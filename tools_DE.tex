\chapter{Tools}

\epigraph{Now that Dennis Yurichev has made this book free (libre), it is a
contribution to the world of free knowledge and free education.
However, for our freedom's sake, we need free (libre) reverse
engineering tools to replace the proprietary tools described in this book.}{Richard M. Stallman}

\section{Binäre Analyse}

Tools die genutzt werden können, wenn kein Prozess gestartet wurde.

\myindex{Hiew}
\myindex{GHex}
\myindex{UNIX!strings}
\myindex{UNIX!xxd}
\myindex{UNIX!od}

\begin{itemize}
\item
(kostenlos, Open Source) \IT{ent}\footnote{\url{http://www.fourmilab.ch/random/}}: Entropie-Analyse-Tool.
Mehr über Entropie: \myref{entropy}.

\item
\label{Hiew}
\IT{Hiew}\footnote{\href{http://go.yurichev.com/17035}{hiew.ru}}:
für kleinere Modifikationen von Code in Binärdateien.
Beinhaltet einen Assembler / Dissassembler.

\item (kostenlos, Open Source) \IT{GHex}\footnote{\url{https://wiki.gnome.org/Apps/Ghex}}: Einfacher Hex-Editor für Linux.

\item (kostenlos, Open Source) \IT{xxd} und \IT{od}: Standard UNIX-Tools für Dumping.

\item (kostenlos, Open Source) \IT{strings}: *NIX-Tool für das Suchen von ASCII-Zeichenketten in Binärdateien,
inklusive ausführbaren Dateien.
Sysinternals hat eine Alternative\footnote{\url{https://technet.microsoft.com/en-us/sysinternals/strings}}
die Wide-Charakter-Zeichenketten unterstützt (UTF-16, unter Windows weit verbreitet).

\item (kostenlos, Open Source) \IT{Binwalk}\footnote{\url{http://binwalk.org/}}: Analyse von Firmware-Images.

\item
\myindex{binary grep}
(kostenlos, Open Source) \IT{binary grep}:
ein kleines Tool um jede Byte-Sequenz in einer großen Anzahl von Dateien zu suchen,
inklusive nicht-ausführbaren Dateien: \BGREPURL.
\myindex{rafind2}
Es gibt auch rafind2 in rada.re mit dem gleichen Verwendungszweck.
\end{itemize}

\subsection{Disassembler}

\myindex{IDA}
\myindex{Binary Ninja}
\myindex{BinNavi}
\myindex{objdump}

\begin{itemize}
\item \IT{IDA}. Eine ältere Freeware-Version ist online erhältlich
\footnote{\href{http://go.yurichev.com/17031}{hex-rays.com/products/ida/support/download\_freeware.shtml}}.
\ShortHotKeyCheatsheet: \myref{sec:IDA_cheatsheet}

\item \IT{Binary Ninja}\footnote{\url{http://binary.ninja/}}

\item (kostenlos, Open Source) \IT{zynamics BinNavi}\footnote{\url{https://www.zynamics.com/binnavi.html}}

\item (kostenlos, Open Source) \IT{objdump}: Einfaches Kommandozeilen-Tool für Dumping und zum disassemblieren.

\item (kostenlos, Open Source) \IT{readelf}\footnote{\url{https://sourceware.org/binutils/docs/binutils/readelf.html}}:
Gibt Informationen über ELF-Dateien aus.
\end{itemize}

\subsection{Decompiler}

Es gibt lediglich einen bekannten, öffentlich verfügbaren Decompiler für C-Code in
hoher Qualität: \IT{Hex-Rays}:\\
\href{http://go.yurichev.com/17033}{hex-rays.com/products/decompiler/}

Mehr darüber: \myref{hex_rays}.

\subsection{Vergleichen von Patches}

Diese Tools können genutzt werden wenn die Original-Version einer ausführbaren Datei
mit einer veränderten Version verglichen werden soll, oder um herauszufinden was
verändert wurde und warum.

\begin{itemize}
\item (kostenlos) \IT{zynamics BinDiff}\footnote{\url{https://www.zynamics.com/software.html}}

\item (kostenlos, Open Source) \IT{Diaphora}\footnote{\url{https://github.com/joxeankoret/diaphora}}
\end{itemize}

\section{Live-Analyse}

Tools die im Live-System oder auf laufende Prozesse angewandt werden können.

\subsection{Debugger}

\myindex{\olly}
\myindex{Radare}
\myindex{GDB}
\myindex{tracer}
\myindex{LLDB}
\myindex{WinDbg}
\myindex{IDA}

\begin{itemize}
\item (kostenlos) \IT{OllyDbg}.
Sehr populärer user-mode Debugger für die Win32-Architektur\footnote{\href{http://go.yurichev.com/17032}{ollydbg.de}}.
\ShortHotKeyCheatsheet: \myref{sec:Olly_cheatsheet}

\item (kostenlos, Open Source) \IT{GDB}.
Nicht sehr populärer Debugger unter Reverse Engineers, da eher für Programmierer gemacht.
Einige Kommandos: \myref{sec:GDB_cheatsheet}.
Es gibt eine grafische Oberfläche für GDB, ``GDB dashboard''\footnote{\url{https://github.com/cyrus-and/gdb-dashboard}}.

\item (kostenlos, Open Source) \IT{LLDB}\footnote{\url{http://lldb.llvm.org/}}.

\item \IT{WinDbg}\footnote{\url{https://developer.microsoft.com/en-us/windows/hardware/windows-driver-kit}}:
Kernel-Debugger für Windows.

\item \IT{IDA} hat einen internen Debugger.

\item (kostenlos, Open Source) \IT{Radare} \ac{AKA} rada.re \ac{AKA} r2\footnote{\url{http://rada.re/r/}}.
Es existiert auch eine GUI: \IT{ragui}\footnote{\url{http://radare.org/ragui/}}.

\item (kostenlos, Open Source) \IT{tracer}.
\label{tracer}
Der Auto benutzt oft \IT{tracer}
\footnote{\href{http://go.yurichev.com/17338}{yurichev.com}}
anstatt Debugger.

Der Autor dieses Buchs hat irgendwann aufgehört Debugger zu nutzen, da alles was er von diesen
brauchte die Funktionsargumente während der Ausführung oder die Zustände der Register an einem
bestimmten Punkt anzuzeigen.
Jedes Mal den Debugger zu starten ist zu aufwändig, deswegen entstand das kleine Tool \IT{tracer}.
Es funktioniert in der Kommandozeile und erlaubt es Funktionsauführungen abzufangen,
Breakpoints an beliebigen Stellen zu setzen und Register-Zustände zu lesen und ändern.

\IT{tracer} wird nicht weiterentwickelt, weil es als Demonstrationstool für dieses Buch entstand
und nicht als Tool für den Alltag.
\end{itemize}

%\subsection{Library calls tracing}
%
%\IT{ltrace}\footnote{\url{http://www.ltrace.org/}}.
%
%\subsection{System calls tracing}
%
%\label{strace}
%\myindex{strace}
%\myindex{dtruss}
%\subsubsection{strace / dtruss}
%
%\myindex{syscall}
%It shows which system calls (syscalls(\myref{syscalls})) are called by a process right now.
%
%For example:
%
%\begin{lstlisting}
%# strace df -h
%
%...
%
%access("/etc/ld.so.nohwcap", F_OK)      = -1 ENOENT (No such file or directory)
%open("/lib/i386-linux-gnu/libc.so.6", O_RDONLY|O_CLOEXEC) = 3
%read(3, "\177ELF\1\1\1\0\0\0\0\0\0\0\0\0\3\0\3\0\1\0\0\0\220\232\1\0004\0\0\0"..., 512) = 512
%fstat64(3, {st_mode=S_IFREG|0755, st_size=1770984, ...}) = 0
%mmap2(NULL, 1780508, PROT_READ|PROT_EXEC, MAP_PRIVATE|MAP_DENYWRITE, 3, 0) = 0xb75b3000
%\end{lstlisting}
%
%\myindex{\MacOSX}
%\MacOSX has dtruss for doing the same.
%
%\myindex{Cygwin}
%Cygwin also has strace, but as far as it's known, it works only for .exe-files
%compiled for the cygwin environment itself.
%
%\subsection{Network sniffing}
%
%\IT{Sniffing} is intercepting some information you may be interested in.
%
%(Free, open-source) \IT{Wireshark}\footnote{\url{https://www.wireshark.org/}} for network sniffing.
%It has also capability for USB sniffing\footnote{\url{https://wiki.wireshark.org/CaptureSetup/USB}}.
%
%Wireshark has a younger (or older) brother \IT{tcpdump}\footnote{\url{http://www.tcpdump.org/}}, simpler command-line tool.
%
%\subsection{Sysinternals}
%
%\myindex{Sysinternals}
%(Free) Sysinternals (developed by Mark Russinovich)
%\footnote{\url{https://technet.microsoft.com/en-us/sysinternals/bb842062}}.
%At least these tools are important and worth studying: Process Explorer, Handle, VMMap, TCPView, Process Monitor.
%
%\subsection{Valgrind}
%
%(Free, open-source) a powerful tool for detecting memory leaks: \url{http://valgrind.org/}.
%Due to its powerful \ac{JIT} mechanism, Valgrind is used as a framework for other tools.
%
%% TODO network fuzzing
%
%\subsection{Emulators}
%
%\begin{itemize}
%\item (Free, open-source) \IT{QEMU}\footnote{\url{http://qemu.org}}: emulator for various CPUs and architectures.
%
%\item (Free, open-source) \IT{DosBox}\footnote{\url{https://www.dosbox.com/}}: MS-DOS emulator, mostly used for retrogaming.
%
%\item (Free, open-source) \IT{SimH}\footnote{\url{http://simh.trailing-edge.com/}}: emulator of ancient computers, mainframes, etc.
%\end{itemize}
%
%\section{Other tools}
%
%\IT{Microsoft Visual Studio Express}
%\footnote{\href{http://go.yurichev.com/17034}{visualstudio.com/en-US/products/visual-studio-express-vs}}:
%Stripped-down free version of Visual Studio, convenient for simple experiments.
%
%Some useful options: \myref{sec:MSVC_options}.
%
%There is a website named ``Compiler Explorer'', allowing to compile small code snippets and see output
%in various GCC versions and architectures
%(at least x86, ARM, MIPS): \url{http://gcc.beta.godbolt.org/}---I would use it myself for the book if I would know about it!
%
%\subsection{Calculators}
%
%Good calculator for reverse engineer's needs should support at least decimal, hexadecimal and binary bases,
%as well as many important operations like XOR and shifts.
%
%\begin{itemize}
%
%\item IDA has built-in calculator (``?'').
%
%\item rada.re has \IT{rax2}.
%
%\item \url{https://github.com/dennis714/progcalc}
%
%\item As a last resort, standard calculator in Windows has programmer's mode.
%
%\end{itemize}
%
%\section{Something missing here?}
%
%If you know a great tool not listed here, please drop me a note:\\
%\TT{\EMAIL}.
%
