% To translators: don't bother to translate this... english-only version.
\vspace*{\fill}

\iffalse
\large \textbf{Subscribe to news about my other articles and blog posts:}

\bigskip

\begin{itemize}

\item \url{https://twitter.com/yurichev}

\item \url{https://www.facebook.com/dennis.yurichev.5}

\end{itemize}

\bigskip
\fi
% ---------------------------------------
\huge Advertisement: reverse engineering services
\normalsize

\bigskip
\bigskip

I can't accept full-time job offers, I mostly work remotely on small tasks, like these:

\large Decrypting a database, managing unknown type of files \normalsize

%Due to NDA agreement, I can't reveal many details about the last case, but the case in \myref{encrypted_DB1} section
%is heavily based on a real case.

\large Rewriting some kind of old EXE or DLL file back to C/C++ \normalsize

I tried many jobs in my life, but, surprisingly (even to myself),
the job I'm the most proud of is rewriting large piece(s) of compiled code back to C/C++.
This is an extremely boring and slow process, I once spent more than a year on rewriting 100KB DLL to pure C,
and it was like full-time job.
And this is also expensive.

\large Dongles \normalsize

Occasionally I do \href{https://en.wikipedia.org/wiki/Software_protection_dongle}{software copy-protection dongle} replacements or dongle emulators. In general, it is somewhat unlawful to break software protection, so I can do this only if these conditions are met:

\begin{itemize}
\item software company who developed the software product does not exist anymore to my best knowledge;
\item the software product is older than 10 years;
\item you have a dongle to read information from it. In other words, I can only help to those who still uses some very old software, completely satisfied with it, but afraid of dongle electrical breakage and there are no company who can still sell the dongle replacement.
\end{itemize}

These includes ancient MS-DOS and UNIX software. Software for exotic computer architectures (like MIPS, DEC Alpha, PowerPC) accepted as well.

Examples of my work you may find here:

\begin{itemize}
%\item My book devoted to reverse engineering has a part about copy-protection dongles: \ref{dongles}.
\item This book devoted to reverse engineering has a part about copy-protection dongles.
\item \href{http://yurichev.com/writings/z3_rockey.pdf}{Finding unknown algorithm using only input/output pairs and Z3 SMT solver article}
\item \href{http://yurichev.com/blog/56/}{About MicroPhar (93c46-based dongle) emulation in DosBox}.
\item \href{http://conus.info/dongle/src/microph.asm}{Source code of DOS MicroPhar emulator using EMM386 I/O interception API}
\end{itemize}

\bigskip

I could also try (binary) code audit.
I can try to find vulnerabilities in your software, before others will do it.
This is like penetration testing.
I can try to work with binary code without source code.

You must also be a legal owner of the software product.

E-Mail: \GTT{\EMAIL}.

\iffalse
\bigskip

\huge Please donate
\normalsize

\bigskip

\dots to this project so I can continue to work on the book and other articles: \\
\url{https://yurichev.com/donate.html}.

\bigskip
\bigskip
\bigskip

\huge Attention: Opinion Poll
\normalsize

\bigskip
\bigskip
\bigskip

I have an idea to replace all the OllyDbg examples in the book with examples using some other debugger.
I have nothing against OllyDbg, but it has a GUI and uses small fonts, and the screenshots are somewhat unsuitable for the book.

Maybe I could use GDB, rada.re, WinDbg, or maybe some other console debugger?

What do you think about it?
Should I leave OllyDbg examples, or would radare examples would be OK?

E-Mail: \GTT{\EMAIL}.
\fi

\vspace*{\fill}
%\vfill
