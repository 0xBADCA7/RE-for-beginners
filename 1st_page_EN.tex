% To translators: don't bother to translate this... english-only version.
%\vspace*{\fill}

(Advertisement: reverse engineering services.)

\begin{itemize}
\item Decrypting a database, getting into files of unknown type

\item Rewriting some kind of old EXE or DLL file back to C/C++

\item Dongles

Occasionally I do \href{https://en.wikipedia.org/wiki/Software_protection_dongle}{software copy-protection dongle} replacements or dongle emulators.

In general, it is somewhat unlawful to break software protection, so I can do this only if these conditions are met:
the software company who developed the software product does not exist anymore to my best knowledge;
the software product is older than 10 years;
you have a dongle to read information from it. In other words, I can only help to those who still uses some very old software, completely satisfied with it, but afraid of dongle electrical breakage and there are no company who can still sell the dongle replacement.

These includes ancient MS-DOS and UNIX software. Software for exotic computer architectures (like MIPS, DEC Alpha, PowerPC) accepted as well.

Examples of my work you may find in this book.

\item Code audit

I could also try (binary) code audit.
I can try to find vulnerabilities in your software, before others will do it.
This is like penetration testing.
I can try to work with binary code without source code.

You must also be a legal owner of the software product.

E-Mail: \EMAIL, Skype: dennis.yurichev, Telegram: @yurichev.
\end{itemize}

\bigskip

Please donate to this project so I can continue to work on the book and other articles. \\
Bitcoin: 1LLa7aqQbRmCYbccnCNwaLgxK9jFPTqPw

\bigskip

Stay tuned:

My
\href{https://twitter.com/yurichev}{Twitter},
\href{https://www.facebook.com/dennis.yurichev.5}{Facebook},
\href{https://ua.linkedin.com/in/dennis-yurichev-5a8368132}{Linkedin}.

My Telegram channel: @yurichev\_news

%\vspace*{\fill}
%\vfill
