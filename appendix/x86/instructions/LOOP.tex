\myindex{x86!\Instructions!LOOP}
  \item[LOOP] (M) \RU{\glslink{decrement}{декремент}}\EN{\gls{decrement}} CX/ECX/RCX,
  \RU{переход если он всё еще не ноль}\EN{jump if it is still not zero}.

\RU{Инструкцию LOOP очень часто использовали в DOS-коде, который работал внешними устройствами.
Чтобы сделать небольшую задержку, делали так:}%
\EN{LOOP instruction was often used in DOS-code which works with external devices.
To add small delay, this was done:}

\begin{lstlisting}[style=customasmx86]
	MOV	CX, nnnn
LABEL:	LOOP	LABEL
\end{lstlisting}

\RU{Недостаток очевиден: длительность задержки сильно зависит от скорости \ac{CPU}.}%
\EN{Drawback is obvious: length of delay depends on \ac{CPU} speed.}

