\section{x86}

\subsection{Terminologie}

Geläufig für 16-Bit (8086/80286), 32-Bit (80386, etc.), 64-Bit.

\myindex{IEEE 754}
\myindex{MS-DOS}
\begin{description}
	\item[Byte] 8-Bit.
		Die DB Assembler-Direktive wird zum Definieren von Variablen und Arrays genutzt.
		Bytes werden in dem 8-Bit-Teil der folgenden Register übergeben:
		\TT{AL/BL/CL/DL/AH/BH/CH/DH/SIL/DIL/R*L}.
	\item[Wort] 16-Bit.
		DW Assembler-Direktive \dittoclosing.
		Bytes werden in dem 16-Bit-Teil der folgenden Register übergeben:
			\TT{AX/BX/CX/DX/SI/DI/R*W}.
	\item[Doppelwort] (\q{dword}) 32-Bit.
		DD Assembler-Direktive \dittoclosing.
		Doppelwörter werden in Registern (x86) oder dem 32-Bit-Teil der Register (x64) übergeben.
		In 16-Bit-Code werden Doppelwörter in 16-Bit-Registerpaaren übergeben.
	\item[zwei Doppelwörter] (\q{qword}) 64-Bit.
		DQ Assembler-Direktive \dittoclosing.
		In 32-Bit-Umgebungen werden diese in 32-Bit-Registerpaaren übergeben.
	\item[tbyte] (10 Byte) 80-Bit oder 10 Bytes (für IEEE 754 FPU Register).
	\item[paragraph] (16 Byte) --- Bezeichnung war in MS-DOS Umgebungen gebräuchlich.
\end{description}

\myindex{Windows!API}

Datentypen der selben Breite (BYTE, WORD, DWORD) entsprechen auch denen in der Windows \ac{API}.

% TODO German Translation (DSiekmeier)
\subsection{\RU{Регистры общего пользования}\EN{General purpose registers}}

\RU{Ко многим регистрам можно обращаться как к частям размером в байт или 16-битное слово}
\EN{It is possible to access many registers by byte or 16-bit word parts}.
\RU{Это всё --- наследие от более старых процессоров Intel (вплоть до 8-битного 8080),
все еще поддерживаемое для обратной совместимости}\EN{It is all inheritance from older Intel CPUs (up to the 8-bit 8080) 
still supported for backward compatibility}.
\EN{Older 8-bit CPUs (8080) had 16-bit registers divided by two.}
\RU{Старые 8-битные процессоры 8080 имели 16-битные регистры, разделенные на две части.}
\EN{Programs written for 8080 could access the low byte part of 16-bit registers, high byte part
or the whole 16-bit register.}
\RU{Программы, написанные для 8080 имели доступ к младшему байту 16-битного регистра, к старшему
байту или к целому 16-битному регистру.}
\EN{Perhaps, this feature was left in 8086 as a helper for easier porting.}
\RU{Вероятно, эта возможность была оставлена в 8086 для более простого портирования.}
\RU{В \ac{RISC} процессорах, такой возможности, как правило, нет}\EN{This feature
is usually not present in \ac{RISC} CPUs}.

\myindex{x86-64}
\RU{Регистры, имеющие префикс \TT{R-} появились только в x86-64, а префикс \TT{E-} ---в 80386.}
\EN{Registers prefixed with \TT{R-} appeared in x86-64, and those prefixed with \TT{E-}---in 80386.}
\RU{Таким образом, R-регистры 64-битные, а E-регистры --- 32-битные.}
\EN{Thus, R-registers are 64-bit, and E-registers---32-bit.}

\RU{В x86-64 добавили еще 8 \ac{GPR}: R8-R15}
\EN{8 more \ac{GPR}'s were added in x86-86: R8-R15}.

N.B.: \RU{В документации от Intel, для обращения к самому младшему байту к имени регистра
нужно добавлять суффикс \IT{L}: \IT{R8L}, но \ac{IDA} называет эти регистры добавляя суффикс \IT{B}: \IT{R8B}}
\EN{In the Intel manuals the byte parts of these registers are prefixed by \IT{L}, e.g.: \IT{R8L}, but \ac{IDA}
names these registers by adding the \IT{B} suffix, e.g.: \IT{R8B}}.

\subsubsection{RAX/EAX/AX/AL}
\RegTableOne{RAX}{EAX}{AX}{AH}{AL}

\ac{AKA} \RU{аккумулятор}\EN{accumulator}.
\RU{Результат функции обычно возвращается через этот регистр}
\EN{The result of a function is usually returned via this register}.

\subsubsection{RBX/EBX/BX/BL}
\RegTableOne{RBX}{EBX}{BX}{BH}{BL}

\subsubsection{RCX/ECX/CX/CL}
\RegTableOne{RCX}{ECX}{CX}{CH}{CL}

\ac{AKA} \RU{счетчик}\EN{counter}: 
\RU{используется в этой роли в инструкциях с префиксом REP и в инструкциях сдвига}
\EN{in this role it is used in REP prefixed instructions and also in shift instructions}
(SHL/SHR/RxL/RxR).

\subsubsection{RDX/EDX/DX/DL}
\RegTableOne{RDX}{EDX}{DX}{DH}{DL}

\subsubsection{RSI/ESI/SI/SIL}
\RegTableTwo{RSI}{ESI}{SI}{SIL}

\ac{AKA} \q{source index}. \RU{Используется как источник в инструкциях}\EN{Used as source in the instructions} 
REP MOVSx, REP CMPSx.

\subsubsection{RDI/EDI/DI/DIL}
\RegTableTwo{RDI}{EDI}{DI}{DIL}

\ac{AKA} \q{destination index}. \RU{Используется как указатель на место назначения в инструкции}
\EN{Used as a pointer to the destination in the instructions} REP MOVSx, REP STOSx.

% TODO навести тут порядок
\subsubsection{R8/R8D/R8W/R8L}
\RegTableFour{R8}{R8D}{R8W}{R8L}

\subsubsection{R9/R9D/R9W/R9L}
\RegTableFour{R9}{R9D}{R9W}{R9L}

\subsubsection{R10/R10D/R10W/R10L}
\RegTableFour{R10}{R10D}{R10W}{R10L}

\subsubsection{R11/R11D/R11W/R11L}
\RegTableFour{R11}{R11D}{R11W}{R11L}

\subsubsection{R12/R12D/R12W/R12L}
\RegTableFour{R12}{R12D}{R12W}{R12L}

\subsubsection{R13/R13D/R13W/R13L}
\RegTableFour{R13}{R13D}{R13W}{R13L}

\subsubsection{R14/R14D/R14W/R14L}
\RegTableFour{R14}{R14D}{R14W}{R14L}

\subsubsection{R15/R15D/R15W/R15L}
\RegTableFour{R15}{R15D}{R15W}{R15L}

\subsubsection{RSP/ESP/SP/SPL}
\RegTableFour{RSP}{ESP}{SP}{SPL}

\ac{AKA} \gls{stack pointer}. \RU{Обычно всегда указывает на текущий стек, кроме тех случаев,
когда он не инициализирован}\EN{Usually points to the current stack except in those cases when it is not yet initialized}.

\subsubsection{RBP/EBP/BP/BPL}
\RegTableFour{RBP}{EBP}{BP}{BPL}

\ac{AKA} frame pointer. \RU{Обычно используется для доступа к локальным переменным функции и аргументам,
Больше о нем}
\EN{Usually used for local variables and accessing the arguments of the function. More about it}: (\myref{stack_frame}).

\subsubsection{RIP/EIP/IP}

\begin{center}
\begin{tabular}{ | l | l | l | l | l | l | l | l | l |}
\hline
\RegHeaderTop \\
\hline
\RegHeader \\
\hline
\multicolumn{8}{ | c | }{RIP\textsuperscript{x64}} \\
\hline
\multicolumn{4}{ | c | }{} & \multicolumn{4}{ c | }{EIP} \\
\hline
\multicolumn{6}{ | c | }{} & \multicolumn{2}{ c | }{IP} \\
\hline
\end{tabular}
\end{center}

\ac{AKA} \q{instruction pointer}
\footnote{\RU{Иногда называется также}\EN{Sometimes also called} \q{program counter}}.
\RU{Обычно всегда указывает на инструкцию, которая сейчас будет исполняться
Напрямую модифицировать регистр нельзя, хотя можно делать так (что равноценно)}%
\EN{Usually always points to the instruction to be executed right now.
Cannot be modified, however, it is possible to do this (which is equivalent)}:

\begin{lstlisting}
MOV EAX, ...
JMP EAX
\end{lstlisting}

\RU{Либо}\EN{Or}:

\begin{lstlisting}
PUSH value
RET
\end{lstlisting}

\subsubsection{CS/DS/ES/SS/FS/GS}

\RU{16-битные регистры, содержащие селектор кода}\EN{16-bit registers containing code selector} (CS), 
\RU{данных}\EN{data selector} (DS), \RU{стека}\EN{stack selector} (SS).\\
\\
\myindex{TLS}
\myindex{Windows!TIB}
FS \InENRU win32 \RU{указывает на}\EN{points to} \ac{TLS}, \RU{а в Linux на эту роль был выбран GS}
\EN{GS took this role in Linux}.
\RU{Это сделано для более быстрого доступа к \ac{TLS} и прочим структурам там вроде \ac{TIB}}
\EN{It is made so for faster access to the \ac{TLS} and other structures like the \ac{TIB}}.
\\
\RU{В прошлом эти регистры использовались как сегментные регистры}
\EN{In the past, these registers were used as segment registers} (\myref{8086_memory_model}).

\subsubsection{\RU{Регистр флагов}\EN{Flags register}}
\myindex{x86!\Registers!\Flags}
\label{EFLAGS}
\ac{AKA} EFLAGS.

\small
\begin{center}
\begin{tabular}{ | l | l | l | }
\hline
\headercolor{} \RU{Бит}\EN{Bit} (\RU{маска}\EN{mask}) &
\headercolor{} \RU{Аббревиатура}\EN{Abbreviation} (\RU{значение}\EN{meaning}) &
\headercolor{} \RU{Описание}\EN{Description} \\
\hline
0 (1) & CF (Carry) & \RU{Флаг переноса.} \\
      &            & \RU{Инструкции}\EN{The} CLC/STC/CMC \RU{используются}\EN{instructions are used} \\
      &            & \RU{для установки/сброса/инвертирования этого флага}\EN{for setting/resetting/toggling this flag} \\
\hline
2 (4) & PF (Parity) & \RU{Флаг четности }(\myref{parity_flag}). \\
\hline
4 (0x10) & AF (Adjust) & \RU{Существует только для работы с \ac{BCD}-числами}
			\EN{Exist solely for work with \ac{BCD}-numbers} \\
\hline
6 (0x40) & ZF (Zero) & \RU{Выставляется в}\EN{Setting to} 0 \\
         &           & \RU{если результат последней операции был равен}\EN{if the last operation's result is equal to} 0. \\
\hline
7 (0x80) & SF (Sign) & \RU{Флаг знака.} \\
\hline
8 (0x100) & TF (Trap) & \RU{Применяется при отладке}\EN{Used for debugging}. \\
&         &             \RU{Если включен, то после исполнения каждой инструкции}\EN{If turned on, an exception is to be} \\
&         &             \RU{будет сгенерировано исключение}\EN{generated after each instruction's execution}. \\
\hline
9 (0x200) & IF (Interrupt enable) & \RU{Разрешены ли прерывания}\EN{Are interrupts enabled}. \\
          &                       & \RU{Инструкции}\EN{The} CLI/STI \RU{используются}\EN{instructions are used} \\
	  &                       & \RU{для установки/сброса этого флага}\EN{for setting/resetting the flag} \\
\hline
10 (0x400) & DF (Direction) & \RU{Задается направление для инструкций}\EN{A directions is set for the} \\
           &                & REP MOVSx/CMPSx/LODSx/SCASx\EN{ instructions}.\\
           &                & \RU{Инструкции}\EN{The} CLD/STD \RU{используются}\EN{instructions are used} \\
	   &                & \RU{для установки/сброса этого флага}\EN{for setting/resetting the flag} \\
	   &                & \EN{See also}\RU{См.также}: \myref{memmove_and_DF}. \\
\hline
11 (0x800) & OF (Overflow) & \RU{Переполнение.} \\
\hline
12, 13 (0x3000) & IOPL (I/O privilege level)\textsuperscript{i286} & \\
\hline
14 (0x4000) & NT (Nested task)\textsuperscript{i286} & \\
\hline
16 (0x10000) & RF (Resume)\textsuperscript{i386} & \RU{Применяется при отладке}\EN{Used for debugging}. \\
             &                  & \RU{Если включить,}\EN{The CPU ignores the hardware} \\
	     &                  & \RU{CPU проигнорирует хардварную точку останова в DRx}\EN{breakpoint in DRx if the flag is set}. \\
\hline
17 (0x20000) & VM (Virtual 8086 mode)\textsuperscript{i386} & \\
\hline
18 (0x40000) & AC (Alignment check)\textsuperscript{i486} & \\
\hline
19 (0x80000) & VIF (Virtual interrupt)\textsuperscript{i586} & \\
\hline
20 (0x100000) & VIP (Virtual interrupt pending)\textsuperscript{i586} & \\
\hline
21 (0x200000) & ID (Identification)\textsuperscript{i586} & \\
\hline
\end{tabular}
\end{center}
\normalsize

\RU{Остальные флаги зарезервированы}\EN{All the rest flags are reserved}.

\subsection{FPU \registers{}}

\myindex{x86!FPU}
8 80-\RU{битных регистров работающих как стек}\EN{bit registers working as a stack}: ST(0)-ST(7).
N.B.: \ac{IDA} \RU{называет}\EN{calls} ST(0) \RU{просто}\EN{as just} ST.
\RU{Числа хранятся в формате}\EN{Numbers are stored in the} IEEE 754\EN{ format}.

\RU{Формат значения \IT{long double}}\EN{\IT{long double} value format}:

\bigskip
% a hack used here! http://tex.stackexchange.com/questions/73524/bytefield-package
\begin{center}
\begingroup
\makeatletter
\let\saved@bf@bitformatting\bf@bitformatting
\renewcommand*{\bf@bitformatting}{%
	\ifnum\value{header@val}=21 %
	\value{header@val}=62 %
	\else\ifnum\value{header@val}=22 %
	\value{header@val}=63 %
	\else\ifnum\value{header@val}=23 %
	\value{header@val}=64 %
	\else\ifnum\value{header@val}=30 %
	\value{header@val}=78 %
	\else\ifnum\value{header@val}=31 %
	\value{header@val}=79 %
	\fi\fi\fi\fi\fi
	\saved@bf@bitformatting
}%
\begin{bytefield}[bitwidth=0.03\linewidth]{32}
	\bitheader[endianness=big]{0,21,22,23,30,31} \\
	\bitbox{1}{S} &
	\bitbox{8}{\RU{экспонента}\EN{exponent}} &
	\bitbox{1}{I} &
	\bitbox{22}{\RU{мантисса}\EN{mantissa or fraction}}
\end{bytefield}
\endgroup
\end{center}

\begin{center}
( S\EMDASH{}\RU{знак}\EN{sign}, I\EMDASH{}\RU{целочисленная часть}\EN{integer part} )
\end{center}

\label{FPU_control_word}
\subsubsection{\RU{Регистр управления}\EN{Control Word}}

\RU{Регистр, при помощи которого можно задавать поведение}\EN{Register controlling the behavior of the}
\ac{FPU}.

\small
\begin{center}
\begin{tabular}{ | l | l | l | }
\hline
\RU{Бит}\EN{Bit} &
\RU{Аббревиатура (значение)}\EN{Abbreviation (meaning)} &
\RU{Описание}\EN{Description} \\
\hline
0   & IM (Invalid operation Mask) & \\
\hline
1   & DM (Denormalized operand Mask) & \\
\hline
2   & ZM (Zero divide Mask) & \\
\hline
3   & OM (Overflow Mask) & \\
\hline
4   & UM (Underflow Mask) & \\
\hline
5   & PM (Precision Mask) & \\
\hline
7   & IEM (Interrupt Enable Mask) & \RU{Разрешение исключений, по умолчанию 1 (запрещено)}
\EN{Exceptions enabling, 1 by default (disabled)} \\
\hline
8, 9 & PC (Precision Control) & \RU{Управление точностью} \\
     &                        & 00 ~--- 24 \RU{бита}\EN{bits} (REAL4) \\
     &                        & 10 ~--- 53 \RU{бита}\EN{bits} (REAL8) \\
     &                        & 11 ~--- 64 \RU{бита}\EN{bits} (REAL10) \\
\hline
10, 11 & RC (Rounding Control) & \RU{Управление округлением} \\
       &                       & 00 ~--- \RU{(по умолчанию) округлять к ближайшему}\EN{(by default) round to nearest} \\
       &                       & 01 ~--- \RU{округлять к}\EN{round toward} $-\infty$ \\
       &                       & 10 ~--- \RU{округлять к}\EN{round toward} $+\infty$ \\
       &                       & 11 ~--- \RU{округлять к}\EN{round toward} 0 \\
\hline
12 & IC (Infinity Control) & 0 ~--- (\RU{по умолчанию}\EN{by default}) \RU{считать}\EN{treat} $+\infty$ \AndENRU $-\infty$ \RU{за беззнаковое}\EN{as unsigned} \\
   &                       & 1 ~--- \RU{учитывать и}\EN{respect both} $+\infty$ \AndENRU $-\infty$ \\
\hline
\end{tabular}
\end{center}
\normalsize

\RU{Флагами}\EN{The} PM, UM, OM, ZM, DM, IM 
\RU{задается, генерировать ли исключения в случае соответствующих ошибок}
\EN{flags define if to generate exception in the case of a corresponding error}.

\subsubsection{\RU{Регистр статуса}\EN{Status Word}}

\label{FPU_status_word}
\RU{Регистр только для чтения}\EN{Read-only register}.

\small
\begin{center}
\begin{tabular}{ | l | l | l | }
\hline
\RU{Бит}\EN{Bit} &
\RU{Аббревиатура (значение)}\EN{Abbreviation (meaning)} &
\RU{Описание}\EN{Description} \\
\hline
15   & B (Busy) & \RU{Работает ли сейчас FPU}\EN{Is FPU do something} (1)
\RU{или закончил и результаты готовы}\EN{or results are ready} (0) \\
\hline
14   & C3 & \\
\hline
13, 12, 11 & TOP & \RU{указывает, какой сейчас регистр является нулевым}
\EN{points to the currently zeroth register} \\
\hline
10 & C2 & \\
\hline
9  & C1 & \\
\hline
8  & C0 & \\
\hline
7  & IR (Interrupt Request) & \\
\hline
6  & SF (Stack Fault) & \\
\hline
5  & P (Precision) & \\
\hline
4  & U (Underflow) & \\
\hline
3  & O (Overflow) & \\
\hline
2  & Z (Zero) & \\
\hline
1  & D (Denormalized) & \\
\hline
0  & I (Invalid operation) & \\
\hline
\end{tabular}
\end{center}
\normalsize

\RU{Биты}\EN{The} SF, P, U, O, Z, D, I \RU{сигнализируют об исключениях}
\EN{bits signal about exceptions}.

\RU{О}\EN{About the} C3, C2, C1, C0 \RU{читайте больше}\EN{you can read more here}: (\myref{Czero_etc}).

N.B.: \RU{когда используется регистр ST(x), FPU прибавляет $x$ к TOP по модулю 8 и получается номер
внутреннего регистра}\EN{When ST(x) is used, the FPU adds $x$ to TOP (by modulo 8) and that is how it gets 
the internal register's number}.

\subsubsection{Tag Word}

\RU{Этот регистр отражает текущее содержимое регистров чисел}
\EN{The register has current information about the usage of numbers registers}.

\begin{center}
\begin{tabular}{ | l | l | l | }
\hline
\RU{Бит}\EN{Bit} & \RU{Аббревиатура (значение)}\EN{Abbreviation (meaning)} \\
\hline
15, 14 & Tag(7) \\
\hline
13, 12 & Tag(6) \\
\hline
11, 10 & Tag(5) \\
\hline
9, 8 & Tag(4) \\
\hline
7, 6 & Tag(3) \\
\hline
5, 4 & Tag(2) \\
\hline
3, 2 & Tag(1) \\
\hline
1, 0 & Tag(0) \\
\hline
\end{tabular}
\end{center}

\EN{Each tag contains information about a physical FPU register (R(x)), not logical (ST(x)).}
\RU{Каждый тэг содержит информацию о физическом регистре FPU (R(x)), но не логическом (ST(x)).}

\RU{Для каждого тэга}\EN{For each tag}:

\begin{itemize}
\item 00 ~--- \RU{Регистр содержит ненулевое значение}\EN{The register contains a non-zero value}
\item 01 ~--- \RU{Регистр содержит 0}\EN{The register contains 0}
\EN{\item 10 ~--- The register contains a special value (\ac{NAN}, $\infty$, or denormal)}
\RU{\item 10 ~--- Регистр содержит специальное число (\ac{NAN}, $\infty$, или денормализованное число)}
\item 11 ~--- \RU{Регистр пуст}\EN{The register is empty}
\end{itemize}

\subsection{SIMD \registers{}}

\subsubsection{MMX \registers{}}

8 64-\RU{битных регистров}\EN{bit registers}: MM0..MM7.

\subsubsection{SSE \AndENRU AVX \registers{}}

\myindex{x86-64}
SSE: 8 128-\RU{битных регистров}\EN{bit registers}: XMM0..XMM7.
\RU{В}\EN{In the} x86-64 \RU{добавлено еще 8 регистров}\EN{8 more registers were added}: XMM8..XMM15.

AVX \RU{это расширение всех регистры до 256 бит}\EN{is the extension of all these registers to 256 bits}.

\input{appendix/x86/DRx}

% TODO: control registers
 % subsection
\input{appendix/x86/instructions} % subsection
\subsection{npad}
\label{sec:npad}

\RU{Это макрос в ассемблере, для выравнивания некоторой метки по некоторой границе.}
\EN{It is an assembly language macro for aligning labels on a specific boundary.}
\DE{Dies ist ein Assembler-Makro um Labels an bestimmten Grenzen auszurichten.}

\RU{Это нужно для тех \IT{нагруженных} меток, куда чаще всего передается управление, например, 
начало тела цикла. 
Для того чтобы процессор мог эффективнее вытягивать данные или код из памяти, через шину с памятью, 
кэширование, итд.}
\EN{That's often needed for the busy labels to where the control flow is often passed, e.g., loop body starts.
So the CPU can load the data or code from the memory effectively, through the memory bus, cache lines, etc.}
\DE{Dies ist oft nützlich Labels, die oft Ziel einer Kotrollstruktur sind, wie Schleifenköpfe.
Somit kann die CPU Daten oder Code sehr effizient vom Speicher durch den Bus, den Cache, usw. laden.}

\RU{Взято из}\EN{Taken from}\DE{Entnommen von} \TT{listing.inc} (MSVC):

\myindex{x86!\Instructions!NOP}
\RU{Это, кстати, любопытный пример различных вариантов \NOP{}-ов. 
Все эти инструкции не дают никакого эффекта, но отличаются разной длиной.}
\EN{By the way, it is a curious example of the different \NOP variations.
All these instructions have no effects whatsoever, but have a different size.}
\DE{Dies ist übrigens ein Beispiel für die unterschiedlichen \NOP-Variationen.
Keine dieser Anweisungen hat eine Auswirkung, aber alle haben eine unterschiedliche Größe.}

\RU{Цель в том, чтобы была только одна инструкция, а не набор NOP-ов, 
считается что так лучше для производительности CPU.}
\EN{Having a single idle instruction instead of couple of NOP-s,
is accepted to be better for CPU performance.}
\DE{Eine einzelne Idle-Anweisung anstatt mehrerer NOPs hat positive Auswirkungen
auf die CPU-Performance.}

\begin{lstlisting}[style=customasmx86]
;; LISTING.INC
;;
;; This file contains assembler macros and is included by the files created
;; with the -FA compiler switch to be assembled by MASM (Microsoft Macro
;; Assembler).
;;
;; Copyright (c) 1993-2003, Microsoft Corporation. All rights reserved.

;; non destructive nops
npad macro size
if size eq 1
  nop
else
 if size eq 2
   mov edi, edi
 else
  if size eq 3
    ; lea ecx, [ecx+00]
    DB 8DH, 49H, 00H
  else
   if size eq 4
     ; lea esp, [esp+00]
     DB 8DH, 64H, 24H, 00H
   else
    if size eq 5
      add eax, DWORD PTR 0
    else
     if size eq 6
       ; lea ebx, [ebx+00000000]
       DB 8DH, 9BH, 00H, 00H, 00H, 00H
     else
      if size eq 7
	; lea esp, [esp+00000000]
	DB 8DH, 0A4H, 24H, 00H, 00H, 00H, 00H 
      else
       if size eq 8
        ; jmp .+8; .npad 6
	DB 0EBH, 06H, 8DH, 9BH, 00H, 00H, 00H, 00H
       else
        if size eq 9
         ; jmp .+9; .npad 7
         DB 0EBH, 07H, 8DH, 0A4H, 24H, 00H, 00H, 00H, 00H
        else
         if size eq 10
          ; jmp .+A; .npad 7; .npad 1
          DB 0EBH, 08H, 8DH, 0A4H, 24H, 00H, 00H, 00H, 00H, 90H
         else
          if size eq 11
           ; jmp .+B; .npad 7; .npad 2
           DB 0EBH, 09H, 8DH, 0A4H, 24H, 00H, 00H, 00H, 00H, 8BH, 0FFH
          else
           if size eq 12
            ; jmp .+C; .npad 7; .npad 3
            DB 0EBH, 0AH, 8DH, 0A4H, 24H, 00H, 00H, 00H, 00H, 8DH, 49H, 00H
           else
            if size eq 13
             ; jmp .+D; .npad 7; .npad 4
             DB 0EBH, 0BH, 8DH, 0A4H, 24H, 00H, 00H, 00H, 00H, 8DH, 64H, 24H, 00H
            else
             if size eq 14
              ; jmp .+E; .npad 7; .npad 5
              DB 0EBH, 0CH, 8DH, 0A4H, 24H, 00H, 00H, 00H, 00H, 05H, 00H, 00H, 00H, 00H
             else
              if size eq 15
               ; jmp .+F; .npad 7; .npad 6
               DB 0EBH, 0DH, 8DH, 0A4H, 24H, 00H, 00H, 00H, 00H, 8DH, 9BH, 00H, 00H, 00H, 00H
              else
	       %out error: unsupported npad size
               .err
              endif
             endif
            endif
           endif
          endif
         endif
        endif
       endif
      endif
     endif
    endif
   endif
  endif
 endif
endif
endm
\end{lstlisting}
 % subsection
