\clearpage
\subsection{Norton Guide: simplest possible 1-byte XOR encryption}
\label{norton_guide}

Norton Guide\footnote{\href{http://go.yurichev.com/17116}{wikipedia}} was popular in the epoch of MS-DOS, it was a resident program that worked as a hypertext reference manual.

Norton Guide's databases are files with the extension .ng, the contents of which look encrypted:

\begin{figure}[H]
\centering
\myincludegraphics{ff/XOR/ng/ng1.png}
\caption{Very typical look}
\end{figure}

Why did we think that it's encrypted but not compressed?

We see that the 0x1A byte (looking like \q{$\rightarrow$}) occurs often, it would not be possible in a compressed file.

We also see long parts that consist only of Latin letters, and they look like strings in an unknown
language.

\clearpage
Since the 0x1A byte occurs so often, we can try to decrypt the file, assuming that it's encrypted
by the simplest XOR-encryption.

If we apply XOR with the 0x1A constant to each byte in Hiew, we can see familiar English text strings:

\begin{figure}[H]
\centering
\myincludegraphics{ff/XOR/ng/ng2.png}
\caption{Hiew XORing with 0x1A}
\end{figure}

XOR encryption with one single constant byte is the simplest possible encryption method, which is, nevertheless,
encountered sometimes.

Now we understand why the 0x1A byte is occurring so often: because there are so many zero bytes and they
were replaced by 0x1A in encrypted form.

But the constant might be different.
In this case, we could try every constant in the 0..255 range and look for something familiar in the decrypted
file. 256 is not so much.

More about Norton Guide's file format: \url{http://go.yurichev.com/17317}.

\subsubsection{Entropy}
\myindex{Wolfram Mathematica}
\myindex{Entropy}

A very important property of such primitive encryption systems is that the information entropy
of the encrypted/decrypted block is the same.

Here is my analysis in Wolfram Mathematica 10.

\begin{lstlisting}[caption=Wolfram Mathematica 10,style=custommath]
In[1]:= input = BinaryReadList["X86.NG"];

In[2]:= Entropy[2, input] // N
Out[2]= 5.62724

In[3]:= decrypted = Map[BitXor[#, 16^^1A] &, input];

In[4]:= Export["X86_decrypted.NG", decrypted, "Binary"];

In[5]:= Entropy[2, decrypted] // N
Out[5]= 5.62724

In[6]:= Entropy[2, ExampleData[{"Text", "ShakespearesSonnets"}]] // N
Out[6]= 4.42366
\end{lstlisting}

What we do here is load the file, get its entropy, decrypt it, save it and get the entropy again (the same!).

Mathematica also offers some well-known English language texts for analysis.

So we also get the entropy of Shakespeare's sonnets, and it is close to the entropy of the file we just analyzed.

The file we analyzed consists of English language sentences, which are close to the language 
of Shakespeare.

And the XOR-ed bitwise English language text has the same entropy.

% I checked!
However, this is not true when the file is XOR-ed with a pattern larger than one byte.

The file we analyzed can be downloaded here: \url{http://go.yurichev.com/17350}.

\myparagraph{One more word about base of entropy}

\newcommand{\FNENTURL}{\footnote{\url{http://www.fourmilab.ch/random/}}}

Wolfram Mathematica calculates entropy with base of $e$ (base of the natural logarithm),
and the UNIX \IT{ent} utility\FNENTURL uses base 2.

So we set base 2 explicitly in \TT{Entropy} command, so Mathematica will give us the same results as the \IT{ent} utility.

