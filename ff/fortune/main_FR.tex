\section{\IT{fortune} programme d'indexation de fichier}

(Cette partie est tout d'abord apparue sur mon blog le 25 avril 2015.)

\IT{fortune} est un programme UNIX bien connu qui affiche une phrase aléatoire depuis
une collection.
Certains geeks ont souvent configuré leur système de telle manière que \IT{fortune}
puisse être appelé après la connexion.
\IT{fortune} prend les phrases depuis des fichiers texte se trouvant dans \IT{/usr/share/games/fortunes}
(sur Ubuntu Linux).
Voici un exemple (fichier texte \q{fortunes}):

\begin{lstlisting}
A day for firm decisions!!!!!  Or is it?
%
A few hours grace before the madness begins again.
%
A gift of a flower will soon be made to you.
%
A long-forgotten loved one will appear soon.

Buy the negatives at any price.
%
A tall, dark stranger will have more fun than you.
%
...
\end{lstlisting}

Donc, il s'agit juste de phrases, parfois sur plusieurs lignes, séparées par le
signe pourcent.
La tâche du programme \IT{fortune} est de trouver une phrase aléatoire et de l'afficher.
Afin de réaliser ceci, il doit scanner le fichier texte complet, compter les phrases,
en choisir une aléatoirement et l'afficher.
Mais le fichier texte peut être très gros, et même sur les ordinateurs modernes, cet
algorithme naïf est du gaspillage de ressource.
La façon simple de procéder est de garder un fichier index binaire contenant l'offset
de chaque phrase du fichier texte.
Avec le fichier index, le programme \IT{fortune} peut fonctionner beaucoup plus vite:
il suffit de choisir un index aléatoirement, prendre son offset, se déplacer à cet
offset dans le fichier texte et d'y lire la phrase.
C'est ce qui est effectivement fait dans le programme \IT{fortune}.
Examinons ce qu'il y a dans ce fichier index (ce sont les fichiers .dat dans le même
répertoire) dans un éditeur hexadécimal.
Ce programme est open-source bien sûr, mais intentionnellement, je ne vais pas jeter
un coup d'\oe{}il dans le code source.

\lstinputlisting{ff/fortune/1.lst}

Sans aide particulière, nous pouvons voir qu'il y a quatre éléments de 4 octets sur
chaque ligne de 16 octets.
Peut-être que c'est notre tableau d'index.
J'essaye de charger tout le fichier comme un tableau d'entier 32-bit dans Wolfram
Mathematica:

\begin{lstlisting}[style=custommath]
In[]:= BinaryReadList["c:/tmp1/fortunes.dat", "UnsignedInteger32"]

Out[]= {33554432, 2936078336, 3137339392, 251658240, 0, 37, 0, \
721420288, 1610612736, 2399141888, 3741319168, 335609856, 1208025088, \
2080440320, 2868969472, 3858825216, 537001984, 989986816, 2046951424, \
3305242624, 67305472, 1023606784, 1745027072, 2801991680, 3775070208, \
419692544, 755236864, 2130968576, 2902720512, 3573809152, 84213760, \
990183424, 1678049280, 2181365760, 2902786048, 3456434176, \
4144300032, 470155264, 1627783168, 2047213568, 3506831360, 168230912, \
1392967680, 2584150016, 4161208320, 654835712, 1493696512, \
2332557312, 2684878848, 3288858624, 3775397888, 4178051072, \
...
\end{lstlisting}

Nope, quelque chose est faux, les nombres sont étrangement gros.
Mais retournons à la sortie de \IT{od}: chaque élément de 4 octets a deux octets nuls
et deux octets non nuls.
Donc les offsets (au moins au début du fichier) sont au maximum 16-bit.
Peut-être qu'un endianness différent est utilisé dans le fichier?
L'endianness par défaut dans Mathematica est little-endian, comme utilisé dans les
CPUs Intel.
Maintenant, je le change en big-endian:

\begin{lstlisting}[style=custommath]
In[]:= BinaryReadList["c:/tmp1/fortunes.dat", "UnsignedInteger32", 
 ByteOrdering -> 1]

Out[]= {2, 431, 187, 15, 0, 620756992, 0, 43, 96, 143, 223, 276, \
328, 380, 427, 486, 544, 571, 634, 709, 772, 829, 872, 935, 993, \
1049, 1069, 1151, 1197, 1237, 1285, 1339, 1380, 1410, 1453, 1486, \
1527, 1564, 1633, 1658, 1745, 1802, 1875, 1946, 2040, 2087, 2137, \
2187, 2208, 2244, 2273, 2297, 2343, 2371, 2425, 2467, 2531, 2581, \
2637, 2654, 2698, 2726, 2751, 2799, 2840, 2883, 2913, 2958, 3023, \
3066, 3131, 3174, 3205, 3257, 3282, 3330, 3387, 3431, 3500, 3552, \
...
\end{lstlisting}

Oui, c'est quelque chose de lisible.
Je choisi un élément au hasard (3066) qui s'écrit 0xBFA en format hexadécimal.
J'ouvre le fichier texte 'fortunes' dans un éditeur hexadécimal, je met l'offset
0xBFA et je vois cette phrase:

\lstinputlisting{ff/fortune/2.lst}

Ou:

\begin{lstlisting}
Do what comes naturally.  Seethe and fume and throw a tantrum.
%
\end{lstlisting}

D'autres offsets peuvent aussi être essayés, oui, ils sont valides.

Je peux aussi vérifier dans Mathematica que chaque élément consécutif est plus grand
que le précédent.
I.e., les éléments du tableau sont croissants.
Dans le jargon mathématiques, ceci est appelé \IT{fonction monotone strictement croissante}.

\begin{lstlisting}[style=custommath]
In[]:= Differences[input]

Out[]= {429, -244, -172, -15, 620756992, -620756992, 43, 53, 47, \
80, 53, 52, 52, 47, 59, 58, 27, 63, 75, 63, 57, 43, 63, 58, 56, 20, \
82, 46, 40, 48, 54, 41, 30, 43, 33, 41, 37, 69, 25, 87, 57, 73, 71, \
94, 47, 50, 50, 21, 36, 29, 24, 46, 28, 54, 42, 64, 50, 56, 17, 44, \
28, 25, 48, 41, 43, 30, 45, 65, 43, 65, 43, 31, 52, 25, 48, 57, 44, \
69, 52, 62, 73, 62, 53, 37, 68, 71, 50, 41, 57, 69, 58, 70, 45, 54, \
38, 45, 50, 42, 61, 47, 43, 62, 189, 61, 56, 30, 85, 63, 48, 61, 58, \
81, 50, 55, 63, 83, 80, 49, 42, 94, 54, 67, 81, 52, 57, 68, 43, 28, \
120, 64, 53, 81, 33, 82, 88, 29, 61, 32, 75, 63, 70, 47, 101, 60, 79, \
33, 48, 65, 35, 59, 47, 55, 22, 43, 35, 102, 53, 80, 65, 45, 31, 29, \
69, 32, 25, 38, 34, 35, 49, 59, 39, 41, 18, 43, 41, 83, 37, 31, 34, \
59, 72, 72, 81, 77, 53, 53, 50, 51, 45, 53, 39, 70, 54, 103, 33, 70, \
51, 95, 67, 54, 55, 65, 61, 54, 54, 53, 45, 100, 63, 48, 65, 71, 23, \
28, 43, 51, 61, 101, 65, 39, 78, 66, 43, 36, 56, 40, 67, 92, 65, 61, \
31, 45, 52, 94, 82, 82, 91, 46, 76, 55, 19, 58, 68, 41, 75, 30, 67, \
92, 54, 52, 108, 60, 56, 76, 41, 79, 54, 65, 74, 112, 76, 47, 53, 61, \
66, 53, 28, 41, 81, 75, 69, 89, 63, 60, 18, 18, 50, 79, 92, 37, 63, \
88, 52, 81, 60, 80, 26, 46, 80, 64, 78, 70, 75, 46, 91, 22, 63, 46, \
34, 81, 75, 59, 62, 66, 74, 76, 111, 55, 73, 40, 61, 55, 38, 56, 47, \
78, 81, 62, 37, 41, 60, 68, 40, 33, 54, 34, 41, 36, 49, 44, 68, 51, \
50, 52, 36, 53, 66, 46, 41, 45, 51, 44, 44, 33, 72, 40, 71, 57, 55, \
39, 66, 40, 56, 68, 43, 88, 78, 30, 54, 64, 36, 55, 35, 88, 45, 56, \
76, 61, 66, 29, 76, 53, 96, 36, 46, 54, 28, 51, 82, 53, 60, 77, 21, \
84, 53, 43, 104, 85, 50, 47, 39, 66, 78, 81, 94, 70, 49, 67, 61, 37, \
51, 91, 99, 58, 51, 49, 46, 68, 72, 40, 56, 63, 65, 41, 62, 47, 41, \
43, 30, 43, 67, 78, 80, 101, 61, 73, 70, 41, 82, 69, 45, 65, 38, 41, \
57, 82, 66}
\end{lstlisting}

Comme on le voit, excepté les 6 premières valeurs (qui appartiennent sans doute à
l'entête du fichier d'index), tous les nombres sont en fait la longueur des phrases
de texte (l'offset de la phrase suivante moins l'offset de la phrase courante est
la longueur de la phrase courante).

Il est très important de garder à l'esprit que l'endianness peut être confondu avec
un début de tableau incorrect.
En effet, dans la sortie d'\IT{od} nous voyons que chaque élément débutait par deux
zéros.
Mais lorsque nous décalons les des octets de chaque côté, nous pouvons interpréter
ce tableau comme little-endian:

\lstinputlisting{ff/fortune/3.lst}

Si nous interprétons ce tableau en little-endian, le premier élément est 0x4801,
le second est 0x7C01, etc.
La partie 8-bit haute de chacune de ces valeurs 16-bit nous semble être aléatoire,
et la partie 8-bit basse semble être ascendante.

Mais je suis sûr que c'est un tableau en big-endian, car le tout dernier élément
32-bit du fichier est big-endian.
(\IT{00 00 5f c4} ici):

\lstinputlisting{ff/fortune/4.lst}

Peut-être que le développeur du programme \IT{fortune} avait un ordinateur big-endian
ou peut-être a-t-il été porté depuis quelque chose comme ça.

Ok, donc le tableau est big-endian, et, à en juger avec bon sens, la toute première
phrase dans le fichier texte doit commencer à l'offset zéro. Donc la valeur zéro
doit se trouver dans le tableau quelque part au tout début.
Nous avons un couple d'élément à zéro au début. Mais le second est plus tentant:
43 se trouve juste après et 43 est un offset valide sur une phrase anglaise correcte
dans le fichier texte.

Le dernier élément du tableau est 0x5FC4, et il n'y a pas de tel octet à cet offset
dans le fichier texte.
Donc le dernier élément du tableau pointe au delà de la fin du fichier.
C'est probablement ainsi car la longueur de la phrase est calculée comme la différence
entre l'offset de la phrase courante et l'offset de la phrase suivante.
Ceci est plus rapide que de lire la chaîne à la recherche du caractère pourcent.
Mais ceci ne fonctionne pas pour le dernier élément.
Donc un élément \IT{fictif} est aussi ajouté à la fin du tableau.

Donc les 6 première valeur entière 32-bit sont une sorte d'en-tête.

Oh, j'ai oublié de compter les phrases dans le fichier texte:

\lstinputlisting{ff/fortune/5.lst}

Le nombre de phrases peut être présent dans dans l'index, mais peut-être pas.
Dans le cas de fichiers d'index simples, le nombre d'éléments peut être facilement
déduit de la taille du fichier d'index.
Quoiqu'il en soit, il y a 432 phrases dans le fichier texte.
Et nous voyons quelque chose de très familier dans le second élément (la valeur 431).
J'ai vérifié dans d'autres fichiers (literature.dat et riddles.dat sur Ubuntu Linux)
et oui, le second élément 32-bit est bien le nombre de phrases moins 1.
Pourquoi \IT{moins 1}? Peut-être que ceci n'est pas le nombre de phrases, mais plutôt
le numéro de la dernière phrase (commençant à zéro)?

Et il y a d'autres éléments dans l'entête.
Dans Mathematica, je charge chacun des trois fichiers disponible et je regarde leur en-tête:

\begin{figure}[H]
\centering
\myincludegraphics{ff/fortune/mathematica.png}
\end{figure}

Je n'ai aucune idée de la signification des autres valeurs, excepté la taille du
fichier d'index.
Certains champs sont les même pour tous les fichiers, d'autres non.
D'après mon expérience, ils peuvent être:

\begin{itemize}
\item signature de fichier;
\item version de fichier;
\item checksum;
\item des flags;
\item peut-être même des identifiants de langages;
\item timestamp du fichier texte, donc le programme \IT{fortune} regénèrerait le
fichier d'index isi l'utilisateur modifiait le fichier texte.
\end{itemize}

Par exemple, les fichiers Oracle .SYM (\myref{Oracle_SYM_files_example}) qui contiennent
la table des symboles pour les fichiers DLL, contiennent aussi un timestamp correspondant
au fichier DLL, afin d'être sûr qu'il est toujours valide.

D'un autre côté, les timestamps des fichiers textes et des fichiers d'index peuvent
être désynchronisés après archivage/désarchivage/installation/déploiement/etc.

Mais ce ne sont pas des timestamps, d'après moi. La manière la plus compacte de représenter
la date et l'heure est la valeur UNIX, qui est un nombre 32-bit. Nous ne voyons rien
de tel ici. D'autres façons de les représenter sont encore moins compactes.

Donc, voici supposément comment fonctionne l'algorithme de \IT{fortune}:

\begin{itemize}
\item prendre le nombre du second élément de la dernière phrase;
\item générer un nombre aléatoire dans l'intervalle 0..number\_of\_last\_phrase;
\item trouver l'élément correspondant dans le tableau des offsets, prendre aussi
l'offset suivant;
\item écrire sur \IT{stdout} tous les caractères depuis le fichier texte en commençant
à l'offset jusqu'à l'offset suivant moins 2 (afin d'ignorer le caractère pourcent
terminal et le caractère de la phrase suivante).
\end{itemize}

\subsection{Hacking}

Effectuons quelques essais afin de vérifier nos hypothèses.
Je vais créer ce fichier texte dans le chemin et le nom \IT{/usr/share/games/fortunes/fortunes}:

\begin{lstlisting}
Phrase one.
%
Phrase two.
%
\end{lstlisting}

Puis, ce fichier fortunes.dat. Je prend l'entête du fichier original fortunes.dat,
j'ai mis à zéro changé le second champ (nombre de phrases) et j'ai laissé deux éléments
dans le tableau: 0 et 0x1c, car la longueur totale du fichier texte \IT{fortunes}
est 28 (0x1c) octets:

\lstinputlisting{ff/fortune/6.lst}

Maintenant, je le lance:

\lstinputlisting{ff/fortune/7.lst}

Quelque chose ne va pas. Mettons le second champ à 1:

\lstinputlisting{ff/fortune/8.lst}

Maintenant, ça fonctionne. Il affiche seulement la première phrase:

\lstinputlisting{ff/fortune/9.lst}

Hmmm. Laissons seulement un élément dans le tableau (0) sans le terminer:

\lstinputlisting{ff/fortune/10.lst}

Le programme Fortune montre toujours seulement la première phrase.

De cet essai nous apprenons que la signe pourcent dans le fichier texte est analysé
et la taille n'est pas calculée comme je l'avais déduit avant, peut-être que même
l'élément terminal du tableau n'est pas utilisé.
Toutefois, il peut toujours l'être. Et peut-être qu'il l'a été dans le passé?

% TODO second phrase!

\subsection{Les fichiers}

Pour les besoins de la démonstration, je ne regarde toujours pas dans le code source
de \IT{fortune}.
Si vous soulez essayer de comprendre la signification des autres valeurs dans l'entête
du fichier d'index, vous pouvez essayer de le faire sans regarder dans le code source.
Les fichiers que j'ai utilisé sous Ubuntu Linux 14.04 sont ici: \url{http://beginners.re/examples/fortune/},
les fichiers bricolés le sont aussi.

Oh, j'ai pris les fichiers de la version x64 d'Ubuntu, mais les éléments du tableau
ont toujours une taille de 32 bit.
C'est parce que les fichiers de texte de \IT{fortune} ne vont sans doute jamais dépasser
une taille de 4\ac{GiB}.
Mais s'ils le devaient, tous les éléments devraient avoir une taille de 64 bit afin
de pouvoir stocker un offset de dans un fichier texte plus gros que 4GiB.

Pour les lecteurs impatients, le code source de \IT{fortune} est ici:
\url{https://launchpad.net/ubuntu/+source/fortune-mod/1:1.99.1-3.1ubuntu4}.

