\vspace*{\fill}

\huge
    请参加一个小调查
\normalsize

\bigskip
\bigskip
\bigskip

\dots 地址在此: \url{https://beginners.re/survey.html}。
您的参与对作者至关重要!

\fi % CHINESE

\ifdefined\CHINESE

\bigskip
\bigskip
\bigskip

\huge
    \CN{我能提供的服务}
\normalsize

\bigskip
\bigskip
\bigskip


您正阅读的本书是\href{http://beginners.re/}{免费的}并已经\href{https://github.com/dennis714/RE-for-beginners/}{以开放源代码的方式发布}。
但我有时需要做一些工作以获取收入。所以,很抱歉将我个人的广告信息发布在这里。

\iffalse
\Large 需要文档? \normalsize

我可以为一部分API、程序语言和框架等来编写文档/参考资料/使用手册。

通常,我擅于为各种API/程序语言的功能找出简明扼要的样例。
而本书便是样例之一。
我可以长期稳定地进行文档编写和维护工作。

另一方面,我的英语水平远未达到能称之为『熟练』的水平。
并且,我往往需要较长的时间以深入了解一个未知领域的产品。

然而,我将乐于重新修订现有的文档项目。

我所钦佩的,是Wolfram Mathematica所提供的文档: \url{http://reference.wolfram.com/language/}。
\fi

\Large 逆向工程 \normalsize

我并不能接受全职的工作请求。通常我可以远程完成一些小任务,例如:

\large 解密一个数据库,或对未知类型的文件进行管理 \normalsize

根据NDA协议,我不能在此披露更多关于案例的细节,但是在\myref{encrypted_DB1}一节阐述的内容很大一部分基于一个真实案例。

\large 将旧版本的EXE或DLL文件重写为C/C++代码 \normalsize

\large 加密狗\footnote{接在计算机上的外接设备,通常用于防止软件被拷贝或修改} \normalsize

我偶尔会制作\href{https://en.wikipedia.org/wiki/Software_protection_dongle}{加密狗}的替代品或模拟器。通常,破坏软件的保护措施是违反法律的,所以我仅在满足以下条件的时候才进行此类工作:

\begin{itemize}
\item 据我所知,开发该软件的软件公司已不复存在;
\item 该软件已经有10年以上的寿命;
\item 你持有一个可以从中读取信息的加密狗设备。换句话说,我只为那些仍在使用很老的软件、并完全满足于其功能,却担心因加密狗损坏而没有途径购买并替换的人服务。
\end{itemize}

上文所述的工作也涵盖了古老的MS-DOS和UNIX程序的情况。另外,我也接受基于其他计算机架构(如MIPS, DEC Alpha, PowerPC)的工作。

关于我的工作,您可以在这里看到一些样例:

\begin{itemize}
\item 我这本关于逆向工程的书中,有一节专门介绍了用于防止拷贝的加密狗设备: \ref{dongles}。
\item \href{http://yurichev.com/writings/z3_rockey.pdf}{仅通过输入、输出对和Z3 SMT求解器探索未知算法的文章}
\item \href{http://yurichev.com/blog/56/}{关于在DosBox中模拟MicroPhar(一种基于93c46的加密狗)的细节}
\item \href{http://conus.info/dongle/src/microph.asm}{基于EMM386的I/O中断API的DOS MicroPhar模拟器源码}
\end{itemize}

\large 联系我 \normalsize

E-Mail: \GTT{\EMAIL}

\large 依旧想要雇用全职的逆向工程师/安全研究员? \normalsize

您可以尝试\href{https://www.reddit.com/r/ReverseEngineering/comments/49cza0/rreverseengineerings_2015_triannual_hiring_thread/}{Reddit的逆向工程话题的招聘版面}。
另外,这里还有一个俄语论坛,其中包含了一个\href{https://forum.reverse4you.org/forumdisplay.php?f=252}{关于逆向工程工作的版面}。

\vspace*{\fill}
\vfill
