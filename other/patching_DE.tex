\section{Patchen von ausführbaren Dateien}

\subsection{Zeichenketten}

Zeichenketten in C können sehr einfach mit einem Hex-Editor verändert werden,
sofern sie nicht verschlüsselt sind.
Diese Technik kann sogar von denen angewandt werden, die nicht viel von Maschinencode
und Formaten von ausführbaren Dateien verstehen.
Die neue Zeichenkette darf nicht größer sein als die alte, da sonst die Gefahr
groß ist, dass ein anderer Wert oder Code überschrieben wird.
\myindex{MS-DOS}

Mit dieser Methode wurde in der MS-DOS-Ära eine Vielzahl von Software \IT{übersetzt},
zumindest in den ehemaligen USSR-Staaten in den 80er- und 90er-Jahren.
Aus diesem Grund existieren einige seltsame Abkürzungen in den \IT{übersetzten}
Programmen: es gab keinen Platz für längere Zeichenketten.

\myindex{Borland Delphi}

In Delphi müssen die Längen der Zeichenketten falls nötig korrigiert werden.

\subsection{x86-Code}
\label{x86_patching}

Häufige Aufgaben beim Patchen sind:

\myindex{x86!\Instructions!NOP}
\begin{itemize}

\item 
Eine der häufigsten Aufgaben ist das Deaktivieren bestimmter Anweisungen. Oft
wird dies durch Austauschen des Bytes durch \TT{0x90} (\ac{NOP}).

\item
Bedingte Sprünge, die den Opcode wie \TT{74 xx} (\JZ) haben, können durch
\ac{NOP}s ersetzt werden.

Es ist möglich alle bedingten Sprünge zu deaktivieren, in dem eine 0 in das
zweite Byte geschrieben wird (\IT{Sprung-Offset}).

\myindex{x86!\Instructions!JMP}
\item 
Eine weitere häufige Aufgabe ist es einen bedingten Sprung immer ausführen zu
lassen: dies kann durch Schreiben von \TT{0xEB}, was für \JMP steht, anstatt des
Opcodes erreicht werden.

\myindex{x86!\Instructions!RET}
\myindex{stdcall}
\item Die Ausführung einer Funktion kann deaktiviert werden, wenn \RETN (0xC3) an
den Anfang geschrieben wird. Dies gilt für alle Funktionen außer \TT{stdcall}
(\myref{sec:stdcall}).
Um \TT{stdcall}-Funktionen zu patchen muss die Anzahl der Argumente bekannt sein
(zum Beispiel durch Finden der \RETN-Anweisung in der Funktion) und die \RETN-Anweisung
mit einem 16-Bit-Argument (0xC2) angewendet werden.

\myindex{x86!\Instructions!MOV}
\myindex{x86!\Instructions!XOR}
\myindex{x86!\Instructions!INC}
\item Manchmal muss eine deaktivierte Funktion den Wert 0 oder 1 zurückgeben.
Dies kann durch \TT{MOV EAX, 0} oder \TT{MOV EAX, 1} erreicht werden, was aber relativ
ausführlich ist.
Ein besserer Weg ist \TT{XOR EAX, EAX} (2 Byte \TT{0x31 0xC0}) oder \TT{XOR EAX, EAX / INC EAX}
(3 Byte \TT{0x31 0xC0 0x40}).

\end{itemize}

Eine Software kann gegen Manipulation geschützt sein.

Dieser Schutz ist häufig realisiert indem der ausführbare Code gelesen und ein
passende Checksumme errechnet wird.
Aus diesem Grund muss der Code gelesen werden bevor die Schutzfunktion aktiviert
wird. Die Stelle kann durch setzen eines Breakpoints beim Lesen von Speicher
herausgefunden werden.

\myindex{tracer}
\tracer hat für diesen Zweck die BPM-Option.

Die Relocs (\myref{subsec:relocs}) in ausführbaren PE-Dateien sollten nicht
verändert werden, da der Windows-Laser den neuen, veränderten Code möglicherweise
überschreibt.
\myindex{Hiew}
(In Hiew sind die Stellen grau markiert, zum Beispiel: \figref{fig:scanf_ex3_hiew_1}).

Eine Möglichkeit ist es Sprünge zu schreiben, welche die Relocs umgehen oder die
Reloc-Tabelle muss editiert werden.
