\section{Anomalies des compilateurs}
\label{anomaly:Intel}
\myindex{\CompilerAnomaly}

\subsection{\oracle 11.2 et Intel C++ 10.1}

\myindex{Intel C++}
\myindex{\oracle}
\myindex{x86!\Instructions!JZ}

Le compilateur Intel C++ 10.1, qui fut utilisé pour la compilation de \oracle 11.2 pour Linux86, 
émettait parfois deux instructions \JZ successives, sans que la seconde instruction soit jamais 
référencée. Elle était donc inutile.

\begin{lstlisting}[caption=kdli.o from libserver11.a,style=customasmx86]
.text:08114CF1                   loc_8114CF1: ; CODE XREF: __PGOSF539_kdlimemSer+89A
.text:08114CF1                                ; __PGOSF539_kdlimemSer+3994
.text:08114CF1 8B 45 08              mov     eax, [ebp+arg_0]
.text:08114CF4 0F B6 50 14           movzx   edx, byte ptr [eax+14h]
.text:08114CF8 F6 C2 01              test    dl, 1
.text:08114CFB 0F 85 17 08 00 00     jnz     loc_8115518
.text:08114D01 85 C9                 test    ecx, ecx
.text:08114D03 0F 84 8A 00 00 00     jz      loc_8114D93
.text:08114D09 0F 84 09 08 00 00     jz      loc_8115518
.text:08114D0F 8B 53 08              mov     edx, [ebx+8]
.text:08114D12 89 55 FC              mov     [ebp+var_4], edx
.text:08114D15 31 C0                 xor     eax, eax
.text:08114D17 89 45 F4              mov     [ebp+var_C], eax
.text:08114D1A 50                    push    eax
.text:08114D1B 52                    push    edx
.text:08114D1C E8 03 54 00 00        call    len2nbytes
.text:08114D21 83 C4 08              add     esp, 8
\end{lstlisting}

\begin{lstlisting}[caption=from the same code,style=customasmx86]
.text:0811A2A5                   loc_811A2A5: ; CODE XREF: kdliSerLengths+11C
.text:0811A2A5                                ; kdliSerLengths+1C1
.text:0811A2A5 8B 7D 08              mov     edi, [ebp+arg_0]
.text:0811A2A8 8B 7F 10              mov     edi, [edi+10h]
.text:0811A2AB 0F B6 57 14           movzx   edx, byte ptr [edi+14h]
.text:0811A2AF F6 C2 01              test    dl, 1
.text:0811A2B2 75 3E                 jnz     short loc_811A2F2
.text:0811A2B4 83 E0 01              and     eax, 1
.text:0811A2B7 74 1F                 jz      short loc_811A2D8
.text:0811A2B9 74 37                 jz      short loc_811A2F2
.text:0811A2BB 6A 00                 push    0
.text:0811A2BD FF 71 08              push    dword ptr [ecx+8]
.text:0811A2C0 E8 5F FE FF FF        call    len2nbytes
\end{lstlisting}

Il s'agit probablement d'un bug du générateur de code du compilateur qui ne fut pas découvert durant 
les tests de celui-ci car le code produit fonctionnait conformément aux résultats attendus.

\subsection{MSVC 6.0}

Je viens juste de trouver celui-ci dans un vieux fragment de code :

\begin{lstlisting}[style=customasmx86]
                 fabs
                 fild    [esp+50h+var_34]
                 fabs
                 fxch    st(1) ; première instruction
                 fxch    st(1) ; seconde instruction
                 faddp   st(1), st
                 fcomp   [esp+50h+var_3C]
                 fnstsw  ax
                 test    ah, 41h
                 jz      short loc_100040B7
\end{lstlisting}

\myindex{x86!\Instructions!FXCH}
La première instruction \INS{FXCH} interverti les valeurs de \TT{ST(0)} et \TT{ST(1)}. La seconde 
effectue la même opération. Combinées, elles ne produisent donc aucun effet.
Cet extrait provient d'un programme qui utilise la librairie MFC42.dll, il a donc du être compilé 
avec MSVC 6.0 ou 5.0 ou peut-être même MSVC 4.2 qui date des années 90.

Cette paire d'instruction ne produit aucun effet, ce qui expliquerait qu'elle n'ait pas été détectée 
lors des tests du compilateur MSVC. Ou bien j'ai loupé quelque chose ...



\subsection{Résumé}

Des anomalies constatées dans d'autres compilateurs figurent également dans ce livre: 
\myref{anomaly:LLVM}, \myref{loops_iterators_loop_anomaly}, \myref{Keil_anomaly},
\myref{MSVC2013_anomaly},
\myref{MSVC_double_JMP_anomaly},
\myref{MSVC2012_anomaly}.

Ces cas sont exposés dans ce livre afin de démontrer que ces compilateurs comportent leurs propres 
erreurs et qu'il convient de ne pas toujours se triturer le cerveau en tentant de comprendre 
pourquoi le compilateur a généré un code aussi étrange.

