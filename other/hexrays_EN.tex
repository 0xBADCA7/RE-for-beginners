\section{My experience with Hex-Rays 2.2.0}
\myindex{Hex-Rays}
\label{hex_rays}

\subsection{Bugs}

There are couple of bugs.

First of all, Hex-Rays is getting lost when \ac{FPU} instructions are interleaved (by compiler codegenerator) with others.

For example, this:

\begin{lstlisting}
f               proc    near

        	lea     eax, [esp+4]
	        fild    dword ptr [eax]
                lea     eax, [esp+8]
        	fild    dword ptr [eax]
	        fabs
                fcompp
        	fnstsw  ax
	        test    ah, 1
                jz      l01

        	mov     eax, 1
	        retn
l01:
                mov     eax, 2
	        retn

f               endp
\end{lstlisting}

\dots will be correcly decompiled to:

\begin{lstlisting}
signed int __cdecl f(signed int a1, signed int a2)
{
  signed int result; // eax@2

  if ( fabs((double)a2) >= (double)a1 )
    result = 2;
  else
    result = 1;
  return result;
}
\end{lstlisting}

But let's comment one of the instructions at the end:

\begin{lstlisting}
...
l01:
	        ;mov    eax, 2
        	retn
...
\end{lstlisting}

\dots we getting an obvious bug:

\begin{lstlisting}
void __cdecl f(char a1, char a2)
{
  fabs((double)a2);
}
\end{lstlisting}

This is another bug:

\begin{lstlisting}
extrn f1:dword
extrn f2:dword

f               proc    near

	        fld     dword ptr [esp+4]
        	fadd    dword ptr [esp+8]
                fst     dword ptr [esp+12]
	        fcomp   ds:const_100
                fld     dword ptr [esp+16]      ; comment this instruction and it will be OK
	        fnstsw  ax
        	test    ah, 1

                jnz     short l01

	        call    f1
        	retn
l01:
	        call    f2
        	retn

f               endp

...

const_100       dd 42C80000h            ; 100.0
\end{lstlisting}

Result:

\begin{lstlisting}
int __cdecl f(float a1, float a2, float a3, float a4)
{
  double v5; // st7@1
  char v6; // c0@1
  int result; // eax@2

  v5 = a4;
  if ( v6 )
    result = f2(v5);
  else
    result = f1(v5);
  return result;
}
\end{lstlisting}

\IT{v6} variable has \IT{char} type and if you'll try to compile this code, compiler will warn you about variable
usage before assignment.

Another bug: \INS{FPATAN} instruction is correctly decompiled into \IT{atan2()}, but arguments are swapped.

\subsection{Odd peculiarities}

Hex-Rays too often promotes 32-bit \IT{int} to 64-bit one.
Here is example:

\begin{lstlisting}
f               proc    near

	        mov     eax, [esp+4]
                cdq
	        xor     eax, edx
        	sub     eax, edx
                ; EAX=abs(a1)

	        sub     eax, [esp+8]
        	; EAX=EAX-a2

                ; EAX at this point somehow gets promoted to 64-bit (RAX)

	        cdq
        	xor     eax, edx
                sub     eax, edx
	        ; EAX=abs(abs(a1)-a2)

                retn

f               endp
\end{lstlisting}

Result:

\begin{lstlisting}
int __cdecl f(int a1, int a2)
{
  __int64 v2; // rax@1

  v2 = abs(a1) - a2;
  return (HIDWORD(v2) ^ v2) - HIDWORD(v2);
}
\end{lstlisting}

Perhaps, this is result of \INS{CDQ} instruction? I'm not sure.
Anyway, whenever you see \IT{\_\_int64} type in 32-bit code, pay attention.

This is also weird:

\begin{lstlisting}
f               proc    near

	        mov     esi, [esp+4]

        	lea     ebx, [esi+10h]
                cmp     esi, ebx
	        jge     short l00

                cmp     esi, 1000
	        jg      short l00

                mov     eax, 2
	        retn

l00:
	        mov     eax, 1
        	retn

f               endp
\end{lstlisting}

Result:

\begin{lstlisting}
signed int __cdecl f(signed int a1)
{
  signed int result; // eax@3

  if ( __OFSUB__(a1, a1 + 16) ^ 1 && a1 <= 1000 )
    result = 2;
  else
    result = 1;
  return result;
}
\end{lstlisting}

The code is correct, but needs manual intervention.

Sometimes, Hex-Rays doesn't fold division by multiplication code:

\begin{lstlisting}
f               proc    near

        	mov     eax, [esp+4]
	        mov     edx, 2AAAAAABh
                imul    edx
        	mov     eax, edx

	        retn

f               endp
\end{lstlisting}

Result:

\begin{lstlisting}
int __cdecl f(int a1)
{
  return (unsigned __int64)(715827883i64 * a1) >> 32;
}
\end{lstlisting}

This can be folded manually.

Many of these peculiarities can be solved by manual reordering of instructions, recompiling assembly code,
and then feeding it to Hex-Rays again.

\subsection{Silence}

\begin{lstlisting}
extrn some_func:dword

f               proc    near

        	mov     ecx, [esp+4]
	        mov     eax, [esp+8]
                push    eax
        	call    some_func
	        add     esp, 4

                ; use ECX
        	mov     eax, ecx

	        retn

f               endp
\end{lstlisting}

Result:

\begin{lstlisting}
int __cdecl f(int a1, int a2)
{
  int v2; // ecx@1

  some_func(a2);
  return v2;
}
\end{lstlisting}

\IT{v2} variable (from ECX) is lost \dots
Yes, this code is incorrect (ECX value doesn't saved during call to another function),
but it would be good for Hex-Rays to give a warning.

Another one:

\begin{lstlisting}
extrn some_func:dword

f               proc    near

	        call    some_func
        	jnz     l01

                mov     eax, 1
	        retn
l01:
	        mov     eax, 2
        	retn

f               endp
\end{lstlisting}

Result:

\begin{lstlisting}
signed int f()
{
  char v0; // zf@1
  signed int result; // eax@2

  some_func();
  if ( v0 )
    result = 1;
  else
    result = 2;
  return result;
}
\end{lstlisting}

Again, warning would be great.

Anyway, whenever you see variable of \IT{char} type, or variable which is used without initialization, this is clear sign
that something went wrong and needs manual intervention.

\subsection{Comma}
\myindex{\CLanguageElements!Comma}

Comma in C/C++ has a bad fame, because it can lead to a confusing code.

Quick quiz, what does this C/C++ function returns?

\begin{lstlisting}
int f()
{
	return 1, 2;
};
\end{lstlisting}

It's 2: when compiler encounters comma-expression, it generates code which executes all sub-expressions, and
\IT{returns} value of the last sub-expression.

I've seen something like that in production code:

\begin{lstlisting}
if (cond)
	return global_var=123, 456; // 456 is returned
else
	return global_var=789, 321; // 321 is returned
\end{lstlisting}

Apparently, programmer wanted to make code slightly shorter without additional curly brackets.
In other words, comma allows to pack couple of expressions into one, without forming
statement/code block inside of curly brackets.

\myindex{Scheme}
\myindex{Racket}
Comma in C/C++ is close to \TT{begin} in Scheme/Racket: \url{https://docs.racket-lang.org/guide/begin.html}.

Perhaps, the only widely accepted usage of comma is in \IT{for()} statements:

\begin{lstlisting}
char *s="hello, world";
for(int i=0; *s; s++, i++);
; i = string lenght
\end{lstlisting}

Both \IT{s++} and \IT{i++} are executed at each loop iteration.

Read more: \url{http://stackoverflow.com/questions/52550/what-does-the-comma-operator-do-in-c}.

\myindex{\CLanguageElements!Short-circuit}
I'm writing all this because Hex-Rays produces (at least in my case) code which is rich with both commas and short-circuit
expressions.
For example, this is real output from Hex-Rays:

\begin{lstlisting}
 if ( a >= b || (c = a, (d[a] - e) >> 2 > f) )
    {
    	...
\end{lstlisting}

This is correct, it compiles and works, and let god help you to understand it.
Here is it rewritten:

\begin{lstlisting}
if (cond1 || (comma_expr, cond2))
{
	...
\end{lstlisting}

Short-circuit is effective here: first \IT{cond1} is checked, if it's \IT{true}, \IT{if()} body is executed, the rest
of \IT{if()} expression is ignored completely.
If \IT{cond1} is \IT{false}, \IT{comma\_expr} is executed (in the previous example, \IT{a} gets copied to \IT{c}),
then \IT{cond2} is checked.
If \IT{cond2} is \IT{true}, \IT{if()} body gets executed, or not.
In other words, \IT{if()} body gets executed if \IT{cond1} is \IT{true} or \IT{cond2} is \IT{true},
but if the latter is \IT{true}, \IT{comma\_expr} is also executed.

Now you can see why comma is so notorious.

\textbf{A word about short-circuit.}
A common beginner's misconception is that sub-conditions are checked in some unspecified order, which is not true.
In \TT{a | b | c} expression, $a$, $b$ and $c$ gets evaluated in unspecified order, so that is why \TT{||} has also been
added to C/C++, to apply short-circuit explicitly.

\subsection{Data types}

Data types is a problem for decompilers.

Hex-Rays can be blind to arrays in local stack, if they weren't set correctly before decompilation.
Same story about global arrays.

Another problem is too big functions, where a single slot in local stack can be used by several variables
across function's execution.
It's not a rare case when a slot is used for \IT{int}-variable, then for pointer, then for \IT{float}-variable.
Hex-Rays correctly decompiles it: it creates a variable with some type, then cast it to another type in various
parts of functions.
This problem has been solved by me by manual splitting big function into several smaller.
Just make local variables as global ones, etc, etc.
And don't forget about tests.

\subsection{Long and messed expressions}

Sometimes, during rewriting, you can end up with long and hard to understand expressions in \IT{if()} constructs, like:

\begin{lstlisting}
if ((! (v38 && v30 <= 5 && v27 != -1)) && ((! (v38 && v30 <= 5) && v27 != -1) || (v24 >= 5 || v26)) && v25)
{
...
}
\end{lstlisting}

Wolfram Mathematica can minimize some of them, using \TT{BooleanMinimize[]} function:

\begin{lstlisting}
In[1]:= BooleanMinimize[(! (v38 && v30 <= 5 && v27 != -1)) && v38 && v30 <= 5 && v25 == 0]

Out[1]:= v38 && v25 == 0 && v27 == -1 && v30 <= 5
\end{lstlisting}

There is even better way, to find common subexpressions:

\begin{lstlisting}
In[2]:= Experimental`OptimizeExpression[(! (v38 && v30 <= 5 && 
      v27 != -1)) && ((! (v38 && v30 <= 5) && 
      v27 != -1) || (v24 >= 5 || v26)) && v25]

Out[2]= Experimental`OptimizedExpression[
 Block[{Compile`$1, Compile`$2}, Compile`$1 = v30 <= 5; 
  Compile`$2 = 
   v27 != -1; ! (v38 && Compile`$1 && 
      Compile`$2) && ((! (v38 && Compile`$1) && Compile`$2) || 
     v24 >= 5 || v26) && v25]]
\end{lstlisting}

Mathematica adds two new variables: \TT{Compile`\$1} and \TT{Compile`\$2}, values of which will be used several times in expression.
So we can add two additional variables.

\subsection{My plan}

\begin{itemize}
\item Split big functions (and don't forget about tests).
Sometimes it's very helpful to form new functions out of big loop bodies.

\item Check/set data type of variables, arrays, etc.

\item If you see odd result, \IT{dangling} variable (which used before initialization), try to swap instructions manually,
recompile it and feed to Hex-Rays again.
\end{itemize}

\subsection{Summary}

Nevertheless, quality of Hex-Rays 2.2.0 is very, very good.
It makes life way easier.

