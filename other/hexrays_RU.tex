\section{Мой опыт с Hex-Rays 2.2.0}
\myindex{Hex-Rays}
\label{hex_rays}

\subsection{Ошибки}

Есть несколько ошибок.

Прежде всего, Hex-Rays теряется, когда инструкции \ac{FPU} перемежаются (кодегенератором компилятора) с другими.

Например:

\begin{lstlisting}
f               proc    near

        	lea     eax, [esp+4]
	        fild    dword ptr [eax]
                lea     eax, [esp+8]
        	fild    dword ptr [eax]
	        fabs
                fcompp
        	fnstsw  ax
	        test    ah, 1
                jz      l01

        	mov     eax, 1
	        retn
l01:
                mov     eax, 2
	        retn

f               endp
\end{lstlisting}

\dots будет корректино декомпилировано в:

\begin{lstlisting}
signed int __cdecl f(signed int a1, signed int a2)
{
  signed int result; // eax@2

  if ( fabs((double)a2) >= (double)a1 )
    result = 2;
  else
    result = 1;
  return result;
}
\end{lstlisting}

Но давайте закомментируем одну инструкцию в конце:

\begin{lstlisting}
...
l01:
	        ;mov    eax, 2
        	retn
...
\end{lstlisting}

\dots получаем явную ошибку:

\begin{lstlisting}
void __cdecl f(char a1, char a2)
{
  fabs((double)a2);
}
\end{lstlisting}

Вот еще ошибка:

\begin{lstlisting}
extrn f1:dword
extrn f2:dword

f               proc    near

	        fld     dword ptr [esp+4]
        	fadd    dword ptr [esp+8]
                fst     dword ptr [esp+12]
	        fcomp   ds:const_100
                fld     dword ptr [esp+16]      ; §закомментируйте эту инструкцию, и всё будет хорошо§
	        fnstsw  ax
        	test    ah, 1

                jnz     short l01

	        call    f1
        	retn
l01:
	        call    f2
        	retn

f               endp

...

const_100       dd 42C80000h            ; 100.0
\end{lstlisting}

Результат:

\begin{lstlisting}
int __cdecl f(float a1, float a2, float a3, float a4)
{
  double v5; // st7@1
  char v6; // c0@1
  int result; // eax@2

  v5 = a4;
  if ( v6 )
    result = f2(v5);
  else
    result = f1(v5);
  return result;
}
\end{lstlisting}

У переменной \IT{v6} тип \IT{char}, и если вы попытаетесь скомпилировать этот код, компилятор выдаст предупреждение о том,
что переменная используется перед тем, как была инициализирована.

Еще ошибка: инструкция \INS{FPATAN} корректно декомпилируется в \IT{atan2()}, но аргументы перепутаны.

\subsection{Странности}

Hex-Rays часто конвертирует 32-битный \IT{int} в 64-битный.
Вот пример:

\begin{lstlisting}
f               proc    near

	        mov     eax, [esp+4]
                cdq
	        xor     eax, edx
        	sub     eax, edx
                ; EAX=abs(a1)

	        sub     eax, [esp+8]
        	; EAX=EAX-a2

                ; §в этом месте, EAX каким-то образом был сконвертирован в 64-битный (RAX)§

	        cdq
        	xor     eax, edx
                sub     eax, edx
	        ; EAX=abs(abs(a1)-a2)

                retn

f               endp
\end{lstlisting}

Результат:

\begin{lstlisting}
int __cdecl f(int a1, int a2)
{
  __int64 v2; // rax@1

  v2 = abs(a1) - a2;
  return (HIDWORD(v2) ^ v2) - HIDWORD(v2);
}
\end{lstlisting}

Возможно, это результат инструкции \INS{CDQ}? Я не уверен.
Так или иначе, если вы видите тип \IT{\_\_int64} вы 32-битном коде, обращайте внимание.

Это тоже странно:

\begin{lstlisting}
f               proc    near

	        mov     esi, [esp+4]

        	lea     ebx, [esi+10h]
                cmp     esi, ebx
	        jge     short l00

                cmp     esi, 1000
	        jg      short l00

                mov     eax, 2
	        retn

l00:
	        mov     eax, 1
        	retn

f               endp
\end{lstlisting}

Результат:

\begin{lstlisting}
signed int __cdecl f(signed int a1)
{
  signed int result; // eax@3

  if ( __OFSUB__(a1, a1 + 16) ^ 1 && a1 <= 1000 )
    result = 2;
  else
    result = 1;
  return result;
}
\end{lstlisting}

Код корректный, но требует ручного вмешательства.

Иногда Hex-Rays не сокращает деление через умножение:

\begin{lstlisting}
f               proc    near

        	mov     eax, [esp+4]
	        mov     edx, 2AAAAAABh
                imul    edx
        	mov     eax, edx

	        retn

f               endp
\end{lstlisting}

Результат:

\begin{lstlisting}
int __cdecl f(int a1)
{
  return (unsigned __int64)(715827883i64 * a1) >> 32;
}
\end{lstlisting}

Это можно сократить вручную.

Многие из этих странностей можно решить при помощи ручного переупорядочивания инструкций, перекомпиляции ассемблерного кода,
и затем подачи его снова на вход Hex-Rays.

\subsection{Безмолвие}

\begin{lstlisting}
extrn some_func:dword

f               proc    near

        	mov     ecx, [esp+4]
	        mov     eax, [esp+8]
                push    eax
        	call    some_func
	        add     esp, 4

                ; используем ECX
        	mov     eax, ecx

	        retn

f               endp
\end{lstlisting}

Результат:

\begin{lstlisting}
int __cdecl f(int a1, int a2)
{
  int v2; // ecx@1

  some_func(a2);
  return v2;
}
\end{lstlisting}

Переменная \IT{v2} (из ECX) потерялась \dots
Да, код некорректный (значение в регистре ECX не сохраняется после вызова другой ф-ции),
но для Hex-Rays было бы неплохо выдать предупреждение.

Вот еще:

\begin{lstlisting}
extrn some_func:dword

f               proc    near

	        call    some_func
        	jnz     l01

                mov     eax, 1
	        retn
l01:
	        mov     eax, 2
        	retn

f               endp
\end{lstlisting}

Результат:

\begin{lstlisting}
signed int f()
{
  char v0; // zf@1
  signed int result; // eax@2

  some_func();
  if ( v0 )
    result = 1;
  else
    result = 2;
  return result;
}
\end{lstlisting}

И снова, предупреждение бы помогло.

Так или иначе, если вы видите переменную типа \IT{char}, которая используется без инициализации, это явный признак того,
что что-то пошло не так и требует ручного вмешательства.

\subsection{Запятая}
\myindex{\CLanguageElements!Запятая}

Запятая в Си/Си++ имеет дурную славу, потому что она приводит к малопонятному коду.

Вот простая задача, что возвращает эта ф-ция на Си/Си++?

\begin{lstlisting}
int f()
{
	return 1, 2;
};
\end{lstlisting}

Это 2: когда компилятор встречает выражение с запятой, он генерирует код, исполняющий все подвыражения, и \IT{возвращает}
значение последнего подвыражения.

Я видел такое в реальном коде:

\begin{lstlisting}
if (cond)
	return global_var=123, 456; // возвращается 456
else
	return global_var=789, 321; // возвращается 321
\end{lstlisting}

Вероятно, автор хотел сделать код немного короче, без дополнительных фигурных скобок.
Другими словами, запятая позволяет упаковать несколько выражений в одно, без формирования
блока внутри фигурных скобок.

\myindex{Scheme}
\myindex{Racket}
Запятая в Си/Си++ близка к \TT{begin} в Scheme/Racket: \url{https://docs.racket-lang.org/guide/begin.html}.

Вероятн, есть только одно популярное и всеми одобренное использование запятой, это в выражении \IT{for()}:

\begin{lstlisting}
char *s="hello, world";
for(int i=0; *s; s++, i++);
; i = длина строки
\end{lstlisting}

И \IT{s++} и \IT{i++} исполняются на каждой итерации цикла.

Читайте больше об этом: \url{http://stackoverflow.com/questions/52550/what-does-the-comma-operator-do-in-c}.

\myindex{\CLanguageElements!Short-circuit}
Я пишу всё это потому что Hex-Rays выдает код (как минимум, в моем случае) очень богатый и на запятые и на
short-circuit-выражения (\IT{короткое замыкание}).
Например, вот реальный пример работы Hex-Rays:

\begin{lstlisting}
 if ( a >= b || (c = a, (d[a] - e) >> 2 > f) )
    {
    	...
\end{lstlisting}

Это корректно, оно компилируется и работает, и да поможет вам бог понять, как.
Вот оно переписанное:

\begin{lstlisting}
if (cond1 || (comma_expr, cond2))
{
	...
\end{lstlisting}

Здесь работает \IT{short-circuit} (\IT{короткое замыкание}): в начале проверяется \IT{cond1}, и если оно истинно,
исполняется тело \IT{if()}, и остальная часть выражения \IT{if()} полностью игнорируется.
Если \IT{cond1} ложно, тогда исполняется \IT{comma\_expr} (в предыдущем примере, \IT{a} копируется в \IT{c}),
затем проверяется \IT{cond2}.
Если \IT{cond2} истинно, исполняется тело \IT{if()}, или нет.
Другими словами, тело \IT{if()} исполняется, если \IT{cond1} истинно, или если \IT{cond2} истинно,
но если последнее истинно, исполняется также \IT{comma\_expr}.

Теперь вы видите, почему запятая имеет такую славу.

\textbf{Еще о short-circuit (короткое замыкание).}
Частое заблуждение начинающих в том, что подусловия проверяются в каком-то незаданном порядке, и это не верно.
В выражении \TT{a | b | c}, $a$, $b$ и $c$ исполняются в незаданном порядке, и вот почему в Си/Си++ был также добавлен
оператор \TT{||}, чтобы явно применять \IT{short-circuit}.

\subsection{Типы данных}

Типы данных это проблема для декомпиляторов.

Hex-Rays может быть слеп к массивам в локальном стеке, если они не были корректно обозначены перед декомпиляцией.
Та же история с глобальными массивами.

Другая проблема это большие ф-ции, где один и тот же слот в локальном стеке может использоваться разными переменными
во время исполнения ф-ции.
Нередкий случай, когда слот в начале используется для \IT{int}-переменной, затем для указателя, затем для переменной типа
\IT{float}.
Hex-Rays корректно декомпилирует это: он создает переменную с каким-то типом, затем приводит её к другому типу в разных частях
ф-ции.
Эта проблема решалась мною ручным разделением ф-ции на несколько меньших.
Просто сделайте локальные переменные глобальными, итд, итд.
И не забывайте о тестах.

\subsection{Мой план}

\begin{itemize}
\item Разделить большие ф-ции (и не забывать о тестах).
Иногда очень полезно формировать новые ф-ции из тел циклов.

\item Проверяйте/устанавливайте тип данных для переменных, массивов, итд.

\item Если вы видите странный результат, \IT{висящую} переменную (которая используется перед инициализацией),
попробуйте поменять инструкции вручную, перекомпилировать и снова подать на вход Hex-Rays-у.
\end{itemize}

\subsection{Итог}

Тем не мнее, качество Hex-Rays 2.2.0 очень и очень хорошее.
Он делает жизнь легче.

