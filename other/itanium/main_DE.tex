\section{Itanium}
\label{itanium}
\myindex{Itanium}

Auch wenn fast gescheitert, ist der Intel Itanium (\ac{IA64}) eine sehr interessante
Architektur.

Während \ac{OOE}-CPUs entscheiden wie die Anweisungen neu organisiert werden und
diese parallel ausführen, war \ac{EPIC} ein Versuch diese Entscheidung dem Compiler
zu überlassen: das Gruppieren der Anweisungen soll während des Kompilierens erfolgen.

Dies führte zu einer berüchtigten Komplexität der Compiler.

Hier ist ein Beispiel von \ac{IA64}-Code, ein einfacher kryptografischer Algorithmus
aus dem Linux-Kernel:

\lstinputlisting[caption=Linux kernel 3.2.0.4,style=customc]{other/itanium/tea_from_linux.c}

Nachfolgend das Ergebnis des Compilers:

\lstinputlisting[caption=Linux Kernel 3.2.0.4 for Itanium 2 (McKinley)]{other/itanium/ia64_linux_3.2.0.4_mckinley.lst}

Zunächst sind alle \ac{IA64}-Anweisungen in Pakete von 3 Anweisungen zusammengefasst.

Jedes Paket hat eine Größe von 16 Byte (128 Bit) und besteht aus Template-Code
(5 Bit) und drei Anweisungen (je 41 Bit).

\IDA zeigt die Pakete als 6+6+4 Byte, das Muster ist leicht zu erkennen.

Alle drei Anweisungen von jedem Paket wird in der Regel gleichzeitig ausgeführt,
außer eine der Anweisungen enthält ein \q{Stop-Bit}.

%Supposedly, Intel and HP engineers gathered statistics on most frequent instruction patterns and decided to bring
%bundle types (\ac{AKA} \q{templates}): a bundle code defines the instruction types in the bundle.
%There are 12 of them.
%
%For example, the zeroth bundle type is \TT{MII}, which implies 
%the first instruction is Memory (load or store), the second and third ones are I (integer instructions).
%
%Another example is the bundle of type 0x1d: \TT{MFB}:
%the first instruction is Memory (load or store), the second one is Float 
%(\ac{FPU} instruction), and the third is Branch (branch instruction).
%
%If the compiler cannot pick a suitable instruction for the relevant bundle slot, it may insert a \ac{NOP}:
%you can see here the
%\TT{nop.i} instructions (\ac{NOP} at the place where the integer instruction might be) or \TT{nop.m} 
%(a memory instruction might be at this slot).
%
%\ac{NOP}s are inserted automatically when one uses assembly language manually.
%
%And that is not all. Bundles are also grouped.
%
%Each bundle may have a \q{stop bit},
%so all the consecutive bundles with a terminating bundle which has the \q{stop bit} 
%can be executed simultaneously.
%
%In practice, Itanium 2 can execute 2 bundles at once, resulting in the execution of 6 instructions at once.
%
%So all instructions inside a bundle and a bundle group cannot interfere with each other 
%(i.e., must not have data hazards).
%
%If they do, the results are to be undefined.
%
%Each stop bit is marked in assembly language as two semicolons (\TT{;;}) after the instruction.
%
%So, the instructions at [90-ac] may be executed simultaneously:
%they do not interfere. The next group is [b0-cc].
%
%We also see a stop bit at 10c.
%The next instruction at 110 has a stop bit too.
%
%This implies that these instructions must be executed isolated from all others (as in \ac{CISC}).
%
%Indeed: the next instruction at 110 uses the result from the previous one (the value in register r26),
%so they cannot be executed at the same time.
%
%Apparently, the compiler was not able to find a better way to parallelize the instructions,
%in other words, to load \ac{CPU} as much as possible, hence too much stop bits and \ac{NOP}s.
%
%Manual assembly programming is a tedious job as well: the programmer has to group the instructions manually.
%
%The programmer is still able to add stop bits to each instructions, but this will degrade
%the performance that Itanium was made for.
%
%An interesting examples of manual \ac{IA64} assembly code can be found in the Linux kernel's sources:
%
%\url{http://go.yurichev.com/17322}.
%
%Another introductory paper on Itanium assembly:
%[Mike Burrell, \IT{Writing Efficient Itanium 2 Assembly Code} (2010)]\footnote{\AlsoAvailableAs \url{http://yurichev.com/mirrors/RE/itanium.pdf}},
%[papasutra of haquebright, \IT{WRITING SHELLCODE FOR IA-64} (2001)]\footnote{\AlsoAvailableAs \url{http://phrack.org/issues/57/5.html}}.
%
%Another very interesting Itanium feature is the \IT{speculative execution} and the NaT (\q{not a thing}) bit,
%somewhat resembling \gls{NaN} numbers: \\
%\href{http://go.yurichev.com/17323}{MSDN}.
%
