\section{Itanium}
\label{itanium}
\myindex{Itanium}

Auch wenn fast gescheitert, ist der Intel Itanium (\ac{IA64}) eine sehr interessante
Architektur.

Während \ac{OOE}-CPUs entscheiden wie die Anweisungen neu organisiert werden und
diese parallel ausführen, war \ac{EPIC} ein Versuch diese Entscheidung dem Compiler
zu überlassen: das Gruppieren der Anweisungen soll während des Kompilierens erfolgen.

Dies führte zu einer berüchtigten Komplexität der Compiler.

Hier ist ein Beispiel von \ac{IA64}-Code, ein einfacher kryptografischer Algorithmus
aus dem Linux-Kernel:

\lstinputlisting[caption=Linux kernel 3.2.0.4,style=customc]{other/itanium/tea_from_linux.c}

Nachfolgend das Ergebnis des Compilers:

\lstinputlisting[caption=Linux Kernel 3.2.0.4 for Itanium 2 (McKinley)]{other/itanium/ia64_linux_3.2.0.4_mckinley.lst}

Zunächst sind alle \ac{IA64}-Anweisungen in Pakete von 3 Anweisungen zusammengefasst.

Jedes Paket hat eine Größe von 16 Byte (128 Bit) und besteht aus Template-Code
(5 Bit) und drei Anweisungen (je 41 Bit).

\IDA zeigt die Pakete als 6+6+4 Byte, das Muster ist leicht zu erkennen.

Alle drei Anweisungen von jedem Paket wird in der Regel gleichzeitig ausgeführt,
außer eine der Anweisungen enthält ein \q{Stop-Bit}.

Vermutlich haben die Intel- und HP-Ingenieure Statistiken über die am meisten verwendeten
Anweisungsmuster erhoben und entschieden die Pakettypen zu erstellen (\ac{AKA} \q{Templates}): 
ein Paket-Code definiert den Anweisungstyp im Paket.
Es existieren 12 von ihnen.

Beispielsweise ist der nullte Pakettyp \TT{MII}, was impliziert, dass die erste
Anweisung Speicher (Lesen oder Schreiben) ist und die zweite und dritte jeweils
eine Integer-Anweisung ist.

Ein weiteres Beispiel ist das Paket vom Typ 0x1d: \TT{MFB}:
die erste Anweisung ist betrifft wieder den Speicher (Lesen oder Schreiben), die
zweite eine Fließkomma (\ac{FPU} Anweisung) und die dritte ein Springbefehl.

Wenn der Compiler keine passende Anweisung für den entsprechenden Paketplatz finden
kann, ist es möglich, dass er ein \ac{NOP} einfügt: man kann hier die \TT{nop.i}-Anweisung
(\ac{NOP} anstelle einer Integer-Anweisung) oder \TT{nop.m} (anstelle einer Speicheroperation)
sehen .

\ac{NOP}s werden automatisch eingefügt wenn mit Assembler gearbeitet wird.

Dies ist nicht alles: Pakete können ebenfalls grupiert werden.

Jedes Paket kann ein  \q{Stop-Bit} enthalten, so dass alle aufeinander folgenden
Pakete mit einem terminierenden Paket (mit \q{Stop-Bit}) gleichzeitig verarbeitet
werden können.

In der Praxis kann Itanium 2 gleichzeitig zwei Pakete ausführen, was zu sechs
Anweisungen führt.

Also kann keine der Anweisungen innerhalb einer Paket-Gruppe mit einer anderen
interagieren (es kann also nicht zu Datenkonflikten kommen).

Falls sie auftreten können die Ergebnisse undefiniert sein.

Jedes Stop-Bit ist in Assembler mit zwei Semikolons (\TT{;;}) nach de Anweisung markiert.

Die Anweisungen bei [90-ac] können also simultan ausgeführt werden: sie beeinflussen
sich gegenseitig nicht. Die nächste Gruppe ist [b0-cc].

Hier ist auch das Stop-Bit bei 10c zu sehen
Die nächste Anweisung bei 110 hat ebenfalls ein Stop-Bit.

Dies impliziert dass diese Anweisungen von allen anderen getrennt ausgeführt werden
müssen (wie in \ac{CISC}).

Außerdem ist zu sehen, dass die Anweisung nach 110 das Ergebnis der vorangehenden
benutzt (Den Wert im Register r26), dementsprechend können sie nicht gleichzeitig
ausgeführt werden.

Anscheinend war der Compiler nicht in der Lage einen besseren Weg zum Parallelisieren
der Anweisungen zu finden, also die \ac{CPU} so weit wie möglich auszulasten. Daher
die vielen Stop-Bits und \ac{NOP}-Anweisungen.

Manuelle Assembler-Programmierung ist ein mühsamer Job: der Programmierer muss
die Anweisungen selber in Gruppen einteilen.

Der Programmierer ist immer noch in der Lage Stop-Bits zu jeder Anweisung hinzuzufügen,
doch dies wird die Geschwindigkeit heruntersetzen für die Itanium gemacht wurde.

Ein interessantes Beispiel von manuellem Assembler-Code in \ac{IA64} kann im Code
des Linux-Kernels gefunden werden:

\url{http://go.yurichev.com/17322}.

Eine weitere Einführung für den Itanium-Assembler:
[Mike Burrell, \IT{Writing Efficient Itanium 2 Assembly Code} (2010)]\footnote{\AlsoAvailableAs \url{http://yurichev.com/mirrors/RE/itanium.pdf}},
[papasutra of haquebright, \IT{WRITING SHELLCODE FOR IA-64} (2001)]\footnote{\AlsoAvailableAs \url{http://phrack.org/issues/57/5.html}}.

Weitere sehr interessante Itanium-Features sind \IT{speculative execution} und das
NaT (\q{not a thing})-Bit, was in gewisser Weise \gls{NaN}-Zahlen ähnelt:

\href{http://go.yurichev.com/17323}{MSDN}
