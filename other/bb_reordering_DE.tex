\section{Basic Block Reordering}

% TODO __builtin_expect in GCC?

\subsection{Profile-guided Optimization}
\label{PGO}

\myindex{\oracle}
\myindex{Intel C++}

Diese Optimierungsmethode kann einige \gls{basic block}s zu anderen Sektionen der
ausführbaren Datei verschieben.

%Obviously, there are parts of a function which are executed more frequently (e.g., loop bodies)
%and less often (e.g., error reporting code, exception handlers).
%
%The compiler adds instrumentation code into the executable, then the developer runs it with
%a lot of tests to collect statistics.
%
%Then the compiler, with the help of the statistics gathered,
%prepares final the executable file with all infrequently executed code moved into another section.
%
%As a result, all frequently executed function code is compacted, and that is very important
%for execution speed and cache usage.
%
%An example from \oracle code, which was compiled with Intel C++:
%
%\begin{lstlisting}[caption=orageneric11.dll (win32),style=customasmx86]
%                public _skgfsync
%_skgfsync       proc near
%
%; address 0x6030D86A
%
%                db      66h
%                nop
%                push    ebp
%                mov     ebp, esp
%                mov     edx, [ebp+0Ch]
%                test    edx, edx
%                jz      short loc_6030D884
%                mov     eax, [edx+30h]
%                test    eax, 400h
%                jnz     __VInfreq__skgfsync  ; write to log
%continue:
%                mov     eax, [ebp+8]
%                mov     edx, [ebp+10h]
%                mov     dword ptr [eax], 0
%                lea     eax, [edx+0Fh]
%                and     eax, 0FFFFFFFCh
%                mov     ecx, [eax]
%                cmp     ecx, 45726963h
%                jnz     error                ; exit with error
%                mov     esp, ebp
%                pop     ebp
%                retn
%_skgfsync       endp
%
%...
%
%; address 0x60B953F0
%
%__VInfreq__skgfsync:
%                mov     eax, [edx]
%                test    eax, eax
%                jz      continue
%                mov     ecx, [ebp+10h]
%                push    ecx
%                mov     ecx, [ebp+8]
%                push    edx
%                push    ecx
%                push    offset ... ; "skgfsync(se=0x%x, ctx=0x%x, iov=0x%x)\n"
%                push    dword ptr [edx+4]
%                call    dword ptr [eax] ; write to log
%                add     esp, 14h
%                jmp     continue
%
%error:
%                mov     edx, [ebp+8]
%                mov     dword ptr [edx], 69AAh ; 27050 "function called with invalid FIB/IOV structure"
%                mov     eax, [eax]
%                mov     [edx+4], eax
%                mov     dword ptr [edx+8], 0FA4h ; 4004
%                mov     esp, ebp
%                pop     ebp
%                retn
%; END OF FUNCTION CHUNK FOR _skgfsync
%\end{lstlisting}
%
%The distance of addresses between these two code fragments is almost 9 MB.
%
%All infrequently executed code was placed at the end of the code section of the DLL file,
%among all function parts.
%
%This part of the function was marked by the Intel C++ compiler with the \TT{VInfreq} prefix.
%
%Here we see that a part of the function that writes to a log file (presumably in case of error or warning
%or something like that) which was probably not executed very often when Oracle's developers gathered 
%statistics (if it was executed at all).
%
%The writing to log basic block eventually returns the control flow to the \q{hot} part of the function.
%
%Another \q{infrequent} part is the \gls{basic block} returning error code 27050.
%
%In Linux ELF files, all infrequently executed code is moved by Intel C++ into the separate 
%\TT{text.unlikely} section, leaving all \q{hot} code in the \TT{text.hot} section.
%
%From a reverse engineer's perspective, this information may help to split the function
%into its core and error handling parts.
