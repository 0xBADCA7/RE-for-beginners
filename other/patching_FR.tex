\section{Modification de fichier exécutable}

\subsection{Chaînes de caractères}

Les chaînes de caractères du langage C sont les plus faciles à modifier (sauf si elles sont chiffrées)
avec un simple éditeur hexadécimal. La technique peut être mise en oeuvre même par ceux qui ne 
connaissent ni le langage machine, ni le format des fichiers exécutables.
Il faut toutefois que la nouvelle chaîne de caractères ne soit pas plus longue que l'ancienne, sinon 
il existe un risque d'écraser une autre valeur ou du code exécutable.

\myindex{MS-DOS}

C'est en utilisant cette méthode que de nombreux programmes ont été traduits à l'époque de MS-DOS,
du moins dans les années 80 et 90 dans l'ancienne URSS. Elle aboutissait parfois à la présence de 
quelques abbréviations folkloriques dans la traduction, faute de place pour des chaînes plus longues.

\myindex{Borland Delphi}

En ce qui concerne Delphi, la taille de la chaîne de caractères doit elle aussi être ajustée.

\subsection{code x86}
\label{x86_patching}

Les tâches de modification les plus courantes sont:

\myindex{x86!\Instructions!NOP}
\begin{itemize}

\item 
L'un des travaux les plus courant consiste à désactiver une instruction en l'écrasant avec des 
octets \TT{0x90} (\ac{NOP}).

\item Les branchements conditionnels qui utilisent un code instruction tel que \TT{74 xx} (\JZ), 
peuvent être réécrits avec deux instructions \ac{NOP}.

Une autre technique consiste à désactiver un branchement conditionnel en écrasant le second octet 
avec la valeur 0 (\IT{jump offset}).

\myindex{x86!\Instructions!JMP}
\item 
Une autre tâche courante consiste à faire en sorte qu'un branchement conditionnel soit effectué 
systématiquement. On y parvient en remplacant le code instruction par \TT{0xEB} qui correspond à 
l'instruction \JMP.

\myindex{x86!\Instructions!RET}
\myindex{stdcall}
\item L'exécution d'une fonction peut être désactivée en remplacant le premier octet par \RETN (0xC3).
Les fonctions dont la convention d'appel est \TT{stdcall} (\myref{sec:stdcall}) font exception. 
Pour les modifier, il faut déterminer le nombre d'arguments (par exemple en trouvant une instruction 
\RETN au sein de la fonction), puis en utilisant l'instruction \RETN accompagnée d'un argument sur 
deux octets (0xC2).

\myindex{x86!\Instructions!MOV}
\myindex{x86!\Instructions!XOR}
\myindex{x86!\Instructions!INC}
\item Il arrive qu'une fonction que l'on a désactivé doive retourner une valeur 0 ou 1. Certes on 
peut utiliser \TT{MOV EAX, 0} ou \TT{MOV EAX, 1}, mais cela occupe un peu trop d'espace.\\
Une meilleure approche consiste à utiliser \TT{XOR EAX, EAX} (2 octets \TT{0x31 0xC0}) ou 
\TT{XOR EAX, EAX / INC EAX} (3 octets \TT{0x31 0xC0 0x40}).

\end{itemize}

Un logiciel peut être protégé contre les modifications. Le plus souvent la protection consiste à 
lire le code du programme (en mémoire) et à en calculer une valeur de contrôle.
Cette technique nécessite que la protection lise le code avant de pouvoir agir. Elle peut donc être 
détectée en positionnant un point d'arrêt déclenché par la lecture de la mémoire contenant le code.

\myindex{tracer}
\tracer possède l'option BPM pour ce faire.

La partie du fichier au format PE qui contient les informations de relocation (\myref{subsec:relocs}) 
ne doivent pas être modifiées par les patchs car le chargeur Windows risquerait d'écraser les 
modifications que apportées.
\myindex{Hiew}
(Ces parties sont présentées sous forme grisées dans Hiew, par exemple:
\figref{fig:scanf_ex3_hiew_1}).

En dernier ressort, il est possible d'effectuer des modifications qui contournent les relocations, 
ou de modifier directement la table des relocations.
