\subsubsection{ARM}

\myparagraph{\NonOptimizingKeilVI (\ARMMode)}

\lstinputlisting[label=Keil_number_sign,style=customasm]{patterns/09_loops/simple/ARM/Keil_ARM_O0.asm}

Der Z�hler $i$ wird im Register \Reg{4} gespeichert. 
Der Befehl \INS{MOV R4, \#2} initialisiert $i$.
Die Befehle \INS{MOV R0, R4} und \INS{BL printing\_function} bilden den K�rper
der Schleife; der erste Befehl bereitet das Argument f�r die \ttf Funktion vor
und der zweite ruft die Funktion auf.

\myindex{ARM!\Instructions!ADD}
Der Befehl \INS{ADD R4, R4, \#1} erh�ht die $i$ Variable in jedem Durchlauf um
1.
\myindex{ARM!\Instructions!CMP}
\myindex{ARM!\Instructions!BLT}
\INS{CMP R4, \#0xA} vergleicht $i$ mit \TT{0xA} (10). 
Der n�chste Befehl \INS{BLT} (\IT{Branch Less Than}) springt, falls $i$ kleiner
als 10 ist. Sonst wird 0 in das Register \Reg{0} geschrieben (unsere Funktion
liefert den Wert 0 zur�ck) und die Funktionsausf�hrung wird beendet.

\myparagraph{\OptimizingKeilVI (\ThumbMode)}

\lstinputlisting[style=customasm]{patterns/09_loops/simple/ARM/Keil_thumb_O3.asm}

Praktisch das gleiche.

\myparagraph{\OptimizingXcodeIV (\ThumbTwoMode)}
\label{ARM_unrolled_loops}

\lstinputlisting[style=customasm]{patterns/09_loops/simple/ARM/xcode_thumb_O3.asm}

In meiner \ttf Funktion befand sich tats�chlich Folgendes:

\begin{lstlisting}
void printing_function(int i)
{
    printf ("%d\n", i);
};
\end{lstlisting}

\myindex{Unrolled loop}
\myindex{Inline code}
Also hat LLVM die Schleife nicht nur \IT{unrolled} sondern auch die einfache
Funktion \ttf \IT{inlined} und den K�rper der Schleife acht Mal generiert,
anstatt die Schleife aufzurufen. 

Dies ist m�glich, wenn eine Funktion sehr einfach ist (wie meine) und wenn sie
nicht allzu oft aufgerufen wird (wie hier).


\myparagraph{ARM64: \Optimizing GCC 4.9.1}

\lstinputlisting[caption=\Optimizing GCC
4.9.1,style=customasm]{patterns/09_loops/simple/ARM/ARM64_GCC491_O3_DE.s}

\myparagraph{ARM64: \NonOptimizing GCC 4.9.1}

\lstinputlisting[caption=\NonOptimizing GCC 4.9.1
-fno-inline,style=customasm]{patterns/09_loops/simple/ARM/ARM64_GCC491_O3_DE.s}
