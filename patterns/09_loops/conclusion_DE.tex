\subsection{\Zusammenfassung{}}

Gerüst einer Schleife von einschließlich 2 bis einschließlich 9:

\lstinputlisting[caption=x86,style=customasmx86]{patterns/09_loops/skeleton_x86_2_9_optimized_DE.lst}

Das Inkrementieren kann in nicht optimiertem Code durch 3 Instruktionen
dargestellt werden:

\lstinputlisting[caption=x86,style=customasmx86]{patterns/09_loops/skeleton_x86_2_9_DE.lst}

Falls der Körper einer Schleife besonders kurz ist, kann ein Register als Zähler
verwendet werden:

\lstinputlisting[caption=x86,style=customasmx86]{patterns/09_loops/skeleton_x86_2_9_reg_DE.lst}

Einige Teile der Schleife können vom Compiler in unterschiedlichen Reihenfolgen
generiert werden:

\lstinputlisting[caption=x86,style=customasmx86]{patterns/09_loops/skeleton_x86_2_9_order_DE.lst}

Normalerweise wird die Bedingung \IT{vor} dem Körper geprüft, aber der Compiler
kann den Code auch so anordnen, dass die Bedingung \IT{nach} dem Körper geprüft
wird.

Dies geschieht dann, wenn der Compiler sicher sein kann, dass die Bedingung im
ersten Durchlauf stets \IT{wahr} ist, sodass der Körper der Schleife mindestens
einmal tatsächlich ausgeführt wird:

\lstinputlisting[caption=x86,style=customasmx86]{patterns/09_loops/skeleton_x86_2_9_reorder_DE.lst}

\myindex{x86!\Instructions!LOOP}

Verwendung des \IT{LOOP} Befehls. Sehr selten, Compiler verwenden ihn nicht.
Wenn er auftaucht, ist dies ein Zeichen dafür, dass das entsprechende
Codesegment von Hand geschrieben worden ist:

 
\lstinputlisting[caption=x86,style=customasmx86]{patterns/09_loops/skeleton_x86_loop_DE.lst}

ARM. 

Das \Reg{4} Register fungiert in diesem Beispiel als Zähler:

 
\lstinputlisting[caption=ARM,style=customasmARM]{patterns/09_loops/skeleton_ARM_DE.lst}

% TODO MIPS
