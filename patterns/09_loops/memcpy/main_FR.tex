\subsection{Routine de copie de blocs de mémoire}
\label{loop_memcpy}

Les routines réelles de copie de mémoire copient 4 ou 8 octets à chaque itération,
utilisent \ac{SIMD}, la vectorisation, etc.
Mais dans un but didactique, cet exemple est le plus simple possible.

\lstinputlisting[style=customc]{memcpy.c}

\subsubsection{Implémentation simple}

\lstinputlisting[caption=GCC 4.9 x64 optimisé pour la taille (-Os),style=customasmx86]{patterns/09_loops/memcpy/memcpy_GCC49_x64_Os_FR.s}

\lstinputlisting[caption=GCC 4.9 ARM64 optimisé pour la taille (-Os),style=customasmARM]{patterns/09_loops/memcpy/memcpy_GCC49_ARM64_Os_FR.s}

\lstinputlisting[caption=\OptimizingKeilVI (\ThumbMode),style=customasmARM]{patterns/09_loops/memcpy/memcpy_Keil_Thumb_O3_FR.s}

\subsubsection{ARM en mode ARM}

Keil en mode ARM tire pleinement avantage des suffixes conditionnels:

\lstinputlisting[caption=\OptimizingKeilVI (\ARMMode),style=customasmARM]{patterns/09_loops/memcpy/memcpy_Keil_ARM_O3_FR.s}

C'est pourquoi il y a seulement une instruction de branchement au lieu de 2.

\subsubsection{MIPS}

\lstinputlisting[caption=GCC 4.4.5 optimisé pour la taille (-Os) (IDA),style=customasmMIPS]{patterns/09_loops/memcpy/memcpy_MIPS_Os_IDA_FR.lst}

\myindex{MIPS!\Instructions!LBU}
\myindex{MIPS!\Instructions!SB}

Nous avons ici deux nouvelles instructions: \INS{LBU} (\q{Load Byte Unsigned} charger
un octet non signé) et \INS{SB} (\q{Store Byte} stocker un octet).

Tout comme en ARM, tous les registres MIPS ont une taille de 32-bit, il n'y en a
pas d'un octet de large comme en x86.

Donc, lorsque l'on travaille avec des octets seuls, nous devons utiliser un registre
de 32-bit pour chacun d'entre eux.

\INS{LBU} charge un octet et met les autres bits à zéro (\q{Unsigned}).
\myindex{MIPS!\Instructions!LB}

En revanche, l'instruction \INS{LB} (\q{Load Byte}) étend le signe de l'octet chargé
sur 32-bit.

\INS{SB} écrit simplement un octet depuis les 8 bits de poids faible d'un registre
dans la mémoire.

\subsubsection{Vectorisation}

GCC \Optimizing peut faire beaucoup mieux avec cet exemple: \myref{vec_memcpy}.
