\subsubsection{MIPS}

\myindex{MIPS!\Registers!FCCR}
Le coprocesseur du processeur MIPS possède un bit de condition qui peut être mis
par le FPU et lu par le CPU.

Les premiers MIPSs avaient seulement un bit de condition (appelé FCC0), les derniers
en ont 8 (appelés FCC7-FCC0).

Ce bit (ou ces bits) sont situés dans un registre appelé FCCR.

\lstinputlisting[caption=\Optimizing GCC 4.4.5 (IDA),style=customasmMIPS]{patterns/12_FPU/3_comparison/MIPS_O3_IDA_FR.lst}

\myindex{MIPS!\Instructions!C.LT.D}
\INS{C.LT.D} compare deux valeurs.
\GTT{LT} est la condition \q{Less Than} (plus petit que).
\GTT{D} implique des valeurs de type \Tdouble.
Suivant le résultat de la comparaison, le bit de condition FCC0 est mis à 1 ou à
0.

\myindex{MIPS!\Instructions!BC1T}
\myindex{MIPS!\Instructions!BC1F}
\INS{BC1T} teste le bit FCC0 et saute si le bit est mis à 1.
\GTT{T} signifie que le saut sera effectué si le bit est mis à 1 (\q{True}).
Il y a aussi une instruction \INS{BC1F} qui saute si le bit n'est pas mis (donc
est à 0) (\q{False}).

Dépendant du saut, un des arguments de la fonction est placé dans \$F0.
