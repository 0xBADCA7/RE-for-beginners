\subsubsection{ARM}

\myparagraph{\OptimizingXcodeIV (\ARMMode)}

\lstinputlisting[caption=\OptimizingXcodeIV (\ARMMode),style=customasmARM]{patterns/12_FPU/3_comparison/ARM/Xcode_ARM_FR.lst}

\myindex{ARM!\Registers!APSR}
\myindex{ARM!\Registers!FPSCR}
Un cas très simple.
Les valeurs en entrée sont placées dans les registres \GTT{D17} et \GTT{D16} puis
comparées en utilisant l'instruction \INS{VCMPE}.

Tout comme dans le coprocesseur x86, le coprocesseur ARM a son propre registre de
flags (\ac{FPSCR}), puisqu'il est nécessaire de stocker des flags spécifique au coprocesseur.
% TODO -> расписать регистр по битам
\myindex{ARM!\Instructions!VMRS}
Et tout comme en x86, il n'y a pas d'instruction de saut conditionel qui teste des
bits dans le registre de status du coprocesseur.
Donc il y a \INS{VMRS}, qui copie 4 bits (N, Z, C, V) du mot d'état du coprocesseur
dans les bits du registre de status \IT{général} (\ac{APSR}).

\myindex{ARM!\Instructions!VMOVGT}
\INS{VMOVGT} est l'analogue de l'instruction \INS{MOVGT} pour D-registres, elle s'exécute
si une opérande est plus grande que l'autre lors de la comparaison (\IT{GT---Greater Than}).

Si elle est exécutée, la valeur de $a$ sera écrite dans \GTT{D16} (ce qui est écrit
en ce moment dans \GTT{D17}).
Sinon, la valeur de $b$ reste dans le registre \GTT{D16}.

\myindex{ARM!\Instructions!VMOV}

La pénultième instruction \INS{VMOV} prépare la valeur dans la registre \GTT{D16}
afin de la renvoyer dans la paire de registres \Reg{0} et \Reg{1}.

\myparagraph{\OptimizingXcodeIV (\ThumbTwoMode)}

\begin{lstlisting}[caption=\OptimizingXcodeIV (\ThumbTwoMode),style=customasmARM]
VMOV            D16, R2, R3 ; b
VMOV            D17, R0, R1 ; a
VCMPE.F64       D17, D16
VMRS            APSR_nzcv, FPSCR
IT GT 
VMOVGT.F64      D16, D17
VMOV            R0, R1, D16
BX              LR
\end{lstlisting}

Presque comme dans l'exemple précédent, toutefois légèrement différent.
Comme nous le savons déjà, en mode ARM, beaucoup d'instructions peuvent avoir un
prédicat de condition.
Mais il n'y a rien de tel en mode Thumb.
Il n'y a pas d'espace dans les instructions sur 16-bit pour 4 bits dans lesquels
serait encodée la condition.

\myindex{ARM!\ThumbTwoMode}

Toutefois, cela à été étendu en un mode Thumb-2 pour rendre possible de spécifier
un prédicat aux instructions de l'ancien mode Thumb.
Ici, dans le listing généré par \IDA, nous voyons l'instruction \INS{VMOVGT}, comme
dans l'exemple précédent.

En fait, le \INS{VMOV} usuel est encodé ici, mais \IDA lui ajoute le suffixe \GTT{-GT},
puisque que l'instruction \INS{IT GT} se trouve juste avant.

\label{ARM_Thumb_IT}
\myindex{ARM!\Instructions!IT}
\myindex{ARM!if-then block}
L'instruction \INS{IT} défini ce que l'on appelle un \IT{bloc if-then}.

Après cette instruction, il est possible de mettre jusqu'à 4 instructions, chacune
d'entre elles ayant un suffixe de prédicat.
Dans notre exemple, \INS{IT GT} implique que l'instruction suivante ne sera exécutée
que si la condition \IT{GT} (\IT{Greater Than} plus grand que) est vraie.

\myindex{Angry Birds}
Voici un exemple de code plus complexe, à propos, d'Angry Birds (pour iOS):

\begin{lstlisting}[caption=Angry Birds Classic,style=customasmARM]
...
ITE NE
VMOVNE          R2, R3, D16
VMOVEQ          R2, R3, D17
BLX             _objc_msgSend ; not suffixed
...
\end{lstlisting}

\INS{ITE} est l'acronyme de \IT{if-then-else}

et elle encode un suffixe pour les deux prochaines instructions.

La première instruction est exécutée si la condition emcodée dans \INS{ITE} (\IT{NE, not equal})
est vraie, et la seconde---si la condition n'est pas vraie (l'inverse de la condition
\GTT{NE} est \GTT{EQ} (\IT{equal})).

L'instruction qui suit le second \INS{VMOV} (ou \INS{VMOVEQ}) est normale, non suffixée
(\INS{BLX}).

\myindex{Angry Birds}
Un autre exemple qui est légèrement plus difficile, qui est aussi d'Angry Birds:

\begin{lstlisting}[caption=Angry Birds Classic,style=customasmARM]
...
ITTTT EQ
MOVEQ           R0, R4
ADDEQ           SP, SP, #0x20
POPEQ.W         {R8,R10}
POPEQ           {R4-R7,PC}
BLX             ___stack_chk_fail ; not suffixed
...
\end{lstlisting}

Les quatres symboles \q{T} dans le mnémonique de l'instruction signifient que les quatres
instructions suivantes seront exécutées si la condition est vraie.

C'est pourquoi \IDA ajoute le suffixe \GTT{-EQ} à chacune d'entre elles.

Et si il y avait, par exemple, \INS{ITEEE EQ} (\IT{if-then-else-else-else}),
alors les suffixes seraient mis comme suit:

\begin{lstlisting}
-EQ
-NE
-NE
-NE
\end{lstlisting}

\myindex{Angry Birds}
Un autre morceau de code d'Angry Birds:

\begin{lstlisting}[caption=Angry Birds Classic,style=customasmARM]
...
CMP.W           R0, #0xFFFFFFFF
ITTE LE
SUBLE.W         R10, R0, #1
NEGLE           R0, R0
MOVGT           R10, R0
MOVS            R6, #0         ; not suffixed
CBZ             R0, loc_1E7E32 ; not suffixed
...
\end{lstlisting}

\INS{ITTE} (\IT{if-then-then-else})

implique que les 1ère et 2ème instructions seront exécutées si la condition \GTT{LE}
(\IT{Less or Equal} moins ou égal) est vraie, et que la 3ème---si la condition inverse
(\GTT{GT}---\IT{Greater Than} plus grand que) est vraie.

En général, les compilateurs ne génèrent pas toutes les combinaisons possible.
\myindex{Angry Birds}

Par exemple, dans le jeu Angry Birds mentionné ((\IT{classic} version pour iOS)
seules les les variantes suivantes de l'instruction \INS{IT} sont utilisées:
\INS{IT}, \INS{ITE}, \INS{ITT}, \INS{ITTE}, \INS{ITTT}, \INS{ITTTT}.
\myindex{\GrepUsage}
Comment savoir cela?
Dans \IDA, il est possible de produire un listing dans un fichier, ce qui a été utilisé
pour en créer un avec l'option d'afficher 4 octets pour chaque opcode.
Ensuite, en connaissant la partie haute de l'opcode de 16-bit (\GTT{0xBF} pour \INS{IT}),
nous utilisons \GTT{grep} ainsi:

\begin{lstlisting}
cat AngryBirdsClassic.lst | grep " BF" | grep "IT" > results.lst
\end{lstlisting}

\myindex{ARM!\ThumbTwoMode}

À propos, si vous programmez en langage d'assemblage ARM pour le mode Thumb-2, et
que vous ajoutez des suffixes conditionnels, l'assembleur ajoutera automatiquement
l'instructions \INS{IT} avec les flags là où ils sont nécessaires.

\myparagraph{\NonOptimizingXcodeIV (\ARMMode)}

\begin{lstlisting}[caption=\NonOptimizingXcodeIV (\ARMMode),style=customasmARM]
b               = -0x20
a               = -0x18
val_to_return   = -0x10
saved_R7        = -4

                STR             R7, [SP,#saved_R7]!
                MOV             R7, SP
                SUB             SP, SP, #0x1C
                BIC             SP, SP, #7
                VMOV            D16, R2, R3
                VMOV            D17, R0, R1
                VSTR            D17, [SP,#0x20+a]
                VSTR            D16, [SP,#0x20+b]
                VLDR            D16, [SP,#0x20+a]
                VLDR            D17, [SP,#0x20+b]
                VCMPE.F64       D16, D17
                VMRS            APSR_nzcv, FPSCR
                BLE             loc_2E08
                VLDR            D16, [SP,#0x20+a]
                VSTR            D16, [SP,#0x20+val_to_return]
                B               loc_2E10

loc_2E08
                VLDR            D16, [SP,#0x20+b]
                VSTR            D16, [SP,#0x20+val_to_return]

loc_2E10
                VLDR            D16, [SP,#0x20+val_to_return]
                VMOV            R0, R1, D16
                MOV             SP, R7
                LDR             R7, [SP+0x20+b],#4
                BX              LR
\end{lstlisting}

Presque la même chose que nous avons déjà vu, mais ici il y a beaucoup de code redondant
car les variables $a$ et $b$ sont stockées sur la pile locale, tout comme la valeur
de retour.

\myparagraph{\OptimizingKeilVI (\ThumbMode)}

\begin{lstlisting}[caption=\OptimizingKeilVI (\ThumbMode),style=customasmARM]
                PUSH    {R3-R7,LR}
                MOVS    R4, R2
                MOVS    R5, R3
                MOVS    R6, R0
                MOVS    R7, R1
                BL      __aeabi_cdrcmple
                BCS     loc_1C0
                MOVS    R0, R6
                MOVS    R1, R7
                POP     {R3-R7,PC}

loc_1C0
                MOVS    R0, R4
                MOVS    R1, R5
                POP     {R3-R7,PC}
\end{lstlisting}


Keil ne génère pas les instructions pour le FPU car il ne peut pas être sûr qu'elles
sont supportées sur le CPU cible, et cela ne peut pas être fait directement en comparant
les bits.
%TODO1: why?
Donc il appelle une fonction d'une bibliothèque externe pour effectuer la comparaison:
\GTT{\_\_aeabi\_cdrcmple}.
\myindex{ARM!\Instructions!BCS}

N.B. Le résultat de la comparaison est laissé dans les flags par cette fonction,
donc l'instruction \INS{BCS} (\IT{Carry set---Greater than or equal} plus grand ou
égal) fonctionne sans code additionel.
