\subsubsection{ARM64}

\myparagraph{\Optimizing GCC (Linaro) 4.9}

\lstinputlisting[style=customasmARM]{patterns/12_FPU/3_comparison/ARM/ARM64_GCC_O3_DE.lst}
Der ARM64 \ac{ISA} verfügt über FPU-Befehle, die der Einfachheit halber die Flags der CPU \ac{APSR} anstelle von
\ac{FPSCR} setzen.
Die \ac{FPU} ist hier kein separates Gerät mehr (zumindest logisch).

\myindex{ARM!\Instructions!FCMPE}
Wir finden hier \INS{FCMPE}. Er vergleicht die beiden über \RegD{0} und \RegD{1} übergebenen Werte (dabei handelt es
sich um das erste und zweite Argument der Funktion) und setzt \ac{APSR} die Flags (N, Z, C, V).

\myindex{ARM!\Instructions!FCSEL}
\INS{FCSEL} (\IT{Floating Conditional Select}) kopiert den Wert von \RegD{0} oder \RegD{1} nach \RegD{0}, abhängig von
der Bedingung (\GTT{GT}---Greater Than), und verwendet wiederum Flags im \ac{APSR} Register anstatt derer von
\ac{FPSCR}.

Dies ist im Vergleich zum Befehlssatz alter CPUs deutlich praktischer.

Falls die Bedingung wahr ist (\GTT{GT}), dann wird der Wert von \RegD{0} nach \RegD{0} kopiert (d.h. es geschieht
nichts).
Falls die Bedingung falsch ist, wird der Wert von \RegD{1} nach \RegD{0} kopiert.

\myparagraph{\NonOptimizing GCC (Linaro) 4.9}

\lstinputlisting[style=customasmARM]{patterns/12_FPU/3_comparison/ARM/ARM64_GCC_DE.lst}
Der nicht optimierende GCC ist weniger kompakt.

Zunächst speichert die Funktion ihre Eingabewerte auf dem lokalen Stack (\IT{Register Save Area}), danach lädt der Code
die Werte erneut in die Register \RegX{0}/\RegX{1} und kopiert sie schließlich nach \RegD{0}/\RegD{1}, um sie mittels
\INS{FCMPE} zu vergleichen.
Eine Menge redundanter Code, aber so arbeitet ein nicht optimierender Compiler nun einmal.
\INS{FCMPE} vergleich die Werte und setzt die \ac{APSR} Flags.
Zu diesem Zeitpunkt entscheidet sich der Compiler noch nicht für den praktischeren \INS{FCSEL} Befehl und arbeitet
stattdessen mit herkömmlichen Methoden:
er verwendet den \INS{BLE} Befehl (\IT{Branch if Less than or Equal}).
Im ersten Fall ($a>b$) wird der Wert von $a$ nach \RegX{0} geladen. 
Im anderen Fall ($a<=b$) wird der Wert von $b$ nach \RegX{0} geladen.
Schließlich wird der Wert aus \RegX{0} nach \RegD{0} kopiert, denn der Rückgabewert muss sich in diesem Register
befinden.


\mysubparagraph{\Exercise}
Dem Leser bleibt als Übung, den vorstehenden Code zu optimieren, indem manuell die redundanten Instruktionen entfernt
werden ohne dabei neue einzuführen (wie etwa \INS{FCSEL}).

\myparagraph{\Optimizing GCC (Linaro) 4.9---float}
Wir wollen nun dieses Beispiel umschreiben, indem wir \Tfloat anstelle von \Tdouble verwenden.

\begin{lstlisting}[style=customc]
float f_max (float a, float b)
{
	if (a>b)
		return a;

	return b;
};
\end{lstlisting}

\lstinputlisting[style=customasmARM]{patterns/12_FPU/3_comparison/ARM/ARM64_GCC_O3_float_DE.lst}
Es ist der gleiche Code, aber hier werden die S-Register anstelle der D-Register verwendet.
Das ist darauf zurückzuführen, dass der \Tfloat Typ in 32-Bit-S-Registern übergeben wird (welche in Wirklichkeit nichts
anderes als die niederen Teile der 64-Bit-D-Register sind).
