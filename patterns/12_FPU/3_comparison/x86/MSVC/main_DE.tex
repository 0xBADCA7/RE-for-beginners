\myparagraph{\NonOptimizing MSVC}

MSVC 2010 erzeugt den folgenden Code:

\lstinputlisting[caption=\NonOptimizing MSVC
2010,style=customasmx86]{patterns/12_FPU/3_comparison/x86/MSVC/MSVC_DE.asm}

\myindex{x86!\Instructions!FLD}

Der Befehl \FLD lädt \GTT{\_b} nach \ST{0}.

\label{Czero_etc}
\newcommand{\Czero}{\GTT{C0}\xspace}
\newcommand{\Ctwo}{\GTT{C2}\xspace}
\newcommand{\Cthree}{\GTT{C3}\xspace}
\newcommand{\CThreeBits}{\Cthree/\Ctwo/\Czero}

\myindex{x86!\Instructions!FCOMP}
\FCOMP verlgeicht den Wert in \ST{0} mit dem Wert, der sich in \GTT{\_a}
befindet und setzt die \CThreeBits im FPU Status Register entsprechend.
Das Statusregister ist ein 16-Bit-Register, das den aktueller Zustand der FPU
abbildet.

Nachdem die Bits gesetzt worden sind, nimmer der \FCOMP Befehl auch eine
Variable vom Stack. Dieses Verhalten unterscheidet ihn von \FCOM, der einfach
zwei Werte vergleicht und den Stack unangetastet lässt.

Leider verfügen CPUs vor Intel P6\footnote{Intel P6 ist Pentium Pro, Pentium II,
etc.}über keinerlei bedingte Sprungbefehle, die die \CThreeBits prüfen.

After the bits are set, the \FCOMP instruction also pops one variable from the stack. 
This is what distinguishes it from \FCOM, which is just compares values, leaving the stack in the same state.
Vielleicht ist diese Tatsache historisch begründet (man erinnere sich: die FPU
war früher ein eigener Chip).\\
Moderne CPUs, beginnend mit Intel P6 haben \FCOMI/\FCOMIP/\FUCOMI/\FUCOMIP
Befehle~--welche im Prinzip das gleiche tun, aber die \ZF/\PF/\CF Flags der CPU
verändern können.

\myindex{x86!\Instructions!FNSTSW}
Der \FNSTSW Befehl kopiert das FPU Statusregister nach \AX.
\CThreeBits werden an den Stellen 14/10/8 abgelegt, sie befinden sich im \AX
Register an den gleichen Stellen und sie werden alle in höherwertigen Teil von
\AX{}~---\AH{} abgelegt.

\begin{itemize}
\item Falls in unserem Beispiel $b>a$, dann werden die \CThreeBits Bits wie
folgt gesetzt: 0, 0, 0.
\item Falls $a>b$, dann ist das Bitmuster: 0, 0, 1.
\item Falls $a=b$, dann ist das Bitmuster: 1, 0, 0.
\item

Wenn das Ergebnis (z.B. im Fehlerfall) ungeordnet ist, dann werden die Bits wie
folgt gesetzt: 1,1,1.
\end{itemize}
% TODO: table here?
So werden die \CThreeBits Bits im \AX Register angeordnet:

\begin{center}
\begin{bytefield}[endianness=big,bitwidth=0.03\linewidth]{16}
\bitheader{14,10,9,8} \\
\bitbox{1}{} & 
\bitbox{1}{C3} & 
\bitbox{3}{} & 
\bitbox{1}{C2} & 
\bitbox{1}{C1} & 
\bitbox{1}{C0} &
\bitbox{8}{}
\end{bytefield}
\end{center}


So werden die \CThreeBits Bits im \AH Register angeordnet:

\begin{center}
\begin{bytefield}[endianness=big,bitwidth=0.03\linewidth]{8}
\bitheader{6,2,1,0} \\
\bitbox{1}{} & 
\bitbox{1}{C3} & 
\bitbox{3}{} & 
\bitbox{1}{C2} & 
\bitbox{1}{C1} & 
\bitbox{1}{C0}
\end{bytefield}
\end{center}

Nach der Ausführung von \INS{test ah, 5}\footnote{5=101b} werden nur die \Czero
und \Ctwo Bits (an den Stellen 0 und 2) betrachtet, alle übrigen Bits werden
einfach überlesen.

\label{parity_flag}
\myindex{x86!\Registers!\Flags!Parity flag}
Werfen wir nun einen Blick auf ein anderes bemerkenswertes historisches
Überbleibsel: das \IT{parity flag}.

Dieses Flag wird auf 1 gesetzt, falls die Anzahl der Einsen im Ergebnis der
letzten Berechnung gerade ist und auf 1, falls dies nicht der Fall ist.

Schlagen wir in der Wikipedia
nach\footnote{\href{http://go.yurichev.com/17131}{wikipedia.org/wiki/Parity\_flag}}:

%TODO Quotation has been translated from English wiki article, since the
% correspondig German article doesn't offer such information.
\begin{framed}
\begin{quotation}
Ein guter Grund das Parity Flag abzufragen, hat tatsächlich gar nichts mit
Parität zu tun. Die FPU hat vier Bedingungsflags (C0 bis C3), aber diese können
nicht direkt abgefragt werden, sondern müssen zunächst in das Flags Register
kopiert werden. Wenn dies geschieht, wird C0 im Carry Flag abgelegt, C2 im
Parity Flag und C3 im Zero Flag.
Das C2 Flag ist gesetzt, wenn z.B. unvergleichbare Fließkommawerte (NaN oder
nicht unterstütztes Format) über der \FUCOM Befehl miteinander verglichen
werden.\textit{(Übersetzung aus der englischen Wikipedia.)}
\end{quotation}
\end{framed}

Wie in der Wikipedia dargestellt wird das Parity Flag manchmal im FPU Code
verwendet; schauen wir uns genauer an wie das funktioniert.

\myindex{x86!\Instructions!JP}
Das \PF Flag wird auf 1 gesetzt, wenn sowohl \Czero als auch \Ctwo beide 0 oder
beide 1 sind. In diesem Fall wird der nachfolgende Sprung \JP(\IT{jump if
PF==1}) ausgeführt.
Wenn wir die Werte der \CThreeBits in den unterschiedlichen Fällen betrachten,
dann sehen wir, dass der bedingte Sprung \JP in zwei Fällen ausgeführt wird:
wenn $b>a$ oder wenn $a=b$ (das \Cthree Bit wird hier nicht betrachtet, da es
durch den Befehl \INS{test ah,5}) gelöscht wurde).

Der Rest ist leicht nachvollziehbar.
Denn der bedingte Sprung ausgeführt wurde, lädt \FLD den Wert von \GTT{\_b} nach
\ST{0} und wenn nicht, wird der Wert von \GTT{\_a} dorthin geladen.

\mysubparagraph{Was ist mit der Abfrage von \Ctwo?}
Das \Ctwo Flag wird im Fehlerfall (\gls{NaN}, etc.) gesetzt, aber unser Code
prüft dies nicht. 
Wenn sich der Programmierer für FPU Fehler interessiert, muss er zusätzliche
Abfragen hinzufügen.

\clearpage
\mysubparagraph{Erstes \olly Beispiel: a=1.2 und b=3.4}
\myindex{\olly}

Laden wir das Beispiel in \olly:

\begin{figure}[H]
\centering
\myincludegraphics{patterns/12_FPU/3_comparison/x86/MSVC/olly1_1.png}
\caption{\olly: erstes \FLD wurde ausgeführt}
\label{fig:FPU_comparison_case1_olly1}
\end{figure}
Die aktuellen Parameter der Funktion sind: $a=1.2$ und $b=3.4$ (Wir finden sie
auf dem Stack zwei 32-Bit-Werte). $b$ (3.4) wurde bereits nach \ST{0} geladen.
Jetzt wird \FCOMP ausgeführt. 
\olly zeigt das zweite Argument von \FCOMP, welches sich jetzt auf dem Stack
befindet.

\clearpage
\FCOMP wurde ausgeführt:

\begin{figure}[H]
\centering
\myincludegraphics{patterns/12_FPU/3_comparison/x86/MSVC/olly1_2.png}
\caption{\olly: \FCOMP wurde ausgeführt}
\label{fig:FPU_comparison_case1_olly2}
\end{figure}
Wir sehen den Status der Flags der \ac{FPU}: sämtlich null. 
Den geholten Wert finden wir unter \ST{7} wieder, auf die Gründe dafür sind
wir bereits eingegangen:\myref{FPU_is_rather_circular_buffer}.

\clearpage
\FNSTSW wurde ausgeführt:
\begin{figure}[H]
\centering
\myincludegraphics{patterns/12_FPU/3_comparison/x86/MSVC/olly1_3.png}
\caption{\olly: \FNSTSW wurde ausgeführt}
\label{fig:FPU_comparison_case1_olly3}
\end{figure}
Wir sehen, dass das \GTT{AX} Register Nullen enthält: das passt, da alle Flag
auf Null gesetzt sind. (\olly disassembliert hier den Befehl \FNSTSW als
\INS{FSTSW}--die beiden sind synonym).

\clearpage
\TEST wurde ausgeführt:

\begin{figure}[H]
\centering
\myincludegraphics{patterns/12_FPU/3_comparison/x86/MSVC/olly1_4.png}
\caption{\olly: \TEST wurde ausgeführt}
\label{fig:FPU_comparison_case1_olly4}
\end{figure}

Das \GTT{PF} Flag ist auf 1 gesetzt.

Begründung: die Anzahö der auf null gesetzten Bits ist null, und Null ist eine
gerade Zahl.
\olly disassembliert \INS{JP} als \ac{JPE}--die beiden sind synonym--und dieser
Befehl wird als nächstes ausgeführt.

\clearpage
\ac{JPE} wird ausgeführt, \FLD lädt den Wert von $b$ (3.4) nach \ST{0}:

\begin{figure}[H]
\centering
\myincludegraphics{patterns/12_FPU/3_comparison/x86/MSVC/olly1_5.png}
\caption{\olly: das zweite \FLD wurde ausgeführt}
\label{fig:FPU_comparison_case1_olly5}
\end{figure}

Die Funktion beendet ihre Arbeit.

\clearpage
\mysubparagraph{Zweites \olly Beispiel: a=5.6 und b=-4}

Laden wir unser Beispiel in \olly:

\begin{figure}[H]
\centering
\myincludegraphics{patterns/12_FPU/3_comparison/x86/MSVC/olly2_1.png}
\caption{\olly: erstes \FLD ausgeführt}
\label{fig:FPU_comparison_case2_olly1}
\end{figure}
Die aktuellen Funktionsparameter sind: $a=5.6$ und $b=-4$.
$b$ (-4) wurde bereits nach \ST{0} geladen.
\FCOMP wird jetzt ausgeführt.
\olly zeigt das zweite Argument von \FCOMP, welches sich jetzt auf dem Stack
befindet.

\clearpage
\FCOMP wird ausgeführt:

\begin{figure}[H]
\centering
\myincludegraphics{patterns/12_FPU/3_comparison/x86/MSVC/olly2_2.png}
\caption{\olly: \FCOMP wird ausgeführt}
\label{fig:FPU_comparison_case2_olly2}
\end{figure}
Wir betrachten den Status der \ac{FPU} Flas: alle null, außer \Czero.

\clearpage
\FNSTSW wird ausgeführt:

\begin{figure}[H]
\centering
\myincludegraphics{patterns/12_FPU/3_comparison/x86/MSVC/olly2_3.png}
\caption{\olly: \FNSTSW wird ausgeführt}
\label{fig:FPU_comparison_case2_olly3}
\end{figure}
Wir sehen, dass das \GTT{AX} Register den Wert \GTT{0x100} enthält: das \Czero
Flag sitzt auf dem achten Bit.

\clearpage
\TEST wird ausgeführt:

\begin{figure}[H]
\centering
\myincludegraphics{patterns/12_FPU/3_comparison/x86/MSVC/olly2_4.png}
\caption{\olly: \TEST wird ausgeführt}
\label{fig:FPU_comparison_case2_olly4}
\end{figure}

Das \GTT{PF}  Flag wird gelöscht. 
Begründung: die Anzahl der gesetzten Bits in \GTT{0x100} ist 1 und 1 ist eine
ungerade Zahl. \ac{JPE} wird jetzt übersprungen.

\clearpage
\ac{JPE} wurde nicht ausgelöst, sodass \FLD den Wert von $a$ (5.6) nach
\ST{0} lädt:

\begin{figure}[H]
\centering
\myincludegraphics{patterns/12_FPU/3_comparison/x86/MSVC/olly2_5.png}
\caption{\olly: zweites \FLD wurde ausgeführt}
\label{fig:FPU_comparison_case2_olly5}
\end{figure}

Die Funktion beendet ihre Arbeit.

