\clearpage
\mysubparagraph{Erstes \olly Beispiel: a=1.2 und b=3.4}
\myindex{\olly}

Laden wir das Beispiel in \olly:

\begin{figure}[H]
\centering
\myincludegraphics{patterns/12_FPU/3_comparison/x86/MSVC/olly1_1.png}
\caption{\olly: erstes \FLD wurde ausgeführt}
\label{fig:FPU_comparison_case1_olly1}
\end{figure}
Die aktuellen Parameter der Funktion sind: $a=1.2$ und $b=3.4$ (Wir finden sie
auf dem Stack zwei 32-Bit-Werte). $b$ (3.4) wurde bereits nach \ST{0} geladen.
Jetzt wird \FCOMP ausgeführt. 
\olly zeigt das zweite Argument von \FCOMP, welches sich jetzt auf dem Stack
befindet.

\clearpage
\FCOMP wurde ausgeführt:

\begin{figure}[H]
\centering
\myincludegraphics{patterns/12_FPU/3_comparison/x86/MSVC/olly1_2.png}
\caption{\olly: \FCOMP wurde ausgeführt}
\label{fig:FPU_comparison_case1_olly2}
\end{figure}
Wir sehen den Status der Flags der \ac{FPU}: sämtlich null. 
Den geholten Wert finden wir unter \ST{7} wieder, auf die Gründe dafür sind
wir bereits eingegangen:\myref{FPU_is_rather_circular_buffer}.

\clearpage
\FNSTSW wurde ausgeführt:
\begin{figure}[H]
\centering
\myincludegraphics{patterns/12_FPU/3_comparison/x86/MSVC/olly1_3.png}
\caption{\olly: \FNSTSW wurde ausgeführt}
\label{fig:FPU_comparison_case1_olly3}
\end{figure}
Wir sehen, dass das \GTT{AX} Register Nullen enthält: das passt, da alle Flag
auf Null gesetzt sind. (\olly disassembliert hier den Befehl \FNSTSW als
\INS{FSTSW}--die beiden sind synonym).

\clearpage
\TEST wurde ausgeführt:

\begin{figure}[H]
\centering
\myincludegraphics{patterns/12_FPU/3_comparison/x86/MSVC/olly1_4.png}
\caption{\olly: \TEST wurde ausgeführt}
\label{fig:FPU_comparison_case1_olly4}
\end{figure}

Das \GTT{PF} Flag ist auf 1 gesetzt.

Begründung: die Anzahö der auf null gesetzten Bits ist null, und Null ist eine
gerade Zahl.
\olly disassembliert \INS{JP} als \ac{JPE}--die beiden sind synonym--und dieser
Befehl wird als nächstes ausgeführt.

\clearpage
\ac{JPE} wird ausgeführt, \FLD lädt den Wert von $b$ (3.4) nach \ST{0}:

\begin{figure}[H]
\centering
\myincludegraphics{patterns/12_FPU/3_comparison/x86/MSVC/olly1_5.png}
\caption{\olly: das zweite \FLD wurde ausgeführt}
\label{fig:FPU_comparison_case1_olly5}
\end{figure}

Die Funktion beendet ihre Arbeit.

\clearpage
\mysubparagraph{Zweites \olly Beispiel: a=5.6 und b=-4}

Laden wir unser Beispiel in \olly:

\begin{figure}[H]
\centering
\myincludegraphics{patterns/12_FPU/3_comparison/x86/MSVC/olly2_1.png}
\caption{\olly: erstes \FLD ausgeführt}
\label{fig:FPU_comparison_case2_olly1}
\end{figure}
Die aktuellen Funktionsparameter sind: $a=5.6$ und $b=-4$.
$b$ (-4) wurde bereits nach \ST{0} geladen.
\FCOMP wird jetzt ausgeführt.
\olly zeigt das zweite Argument von \FCOMP, welches sich jetzt auf dem Stack
befindet.

\clearpage
\FCOMP wird ausgeführt:

\begin{figure}[H]
\centering
\myincludegraphics{patterns/12_FPU/3_comparison/x86/MSVC/olly2_2.png}
\caption{\olly: \FCOMP wird ausgeführt}
\label{fig:FPU_comparison_case2_olly2}
\end{figure}
Wir betrachten den Status der \ac{FPU} Flas: alle null, außer \Czero.

\clearpage
\FNSTSW wird ausgeführt:

\begin{figure}[H]
\centering
\myincludegraphics{patterns/12_FPU/3_comparison/x86/MSVC/olly2_3.png}
\caption{\olly: \FNSTSW wird ausgeführt}
\label{fig:FPU_comparison_case2_olly3}
\end{figure}
Wir sehen, dass das \GTT{AX} Register den Wert \GTT{0x100} enthält: das \Czero
Flag sitzt auf dem achten Bit.

\clearpage
\TEST wird ausgeführt:

\begin{figure}[H]
\centering
\myincludegraphics{patterns/12_FPU/3_comparison/x86/MSVC/olly2_4.png}
\caption{\olly: \TEST wird ausgeführt}
\label{fig:FPU_comparison_case2_olly4}
\end{figure}

Das \GTT{PF}  Flag wird gelöscht. 
Begründung: die Anzahl der gesetzten Bits in \GTT{0x100} ist 1 und 1 ist eine
ungerade Zahl. \ac{JPE} wird jetzt übersprungen.

\clearpage
\ac{JPE} wurde nicht ausgelöst, sodass \FLD den Wert von $a$ (5.6) nach
\ST{0} lädt:

\begin{figure}[H]
\centering
\myincludegraphics{patterns/12_FPU/3_comparison/x86/MSVC/olly2_5.png}
\caption{\olly: zweites \FLD wurde ausgeführt}
\label{fig:FPU_comparison_case2_olly5}
\end{figure}

Die Funktion beendet ihre Arbeit.
