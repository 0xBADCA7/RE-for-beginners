\myparagraph{\Optimizing GCC 4.4.1}

\lstinputlisting[caption=\Optimizing GCC
4.4.1,style=customasmx86]{patterns/12_FPU/3_comparison/x86/GCC_O3_DE.asm}

\myindex{x86!\Instructions!JA}
Dies ist fast das gleiche, außer dass \JA nach \SAHF verwendet wird.
Tatsächlich prüfen die bedingte Sprungbefehle, die vorzeichenlose Zahlen auf
\q{größer}, \q{kleiner} oder \q{gleich} prüfen (das sind \JA, \JAE, \JB, \JBE,
\JE/\JZ, \JNA, \JNAE, \JNB, \JNBE, \JNE/\JNZ) lediglich die Flags \CF und
\ZF.\\\\
Erinnern wir uns, an welchen Stellen die \CThreeBits sich im \GTT{AH} Register
befinden, nachdem der Befehl \INS{FSTSW}/\FNSTSW ausgeführt wurde:

\begin{center}
\begin{bytefield}[endianness=big,bitwidth=0.03\linewidth]{8}
\bitheader{6,2,1,0} \\
\bitbox{1}{} & 
\bitbox{1}{C3} & 
\bitbox{3}{} & 
\bitbox{1}{C2} & 
\bitbox{1}{C1} & 
\bitbox{1}{C0}
\end{bytefield}
\end{center}

Halten wir uns auch vor Augen wie die Bits aus \GTT{AH} in den CPU Flags nach
der Ausführung von \SAHF abgelegt werden:

\begin{center}
\begin{bytefield}[endianness=big,bitwidth=0.03\linewidth]{8}
\bitheader{7,6,4,2,0} \\
\bitbox{1}{SF} & 
\bitbox{1}{ZF} & 
\bitbox{1}{} & 
\bitbox{1}{AF} & 
\bitbox{1}{} & 
\bitbox{1}{PF} & 
\bitbox{1}{} & 
\bitbox{1}{CF}
\end{bytefield}
\end{center}

Nach dem Vergleich werden die \Cthree und \Czero Bits nach \ZF und \CF
verschoben, sodass der bedingte Sprung danach funktionieren kann. \JA wird
ausgeüführt, falls sowohl \CF als auch \ZF gleich 0 sind.

Hierbei können alle hier aufgelisteten Sprungbefehle nach einem \FNSTSW/\SAHF
Befehlspaar verwendet werden. 

Offenbar wurden die \CThreeBits Status Bits der CPU dort bewusst platziert,
sodass diese leicht auf die CPU Flags übertragen werden können, ohne dass
zusätzliche Vertauschungen notwendig sind.