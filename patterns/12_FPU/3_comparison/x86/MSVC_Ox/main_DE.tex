\myparagraph{\Optimizing MSVC 2010}

\lstinputlisting[caption=\Optimizing MSVC 2010,style=customasmx86]{patterns/12_FPU/3_comparison/x86/MSVC_Ox/MSVC_DE.asm}

\myindex{x86!\Instructions!FCOM}
\FCOM unterscheidet sich von \FCOMP in dem Sinne, dass es nur die Werte vergleicht ohne den FPU Stack zu verändern. 
Anders als im vorangehenden Beispiel liegen die Operanden hier in umgekehrter Reihenfolge um vor. Dies ist der Grund
warum das Ergebnis dieses Vergleichs bezüglich der \CThreeBits unterschiedlich ist:

\begin{itemize}
\item Falls $a>b$ in unserem Beispiel, dann werden die \CThreeBits Bits wie folgt gesetzt: 0, 0, 0.
\item Falls $b>a$, dann ist das Bitmuster: 0, 0, 1.
\item Falls $a=b$, dann ist das Bitmuster: 1, 0, 0.
\end{itemize}
% TODO: table?
Der Befehl \INS{test ah, 65} setzt zwei Bits~---\Cthree und \Czero. 
Beide werden auf 0 gesetzt, falls $a>b$: in diesem Fall wird der \JNE Sprungbefehl nicht ausgeführt.
Dann folgt \INS{FSTP ST(1)}~---dieser Befehl kopiert den Wert von \ST{0} in den Operanden und holt einen Wert vom FPU
Stack.
Mit anderen Worten, der Befehl kopiert \ST{0} (in dem sich gerade der Wert von \GTT{\_a} befindet) nach \ST{1}. 
Anschließend befinden sich zwei Kopien von {\_a} oben auf dem Stack.
Nun wird ein Wert wieder vom Stack geholt. Schließlich enthält \ST{0} den Wert {\_a} und die Funktion beendet sich.

Der bedingte Sprung \JNE wird in zwei Fällen ausgeführt: wenn $b>a$ oder wenn $a=b$.
\ST{0} wird nach \ST{0} kopiert; dabei handelt es sich um eine Operation ohne Wirkung (\ac{NOP}), dann wird ein Wert
 vom Stack geholt und in \ST{0} steht dann was sich vorher in \ST{1} befunden hat, nämlich {\_b}. Danach beendet sich
 die Funktion.
Der Grund dafür, dass dieser Befehl hier erzeugt wird ist wahrscheinlich, dass die \ac{FPU} über keinen anderen Befehl
verfügt um einen Wert vom Stack zu holen und anschließend zu entsorgen.

\clearpage
\mysubparagraph{Erstes \olly Beispiel: a=1.2 und b=3.4}
\myindex{\olly}

Beide \FLD werden ausgeführt:

\begin{figure}[H]
\centering
\myincludegraphics{patterns/12_FPU/3_comparison/x86/MSVC_Ox/olly1_1.png}
\caption{\olly: beide \FLD werden ausgeführt}
\label{fig:FPU_comparison_Ox_case1_olly1}
\end{figure}

\FCOM wird ausgeführt: 
\olly zeigt die Inhalte \ST{0} und \ST{1} übersichtlich an.

\clearpage
\FCOM wurde ausgeführt:

\begin{figure}[H]
\centering
\myincludegraphics{patterns/12_FPU/3_comparison/x86/MSVC_Ox/olly1_2.png}
\caption{\olly: \FCOM wurde ausgeführt}
\label{fig:FPU_comparison_Ox_case1_olly2}
\end{figure}

\Czero ist gesetzt, alle anderen Flags sind gelöscht.

\clearpage
\FNSTSW wurde ausgeführt, \GTT{AX}=0x3100:

\begin{figure}[H]
\centering
\myincludegraphics{patterns/12_FPU/3_comparison/x86/MSVC_Ox/olly1_3.png}
\caption{\olly: \FNSTSW wird ausgeführt}
\label{fig:FPU_comparison_Ox_case1_olly3}
\end{figure}

\clearpage
\TEST wird ausgeführt:

\begin{figure}[H]
\centering
\myincludegraphics{patterns/12_FPU/3_comparison/x86/MSVC_Ox/olly1_4.png}
\caption{\olly: \TEST wird ausgeführt}
\label{fig:FPU_comparison_Ox_case1_olly4}
\end{figure}

ZF=0, conditional bedingter Sprung wird jetzt ausgeführt.

\clearpage
\INS{FSTP ST} (oder \FSTP \ST{0}) wurde ausgeführt~---1.2 wurde vom Stack geholt und 3.4 bleibt obenauf liegen:

\begin{figure}[H]
\centering
\myincludegraphics{patterns/12_FPU/3_comparison/x86/MSVC_Ox/olly1_5.png}
\caption{\olly: \FSTP wird ausgeführt}
\label{fig:FPU_comparison_Ox_case1_olly5}
\end{figure}

Wir sehen, dass das Ergebnis des Befehls \INS{FSTP ST} dem Holen eines Wertes vom FPU Stack entspricht. 

\clearpage
\mysubparagraph{Second \olly example: a=5.6 and b=-4}

Beide \FLD werden ausgeführt:

\begin{figure}[H]
\centering
\myincludegraphics{patterns/12_FPU/3_comparison/x86/MSVC_Ox/olly2_1.png}
\caption{\olly: beide \FLD werden ausgeführt}
\label{fig:FPU_comparison_Ox_case2_olly1}
\end{figure}

\FCOM wird gleich ausgeführt.

\clearpage
\FCOM wurde ausgeführt:

\begin{figure}[H]
\centering
\myincludegraphics{patterns/12_FPU/3_comparison/x86/MSVC_Ox/olly2_2.png}
\caption{\olly: \FCOM ist beendet}
\label{fig:FPU_comparison_Ox_case2_olly2}
\end{figure}

Alle Bedingungs-Flags sind gelöscht.

\clearpage
\FNSTSW ist abgearbeitet, \GTT{AX}=0x3000:

\begin{figure}[H]
\centering
\myincludegraphics{patterns/12_FPU/3_comparison/x86/MSVC_Ox/olly2_3.png}
\caption{\olly: \FNSTSW wurde ausgeführt}
\label{fig:FPU_comparison_Ox_case2_olly3}
\end{figure}

\clearpage
\TEST wurde ausgeführt:

\begin{figure}[H]
\centering
\myincludegraphics{patterns/12_FPU/3_comparison/x86/MSVC_Ox/olly2_4.png}
\caption{\olly: \TEST wurde ausgeführt}
\label{fig:FPU_comparison_Ox_case2_olly4}
\end{figure}

ZF=1, der Sprung wird jetzt nicht ausgeführt.

\clearpage
\FSTP \ST{1} wurde ausgeführt: der Wert 5.6 liegt jetzt oben auf dem FPU Stack.

\begin{figure}[H]
\centering
\myincludegraphics{patterns/12_FPU/3_comparison/x86/MSVC_Ox/olly2_5.png}
\caption{\olly: \FSTP wurde ausgeführt}
\label{fig:FPU_comparison_Ox_case2_olly5}
\end{figure}
Wir erkennen, dass der Befehl \FSTP \ST{1} wie folgt funktioniert: er lässt das oberste Element des Stacks an seinem Platz, löscht aber das Register \ST{1}.

