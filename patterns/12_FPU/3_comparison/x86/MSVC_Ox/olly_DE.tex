\clearpage
\mysubparagraph{Erstes \olly Beispiel: a=1.2 und b=3.4}
\myindex{\olly}

Beide \FLD werden ausgeführt:

\begin{figure}[H]
\centering
\myincludegraphics{patterns/12_FPU/3_comparison/x86/MSVC_Ox/olly1_1.png}
\caption{\olly: beide \FLD werden ausgeführt}
\label{fig:FPU_comparison_Ox_case1_olly1}
\end{figure}

\FCOM wird ausgeführt: 
\olly zeigt die Inhalte \ST{0} und \ST{1} übersichtlich an.

\clearpage
\FCOM wurde ausgeführt:

\begin{figure}[H]
\centering
\myincludegraphics{patterns/12_FPU/3_comparison/x86/MSVC_Ox/olly1_2.png}
\caption{\olly: \FCOM wurde ausgeführt}
\label{fig:FPU_comparison_Ox_case1_olly2}
\end{figure}

\Czero ist gesetzt, alle anderen Flags sind gelöscht.

\clearpage
\FNSTSW wurde ausgeführt, \GTT{AX}=0x3100:

\begin{figure}[H]
\centering
\myincludegraphics{patterns/12_FPU/3_comparison/x86/MSVC_Ox/olly1_3.png}
\caption{\olly: \FNSTSW wird ausgeführt}
\label{fig:FPU_comparison_Ox_case1_olly3}
\end{figure}

\clearpage
\TEST wird ausgeführt:

\begin{figure}[H]
\centering
\myincludegraphics{patterns/12_FPU/3_comparison/x86/MSVC_Ox/olly1_4.png}
\caption{\olly: \TEST wird ausgeführt}
\label{fig:FPU_comparison_Ox_case1_olly4}
\end{figure}

ZF=0, conditional bedingter Sprung wird jetzt ausgeführt.

\clearpage
\INS{FSTP ST} (oder \FSTP \ST{0}) wurde ausgeführt~---1.2 wurde vom Stack geholt und 3.4 bleibt obenauf liegen:

\begin{figure}[H]
\centering
\myincludegraphics{patterns/12_FPU/3_comparison/x86/MSVC_Ox/olly1_5.png}
\caption{\olly: \FSTP wird ausgeführt}
\label{fig:FPU_comparison_Ox_case1_olly5}
\end{figure}

Wir sehen, dass das Ergebnis des Befehls \INS{FSTP ST} dem Holen eines Wertes vom FPU Stack entspricht. 

\clearpage
\mysubparagraph{Second \olly example: a=5.6 and b=-4}

Beide \FLD werden ausgeführt:

\begin{figure}[H]
\centering
\myincludegraphics{patterns/12_FPU/3_comparison/x86/MSVC_Ox/olly2_1.png}
\caption{\olly: beide \FLD werden ausgeführt}
\label{fig:FPU_comparison_Ox_case2_olly1}
\end{figure}

\FCOM wird gleich ausgeführt.

\clearpage
\FCOM wurde ausgeführt:

\begin{figure}[H]
\centering
\myincludegraphics{patterns/12_FPU/3_comparison/x86/MSVC_Ox/olly2_2.png}
\caption{\olly: \FCOM ist beendet}
\label{fig:FPU_comparison_Ox_case2_olly2}
\end{figure}

Alle Bedingungs-Flags sind gelöscht.

\clearpage
\FNSTSW ist abgearbeitet, \GTT{AX}=0x3000:

\begin{figure}[H]
\centering
\myincludegraphics{patterns/12_FPU/3_comparison/x86/MSVC_Ox/olly2_3.png}
\caption{\olly: \FNSTSW wurde ausgeführt}
\label{fig:FPU_comparison_Ox_case2_olly3}
\end{figure}

\clearpage
\TEST wurde ausgeführt:

\begin{figure}[H]
\centering
\myincludegraphics{patterns/12_FPU/3_comparison/x86/MSVC_Ox/olly2_4.png}
\caption{\olly: \TEST wurde ausgeführt}
\label{fig:FPU_comparison_Ox_case2_olly4}
\end{figure}

ZF=1, der Sprung wird jetzt nicht ausgeführt.

\clearpage
\FSTP \ST{1} wurde ausgeführt: der Wert 5.6 liegt jetzt oben auf dem FPU Stack.

\begin{figure}[H]
\centering
\myincludegraphics{patterns/12_FPU/3_comparison/x86/MSVC_Ox/olly2_5.png}
\caption{\olly: \FSTP wurde ausgeführt}
\label{fig:FPU_comparison_Ox_case2_olly5}
\end{figure}
Wir erkennen, dass der Befehl \FSTP \ST{1} wie folgt funktioniert: er lässt das oberste Element des Stacks an seinem Platz, löscht aber das Register \ST{1}.
