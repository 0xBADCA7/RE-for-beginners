\myparagraph{\Optimizing MSVC 2010}

\lstinputlisting[caption=\Optimizing MSVC 2010,style=customasm]{patterns/12_FPU/3_comparison/x86/MSVC_Ox/MSVC_EN.asm}

\myindex{x86!\Instructions!FCOM}

\FCOM differs from \FCOMP in the sense that it just compares the values and doesn't change the FPU stack. 
Unlike the previous example, here the operands are in reverse order, 
which is why the result of the comparison in \CThreeBits is different:

\begin{itemize}
\item If $a>b$ in our example, then \CThreeBits bits are to be set as: 0, 0, 0.
\item If $b>a$, then the bits are: 0, 0, 1.
\item If $a=b$, then the bits are: 1, 0, 0.
\end{itemize}
% TODO: table?

The \INS{test ah, 65} instruction leaves just two bits~---\Cthree and \Czero. 
Both will be zero if $a>b$: in that case the \JNE jump will not be triggered. 
Then \INS{FSTP ST(1)} follows~---this instruction copies the value from \ST{0} to the operand and 
pops one value from the FPU stack.
In other words, the instruction copies \ST{0} (where the value of \GTT{\_a} is now) into \ST{1}.
After that, two copies of {\_a} are at the top of the stack. 
Then, one value is popped.
After that, \ST{0} contains {\_a} and the function is finishes.

The conditional jump \JNE is triggering in two cases: if $b>a$ or $a=b$. 
\ST{0} is copied into \ST{0}, it is just like an idle (\ac{NOP}) operation, then one value 
is popped from the stack and the top of the stack (\ST{0}) is contain what has been in \ST{1} before 
(that is {\_b}). 
Then the function finishes. 
The reason this instruction is used here probably is because the \ac{FPU} 
has no other instruction to pop a value from the stack and discard it.

\clearpage
\mysubparagraph{\olly}
\myindex{\olly}

Let's try this example in \olly.
The input value of the function (2) is loaded into \EAX: 

\begin{figure}[H]
\centering
\myincludegraphics{patterns/08_switch/2_lot/olly1.png}
\caption{\olly: function's input value is loaded in \EAX}
\label{fig:switch_lot_olly1}
\end{figure}

\clearpage
The input value is checked, is it bigger than 4? 
If not, the \q{default} jump is not taken:
\begin{figure}[H]
\centering
\myincludegraphics{patterns/08_switch/2_lot/olly2.png}
\caption{\olly: 2 is no bigger than 4: no jump is taken}
\label{fig:switch_lot_olly2}
\end{figure}

\clearpage
Here we see a jumptable:

\begin{figure}[H]
\centering
\myincludegraphics{patterns/08_switch/2_lot/olly3.png}
\caption{\olly: calculating destination address using jumptable}
\label{fig:switch_lot_olly3}
\end{figure}

Here we've clicked \q{Follow in Dump} $\rightarrow$ \q{Address constant}, so now we see the \IT{jumptable} in the data window.
These are 5 32-bit values\footnote{They are underlined by \olly because
these are also FIXUPs: \myref{subsec:relocs}, we are going to come back to them later}.
\ECX is now 2, so the third element (can be indexed as 2\footnote{About indexing, see also: \ref{arrays_at_one}}) of the table is to be used.
It's also possible to click \q{Follow in Dump} $\rightarrow$ 
\q{Memory address} and \olly will show the element addressed by the \JMP instruction. 
That's \TT{0x010B103A}.

\clearpage
After the jump we are at \TT{0x010B103A}: the code printing \q{two} will now be executed:

\begin{figure}[H]
\centering
\myincludegraphics{patterns/08_switch/2_lot/olly4.png}
\caption{\olly: now we at the \IT{case:} label}
\label{fig:switch_lot_olly4}
\end{figure}

