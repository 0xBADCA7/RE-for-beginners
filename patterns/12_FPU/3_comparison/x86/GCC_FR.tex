\myparagraph{GCC 4.4.1}

\lstinputlisting[caption=GCC 4.4.1,style=customasmx86]{patterns/12_FPU/3_comparison/x86/GCC_FR.asm}

\myindex{x86!\Instructions!FUCOMPP}

\FUCOMPP{} est presque comme \FCOM, mais dépile deux valeurs de la pile et traite
les \q{non-nombres} différement.

\myindex{Non-a-numbers (NaNs)}
Quelques informations à propos des \IT{not-a-numbers} (non-nombres).

\newcommand{\NANFN}{\footnote{\href{http://go.yurichev.com/17130}{wikipedia.org/wiki/NaN}}}

Le FPU est capable de traiter les valeurs spéciales que sont les \IT{not-a-numbers}
(non-nombres) ou \gls{NaN}s\NANFN.
Ce sont les infinis, les résultat de division par 0, etc.
Les non-nombres peuvent être \q{quiet} et \q{signaling}. Il est possible de continuer
à travailler ave les \q{quiet} NaNs, mais si l'on essaye de faire une opération avec
un \q{signaling} NaNs, une exception est levée.

\myindex{x86!\Instructions!FCOM}
\myindex{x86!\Instructions!FUCOM}

\FCOM lève une exception si un des opérandes est \gls{NaN}.
\FUCOM lève une exception seulement si un des opérandes est un signaling \gls{NaN}
(SNaN).

\myindex{x86!\Instructions!SAHF}
\label{SAHF}

L'instruction suivante est \SAHF (\IT{Store AH into Flags} stocker AH dans les Flags)~---est
une instruction rare dans le code non relatif au FPU.
8 bits de AH sont copiés dans les 8-bits bas dans les flags du CPU dans l'ordre suivant:

\begin{center}
\begin{bytefield}[endianness=big,bitwidth=0.03\linewidth]{8}
\bitheader{7,6,4,2,0} \\
\bitbox{1}{SF} & 
\bitbox{1}{ZF} & 
\bitbox{1}{} & 
\bitbox{1}{AF} & 
\bitbox{1}{} & 
\bitbox{1}{PF} & 
\bitbox{1}{} & 
\bitbox{1}{CF}
\end{bytefield}
\end{center}


\myindex{x86!\Instructions!FNSTSW}

Rappelons que \FNSTSW déplace des bits qui nous intéressent (\CThreeBits) dans \AH
et qu'ils sont aux positions 6, 2, 0 du registre \AH.

\begin{center}
\begin{bytefield}[endianness=big,bitwidth=0.03\linewidth]{8}
\bitheader{6,2,1,0} \\
\bitbox{1}{} & 
\bitbox{1}{C3} & 
\bitbox{3}{} & 
\bitbox{1}{C2} & 
\bitbox{1}{C1} & 
\bitbox{1}{C0}
\end{bytefield}
\end{center}


En d'autres mots, la paire d'instructions \INS{fnstsw  ax / sahf} déplace \CThreeBits
dans \ZF, \PF et \CF.

Maintenant, rappelons les valeurs de \CThreeBits sous différentes conditions:

\begin{itemize}
\item Si $a$ est plus grand que $b$ dans notre exemple, alors les \CThreeBits sont
mis à: 0, 0, 0.
\item Si $a$ est plus petit que $b$, alors les bits sont mis à: 0, 0, 1.
\item Si $a=b$, alors: 1, 0, 0.
\end{itemize}
% TODO: table?

En d'autres mots, ces états des flags du CPU sont possible après les
trois instructions \FUCOMPP/\FNSTSW/\SAHF:

\begin{itemize}
\item Si $a>b$, les flags du CPU sont mis à: \GTT{ZF=0, PF=0, CF=0}.
\item Si $a<b$, alors les flags sont mis à: \GTT{ZF=0, PF=0, CF=1}.
\item Et si $a=b$, alors: \GTT{ZF=1, PF=0, CF=0}.
\end{itemize}
% TODO: table?

\myindex{x86!\Instructions!SETcc}
\myindex{x86!\Instructions!JNBE}

Suivant les flags du CPU et les conditions, \SETNBE met 1 ou 0 dans AL.
C'est presque la contrepartie de \JNBE, avec l'exception que \SETcc\footnote{\IT{cc}
est un \IT{condition code}} met 1 ou 0 dans \AL, mais \Jcc effectue un saut ou non.
\SETNBE met 1 seulement si \GTT{CF=0} et \GTT{ZF=0}.
Si ce n'est pas vrai, 0 est mis dans \AL.

Il y a un seul cas où \CF et \ZF sont à 0: si $a>b$.

Alors 1 est mis dans \AL, le \JZ subséquent n'est pas pris et la fonction va renvoyer
{\_a}.
Dans tous les autres cas, {\_b} est renvoyé.

