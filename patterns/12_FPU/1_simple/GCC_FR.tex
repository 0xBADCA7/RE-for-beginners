\myparagraph{GCC}

GCC 4.4.1 (avec l'option \Othree) génère le même code, juste un peu différent:

\lstinputlisting[caption=GCC 4.4.1 \Optimizing,style=customasmx86]{patterns/12_FPU/1_simple/GCC_FR.asm}

La différence est que, tout d'abord, 3.14 est poussé sur la pile (dans \ST{0}), et
ensuite la valeur dans \GTT{arg\_0} est divisée par la valeur dans \ST{0}.

\myindex{x86!\Instructions!FDIVR}

\FDIVR signifie \IT{Reverse Divide}~---pour diviser avec le diviseur et le dividende
échangés l'un avec l'autre.
Il n'y a pas d'instruction de ce genre pour la multiplication puisque c'est une opération
commutative, donc nous avons seulement \FMUL sans son homologue \GTT{-R}.

\myindex{x86!\Instructions!FADDP}

\FADDP ajoute les deux valeurs mais supprime aussi une valeur de la pile.
Après cette opération, \ST{0} contient la somme.

