\subsubsection{MIPS}

MIPS может поддерживать несколько сопроцессоров (вплоть до 4), нулевой из которых\footnote{Если считать с нуля.} это специальный
управляющий сопроцессор, а первый~--- это FPU.

Как и в ARM, сопроцессор в MIPS это не стековая машина. Он имеет 32 32-битных регистра (\$F0-\$F31):

\myref{MIPS_FPU_registers}.
Когда нужно работать с 64-битными значениями типа \Tdouble, используется пара 32-битных F-регистров.

\lstinputlisting[caption=\Optimizing GCC 4.4.5 (IDA),style=customasmMIPS]{patterns/12_FPU/1_simple/MIPS_O3_IDA_RU.lst}

Новые инструкции:

\myindex{MIPS!\Instructions!LWC1}
\myindex{MIPS!\Instructions!DIV.D}
\myindex{MIPS!\Instructions!MUL.D}
\myindex{MIPS!\Instructions!ADD.D}
\begin{itemize}

\item \INS{LWC1} загружает 32-битное слово в регистр первого сопроцессора (отсюда \q{1} в названии инструкции).

\myindex{MIPS!\Pseudoinstructions!L.D}
Пара инструкций \INS{LWC1} может быть объединена в одну псевдоинструкцию \INS{L.D}.

\item \INS{DIV.D}, \INS{MUL.D}, \INS{ADD.D} производят деление, умножение и сложение соответственно 
(\q{.D} в суффиксе означает двойную точность, \q{.S}~--- одинарную точность)

\end{itemize}

\myindex{MIPS!\Instructions!LUI}
\myindex{\CompilerAnomaly}
\label{MIPS_FPU_LUI}
Здесь также имеется странная аномалия компилятора: инструкция \INS{LUI} помеченная нами вопросительным знаком.%

Мне трудно понять, зачем загружать часть 64-битной константы типа \Tdouble в регистр \$V0.

От этих инструкций нет толка.
% TODO did you try checking out compiler source code?
Если кто-то об этом что-то знает, пожалуйста, напишите автору емейл \footnote{\EMAIL}.

