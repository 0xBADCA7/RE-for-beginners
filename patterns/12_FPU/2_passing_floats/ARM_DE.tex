\subsubsection{ARM + \NonOptimizingXcodeIV (\ThumbTwoMode)}
\label{FPU_passing_floats_ARM}

\lstinputlisting[style=customasm]{patterns/12_FPU/2_passing_floats/Xcode_thumb_O0.asm}
Wie bereits vorher erwähnt werden Pointer auf 64-Bit-Fließkommazahlen über ein
Paar von R-Registern übergeben.

Dieser Code ist leicht redundant (sicherlich aufgrund der deaktivierten
Optimierung), da es möglich ist Werte direkt in die R-Register zu laden, ohne
die D-Register zu verwenden.

Wie wir also sehen erhält die \GTT{\_pow} Funktion ihr erster Argument in
\Reg{0} und \Reg{1} und das zweite in \Reg{2} und \Reg{3}. Die Funktion
speichert ihr Ergebnis in \Reg{0} und \Reg{1}. 
Das Ergebnis von \GTT{\_pow} wird zunächst nach \GTT{D16} und
anschließend in das Paar \Reg{1} und \Reg{2} verschoben, von wo aus \printf das
Ergebnis übernimmt. 

\subsubsection{ARM + \NonOptimizingKeilVI (\ARMMode)}

\lstinputlisting[style=customasm]{patterns/12_FPU/2_passing_floats/Keil_ARM_O0.asm}

Die D-Register werden hier nicht verwendet, sondern nur Paare von R-Registern.

\subsubsection{ARM64 + \Optimizing GCC (Linaro) 4.9}

\lstinputlisting[caption=\Optimizing GCC (Linaro)
4.9,style=customasm]{patterns/12_FPU/2_passing_floats/ARM64_DE.s}

Die Konstanten werden nach \RegD{0} und \RegD{1} geladen: \TT{pow()} übernimmt
sie von dort. Das Ergebnis befindet sich nach der Ausführung von \TT{pow()} in
\RegD{0}. 
Es wird ohne weitere Änderung oder Verschiebung an die Funktion \prinft
übergeben, da \printf ganzzahlige Werte und Pointer aus X-Registern,
Fließkommaparameter jedoch aus D-Registern übernimmt.

