\subsubsection{x86}

Regardons ce que nous obtenons avec MSVC 2010:

\lstinputlisting[caption=MSVC 2010,style=customasmx86]{patterns/12_FPU/2_passing_floats/MSVC_FR.asm}

\myindex{x86!\Instructions!FLD}
\myindex{x86!\Instructions!FSTP}

% TODO bug to be fixed here:
\FLD et \FSTP déplacent des variables entre le segment de données et la pile du FPU.
\GTT{pow()}\footnote{une fonction C standard, qui élève un nombre à la puissance
donnée (puissance)} prend deux valeurs depuis la pile du FPU et renvoie son résultat
dans le registre \ST{0}.
\printf prend 8 octets de la pile locale et les interprète comme des variables de
type \Tdouble.

Á prosos, une paire d'instructions \MOV pourrait être utilisée ici pour déplacer
les valeurs depuis la mémoire vers la pile, car les valeurs en mémoire sont stockées
au format IEEE 754, et pow() les prend aussi dans ce format, donc aucune conversion
n'est nécessaire.
C'est comme ceci que c'est fait dans l'exemple suivant, pour ARM: \myref{FPU_passing_floats_ARM}.

