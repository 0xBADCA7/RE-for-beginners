\subsubsection{MIPS}

\lstinputlisting[caption=\Optimizing GCC 4.4.5 (IDA),style=customasmMIPS]{patterns/12_FPU/2_passing_floats/MIPS_O3_IDA_FR.lst}

À nouveau, nous voyons ici \INS{LUI} qui charge une partie 32-bit d'un nombre \Tdouble
dans \$V0.
À nouveau, c'est difficile de comprendre pourquoi.

\myindex{MIPS!\Instructions!MFC1}

La nouvelle instruction pour nous ici est \INS{MFC1} (\q{Move From Coprocessor 1}
charger depuis le coprocesseur 1).
Le FPU est le coprocesseur numéro 1, d'où le \q{1} dans le nom de l'instruction.
Cette instruction transfère des valeurs depuis des registres du coprocesseur dans
les registres du CPU (\ac{GPR}).
Donc à la fin, le résultat de \TT{pow()} est transféré dans les registres \$A3 et
\$A2, et \printf prend une valeur double 64-bit depuis cette paire de registre.

