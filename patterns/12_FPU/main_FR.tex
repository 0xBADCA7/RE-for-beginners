\section{\FPUChapterName}
\label{sec:FPU}

\newcommand{\FNURLSTACK}{\footnote{\href{http://go.yurichev.com/17123}{wikipedia.org/wiki/Stack\_machine}}}
\newcommand{\FNURLFORTH}{\footnote{\href{http://go.yurichev.com/17124}{wikipedia.org/wiki/Forth\_(programming\_language)}}}
\newcommand{\FNURLIEEE}{\footnote{\href{http://go.yurichev.com/17125}{wikipedia.org/wiki/IEEE\_floating\_point}}}
\newcommand{\FNURLSP}{\footnote{\href{http://go.yurichev.com/17126}{wikipedia.org/wiki/Single-precision\_floating-point\_format}}}
\newcommand{\FNURLDP}{\footnote{\href{http://go.yurichev.com/17127}{wikipedia.org/wiki/Double-precision\_floating-point\_format}}}
\newcommand{\FNURLEP}{\footnote{\href{http://go.yurichev.com/17128}{wikipedia.org/wiki/Extended\_precision}}}

Le \ac{FPU} est un dispositif à l'intérieur du \ac{CPU}, spécialement conçu pour
traiter les nombres à virgules flottantes.

Il était appelé \q{coprocesseur} dans le passé et il était en dehors du \ac{CPU}.

\subsection{IEEE 754}

Un nombre au format IEEE 754 consiste en un \IT{signe}, un \IT{significande} (aussi
appelé \IT{fraction}) et un \IT{exposant}.

\subsection{x86}

Ca vaut la peine de jetter un oeil sur les machines à base de piles\FNURLSTACK ou
d'apprendre les bases du langage Forth\FNURLFORTH, avant d'étudier le \ac{FPU} en
x86.

\myindex{Intel!80486}
\myindex{Intel!FPU}
Il est intéressant de savoir que dans le passé (avant le CPU 80486) le coprocesseur
était une puce séparée et n'était pas toujours pré-installé sur la carte mère. Il
était possible de l'acheter séparemment et de l'installer\footnote{Par exemple, John
Carmack a utilisé des valeurs arithmétiques à virgule fixe
(\href{http://go.yurichev.com/17356}{wikipedia.org/wiki/Fixed-point\_arithmetic})
dans son jeu vidéo Doom, stockées dans des registres 32-bit \ac{GPR} (16 bit pour
la partie entière et 16 bit pour la partie fractionnaire), donc Doom pouvait fonctionner
sur des ordinateurs 32-bit sans FPU, i.e., 80386 et 80486 SX.}.

A partir du CPU 80486 DX, le \ac{FPU} est intégré dans le \ac{CPU}.

\myindex{x86!\Instructions!FWAIT}
L'instruction \INS{FWAIT} nous rapelle le fait qu'elle passe le \ac{CPU} dans un
état d'attente, jusqu'à ce que le \ac{FPU} ait fini son traitement.

Un autre rudiment est le fait que les opcodes d'instruction \ac{FPU} commencent
avec ce qui est appelé l'opcode-\q{d'échappement} (\GTT{D8..DF}), i.e., opcodes
passés à un coprocesseur séparé.

\myindex{IEEE 754}
\label{FPU_is_stack}

Le FPU a une pile capable de contenir 8 registres de 80-bit, et chaque registre peut
contenir un nombre au format IEEE 754\FNURLIEEE.

Ce sont \ST{0}..\ST{7}. Par concision, \IDA et \olly montrent \ST{0} comme \GTT{ST},
qui est représenté dans certains livres et manuels comme \q{Stack Top}.

\subsection{ARM, MIPS, x86/x64 SIMD}

En ARM et MIPS le FPU n'a pas de pile, mais un ensemble de registres, qui peuvent
être accédés aléatoirement, comme \ac{GPR}.

La même idéologie est utilisée dans l'extension SIMD des CPUs x86/x64.

\subsection{\CCpp}

\myindex{float}
\myindex{double}

Le standard des langages \CCpp offre au moins deux types de nombres à virgule flottante,
\Tfloat (\IT{simple-précision}\FNURLSP, 32 bits) \footnote{le format des nombres
à virgule flottante simple précision est aussi abordé dans la section \IT{\WorkingWithFloatAsWithStructSubSubSectionName}~(\myref{sec:floatasstruct})}
et \Tdouble  (\IT{double-précision}\FNURLDP, 64 bits).

Dans \InSqBrackets{\TAOCPvolII 246} nous pouvons trouver que \IT{simple-précision}
signifie que la valeur flottante peut être stockée dans un simple mot machine [32-bit],
\IT{double-précision} signifie qu'elle peut être stockée dans deux mots (64 bits).

\myindex{long double}

GCC supporte également le type \IT{long double} (\IT{précision étendue}\FNURLEP,
80 bit), que MSVC ne supporte pas.

Le type \Tfloat nécessite le même nombre de bits que le type \Tint dans les environnements
32-bit, mais la représentation du nombre est complètement différente.

\ifdefined\RUSSIAN
\subsection{Простой пример}

Рассмотрим простой пример:
\fi

\ifdefined\ENGLISH
\subsection{Simple example}

Let's consider this simple example:
\fi

\ifdefined\GERMAN
\subsection{\DEph{}}

\DEph{}

\fi

\ifdefined\FRENCH
\subsection{Exemple simple}

Considérons cet exemple simple:
\fi

\lstinputlisting[style=customc]{patterns/12_FPU/1_simple/simple.c}

\subsubsection{x86}

% subsubsections
\EN{\myparagraph{MSVC}

Compile it in MSVC 2010:

\lstinputlisting[caption=MSVC 2010: \ttf{},style=customasm]{patterns/12_FPU/1_simple/MSVC_EN.asm}

\FLD takes 8 bytes from stack and loads the number into the \ST{0} register, automatically converting 
it into the internal 80-bit format (\IT{extended precision}).

\myindex{x86!\Instructions!FDIV}

\FDIV divides the value in \ST{0} by the number stored at address \\
\GTT{\_\_real@40091eb851eb851f}~---the value 3.14 is encoded there. 
The assembly syntax doesn't support floating point numbers, so 
what we see here is the hexadecimal representation of 3.14 in 64-bit IEEE 754 format.

After the execution of \FDIV \ST{0} holds the \gls{quotient}.

\myindex{x86!\Instructions!FDIVP}

By the way, there is also the \FDIVP instruction, which divides \ST{1} by \ST{0}, 
popping both these values from stack and then pushing the result. 
If you know the Forth language\FNURLFORTH,
you can quickly understand that this is a stack machine\FNURLSTACK.

The subsequent \FLD instruction pushes the value of $b$ into the stack.

After that, the quotient is placed in \ST{1}, and \ST{0} has the value of $b$.

\myindex{x86!\Instructions!FMUL}

The next \FMUL instruction does multiplication: $b$ from \ST{0} is multiplied by value at\\
\GTT{\_\_real@4010666666666666} (the number 4.1 is there) and leaves the result in the \ST{0} register.

\myindex{x86!\Instructions!FADDP}

The last \FADDP instruction adds the two values at top of stack, storing the result in \ST{1} 
and then popping the value of \ST{0}, thereby leaving the result at the top of the stack, in \ST{0}.

The function must return its result in the \ST{0} register, 
so there are no any other instructions except the function epilogue after \FADDP.

\input{patterns/12_FPU/1_simple/olly_EN.tex}
}
\RU{\myparagraph{MSVC}

Компилируем в MSVC 2010:

\lstinputlisting[caption=MSVC 2010: \ttf{},style=customasmx86]{patterns/12_FPU/1_simple/MSVC_RU.asm}

\FLD берет 8 байт из стека и загружает их в регистр \ST{0}, автоматически конвертируя во внутренний 
80-битный формат (\IT{extended precision}).

\myindex{x86!\Instructions!FDIV}
\FDIV делит содержимое регистра \ST{0} на число, лежащее по адресу \GTT{\_\_real@40091eb851eb851f}~--- 
там закодировано значение 3,14. Синтаксис ассемблера не поддерживает подобные числа, 
поэтому мы там видим шестнадцатеричное представление числа 3,14 в формате IEEE 754.

После выполнения \FDIV в \ST{0} остается \glslink{quotient}{частное}.

\myindex{x86!\Instructions!FDIVP}
Кстати, есть ещё инструкция \FDIVP, которая делит \ST{1} на \ST{0}, 
выталкивает эти числа из стека и заталкивает результат. 
Если вы знаете язык Forth\FNURLFORTH, то это как раз оно и есть~--- стековая машина\FNURLSTACK.

Следующая \FLD заталкивает в стек значение $b$.

После этого в \ST{1} перемещается результат деления, а в \ST{0} теперь $b$.

\myindex{x86!\Instructions!FMUL}
Следующий \FMUL умножает $b$ из \ST{0} на значение \\
\GTT{\_\_real@4010666666666666} --- там лежит число 4,1~--- и оставляет результат в \ST{0}.

\myindex{x86!\Instructions!FADDP}
Самая последняя инструкция \FADDP складывает два значения из вершины стека 
в \ST{1} и затем выталкивает значение, лежащее в \ST{0}. 
Таким образом результат сложения остается на вершине стека в \ST{0}.

Функция должна вернуть результат в \ST{0}, так что больше ничего здесь не производится, 
кроме эпилога функции.

\input{patterns/12_FPU/1_simple/olly_RU.tex}
}
\DE{\myparagraph{MSVC}

Kompilieren mit MSVC 2010 liefert:

\lstinputlisting[caption=MSVC 2010:
\ttf{},style=customasm]{patterns/12_FPU/1_simple/MSVC_DE.asm}

\FLD nimmt 8 Byte vom Stack und lädt die Zahl in das \ST{0} Register, wobei
diese automatisch in das interne 80-bit-Format (\IT{erweiterte Genauigkeit})
konvertiert wird.

\myindex{x86!\Instructions!FDIV}
\FDIV teilt den Wert in \ST{0} durch die Zahl, die an der Adresse
\GTT{\_\_real@40091eb851eb851f} gespeichert ist~---der Wert 3.14 ist hier
kodiert.
Die Syntax des Assemblers erlaubt keine Fließkommazahlen, sodass wir hier die
hexadezimale Darstellung von 3.14 im 64-bit IEEE 754 Format finden.

Nach der Ausführung von \FDIV enthält \ST{0} den \gls{Quotienten}.

\myindex{x86!\Instructions!FDIVP}
Es gibt übrigens auch noch den \FDIVP Befehl, welcher \ST{1} durch \ST{0}
teilt, beide Werte vom Stack holt und das Ergebnis ebenfalls auf dem Stack
ablegt.
Wer mit der Sprache Forth \FNURLFRTH vertraut ist, erkennt schnell, dass es sich
hier um eine Stackmaschine\FNURLSTACK handelt.

Der nachfolgende \FLD Befehl speichert den Wert von $b$ auf dem Stack.

Anschließend wir der Quotient in \ST{1} abgelegt und \ST{0} enthält den Wert von
$b$.

\myindex{x86!\Instructions!FMUL}
Der nächste \FMUL Befehl führt folgende Multiplikation aus: $b$ aus Register
\ST{0} wird mit dem Wert an der Speicherstelle \GTT{\_\_real@4010666666666666}
(hier befindet sich die Zahl 4.1) multipliziert und hinterlässt das Ergebnis im
\ST{ß} Register.

\myindex{x86!\Instructions!FADDP}
Der letzte \FADDP Befehl addiert die beiden Werte, die auf dem Stack zuoberst
liegen, speichet das Ergebnis in \ST{1} und holt dann den Wert von \ST{0} vom
Stack, wobei das oberste Element auf dem Stack in \ST{0} gespeichert wird.

Die Funktion muss ihr Ergebnis im \ST{0} Register zurückgeben, sodass außer dem
Funktionsepilog nach \FADDP keine weiteren Befehle mehr folgen.

\input{patterns/12_FPU/1_simple/olly_DE.tex}
}

\EN{\myparagraph{GCC}

GCC 4.4.1 (with \Othree option) emits the same code, just slightly different:

\lstinputlisting[caption=\Optimizing GCC 4.4.1,style=customasmx86]{patterns/12_FPU/1_simple/GCC_EN.asm}

The difference is that, first of all, 3.14 is pushed to the stack (into \ST{0}), and then the value 
in \GTT{arg\_0} is divided by the value in \ST{0}.

\myindex{x86!\Instructions!FDIVR}

\FDIVR stands for \IT{Reverse Divide}~---to divide with divisor and dividend swapped with each other. 
There is no likewise instruction for multiplication since it is 
a commutative operation, so we just have \FMUL without its \GTT{-R} counterpart.

\myindex{x86!\Instructions!FADDP}

\FADDP adds the two values but also pops one value from the stack. 
After that operation, \ST{0} holds the sum.

}
\RU{\myparagraph{GCC}

GCC 4.4.1 (с опцией \Othree) генерирует похожий код, хотя и с некоторой разницей:

\lstinputlisting[caption=\Optimizing GCC 4.4.1,style=customasm]{patterns/12_FPU/1_simple/GCC_RU.asm}

Разница в том, что в стек сначала заталкивается 3,14 (в \ST{0}), а затем значение 
из \GTT{arg\_0} делится на то, что лежит в регистре \ST{0}.

\myindex{x86!\Instructions!FDIVR}
\FDIVR означает \IT{Reverse Divide}~--- делить, поменяв делитель и делимое местами. 
Точно такой же инструкции для умножения нет, потому что она была бы бессмысленна (ведь умножение 
операция коммутативная), так что остается только \FMUL без соответствующей ей \GTT{-R} инструкции.

\myindex{x86!\Instructions!FADDP}
\FADDP не только складывает два значения, но также и выталкивает из стека одно значение. 
После этого в \ST{0} остается только результат сложения.
}
\DE{\myparagraph{GCC}

GCC 4.4.1 (mit der Option \Othree) erzeugt fast den gleichen Code, nur leicht
verändert.

\lstinputlisting[caption=\Optimizing GCC 4.4.1,style=customasmx86]{patterns/12_FPU/1_simple/GCC_DE.asm} Der Unterschied
besteht darin, dass zuerst 3.14 auf dem Stack (in \ST{0}) abgelegt wird und danach der Wert in \GTT{arg\_0} durch den Wert in \ST{0}
geteilt wird.

\myindex{x86!\Instructions!FDIVR}
\FDIVR steht für \IT{Reverse Divide}~--teilen, wobei Dividend und Divisor
miteinander vertauscht werden. Da es sich bei der Multiplikation um eine
kommutative Operation handelt, gibt es keinen vergleichbaren Befehl für die
Multiplikation. Wir haben es lediglich \FMUL ohne \GTT{-R} Gegenstück zur
Verfügung.

\myindex{x86!\Instructions!FADDP}
\FADDP addiert die beiden Werte und holt auch einen Wert vom Stack. 
Nach der Ausführung steht die Summe in \ST{0}.

}


\EN{\subsubsection{ARM: \OptimizingXcodeIV (\ARMMode)}

Until ARM got standardized floating point support, several processor manufacturers added their own 
instructions extensions.
Then, VFP (\IT{Vector Floating Point}) was standardized.

One important difference from x86 is that in ARM, there
is no stack, you work just with registers.

\lstinputlisting[label=ARM_leaf_example10,caption=\OptimizingXcodeIV (\ARMMode),style=customasmARM]{patterns/12_FPU/1_simple/ARM/Xcode_ARM_O3_EN.asm}

\myindex{ARM!D-\registers{}}
\myindex{ARM!S-\registers{}}

So, we see here new some registers used, with D prefix.

These are 64-bit registers, there are 32 of them, and they can be used both for floating-point numbers 
(double) but also for SIMD (it is called NEON here in ARM).

There are also 32 32-bit S-registers, intended to be used for single precision 
floating pointer numbers (float).

It is easy to memorize: D-registers are for double precision numbers, while
S-registers---for single precision numbers.
More about it: \myref{ARM_VFP_registers}.

Both constants (3.14 and 4.1) are stored in memory in IEEE 754 format.

\myindex{ARM!\Instructions!VLDR}
\myindex{ARM!\Instructions!VMOV}
\INS{VLDR} and \INS{VMOV}, as it can be easily deduced, are analogous to the \INS{LDR} and \MOV instructions,
but they work with D-registers.

It has to be noted that these instructions, just like the D-registers, are intended not only for
floating point numbers, 
but can be also used for SIMD (NEON) operations and this will also be shown soon.

The arguments are passed to the function in a common way, via the R-registers, however
each number that has double precision has a size of 64 bits, so two R-registers are needed to pass each one.

\INS{VMOV D17, R0, R1} at the start, composes two 32-bit values from \Reg{0} and \Reg{1} into one 64-bit value
and saves it to \GTT{D17}.

\INS{VMOV R0, R1, D16} is the inverse operation: what has been in \GTT{D16} 
is split in two registers, \Reg{0} and \Reg{1}, because a double-precision number 
that needs 64 bits for storage, is returned in \Reg{0} and \Reg{1}.

\myindex{ARM!\Instructions!VDIV}
\myindex{ARM!\Instructions!VMUL}
\myindex{ARM!\Instructions!VADD}
\INS{VDIV}, \INS{VMUL} and \INS{VADD}, 
are instruction for processing floating point numbers that compute \gls{quotient}, 
\gls{product} and sum, respectively.

The code for Thumb-2 is same.

\subsubsection{ARM: \OptimizingKeilVI (\ThumbMode)}

\lstinputlisting[style=customasmARM]{patterns/12_FPU/1_simple/ARM/Keil_O3_thumb_EN.asm}

Keil generated code for a processor without FPU or NEON support.

The double-precision floating-point numbers are passed via generic R-registers,
and instead of FPU-instructions, service library functions are called\\
(like \GTT{\_\_aeabi\_dmul}, \GTT{\_\_aeabi\_ddiv}, \GTT{\_\_aeabi\_dadd})
which emulate multiplication, division and addition for floating-point numbers.

Of course, that is slower than FPU-coprocessor, but it's still better than nothing.

By the way, similar FPU-emulating libraries were very popular in the x86 world when coprocessors were rare
and expensive, and were installed only on expensive computers.

\myindex{ARM!soft float}
\myindex{ARM!armel}
\myindex{ARM!armhf}
\myindex{ARM!hard float}

The FPU-coprocessor emulation is called \IT{soft float} or \IT{armel} (\IT{emulation}) in the ARM world, 
while using the coprocessor's FPU-instructions is called \IT{hard float} or \IT{armhf}.

\iffalse
% TODO разобраться...
\myindex{Raspberry Pi}

For example, the Linux kernel for Raspberry Pi is compiled in two variants.

In the \IT{soft float} case, arguments are passed via R-registers, and in the \IT{hard float} case---via D-registers.

And that is what stops you from using armhf-libraries from armel-code or vice versa,
and that is why all the code in Linux distributions must be compiled according to a single convention.
\fi

\subsubsection{ARM64: \Optimizing GCC (Linaro) 4.9}

Very compact code:

\lstinputlisting[caption=\Optimizing GCC (Linaro) 4.9,style=customasmARM]{patterns/12_FPU/1_simple/ARM/ARM64_GCC_O3_EN.s}

\subsubsection{ARM64: \NonOptimizing GCC (Linaro) 4.9}

\lstinputlisting[caption=\NonOptimizing GCC (Linaro) 4.9,style=customasmARM]{patterns/12_FPU/1_simple/ARM/ARM64_GCC_O0_EN.s}

\NonOptimizing GCC is more verbose.

There is a lot of unnecessary value shuffling, including some clearly redundant code 
(the last two \INS{FMOV} instructions). Probably, GCC 4.9 is not yet good in generating ARM64 code.

What is worth noting is that ARM64 has 64-bit registers, and the D-registers are 64-bit ones as well.

So the compiler is free to save values of type \Tdouble in \ac{GPR}s instead of the local stack.
This isn't possible on 32-bit CPUs.

And again, as an exercise, you can try to optimize this function manually, without introducing
new instructions like \INS{FMADD}.
}
\RU{\subsubsection{ARM: \OptimizingXcodeIV (\ARMMode)}

Пока в ARM не было стандартного набора инструкций для работы с числами с плавающей точкой, разные производители процессоров
могли добавлять свои расширения для работы с ними.
Позже был принят стандарт VFP (\IT{Vector Floating Point}).

Важное отличие от x86 в том, что там вы работаете с FPU-стеком, а здесь стека нет, вы работаете просто с регистрами.

\lstinputlisting[label=ARM_leaf_example10,caption=\OptimizingXcodeIV (\ARMMode),style=customasm]{patterns/12_FPU/1_simple/ARM/Xcode_ARM_O3_RU.asm}

\myindex{ARM!D-\registers{}}
\myindex{ARM!S-\registers{}}
Итак, здесь мы видим использование новых регистров с префиксом D.

Это 64-битные регистры. Их 32 и их можно
использовать для чисел с плавающей точкой двойной точности (double) и для 
SIMD (в ARM это называется NEON).

Имеются также 32 32-битных S-регистра. Они применяются для работы с числами 
с плавающей точкой одинарной точности (float).

Запомнить легко: D-регистры предназначены для чисел double-точности, 
а S-регистры~--- для чисел single-точности.

Больше об этом: \myref{ARM_VFP_registers}.

Обе константы (3,14 и 4,1) хранятся в памяти в формате IEEE 754.

\myindex{ARM!\Instructions!VLDR}
\myindex{ARM!\Instructions!VMOV}
Инструкции \INS{VLDR} и \INS{VMOV}, как можно догадаться, это аналоги обычных \INS{LDR} и \MOV, но они работают с D-регистрами.

Важно отметить, что эти инструкции, как и D-регистры, предназначены не только для работы 
с числами с плавающей точкой, но пригодны также и для работы с SIMD (NEON), и позже это также будет видно.

Аргументы передаются в функцию обычным путем через R-регистры, однако 
каждое число, имеющее двойную точность, занимает 64 бита, так что для передачи каждого нужны два R-регистра.

\INS{VMOV D17, R0, R1} в самом начале составляет два 32-битных значения из \Reg{0} и \Reg{1} в одно 64-битное и сохраняет в \GTT{D17}.

\INS{VMOV R0, R1, D16} в конце это обратная процедура: то что было в \GTT{D16} 
остается в двух регистрах \Reg{0} и \Reg{1}, потому что число с двойной точностью, 
занимающее 64 бита, возвращается в паре регистров \Reg{0} и \Reg{1}.

\myindex{ARM!\Instructions!VDIV}
\myindex{ARM!\Instructions!VMUL}
\myindex{ARM!\Instructions!VADD}
\INS{VDIV}, \INS{VMUL} и \INS{VADD}, это инструкции для работы с числами 
с плавающей точкой, вычисляющие, соответственно, \glslink{quotient}{частное}, \glslink{product}{произведение} и сумму.

Код для Thumb-2 такой же.

\subsubsection{ARM: \OptimizingKeilVI (\ThumbMode)}

\lstinputlisting[style=customasm]{patterns/12_FPU/1_simple/ARM/Keil_O3_thumb_RU.asm}

Keil компилировал для процессора, в котором может и не быть поддержки FPU или NEON.
Так что числа с двойной точностью передаются в парах обычных R-регистров,
а вместо FPU-инструкций вызываются сервисные библиотечные функции\\
\GTT{\_\_aeabi\_dmul}, \GTT{\_\_aeabi\_ddiv}, \GTT{\_\_aeabi\_dadd}, эмулирующие умножение, деление и сложение чисел с плавающей точкой.

Конечно, это медленнее чем FPU-сопроцессор, но это лучше, чем ничего.

Кстати, похожие библиотеки для эмуляции сопроцессорных инструкций были очень распространены в x86 
когда сопроцессор был редким и дорогим и присутствовал далеко не во всех компьютерах.

\myindex{ARM!soft float}
\myindex{ARM!armel}
\myindex{ARM!armhf}
\myindex{ARM!hard float}
Эмуляция FPU-сопроцессора в ARM называется \IT{soft float} или \IT{armel} (\IT{emulation}),
а использование FPU-инструкций сопроцессора~--- \IT{hard float} или \IT{armhf}.

\iffalse
% TODO разобраться...
\myindex{Raspberry Pi}
Ядро Linux, например, для Raspberry Pi может поставляться в двух вариантах.

В случае \IT{soft float}, аргументы будут передаваться через R-регистры, 
а в случае \IT{hard float}, через D-регистры.


И это то, что помешает использовать, например, armhf-библиотеки
из armel-кода или наоборот, поэтому, весь код в дистрибутиве Linux должен быть скомпилирован
в соответствии с выбранным соглашением о вызовах.

\fi

\subsubsection{ARM64: \Optimizing GCC (Linaro) 4.9}

Очень компактный код:

\lstinputlisting[caption=\Optimizing GCC (Linaro) 4.9,style=customasm]{patterns/12_FPU/1_simple/ARM/ARM64_GCC_O3_RU.s}

\subsubsection{ARM64: \NonOptimizing GCC (Linaro) 4.9}

\lstinputlisting[caption=\NonOptimizing GCC (Linaro) 4.9,style=customasm]{patterns/12_FPU/1_simple/ARM/ARM64_GCC_O0_RU.s}

\NonOptimizing GCC более многословный.
Здесь много ненужных перетасовок значений, включая явно избыточный код 
(последние две инструкции \INS{GMOV}).
Должно быть, GCC 4.9 пока ещё не очень хорош для генерации кода под ARM64.
Интересно заметить что у ARM64 64-битные регистры и D-регистры так же 64-битные.
Так что компилятор может сохранять значения типа \Tdouble в \ac{GPR} вместо локального стека.
Это было невозможно на 32-битных CPU.
И снова, как упражнение, вы можете попробовать соптимизировать эту функцию вручную, без добавления
новых инструкций вроде \INS{FMADD}.

}
\DE{%TODO
\subsubsection{ARM: \OptimizingXcodeIV (\ARMMode)}
Bis die Unterstützung für Fließkommaarithmetik in ARM standardisiert wurde,
fügten einige Hersteller von Prozessoren ihre eigenen Befehlserweiterungen
hinzu. 
Schließlich wurde VFP (\IT{Vector Floating Point}) standardisiert.

Ein wichtiger Unterschied zum x86 ist, dass in es in ARM keinen Stack gibt,
sondern man nur mit den Registern arbeitet.

\lstinputlisting[label=ARM_leaf_example10,caption=\OptimizingXcodeIV
(\ARMMode),style=customasm]{patterns/12_FPU/1_simple/ARM/Xcode_ARM_O3_DE.asm}

\myindex{ARM!D-\registers{}}
\myindex{ARM!S-\registers{}}

Hier sehen wir, dass einige neue Register mit einem D als Präfix verwendet
werden.

Bei diesen handelt es sich um 64-bit-Register; es gibt 32 von ihnen und sie
können sowohl für Fließkommazahlen (doppelte Genauigkeit (double)) als auch für
SIMD (heißt hier in ARM NEON) benutzt werden.

Es gibt also 32 32-bit-S-Register vorgesehen für Fließkommazahlen in einfacher
Genauigkeit (float).

Es ist leicht zu merken: D-Register sind für Zahlen in doppelter Genauigkeit,
während S-Register für einfache Genauigkeit (engl. single) vorgesehen sind.
Mehr dazu hier:\myref{ARM_VFP_registers}

Beide Konstanten (3.14 und 4.1) werden im IEEE 754 Format im Speicher abgelegt.

\myindex{ARM!\Instructions!VLDR}
\myindex{ARM!\Instructions!VMOV}
Wie man leicht sieht sind \INS{VLDR} und \INS{VMOV} analog zu den \INS{LDR} und
\MOV Befehlen, aber arbeiten auf D-Registern.

Es muss angemerkt werden, dass diese Befehle genau wie die D-Register nicht nur
für Fließkommazahlen vorgesehen sind, sondern ebenfalls für SIMD (NEON)
Operationen verwendet werden können, was wir im folgenden zeigen werden.

Die Paraemter werden der Funktion auf übliche Weise über die R-Register
übergeben, aber da jede Zahl in doppelter Genauigkeit eine Größe von 64 Bit hat
werden jeweils zwei R-Register benötigt, um eine Zahl zu übergeben.

Der Befehl \INS{VMOV D17, R0, R1} zu Beginn, fasst zwei 32-Bit-Werte aus
\Reg{0} und \Reg{1} zu einem 64-Bit-Wert zusammen und speichert diesen in
\GTT{D17}.

\INS{VMOV R0, R1, D16} ist die umgekehrte Operation: was vorher in \GTT{D16}
war, wird in zwei Register, \Reg{0} und \Reg{1} aufgeteilt, denn eine Zahl in
doppelter Genauigkeit, die 64 Bit Speicherplatz benötigt, wird über \Reg{0} und
\Reg{1} zurückgegeben.

\myindex{ARM!\Instructions!VDIV}
\myindex{ARM!\Instructions!VMUL}
\myindex{ARM!\Instructions!VADD}
\INS{VDIV}, \INS{VMUL} und \INS{VADD} sind Befehle zur Verarbeitung von
Fließkommazahlen, die \gls{Quotient}, \gls{Produkt} bzw. \gls{Summe} berechnen.

Der Code für Thumb-2 ist identisch.

\subsubsection{ARM: \OptimizingKeilVI (\ThumbMode)}

\lstinputlisting[style=customasm]{patterns/12_FPU/1_simple/ARM/Keil_O3_thumb_DE.asm}

Keil erzeugte Code für einen Prozessor ohne FPU oder NEON Unterstützung.

Die Fließkommazahlen in doppelter Genauigkeit werden über die üblichen
R-Register übergeben und anstelle von FPU-Befehlen werden Programmbibliotheken
(wie z.B. \GTT{\_\_aeabi\_dmul}, \GTT{\_\_aeabi\_ddiv}, \GTT{\_\_aeabi\_dadd})
aufgerufen, welche Multiplikation, Division und Addition auf Fließkommazahlen
emulieren. 

Diese Vorgehensweise ist natürlich langsamer als der FPU-Koprozessor, aber es
ist besser als nichts.

Übrigens waren ähnliche FPU-emulierende Programmbibliotheken auch in der
x86-Welt sehr beliebt als Koprozessoren selten und teuer waren und nur auf
wertvollen Computern installiert waren.

\myindex{ARM!soft float}
\myindex{ARM!armel}
\myindex{ARM!armhf}
\myindex{ARM!hard float}
Die Emulation des FPU-Koprozessors wird \IT{soft float} oder \IT{armel} (in der
ARM-Welt) genannt, wohingegen die FPU-Befehle des Koprozessors \IT{hard float}
oder \IT{armhf} genannt werden.

\iffalse
% TODO разобраться...
\myindex{Raspberry Pi}
Der Linux Kernel des Raspberry Pi beispielsweise wird in zwei Varianten
kompiliert.

Im Falle von \IT{soft float} werden Parameter über R-Register übergeben und im
Falle von \IT{hard float} über D-Register.

Diese Tatsache hält einen davon ab armhf-Programmbibliotheken für armel-Code
oder umgekehrt zu verwenden und dies ist der Grund warum der gesamte Code in
Linux-Distributionen speziell für eine der beiden Konventionen kompiliert wird.
\fi

\subsubsection{ARM64: \Optimizing GCC (Linaro) 4.9}

Sehr kompakter Code:

\lstinputlisting[caption=\Optimizing GCC (Linaro)
4.9,style=customasm]{patterns/12_FPU/1_simple/ARM/ARM64_GCC_O3_DE.s}

\subsubsection{ARM64: \NonOptimizing GCC (Linaro) 4.9}

\lstinputlisting[caption=\NonOptimizing GCC (Linaro)
4.9,style=customasm]{patterns/12_FPU/1_simple/ARM/ARM64_GCC_O0_DE.s}

\NonOptimizing GCC ist geschwätziger.
Hier findet eine Menge unnützes Verschieben von Werten statt, inklusive einigem
eindeutig redundantem Code (die letzten beiden \INS{FMOV} Befehle). Vermutlich
ist GCC 4.9 noch nicht besonders gut im Erzeugen von ARM64 Code.

Bemerkenswert ist, dass ARM64 64-Bit-Register besitzt und die D-Register
ebenfalls 64 Bit breit sind.

Dadurch steht es dem Compiler frei Werte von Typ \Tdouble in \ac{GPR}s anstelle
auf dem lokalen Stack zu speichern. Dies ist in 32-bit-CPUs nicht möglich.

Wiederum kann man als Übung versuchen diese Funktion manuell zu optimieren ohne
neue Befehl wie \INS{FMADD} einzuführen. 
}
\FR{\subsubsection{ARM: \OptimizingXcodeIV (\ARMMode)}

Jusqu'à la standardisation du support de la virgule flottante, certains fabricants
de processeur ont ajouté leur propre instructions étendues.
Ensuite, VFP (\IT{Vector Floating Point}) a été standardisé.

Une différence importante par rapport au x86 est qu'en ARM, il n'y a pas de pile,
vous travaillez seulement avec des registres.

\lstinputlisting[label=ARM_leaf_example10,caption=\OptimizingXcodeIV (\ARMMode),style=customasmARM]{patterns/12_FPU/1_simple/ARM/Xcode_ARM_O3_FR.asm}

\myindex{ARM!D-\registers{}}
\myindex{ARM!S-\registers{}}

Donc, nous voyons ici que des nouveaux registres sont utilisés, avec le préfixe D.

Ce sont des registres 64-bits, il y en a 32, et ils peuvent être utilisés tant pour
des nombres à virgules flottantes (double) que pour des opérations SIMD (c'est appelé
NEON ici en ARM).

Il y a aussi 32 S-registres 32 bits, destinés à être utilisés pour les nombres à
virgules flottantes simple précision (float).

C'est facile à retenir: les registres D sont pour les nombres en double précision,
tandis que les registres S----pour les nombres en simple précision
Pour aller plus loin: \myref{ARM_VFP_registers}.

Les deux constantes (3.14 et 4.1) sont stockées en mémoire au format IEEE 754.

\myindex{ARM!\Instructions!VLDR}
\myindex{ARM!\Instructions!VMOV}
\INS{VLDR} et \INS{VMOV}, comme il peut en être facilement déduit, sont analogues
aux instructions \INS{LDR} et \MOV, mais travaillent avec des registres D.

Il est à noter que ces instructions, tout comme les registres D, sont destinées non
seulement pour les nombres à virgules flottantes, mais peuvent aussi être utilisées
pour des opérations SIMD (NEON) et cela va être montré bientôt.

Les arguments sont passés à la fonction de manière classique, via les registres-R,
toutefois, chaque nombres en double précision a une taille de 64 bits, donc deux
registres-R sont nécessaires pour passer chacun d'entre eux.

\INS{VMOV D17, R0, R1} au début, combine les deux valeurs 32-bit de \Reg{0} et \Reg{1}
en une valeur 64-bit et la sauve dans \GTT{D17}.

\INS{VMOV R0, R1, D16} est l'opération inverse: ce qui est dans \GTT{D16} est séparé
dans deux registres, \Reg{0} et \Reg{1}, car un nombre en double précision qui
nécessite 64 bit pour le stockage, est renvoyé dans \Reg{0} et \Reg{1}.

\myindex{ARM!\Instructions!VDIV}
\myindex{ARM!\Instructions!VMUL}
\myindex{ARM!\Instructions!VADD}
\INS{VDIV}, \INS{VMUL} and \INS{VADD}, 
sont des instructions pour traiter des nombres à virgule flottante, qui calculent
respectivement le \gls{quotient}, \glslink{product}{produit} et la somme.

Le code pour Thumb-2 est similaire.

\subsubsection{ARM: \OptimizingKeilVI (\ThumbMode)}

\lstinputlisting[style=customasmARM]{patterns/12_FPU/1_simple/ARM/Keil_O3_thumb_FR.asm}

Code généré par Keil pour un processeur sans FPU ou support pour NEON.

Les nombres en virgule flottante double précision sont passés par des registres-R
génériques et au lieu d'instructions FPU, des fonctions d'une bibliothèque de service
sont appelées (comme \GTT{\_\_aeabi\_dmul}, \GTT{\_\_aeabi\_ddiv}, \GTT{\_\_aeabi\_dadd})
qui émulent la multiplication, la division et l'addition pour les nombres à virgule
flottante.

Bien sûr, c'est plus lent qu'un coprocesseur FPU, mais toujours mieux que rien.

A propos, de telles bibliothèques d'émulation de FPU étaient très populaire dans
le monde x86 lorsque les coprocesseurs étaient rares et chers, et étaient installés
seulement dans des ordinateurs coûteux.

\myindex{ARM!soft float}
\myindex{ARM!armel}
\myindex{ARM!armhf}
\myindex{ARM!hard float}

L'émulation d'un coprocesseur FPU est appelée \IT{soft float} ou \IT{armel} (\IT{emulation})
dans le monde ARM, alors que l'utilisation des instructions d'un coprocesseur FPU
est appelée \IT{hard float} ou \IT{armhf}.

\iffalse
% TODO разобраться...
\myindex{Raspberry Pi}

Par exemple, le noyau Linux pour Raspberry Pi est compilé en deux variantes.

Dans le case \IT{soft float}, les arguments sont passés par les registres-R, et dans
le cas \IT{hard float}---par les registes-D.

Et c'est ce qui empêche d'utiliser des bibliothèques armfh pour de code armel ou
vice-versa, et c'est pourquoi le code dans les distributions Linux doit être compilé
suivant une seule convention.
\fi

\subsubsection{ARM64: GCC \Optimizing (Linaro) 4.9}

Code très compact:

\lstinputlisting[caption=GCC \Optimizing (Linaro) 4.9,style=customasmARM]{patterns/12_FPU/1_simple/ARM/ARM64_GCC_O3_FR.s}

\subsubsection{ARM64: GCC \NonOptimizing (Linaro) 4.9}

\lstinputlisting[caption=GCC \NonOptimizing (Linaro) 4.9,style=customasmARM]{patterns/12_FPU/1_simple/ARM/ARM64_GCC_O0_FR.s}

GCC \NonOptimizing est plus verbeux.

Il y a des nombreuses modifications de valeurs inutile, incluant du code clairement
redondant (les deux dernières instructions \INS{FMOV}). Sans doute que GCC 4.9 n'est
pas encore très bon pour la génération de code ARM64.

Il est utile de noter qu'ARM64 possède des registres 64-bit, et que les registres-D
sont aussi 64-bit.

Donc le compilateur est libre de sauver des valeurs de type \Tdouble dans \ac{GPR}s
au lieu de la pile locale.
Ce n'est pas possible sur des CPUs 32-bit.

Et encore, à titre d'exercice, vous pouvez essayer d'optimiser manuellement cette
fonction, sans introduire de nouvelles instructions comme \INS{FMADD}.
}



\EN{\subsubsection{MIPS}

MIPS can support several coprocessors (up to 4), 
the zeroth of which is a special control coprocessor,
and first coprocessor is the FPU.

As in ARM, the MIPS coprocessor is not a stack machine, it has 32 32-bit registers (\$F0-\$F31):
\myref{MIPS_FPU_registers}.

When one needs to work with 64-bit \Tdouble values, a pair of 32-bit F-registers is used.

\lstinputlisting[caption=\Optimizing GCC 4.4.5 (IDA),style=customasm]{patterns/12_FPU/1_simple/MIPS_O3_IDA_EN.lst}

The new instructions here are:

\myindex{MIPS!\Instructions!LWC1}
\myindex{MIPS!\Instructions!DIV.D}
\myindex{MIPS!\Instructions!MUL.D}
\myindex{MIPS!\Instructions!ADD.D}
\begin{itemize}

\item \INS{LWC1} loads a 32-bit word into a register of the first coprocessor (hence \q{1} in instruction name).
\myindex{MIPS!\Pseudoinstructions!L.D}

A pair of \INS{LWC1} instructions may be combined into a \INS{L.D} pseudo instruction.

\item \INS{DIV.D}, \INS{MUL.D}, \INS{ADD.D} do division, multiplication, and addition respectively 
(\q{.D} in the suffix stands for double precision, \q{.S} stands for single precision)

\end{itemize}

\myindex{MIPS!\Instructions!LUI}
\myindex{\CompilerAnomaly}
\label{MIPS_FPU_LUI}

There is also a weird compiler anomaly: the \INS{LUI} instructions that we've marked with a question mark.
It's hard for me to understand why load a part of a 64-bit constant of \Tdouble type into the \$V0 register.
These instructions has no effect.
% TODO did you try checking out compiler source code?
If someone knows more about it, please drop an email to author\footnote{\EMAIL}.

}
\RU{\subsubsection{MIPS}

MIPS может поддерживать несколько сопроцессоров (вплоть до 4), нулевой из которых это специальный
управляющий сопроцессор, а первый~--- это FPU.

Как и в ARM, сопроцессор в MIPS это не стековая машина. Он имеет 32 32-битных регистра (\$F0-\$F31):

\myref{MIPS_FPU_registers}.
Когда нужно работать с 64-битными значениями типа \Tdouble, используется пара 32-битных F-регистров.

\lstinputlisting[caption=\Optimizing GCC 4.4.5 (IDA),style=customasm]{patterns/12_FPU/1_simple/MIPS_O3_IDA_RU.lst}

Новые инструкции:

\myindex{MIPS!\Instructions!LWC1}
\myindex{MIPS!\Instructions!DIV.D}
\myindex{MIPS!\Instructions!MUL.D}
\myindex{MIPS!\Instructions!ADD.D}
\begin{itemize}

\item \INS{LWC1} загружает 32-битное слово в регистр первого сопроцессора (отсюда \q{1} в названии инструкции).

\myindex{MIPS!\Pseudoinstructions!L.D}
Пара инструкций \INS{LWC1} может быть объединена в одну псевдоинструкцию \INS{L.D}.

\item \INS{DIV.D}, \INS{MUL.D}, \INS{ADD.D} производят деление, умножение и сложение соответственно 
(\q{.D} в суффиксе означает двойную точность, \q{.S}~--- одинарную точность)

\end{itemize}

\myindex{MIPS!\Instructions!LUI}
\myindex{\CompilerAnomaly}
\label{MIPS_FPU_LUI}
Здесь также имеется странная аномалия компилятора: инструкция \INS{LUI} помеченная нами вопросительным знаком.%

Мне трудно понять, зачем загружать часть 64-битной константы типа \Tdouble в регистр \$V0.

От этих инструкций нет толка.
% TODO did you try checking out compiler source code?
Если кто-то об этом что-то знает, пожалуйста, напишите автору емейл \footnote{\EMAIL}.

}
\DE{\subsubsection{MIPS}

MIPS unterstütz mehrere Koprozessoren (bis zu 4); der nullte\footnote{TBT} in ein spezeiller
Kontroll-Koprozessor und der erste Koprozessor ist die FPU.

Genau wie in ARM ist der MIPS Koprozessor keine Stackmaschine, sondern hat 32
32-bit-Register (\$F0-\$F31):
\myref{MIPS_FPU_registers}.

Muss man mit 64-bit \Tdouble Werten arbeiten, wird ein Paar 32-bit F-Register
hierfür verwendet.

\lstinputlisting[caption=\Optimizing GCC 4.4.5
(IDA)]{patterns/12_FPU/1_simple/MIPS_O3_IDA_DE.lst}

Die neuen Befehl sind im Einzelnen:

\myindex{MIPS!\Instructions!LWC1}
\myindex{MIPS!\Instructions!DIV.D}
\myindex{MIPS!\Instructions!MUL.D}
\myindex{MIPS!\Instructions!ADD.D}
\begin{itemize}

\item \INS{LWC1} lädt ein 32-bit-Wort in ein Register des ersten Koprozessors
(daher \q{1} im Namen des Befehls).
\myindex{MIPS!\Pseudoinstructions!L.D}

Ein Parr \INS{LWC1} Befehle kann zu einem \INS{L.D} Pseudobefehl zusammengefasst
werden.

\item \INS{DIV.D}, \INS{MUL.D}, \INS{ADD.D} führen Division, Multiplikation bzw.
Addition aus (das \q{D.} im Suffix steht für doppelte Genauigkeit, \q{S.}
bedeutet entsprechend einfache Genauigkeit).

\end{itemize}

\myindex{MIPS!\Instructions!LUI}
\myindex{\CompilerAnomaly}
\label{MIPS_FPU_LUI}
Es gibt ein verrückte Anomalie im Compiler: die \INS{LUI} Befehle, die wir mit
einem Fragezeichen versehen haben. 
Es ist sehr schwer zu verstehen, warum ein Teil einer
64-bit-Konstante vom Typ \Tdouble in das \$V0 Register geladen wird. 
Dieser Befehl hat keine Auswirkungen. 

% TODO did you try checking out compiler source code?
Sollte jemand mehr zu dieser Anomalie wissen, bittet der Autor um eine
Mail\footnote{\EMAIL}.

}
\FR{\subsubsection{MIPS}

MIPS peut supporter plusieurs coprocesseur (jusqu'à 4), le zérotième\footnote{Barbarisme pour rappeler que les indices commencent à zéro.} est un coprocesseur
contrôleur spécial, et celui d'indice 1 est le FPU.

Comme en ARM, le coprocesseur MIPS n'est pas une machine à pile, il comprend 32 registres
32-bit (\$F0-\$F31):
\myref{MIPS_FPU_registers}.

Lorsque l'on doit travailler avec des valeurs \Tdouble 64-bit, une paire de registre-F
32-bit est utilisée.

\lstinputlisting[caption=GCC 4.4.5 \Optimizing (IDA),style=customasmMIPS]{patterns/12_FPU/1_simple/MIPS_O3_IDA_FR.lst}

Les nouvelles instructions ici sont:

\myindex{MIPS!\Instructions!LWC1}
\myindex{MIPS!\Instructions!DIV.D}
\myindex{MIPS!\Instructions!MUL.D}
\myindex{MIPS!\Instructions!ADD.D}
\begin{itemize}

\item \INS{LWC1} charge un mot de 32-bit dans un registre du premier coprocesseur
(d'où le \q{1} dans le nom de l'instruction).
\myindex{MIPS!\Pseudoinstructions!L.D}

Une paire d'instructions \INS{LWC1} peut être combinée en une pseudo instruction \INS{L.D}.

\item \INS{DIV.D}, \INS{MUL.D}, \INS{ADD.D} effectuent respectivement la division,
la multiplication, et l'addition (\q{.D} est le suffixe standard pour la double précision,
\q{.S} pour la simple précision)

\end{itemize}

\myindex{MIPS!\Instructions!LUI}
\myindex{\CompilerAnomaly}
\label{MIPS_FPU_LUI}

Il y a une anomalie bizarre du compilateur: l'instruction \INS{LUI} que nous avons
marqué avec un point d'interrogation.
Il m'est difficile de comprendre pourquoi charger une partie de la constante de type
64-bit \Tdouble dans le registre \$V0. Cette instruction n'a pas d'effet.
% TODO did you try checking out compiler source code?
Si quelqu'un en sait plus sur ceci, s'il vous plaît, envoyez un email à l'auteur\footnote{\EMAIL}.

}


\subsection{\RU{Передача чисел с плавающей запятой в аргументах}\EN{Passing floating point numbers via arguments}\DEph{}}
\myindex{\CStandardLibrary!pow()}

\lstinputlisting[style=customc]{patterns/12_FPU/2_passing_floats/pow.c}

\EN{\subsubsection{x86}

Let's see what we get in (MSVC 2010):

\lstinputlisting[caption=MSVC 2010,style=customasmx86]{patterns/12_FPU/2_passing_floats/MSVC_EN.asm}

\myindex{x86!\Instructions!FLD}
\myindex{x86!\Instructions!FSTP}

\FLD and \FSTP move variables between the data segment and the FPU stack. 
\GTT{pow()}\footnote{a standard C function, raises a number to the given power (exponentiation)}
takes both values from the stack and returns its result in the \ST{0} register.
\printf takes 8 bytes from the local stack and interprets them as \Tdouble type variable.

By the way, a pair of \MOV instructions could be used here for moving values from the memory
into the stack, because the values in memory are stored in IEEE 754 format, and pow() also takes them in this
format, so no conversion is necessary.
That's how it's done in the next example, for ARM: \myref{FPU_passing_floats_ARM}.

}
\RU{\subsubsection{x86}

Посмотрим, что у нас вышло (MSVC 2010):

\lstinputlisting[caption=MSVC 2010,style=customasmx86]{patterns/12_FPU/2_passing_floats/MSVC_RU.asm}

\myindex{x86!\Instructions!FLD}
\myindex{x86!\Instructions!FSTP}
\FLD и \FSTP перемещают переменные из сегмента данных в FPU-стек или обратно. 
\GTT{pow()}\footnote{стандартная функция Си, возводящая число в степень} достает оба значения из стека и 
возвращает результат в \ST{0}. 
\printf берет 8 байт из стека и трактует их как переменную типа \Tdouble.

Кстати, с тем же успехом можно было бы перекладывать эти два числа из памяти в стек при помощи пары \MOV:
 
ведь в памяти числа в формате IEEE 754, pow() также принимает их в том же
формате, и никакая конверсия не требуется.

Собственно, так и происходит в следующем примере с ARM: \myref{FPU_passing_floats_ARM}.

}
\DE{\subsubsection{x86}
Schauen wir uns an, was wir in MSVC 2010 erhalten:

\lstinputlisting[caption=MSVC 2010,style=customasmx86]{patterns/12_FPU/2_passing_floats/MSVC_DE.asm}

\myindex{x86!\Instructions!FLD}
\myindex{x86!\Instructions!FSTP}
% TODO bug to be fixed here:
\FLD und \FSTP verschieben Variablen zwischen Datensegment und dem FPU
Stack.\GTT{pow()}\footnote{eine Standard-C-Funktion, die eine Zahl potenziert}
nimmt beide Werte vom Stack der FPU und gibt ihr Ergebnis über das \ST{0} Register zurück. 
Die Funktion \printf nimmt 8 Byte vom lokalen Stack und interpretiert diese als
Variable von Typ \Tdouble.

Übrigens könnte hier auch ein Paar \MOV Befehle verwendet werden, um die Werte
aus dem Speicher zu holen und auf den Stack zu legen, denn die Werte sind im
Speicher im IEEE 754 Format abgelegt und pow() arbeitet mit diesem Format,
sodass keine Umwandlung notwendig ist.
Genau so wird es im folgenden Beispiel für ARM auch
gemacht:\myref{FPU_passing_floats_ARM}
}

\EN{\subsubsection{ARM + \NonOptimizingXcodeIV (\ThumbTwoMode)}
\label{FPU_passing_floats_ARM}

\lstinputlisting[style=customasm]{patterns/12_FPU/2_passing_floats/Xcode_thumb_O0.asm}

As it was mentioned before, 64-bit floating pointer numbers are passed in R-registers pairs.

This code is a bit redundant (certainly because optimization is turned off), 
since it is possible to load values into the R-registers directly without touching the D-registers.

So, as we see, the \GTT{\_pow} function receives its first argument in \Reg{0} and \Reg{1}, and its second one in \Reg{2} and \Reg{3}. 
The function leaves its result in \Reg{0} and \Reg{1}.
The result of \GTT{\_pow} is moved into \GTT{D16}, then in the \Reg{1} and \Reg{2} pair, from where \printf takes the resulting number.

\subsubsection{ARM + \NonOptimizingKeilVI (\ARMMode)}

\lstinputlisting[style=customasm]{patterns/12_FPU/2_passing_floats/Keil_ARM_O0.asm}

D-registers are not used here, just R-register pairs.

\subsubsection{ARM64 + \Optimizing GCC (Linaro) 4.9}

\lstinputlisting[caption=\Optimizing GCC (Linaro) 4.9,style=customasm]{patterns/12_FPU/2_passing_floats/ARM64_EN.s}

The constants are loaded into \RegD{0} and \RegD{1}: \TT{pow()} takes them from there.
The result will be in \RegD{0} after the execution of \TT{pow()}.
It is to be passed to \printf without any modification and moving, 
because \printf takes arguments of \glslink{integral type}{integral types} 
and pointers from X-registers, and floating point arguments from D-registers.

}
\RU{\subsubsection{ARM + \NonOptimizingXcodeIV (\ThumbTwoMode)}
\label{FPU_passing_floats_ARM}

\lstinputlisting[style=customasmARM]{patterns/12_FPU/2_passing_floats/Xcode_thumb_O0.asm}

Как уже было указано, 64-битные числа с плавающей точкой передаются в парах R-регистров.

Этот код слегка избыточен (наверное, потому что не включена оптимизация), ведь можно было бы 
загружать значения напрямую в R-регистры минуя загрузку в D-регистры.

Итак, видно, что функция \GTT{\_pow} получает первый аргумент в \Reg{0} и \Reg{1}, а второй в \Reg{2} и \Reg{3}. 
Функция оставляет результат в \Reg{0} и \Reg{1}.
Результат работы \GTT{\_pow} перекладывается в \GTT{D16}, 
затем в пару \Reg{1} и \Reg{2}, откуда 
\printf берет это число-результат.

\subsubsection{ARM + \NonOptimizingKeilVI (\ARMMode)}

\lstinputlisting[style=customasmARM]{patterns/12_FPU/2_passing_floats/Keil_ARM_O0.asm}

Здесь не используются D-регистры, используются только пары R-регистров.

\subsubsection{ARM64 + \Optimizing GCC (Linaro) 4.9}

\lstinputlisting[caption=\Optimizing GCC (Linaro) 4.9,style=customasmARM]{patterns/12_FPU/2_passing_floats/ARM64_RU.s}

Константы загружаются в \RegD{0} и \RegD{1}: 
функция \TT{pow()} берет их оттуда.
Результат в \RegD{0} после исполнения \TT{pow()}.
Он пропускается в \printf без всякой модификации и перемещений, 
потому что \printf берет аргументы \glslink{integral type}{интегральных типов} и указатели 
из X-регистров, а аргументы типа плавающей точки из D-регистров.

}
\DE{\subsubsection{ARM + \NonOptimizingXcodeIV (\ThumbTwoMode)}
\label{FPU_passing_floats_ARM}

\lstinputlisting[style=customasm]{patterns/12_FPU/2_passing_floats/Xcode_thumb_O0.asm}
Wie bereits vorher erwähnt werden Pointer auf 64-Bit-Fließkommazahlen über ein
Paar von R-Registern übergeben.

Dieser Code ist leicht redundant (sicherlich aufgrund der deaktivierten
Optimierung), da es möglich ist Werte direkt in die R-Register zu laden, ohne
die D-Register zu verwenden.

Wie wir also sehen erhält die \GTT{\_pow} Funktion ihr erster Argument in
\Reg{0} und \Reg{1} und das zweite in \Reg{2} und \Reg{3}. Die Funktion
speichert ihr Ergebnis in \Reg{0} und \Reg{1}. 
Das Ergebnis von \GTT{\_pow} wird zunächst nach \GTT{D16} und
anschließend in das Paar \Reg{1} und \Reg{2} verschoben, von wo aus \printf das
Ergebnis übernimmt. 

\subsubsection{ARM + \NonOptimizingKeilVI (\ARMMode)}

\lstinputlisting[style=customasm]{patterns/12_FPU/2_passing_floats/Keil_ARM_O0.asm}

Die D-Register werden hier nicht verwendet, sondern nur Paare von R-Registern.

\subsubsection{ARM64 + \Optimizing GCC (Linaro) 4.9}

\lstinputlisting[caption=\Optimizing GCC (Linaro)
4.9,style=customasm]{patterns/12_FPU/2_passing_floats/ARM64_DE.s}

Die Konstanten werden nach \RegD{0} und \RegD{1} geladen: \TT{pow()} übernimmt
sie von dort. Das Ergebnis befindet sich nach der Ausführung von \TT{pow()} in
\RegD{0}. 
Es wird ohne weitere Änderung oder Verschiebung an die Funktion \prinft
übergeben, da \printf ganzzahlige Werte und Pointer aus X-Registern,
Fließkommaparameter jedoch aus D-Registern übernimmt.

}

\EN{\subsubsection{MIPS}

\lstinputlisting[caption=\Optimizing GCC 4.4.5 (IDA),style=customasm]{patterns/12_FPU/2_passing_floats/MIPS_O3_IDA_EN.lst}

And again, we see here \INS{LUI} loading a 32-bit part of a \Tdouble number into \$V0.
And again, it's hard to comprehend why.

\myindex{MIPS!\Instructions!MFC1}

The new instruction for us here is \INS{MFC1} (\q{Move From Coprocessor 1}).
The FPU is coprocessor number 1, hence \q{1} in the instruction name.
This instruction transfers values from the coprocessor's registers to the registers of the CPU (\ac{GPR}).
So at the end the result of \TT{pow()} is moved to registers \$A3 and \$A2, 
and \printf takes a 64-bit double value from this register pair.

}
\RU{\subsubsection{MIPS}

\lstinputlisting[caption=\Optimizing GCC 4.4.5 (IDA),style=customasmMIPS]{patterns/12_FPU/2_passing_floats/MIPS_O3_IDA_RU.lst}

И снова мы здесь видим, как \INS{LUI} загружает 32-битную часть числа типа \Tdouble в \$V0.
И снова трудно понять почему.

\myindex{MIPS!\Instructions!MFC1}
Новая для нас инструкция это \INS{MFC1} (\q{Move From Coprocessor 1}) (копировать из первого сопроцессора).
FPU это сопроцессор под номером 1, вот откуда \q{1} в имени инструкции.
Эта инструкция переносит значения из регистров сопроцессора в регистры основного CPU (\ac{GPR}).
Так что результат исполнения \TT{pow()} в итоге копируется в регистры \$A3 и \$A2
и из этой пары регистров \printf берет его как 64-битное значение типа \Tdouble.

}
\DE{\subsubsection{MIPS}

\lstinputlisting[caption=\Optimizing GCC 4.4.5
(IDA)]{patterns/12_FPU/2_passing_floats/MIPS_O3_IDA_DE.lst}
Und wieder sehen wir hier, dass der Befehl \INS{LUI} einen 32-Bit-Teil einer
\Tdouble Zahl nach \$V0 lädt.
Und wiederum ist es schwer nachzuvollziehen warum dies geschieht.

\myindex{MIPS!\Instructions!MFC1}
Der für uns neue Befehl an dieser Stelle ist \INS{MFC1}(\q{Move From Coprocessor
1}). Die Nummer des FPU-Koprozessors ist 1, daher die \q{1} im Namen des
Befehls. 
Dieser Befehl überträgt Werte aus den Registern des Koprozessors in die Register
der CPU (\ac{GPR}).
Auf diese Weise wird das Ergebnis von \TT{pow()} schließlich in die Register
\$A3 und \$A2 verschoben und \printf übernimmt einen 64-Bit-Wert von doppelter
Genauigkeit aus diesem Registerpaar.}


\subsection{\RU{Пример с сравнением}\EN{Comparison example}}

\RU{Попробуем теперь вот это:}\EN{Let's try this:}

\lstinputlisting[style=customc]{patterns/12_FPU/3_comparison/d_max.c}

\RU{Несмотря на кажущуюся простоту этой функции, понять, как она работает, будет чуть сложнее.}%
\EN{Despite the simplicity of the function, it will be harder to understand how it works.}

% subsections
\subsubsection{x86}

% subsubsections
\EN{\myparagraph{\NonOptimizing MSVC}

MSVC 2010 generates the following:

\lstinputlisting[caption=\NonOptimizing MSVC 2010,style=customasm]{patterns/12_FPU/3_comparison/x86/MSVC/MSVC_EN.asm}

\myindex{x86!\Instructions!FLD}

So, \FLD loads \GTT{\_b} into \ST{0}.

\label{Czero_etc}
\newcommand{\Czero}{\GTT{C0}\xspace}
\newcommand{\Ctwo}{\GTT{C2}\xspace}
\newcommand{\Cthree}{\GTT{C3}\xspace}
\newcommand{\CThreeBits}{\Cthree/\Ctwo/\Czero}

\myindex{x86!\Instructions!FCOMP}

\FCOMP compares the value in \ST{0} with what is in \GTT{\_a} 
and sets \CThreeBits bits in FPU status word register, accordingly. 
This is a 16-bit register that reflects the current state of the FPU.

After the bits are set, the \FCOMP instruction also pops one variable from the stack. 
This is what distinguishes it from \FCOM, which is just compares values, leaving the stack in the same state.

Unfortunately, CPUs before Intel P6
\footnote{Intel P6 is Pentium Pro, Pentium II, etc.} don't have any conditional 
jumps instructions which check the \CThreeBits bits. 
Perhaps, it is a matter of history (recall: FPU was a separate chip in past).\\
Modern CPU starting at Intel P6 have \FCOMI/\FCOMIP/\FUCOMI/\FUCOMIP 
instructions~---which do the same, but modify the \ZF/\PF/\CF CPU flags.

\myindex{x86!\Instructions!FNSTSW}

The \FNSTSW instruction copies FPU the status word register to \AX. 
\CThreeBits bits are placed at positions 14/10/8, 
they are at the same positions in the \AX register and all they are placed in the high part of \AX{}~---\AH{}.

\begin{itemize}
\item If $b>a$ in our example, then \CThreeBits bits are to be set as following: 0, 0, 0.
\item If $a>b$, then the bits are: 0, 0, 1.
\item If $a=b$, then the bits are: 1, 0, 0.
\item

If the result is unordered (in case of error), then the set bits are: 1, 1, 1.
\end{itemize}
% TODO: table here?

This is how \CThreeBits bits are located in the \AX register:

\input{C3_in_AX}

This is how \CThreeBits bits are located in the \AH register:

\input{C3_in_AH}

After the execution of \INS{test ah, 5}\footnote{5=101b}, 
only \Czero and \Ctwo bits (on 0 and 2 position) are considered, all other bits are just
ignored.

\label{parity_flag}
\myindex{x86!\Registers!\Flags!Parity flag}

Now let's talk about the \IT{parity flag}, another notable historical rudiment.

This flag is set to 1 if the number of ones in the result of the last calculation is even, and to 0 if it is odd.

Let's look into Wikipedia\footnote{\href{http://go.yurichev.com/17131}{wikipedia.org/wiki/Parity\_flag}}:

\begin{framed}
\begin{quotation}
One common reason to test the parity flag actually has nothing to do with parity. The FPU has four condition flags 
(C0 to C3), but they cannot be tested directly, and must instead be first copied to the flags register. 
When this happens, C0 is placed in the carry flag, C2 in the parity flag and C3 in the zero flag. 
The C2 flag is set when e.g. incomparable floating point values (NaN or unsupported format) are compared 
with the FUCOM instructions.
\end{quotation}
\end{framed}

As noted in Wikipedia, the parity flag used sometimes in FPU code, let's see how.

\myindex{x86!\Instructions!JP}

The \PF flag is to be set to 1 if both \Czero and \Ctwo are set to 0 or both are 1, in which case
the subsequent \JP (\IT{jump if PF==1}) is triggering. 
If we recall the values of \CThreeBits for various cases,
we can see that the conditional jump 
\JP is triggering in two cases: if $b>a$ or $a=b$ 
(\Cthree bit is not considered here, since it has been cleared by the \INS{test ah, 5} instruction).

It is all simple after that. 
If the conditional jump has been triggered, 
\FLD loads the value of \GTT{\_b} 
in \ST{0}, and if it hasn't been triggered, the value of \GTT{\_a} is loaded there.

\mysubparagraph{And what about checking \Ctwo?}

The \Ctwo flag is set in case of error (\gls{NaN}, etc.), but our code doesn't check it.

If the programmer cares about FPU errors, he/she must add additional checks.

\input{patterns/12_FPU/3_comparison/x86/MSVC/olly_EN.tex}
}
\RU{\myparagraph{\NonOptimizing MSVC}

Вот что выдал MSVC 2010:

\lstinputlisting[caption=\NonOptimizing MSVC 2010,style=customasm]{patterns/12_FPU/3_comparison/x86/MSVC/MSVC_RU.asm}

\myindex{x86!\Instructions!FLD}
Итак, \FLD загружает \GTT{\_b} в регистр \ST{0}.

\label{Czero_etc}
\newcommand{\Czero}{\GTT{C0}\xspace}
\newcommand{\Ctwo}{\GTT{C2}\xspace}
\newcommand{\Cthree}{\GTT{C3}\xspace}
\newcommand{\CThreeBits}{\Cthree/\Ctwo/\Czero}

\myindex{x86!\Instructions!FCOMP}
\FCOMP сравнивает содержимое \ST{0} с тем, что лежит в \GTT{\_a} и выставляет биты \CThreeBits в 
регистре статуса FPU. Это 16-битный регистр отражающий текущее состояние FPU.

После этого инструкция \FCOMP также выдергивает одно значение из стека. 
Это отличает её от \FCOM, которая просто сравнивает значения, оставляя стек в таком же состоянии.

К сожалению, у процессоров до Intel P6
\footnote{Intel P6 это Pentium Pro, Pentium II, и последующие модели} нет инструкций условного перехода,
проверяющих биты \CThreeBits.
Возможно, так сложилось исторически (вспомните о том, что FPU когда-то был вообще отдельным чипом).\\
А у Intel P6 появились инструкции \FCOMI/\FCOMIP/\FUCOMI/\FUCOMIP, делающие то же самое, 
только напрямую модифицирующие флаги \ZF/\PF/\CF.

\myindex{x86!\Instructions!FNSTSW}
Так что \FNSTSW копирует содержимое регистра статуса в \AX. 
Биты \CThreeBits занимают позиции, 
соответственно, 14, 10, 8. В этих позициях они и остаются в регистре \AX, 
и все они расположены в старшей части регистра~--- \AH.

\begin{itemize}
\item Если $b>a$ в нашем случае, то биты \CThreeBits должны быть выставлены так: 0, 0, 0.
\item Если $a>b$, то биты будут выставлены: 0, 0, 1.
\item Если $a=b$, то биты будут выставлены так: 1, 0, 0.
\item Если результат не определен (в случае ошибки), то биты будут выставлены так: 1, 1, 1.
\end{itemize}
% TODO: table here?

Вот как биты \CThreeBits расположены в регистре \AX:

\input{C3_in_AX}

Вот как биты \CThreeBits расположены в регистре \AH:

\input{C3_in_AH}

После исполнения \INS{test ah, 5}\footnote{5=101b} % FIXME: subscript here!
будут учтены только биты \Czero и \Ctwo (на позициях 0 и 2), остальные просто проигнорированы.

\label{parity_flag}
\myindex{x86!\Registers!\Flags!Флаг четности}
Теперь немного о \IT{parity flag}\footnote{флаг четности}. 
Ещё один замечательный рудимент эпохи.

Этот флаг выставляется в 1 если количество единиц в последнем результате четно. 
И в 0 если нечетно.

Заглянем в Wikipedia\footnote{\href{http://go.yurichev.com/17131}{wikipedia.org/wiki/Parity\_flag}}:

\begin{framed}
\begin{quotation}
One common reason to test the parity flag actually has nothing to do with parity. The FPU has four condition flags 
(C0 to C3), but they cannot be tested directly, and must instead be first copied to the flags register. 
When this happens, C0 is placed in the carry flag, C2 in the parity flag and C3 in the zero flag. 
The C2 flag is set when e.g. incomparable floating point values (NaN or unsupported format) are compared 
with the FUCOM instructions.
\end{quotation}
\end{framed}

Как упоминается в Wikipedia, флаг четности иногда используется в FPU-коде и сейчас мы увидим как.

\myindex{x86!\Instructions!JP}
Флаг \PF будет выставлен в 1, если \Czero и \Ctwo оба 1 или оба 0. 
И тогда сработает последующий \JP (\IT{jump if PF==1}). 
Если мы вернемся чуть назад и посмотрим значения \CThreeBits 
для разных вариантов, то увидим, что условный переход \JP сработает в двух случаях: если $b>a$ или если $a=b$ 
(ведь бит \Cthree перестал учитываться после исполнения \INS{test ah, 5}).

Дальше всё просто. Если условный переход сработал, то \FLD загрузит значение \INS{\_b} в \ST{0}, 
а если не сработал, то загрузится \GTT{\_a} и произойдет выход из функции.

\mysubparagraph{А как же проверка флага \Ctwo?}

Флаг \Ctwo включается в случае ошибки (\gls{NaN}, итд.), но наш код его не проверяет.

Если программисту нужно знать, не произошла ли FPU-ошибка, он должен позаботиться об этом
дополнительно, добавив соответствующие проверки.

\input{patterns/12_FPU/3_comparison/x86/MSVC/olly_RU.tex}
}
\DE{\myparagraph{\NonOptimizing MSVC}

MSVC 2010 erzeugt den folgenden Code:

\lstinputlisting[caption=\NonOptimizing MSVC
2010,style=customasmx86]{patterns/12_FPU/3_comparison/x86/MSVC/MSVC_DE.asm}

\myindex{x86!\Instructions!FLD}

Der Befehl \FLD lädt \GTT{\_b} nach \ST{0}.

\label{Czero_etc}
\newcommand{\Czero}{\GTT{C0}\xspace}
\newcommand{\Ctwo}{\GTT{C2}\xspace}
\newcommand{\Cthree}{\GTT{C3}\xspace}
\newcommand{\CThreeBits}{\Cthree/\Ctwo/\Czero}

\myindex{x86!\Instructions!FCOMP}
\FCOMP verlgeicht den Wert in \ST{0} mit dem Wert, der sich in \GTT{\_a}
befindet und setzt die \CThreeBits im FPU Status Register entsprechend.
Das Statusregister ist ein 16-Bit-Register, das den aktueller Zustand der FPU
abbildet.

Nachdem die Bits gesetzt worden sind, nimmer der \FCOMP Befehl auch eine
Variable vom Stack. Dieses Verhalten unterscheidet ihn von \FCOM, der einfach
zwei Werte vergleicht und den Stack unangetastet lässt.

Leider verfügen CPUs vor Intel P6\footnote{Intel P6 ist Pentium Pro, Pentium II,
etc.}über keinerlei bedingte Sprungbefehle, die die \CThreeBits prüfen.

After the bits are set, the \FCOMP instruction also pops one variable from the stack. 
This is what distinguishes it from \FCOM, which is just compares values, leaving the stack in the same state.
Vielleicht ist diese Tatsache historisch begründet (man erinnere sich: die FPU
war früher ein eigener Chip).\\
Moderne CPUs, beginnend mit Intel P6 haben \FCOMI/\FCOMIP/\FUCOMI/\FUCOMIP
Befehle~--welche im Prinzip das gleiche tun, aber die \ZF/\PF/\CF Flags der CPU
verändern können.

\myindex{x86!\Instructions!FNSTSW}
Der \FNSTSW Befehl kopiert das FPU Statusregister nach \AX.
\CThreeBits werden an den Stellen 14/10/8 abgelegt, sie befinden sich im \AX
Register an den gleichen Stellen und sie werden alle in höherwertigen Teil von
\AX{}~---\AH{} abgelegt.

\begin{itemize}
\item Falls in unserem Beispiel $b>a$, dann werden die \CThreeBits Bits wie
folgt gesetzt: 0, 0, 0.
\item Falls $a>b$, dann ist das Bitmuster: 0, 0, 1.
\item Falls $a=b$, dann ist das Bitmuster: 1, 0, 0.
\item

Wenn das Ergebnis (z.B. im Fehlerfall) ungeordnet ist, dann werden die Bits wie
folgt gesetzt: 1,1,1.
\end{itemize}
% TODO: table here?
So werden die \CThreeBits Bits im \AX Register angeordnet:

\input{C3_in_AX}

So werden die \CThreeBits Bits im \AH Register angeordnet:

\input{C3_in_AH}
Nach der Ausführung von \INS{test ah, 5}\footnote{5=101b} werden nur die \Czero
und \Ctwo Bits (an den Stellen 0 und 2) betrachtet, alle übrigen Bits werden
einfach überlesen.

\label{parity_flag}
\myindex{x86!\Registers!\Flags!Parity flag}
Werfen wir nun einen Blick auf ein anderes bemerkenswertes historisches
Überbleibsel: das \IT{parity flag}.

Dieses Flag wird auf 1 gesetzt, falls die Anzahl der Einsen im Ergebnis der
letzten Berechnung gerade ist und auf 1, falls dies nicht der Fall ist.

Schlagen wir in der Wikipedia
nach\footnote{\href{http://go.yurichev.com/17131}{wikipedia.org/wiki/Parity\_flag}}:

%TODO Quotation has been translated from English wiki article, since the
% correspondig German article doesn't offer such information.
\begin{framed}
\begin{quotation}
Ein guter Grund das Parity Flag abzufragen, hat tatsächlich gar nichts mit
Parität zu tun. Die FPU hat vier Bedingungsflags (C0 bis C3), aber diese können
nicht direkt abgefragt werden, sondern müssen zunächst in das Flags Register
kopiert werden. Wenn dies geschieht, wird C0 im Carry Flag abgelegt, C2 im
Parity Flag und C3 im Zero Flag.
Das C2 Flag ist gesetzt, wenn z.B. unvergleichbare Fließkommawerte (NaN oder
nicht unterstütztes Format) über der \FUCOM Befehl miteinander verglichen
werden.\textit{(Übersetzung aus der englischen Wikipedia.)}
\end{quotation}
\end{framed}

Wie in der Wikipedia dargestellt wird das Parity Flag manchmal im FPU Code
verwendet; schauen wir uns genauer an wie das funktioniert.

\myindex{x86!\Instructions!JP}
Das \PF Flag wird auf 1 gesetzt, wenn sowohl \Czero als auch \Ctwo beide 0 oder
beide 1 sind. In diesem Fall wird der nachfolgende Sprung \JP(\IT{jump if
PF==1}) ausgeführt.
Wenn wir die Werte der \CThreeBits in den unterschiedlichen Fällen betrachten,
dann sehen wir, dass der bedingte Sprung \JP in zwei Fällen ausgeführt wird:
wenn $b>a$ oder wenn $a=b$ (das \Cthree Bit wird hier nicht betrachtet, da es
durch den Befehl \INS{test ah,5}) gelöscht wurde).

Der Rest ist leicht nachvollziehbar.
Denn der bedingte Sprung ausgeführt wurde, lädt \FLD den Wert von \GTT{\_b} nach
\ST{0} und wenn nicht, wird der Wert von \GTT{\_a} dorthin geladen.

\mysubparagraph{Was ist mit der Abfrage von \Ctwo?}
Das \Ctwo Flag wird im Fehlerfall (\gls{NaN}, etc.) gesetzt, aber unser Code
prüft dies nicht. 
Wenn sich der Programmierer für FPU Fehler interessiert, muss er zusätzliche
Abfragen hinzufügen.

\input{patterns/12_FPU/3_comparison/x86/MSVC/olly_DE.tex}
}
\FR{\myparagraph{\NonOptimizing MSVC}

MSVC 2010 génère ce qui suit:

\lstinputlisting[caption=\NonOptimizing MSVC 2010,style=customasmx86]{patterns/12_FPU/3_comparison/x86/MSVC/MSVC_FR.asm}

\myindex{x86!\Instructions!FLD}

Ainsi, \FLD charge \GTT{\_b} dans \ST{0}.

\label{Czero_etc}
\newcommand{\Czero}{\GTT{C0}\xspace}
\newcommand{\Ctwo}{\GTT{C2}\xspace}
\newcommand{\Cthree}{\GTT{C3}\xspace}
\newcommand{\CThreeBits}{\Cthree/\Ctwo/\Czero}

\myindex{x86!\Instructions!FCOMP}

\FCOMP compare la valeur dans \ST{0} avec ce qui est dans \GTT{\_a} et met les bits
\CThreeBits du mot registre d'état du FPU, suivant le résultat.
Ceci est un registre 16-bit qui reflète l'état courant du FPU.

Après que les bits ont été mis, l'instruction \FCOMP dépile une variable depuis la
pile.
C'est ce qui la différencie de \FCOM, qui compare juste les valeurs, laissant la
pile dans le même état.

Malheureusement, les CPUs avant les Intel P6\footnote{Intel P6 consiste en les Pentium
Pro, Pentium II, etc.} ne possèdent aucune instruction de saut conditionnel qui teste
les bits \CThreeBits.
Peut-être est-ce une raison historique (rappel: le FPU était une puce séparée dans
le passé).\\Les CPU modernes, à partir des Intel P6 possèdent les instructions \FCOMI/\FCOMIP/\FUCOMI/\FUCOMIP~---qui
font la même chose, mais modifient les flags \ZF/\PF/\CF du CPU.

\myindex{x86!\Instructions!FNSTSW}

L'instruction \FNSTSW copie le le mot du registre d'état du FPU dans \AX.
Les bits \CThreeBits sont placés aux positions 14/10/8, ils sont à la même position
dans le registre \AX et tous sont placés dans la partie haute de \AX{}~---\AH{}.

\begin{itemize}
\item Si $b>a$ dans notre exemple, alors les bits \CThreeBits sont mis comme ceci: 0, 0, 0.
\item Si $a>b$, alors les bits sont: 0, 0, 1.
\item Si $a=b$, alors les bits sont: 1, 0, 0.

Si le résultat n'est pas ordonné (en cas d'erreur), alors les bits sont: 1, 1, 1.
\end{itemize}
% TODO: table here?

Voici comment les bits \CThreeBits sont situés dans le registre \AX:

\input{C3_in_AX}

Voici comment les bits \CThreeBits sont situés dans le registre \AH:

\input{C3_in_AH}

Après l'exécution de \INS{test ah, 5}\footnote{5=101b}, seul les bits \Czero et \Ctwo
(en position 0 et 2) sont considérés, tous les autres bits sont simplement ignorés.

\label{parity_flag}
\myindex{x86!\Registers!\Flags!Parity flag}

Parlons maintenant du \IT{parity flag} (flag de parité), un autre rudiment historique
remarquable.

Ce flag est mis à 1 si le nombre de un dans le résultat du dernier calcul est pair,
et à 0 s'il est impair.

Regardons sur Wikipédia\footnote{\href{http://go.yurichev.com/17131}{wikipedia.org/wiki/Parity\_flag}}:

\begin{framed}
\begin{quotation}
Une raison commune de tester le bit de parité n'a rien à voir avec la parité. Le FPU
possède quatre flags de condition (C0 à C3), mais ils ne peuvent pas être testés
directement, et doivent d'abord être copiés dans le registre d'états.
Lorsque ça se produit, C0 est mis dans le flag de retenue, C2 dans le flag
de parité et C3 dans le flag de zéro.
Le flag C2 est mis lorsque e.g. des valeurs en virgule flottantes incomparable
(NaN ou format non supporté) sont comparées avec l'instruction \FUCOM.
\end{quotation}
\end{framed}

Comme indiqué dans Wikipédia, le flag de parité est parfois utilisé dans du code
FPU, voyons comment.

\myindex{x86!\Instructions!JP}

Le flag \PF est mis à 1 si à la fois \Czero et \Ctwo sont mis à 0 ou si les deux
sont à 1, auquel cas le \JP (\IT{jump if PF==1}) subséquent est déclenché.
Si l'on se rappelle les valeurs de \CThreeBits pour différents cas, nous pouvons
voir que le saut conditionnel \JP est déclenché dans deux cas: si $b>a$ ou $a=b$
(le bit \Cthree n'est pris en considération ici, puisqu'il a été mis à 0 par l'instruction
\INS{test ah, 5}).

C'est très simple ensuite.
Si le saut conditionnel a été déclenché, \FLD charge la valeur de \GTT{\_b} dans
\ST{0}, et sinon, la valeur de \GTT{\_a} est chargée ici.

\mysubparagraph{Et à propos du test de \Ctwo?}

Le flag \Ctwo est mis en cas d'erreur (\gls{NaN}, etc.), mais notre code ne le teste
pas.

Si le programmeur veut prendre en compte les erreurs FPU, il doit ajouter des tests
supplémentaires.

\input{patterns/12_FPU/3_comparison/x86/MSVC/olly_FR.tex}
}

\EN{\myparagraph{\Optimizing MSVC 2010}

\lstinputlisting[caption=\Optimizing MSVC 2010,style=customasmx86]{patterns/12_FPU/3_comparison/x86/MSVC_Ox/MSVC_EN.asm}

\myindex{x86!\Instructions!FCOM}

\FCOM differs from \FCOMP in the sense that it just compares the values and doesn't change the FPU stack. 
Unlike the previous example, here the operands are in reverse order, 
which is why the result of the comparison in \CThreeBits is different:

\begin{itemize}
\item If $a>b$ in our example, then \CThreeBits bits are to be set as: 0, 0, 0.
\item If $b>a$, then the bits are: 0, 0, 1.
\item If $a=b$, then the bits are: 1, 0, 0.
\end{itemize}
% TODO: table?

The \INS{test ah, 65} instruction leaves just two bits~---\Cthree and \Czero. 
Both will be zero if $a>b$: in that case the \JNE jump will not be triggered. 
Then \INS{FSTP ST(1)} follows~---this instruction copies the value from \ST{0} to the operand and 
pops one value from the FPU stack.
In other words, the instruction copies \ST{0} (where the value of \GTT{\_a} is now) into \ST{1}.
After that, two copies of {\_a} are at the top of the stack. 
Then, one value is popped.
After that, \ST{0} contains {\_a} and the function is finishes.

The conditional jump \JNE is triggering in two cases: if $b>a$ or $a=b$. 
\ST{0} is copied into \ST{0}, it is just like an idle (\ac{NOP}) operation, then one value 
is popped from the stack and the top of the stack (\ST{0}) is contain what has been in \ST{1} before 
(that is {\_b}). 
Then the function finishes. 
The reason this instruction is used here probably is because the \ac{FPU} 
has no other instruction to pop a value from the stack and discard it.

\input{patterns/12_FPU/3_comparison/x86/MSVC_Ox/olly_EN.tex}
}
\RU{\myparagraph{\Optimizing MSVC 2010}

\lstinputlisting[caption=\Optimizing MSVC 2010,style=customasmx86]{patterns/12_FPU/3_comparison/x86/MSVC_Ox/MSVC_RU.asm}

\myindex{x86!\Instructions!FCOM}
\FCOM отличается от \FCOMP тем, что просто сравнивает значения и оставляет стек в том же состоянии. 
В отличие от предыдущего примера, операнды здесь в обратном порядке. 
Поэтому и результат сравнения в \CThreeBits будет отличаться:

\begin{itemize}
\item Если $a>b$, то биты \CThreeBits должны быть выставлены так: 0, 0, 0.
\item Если $b>a$, то биты будут выставлены так: 0, 0, 1.
\item Если $a=b$, то биты будут выставлены так: 1, 0, 0.
\end{itemize}
% TODO: table?

Инструкция \INS{test ah, 65} как бы оставляет только два бита~--- \Cthree и \Czero. 
Они оба будут нулями, если $a>b$: в таком случае переход \JNE не сработает. 
\myindex{ARM!\Instructions!FSTP}
Далее имеется инструкция \INS{FSTP ST(1)}~--- эта инструкция копирует 
значение \ST{0} в указанный операнд и выдергивает одно значение из стека. В данном случае, 
она копирует \ST{0} 
(где сейчас лежит~\GTT{\_a})~в~\ST{1}. 
После этого на вершине стека два раза лежит~\GTT{\_a}. Затем одно значение выдергивается. 
После этого в \ST{0} остается~\GTT{\_a} и функция завершается.

Условный переход \JNE сработает в двух других случаях: если $b>a$ или $a=b$. 
\ST{0} скопируется в \ST{0} (как бы холостая операция). 
Затем одно значение из стека вылетит и на вершине стека останется то, что 
до этого лежало в \ST{1} (то~есть~\GTT{\_b}). И функция завершится. 
Эта инструкция используется здесь видимо потому что в FPU 
нет другой инструкции, которая просто выдергивает 
значение из стека и выбрасывает его.

\input{patterns/12_FPU/3_comparison/x86/MSVC_Ox/olly_RU.tex}
}
\DE{\myparagraph{\Optimizing MSVC 2010}

\lstinputlisting[caption=\Optimizing MSVC 2010,style=customasmx86]{patterns/12_FPU/3_comparison/x86/MSVC_Ox/MSVC_DE.asm}

\myindex{x86!\Instructions!FCOM}
\FCOM unterscheidet sich von \FCOMP in dem Sinne, dass es nur die Werte vergleicht ohne den FPU Stack zu verändern. 
Anders als im vorangehenden Beispiel liegen die Operanden hier in umgekehrter Reihenfolge um vor. Dies ist der Grund
warum das Ergebnis dieses Vergleichs bezüglich der \CThreeBits unterschiedlich ist:

\begin{itemize}
\item Falls $a>b$ in unserem Beispiel, dann werden die \CThreeBits Bits wie folgt gesetzt: 0, 0, 0.
\item Falls $b>a$, dann ist das Bitmuster: 0, 0, 1.
\item Falls $a=b$, dann ist das Bitmuster: 1, 0, 0.
\end{itemize}
% TODO: table?
Der Befehl \INS{test ah, 65} setzt zwei Bits~---\Cthree und \Czero. 
Beide werden auf 0 gesetzt, falls $a>b$: in diesem Fall wird der \JNE Sprungbefehl nicht ausgeführt.
Dann folgt \INS{FSTP ST(1)}~---dieser Befehl kopiert den Wert von \ST{0} in den Operanden und holt einen Wert vom FPU
Stack.
Mit anderen Worten, der Befehl kopiert \ST{0} (in dem sich gerade der Wert von \GTT{\_a} befindet) nach \ST{1}. 
Anschließend befinden sich zwei Kopien von {\_a} oben auf dem Stack.
Nun wird ein Wert wieder vom Stack geholt. Schließlich enthält \ST{0} den Wert {\_a} und die Funktion beendet sich.

Der bedingte Sprung \JNE wird in zwei Fällen ausgeführt: wenn $b>a$ oder wenn $a=b$.
\ST{0} wird nach \ST{0} kopiert; dabei handelt es sich um eine Operation ohne Wirkung (\ac{NOP}), dann wird ein Wert
 vom Stack geholt und in \ST{0} steht dann was sich vorher in \ST{1} befunden hat, nämlich {\_b}. Danach beendet sich
 die Funktion.
Der Grund dafür, dass dieser Befehl hier erzeugt wird ist wahrscheinlich, dass die \ac{FPU} über keinen anderen Befehl
verfügt um einen Wert vom Stack zu holen und anschließend zu entsorgen.

\input{patterns/12_FPU/3_comparison/x86/MSVC_Ox/olly_DE.tex}
}
\FR{\myparagraph{MSVC 2010 \Optimizing}

\lstinputlisting[caption=MSVC 2010 \Optimizing,style=customasmx86]{patterns/12_FPU/3_comparison/x86/MSVC_Ox/MSVC_FR.asm}

\myindex{x86!\Instructions!FCOM}

\FCOM diffère de \FCOMP dans le sens où il compare seulement les deux valeurs, et
ne change pas la pile du FPU.
Contrairement à l'exemple précédent, ici les opérandes sont dans l'ordre inverse,
c'est pourquoi le résultat de la comparaison dans \CThreeBits est différent.

\begin{itemize}
\item si $a>b$ dans notre exemple, alors les bits \CThreeBits sont mis comme suit: 0, 0, 0.
\item si $b>a$, alors les bits sont: 0, 0, 1.
\item si $a=b$, alors les bits sont: 1, 0, 0.
\end{itemize}
% TODO: table?

L'instruction \INS{test ah, 65} laisse seulement deux bits~---\Cthree et \Czero.
Les deux seront à zéro si $a>b$: dans ce cas le saut \JNE ne sera pas effectué.
Puis \INS{FSTP ST(1)} suit~---cette instruction copie la valeur de \ST{0} dans l'opérande
et supprime une valeur de la pile du FPU.
En d'autres mots, l'instruction copie \ST{0} (où la valeur de \GTT{\_a} se trouve)
dans \ST{1}.
Après cela, deux copies de {\_a} sont sur le sommet de la pile.
Puis, une valeur est supprimée.
Après cela, \ST{0} contient {\_a} et la fonction se termine.

Le saut conditionnel \JNE est effectué dans deux cas: si $b>a$ ou $a=b$.
\ST{0} est copié dans \ST{0}, c'est comme une opération sans effet (\ac{NOP}), puis
une valeur est supprimée de la pile et le sommet de la pile (\ST{0}) contient la
valeur qui était avant dans \ST{1} (qui est {\_b}).
Puis la fonction se termine.
La raison pour laquelle cette instruction est utilisée ici est sans doute que le
\ac{FPU} n'a pas d'autre instruction pour prendre une valeur sur la pile et la
supprimer.

\input{patterns/12_FPU/3_comparison/x86/MSVC_Ox/olly_FR.tex}
}

\EN{\myparagraph{GCC 4.4.1}

\lstinputlisting[caption=GCC 4.4.1,style=customasmx86]{patterns/12_FPU/3_comparison/x86/GCC_EN.asm}

\myindex{x86!\Instructions!FUCOMPP}

\FUCOMPP{} is almost like \FCOM, but pops both values from the stack and handles
\q{not-a-numbers} differently.

\myindex{Non-a-numbers (NaNs)}
A bit about \IT{not-a-numbers}.

\newcommand{\NANFN}{\footnote{\href{http://go.yurichev.com/17130}{wikipedia.org/wiki/NaN}}}

The FPU is able to deal with special values which are \IT{not-a-numbers} or \gls{NaN}s\NANFN. 
These are infinity, result of division by 0, etc.
Not-a-numbers can be \q{quiet} and \q{signaling}. It is possible to continue to work with \q{quiet} NaNs, 
but if one tries to do any operation with \q{signaling} NaNs, an exception is to be raised.

\myindex{x86!\Instructions!FCOM}
\myindex{x86!\Instructions!FUCOM}

\FCOM raising an exception if any operand is \gls{NaN}. 
\FUCOM raising an exception only if any operand is a signaling \gls{NaN} (SNaN).

\myindex{x86!\Instructions!SAHF}
\label{SAHF}

The next instruction is \SAHF (\IT{Store AH into Flags})~---this is a rare 
instruction in code not related to the FPU. 
8 bits from AH are moved into the lower 8 bits of the CPU flags in the following order:

\input{SAHF_LAHF}

\myindex{x86!\Instructions!FNSTSW}

Let's recall that \FNSTSW moves the bits that interest us (\CThreeBits) into \AH 
and they are in positions 6, 2, 0 of the \AH register:

\input{C3_in_AH}

In other words, the \INS{fnstsw  ax / sahf} instruction pair moves \CThreeBits into \ZF, \PF and \CF.

Now let's also recall the values of \CThreeBits in different conditions:

\begin{itemize}
\item If $a$ is greater than $b$ in our example, then \CThreeBits are to be set to: 0, 0, 0.
\item if $a$ is less than $b$, then the bits are to be set to: 0, 0, 1.
\item If $a=b$, then: 1, 0, 0.
\end{itemize}
% TODO: table?

In other words, these states of the CPU flags are possible
after three \\
\FUCOMPP/\FNSTSW/\SAHF instructions:

\begin{itemize}
\item If $a>b$, the CPU flags are to be set as: \GTT{ZF=0, PF=0, CF=0}.
\item If $a<b$, then the flags are to be set as: \GTT{ZF=0, PF=0, CF=1}.
\item And if $a=b$, then: \GTT{ZF=1, PF=0, CF=0}.
\end{itemize}
% TODO: table?

\myindex{x86!\Instructions!SETcc}
\myindex{x86!\Instructions!JNBE}

Depending on the CPU flags and conditions, \SETNBE stores 1 or 0 to AL. 
It is almost the counterpart of \JNBE, with the exception that \SETcc 
\footnote{\IT{cc} is \IT{condition code}} stores 1 or 0 in \AL, 
but \Jcc does actually jump or not. 
\SETNBE stores 1 only if \GTT{CF=0} and \GTT{ZF=0}. 
If it is not true, 0 is to be stored into \AL.

Only in one case both \CF and \ZF are 0: if $a>b$.

Then 1 is to be stored to \AL, the subsequent \JZ is not to be triggered and the function will return {\_a}. 
In all other cases, {\_b} is to be returned.

}
\RU{\myparagraph{GCC 4.4.1}

\lstinputlisting[caption=GCC 4.4.1,style=customasm]{patterns/12_FPU/3_comparison/x86/GCC_RU.asm}

\myindex{x86!\Instructions!FUCOMPP}
\FUCOMPP~--- это почти то же что и \FCOM, только выкидывает из стека оба значения после сравнения, 
а также несколько иначе реагирует на \q{не-числа}.

\myindex{Не-числа (NaNs)}
Немного о \IT{не-числах}.

\newcommand{\NANFN}{\footnote{\href{http://go.yurichev.com/17129}{ru.wikipedia.org/wiki/NaN}}}

FPU умеет работать со специальными переменными, которые числами не являются и называются \q{не числа} или 
\gls{NaN}\NANFN. 
Это бесконечность, результат деления на ноль, и так далее. Нечисла бывают \q{тихие} и \q{сигнализирующие}. 
С первыми можно продолжать работать и далее, а вот если вы попытаетесь совершить какую-то операцию 
с сигнализирующим нечислом, то сработает исключение.

\myindex{x86!\Instructions!FCOM}
\myindex{x86!\Instructions!FUCOM}
Так вот, \FCOM вызовет исключение если любой из операндов какое-либо нечисло.
\FUCOM же вызовет исключение только если один из операндов именно \q{сигнализирующее нечисло}.

\myindex{x86!\Instructions!SAHF}
\label{SAHF}
Далее мы видим \SAHF (\IT{Store AH into Flags})~--- это довольно редкая инструкция в коде, не использующим FPU. 
8 бит из \AH перекладываются в младшие 8 бит регистра статуса процессора в таком порядке:

\input{SAHF_LAHF}

\myindex{x86!\Instructions!FNSTSW}
Вспомним, что \FNSTSW перегружает интересующие нас биты \CThreeBits в \AH, 
и соответственно они будут в позициях 6, 2, 0 в регистре \AH:

\input{C3_in_AH}

Иными словами, пара инструкций \INS{fnstsw  ax / sahf} перекладывает биты \CThreeBits в флаги \ZF, \PF, \CF.

Теперь снова вспомним, какие значения бит \CThreeBits будут при каких результатах сравнения:

\begin{itemize}
\item Если $a$ больше $b$ в нашем случае, то биты \CThreeBits должны быть выставлены так: 0, 0, 0.
\item Если $a$ меньше $b$, то биты будут выставлены так: 0, 0, 1.
\item Если $a=b$, то так: 1, 0, 0.
\end{itemize}
% TODO: table?

Иными словами, после трех инструкций \FUCOMPP/\FNSTSW/\SAHF возможны такие состояния флагов:

\begin{itemize}
\item Если $a>b$ в нашем случае, то флаги будут выставлены так: \GTT{ZF=0, PF=0, CF=0}.
\item Если $a<b$, то флаги будут выставлены так: \GTT{ZF=0, PF=0, CF=1}.
\item Если $a=b$, то так: \GTT{ZF=1, PF=0, CF=0}.
\end{itemize}
% TODO: table?

\myindex{x86!\Instructions!SETcc}
\myindex{x86!\Instructions!JNBE}
Инструкция \SETNBE выставит в \AL единицу или ноль в зависимости от флагов и условий. 
Это почти аналог \JNBE, за тем лишь исключением, что \SETcc
\footnote{\IT{cc} это \IT{condition code}}
выставляет 1 или 0 в \AL, а \Jcc делает переход или нет. 
\SETNBE запишет 1 только если \GTT{CF=0} и \GTT{ZF=0}. Если это не так, то запишет 0 в \AL.

\CF будет 0 и \ZF будет 0 одновременно только в одном случае: если $a>b$.

Тогда в \AL будет записана 1, последующий условный переход \JZ выполнен не будет 
и функция вернет~\GTT{\_a}. 
В остальных случаях, функция вернет~\GTT{\_b}.
}
\DE{\myparagraph{GCC 4.4.1}

\lstinputlisting[caption=GCC
4.4.1,style=customasmx86]{patterns/12_FPU/3_comparison/x86/GCC_DE.asm}

\myindex{x86!\Instructions!FUCOMPP}
\FUCOMPP{} ist fast wie like \FCOM, nimmt aber beide Werte vom Stand und
behandelt \q{undefinierte Zahlenwerte} anders.


\myindex{Non-a-numbers (NaNs)}
Ein wenig über \IT{undefinierte Zahlenwerte}.

\newcommand{\NANFN}{\footnote{\href{http://go.yurichev.com/17130}{wikipedia.org/wiki/NaN}}}
Die FPU ist in der Lage mit speziellen undefinieten Werten, den sogenannten
\IT{not-a-number}(kurz \gls{NaN})\NANFN umzugehen. Beispiele sind etwa der Wert
unendlich, das Ergebnis einer Division durch 0, etc. Undefinierte Werte können
entwder \q{quiet} oder \q{signaling} sein. Es ist möglich mit \q{quiet} NaNs zu
arbeiten, aber beim Versuch einen Befehl auf \q{signaling} NaNs auszuführen,
wird eine Exception geworfen. 

\myindex{x86!\Instructions!FCOM}
\myindex{x86!\Instructions!FUCOM}
\FCOM erzeugt eine Exception, falls irgendein Operand ein \gls{NaN} ist.
\FUCOM erzeugt eine Exception nur dann, wenn ein Operand eine \q{signaling}
\gls{NaN} (SNaN) ist.

\myindex{x86!\Instructions!SAHF}
\label{SAHF}
Der nächste Befehl ist \SAHF (\IT{Store AH into Flags})~---es handelt sich
hierbei um einen seltenen Befehl, der nicht mit der FPU zusammenhängt.
8 Bits aus AH werden in die niederen 8 Bit der CPU Flags in der folgenden
Reihenfolge verschoben:

\input{SAHF_LAHF}

\myindex{x86!\Instructions!FNSTSW}
Erinnern wir uns, dass \FNSTSW die für uns interessanten Bits (\CThreeBits) auf
den Stellen 6,2,0 im AH Register setzt:

\input{C3_in_AH}
Mit anderen Worten: der Befehl \INS{fnstsw ax / sahf} verschiebt \CThreeBits
nach \ZF, \PF und \CF. 

Überlegen wir uns auch die Werte der \CThreeBits in unterschiedlichen Szenarien:

\begin{itemize} 
  \item Falls in unserem Beispiel $a$ größer als $b$ ist, dann werden die
  \CThreeBits auf 0,0,0 gesetzt.
  \item Falls $a$ kleiner als $b$ ist, werden die Bits auf 0,0,1 gesetzt.
  \item Falls $a=b$, dann werden die Bits auf 1,0,0 gesetzt.
\end{itemize}
% TODO: table?
Mit anderen Worten, die folgenden Zustände der CPU Flags sind nach drei
\FUCMPP/\FNSTSW/\SAHF Befehlen möglich:

\begin{itemize}
\item Falls $a>b$, werden die CPU Flags wie folgt gesetzt \GTT{ZF=0, PF=0,
CF=0}.
\item Falls $a<b$, werden die CPU Flags wie folgt gesetzt: \GTT{ZF=0, PF=0,
CF=1}.
\item Und falls $a=b$, dann gilt: \GTT{ZF=1, PF=0, CF=0}.
\end{itemize}
% TODO: table?

\myindex{x86!\Instructions!SETcc}
\myindex{x86!\Instructions!JNBE}
Abhängig von den CPU Flags und Bedingungen, speichert \SETNBE entweder 1 oder 0
in AL.
Es ist also quasi das Gegenstück von \JNBE mit dem Unterschied, dass \SETcc

Depending on the CPU flags and conditions, \SETNBE stores 1 or 0 to AL. 
It is almost the counterpart of \JNBE, with the exception that \SETcc
\footnote{\IT{cc} is \IT{condition code}} eine 1 oder 0 in \AL speichert, aber
\Jcc tatsächlich auch springt.
\SETNBE speicher 1 nur, falls \GTT{CF=0} und \GTT{ZF=0}.
Wenn dies nicht der Fall ist, dann wird 0 in \AL gespeichert.

Nur in einem Fall sind \CF und \ZF beide 0: falls $a>b$.

In diesem Fall wird 1 in \AL gespeichert, der nachfolgende \JZ Sprung wird nicht
ausgeführt und die Funktion liefert {\_a} zurück. In allen anderen Fällen wird
{\_b} zurückgegeben.}
\FR{\myparagraph{GCC 4.4.1}

\lstinputlisting[caption=GCC 4.4.1,style=customasmx86]{patterns/12_FPU/3_comparison/x86/GCC_FR.asm}

\myindex{x86!\Instructions!FUCOMPP}

\FUCOMPP{} est presque comme \FCOM, mais dépile deux valeurs de la pile et traite
les \q{non-nombres} différemment.

\myindex{Non-a-numbers (NaNs)}
Quelques informations à propos des \IT{not-a-numbers} (non-nombres).

\newcommand{\NANFN}{\footnote{\href{http://go.yurichev.com/17130}{wikipedia.org/wiki/NaN}}}

Le FPU est capable de traiter les valeurs spéciales que sont les \IT{not-a-numbers}
(non-nombres) ou \gls{NaN}s\NANFN.
Ce sont les infinis, les résultat de division par 0, etc.
Les non-nombres peuvent être \q{quiet} et \q{signaling}. Il est possible de continuer
à travailler ave les \q{quiet} NaNs, mais si l'on essaye de faire une opération avec
un \q{signaling} NaNs, une exception est levée.

\myindex{x86!\Instructions!FCOM}
\myindex{x86!\Instructions!FUCOM}

\FCOM lève une exception si un des opérandes est \gls{NaN}.
\FUCOM lève une exception seulement si un des opérandes est un signaling \gls{NaN}
(SNaN).

\myindex{x86!\Instructions!SAHF}
\label{SAHF}

L'instruction suivante est \SAHF (\IT{Store AH into Flags} stocker AH dans les Flags)~---est
une instruction rare dans le code non relatif au FPU.
8 bits de AH sont copiés dans les 8-bits bas dans les flags du CPU dans l'ordre suivant:

\input{SAHF_LAHF}

\myindex{x86!\Instructions!FNSTSW}

Rappelons que \FNSTSW déplace des bits qui nous intéressent (\CThreeBits) dans \AH
et qu'ils sont aux positions 6, 2, 0 du registre \AH.

\input{C3_in_AH}

En d'autres mots, la paire d'instructions \INS{fnstsw  ax / sahf} déplace \CThreeBits
dans \ZF, \PF et \CF.

Maintenant, rappelons les valeurs de \CThreeBits sous différentes conditions:

\begin{itemize}
\item Si $a$ est plus grand que $b$ dans notre exemple, alors les \CThreeBits sont
mis à: 0, 0, 0.
\item Si $a$ est plus petit que $b$, alors les bits sont mis à: 0, 0, 1.
\item Si $a=b$, alors: 1, 0, 0.
\end{itemize}
% TODO: table?

En d'autres mots, ces états des flags du CPU sont possible après les
trois instructions \FUCOMPP/\FNSTSW/\SAHF:

\begin{itemize}
\item Si $a>b$, les flags du CPU sont mis à: \GTT{ZF=0, PF=0, CF=0}.
\item Si $a<b$, alors les flags sont mis à: \GTT{ZF=0, PF=0, CF=1}.
\item Et si $a=b$, alors: \GTT{ZF=1, PF=0, CF=0}.
\end{itemize}
% TODO: table?

\myindex{x86!\Instructions!SETcc}
\myindex{x86!\Instructions!JNBE}

Suivant les flags du CPU et les conditions, \SETNBE met 1 ou 0 dans AL.
C'est presque la contrepartie de \JNBE, avec l'exception que \SETcc\footnote{\IT{cc}
est un \IT{condition code}} met 1 ou 0 dans \AL, mais \Jcc effectue un saut ou non.
\SETNBE met 1 seulement si \GTT{CF=0} et \GTT{ZF=0}.
Si ce n'est pas vrai, 0 est mis dans \AL.

Il y a un seul cas où \CF et \ZF sont à 0: si $a>b$.

Alors 1 est mis dans \AL, le \JZ subséquent n'est pas pris et la fonction va renvoyer
{\_a}.
Dans tous les autres cas, {\_b} est renvoyé.

}

\EN{\myparagraph{\Optimizing GCC 4.4.1}

\lstinputlisting[caption=\Optimizing GCC 4.4.1,style=customasm]{patterns/12_FPU/3_comparison/x86/GCC_O3_EN.asm}

\myindex{x86!\Instructions!JA}

It is almost the same except that \JA is used after \SAHF. 
Actually, conditional jump instructions that check \q{larger}, \q{lesser} or \q{equal} for unsigned number comparison 
(these are \JA, \JAE, \JB, \JBE, \JE/\JZ, \JNA, \JNAE, \JNB, \JNBE, \JNE/\JNZ) check only flags \CF and \ZF.\\
\\
Let's recall where bits \CThreeBits are located in the \GTT{AH} register after the execution of \INS{FSTSW}/\FNSTSW:

\input{C3_in_AH}

Let's also recall, how the bits from \GTT{AH} are stored into the CPU flags the execution of \SAHF:

\input{SAHF_LAHF}

After the comparison, the \Cthree and \Czero bits are moved into \ZF and \CF, so the conditional jumps are able work after. \JA is triggering if both \CF are \ZF zero.

Thereby, the conditional jumps instructions listed here can be used after a \FNSTSW/\SAHF instruction pair.

Apparently, the FPU \CThreeBits status bits were placed there intentionally, to easily map them to base CPU flags without additional permutations?

}
\RU{\myparagraph{\Optimizing GCC 4.4.1}

\lstinputlisting[caption=\Optimizing GCC 4.4.1,style=customasmx86]{patterns/12_FPU/3_comparison/x86/GCC_O3_RU.asm}

\myindex{x86!\Instructions!JA}

Почти всё что здесь есть, уже описано мною, кроме одного: использование \JA после \SAHF. 
Действительно, инструкции условных переходов \q{больше}, \q{меньше} и \q{равно} для сравнения беззнаковых чисел 
(а это \JA, \JAE, \JB, \JBE, \JE/\JZ, \JNA, \JNAE, \JNB, \JNBE, \JNE/\JNZ) проверяют только флаги \CF и \ZF.\\
\\
Вспомним, как биты \CThreeBits располагаются в регистре \GTT{AH} после исполнения \INS{FSTSW}/\FNSTSW:

\input{C3_in_AH}

Вспомним также, как располагаются биты из \GTT{AH} во флагах CPU после исполнения \SAHF:

\input{SAHF_LAHF}

Биты \Cthree и \Czero после сравнения перекладываются в флаги \ZF и \CF так, что перечисленные инструкции переходов могут работать. \JA сработает, если \CF и \ZF обнулены.

Таким образом, перечисленные инструкции условного перехода можно использовать после инструкций \FNSTSW/\SAHF.

Может быть, биты статуса FPU \CThreeBits преднамеренно были размещены таким образом, чтобы переноситься на базовые флаги процессора без перестановок?

}
\DE{\myparagraph{\Optimizing GCC 4.4.1}

\lstinputlisting[caption=\Optimizing GCC
4.4.1,style=customasmx86]{patterns/12_FPU/3_comparison/x86/GCC_O3_DE.asm}

\myindex{x86!\Instructions!JA}
Dies ist fast das gleiche, außer dass \JA nach \SAHF verwendet wird.
Tatsächlich prüfen die bedingte Sprungbefehle, die vorzeichenlose Zahlen auf
\q{größer}, \q{kleiner} oder \q{gleich} prüfen (das sind \JA, \JAE, \JB, \JBE,
\JE/\JZ, \JNA, \JNAE, \JNB, \JNBE, \JNE/\JNZ) lediglich die Flags \CF und
\ZF.\\\\
Erinnern wir uns, an welchen Stellen die \CThreeBits sich im \GTT{AH} Register
befinden, nachdem der Befehl \INS{FSTSW}/\FNSTSW ausgeführt wurde:

\input{C3_in_AH}
Halten wir uns auch vor Augen wie die Bits aus \GTT{AH} in den CPU Flags nach
der Ausführung von \SAHF abgelegt werden:

\input{SAHF_LAHF}
Nach dem Vergleich werden die \Cthree und \Czero Bits nach \ZF und \CF
verschoben, sodass der bedingte Sprung danach funktionieren kann. \JA wird
ausgeüführt, falls sowohl \CF als auch \ZF gleich 0 sind.

Hierbei können alle hier aufgelisteten Sprungbefehle nach einem \FNSTSW/\SAHF
Befehlspaar verwendet werden. 

Offenbar wurden die \CThreeBits Status Bits der CPU dort bewusst platziert,
sodass diese leicht auf die CPU Flags übertragen werden können, ohne dass
zusätzliche Vertauschungen notwendig sind.}
\FR{\myparagraph{GCC 4.4.1 \Optimizing}

\lstinputlisting[caption=GCC 4.4.1 \Optimizing,style=customasmx86]{patterns/12_FPU/3_comparison/x86/GCC_O3_FR.asm}

\myindex{x86!\Instructions!JA}

C'est presque le même. à l'exception que \JA est utilisé après \SAHF.
En fait, les instructions de sauts conditionnels qui vérifient \q{plus}, \q{moins} ou \q{égal} pour
les comparaisons de nombres non signés (ce sont \JA, \JAE, \JB, \JBE, \JE/\JZ, \JNA,
\JNAE, \JNB, \JNBE, \JNE/\JNZ) vérifient seulement les flags \CF et \ZF.\\
\\
Rappelons comment les bits \CThreeBits sont situés dans le registre \GTT{AH} après
l'exécution de \INS{FSTSW}/\FNSTSW:

\input{C3_in_AH}

Rappelons également, comment les bits de \GTT{AH} sont stockés dans les flags du
CPU après l'exécution de \SAHF:

\input{SAHF_LAHF}

Après la comparaison, les bits \Cthree et \Czero sont copiés dans \ZF et \CF, donc
les sauts conditionnels peuvent fonctionner après. \JA est déclenché si \CF et \ZF
sont tout les deux à zéro.

Ainsi, les instructions de saut conditionnel listées ici peuvent être utilisées après
une paire d'instructions \FNSTSW/\SAHF.

Apparemment, les bits de status du FPU \CThreeBits ont été mis ici intentionnellement,
pour facilement les relier aux flags du CPU de base sans permutations supplémentaires?

}

\EN{\myparagraph{GCC 4.8.1 with \Othree optimization turned on}
\label{gcc481_o3}

Some new FPU instructions were added in the P6 Intel family\footnote{Starting at Pentium Pro, Pentium-II, etc.}.
\myindex{x86!\Instructions!FUCOMI}
These are \INS{FUCOMI} (compare operands and set flags of the main CPU) and 
\myindex{x86!\Instructions!FCMOVcc}
\INS{FCMOVcc} (works like \INS{CMOVcc}, but on FPU registers).

Apparently, the maintainers of GCC decided to drop support of pre-P6 Intel CPUs (early Pentiums, 80486, etc.).

And also, the FPU is no longer separate unit in P6 Intel family, so now it is possible to modify/check flags of the main CPU from the FPU.

So what we get is:

\lstinputlisting[caption=\Optimizing GCC 4.8.1,style=customasmx86]{patterns/12_FPU/3_comparison/x86/GCC481_O3_EN.s}

Hard to guess why \INS{FXCH} (swap operands) is here.

It's possible to get rid of it easily by swapping the first two \FLD instructions or by replacing 
\INS{FCMOVBE} (\IT{below or equal}) by \INS{FCMOVA} (\IT{above}).
Probably it's a compiler inaccuracy.

So \INS{FUCOMI} compares \ST{0} ($a$) and \ST{1} ($b$) 
and then sets some flags in the main CPU.
\INS{FCMOVBE} checks the flags and copies \ST{1} 
($b$ here at the moment) to 
\ST{0} ($a$ here) if $ST0 (a) <= ST1 (b)$.
Otherwise ($a>b$), it leaves $a$ in \ST{0}.

The last \FSTP leaves \ST{0} on top of the stack, discarding the contents of \ST{1}.

Let's trace this function in GDB:

\lstinputlisting[caption=\Optimizing GCC 4.8.1 and GDB,numbers=left]{patterns/12_FPU/3_comparison/x86/gdb.txt}

Using \q{ni}, 
let's execute the first two \FLD instructions.

Let's examine the FPU registers (line 33).

As it was mentioned before, the FPU registers set is a circular buffer rather than a stack (\myref{FPU_is_rather_circular_buffer}).
And GDB doesn't show \GTT{STx} registers, but internal the FPU registers (\GTT{Rx}). 
The arrow (at line 35) points to the current top of the stack.

You can also see the \GTT{TOP} register contents in \IT{Status Word} (line 44)---it is 6 now, 
so the stack top is now pointing to internal register 6.

The values of $a$ and $b$ are swapped after \INS{FXCH} is executed (line 54).

\INS{FUCOMI} is executed (line 83). 
Let's see the flags: \CF is set (line 95).

\INS{FCMOVBE} has copied the value of $b$ (see line 104).

\FSTP leaves one value at the top of stack (line 136). 
The value of \GTT{TOP} is now 7, so the FPU stack top is pointing to internal register 7.

}
\RU{\myparagraph{GCC 4.8.1 с оптимизацией \Othree}
\label{gcc481_o3}

В линейке процессоров P6 от Intel 
появились новые FPU-инструкции\footnote{Начиная с Pentium Pro, Pentium-II, итд.}.
\myindex{x86!\Instructions!FUCOMI}
Это \INS{FUCOMI} (сравнить операнды и выставить флаги основного CPU) и
\myindex{x86!\Instructions!FCMOVcc}
\INS{FCMOVcc} (работает как \INS{CMOVcc}, но на регистрах FPU).
Очевидно, разработчики GCC решили отказаться от поддержки процессоров до линейки P6 (ранние Pentium, 80486, итд.).

И кстати, FPU уже давно не отдельная часть процессора в линейке P6, так что флаги основного CPU можно модифицировать из FPU.

Вот что имеем:

\lstinputlisting[caption=\Optimizing GCC 4.8.1,style=customasmx86]{patterns/12_FPU/3_comparison/x86/GCC481_O3_RU.s}

Не совсем понимаю, зачем здесь \INS{FXCH} (поменять местами операнды).

От нее легко избавиться поменяв местами инструкции \FLD либо заменив 
\INS{FCMOVBE} (\IT{below or equal}~--- меньше или равно) на 
\INS{FCMOVA} (\IT{above}~--- больше).

Должно быть, неаккуратность компилятора.

Так что \INS{FUCOMI} сравнивает \ST{0} ($a$) и \ST{1} ($b$) 
и затем устанавливает флаги основного CPU.
\INS{FCMOVBE} проверяет флаги и копирует \ST{1} 
(в тот момент там находится $b$) в 
\ST{0} (там $a$) если $ST0 (a) <= ST1 (b)$.
В противном случае ($a>b$), она оставляет $a$ в \ST{0}.

Последняя \FSTP оставляет содержимое \ST{0} на вершине стека, выбрасывая содержимое \ST{1}.

Попробуем оттрасировать функцию в GDB:

\lstinputlisting[caption=\Optimizing GCC 4.8.1 and GDB,numbers=left]{patterns/12_FPU/3_comparison/x86/gdb.txt}

Используя \q{ni}, дадим первым двум инструкциям \FLD исполниться.

Посмотрим регистры FPU (строка 33).

Как уже было указано ранее, регистры FPU это скорее кольцевой буфер, нежели стек (\myref{FPU_is_rather_circular_buffer}).
И GDB показывает не регистры \GTT{STx}, а внутренние регистры FPU (\GTT{Rx}). 
Стрелка (на строке 35) указывает на текущую вершину стека.

Вы можете также увидеть содержимое регистра \GTT{TOP} в \q{Status Word} (строка 44). Там сейчас 6, так что
вершина стека сейчас указывает на внутренний регистр 6.

Значения $a$ и $b$ меняются местами после исполнения \INS{FXCH} (строка 54).

\INS{FUCOMI} исполнилась (строка 83).
Посмотрим флаги: \CF выставлен (строка 95).

\INS{FCMOVBE} действительно скопировал значение $b$ (см. строку 104).

\FSTP оставляет одно значение на вершине стека (строка 136). 
Значение \GTT{TOP} теперь 7, так что вершина FPU-стека указывает на внутренний регистр 7.
}
\DE{\myparagraph{GCC 4.8.1 mit aktivierter \Othree Optimierung}
\label{gcc481_o3}
Mit der P6 Intel Familie\footnote{Beginnend mit Pentium Pro, Pentium-II, etc.}
wurden einige neue FPU Befehle hinzugefügt. 
\myindex{x86!\Instructions!FUCOMI}
Diese sind \INS{FUCOMI} (vergleiche Operanden und setze Flags der CPU) und 
\myindex{x86!\Instructions!FCMOVcc}
\INS{FCMOVcc} (arbeitet wie \INS{CMOVcc}, aber auf FPU Registern).
Offenbar haben sich die Verwalter von GCC dazu entschieden, den Support von
vor-P6 Intel CPUs (frühe Pentiums, 80486, etc.) einzustellen.

Außerdem ist die FPU nicht länger eine separate Einheit in der P6 Intel Familie,
sodass es nun auch möglich ist, die Flags der CPU von der FPU aus zu prüfen oder
zu verändern.

Wir erhalten also das Folgende:

\lstinputlisting[caption=\Optimizing GCC
4.8.1,style=customasmx86]{patterns/12_FPU/3_comparison/x86/GCC481_O3_DE.s}

Schwer zu sagen, warum \INS{FXCH} (vertausche Operanden) hier verwendet wird.

Es ist möglich, diesen Befehl loszuwerden, indem man die ersten beiden \FLD
Befehle vertauscht oder \INS{FCMOVBE} (\IT{below or equal}) durch \INS{FCMOVA}
(\IT{above}) ersetzt.
Wahrscheinlich handelt es sich hierbei um eine Ungenauigkeit im Compiler.

\INS{FUCOMI} vergleicht also \ST{0} ($a$) und \ST{1} ($b$) und setzt einige
Flags in der CPU. 
\INS{FCMOVBE} prüft die Flags und kopiert \ST{1} (in diesem Moment also $b$)
nach \ST{0} (hier: $a$), falls $ST0 (a) <= ST1 (b)$.
Andernfalls ($a>b$) wird $a$ in \ST{0} belassen.

Der letzte \FSTP Befehl belässt \ST{0} oben auf dem Stack und verwirft den
Inhalt von \ST{1}. 

Verfolgen wir den Funktionsverlauf in GDB:

\lstinputlisting[caption=\Optimizing GCC 4.8.1 and GDB,numbers=left]{patterns/12_FPU/3_comparison/x86/gdb.txt}

Unter Verwendung von \q{ni} führen wir die ersten beiden \FLD Befehle aus.

Sehen wir uns die FPU Register (Zeile 33) an.

Wie bereits erwähnt, bildet der FPU Registersatz einen Ringpuffer und keinen
Stack (\myref{FPU_is_rather_circular_buffer}).
Außerdem zeigt GDB nicht die \GTT{STx} Register, sondern die internen FPU
Register (\GTT{Rx}). 
Der Pfeil (in Zeile 35) zeigt auf das aktuell obere Ende des Stacks.

Wir sehen auch den Inhalt des \GTT{TOP} Registers in \IT{Status Word} (Zeile
44)--hier ist dieser 6, sodass das oberste Element im Stack also aktuell auf das
interne Register 6 zeigt.

Die Werte von $a$ und $b$ werden nach Ausführung von \INS{FXCH} (Zeile 54)
vertauscht.

\INS{FUCOMI} wird ausgeführt (Zeile 83).
Betrachten wir die Flags: \CF ist gesetzt (Zeile 95).

\INS{FCMOVBE} hat den Wert von $b$ kopiert (siehe Zeile 104).

\FSTP lässt einen Wert oben auf dem Stack (Zeile 136).
Der Wert von \GTT{TOP} beträgt jetzt 7, was bedeutet, dass das obere Ende des
FPU Stacks jetzt auf das interne Register 7 zeigt.
}
\FR{\myparagraph{GCC 4.8.1 avec l'option d'optimisation \Othree}
\label{gcc481_o3}

De nouvelles instructions FPU ont été ajoutées avec la famille Intel P6\footnote{À partir du Pentium Pro, Pentium-II, etc.}.
\myindex{x86!\Instructions!FUCOMI}
Ce sont \INS{FUCOMI} (comparer les opérandes et positionner les flags du CPU principal)
et \myindex{x86!\Instructions!FCMOVcc}
\INS{FCMOVcc} (fonctionne comme \INS{CMOVcc}, mais avec les registres du FPU).

Apparement, les mainteneurs de GCC ont décidé de supprimer le support des CPUs Intel
pré-P6 (premier Pentium, 80486, etc.).

Et donc, le FPU n'est plus une unité séparée dans la famille Intel P6, ainsi il est
possible de modifier/vérifier un flag du CPU principal depuis le FPU.

Voici ce que nous obtenons:

\lstinputlisting[caption=GCC 4.8.1 \Optimizing,style=customasmx86]{patterns/12_FPU/3_comparison/x86/GCC481_O3_FR.s}

Difficile de deviner pourquoi \INS{FXCH} (échange les opérandes) est ici.

Il est possible de s'en débarrasser facilement en échangeant les deux premières instructions
\FLD ou en remplaçant \INS{FCMOVBE} (\IT{below or equal} inférieur ou égal) par
\INS{FCMOVA} (\IT{above}).
Il s'agit probablement d'une imprécision du compilateur.

Donc \INS{FUCOMI} compare \ST{0} ($a$) et \ST{1} ($b$) et met certains flags
dans le CPU principal.
\INS{FCMOVBE} vérifie les flags et copie \ST{1} ($b$ ici à ce moment) dans \ST{0}
($a$ ici) si $ST0 (a) <= ST1 (b)$.
Autrement ($a>b$), $a$ est laissé dans \ST{0}.

Le dernier \FSTP laisse \ST{0} sur le sommet de la pile, supprimant le contenu de \ST{1}.

Exécutons pas à pas cette fonction dans GDB:

\lstinputlisting[caption=GCC 4.8.1 \Optimizing and GDB,numbers=left]{patterns/12_FPU/3_comparison/x86/gdb.txt}

En utilisant \q{ni}, exécutons les deux premières instructions \FLD.

Examinons les registres du FPU (ligne 33).

Comme cela a déjà été mentionné, l'ensemble des registres FPU est un buffeur
circulaire plutôt qu'une pile (\myref{FPU_is_rather_circular_buffer}).
Et GDB ne montre pas les registres \GTT{STx}, mais les registre internes du FPU (\GTT{Rx}).
La flêche (à la ligne 35) pointe sur le haut courant de la pile.

Vous pouvez voir le contenu du registre \GTT{TOP} dans le \IT{Status Word} (ligne 44)---c'est
6 maintenant, donc le haut de la pile pointe maintenant sur le registre interne 6.

Les valeurs de $a$ et $b$ sont échangées après l'exécution de \INS{FXCH} (ligne 54).

\INS{FUCOMI} est exécuté (ilgne 83).
Regardons les flags: \CF est mis (ligne 95).

\INS{FCMOVBE} a copié la valeur de $b$ (voir ligne 104).

\FSTP dépose une valeur au sommet de la pile (ligne 136).
La valeur de \GTT{TOP} est maintenant 7, donc le sommet de la pile du FPU pointe
sur le registre interne 7.

}


\EN{\subsubsection{ARM}

\myparagraph{\OptimizingXcodeIV (\ARMMode)}

\lstinputlisting[caption=\OptimizingXcodeIV (\ARMMode),style=customasmARM]{patterns/12_FPU/3_comparison/ARM/Xcode_ARM_EN.lst}

\myindex{ARM!\Registers!APSR}
\myindex{ARM!\Registers!FPSCR}
A very simple case.
The input values are placed into the \GTT{D17} and \GTT{D16} registers and then compared using the \INS{VCMPE} instruction.

Just like in the x86 coprocessor, the ARM coprocessor has its own status and flags register (\ac{FPSCR}),
since there is a necessity to store coprocessor-specific flags.
% TODO -> расписать регистр по битам
\myindex{ARM!\Instructions!VMRS}
And just like in x86, there are no conditional jump instruction in ARM, 
that can check bits in the status register of the coprocessor. 
So there is \INS{VMRS}, which copies 4 bits (N, Z, C, V) from the coprocessor status word into bits of the \IT{general} status register (\ac{APSR}).

\myindex{ARM!\Instructions!VMOVGT}
\INS{VMOVGT} is the analog of the \INS{MOVGT}, 
instruction for D-registers, it executes if one operand is greater than the other while comparing (\IT{GT---Greater Than}). 

If it gets executed, the value of $a$ is to be written into \GTT{D16} (that is currently stored in in \GTT{D17}).
Otherwise the value of $b$ stays in the \GTT{D16} register.

\myindex{ARM!\Instructions!VMOV}

The penultimate instruction \INS{VMOV} prepares the value in the \GTT{D16} register for returning it via the \Reg{0} and \Reg{1}
register pair.

\myparagraph{\OptimizingXcodeIV (\ThumbTwoMode)}

\begin{lstlisting}[caption=\OptimizingXcodeIV (\ThumbTwoMode),style=customasmARM]
VMOV            D16, R2, R3 ; b
VMOV            D17, R0, R1 ; a
VCMPE.F64       D17, D16
VMRS            APSR_nzcv, FPSCR
IT GT 
VMOVGT.F64      D16, D17
VMOV            R0, R1, D16
BX              LR
\end{lstlisting}

Almost the same as in the previous example, however slightly different.
As we already know, many instructions in ARM mode can be supplemented by condition predicate.
But there is no such thing in Thumb mode. 
There is no space in the 16-bit instructions for 4 more bits in which conditions can be encoded.

\myindex{ARM!\ThumbTwoMode}

However, Thumb-2 was extended to make it possible to specify predicates to old Thumb instructions.
Here, in the \IDA-generated listing, we see the \INS{VMOVGT} instruction, as in previous example.

In fact, the usual \INS{VMOV} is encoded there, but \IDA adds the \GTT{-GT} suffix to it, 
since there is a \INS{IT GT} instruction placed right before it.

\label{ARM_Thumb_IT}
\myindex{ARM!\Instructions!IT}
\myindex{ARM!if-then block}
The \INS{IT} instruction defines a so-called \IT{if-then block}. 

After the instruction it is possible to place up to 4 instructions, 
each of them has a predicate suffix.
In our example, \INS{IT GT} implies that the next instruction is to be executed, if the \IT{GT} (\IT{Greater Than}) condition is true.

\myindex{Angry Birds}
Here is a more complex code fragment, by the way, from Angry Birds (for iOS):

\begin{lstlisting}[caption=Angry Birds Classic,style=customasmARM]
...
ITE NE
VMOVNE          R2, R3, D16
VMOVEQ          R2, R3, D17
BLX             _objc_msgSend ; not suffixed
...
\end{lstlisting}

\INS{ITE} stands for \IT{if-then-else} 

and it encodes suffixes for the next two instructions.

The first instruction executes if the condition encoded in \INS{ITE} (\IT{NE, not equal}) is true at, and the second---if the condition is not true.
(The inverse condition of \GTT{NE} is \GTT{EQ} (\IT{equal})).

The instruction followed after the second \INS{VMOV} (or \INS{VMOVEQ}) is a normal one, not suffixed (\INS{BLX}).

\myindex{Angry Birds}
One more that's slightly harder, which is also from Angry Birds:

\begin{lstlisting}[caption=Angry Birds Classic,style=customasmARM]
...
ITTTT EQ
MOVEQ           R0, R4
ADDEQ           SP, SP, #0x20
POPEQ.W         {R8,R10}
POPEQ           {R4-R7,PC}
BLX             ___stack_chk_fail ; not suffixed
...
\end{lstlisting}

Four \q{T} symbols in the instruction mnemonic mean that the four subsequent instructions are to be executed if the condition is true.

That's why \IDA adds the \GTT{-EQ} suffix to each one of them. 

And if there was, for example, \INS{ITEEE EQ} (\IT{if-then-else-else-else}), 
then the suffixes would have been set as follows:

\begin{lstlisting}
-EQ
-NE
-NE
-NE
\end{lstlisting}

\myindex{Angry Birds}
Another fragment from Angry Birds:

\begin{lstlisting}[caption=Angry Birds Classic,style=customasmARM]
...
CMP.W           R0, #0xFFFFFFFF
ITTE LE
SUBLE.W         R10, R0, #1
NEGLE           R0, R0
MOVGT           R10, R0
MOVS            R6, #0         ; not suffixed
CBZ             R0, loc_1E7E32 ; not suffixed
...
\end{lstlisting}

\INS{ITTE} (\IT{if-then-then-else}) 

implies that the 1st and 2nd instructions are to be executed if the \GTT{LE} (\IT{Less or Equal})
condition is true, and the 3rd---if the inverse condition (\GTT{GT}---\IT{Greater Than}) 
is true.

Compilers usually don't generate all possible combinations.
\myindex{Angry Birds}

For example, in the mentioned Angry Birds game (\IT{classic} version for iOS)
only these variants of the \INS{IT} instruction are used: 
\INS{IT}, \INS{ITE}, \INS{ITT}, \INS{ITTE}, \INS{ITTT}, \INS{ITTTT}.
\myindex{\GrepUsage}
How to learn this?
In \IDA It is possible to produce listing files, so it was created with an option to show 4 bytes for each opcode.
Then, knowing the high part of the 16-bit opcode (\INS{IT} is \GTT{0xBF}), we do the following using \GTT{grep}:

\begin{lstlisting}
cat AngryBirdsClassic.lst | grep " BF" | grep "IT" > results.lst
\end{lstlisting}

\myindex{ARM!\ThumbTwoMode}

By the way, if you program in ARM assembly language manually for Thumb-2 mode, 
and you add conditional suffixes,
the assembler will add the \INS{IT} instructions automatically with the required flags where it is necessary.

\myparagraph{\NonOptimizingXcodeIV (\ARMMode)}

\begin{lstlisting}[caption=\NonOptimizingXcodeIV (\ARMMode),style=customasmARM]
b               = -0x20
a               = -0x18
val_to_return   = -0x10
saved_R7        = -4

                STR             R7, [SP,#saved_R7]!
                MOV             R7, SP
                SUB             SP, SP, #0x1C
                BIC             SP, SP, #7
                VMOV            D16, R2, R3
                VMOV            D17, R0, R1
                VSTR            D17, [SP,#0x20+a]
                VSTR            D16, [SP,#0x20+b]
                VLDR            D16, [SP,#0x20+a]
                VLDR            D17, [SP,#0x20+b]
                VCMPE.F64       D16, D17
                VMRS            APSR_nzcv, FPSCR
                BLE             loc_2E08
                VLDR            D16, [SP,#0x20+a]
                VSTR            D16, [SP,#0x20+val_to_return]
                B               loc_2E10

loc_2E08
                VLDR            D16, [SP,#0x20+b]
                VSTR            D16, [SP,#0x20+val_to_return]

loc_2E10
                VLDR            D16, [SP,#0x20+val_to_return]
                VMOV            R0, R1, D16
                MOV             SP, R7
                LDR             R7, [SP+0x20+b],#4
                BX              LR
\end{lstlisting}

Almost the same as we already saw, 
but there is too much redundant code because the $a$ and $b$ variables are stored in the local stack, as well
as the return value.

\myparagraph{\OptimizingKeilVI (\ThumbMode)}

\begin{lstlisting}[caption=\OptimizingKeilVI (\ThumbMode),style=customasmARM]
                PUSH    {R3-R7,LR}
                MOVS    R4, R2
                MOVS    R5, R3
                MOVS    R6, R0
                MOVS    R7, R1
                BL      __aeabi_cdrcmple
                BCS     loc_1C0
                MOVS    R0, R6
                MOVS    R1, R7
                POP     {R3-R7,PC}

loc_1C0
                MOVS    R0, R4
                MOVS    R1, R5
                POP     {R3-R7,PC}
\end{lstlisting}


Keil doesn't generate FPU-instructions since it cannot rely on them being
supported on the target CPU, and it cannot be done by straightforward bitwise comparing.
%TODO1: why?
So it calls an external library function to do the comparison: \GTT{\_\_aeabi\_cdrcmple}. 
\myindex{ARM!\Instructions!BCS}

N.B. The result of the comparison is to be left in the flags by this function, so the following
\INS{BCS} (\IT{Carry set---Greater than or equal})
instruction can work without any additional code.

}
\RU{\subsubsection{ARM}

\myparagraph{\OptimizingXcodeIV (\ARMMode)}

\lstinputlisting[caption=\OptimizingXcodeIV (\ARMMode),style=customasm]{patterns/12_FPU/3_comparison/ARM/Xcode_ARM_RU.lst}

\myindex{ARM!\Registers!APSR}
\myindex{ARM!\Registers!FPSCR}
Очень простой случай.
Входные величины помещаются в \GTT{D17} и \GTT{D16} и сравниваются при помощи инструкции \INS{VCMPE}.
Как и в сопроцессорах x86, сопроцессор в ARM имеет свой собственный регистр статуса и флагов (\ac{FPSCR}),
потому что есть необходимость хранить специфичные для его работы флаги.

% TODO -> расписать регистр по битам
\myindex{ARM!\Instructions!VMRS}
И так же, как и в x86, 
в ARM нет инструкций условного перехода, проверяющих биты в регистре статуса сопроцессора. 
Поэтому имеется инструкция \INS{VMRS}, копирующая 4 бита (N, Z, C, V) 
из статуса сопроцессора в биты \IT{общего} статуса (регистр \ac{APSR}).

\myindex{ARM!\Instructions!VMOVGT}
\INS{VMOVGT} это аналог \INS{MOVGT}, инструкция для D-регистров, срабатывающая, если при сравнении один операнд был больше чем второй
(\IT{GT --- Greater Than}). 

Если она сработает, 
в \GTT{D16} запишется значение $a$, лежащее в тот момент в \GTT{D17}.
В обратном случае в \GTT{D16} остается значение $b$.


\myindex{ARM!\Instructions!VMOV}
Предпоследняя инструкция \INS{VMOV} готовит то, что было в \GTT{D16}, для возврата через 
пару регистров \Reg{0} и \Reg{1}.

\myparagraph{\OptimizingXcodeIV (\ThumbTwoMode)}

\begin{lstlisting}[caption=\OptimizingXcodeIV (\ThumbTwoMode),style=customasmARM]
VMOV            D16, R2, R3 ; b
VMOV            D17, R0, R1 ; a
VCMPE.F64       D17, D16
VMRS            APSR_nzcv, FPSCR
IT GT 
VMOVGT.F64      D16, D17
VMOV            R0, R1, D16
BX              LR
\end{lstlisting}

Почти то же самое, что и в предыдущем примере, за парой отличий.
Как мы уже знаем, многие инструкции в режиме ARM можно дополнять условием.
Но в режиме Thumb такого нет.
В 16-битных инструкций просто нет места для лишних 4 битов, при помощи
которых можно было бы закодировать условие выполнения.

\myindex{ARM!\ThumbTwoMode}
Поэтому в Thumb-2 добавили возможность дополнять \\
Thumb-инструкции условиями.
В листинге, сгенерированном при помощи \IDA, мы видим инструкцию \INS{VMOVGT}, 
такую же как и в предыдущем примере.

В реальности там закодирована обычная инструкция \INS{VMOV}, просто \IDA добавила суффикс \GTT{-GT} к ней, 
потому что перед этой инструкцией стоит \INS{IT GT}.

\label{ARM_Thumb_IT}
\myindex{ARM!\Instructions!IT}
\myindex{ARM!if-then block}
Инструкция \INS{IT} определяет так называемый \IT{if-then block}. 
После этой инструкции можно указывать до четырех инструкций, 
к каждой из которых будет добавлен суффикс условия.

В нашем примере \INS{IT GT} означает,
что следующая за ней инструкция будет исполнена, если условие
\IT{GT} (\IT{Greater Than}) справедливо.

\myindex{Angry Birds}
Теперь более сложный пример. Кстати, из 
Angry Birds (для iOS):

\begin{lstlisting}[caption=Angry Birds Classic,style=customasmARM]
...
ITE NE
VMOVNE          R2, R3, D16
VMOVEQ          R2, R3, D17
BLX             _objc_msgSend ; без суффикса
...
\end{lstlisting}

\INS{ITE} означает \IT{if-then-else} 
и кодирует суффиксы для двух следующих за ней инструкций.

Первая из них исполнится, если условие, закодированное в \INS{ITE} (\IT{NE, not equal}) будет в тот момент справедливо,
а вторая~--- если это условие не сработает.
(Обратное условие от \GTT{NE} это \GTT{EQ} (\IT{equal})).

Инструкция следующая за второй \INS{VMOV} (или VMOEQ) нормальная, без суффикса (\INS{BLX}).

\myindex{Angry Birds}
Ещё чуть сложнее, и снова этот фрагмент из Angry Birds:

\begin{lstlisting}[caption=Angry Birds Classic,style=customasmARM]
...
ITTTT EQ
MOVEQ           R0, R4
ADDEQ           SP, SP, #0x20
POPEQ.W         {R8,R10}
POPEQ           {R4-R7,PC}
BLX             ___stack_chk_fail ; без суффикса
...
\end{lstlisting}

Четыре символа \q{T} в инструкции означают, что четыре последующие инструкции будут исполнены если условие соблюдается.
Поэтому \IDA добавила ко всем четырем инструкциям суффикс \GTT{-EQ}. 
А если бы здесь было, например,
\INS{ITEEE EQ} (\IT{if-then-else-else-else}), 
тогда суффиксы для следующих четырех инструкций были бы расставлены так:

\begin{lstlisting}
-EQ
-NE
-NE
-NE
\end{lstlisting}

\myindex{Angry Birds}
Ещё фрагмент из Angry Birds:

\begin{lstlisting}[caption=Angry Birds Classic,style=customasmARM]
...
CMP.W           R0, #0xFFFFFFFF
ITTE LE
SUBLE.W         R10, R0, #1
NEGLE           R0, R0
MOVGT           R10, R0
MOVS            R6, #0         ; без суффикса
CBZ             R0, loc_1E7E32 ; без суффикса
...
\end{lstlisting}

\INS{ITTE} (\IT{if-then-then-else}) 
означает, что первая и вторая инструкции исполнятся, если условие \GTT{LE} (\IT{Less or Equal})
справедливо, а третья~--- если справедливо обратное условие (\GTT{GT} --- \IT{Greater Than}).

Компиляторы способны генерировать далеко не все варианты.

\myindex{Angry Birds}
Например, в вышеупомянутой игре Angry Birds (версия \IT{classic} для iOS)

встречаются только такие варианты инструкции \INS{IT}: 
\INS{IT}, \INS{ITE}, \INS{ITT}, \INS{ITTE}, \INS{ITTT}, \INS{ITTTT}.
\myindex{\GrepUsage}
Как это узнать?
В \IDA можно сгенерировать листинг (что и было сделано), только в опциях был установлен показ 4 байтов для каждого опкода.

Затем, зная что старшая часть 16-битного опкода (\INS{IT} это \GTT{0xBF}), сделаем при помощи \GTT{grep} это:

\begin{lstlisting}
cat AngryBirdsClassic.lst | grep " BF" | grep "IT" > results.lst
\end{lstlisting}

\myindex{ARM!\ThumbTwoMode}
Кстати, если писать на ассемблере для режима Thumb-2 вручную, и дополнять инструкции суффиксами
условия, то ассемблер автоматически будет добавлять инструкцию \INS{IT} с соответствующими флагами там,
где надо.

\myparagraph{\NonOptimizingXcodeIV (\ARMMode)}

\begin{lstlisting}[caption=\NonOptimizingXcodeIV (\ARMMode),style=customasmARM]
b               = -0x20
a               = -0x18
val_to_return   = -0x10
saved_R7        = -4

                STR             R7, [SP,#saved_R7]!
                MOV             R7, SP
                SUB             SP, SP, #0x1C
                BIC             SP, SP, #7
                VMOV            D16, R2, R3
                VMOV            D17, R0, R1
                VSTR            D17, [SP,#0x20+a]
                VSTR            D16, [SP,#0x20+b]
                VLDR            D16, [SP,#0x20+a]
                VLDR            D17, [SP,#0x20+b]
                VCMPE.F64       D16, D17
                VMRS            APSR_nzcv, FPSCR
                BLE             loc_2E08
                VLDR            D16, [SP,#0x20+a]
                VSTR            D16, [SP,#0x20+val_to_return]
                B               loc_2E10

loc_2E08
                VLDR            D16, [SP,#0x20+b]
                VSTR            D16, [SP,#0x20+val_to_return]

loc_2E10
                VLDR            D16, [SP,#0x20+val_to_return]
                VMOV            R0, R1, D16
                MOV             SP, R7
                LDR             R7, [SP+0x20+b],#4
                BX              LR
\end{lstlisting}

Почти то же самое, что мы уже видели, 
но много избыточного кода из-за хранения $a$ и $b$, 
а также выходного значения, в локальном стеке.


\myparagraph{\OptimizingKeilVI (\ThumbMode)}

\begin{lstlisting}[caption=\OptimizingKeilVI (\ThumbMode),style=customasmARM]
                PUSH    {R3-R7,LR}
                MOVS    R4, R2
                MOVS    R5, R3
                MOVS    R6, R0
                MOVS    R7, R1
                BL      __aeabi_cdrcmple
                BCS     loc_1C0
                MOVS    R0, R6
                MOVS    R1, R7
                POP     {R3-R7,PC}

loc_1C0
                MOVS    R0, R4
                MOVS    R1, R5
                POP     {R3-R7,PC}
\end{lstlisting}

Keil не генерирует FPU-инструкции, потому что не 
рассчитывает на то, что они будет поддерживаться, а простым сравнением побитово здесь не обойтись.

%TODO1: why?
Для сравнения вызывается библиотечная функция \GTT{\_\_aeabi\_cdrcmple}. 
\myindex{ARM!\Instructions!BCS}

N.B. Результат сравнения эта функция оставляет в флагах, чтобы следующая за вызовом инструкция
\INS{BCS} (\IT{Carry set~--- Greater than or equal})
могла работать без дополнительного кода.

}
\EN{\subsubsection{ARM64}

\myparagraph{\Optimizing GCC (Linaro) 4.9}

\lstinputlisting[style=customasm]{patterns/12_FPU/3_comparison/ARM/ARM64_GCC_O3_EN.lst}

The ARM64 \ac{ISA} has FPU-instructions 
which set \ac{APSR} the CPU flags instead of \ac{FPSCR} for convenience.
The\ac{FPU} is not a separate device here anymore (at least, logically).
\myindex{ARM!\Instructions!FCMPE}
Here we see \INS{FCMPE}. It compares the two values passed in \RegD{0} and \RegD{1} (which are the first and second arguments of the function)
and sets \ac{APSR} flags (N, Z, C, V).

\myindex{ARM!\Instructions!FCSEL}
\INS{FCSEL} (\IT{Floating Conditional Select}) copies the value of \RegD{0} or \RegD{1} into \RegD{0} depending on the condition (\GTT{GT}---Greater Than),
and again, it uses flags in \ac{APSR} register instead of \ac{FPSCR}.

This is much more convenient, compared to the instruction set in older CPUs.

If the condition is true (\GTT{GT}), then the value of \RegD{0} 
is copied into \RegD{0} (i.e., nothing happens).
If the condition is not true, the value of \RegD{1} 
is copied into \RegD{0}.

\myparagraph{\NonOptimizing GCC (Linaro) 4.9}

\lstinputlisting[style=customasm]{patterns/12_FPU/3_comparison/ARM/ARM64_GCC_EN.lst}

Non-optimizing GCC is more verbose.

First, the function saves its input argument values in the local stack (\IT{Register Save Area}).
Then the code reloads these values into registers
\RegX{0}/\RegX{1} and finally copies them to 
\RegD{0}/\RegD{1} to be compared using \INS{FCMPE}. 
A lot of redundant code, 
but that is how non-optimizing compilers work.
\INS{FCMPE} compares the values and sets the \ac{APSR} flags.
At this moment, 
the compiler is not thinking yet about the more convenient \INS{FCSEL} instruction, so it proceed using old methods: 
using the \INS{BLE} instruction (\IT{Branch if Less than or Equal}).
In the first case ($a>b$), the value of $a$ gets loaded 
into \RegX{0}.
In the other case ($a<=b$), the value of $b$ gets loaded into 
\RegX{0}.
Finally, the value from \RegX{0} gets copied into \RegD{0}, 
because the return value needs to be in this 
register.

\mysubparagraph{\Exercise}

As an exercise, you can try optimizing this piece of code 
manually by removing redundant instructions and not introducing new ones (including \INS{FCSEL}).

\myparagraph{\Optimizing GCC (Linaro) 4.9---float}

Let's also rewrite this example to use \Tfloat instead of \Tdouble.

\begin{lstlisting}[style=customc]
float f_max (float a, float b)
{
	if (a>b)
		return a;

	return b;
};
\end{lstlisting}

\lstinputlisting[style=customasmARM]{patterns/12_FPU/3_comparison/ARM/ARM64_GCC_O3_float_EN.lst}

It is the same code, but the S-registers are used instead of D- ones.
It's because numbers of type \Tfloat are passed in 32-bit S-registers (which are in fact the lower parts of the 64-bit D-registers).

}
\RU{\subsubsection{ARM64}

\myparagraph{\Optimizing GCC (Linaro) 4.9}

\lstinputlisting[style=customasm]{patterns/12_FPU/3_comparison/ARM/ARM64_GCC_O3_RU.lst}

В ARM64 \ac{ISA} теперь есть FPU-инструкции, устанавливающие флаги CPU \ac{APSR} вместо \ac{FPSCR} для удобства.
\ac{FPU} больше не отдельное устройство (по крайней мере логически).
\myindex{ARM!\Instructions!FCMPE}
Это \INS{FCMPE}. Она сравнивает два значения, переданных в \RegD{0} и \RegD{1} 
(а это первый и второй аргументы функции) и выставляет флаги в \ac{APSR} (N, Z, C, V).

\myindex{ARM!\Instructions!FCSEL}
\INS{FCSEL} (\IT{Floating Conditional Select}) копирует значение \RegD{0} или
\RegD{1} в \RegD{0} в зависимости от условия 
(\GTT{GT} --- Greater Than --- больше чем),
и снова, она использует флаги в регистре \ac{APSR} вместо \ac{FPSCR}.
Это куда удобнее, если сравнивать с тем набором инструкций, что был в процессорах раньше.

Если условие верно (\GTT{GT}), тогда значение из \RegD{0} копируется в \RegD{0} (т.е. ничего не происходит).
Если условие не верно, то значение \RegD{1} копируется в \RegD{0}.

\myparagraph{\NonOptimizing GCC (Linaro) 4.9}

\lstinputlisting[style=customasm]{patterns/12_FPU/3_comparison/ARM/ARM64_GCC_RU.lst}

Неоптимизирующий GCC более многословен.
В начале функция сохраняет значения входных аргументов в локальном стеке (\IT{Register Save Area}).
Затем код перезагружает значения в регистры
\RegX{0}/\RegX{1} и наконец копирует их в 
\RegD{0}/\RegD{1} для сравнения инструкцией \INS{FCMPE}. 
Много избыточного кода, но так работают неоптимизирующие компиляторы.
\INS{FCMPE} сравнивает значения и устанавливает флаги в \ac{APSR}.
В этот момент компилятор ещё не думает о более удобной инструкции \INS{FCSEL}, так что он работает старым 
методом: 
использует инструкцию \INS{BLE} (\IT{Branch if Less than or Equal} (переход если меньше или равно)).
В одном случае ($a>b$) значение $a$ перезагружается в \RegX{0}.
В другом случае ($a<=b$) значение $b$ загружается в \RegX{0}.
Наконец, значение из \RegX{0} копируется в \RegD{0}, 
потому что возвращаемое значение оставляется в этом регистре.

\mysubparagraph{\Exercise}

Для упражнения вы можете попробовать оптимизировать этот фрагмент кода вручную, удалив избыточные инструкции,
но не добавляя новых (включая \INS{FCSEL}).

\myparagraph{\Optimizing GCC (Linaro) 4.9: float}

Перепишем пример. Теперь здесь \Tfloat вместо \Tdouble.

\begin{lstlisting}
float f_max (float a, float b)
{
	if (a>b)
		return a;

	return b;
};
\end{lstlisting}

\lstinputlisting[style=customasm]{patterns/12_FPU/3_comparison/ARM/ARM64_GCC_O3_float_RU.lst}

Всё то же самое, только используются S-регистры вместо D-.
Так что числа типа \Tfloat передаются в 32-битных S-регистрах (а это младшие части 64-битных D-регистров).

}
\EN{\subsubsection{MIPS}

\myindex{MIPS!\Registers!FCCR}
The co-processor of the MIPS processor has a condition bit which can be set in the FPU and checked in the CPU.

Earlier MIPS-es have only one condition bit (called FCC0), later ones have 8 (called FCC7-FCC0).

This bit (or bits) are located in the register called FCCR.

\lstinputlisting[caption=\Optimizing GCC 4.4.5 (IDA),style=customasmMIPS]{patterns/12_FPU/3_comparison/MIPS_O3_IDA_EN.lst}

\myindex{MIPS!\Instructions!C.LT.D}
\INS{C.LT.D} compares two values. 
\GTT{LT} is the condition \q{Less Than}.
\GTT{D} implies values of type \Tdouble.
Depending on the result of the comparison, the FCC0 condition bit is either set or cleared.

\myindex{MIPS!\Instructions!BC1T}
\myindex{MIPS!\Instructions!BC1F}
\INS{BC1T} checks the FCC0 bit and jumps if the bit is set.
\GTT{T} means that the jump is to be taken if the bit is set (\q{True}).
There is also the instruction \INS{BC1F} which jumps if the bit is cleared (\q{False}).

Depending on the jump, one of function arguments is placed into \$F0.
}
\RU{\subsubsection{MIPS}

\myindex{MIPS!\Registers!FCCR}

В сопроцессоре MIPS есть бит результата, который устанавливается в FPU и проверяется в CPU.

Ранние MIPS имели только один бит (с названием FCC0), а у поздних их 8 (с названием FCC7-FCC0).
Этот бит (или биты) находятся в регистре с названием FCCR.

\lstinputlisting[caption=\Optimizing GCC 4.4.5 (IDA),style=customasm]{patterns/12_FPU/3_comparison/MIPS_O3_IDA_RU.lst}

\myindex{MIPS!\Instructions!C.LT.D}
\INS{C.LT.D} сравнивает два значения. 
\GTT{LT} это условие \q{Less Than} (меньше чем).
\GTT{D} означает переменные типа \Tdouble.

В зависимости от результата сравнения, бит FCC0 устанавливается или очищается.

\myindex{MIPS!\Instructions!BC1T}
\myindex{MIPS!\Instructions!BC1F}
\INS{BC1T} проверяет бит FCC0 и делает переход, если бит выставлен.
\GTT{T} означает, что переход произойдет если бит выставлен (\q{True}).
Имеется также инструкция \INS{BC1F} которая сработает, если бит сброшен (\q{False}).

В зависимости от перехода один из аргументов функции помещается в регистр \$F0.

}


\subsection{Quelques constantes}

Il est facile de trouver la représentation de certaines constantes pour des nombres
encodés au format IEEE 754 sur Wikipedia.
Il est intéressant de savoir que 0,0 en IEEE 754 est représenté par 32 bits à zéro
(pour la simple précision) ou 64 bits à zéro (pour la double).
Donc pour mettre une variable flottante à 0,0 dans un registre ou en mémoire, on
peut utiliser l'instruction \MOV ou \TT{XOR reg, reg}.
\myindex{\CStandardLibrary!memset()}
Ceci est utilisable pour les structures où des variables de types variés sont présentes.
Avec la fonction usuelle memset() il est possible de mettre toutes les variables
entières à 0, toutes les variables booléennes à \IT{false}, tous les pointeurs à
NULL, et toutes les variables flottantes (de n'importe quelle précision) à 0,0.

\subsection{Copie}

On peut tout d'abord penser qu'il faut utiliser les instructions \INS{FLD}/\INS{FST}
pour charger et stocker (et donc, copier) des valeurs IEEE 754.
Néanmoins, la même chose peut-être effectuée plus facilement avec l'instruction usuelle
\INS{MOV}, qui, bien sûr, copie les valeurs au niveau binaire.

\subsection{Pile, calculateurs et notation polonaise inverse}

\myindex{Notation polonaise inverse}

Maintenant nous comprenons pourquoi certains anciens calculateurs utilisent la notation
Polonaise inverse
\footnote{\href{http://go.yurichev.com/17354}{wikipedia.org/wiki/Reverse\_Polish\_notation}}.

Par exemple, pour aditionner 12 et 34, on doit entrer 12, puis 34, et presser le signe
\q{plus}.

C'est parce que les anciens calculateurs étaient juste des implémentations de machine
à pile, et c'était bien plus simple que de manipuler des expressions complexes avec
parenthèses.

\subsection{80 bits?}

\myindex{Punched card}

Représentation interne des nombres dans le FPU --- 80-bit.
Nombre étrange, car il n'est pas de la forme $2^n$.
Il y a une hypothèse que c'est probablement dû à des raisons historiques---le standard
IBM de carte perforée peut encoder 12 lignes de 80 bits.
La résolution en mode texte de $80\cdot 25$ était aussi très populaire dans le passé.

Il y a une autre explication sur Wikipedia: \url{https://en.wikipedia.org/wiki/Extended_precision}.

Si vous en savez plus, s'il vous plait envoyez un email à l'auteur: \EMAIL{}.

\subsection{x64}

Sur la manière dont sont traités les nombres à virgules flottante en x86-64, lire
ici: \myref{floating_SIMD}.

% sections
\subsection{\Exercise}

\begin{itemize}
	\item \url{http://challenges.re/27}
\end{itemize}


