\subsection{典型的なスタックレイアウト}

最初の命令を実行する前の、
関数の開始時の32ビット環境での典型的なスタックレイアウトは次のようになります。

\input{patterns/02_stack/stack_layout}

% I think this only applies to RISC architectures
% that don't have a POP instruction that only lets you read one value
% (ie. ARM and MIPS).
% In x86, the return address is saved before entering the function,
% and the function does not have the chance to save the frame pointer.
% Also, you should mention that this is how the stack looks like
% right after the function prologue,
% which some readers might think is the first instruction,
% but is needed to save the frame pointer.
