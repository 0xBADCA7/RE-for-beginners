\section{\Stack}
\label{sec:stack}
\myindex{\Stack}

La pile est une des structures de données les plus fondamentales en informatique.
\footnote{\href{http://go.yurichev.com/17119}{wikipedia.org/wiki/Call\_stack}}.
\ac{AKA} \ac{LIFO}.

Techniquement, il s'agit d'un bloc de mémoire présent dans l'espace d'adressage
d'un processus et qui est utilisé par le registre \ESP ou \RSP en x86 ou x64,
ou par le registre \ac{SP} en ARM comme un pointeur dans ce bloc mémoire. 

\myindex{ARM!\Instructions!PUSH}
\myindex{ARM!\Instructions!POP}
\myindex{x86!\Instructions!PUSH}
\myindex{x86!\Instructions!POP}
Les instructions d'accès à la pile sont \PUSH et \POP (en x86 ainsi qu'en ARM Thumb-mode).
\PUSH soustrait à \ESP/\RSP/\ac{SP} 4 en mode 32-bit (ou 8 en mode 64-bit) et écrit
ensuite le contenu de l'opérande associée à l'adresse mémoire pointée par \ESP/\RSP/\ac{SP}.

\POP est l'operation inverse: elle récupére la donnée depuis l'adresse mémoire pointée par \ac{SP},
l'écrit dans l'opérande associée (souvent un registre) puis ajoute 4 (ou 8) au \glslink{stack pointer}{pointeur de pile}.

Après une allocation sur la pile, le \glslink{stack pointer}{pointeur de pile} pointe sur le bas de la pile.
\PUSH décrémente le \gls{stack pointer} et \POP l'incrémente.

Le bas de la pile représente en réalité le début de la mémoire allouée pour
 le bloc de pile. Cela semble étrange, mais c'est comme ça.

ARM supporte à la fois les piles ascendantes et descendantes.

\myindex{ARM!\Instructions!STMFD}
\myindex{ARM!\Instructions!LDMFD}
\myindex{ARM!\Instructions!STMED}
\myindex{ARM!\Instructions!LDMED}
\myindex{ARM!\Instructions!STMFA}
\myindex{ARM!\Instructions!LDMFA}
\myindex{ARM!\Instructions!STMEA}
\myindex{ARM!\Instructions!LDMEA}

Par exemple les instructions \ac{STMFD}/\ac{LDMFD}, \ac{STMED}/\ac{LDMED} sont utilisées pour gérer les piles
descendantes (qui grandissent vers le bas en commençant avec une adresse haute et évoluent vers une plus basse).

Les instructions \ac{STMFA}/\ac{LDMFA}, \ac{STMEA}/\ac{LDMEA} sont utilisées pour gérer les piles montantes
(qui grandissent vers les adresses hautes de l'adresse space, en commençant avec une adresse située en bas de l'adresse space)

% It might be worth mentioning that STMED and STMEA write first,
% and then move the pointer,
% and that LDMED and LDMEA move the pointer first, and then read.
% In other words, ARM not only lets the stack grow in a non-standard direction,
% but also in a non-standard order.
% Maybe this can be in the glossary, which would explain why E stands for "empty".

\subsection{Pourquoi la pile grandit en descendant ?}
\label{stack_grow_backwards}

Intuitivement, on pourrait penser que la pile grandit vers le haut, i.e. vers des
adresses plus élevées, comme n'importe qu'elle autre structure de données.

La raison pour laquelle la pile grandit vers le bas est probablement historique.
Dans le passé, les ordinateurs étaient énormes et occupaient des piéces entières,
il était facile de diviser la mémoire en deux parties, une pour le \gls{heap} et
une pour la pile.
Evidemment, on ignorait quelle serait la taille du \gls{heap} et de la pile durant
l'éxécution du progamme, donc cette solution était la plus simple possible.

\input{patterns/02_stack/stack_and_heap}

Dans \RitchieThompsonUNIX on peut lire:

\begin{framed}
\begin{quotation}
The user-core part of an image is divided into three logical segments. The program text segment begins at location 0 in the virtual address space. During execution, this segment is write-protected and a single copy of it is shared among all processes executing the same program. At the first 8K byte boundary above the program text segment in the virtual address space begins a nonshared, writable data segment, the size of which may be extended by a system call. Starting at the highest address in the virtual address space is a pile segment, which automatically grows downward as the hardware's pile pointer fluctuates.
\end{quotation}
\end{framed}

Cela nous rappelle comment certains étudiants prennent des notes pour deux cours différents dans
un seul et même cahier en prenant un cours d'un côté du cahier, et l'autre cours de l'autre côté.
Les notes de cours finissent par se rencontrer à un moment dans le cahier quand il n'y a plus de place.

% I think if we want to expand on this analogy,
% one might remember that the line number increases as as you go down a page.
% So when you decrease the address when pushing to the stack, visually,
% the stack does grow upwards.
% Of course, the problem is that in most human languages,
% just as with computers,
% we write downwards, so this direction is what makes buffer overflows so messy.

\subsection{Quel est le rôle de la pile ?}

% subsections
\subsubsection{Sauvegarder l'adresse de retour de la fonction}

\myparagraph{x86}

\myindex{x86!\Instructions!CALL}
Lorsque l'on appelle une fonction avec une instruction \CALL, l'adresse du point
exactement après cette dernière est sauvegardée sur la pile et un saut inconditionnel
à l'adresse de l'opérande \CALL est exécuté.

\myindex{x86!\Instructions!PUSH}
\myindex{x86!\Instructions!JMP}
L'instruction \CALL est équivalente à la\\
paire d'instructions \INS{PUSH address\_after\_call / JMP operand}.

\myindex{x86!\Instructions!RET}
\myindex{x86!\Instructions!POP}
\RET va chercher une valeur sur la pile et y saute~---ce qui est équivalent à
la paire d'instructions \TT{POP tmp / JMP tmp}.

\myindex{\Stack!\FR{Débordement de pile}}
\myindex{\Recursion}
Déborder de la pile est très facile. Il suffit de lancer une récursion éternelle:

\begin{lstlisting}[style=customc]
void f()
{
	f();
};
\end{lstlisting}

MSVC 2008 signale le problème:

\begin{lstlisting}
c:\tmp6>cl ss.cpp /Fass.asm
Microsoft (R) 32-bit C/C++ Optimizing Compiler Version 15.00.21022.08 for 80x86
Copyright (C) Microsoft Corporation.  All rights reserved.

ss.cpp
c:\tmp6\ss.cpp(4) : warning C4717: 'f' : recursive on all control paths, function will cause runtime stack overflow
\end{lstlisting}

\dots mais génère tout de même le code correspondant:

\begin{lstlisting}[style=customasmx86]
?f@@YAXXZ PROC			; f
; File c:\tmp6\ss.cpp
; Line 2
	push	ebp
	mov	ebp, esp
; Line 3
	call	?f@@YAXXZ	; f
; Line 4
	pop	ebp
	ret	0
?f@@YAXXZ ENDP			; f
\end{lstlisting}

\dots Si nous utilisons l'option d'optimisation du compilateur (option \TT{\Ox})
le code optimisé ne va pas déborder de la pile et au lieu de cela va fonctionner
\IT{correctemment}\footnote{ironique ici}:

\begin{lstlisting}[style=customasmx86]
?f@@YAXXZ PROC			; f
; File c:\tmp6\ss.cpp
; Line 2
$LL3@f:
; Line 3
	jmp	SHORT $LL3@f
?f@@YAXXZ ENDP			; f
\end{lstlisting}

GCC 4.4.1 génère un code similaire dans les deux cas, sans, toutefois émettre
d'avertissement à propos de ce problème.

\myparagraph{ARM}

\myindex{ARM!\Registers!Link Register}
Les programmes ARM utilisent également la pile pour sauver les adresses de retour,
mais différemment.
Comme mentionné dans \q{\HelloWorldSectionName}~(\myref{sec:hw_ARM}),
\ac{RA} est sauvegardé dans \ac{LR} (\gls{link register}).
Si l'on a toutefois besoin d'appeler une autre fonction et d'utiliser le registre
\ac{LR} une fois de plus, sa valeur doit être sauvegardée.
\myindex{Function prologue}
Usuellement, cela se fait dans le prologue de la fonction.

\myindex{ARM!\Instructions!PUSH}
\myindex{ARM!\Instructions!POP}
Souvent, nous voyons des instructions comme \INS{PUSH {R4-R7,LR}} en même temps
que cette instruction dans le prologue \INS{POP {R4-R7,PC}}---ces registres qui
sont utilisés dans la fonction sont sauvegardés sur la pile, \ac{LR} inclus.

\myindex{ARM!Fonction leaf} % FIXME traduire avec feuille ?
Néanmoins, si une fonction n'appelle jamais d'autre fonction, dans la terminologie
\ac{RISC} elle est appelée \IT{\glslink{leaf function}{fonction leaf}}\footnote{\href{http://go.yurichev.com/17064}{infocenter.arm.com/help/index.jsp?topic=/com.arm.doc.faqs/ka13785.html}}.
Ceci a comme conséquence que les fonctions leaf ne sauvegardent pas le registre
\ac{LR} (car elles ne le modifient pas).
Si une telle fonction est petite et utilise un petit nombre de registres, elle
peut ne pas utiliser du tout la pile.
Ainsi, il est possible d'appeler des fonctions leaf sans utiliser la pile.
Ce qui peut être plus rapide sur des vieilles machines x86 car la mémoire externe
n'est pas utilisée pour la pile
\footnote{Il y a quelques temps, sur PDP-11 et VAX, l'instruction CALL (appel d'autres fonctions) était coûteux; jusqu'à 50\%
du temps d'exécution pouvait être passé à ça, il était donc considèré qu'avoir un grand nombre de petite fonction était un \gls{anti-pattern} \InSqBrackets{\TAOUP Chapter 4, Part II}.}.
Cela peut être utile pour des situations où la mémoire pour la pile n'est pas
encore allouée ou disponible.

Quelques exemples de fonctions leaf:
\myref{ARM_leaf_example1}, \myref{ARM_leaf_example2},
\myref{ARM_leaf_example3}, \myref{ARM_leaf_example4}, \myref{ARM_leaf_example5},
\myref{ARM_leaf_example6}, \myref{ARM_leaf_example7}, \myref{ARM_leaf_example10}.


\subsubsection{Passage des arguments de fonction}

Le moyen le plus utilisé pour passer des arguments en x86 est appelé \q{cdecl}:

\begin{lstlisting}[style=customasmx86]
push arg3
push arg2
push arg1
call f
add esp, 12 ; 4*3=12
\end{lstlisting}

La fonction \glslink{callee}{appelée} reçoit ses arguments par la pile.

Voici donc comment sont stockés les arguments sur la pile avant l'exécution
de la première instruction de la fonction \ttf{}:

\begin{center}
\begin{tabular}{ | l | l | }
\hline
ESP & return address \\
\hline
ESP+4 & \argument \#1, \MarkedInIDAAs{} \TT{arg\_0} \\
\hline
ESP+8 & \argument \#2, \MarkedInIDAAs{} \TT{arg\_4} \\
\hline
ESP+0xC & \argument \#3, \MarkedInIDAAs{} \TT{arg\_8} \\
\hline
\dots & \dots \\
\hline
\end{tabular}
\end{center}

Pour plus d'information sur les conventions d'appel, voir cette section~(\myref{sec:callingconventions}).

\par
A propos, la fonction \glslink{callee}{appelée} n'a aucune d'information sur le
nombre d'arguments qui ont été passés.
Les fonctions C avec un nombre variable d'arguments (comme \printf) déterminent
leur nombre en utilisant les spécificateurs de la chaîne de format (qui commencent
pas le symbole \%).

Si nous écrivons quelque comme:

\begin{lstlisting}
printf("%d %d %d", 1234);
\end{lstlisting}

%% \printf will print 1234, and then two random numbers\footnote{Not random in strict sense, but rather unpredictable: \myref{noise_in_stack}}, which were lying next to it in the stack.
\printf va afficher 1234, et deux autres nombres aléatoires\footnote{Pas aléatoire
dans le sens strict du terme, mais plutôt imprévisibles: \myref{noise_in_stack}}},
qui sont situés à côté dans la pile.

\par
C'est pourquoi la façon dont la fonction \main est déclarée n'est pas très importante:
comme \main, \\\TT{main(int argc, char *argv[])} ou \TT{main(int argc, char *argv[], char *envp[])}.

En fait, le code-\ac{CRT} appelle \main, schématiquement, de cette façon:
	
\begin{lstlisting}[style=customasmx86]
push envp
push argv
push argc
call main
...
\end{lstlisting}

Si vous déclarez \main comme \main sans argument, ils sont néanmoins toujours présents
sur la pile, mais ne sont pas utilisés.
Si vous déclarez \main as comme \TT{main(int argc, char *argv[])},
vous pourrez utiliser les deux premiers arguments, et le troisième restera \q{invisible}
pour votre fonction.
Il est même possible de déclarer \main comme \TT{main(int argc)}, cela fonctionnera.

\myparagraph{Autres façons de passer les arguments}

Il est à noter que rien n'oblige les programmeurs à passer les arguments à travers
la pile. Ce n'est pas une exigence.
On peut implémenter n'importe quelle autre méthode sans utiliser du tout la pile.

Une méthode répandue chez les débutants en assembleur est de passer les arguments
par des variables globales, comme:

\lstinputlisting[caption=Code assembleur,style=customasmx86]{patterns/02_stack/global_args.asm}

Mais cette méthode a un inconvénient évident: la fonction \IT{do\_something()}
ne peut pas s'appeler elle-même récursivement (ou par une autre fonction),
car il faudrait écraser ses propres arguments.
La même histoire avec les variables locales: si vous les stocker dans des variables
globales, la fonction ne peut pas s'appeler elle-même.
Et ce n'est pas thread-safe.
\footnote{Correctemment implémenter, chaque thread aurait sa propre pile avec ses propres arguments/variables.}.
Une méthode qui stocke ces informations sur la pile rend cela plus facile---elle
peut contenir autant d'arguments de fonctions et/ou de valeurs, que la pile a d'espace.

\InSqBrackets{\TAOCPvolI{}, 189} mentionne un schéma encore plus étrange, particulièrement
pratique sur les IBM System/360.

\myindex{MS-DOS}
\myindex{x86!\Instructions!INT}

MS-DOS a une manière de passer tout les arguments de fonctions via des registres,
par exemple, c'est un morceau de code pour un ancien MS-DOS 16-bit qui affiche
``Hello, world!'':

\begin{lstlisting}[style=customasmx86]
mov  dx, msg      ; address of message
mov  ah, 9        ; 9 means "print string" function
int  21h          ; DOS "syscall"

mov  ah, 4ch      ; "terminate program" function
int  21h          ; DOS "syscall"

msg  db 'Hello, World!\$'
\end{lstlisting}

\myindex{fastcall}
C'est presque similaire à la méthode \myref{fastcall}.
Et c'est aussi très similaire aux appels systèmes sous Linux (\myref{linux_syscall}) et Windows.

\myindex{x86!\Flags!CF}
Si une fonction MS-DOS devait renvoyer une valeur booléenne (i.e., un simple bit,
souvent pour indiquer un état d'erreur), le flag \TT{CF} était souvent utilisé.

Par exemple:

\begin{lstlisting}[style=customasmx86]
mov ah, 3ch       ; create file
lea dx, filename
mov cl, 1
int 21h
jc  error
mov file_handle, ax
...
error:
...
\end{lstlisting}

En cas d'erreur, le flag \TT{CF} est mis. Sinon, le handle du fichier nouvellement
créer est retourné via \TT{AX}.

Cette méthode est encore utilisée par les programmeurs en langage d'assemblage.
Dans le code source de Windows Research Kernel (qui est très similaire à Windows
2003) nous pouvons trouver quelque chose comme ça (file \IT{base/ntos/ke/i386/cpu.asm}):

\begin{lstlisting}[style=customasmx86]
        public  Get386Stepping
Get386Stepping  proc

        call    MultiplyTest            ; Perform multiplication test
        jnc     short G3s00             ; if nc, muttest is ok
        mov     ax, 0
        ret
G3s00:
        call    Check386B0              ; Check for B0 stepping
        jnc     short G3s05             ; if nc, it's B1/later
        mov     ax, 100h                ; It is B0/earlier stepping
        ret

G3s05:
        call    Check386D1              ; Check for D1 stepping
        jc      short G3s10             ; if c, it is NOT D1
        mov     ax, 301h                ; It is D1/later stepping
        ret

G3s10:
        mov     ax, 101h                ; assume it is B1 stepping
        ret

	...

MultiplyTest    proc

        xor     cx,cx                   ; 64K times is a nice round number
mlt00:  push    cx
        call    Multiply                ; does this chip's multiply work?
        pop     cx
        jc      short mltx              ; if c, No, exit
        loop    mlt00                   ; if nc, YEs, loop to try again
        clc
mltx:
        ret

MultiplyTest    endp
\end{lstlisting}


\subsubsection{Stockage des variables locales}

Une fonction peut allouer de l'espace sur la pile pour ses variables locales
simplement en décrémentant le \glslink{stack pointer}{pointeur de pile} vers le
bas de la pile.

% I think here, "stack bottom" means the lowest address in the stack space,
% but the reader might also think it means towards the top of the stack space,
% like in a pop, so you might change "towards the stack bottom" to
% "towards the lowest address of the stack", or just take it out,
% since "decreasing" also suggests that.

Donc, c'est très rapide, peu importe combien de variables locales sont définies.
Ce n'est pas une nécessité de stocker les variables locales sur la pile.
Vous pouvez stocker les variables locales où bon vous semble,
mais c'est traditionnellement fait comme cela.

\EN{\subsubsection{x86: alloca() function}
\label{alloca}
\myindex{\CStandardLibrary!alloca()}

\newcommand{\AllocaSrcPath}{C:\textbackslash{}Program Files (x86)\textbackslash{}Microsoft Visual Studio 10.0\textbackslash{}VC\textbackslash{}crt\textbackslash{}src\textbackslash{}intel}

It is worth noting the \TT{alloca()} function
\footnote{In MSVC, the function implementation can be found in \TT{alloca16.asm} and \TT{chkstk.asm} in \\
\TT{\AllocaSrcPath{}}}.
This function works like \TT{malloc()}, but allocates memory directly on the stack.
% page break added to prevent "\vref on page boundary" error. it may be dropped in future.
The allocated memory chunk does not have to be freed via a \TT{free()} function call, \\
since the function epilogue (\myref{sec:prologepilog}) returns \ESP back to its initial state and 
the allocated memory is just \IT{dropped}.
It is worth noting how \TT{alloca()} is implemented.
In simple terms, this function just shifts \ESP downwards toward the stack bottom by the number of bytes you need and sets \ESP as a pointer to the \IT{allocated} block.

Let's try:

\lstinputlisting[style=customc]{patterns/02_stack/04_alloca/2_1.c}

\TT{\_snprintf()} function works just like \printf, but instead of dumping the result into \gls{stdout} (e.g., to terminal or 
console), it writes it to the \TT{buf} buffer. Function \puts copies the contents of \TT{buf} to \gls{stdout}. Of course, these two
function calls might be replaced by one \printf call, but we have to illustrate small buffer usage.

\myparagraph{MSVC}

Let's compile (MSVC 2010):

\lstinputlisting[caption=MSVC 2010,style=customasm]{patterns/02_stack/04_alloca/2_2_msvc.asm}

\myindex{Compiler intrinsic}
The sole \TT{alloca()} argument is passed via \EAX (instead of pushing it into the stack)
\footnote{It is because alloca() is rather a compiler intrinsic (\myref{sec:compiler_intrinsic}) than a normal function.
One of the reasons we need a separate function instead of just a couple of instructions in the code,
is because the \ac{MSVC} alloca() implementation also has code which reads from the memory just allocated, in order to let the \ac{OS} map
physical memory to this \ac{VM} region.
After the \TT{alloca()} call, \ESP points to the block of 600 bytes and we can use it as memory for the \TT{buf} array.}.

\myparagraph{GCC + \IntelSyntax}

GCC 4.4.1 does the same without calling external functions:

\lstinputlisting[caption=GCC 4.7.3,style=customasm]{patterns/02_stack/04_alloca/2_1_gcc_intel_O3_EN.asm}

\myparagraph{GCC + \ATTSyntax}

Let's see the same code, but in AT\&T syntax:

\lstinputlisting[caption=GCC 4.7.3,style=customasm]{patterns/02_stack/04_alloca/2_1_gcc_ATT_O3.s}

\myindex{\ATTSyntax}
The code is the same as in the previous listing.

By the way, \INS{movl \$3, 20(\%esp)} corresponds to
\INS{mov DWORD PTR [esp+20], 3} in Intel-syntax.
In the AT\&T syntax, the register+offset format of addressing memory looks like
\TT{offset(\%{register})}.

}
\FR{\subsubsection{x86: alloca() function}
\label{alloca}
\myindex{\CStandardLibrary!alloca()}

\newcommand{\AllocaSrcPath}{C:\textbackslash{}Program Files (x86)\textbackslash{}Microsoft Visual Studio 10.0\textbackslash{}VC\textbackslash{}crt\textbackslash{}src\textbackslash{}intel}

Intéressons-nous à la fonction \TT{alloca()}
\footnote{Avec MSVC, l'implémentation de cette fonction peut être trouvée dans les fichiers \TT{alloca16.asm} et \TT{chkstk.asm} dans \\
\TT{\AllocaSrcPath{}}}

Cette fonction fonctionne comme \TT{malloc()}, mais alloue de la mémoire directement sur la pile.
% page break added to prevent "\vref on page boundary" error. it may be dropped in future.
Le bout de mémoire ne doit pas être libéré via un appel à la fonction \TT{free()}, \\
puisque l'épilogue de fonction (\myref{sec:prologepilog}) retourne \ESP à son état initial précédant ce qui va automatiquement désallouer ce bout de mémoire.

Intéressons-nous à l'implémentation d'\TT{alloca()}.
Cette fonction décale simplement \ESP du nombre d'octets demandé vers le bas et vers le fond de la pile et définit \ESP en tant que pointeur vers la mémoire \IT{allouée}.

Essayons :

\lstinputlisting[style=customc]{patterns/02_stack/04_alloca/2_1.c}

La fonction \TT{\_snprintf()} fonctionne comme \printf, mais au lieu d'afficher le résultat sur \gls{stdout} (ex., dans un terminal ou une console), il l'écrit dans le buffer \TT{buf}. La fonction \puts copie le contenu de \TT{buf} dans \gls{stdout}. Évidemment, ces deux appels de fonctions peuvent être remplacés par un seul appel à la fonction \printf, mais nous devons illustrer l'utilisation de petit buffer.

\myparagraph{MSVC}

Compilons (MSVC 2010) :

\lstinputlisting[caption=MSVC 2010,style=customasmx86]{patterns/02_stack/04_alloca/2_2_msvc.asm}

\myindex{Compiler intrinsic}
Le seul argument d'\TT{alloca()} est passé via \EAX (au lieu de le mettre sur la pile )
\footnote{C'est parce que alloca() est plutôt une fonctionnalité intrinsèque du compilateur (\myref{sec:compiler_intrinsic}) qu'une fonction normale. Une des raisons pour lequelle nous avons besoin d'une fonction séparée au lieu de quelques instructions dans le code, est parce que l'implementation d'alloca() par \ac{MSVC} à également du code qui lit depuis la mémoire récemment allouée pour laisser l'\ac{OS} mapper la memoire physique vers la \ac{mémoire virtuelle}. Aprés l'appel à la fonction \TT{alloca()}, \ESP pointe sur un bloc de 600 octets que nous pouvons utiliser pour le tableau \TT{buf}.}

\myparagraph{GCC + \IntelSyntax}

GCC 4.4.1 fait la même chose sans effectuer d'appel à des fonctions externes :

\lstinputlisting[caption=GCC 4.7.3,style=customasmx86]{patterns/02_stack/04_alloca/2_1_gcc_intel_O3_FR.asm}

\myparagraph{GCC + \ATTSyntax}

Voyons le même code mais avec la syntaxe AT\&T :

\lstinputlisting[caption=GCC 4.7.3,style=customasmx86]{patterns/02_stack/04_alloca/2_1_gcc_ATT_O3.s}

\myindex{\ATTSyntax}
Le code est le même que le précédent.

Au fait, \INS{movl \$3, 20(\%esp)} correspond à
\INS{mov DWORD PTR [esp+20], 3} avec la syntaxe intel.
Dans la syntaxe AT\&T, le format registre+offset pour l'adressage mémoire
ressemble à \TT{offset(\%{register})}.
}
\RU{\subsubsection{x86: Функция alloca()}
\label{alloca}
\myindex{\CStandardLibrary!alloca()}

\newcommand{\AllocaSrcPath}{C:\textbackslash{}Program Files (x86)\textbackslash{}Microsoft Visual Studio 10.0\textbackslash{}VC\textbackslash{}crt\textbackslash{}src\textbackslash{}intel}

Интересен случай с функцией \TT{alloca()}
\footnote{В MSVC, реализацию функции можно посмотреть в файлах \TT{alloca16.asm} и \TT{chkstk.asm} в \\
\TT{\AllocaSrcPath{}}}. 
Эта функция работает как \TT{malloc()}, но выделяет память прямо в стеке.
Память освобождать через \TT{free()} не нужно, так как эпилог функции~(\myref{sec:prologepilog})
вернет \ESP в изначальное состояние и выделенная память просто \IT{выкидывается}.
Интересна реализация функции \TT{alloca()}.
Эта функция, если упрощенно, просто сдвигает \ESP вглубь стека на столько байт, сколько вам нужно и возвращает \ESP в качестве указателя на выделенный блок.

Попробуем:

\lstinputlisting[style=customc]{patterns/02_stack/04_alloca/2_1.c}

Функция \TT{\_snprintf()} работает так же, как и \printf, только вместо выдачи результата в \gls{stdout} (т.е. на терминал или в консоль),
записывает его в буфер \TT{buf}. Функция \puts выдает содержимое буфера \TT{buf} в \gls{stdout}. Конечно, можно было бы
заменить оба этих вызова на один \printf, но здесь нужно проиллюстрировать использование небольшого буфера.

\myparagraph{MSVC}

Компилируем (MSVC 2010):

\lstinputlisting[caption=MSVC 2010,style=customasmx86]{patterns/02_stack/04_alloca/2_2_msvc.asm}

\myindex{Compiler intrinsic}
Единственный параметр в \TT{alloca()} передается через \EAX, а не как обычно через стек
\footnote{Это потому, что alloca()~--- это не сколько функция, сколько т.н. \IT{compiler intrinsic} (\myref{sec:compiler_intrinsic})
Одна из причин, почему здесь нужна именно функция, а не несколько инструкций прямо в коде в том, что в реализации 
функции alloca() от \ac{MSVC}
есть также код, читающий из только что выделенной памяти, чтобы \ac{OS} подключила физическую память к этому региону \ac{VM}.
После вызова \TT{alloca()} \ESP указывает на блок в 600 байт, который мы можем использовать под \TT{buf}.}.

\myparagraph{GCC + \IntelSyntax}

А GCC 4.4.1 обходится без вызова других функций:

\lstinputlisting[caption=GCC 4.7.3,style=customasmx86]{patterns/02_stack/04_alloca/2_1_gcc_intel_O3_RU.asm}

\myparagraph{GCC + \ATTSyntax}

Посмотрим на тот же код, только в синтаксисе AT\&T:

\lstinputlisting[caption=GCC 4.7.3,style=customasmx86]{patterns/02_stack/04_alloca/2_1_gcc_ATT_O3.s}

\myindex{\ATTSyntax}
Всё то же самое, что и в прошлом листинге.

Кстати, \INS{movl \$3, 20(\%esp)}~--- это аналог \INS{mov DWORD PTR [esp+20], 3} в синтаксисе Intel.
Адресация памяти в виде \IT{регистр+смещение} записывается в синтаксисе AT\&T как \TT{смещение(\%{регистр})}.

}
\PTBR{\subsubsection{x86: a função alloca()}
\label{alloca}
\myindex{\CStandardLibrary!alloca()}

\newcommand{\AllocaSrcPath}{C:\textbackslash{}Program Files (x86)\textbackslash{}Microsoft Visual Studio 10.0\textbackslash{}VC\textbackslash{}crt\textbackslash{}src\textbackslash{}intel}

A função \TT{alloca()}
\footnote{No MSVC, a implementação da função pode ser encontrada nos arquivos \TT{alloca16.asm} e \TT{chkstk.asm} em \\
\TT{\AllocaSrcPath{}}}
funciona da mesma maneira que \TT{malloc()}, mas aloca memória diretamente na pilha.
O bloco de memória alocado não precisa ser limpo através da chamada da função free(),
desde que o rodapé da função (\myref{sec:prologepilog}) retorna \ESP de volta para seu estado inicial e a memória alocada é simplesmente desassociada.
Sobre como a função \TT{alloca()} é implementada, em termos simples, essa função só desloca \ESP para baixo 
(em direção ao fundo da pilha) pelo número de bytes que você precisa e define o ESP como um ponteiro para o bloco alocado.

\RU{Попробуем:}\EN{Let's try:}\PTBR{Vamos tentar:}

\lstinputlisting[style=customc]{patterns/02_stack/04_alloca/2_1.c}

A função \TT{\_snprintf()} funciona exatamente como \printf, mas ao invés de jogar o resultado em stdout
(terminal ou console, por exemplo), ela escreve no buffer buf.
A função \puts copia o conteúdo para um buf do stdout.
Lógico, essas duas chamadas de funções podem ser substituídas por um \printf, mas nós temos que ilustrar o uso pequeno do buffer.

\myparagraph{MSVC}

Vamos compilar (MSVC 2010):

\lstinputlisting[caption=MSVC 2010,style=customasmx86]{patterns/02_stack/04_alloca/2_2_msvc.asm}

\myindex{Compiler intrinsic}
O único argumento da função alloca() é passado via EAX (ao invés de ser empurrado na pilha)
\footnote{Isso é devido ao fato de que alloca() é mais nativa do compilador do que uma função normal (\myref{sec:compiler_intrinsic}).
Um dos motivos que se faz necessário o separamento da função ao invés de um pouco de linhas de código no código,
é porque a implementação da alloca() no MSVC também tem código que é lido da memória que acabou de ser alocada,
para deixar o sistema operacional mapear a memória física para essa região da memória virtual.}.

Depois da chamada de \TT{alloca()}, \ESP aponta para o bloco de 600 bytes que nós podemos usar como memória para o array.

\myparagraph{GCC + \IntelSyntax}

\PTBRph{}

}
\ITA{\subsubsection{x86: la funzione alloca() }
\label{alloca}
\myindex{\CStandardLibrary!alloca()}

\newcommand{\AllocaSrcPath}{C:\textbackslash{}Program Files (x86)\textbackslash{}Microsoft Visual Studio 10.0\textbackslash{}VC\textbackslash{}crt\textbackslash{}src\textbackslash{}intel}

Vale la pena esaminare la funzione \TT{alloca()}
\footnote{In MSVC, l'implementazione della funzione si trova in \TT{alloca16.asm} e \TT{chkstk.asm} in \\
\TT{\AllocaSrcPath{}}}.
Questa funzione opera come \TT{malloc()}, ma alloca memoria direttamente nello stack.
% page break added to prevent "\vref on page boundary" error. it may be dropped in future.
Il pezzo di memoria allocato non necessita di essere liberato tramite una chiamata alla funzione \TT{free()} function call, \\
poiche' l'epilogo della funzione (\myref{sec:prologepilog}) ripristina \ESP al suo valore iniziale e la memoria allocata viene semplicemente \IT{abbandonata}.
Vale anche la pena notare come e' implementata la funzione \TT{alloca()}.
In termini semplici, questa funzione shifta \ESP in basso, verso la base dello stack, per il numero di byte necessari e setta \ESP  
per puntare al blocco \IT{allocato}.

Proviamo:

\lstinputlisting[style=customc]{patterns/02_stack/04_alloca/2_1.c}

La funzione \TT{\_snprintf()} opera come \printf, ma invece di inviare il risultato a \gls{stdout} (es. al terminale o console),
lo scrive nel buffer \TT{buf}. La funzione \puts copia il contenuto di \TT{buf} in \gls{stdout}.
Ovviamente questo due chiamate potrebbero essere rimpiazzate da una sola chiamata a \printf, ma in questo caso era necessario per illustrare
l'uso di un piccolo buffer.

\myparagraph{MSVC}

Compiliamo (MSVC 2010):

\lstinputlisting[caption=MSVC 2010,style=customasmx86]{patterns/02_stack/04_alloca/2_2_msvc.asm}

\myindex{Compiler intrinsic}
L'unico argomento di \TT{alloca()} e' passato tramite il registro \EAX (anziche' metterlo nello stack)
\footnote{Questo perche' alloca() e' una "compiler intrinsic" (\myref{sec:compiler_intrinsic}) piuttosto che una funzione normale.
Una delle ragioni per cui abbiamo bisogno di una funzione separata, invece di un paio di istruzioni nel codice, e' che
l'implementazione di alloca() di \ac{MSVC} ha anche del codice che legge dalla memoria appena llocata, per far si che l'\ac{OS} effettui il mapping
della memoria fisica in questa regione della \ac{VM}.
Dopo la chiamata a \TT{alloca()} , \ESP punta al blocco di 600 byte, ed e' possibile utilizzarlo come memoria per l'array \TT{buf}.}.

\myparagraph{GCC + \IntelSyntax}

GCC 4.4.1 fa lo stesso senza chiamare funzioni esterne:

\lstinputlisting[caption=GCC 4.7.3,style=customasmx86]{patterns/02_stack/04_alloca/2_1_gcc_intel_O3_EN.asm}

\myparagraph{GCC + \ATTSyntax}

Esaminiamo lo stesso codice, ma in sintassi AT\&T:

\lstinputlisting[caption=GCC 4.7.3,style=customasmx86]{patterns/02_stack/04_alloca/2_1_gcc_ATT_O3.s}

\myindex{\ATTSyntax}
The code e' uguale a quello del listato precedente.

A proposito, \INS{movl \$3, 20(\%esp)} corrisponde a \INS{mov DWORD PTR [esp+20], 3} in sintassi Intel.
In sintassi AT\&T, il formato registro+offset per indirizzare memoria appare come \TT{offset(\%{register})}.
}
\DE{\subsubsection{x86: alloca() Funktion}
\label{alloca}
\myindex{\CStandardLibrary!alloca()}

\newcommand{\AllocaSrcPath}{C:\textbackslash{}Program Files (x86)\textbackslash{}Microsoft Visual Studio 10.0\textbackslash{}VC\textbackslash{}crt\textbackslash{}src\textbackslash{}intel}

Es macht Sinn einen Blick auf die \TT{alloca()} Funktion zu werfen
\footnote{In MSVC, kann die Funktions Implementierung in \TT{alloca16.asm} und \TT{chkstk.asm} in \\
\TT{\AllocaSrcPath{}}} gefunden werden.
Diese Funktion arbeitet wie \TT{malloc()}, nur das sie Speicher direkt auf dem Stack bereit stellt.

Der allozierte Speicher Chunk muss nicht wieder mit \TT{free()} freigegeben werden, weil
der Funktions Epilog (\myref{sec:prologepilog}) das \ESP Register wieder in seinen ursprünglichen 
Zustand versetzt und der allozierte Speicher wird einfach \IT{verworfen}. 
Es macht Sinn sich anzuschauen wie \TT{alloca()} implementiert ist.
Mit einfachen Begriffen erklärt, diese Funktion verschiebt \ESP in Richtung des Stack ende mit der 
Anzahl der Bytes die alloziert werden müssen und setzt \ESP als einen Zeiger auf den \IT{allozierten} block.

Beispiel:

\lstinputlisting{patterns/02_stack/04_alloca/2_1.c}


Die \TT{\_snprintf()} Funktion arbeitetet genau wie \printf, nur statt die Ergebnisse nach \gls{stdout} aus zu geben ( bsp. auf dem Terminal oder Konsole), schreibt sie in den \TT{buf} buffer. Die Funktion \puts kopiert den Inhalt aus \TT{buf} nach \gls{stdout}. Sicher könnte man die beiden Funktions Aufrufe könnten durch einen \printf Aufruf ersetzt werden, aber wir sollten einen genaueren Blick auf die Benutzung kleiner Buffer anschauen.

\myparagraph{MSVC}

Compilierung mit MSVC 2010: 

\lstinputlisting[caption=MSVC 2010]{patterns/02_stack/04_alloca/2_2_msvc.asm}

\myindex{Compiler intrinsisch}
Das einzige \\TT{alloca()} Argument wird über \EAX übergeben (anstatt es erst auf den Stack zu pushen)
\footnote{Das liegt daran, das alloca() Verhalten Compiler intrinsisch bestimmt (\myref{sec:compiler_intrinsic}) im Gegensatz zu einer normalen Funktion. Einer der Gründe dafür das man braucht eine separate Funktion braucht, statt ein paar Code Instruktionen im Code,  ist weil die \ac{MSCV} alloca() Implementierung ebenfalls Code hat welcher aus dem gerade allozierten Speicher gelesen wird. Damit in Folge das \ac{Betriebssystem} physikalischen Speicher in dieser \ac{VM} Region zu allozieren. Nach dem \TT{alloca()} Aufruf, zeigt \ESP auf den Block von 600 Bytes der nun als Speicher für das \TT{buf} Array dienen kann.}.

\myparagraph{GCC + \IntelSyntax}

GCC 4.4.1 macht das selbe, aber ohne externe Funktions aufrufe.

\lstinputlisting[caption=GCC 4.7.3]{patterns/02_stack/04_alloca/2_1_gcc_intel_O3_EN.asm}

\myparagraph{GCC + \ATTSyntax}

Nun der gleiche Code, aber in AT\&T Syntax:

\lstinputlisting[caption=GCC 4.7.3]{patterns/02_stack/04_alloca/2_1_gcc_ATT_O3.s}

\myindex{\ATTSyntax}
Der Code ist der gleiche wie im vorherigen listig.

Übrigens, \INS{movl \$3, 20(\%esp)} in AT\&T Syntax wird zu \
\INS{mov DWORD PTR [esp+20], 3} in Intel-syntax.
In der AT\&T Syntax, sehen Register+Offset Formatierungen einer Adresse so aus:
\TT{offset(\%{register})}.
}
\PL{\subsubsection{x86: Funkcja alloca()}
\label{alloca}
\myindex{\CStandardLibrary!alloca()}

\newcommand{\AllocaSrcPath}{C:\textbackslash{}Program Files (x86)\textbackslash{}Microsoft Visual Studio 10.0\textbackslash{}VC\textbackslash{}crt\textbackslash{}src\textbackslash{}intel}

Przypadek z funkcją \TT{alloca()} jest całkiem ciekawy
\footnote{W MSVC, implementację funkcji można podejrzeć w plikach \TT{alloca16.asm} i \TT{chkstk.asm} w \\
\TT{\AllocaSrcPath{}}}. 
Ta funkcja działa jak \TT{malloc()}, ale przydziela pamięć od razu na stosie.
Nie potrzebne jest zwalnianie pamięci \TT{free()}, dlatego że epilog funkcji~(\myref{sec:prologepilog})
przywróci \ESP do stanu początkowego i przeznaczona pamięć \IT{zostaje wyrzucona}.
Ciekawa jest również realizacja tej funkcji.
Ona, w skrócie, po prostu przesuwa \ESP wgłąb stosu i zwraca \ESP jako wskaźnik na przydzielony obszar.

Spróbujmy:

\lstinputlisting[style=customc]{patterns/02_stack/04_alloca/2_1.c}

Funkcja \TT{\_snprintf()} działa tak samo, jak i \printf, tylko zamiast wyprowadzenia wyniku na wyjście standardowe \gls{stdout} (czyli do terminalu),
ona go zapisuje do buforu \TT{buf}. Funkcja \puts, z kolei, wyrzuca zawartość buforu \TT{buf} na \gls{stdout}. Oczywiście można by było tutaj zamienić
tę parę instrukcji na \printf, ale tutaj chcielibyśmy zobaczyć wykorzystanie niewielkiego buforu.

\myparagraph{MSVC}

Skompilujmy (MSVC 2010):

\lstinputlisting[caption=MSVC 2010,style=customasmx86]{patterns/02_stack/04_alloca/2_2_msvc.asm}

\myindex{Compiler intrinsic}
Jedyny parametr \TT{alloca()} jest przekazywany przez \EAX, a nie jak zwykle, przez stos
\footnote{To dlatego, że alloca()~--- to nie tyle co funkcja, a raczej \IT{compiler intrinsic} (\myref{sec:compiler_intrinsic})
Jedną z przyczyn, flaczego tu potrzeba funkcji, a nie kilku instrukcji w samym kodzie, polega na tym, że w realizacji
funkcji alloca() w \ac{MSVC}
zawarty również kod, czytający z dopiero co przydzielonej pamięci po to, żeby \ac{OS} zaalokowała pamięć fizyczną dla tego obszaru \ac{VM}.
Po wywołaniu \TT{alloca()} \ESP wskazuje na blok o długości 600 bajtów, z którego możemy korzystać na potrzeby naszego \TT{buf}.}.

\myparagraph{GCC + \IntelSyntax}

GCC 4.4.1 nie wywołuje innych funkcji:

\lstinputlisting[caption=GCC 4.7.3,style=customasmx86]{patterns/02_stack/04_alloca/2_1_gcc_intel_O3_RU.asm}

\myparagraph{GCC + \ATTSyntax}

Spójrzmy na ten sam kod w syntaksie AT\&T:

\lstinputlisting[caption=GCC 4.7.3,style=customasmx86]{patterns/02_stack/04_alloca/2_1_gcc_ATT_O3.s}

\myindex{\ATTSyntax}
Wygląda to tak samo jak i poprzedni listing.

A propos, \INS{movl \$3, 20(\%esp)}~--- jest analogiem do \INS{mov DWORD PTR [esp+20], 3} w syntaksie Intel.
Adresowanie pamięci typu \IT{rejestr+przesunięcie} jest zapisywane w syntaksie AT\&T jako \TT{przesunięcie(\%{rejestr})}.


}
\JPN{\subsubsection{x86: alloca()関数}
\label{alloca}
\myindex{\CStandardLibrary!alloca()}

\newcommand{\AllocaSrcPath}{C:\textbackslash{}Program Files (x86)\textbackslash{}Microsoft Visual Studio 10.0\textbackslash{}VC\textbackslash{}crt\textbackslash{}src\textbackslash{}intel}

\TT{alloca()}関数に注目することは重要です
\footnote{MSVCでは、関数の実装は\TT{\AllocaSrcPath{}}の\TT{alloca16.asm} と \TT{chkstk.asm}にあります}
この関数は\TT{malloc()}のように動作しますが、スタックに直接メモリを割り当てます。 
% page break added to prevent "\vref on page boundary" error. it may be dropped in future.
関数のエピローグ(\myref{sec:prologepilog})は \ESP を初期状態に戻し、割り当てられたメモリは単に\IT{破棄}されるため、
割り当てられたメモリチャンクは\TT{free()}関数呼び出しで解放する必要はありません。 \TT{alloca()}がどのように実装されているかは注目に値する。
簡単に言えば、この関数は必要なバイト数だけスタック底部に向かって \ESP を下にシフトさせ、\IT{割り当てられた}ブロックへのポインタとして \ESP を設定します。

やってみましょう。

\lstinputlisting[style=customc]{patterns/02_stack/04_alloca/2_1.c}

\TT{\_snprintf()}関数は \printf と同じように動作しますが、結果を\gls{stdout}(ターミナルやコンソールなど)
にダンプする代わりに、\TT{buf}バッファに書き込みます。 \puts 関数は\TT{buf}の内容を\gls{stdout}にコピーします。
もちろん、これらの2つの関数呼び出しは1つの \printf 呼び出しで置き換えることができますが、小さなバッファの使用法を説明する必要があります。

\myparagraph{MSVC}

コンパイルしてみましょう(MSVC 2010で)

\lstinputlisting[caption=MSVC 2010,style=customasmx86]{patterns/02_stack/04_alloca/2_2_msvc.asm}

\myindex{Compiler intrinsic}
\TT{alloca()}の唯一の引数は \EAX 経由で(スタックにプッシュするのではなく)渡されます。
\footnote{alloca()はコンパイラ組み込み関数((\myref{sec:compiler_intrinsic}))ではなく、通常の関数です。 
\ac{MSVC}のalloca()の実装には、割り当てられたメモリから読み込むコードが含まれているため、\ac{OS}が物理メモリをVM領域にマップするために、
コード内の命令が数個ではなく別々の関数を必要とする理由の1つです。 \TT{alloca()}呼び出しの後、ESPは600バイトのブロックを指し、\TT{buf}配列のメモリとして使用できます。}

\myparagraph{GCC + \IntelSyntax}

GCC 4.4.1は、外部関数を呼び出すことなく同じことを行います

\lstinputlisting[caption=GCC 4.7.3,style=customasmx86]{patterns/02_stack/04_alloca/2_1_gcc_intel_O3_EN.asm}

\myparagraph{GCC + \ATTSyntax}

同じコードをAT\&T構文で見てみましょう

\lstinputlisting[caption=GCC 4.7.3,style=customasmx86]{patterns/02_stack/04_alloca/2_1_gcc_ATT_O3.s}

\myindex{\ATTSyntax}
コードは前のリストと同じです。

ちなみに、\INS{movl \$3, 20(\%esp)}は、
Intel構文の\INS{mov DWORD PTR [esp+20], 3}に対応しています。 
AT\&Tの構文では、アドレス指定メモリのレジスタ+オフセット形式は
\TT{offset(\%{register})}のように見えます。
}

\subsubsection{(Windows) SEH}
\myindex{Windows!Structured Exception Handling}

\ifdefined\RUSSIAN
В стеке хранятся записи \ac{SEH} для функции (если они присутствуют).
Читайте больше о нем здесь: (\myref{sec:SEH}).
\fi % RUSSIAN

\ifdefined\ENGLISH
\ac{SEH} records are also stored on the stack (if they are present).
Read more about it: (\myref{sec:SEH}).
\fi % ENGLISH

\ifdefined\BRAZILIAN
\ac{SEH} também são guardados na pilha (se estiverem presentes).
\PTBRph{}: (\myref{sec:SEH}).
\fi % BRAZILIAN

\ifdefined\ITALIAN
I record \ac{SEH}, se presenti, sono anch'essi memorizzati nello stack.
Maggiori informazioni qui: (\myref{sec:SEH}).
\fi % ITALIAN

\ifdefined\FRENCH
Les enregistrements \ac{SEH} sont aussi stockés dans la pile (s'ils sont présents).
Lire à ce propos: (\myref{sec:SEH}).
\fi % FRENCH


\ifdefined\POLISH
Na stosie są przechowywane wpisy \ac{SEH} dla funkcji (jeśli są one obecne).
Więcej o tym tutaj: (\myref{sec:SEH}).
\fi % POLISH

\ifdefined\JAPANESE
\ac{SEH}レコードはスタックにも格納されます(存在する場合)。
それについてもっと読む:(\myref{sec:SEH})
\fi % JAPANESE

\ifdefined\ENGLISH
\subsubsection{Buffer overflow protection}

More about it here~(\myref{subsec:bufferoverflow}).
\fi

\ifdefined\RUSSIAN
\subsubsection{Защита от переполнений буфера}

Здесь больше об этом~(\myref{subsec:bufferoverflow}).
\fi

\ifdefined\BRAZILIAN
\subsubsection{Proteção contra estouro de buffer}

Mais sobre aqui~(\myref{subsec:bufferoverflow}).
\fi

\ifdefined\ITALIAN
\subsubsection{Protezione da buffer overflow}

Maggiori informazioni qui~(\myref{subsec:bufferoverflow}).
\fi

\ifdefined\FRENCH
\subsubsection{Protection contre les débordements de tampon}

Lire à ce propos~(\myref{subsec:bufferoverflow}).
\fi


\ifdefined\POLISH
\subsubsection{Metody zabiezpieczenia przed przepełnieniem stosu}

Więcej o tym tutaj~(\myref{subsec:bufferoverflow}).
\fi

\ifdefined\JAPANESE
\subsubsection{バッファオーバーフロー保護}

詳細はこちら~(\myref{subsec:bufferoverflow})
\fi

\subsubsection{Désallocation automatique de données dans la pile}

Peut-être que la raison pour laquelle les variables locales et les enregistrements SEH sont stockés dans la
pile est qu'ils sont automatiquement libérés quand la fonction se termine en utilisant simplement une
instruction pour corriger la position du pointeur de pile (souvent \ADD).
Les arguments de fonction sont aussi désalloués automatiquement à la fin de la fonction.
À l'inverse, toutes les données allouées sur le \IT{heap} doivent être désallouées de façon explicite.

% sections
%\subsection{ Disposition typique de la pile }

Avant l'exécution de la première instruction d'une fonction, la pile ressemble généralement à ceci:

\input{patterns/02_stack/stack_layout}
 % TBT
\EN{\subsection{Noise in stack}

Often in this book \q{noise} or \q{garbage} values in the stack or memory are mentioned.
Where do they come from?
These are what has been left there after other functions' executions.
Short example:

\lstinputlisting[style=customc]{patterns/02_stack/08_noise/st.c}

Compiling \dots

\lstinputlisting[caption=\NonOptimizing MSVC 2010,style=customasmx86]{patterns/02_stack/08_noise/st.asm}

The compiler will grumble a little bit\dots

\begin{lstlisting}
c:\Polygon\c>cl st.c /Fast.asm /MD
Microsoft (R) 32-bit C/C++ Optimizing Compiler Version 16.00.40219.01 for 80x86
Copyright (C) Microsoft Corporation.  All rights reserved.

st.c
c:\polygon\c\st.c(11) : warning C4700: uninitialized local variable 'c' used
c:\polygon\c\st.c(11) : warning C4700: uninitialized local variable 'b' used
c:\polygon\c\st.c(11) : warning C4700: uninitialized local variable 'a' used
Microsoft (R) Incremental Linker Version 10.00.40219.01
Copyright (C) Microsoft Corporation.  All rights reserved.

/out:st.exe
st.obj
\end{lstlisting}

But when we run the compiled program \dots

\begin{lstlisting}
c:\Polygon\c>st
1, 2, 3
\end{lstlisting}

Oh, what a weird thing! We did not set any variables in \TT{f2()}. 
These are \q{ghosts} values, which are still in the stack.

\clearpage
Let's load the example into \olly:

\begin{figure}[H]
\centering
\myincludegraphics{patterns/02_stack/08_noise/olly1.png}
\caption{\olly: \TT{f1()}}
\label{fig:stack_noise_olly1}
\end{figure}

When \TT{f1()} assigns the variables $a$, $b$ and $c$, their values are stored at the address \TT{0x1FF860} and so on.

\clearpage
And when \TT{f2()} executes:

\begin{figure}[H]
\centering
\myincludegraphics{patterns/02_stack/08_noise/olly2.png}
\caption{\olly: \TT{f2()}}
\label{fig:stack_noise_olly2}
\end{figure}

... $a$, $b$ and $c$ of \TT{f2()} are located at the same addresses!
No one has overwritten the values yet, so at that point they are still untouched.
So, for this weird situation to occur, several functions have to be called one after another and
\ac{SP} has to be the same at each function entry (i.e., they have the same number
of arguments). Then the local variables will be located at the same positions in the stack.
Summarizing, all values in the stack (and memory cells in general) have values left there from previous function executions.
They are not random in the strict sense, but rather have unpredictable values.
Is there another option?
It would probably be possible to clear portions of the stack before each function execution,
but that's too much extra (and unnecessary) work.

\subsubsection{MSVC 2013}

The example was compiled by MSVC 2010.
But the reader of this book made attempt to compile this example in MSVC 2013, ran it, and got all 3 numbers reversed:%

\begin{lstlisting}
c:\Polygon\c>st
3, 2, 1
\end{lstlisting}

Why?
I also compiled this example in MSVC 2013 and saw this:


\begin{lstlisting}[caption=MSVC 2013,style=customasmx86]
_a$ = -12	; size = 4
_b$ = -8	; size = 4
_c$ = -4	; size = 4
_f2	PROC

...

_f2	ENDP

_c$ = -12	; size = 4
_b$ = -8	; size = 4
_a$ = -4	; size = 4
_f1	PROC

...

_f1	ENDP
\end{lstlisting}

Unlike MSVC 2010, MSVC 2013 allocated a/b/c variables in function \TT{f2()} in reverse order.%
And this is completely correct, because \CCpp standards has no rule, in which order local variables must be allocated in the local stack, if at all.
The reason of difference is because MSVC 2010 has one way to do it, and MSVC 2013 has supposedly something changed inside of compiler guts, so it behaves slightly different.

}
\RU{\subsection{Мусор в стеке}

Часто в этой книге говорится о \q{шуме} или \q{мусоре} в стеке или памяти.
Откуда он берется?
Это то, что осталось там после исполнения предыдущих функций.

Короткий пример:

\lstinputlisting[style=customc]{patterns/02_stack/08_noise/st.c}

Компилируем\dots

\lstinputlisting[caption=\NonOptimizing MSVC 2010,style=customasmx86]{patterns/02_stack/08_noise/st.asm}

Компилятор поворчит немного\dots

\begin{lstlisting}
c:\Polygon\c>cl st.c /Fast.asm /MD
Microsoft (R) 32-bit C/C++ Optimizing Compiler Version 16.00.40219.01 for 80x86
Copyright (C) Microsoft Corporation.  All rights reserved.

st.c
c:\polygon\c\st.c(11) : warning C4700: uninitialized local variable 'c' used
c:\polygon\c\st.c(11) : warning C4700: uninitialized local variable 'b' used
c:\polygon\c\st.c(11) : warning C4700: uninitialized local variable 'a' used
Microsoft (R) Incremental Linker Version 10.00.40219.01
Copyright (C) Microsoft Corporation.  All rights reserved.

/out:st.exe
st.obj
\end{lstlisting}

Но когда мы запускаем\dots

\begin{lstlisting}
c:\Polygon\c>st
1, 2, 3
\end{lstlisting}

Ох. Вот это странно. Мы ведь не устанавливали значения никаких переменных в \TT{f2()}. 
Эти значения --- это \q{привидения}, которые всё ещё в стеке.

\clearpage
Загрузим пример в \olly:

\begin{figure}[H]
\centering
\myincludegraphics{patterns/02_stack/08_noise/olly1.png}
\caption{\olly: \TT{f1()}}
\label{fig:stack_noise_olly1}
\end{figure}

Когда \TT{f1()} заполняет переменные $a$, $b$ и $c$ они сохраняются по адресу \TT{0x1FF860}, итд.

\clearpage
А когда исполняется \TT{f2()}:

\begin{figure}[H]
\centering
\myincludegraphics{patterns/02_stack/08_noise/olly2.png}
\caption{\olly: \TT{f2()}}
\label{fig:stack_noise_olly2}
\end{figure}

... $a$, $b$ и $c$ в функции \TT{f2()} находятся по тем же адресам!
Пока никто не перезаписал их, так что они здесь в нетронутом виде.
Для создания такой странной ситуации несколько функций должны исполняться друг за другом
и \ac{SP} должен быть одинаковым при входе в функции, т.е. у функций должно быть равное количество
аргументов). Тогда локальные переменные будут расположены в том же месте стека.
Подводя итоги, все значения в стеке (да и памяти вообще) это значения оставшиеся от 
исполнения предыдущих функций.
Строго говоря, они не случайны, они скорее непредсказуемы.
А как иначе?
Можно было бы очищать части стека перед исполнением каждой функции,
но это слишком много лишней (и ненужной) работы.

\subsubsection{MSVC 2013}

Этот пример был скомпилирован в MSVC 2010.
Но один читатель этой книги сделал попытку скомпилировать пример в MSVC 2013, запустил и увидел 3 числа в обратном порядке:

\begin{lstlisting}
c:\Polygon\c>st
3, 2, 1
\end{lstlisting}

Почему?
Я также попробовал скомпилировать этот пример в MSVC 2013 и увидел это:

\begin{lstlisting}[caption=MSVC 2013,style=customasmx86]
_a$ = -12	; size = 4
_b$ = -8	; size = 4
_c$ = -4	; size = 4
_f2	PROC

...

_f2	ENDP

_c$ = -12	; size = 4
_b$ = -8	; size = 4
_a$ = -4	; size = 4
_f1	PROC

...

_f1	ENDP
\end{lstlisting}

В отличии от MSVC 2010, MSVC 2013 разместил переменные a/b/c в функции \TT{f2()} в обратном порядке.
И это полностью корректно, потому что в стандартах \CCpp нет правила, в каком порядке локальные переменные должны быть размещены в локальном стеке, если вообще.
Разница есть из-за того что MSVC 2010 делает это одним способом, а в MSVC 2013, вероятно, что-то немного изменили во внутренностях компилятора, так что он ведет себя слегка иначе.

}
\ITA{\subsection{Rumore nello stack}

In questo libro si fa spesso riferimento a \q{rumore} o \q{spazzatura} (garbage) nello stack o in memoria.

Da dove arrivano?
Sono cio' che resta dopo l'esecuzione di altre funzioni.
Un piccolo esempio:

\lstinputlisting[style=customc]{patterns/02_stack/08_noise/st.c}

Compilando si ottiene:

\lstinputlisting[caption=\NonOptimizing MSVC 2010,style=customasm]{patterns/02_stack/08_noise/st.asm}

Il compilatore si lamentera' un pochino\dots

\begin{lstlisting}
c:\Polygon\c>cl st.c /Fast.asm /MD
Microsoft (R) 32-bit C/C++ Optimizing Compiler Version 16.00.40219.01 for 80x86
Copyright (C) Microsoft Corporation.  All rights reserved.

st.c
c:\polygon\c\st.c(11) : warning C4700: uninitialized local variable 'c' used
c:\polygon\c\st.c(11) : warning C4700: uninitialized local variable 'b' used
c:\polygon\c\st.c(11) : warning C4700: uninitialized local variable 'a' used
Microsoft (R) Incremental Linker Version 10.00.40219.01
Copyright (C) Microsoft Corporation.  All rights reserved.

/out:st.exe
st.obj
\end{lstlisting}

Ma quando avvieremo il programma \dots

\begin{lstlisting}
c:\Polygon\c>st
1, 2, 3
\end{lstlisting}

Oh, che cosa strana! Non abbiamo impostato il valore di alcuna variabile in \TT{f2()}. 
Si tratta di valori \q{fantasma} , che si trovano ancora nello stack.

\clearpage
Carichiamo l'esempio in \olly:

\begin{figure}[H]
\centering
\myincludegraphics{patterns/02_stack/08_noise/olly1.png}
\caption{\olly: \TT{f1()}}
\label{fig:stack_noise_olly1}
\end{figure}

Quando \TT{f1()} assegna le variabili $a$, $b$ e $c$, i loro valori sono memorizzati all'indirizzo \TT{0x1FF860} e seguenti.

\clearpage
E quando viene eseguita \TT{f2()}:

\begin{figure}[H]
\centering
\myincludegraphics{patterns/02_stack/08_noise/olly2.png}
\caption{\olly: \TT{f2()}}
\label{fig:stack_noise_olly2}
\end{figure}

... $a$, $b$ e $c$ di \TT{f2()} si trovano agli stessi indirizzi!
Nessuno ha ancora sovrascritto quei valori, e a quel punto restano intatti.
Quindi, affinche' questa strana situazione si verifichi, piu' funzioni devono essere chiamate una dopo l'altra e
\ac{SP} deve essere uguale ad ogni ingresso nella funzione (ovvero le funzioni devono avere lo stesso numero di argomenti).
A quel punto le variabili locali si troveranno nelle stesse posizioni nello stack.
Per riassumere, tutti i valori nello stack (e nelle celle di memoria in generale) hanno valori lasciati li' dall'esecuzione di funzioni precedenti.
Non sono letteralmente randomici, piuttosto hanno valori non predicibili.
C'e' un'altra opzione?
Sarebbe possibile ripulire porzioni dello stack prima di ogni esecuzione di una funzione, ma sarebbe un lavoro extra inutile.

\subsubsection{MSVC 2013}

L'esempio e' stato compilato con MSVC 2010.
Un lettore di questo libro ha provato a compilare l'esempio con MSVC 2013, lo ha eseguito, ed ha ottenuto i 3 numeri in ordine inverso:%

\begin{lstlisting}
c:\Polygon\c>st
3, 2, 1
\end{lstlisting}

Perche'?
Ho compilato anche io l'esempio in MSVC 2013 ed ho visto questo:


\begin{lstlisting}[caption=MSVC 2013,style=customasm]
_a$ = -12	; size = 4
_b$ = -8	; size = 4
_c$ = -4	; size = 4
_f2	PROC

...

_f2	ENDP

_c$ = -12	; size = 4
_b$ = -8	; size = 4
_a$ = -4	; size = 4
_f1	PROC

...

_f1	ENDP
\end{lstlisting}

Contrariamente a MSVC 2010, MSVC 2013 ha allocato le variabili a/b/c nella funzione \TT{f2()} in ordine inverso.%
E cio' e' del tutto corretto, perche' lo standard \CCpp non ha una regola che definisce in quale ordine le variabili locali devono essere allocate nello stack.
La ragione per cui si presenta questa differenza e' che MSVC 2010 ha un solo modo per farlo, mentre MSVC 2013 ha probabilmente subito modifiche all'interno del compilatore, e si comporta quindi in modo leggermente diverso. 
}
\DE{\subsection{Rauschen auf dem Stack}

Oft wird in diesem Buch von \q{rauschen} oder \q{garbage} Werten im Bezug
auf den Stack gesprochen. Woher kommen diese Werte? Das sind Überbleibsel der
Ausführung von anderen Funktionen. Zum Beispiel: 

\lstinputlisting{patterns/02_stack/08_noise/st.c}

Compilieren \dots

\lstinputlisting[caption=\NonOptimizing MSVC 2010]{patterns/02_stack/08_noise/st.asm}

Hier wird sich der Compiler ein bisschen beschweren\dots

\begin{lstlisting}
c:\Polygon\c>cl st.c /Fast.asm /MD
Microsoft (R) 32-bit C/C++ Optimizing Compiler Version 16.00.40219.01 for 80x86
Copyright (C) Microsoft Corporation.  All rights reserved.

st.c
c:\polygon\c\st.c(11) : warning C4700: uninitialized local variable 'c' used
c:\polygon\c\st.c(11) : warning C4700: uninitialized local variable 'b' used
c:\polygon\c\st.c(11) : warning C4700: uninitialized local variable 'a' used
Microsoft (R) Incremental Linker Version 10.00.40219.01
Copyright (C) Microsoft Corporation.  All rights reserved.

/out:st.exe
st.obj
\end{lstlisting}

Aber wenn wir das compilierte Programm laufen lassen\dots

\begin{lstlisting}
c:\Polygon\c>st
1, 2, 3
\end{lstlisting}

sieh an! Wir haben keine Variablen gesetzt in \TT{f2()}.
Das sind \q{Geister} Werte, welche noch immer auf dem Stack rumliegen.

\clearpage
Lasst uns das Beispiel in \olly laden:

\begin{figure}[H]
\centering
\myincludegraphics{patterns/02_stack/08_noise/olly1.png}
\caption{\olly: \TT{f1()}}
\label{fig:stack_noise_olly1}
\end{figure}

Wenn \TT{f1()} den Variablen $a$, $b$ und $c$ ihre Werte zuordnet, wird ihre Adresse bei \TT{0x1FF860} gespeichert und so weiter. 

\clearpage
Und wenn \TT{f2()} ausgeführt wird:

\begin{figure}[H]
\centering
\myincludegraphics{patterns/02_stack/08_noise/olly2.png}
\caption{\olly: \TT{f2()}}
\label{fig:stack_noise_olly2}
\end{figure}

... liegen $a$, $b$ und $c$ von \TT{f2()}  an den gleichen Adressen!
Nichts hat bis jetzt Ihre Werte überschrieben und sie sind bisher
unberührt geblieben. Also, damit diese seltsame Situation eintritt, müssen
mehrere Funktionen nacheinander aufgerufen werden und \ac{SP} muss gleich 
sein für jede Funktions Instruktion ( z.B. die Funktionen haben die 
gleiche Anzahl an Argumenten). Dann werden die lokalen Variablen an den
gleichen Positionen im Stack liegen. Zusammen fassend kann man sagen, alle
Werte auf dem Stack (und Speicherzellen im allgemeinen) beinhalten Werte
von vorhergehenden Funktions aufrufen. Diese Werte sind nicht zufällig 
im klassischem Sinn, eher unvorhersehbar. Es wäre wahrscheinlich möglich 
Teile des Stacks auf zu räumen vor jedem Funktions Aufruf, aber das wäre
zu viel zusätzliche (und unnötige) Arbeit. 

\subsubsection{MSVC 2013}

Das Beispiel wurde compiliert mit dem MSVC 2010 Compiler.
Allerdings haben die Leser dieses Buch auch schon geschafft das Beispiel mit MSVC 2013 zu compilieren, sie haben es geschafft es zum laufen zu bringen und alle drei Nummern zu reversen.

\begin{lstlisting}
c:\Polygon\c>st
3, 2, 1
\end{lstlisting}

Warum? 
Ich habe das Beispiel auch mit MSCV 2013 compiliert und habe folgendes beobachtet:

\begin{lstlisting}[caption=MSVC 2013]
_a$ = -12	; size = 4
_b$ = -8	; size = 4
_c$ = -4	; size = 4
_f2	PROC

...

_f2	ENDP

_c$ = -12	; size = 4
_b$ = -8	; size = 4
_a$ = -4	; size = 4
_f1	PROC

...

_f1	ENDP
\end{lstlisting}

Im Gegensatz zu MSVC 2010, alloziert MSCV 2013 die Variablen in der Funktion \TT{f2()} in umgekehrter Reihenfolge.%
Was auch vollkommen Korrekt ist, weil es im \CCpp Standard keine Vorschriften gibt, in welcher Reihenfolge Variablen auf dem Stack alloziert werden müssen. Der Grund für den Unterschied liegt daran das MSCV 2010 eine Methode genutzt hat um  die allozierung durch zu führen und in MSCV 2013 wurde scheinbar eine anpassung im Compiler inneren gemacht, so das sich MSCV 2013 leicht anders verhält.

}

\subsection{\Exercise}

\begin{itemize}
	\item \url{http://challenges.re/27}
\end{itemize}


