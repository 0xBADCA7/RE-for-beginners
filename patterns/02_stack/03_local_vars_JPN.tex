\subsubsection{ローカル変数記憶域}

関数は、スタックの底に向かって\gls{stack pointer}を減らすだけで、
ローカル変数のためにスタックに領域を割り当てることができます。

% I think here, "stack bottom" means the lowest address in the stack space,
% but the reader might also think it means towards the top of the stack space,
% like in a pop, so you might change "towards the stack bottom" to
% "towards the lowest address of the stack", or just take it out,
% since "decreasing" also suggests that.

したがって、どれだけ多くのローカル変数が定義されていても、非常に高速です。 
スタックにローカル変数を格納する必要もありません。 
あなたは好きな場所にローカル変数を格納することができますが、
伝統的にはこれがどのように行われています。
