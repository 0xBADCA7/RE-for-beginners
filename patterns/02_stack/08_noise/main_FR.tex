\subsection{Bruit dans la pile}
\label{bruit_dans_la_pile}

\epigraph{Quand quelqu'un dit que quelques chose est aléatoire,
  ce que cela signifie en pratique c'est qu'il n'est pas capable de
  voir les régularités de cette chose}{Stephen Wolfram, A New Kind of Science.}

Dans ce livre les valeurs dites \q{bruité} ou \q{poubelle} présente dans la pile ou dans la mémoire sont souvent mentionner.

D'où viennent-elles ?
Ces valeurs ont été laissées sur la pile après l'exécution de fonction précédente.
Par exemple: 

\lstinputlisting[style=customc]{patterns/02_stack/08_noise/st.c}

Compilons \dots

\lstinputlisting[caption=\NonOptimizing MSVC 2010,style=customasmx86]{patterns/02_stack/08_noise/st.asm}

Le compilateur va rouspéter un peu\dots 

\begin{lstlisting}
c:\Polygon\c>cl st.c /Fast.asm /MD
Microsoft (R) 32-bit C/C++ Optimizing Compiler Version 16.00.40219.01 for 80x86
Copyright (C) Microsoft Corporation.  All rights reserved.

st.c
c:\polygon\c\st.c(11) : warning C4700: uninitialized local variable 'c' used
c:\polygon\c\st.c(11) : warning C4700: uninitialized local variable 'b' used
c:\polygon\c\st.c(11) : warning C4700: uninitialized local variable 'a' used
Microsoft (R) Incremental Linker Version 10.00.40219.01
Copyright (C) Microsoft Corporation.  All rights reserved.

/out:st.exe
st.obj
\end{lstlisting}

Mais quand nous lançons le programme compilé \dots

\begin{lstlisting}
c:\Polygon\c>st
1, 2, 3
\end{lstlisting}

Quel résultat étrange ! Aucune variables n'a été initialisées dans \TT{f2()}.
Ce sont des valeurs \q{fantôme} qui sont toujours dans la pile.

\clearpage
Chargeons cet exemple dans \olly:

\begin{figure}[H]
\centering
\myincludegraphics{patterns/02_stack/08_noise/olly1.png}
\caption{\olly: \TT{f1()}}
\label{fig:stack_noise_olly1}
\end{figure}

Quand \TT{f1()} assigne les variable $a$, $b$ et $c$, leurs valeurs sont stockées à l'adresse \TT{0x1FF860} et ainsi de suite.

\clearpage
Et quand \TT{f2()} s'exécute:

\begin{figure}[H]
\centering
\myincludegraphics{patterns/02_stack/08_noise/olly2.png}
\caption{\olly: \TT{f2()}}
\label{fig:stack_noise_olly2}
\end{figure}

... $a$, $b$ et $c$ de la fonction \TT{f2()} sont situées à la même adresse !
Aucunes autre fonction n'a encore écrasées ces valeurs, elles sont donc encore inchangées. Pour que cette situation arrive, il faut que plusieurs fonctions soit appelées les unes après les autres et \ac{SP} doit être le même sur chaque début de fonctions (i.e., les fonctions doivent avoir le même nombre d'arguments). Les variables locales seront donc positionnées au même endroit dans la pile. Pour résumer, toutes les valeurs sur la pile sont des valeurs laissées par des appels de fonction précédent. Ces valeurs laissées sur la pile ne sont pas réellement aléatoire dans le sens strict du terme, mais elles sont imprévisibles.
Y a t'il une autre option ?
Il serait probablement possible de nettoyer des parties de la pile avant chaque nouvelle exécution de fonction, mais cela engendrerait du travail et du temps d'exécution (non nécessaire) en plus.

\subsubsection{MSVC 2013}

Cet exemple a été compilé avec MSVC 2010.
Si vous essayez de compiler cet exemple avec MSVC 2013 et de l'exécuter, ces 3 nombres seront inversé:%

\begin{lstlisting}
c:\Polygon\c>st
3, 2, 1
\end{lstlisting}

Pourquoi ?
J'ai aussi compilé cet exemple avec MSVC 2013 et constaté ceci: 


\begin{lstlisting}[caption=MSVC 2013,style=customasmx86]
_a$ = -12	; size = 4
_b$ = -8	; size = 4
_c$ = -4	; size = 4
_f2	PROC

...

_f2	ENDP

_c$ = -12	; size = 4
_b$ = -8	; size = 4
_a$ = -4	; size = 4
_f1	PROC

...

_f1	ENDP
\end{lstlisting}

Contrairement à MSVC 2010, MSVC 2013 alloue les variables a/b/c dans la fonction \TT{f2()} dans l'ordre inverse puisqu'il se comporte différemment en raison d'un changement supposé dans son fonctionnement interne.%
Ceci est correct, car le standard du \CCpp n'a aucune règle sur l'ordre d'allocation des variables locales sur la pile.
