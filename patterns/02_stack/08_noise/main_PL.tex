\subsection{Śmieci na stosie}
\label{noise_in_stack}

\epigraph{When one says that something seems random, what one usually
means in practice is that one cannot see any regularities in it.}{Stephen Wolfram, A New Kind of Science.}

Często w tej książce się wspomina o \q{śmieciach} na stosie lub w pamięci.
Skąd one się biorą?
Są to pozostałości po poprzednich funkcjach.

Krótki przykład:

\lstinputlisting[style=customc]{patterns/02_stack/08_noise/st.c}

Kompilujemy\dots

\lstinputlisting[caption=\NonOptimizing MSVC 2010,style=customasmx86]{patterns/02_stack/08_noise/st.asm}

Kompilator się trochę oburzy\dots

\begin{lstlisting}
c:\Polygon\c>cl st.c /Fast.asm /MD
Microsoft (R) 32-bit C/C++ Optimizing Compiler Version 16.00.40219.01 for 80x86
Copyright (C) Microsoft Corporation.  All rights reserved.

st.c
c:\polygon\c\st.c(11) : warning C4700: uninitialized local variable 'c' used
c:\polygon\c\st.c(11) : warning C4700: uninitialized local variable 'b' used
c:\polygon\c\st.c(11) : warning C4700: uninitialized local variable 'a' used
Microsoft (R) Incremental Linker Version 10.00.40219.01
Copyright (C) Microsoft Corporation.  All rights reserved.

/out:st.exe
st.obj
\end{lstlisting}

Ale kiedy uruchamiamy\dots

\begin{lstlisting}
c:\Polygon\c>st
1, 2, 3
\end{lstlisting}

Dziwne, przecież nie ustawialiśmy żadnych zmiennych w \TT{f2()}. 
Te wartości --- to są \q{duchy}, które wciąż się znajdują na stosie.

\clearpage
Uruchomimy przykład w \olly:

\begin{figure}[H]
\centering
\myincludegraphics{patterns/02_stack/08_noise/olly1.png}
\caption{\olly: \TT{f1()}}
\label{fig:stack_noise_olly1}
\end{figure}

Kiedy \TT{f1()} ustawia zmienne $a$, $b$ i $c$ one są zapisywane pod adresem \TT{0x1FF860}, itd.

\clearpage
A kiedy jest wykonywana \TT{f2()}:

\begin{figure}[H]
\centering
\myincludegraphics{patterns/02_stack/08_noise/olly2.png}
\caption{\olly: \TT{f2()}}
\label{fig:stack_noise_olly2}
\end{figure}

... $a$, $b$ i $c$ w funkcji \TT{f2()} znajdują się na tych samych miejscach!
Taka dziwna sytuacja powstaje, kiedy kilka funkcji jest wykonywane po sobie
i \ac{SP} powinien być taki sam przy wejściu do funkcji, tzn funkcja powinna mieć taką samą ilość argumentów. 
Wtedy zmienne lokalne będą alokowane na te same miejsca.
Podsumowując, wszystkie wartości na stosie (i ogólnie w pamięci) to wartości pozostałe po poprzedzających funkcjach
Nie są one przypadkowe, lecz nieprzewidywalne.
A jak inaczej?
Można by było czyścić stos przed wykonywaniem kolejnej funkcji,
ale to za dużo zbędnej (i nieporzebnej) roboty.

\subsubsection{MSVC 2013}

Ten przykład był skompilowany w MSVC 2010.
Ale jeden czytelnik tej książki spróbował skompilować to w MSVC 2013, uruchomił i zobaczył 3 liczby w odwrotnej kolejności:

\begin{lstlisting}
c:\Polygon\c>st
3, 2, 1
\end{lstlisting}

Dkaczego?
Również spróbowałem skompilować ten przykład w MSVC 2013 i zobaczyłem to:

\begin{lstlisting}[caption=MSVC 2013,style=customasmx86]
_a$ = -12	; size = 4
_b$ = -8	; size = 4
_c$ = -4	; size = 4
_f2	PROC

...

_f2	ENDP

_c$ = -12	; size = 4
_b$ = -8	; size = 4
_a$ = -4	; size = 4
_f1	PROC

...

_f1	ENDP
\end{lstlisting}

W odróżnieniu od MSVC 2010, MSVC 2013 rozmieścił zmienne a/b/c w funkcji \TT{f2()} w kolejności odwrotnej.
I jest to całkowicie poprawne, ponieważ w \CCpp nie ma zdefiniowanego standardu, reguły która by wyznaczała w jakiej kolejności zmienne lokalne powinne być usytuowane na stosie.


