\subsubsection{Passaggio di argomenti alle funzioni}

Il modo piu' diffuso di passare parametri in x86 e' detto \q{cdecl}:

\begin{lstlisting}[style=customasm]
push arg3
push arg2
push arg1
call f
add esp, 12 ; 4*3=12
\end{lstlisting}

La funzioni chiamate, \Gls{callee}, ricevono i propri argomenti tramite lo stack pointer.
Quindi e' cosi' che i valori degli arcomenti sono posizionati nello stack prima dell'esecuzione della prima istruzione della funzione \ttf{}:

\begin{center}
\begin{tabular}{ | l | l | }
\hline
ESP & return address \\
\hline
ESP+4 & \argument \#1, \MarkedInIDAAs{} \TT{arg\_0} \\
\hline
ESP+8 & \argument \#2, \MarkedInIDAAs{} \TT{arg\_4} \\
\hline
ESP+0xC & \argument \#3, \MarkedInIDAAs{} \TT{arg\_8} \\
\hline
\dots & \dots \\
\hline
\end{tabular}
\end{center}

Per ulteriori informazioni su altri tipi di convenzioni di chiamata (calling conventions), fare riferimento alla sezione~(\myref{sec:callingconventions}).
Vale la pena notare che non c'e' nulla che obbliga il programmatore a passare gli argomenti attraverso lo stack. Non e' un requisito necessario.
Si potrebbe implementare un qualunque altro metodo anche senza usare per niente lo stack.

Ad esempio e' possibile allocare spazio per gli argomenti nello \gls{heap}, riempirlo e passarlo ad una funzione tramite un puntatore a questo blocco di memoria, nel registro \EAX.
E cio' funzionerebbe.

% TBT: \InSqBrackets{\TAOCPvolI{}, 189} mentions even weirder schemes particularly convenient on IBM System/360.

Ad ogni modo, in x86 e ARM e' pratica diffusa e conveniente quella di usare lo stack per questo scopo.

\par
A proposito, la funzione chiamata (\gls{callee}) non ha alcuna informazione su quanti argomenti sono stati passati.
Le funzioni C functions con un numero variabile di argomenti (come \printf) determinano questo numero utilizzando i format string specifiers (che iniziano con il simbolo \% ).

Se scriviamo una cosa come:

\begin{lstlisting}
printf("%d %d %d", 1234);
\end{lstlisting}

\printf stampera' 1234, e due numeri randomici, che si trovano vicini a 1234 nello stack.

\par
Questo spiega perche' non e' molto importante come si dichiara la funzione \main : \main, \TT{main(int argc, char *argv[])} oppure \TT{main(int argc, char *argv[], char *envp[])}.

Infatti, il codice \ac{CRT} chiama \main grossomodo cosi':

\begin{lstlisting}[style=customasm]
push envp
push argv
push argc
call main
...
\end{lstlisting}

Se si dichiara \main come \main senza arogmenti, saranno comunque presenti nello stack anche se non usati.
Se si dichiara \main come \TT{main(int argc, char *argv[])},
si potranno usare i primi due argomenti, ed il terzo restera' \q{invisibile} alla funzione.
E' inoltre possibile dichiarare anche \TT{main(int argc)}, e funzionera'.
