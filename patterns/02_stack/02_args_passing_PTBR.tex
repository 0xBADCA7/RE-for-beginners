\subsubsection{Passando argumento de funções}

A maneira mais comum de se passar parâmetros no x86 é chamada \q{cdecl}:

\begin{lstlisting}[style=customasmx86]
push arg3
push arg2
push arg1
call f
add esp, 12 ; 4*3=12
\end{lstlisting}

Uma função chamada recebe seus argumentos pelo ponteiro da pilha.

Portanto, é assim que os valores dos argumentos são alocados na pilha antes da execução das primeiras intruções da função \ttf{}:

\begin{center}
\begin{tabular}{ | l | l | }
\hline
ESP & endereço de retorno \\
\hline
ESP+4 & \argument \#1, \MarkedInIDAAs{} \TT{arg\_0} \\
\hline
ESP+8 & \argument \#2, \MarkedInIDAAs{} \TT{arg\_4} \\
\hline
ESP+0xC & \argument \#3, \MarkedInIDAAs{} \TT{arg\_8} \\
\hline
\dots & \dots \\
\hline
\end{tabular}
\end{center}

\PTBRph{}

Vale ressaltar que nada obriga o programador a passar os argumentos pela pilha. Não é um requerimento.
Você pode implementar qualquer outro método usando a pilha da maneira que desejar.

Por exemplo, é possível alocar um espaço para argumentos na \gls{heap}, preencher e passar para a função via um ponteiro para esse bloco no registrador \EAX{}.
% TBT: \InSqBrackets{\TAOCPvolI{}, 189} mentions even weirder schemes particularly convenient on IBM System/360.

Isso vai funcionar, entretando, é de senso comum no x86 e ARM a usar a pilha para esse fim.

\par
A propósito, a função chamada não tem nenhuma informação sobre quantos argumentos foram passados.
Funções em C com um número variável de argumentos (como \printf) determina seu número usando formatações específicas de string (que começam com o símbolo \%).

Se nós escrevermos algo como:

\begin{lstlisting}
printf("%d %d %d", 1234);
\end{lstlisting}

\printf vai mostrar 1234, e então dois números aleatórios, que estariam próximos a stack.

\par
É por isso que não é muito importante como declaramos a função \main{}: como \main{}, \TT{main(int argc, char *argv[])} ou \TT{main(int argc, char *argv[], char *envp[])}.

Na verdade, o código da C Runtime Library está chamando grosseiramente \main{} dessa maneira:
	
\begin{lstlisting}[style=customasmx86]
push envp
push argv
push argc
call main
...
\end{lstlisting}

Se você declarar \main como \main sem argumentos, mesmo assim eles ainda estarão presentes na pilha, mas não são usados.
Se você declarar \main como \TT{main(int argc, char *argv[])}, você será capaz de utilizar os primeiros dois argumentos e o terceiro vai continuar \q{invisível} para a sua função.
Da mesma maneira, é possível declarar a \main como \TT{main(int argc)} e ainda assim vai funcionar.

