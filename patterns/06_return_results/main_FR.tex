\section{Plus sur le renvoi des résultats}

\myindex{x86!\Registers!EAX}

En x86, le résultat de l'exécution d'une fonction est d'habitude renvoyé
\footnote{\Seealso: MSDN: Return Values (C++): \href{http://go.yurichev.com/17258}{MSDN}}
dans le registre \EAX.

Si il est de type octet ou un caractère (\Tchar), alors la partie basse du registre
\EAX (\AL) est utilisée.
Si une fonction renvoie un nombre de type \Tfloat, le registre \ST{0} du FPU est utilisé. % what about double ?
\myindex{ARM!\Registers!R0}
En ARM, d'habitude, le résultat est renvoyé dans le registre \Reg{0}.

\subsection{Tentative d'utilisation du résultat d'une fonction renvoyant \Tvoid}

Donc, que se passe-t-il si le type de retour de la fonction \main a été déclaré
du type \Tvoid et non pas \Tint?
Le soi-disant code de démarrage (startup-code) appelle \main grosso-modo de la façon
suivante:

\begin{lstlisting}[style=customasmx86]
push envp
push argv
push argc
call main
push eax
call exit
\end{lstlisting}

En d'autres mots:

\begin{lstlisting}[style=customc]
exit(main(argc,argv,envp));
\end{lstlisting}

Si vous déclarez \main comme renvoyant \Tvoid, rien ne sera renvoyé explicitement
(en utilisant la déclaration \IT{return}), alors quelque chose d'inconnu, qui aura
été stocké dans la registre \EAX lors de l'exécution de \main sera l'unique argument
de la fonction exit().
Il y aura probablement une valeur aléatoire, laissée lors de l'exécution de la fonction,
donc le code de retour du programme est pseudo-aléatoire.
\par
Illustrons ce fait:
Notez bien  que la fonction \main a un type de retour \Tvoid:

\begin{lstlisting}[style=customc]
#include <stdio.h>

void main()
{
	printf ("Hello, world!\n");
};
\end{lstlisting}

Compilons-le sous Linux.

\myindex{puts() instead of printf()}
GCC 4.8.1 a remplacé \printf par \puts (nous avons vu ceci avant: \myref{puts}),
mais c'est OK, puisque \puts renvoie le nombre de caractères écrit, tout comme
\printf.
Remarquez que le registre \EAX n'est pas mis à zéro avant la fin de \main.

Cece implique que la valeur de \EAX à la fin de \main contient ce que \puts y avait mis.

\begin{lstlisting}[caption=GCC 4.8.1,style=customasmx86]
.LC0:
	.string	"Hello, world!"
main:
	push	ebp
	mov	ebp, esp
	and	esp, -16
	sub	esp, 16
	mov	DWORD PTR [esp], OFFSET FLAT:.LC0
	call	puts
	leave
	ret
\end{lstlisting}

\myindex{bash}

Écrivons un script bash qui montre le code de retour:

\begin{lstlisting}[caption=tst.sh]
#!/bin/sh
./hello_world
echo \$?
\end{lstlisting}

Et lançons le:

\begin{lstlisting}
\$ tst.sh 
Hello, world!
14
\end{lstlisting}

14 est le nombre de caractères écrits.
Le nombre de caractères affichés est \textit{passé} de \printf{}, à travers \TT{EAX}/\TT{RAX}, dans le \q{code de retour}.

\myindex{Hex-Rays}
A propos, lorsque l'on décompile du C++ dans Hex-Rays, nous rencontrons souvent
une fonction qui se termine par un destructeur d'une classe:

\begin{lstlisting}[style=customasmx86]
...

call    ??1CString@@QAE@XZ ; CString::~CString(void)
mov     ecx, [esp+30h+var_C]
pop     edi
pop     ebx
mov     large fs:0, ecx
add     esp, 28h
retn
\end{lstlisting}

Dans le standard C++, le destructeur ne renvoie rien, mais lorsque Hex-Rays n'en % FIXME \IDA ?
sait rien, et pense que le destructeur et cette fonction renvoient tout deux un \Tint

\begin{lstlisting}[style=customc]
...

	return CString::~CString(&Str);
}
\end{lstlisting}

\subsection{Que se passe-t-il si on n'utilise pas le résultat de la fonction?}

\printf renvoie le nombre de caractères écrit avec succès, mais, en pratique, ce
résultat est rarement utilisé.

Il est aussi possible d'appeler une fonction dont la finalité est de renvoyer une valeur,
et de ne pas l'utiliser:

\begin{lstlisting}[style=customc]
int f()
{
    // skip first 3 random values:
    rand();
    rand();
    rand();
    // and use 4th:
    return rand();
};
\end{lstlisting}

Le résultat de la fonction rand() est mis dans \EAX, dans les quatre cas.

Mais dans les 3 premiers, la valeur dans \EAX n'est pas utilisée.

\subsection{Renvoyer une structure}

\myindex{\CLanguageElements!return}

Revenons au fait que la valeur de retour est passée par le registre \EAX.

C'est pourquoi les vieux compilateurs C ne peuvent pas créer de fonction capable
de renvoyer quelque chose qui ne tient pas dans un registre (d'habitude \Tint), mais
si besoin, les informations doivent être renvoyées via un pointeur passé en argument.

Donc, d'habitude, si une fonction doit renvoyer plusieurs valeurs, elle en renvoie
une seule, et le reste---par des pointeurs.

Maintenant, il est possible de renvoyer. disons, une structure entière, mais ce
n'est pas encore très populaire.
Si une fonction doit renvoyer une grosse structure, la fonction \glslink{caller}{appelante}
doit l'allouer et passer un pointeur sur cette dernière via le premier argument,
de manière transparente pour le programmeur.
C'est presque la même chose que de passer un pointeur manuellement dans le premier
argument, mais le compilateur le cache.

Petit exemple:

\lstinputlisting[style=customc]{patterns/06_return_results/6_1.c}

\dots ce que nous obtenons (MSVC 2010 \Ox):

\lstinputlisting[style=customasmx86]{patterns/06_return_results/6_1.asm}

Ici, le nom de la macro interne pour passer le pointeur sur une structure est \GTT{\$T3853}.

\myindex{\CLanguageElements!C99}
Cet exemple peut être réécrit en utilisant les extensions C99 du langage:

\lstinputlisting[style=customc]{patterns/06_return_results/6_1_C99.c}

\lstinputlisting[caption=GCC 4.8.1,style=customasmx86]{patterns/06_return_results/6_1_C99.asm}

Comme on le voit, la fonction remplit simplement les champs de la structure allouée
par la fonction appelante, comme si un pointeur sur la structure avait été passé.
Donc, il n'y a pas d'impact négatif sur les performances.
