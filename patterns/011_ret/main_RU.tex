\section{Простейшая функция}

Наверное, простейшая из возможных функций это та что возвращает некоторую константу:

Вот, например:

\lstinputlisting[caption=Код на \CCpp]{patterns/011_ret/1.c}

Скомпилируем её!

\subsection{x86}

И вот что делает оптимизирующий GCC:

\lstinputlisting[caption=\Optimizing GCC/MSVC (\assemblyOutput)]{patterns/011_ret/1.s}

\myindex{x86!\Instructions!RET}
Здесь только две инструкции. Первая помещает значение 123 в регистр \EAX, который используется
для передачи возвращаемых значений. Вторая это \RET, которая возвращает управление в вызывающую функцию.

Вызывающая функция возьмет результат из регистра \EAX.

\subsection{ARM}

А что насчет ARM?

\lstinputlisting[caption=\OptimizingKeilVI (\ARMMode) ASM Output]{patterns/011_ret/1_Keil_ARM_O3.s}

ARM использует регистр \Reg{0} для возврата значений, так что здесь 123 помещается в \Reg{0}.

Адрес возврата (\ac{RA}) в ARM не сохраняется в локальном стеке, а в регистре \ac{LR}.
Так что инструкция \INS{BX LR} делает переход по этому адресу, и это то же самое что и вернуть управление
в вызывающую ф-цию.

\myindex{ARM!\Instructions!MOV}
\myindex{x86!\Instructions!MOV}
Нужно отметить, что название инструкции \MOV в x86 и ARM сбивает с толку.

На самом деле, данные не \IT{перемещаются}, а скорее \IT{копируются}.

\subsection{MIPS}

\label{MIPS_leaf_function_ex1}
Есть два способа называть регистры в мире MIPS. По номеру (от \$0 до \$31) или по псевдоимени (\$V0, \$A0, итд.).

Вывод на ассемблере в GCC показывает регистры по номерам:

\lstinputlisting[caption=\Optimizing GCC 4.4.5 (\assemblyOutput)]{patterns/011_ret/MIPS.s}

\dots а \IDA --- по псевдоименам:

\lstinputlisting[caption=\Optimizing GCC 4.4.5 (IDA)]{patterns/011_ret/MIPS_IDA.lst}

Так что регистр \$2 (или \$V0) используется для возврата значений.
\myindex{MIPS!\Pseudoinstructions!LI}
\INS{LI} это ``Load Immediate'', и это эквивалент \MOV в MIPS.

\myindex{MIPS!\Instructions!J}
Другая инструкция это инструкция перехода (J или JR), которая возвращает управление в \glslink{caller}{вызывающую ф-цию}, переходя по адресу в регистре \$31 (или \$RA).

Это аналог регистра \ac{LR} в ARM.

\myindex{MIPS!Branch delay slot}
Но почему инструкция загрузки (LI) и инструкция перехода (J или JR) поменяны местами? Это артефакт \ac{RISC} и называется он ``branch delay slot''.

На самом деле, нам не нужно вникать в эти детали. Нужно просто запомнить: в MIPS инструкция после инструкции перехода исполняется \IT{перед} инструкцией перехода.

Таким образом, инструкция перехода всегда поменяна местами с той, которая должна быть исполнена перед ней.

\subsubsection{Еще кое-что об именах инструкций и регистров в MIPS}

Имена регистров и инструкций в мире MIPS традиционно пишутся в нижнем регистре.
Но мы будем использовать верхний регистр, потому что имена инструкций и регистров других \ac{ISA} в этой книге так же в верхнем регистре.

