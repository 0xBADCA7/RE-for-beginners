\section{Valeur de retour}
\label{ret_val_func}

Une autre fonction simple est celle qui retourne juste une valeur constante:

La voici:

\lstinputlisting[caption=Code \CCpp,style=customc]{patterns/011_ret/1.c}

Compilons la!

\subsection{x86}

Voici ce que les compilateurs GCC et MSVC produisent sur une plateforme x86:

\lstinputlisting[caption=GCC/MSVC \Optimizing (\assemblyOutput),style=customasmx86]{patterns/011_ret/1.s}

\myindex{x86!\Instructions!RET}
Il y a juste deux instructions: la première place la valeur 123 dans le registre \EAX, qui est par convention le registre utilisé pour stocker la valeur renvoyée d'une fonction et la seconde est \RET, qui retourne l'exécution vers l'\glslink{caller}{appelant}.

L'appelant prendra le résultat de cette fonction dans le registre \EAX.

\subsection{ARM}

Il y a quelques différences sur la platforme ARM:

\lstinputlisting[caption=\OptimizingKeilVI (\ARMMode) ASM Output,style=customasmARM]{patterns/011_ret/1_Keil_ARM_O3.s}

ARM utilise le registre \Reg{0} pour renvoyer le résultat d'une fonction, donc 123 est copié dans \Reg{0}.

\myindex{ARM!\Instructions!MOV}
\myindex{x86!\Instructions!MOV}

Il est à noter que l'instruction \MOV est trompeuse pour les plateformes x86 et ARM \ac{ISA}s.

La donnée n'est en réalité pas \IT{déplacée (moved)} mais \IT{copiée}.

\subsection{MIPS}

\label{MIPS_leaf_function_ex1}

La sortie de l'assembleur GCC ci-dessous indique les registres par numéro:

\lstinputlisting[caption=GCC 4.4.5 \Optimizing (\assemblyOutput),style=customasmMIPS]{patterns/011_ret/MIPS.s}

\dots tandis qu'\IDA le fait---avec les pseudo noms:

\lstinputlisting[caption=GCC 4.4.5 \Optimizing (IDA),style=customasmMIPS]{patterns/011_ret/MIPS_IDA.lst}

Le registre \$2 (ou \$V0) est utilisé pour stocker la valeur de retour de la fonction.
\myindex{MIPS!\Pseudoinstructions!LI}
\INS{LI} signifie ``Load Immediate'' et est l'équivalent MIPS de \MOV.

\myindex{MIPS!\Instructions!J}

L'autre instruction est l'instruction de saut (J ou JR) qui retourne le flux d'exécution vers l'\glslink{caller}{appelant}.

\myindex{MIPS!Branch delay slot}

Vous pouvez vous demander pourquoi la position de l'instruction d'affectation de
valeur immédiate (LI) et l'instruction de saut (J ou JR) sont échangées. Ceci est
dû à une fonctionnalité du \ac{RISC} appelée ``branch delay slot'' (slot de délai
de branchement).

La raison de cela est du à une bizarrerie dans l'architecture de certains RISC \ac{ISA}s et n'est pas importante pour nous. Nous gardons juste en tête qu'en MIPS, l'instruction qui suit une instruction de saut ou de branchement est exécutée \IT{avant} l'instruction de saut ou de branchement elle-même.

Par conséquent, les instructions de branchement échangent toujours leur place avec l'instruction qui doit être exécutée avant.

% A footnote/link to http://en.wikipedia.org/wiki/Delay_slot#Branch_delay_slots or
% something similar might be useful for the people more interested in it.

\subsection{En pratique}

Les fonctions qui retournent simplement 1 (\IT{true}) ou 0 (\IT{false}) sont vraiment fréquentes en pratique.

Les plus petits utilitaires UNIX standard, \IT{/bin/true} et \IT{/bin/false} renvoient
respectivement 0 et 1, comme code de retour.
(un code retour de zéro signifie en général succès, non-zéro une erreur).

