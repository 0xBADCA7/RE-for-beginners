\section{Die einfachste Funktion}

Die einfachste, mögliche Funktion ist vermutlich eine, die lediglich einen konstanten Wert zurückgibt.

Hier ist sie:

\lstinputlisting[caption=\DE{\CCpp Code}]{patterns/011_ret/1.c}

Und nun in kompilierter Version!

\subsection{x86}

Nachfolgend das, was sowohl der optimierende GCC- als auch MSVC-Compiler auf einer x86-Plattform erzeugt:

\lstinputlisting[caption=\Optimizing GCC/MSVC (\assemblyOutput)]{patterns/011_ret/1.s}

\myindex{x86!\Instructions!RET}
Es gibt zwei Anweisungen: die erste platziert den Wert 123 in das \EAX-Register, welches per Konvention
als Speicherplatz für den Rückabewert genutzt wird. Die zweite ist \RET, die die Ausführung wieder an
die aufrufende Funktion übergibt.

Diese wird das Ergebnis vom \EAX-Register übernehmen.

\subsection{ARM}

Auf der ARM-Plattform gibt es einige kleine Unterschiede:

\lstinputlisting[caption=\OptimizingKeilVI (\ARMMode) Assembler-Ausgabe]{patterns/011_ret/1_Keil_ARM_O3.s}

ARM nutzt das Register \Reg{0} für das Speichern des Rückgabewerts der Funktion. Also wird in diesem Beispiel
der Wert 123 dorthin kopiert.

%Maybe explain what a link register is, or if it is just a normal register, say so?
Die Rücksprungadresse wird nicht auf dem lokalen Stack sondern im Link-Register gespeichert.
Die Anweisung \INS{BX LR} führt dazu, dass die Ausführung an dieser Stelle fortgeführt wird. In diesem Fall wird
also die Kontrolle wieder an die aufrufende Funktion übergeben.

\myindex{ARM!\Instructions!MOV}
\myindex{x86!\Instructions!MOV}
Erwähnenswert ist der irreführende Name der \MOV-Anweisung sowohl beim x86- als auch ARM-Befehlssatz:
die Daten werden nicht \IT{verschoben} sondern \IT{kopiert}.

\subsection{MIPS}

\label{MIPS_leaf_function_ex1}
Es gibt zwei verschiedene Konventionen bei der Benamung von Registern in der MIPS-Welt:
mit einer Nummer (von \$0 bis \$31) oder mit einem Pseudonamen (\$V0, \$A0, usw.).

GCC benamt in der Ausgabe die Register mit Nummern:

\lstinputlisting[caption=\Optimizing GCC 4.4.5 (\assemblyOutput)]{patterns/011_ret/MIPS.s}

\dots während \IDA Pseudonamen verwendet:

\lstinputlisting[caption=\Optimizing GCC 4.4.5 (IDA)]{patterns/011_ret/MIPS_IDA.lst}

Das \$2 (oder \$V0)-Register wird zum Speichern des Rückgabewerts genutzt.
\myindex{MIPS!\Pseudoinstructions!LI}
\INS{LI} steht für ''Load Immediate'' und ist das MIPS-Äquivalent zu \MOV.

\myindex{MIPS!\Instructions!J}
Die anderen Anweisungen sind Sprungbefehle (J oder JR), die die Ausführung wieder an die aufrufende Funktion übergeben,
indem an die Adresse gesprungen wird die im \$31 (oder \$RA)-Register.

Dieses Register ist analog zum \ac{LR} in der ARM-Architektur.

\myindex{MIPS!Branch delay slot}
Möglicherweise wundert man sich warum die Positionen der Lade- (LI) und Sprunganweisung (J or JR) vertauscht sind.
Dies geschieht durch ein \ac{RISC}-Feature das ''branch delay slot'' genannt wird.

Die Begründung liegt in der Eigenart einiger RISC-Befehlssets die hier jedoch nicht so wichtig ist. Man sollte
aber im Hinterkopf behalten, dass bei MIPS die Anweisung \IT{nach} dem Sprungbefehl noch \IT{vor} dieser ausgeführt wird.

Als Konsequenz sind Verzweigungsbefehle immer mit der Anweisung vertauscht, die zuvor ausgeführt werden muss.
% A footnote/link to http://en.wikipedia.org/wiki/Delay_slot#Branch_delay_slots or
% something similar might be useful for the people more interested in it.

\subsubsection{Anmerkungen zu MIPS-Anweisungen / Registernamen}

Register- und Anweisungsnamen sind in der MIPS-Welt traditionellerweise in Kleinbuchstaben geschrieben.
Aus Gründen der Einheitlichkeit wird in diesem Buch jedoch die Großschreibung bevorzugt.
