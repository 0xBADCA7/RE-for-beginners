\section{Fonction prologue et épilogue}
\label{sec:prologepilog}
\myindex{Function epilogue}
\myindex{Function prologue}

Un prologue de fonction est une séquence particulière d'instructions située au début d'une fonction. Il ressemble souvent à ce morceau de code:

\begin{lstlisting}[style=customasmx86]
    push    ebp
    mov     ebp, esp
    sub     esp, X
\end{lstlisting}

Ce que ces instructions font: sauvent la valeur du registre \EBP dans la pile (push ebp), sauvent la valeur actuelle du registre \ESP dans le
 registre \EBP (mov ebp, esp) et enfin allouent de la mémoire dans la pile pour les variables locales de la fonction (sub esp, X).

La valeur du registre \EBP reste la même durant la période où la fonction s'exécute et est utilisée pour accéder aux variables locales et aux arguments de la fonction.

Le registre \ESP peut aussi être utilisé pour accéder aux variables locales et aux arguments de la fonction, cependant cette approche n'est
 pas pratique car sa valeur est susceptible de changer au cours de l'exécution de cette fonction.

L'épilogue de fonction libère la mémoire allouée dans la pile (mov esp, ebp), restaure l'ancienne valeur de \EBP précédemment sauvegardée dans la pile (pop ebp) puis rend l'exécution au \gls{caller} (ret 0).

\begin{lstlisting}[style=customasmx86]
    mov    esp, ebp
    pop    ebp
    ret    0
\end{lstlisting}

% what about calling convention?
Les prologues et épilogues de fonction sont généralement détectés par les désassembleurs pour déterminer où une fonction commence et où elle se termine.

\subsection{\Recursion}

\myindex{\Recursion}
Les prologues et épilogues de fonction peuvent affecter négativement les performances de récursion.

Plus d'information sur la récursivité dans ce livre: \myref{Recursion_and_tail_call}.
