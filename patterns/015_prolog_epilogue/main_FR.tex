\section{Function prologue and epilogue}
\label{sec:prologepilog}
\myindex{Function epilogue}
\myindex{Function prologue}

Un prologue de fonction est une séquence particuliaire d'instructions situé au début d'une fonction. Il ressemble souvent à ce morceau de code:

\begin{lstlisting}[style=customasmx86]
    push    ebp
    mov     ebp, esp
    sub     esp, X
\end{lstlisting}

Ce que ces instructions font: sauve la valeur du registre \EBP dans la stack (push ebp), sauve la valeur actuel du registre \ESP dans le registre \EBP (mov ebp, esp) et enfin alloue de la mémoire dans la stack pour les variables locales de la fonction (sub esp, X).

La valeur du registre \EBP réste la même durant la période où la fonction s'execute et est utilisé pour accéder au variable local et au arguments de la fonction.

Le registre \ESP peut aussi être utilisé pour accéder au variable local et au arguments de la fonction, cependant cette approche n'est pas pratique car sa valeur est susceptible de changer au cours de l'execution de cette fonction.

L'épilogue de fonction libère la mémoire alloué dans la stack (mov esp, ebp), restaure l'ancienne valeur de \EBP précédement sauvegarder dans la stack (pop ebp) puis rend l'execution au \gls{caller} (ret 0).

\begin{lstlisting}[style=customasmx86]
    mov    esp, ebp
    pop    ebp
    ret    0
\end{lstlisting}

% what about calling convention?
Les prologues et épilogues de fonction sont généralement détecté par les désassembleurs pour déterminer où une fonction commence et où elle se termine. 

\subsection{\Recursion}

\myindex{\Recursion}
Les prologues et épilogues de fonction peuvent affécter négativement les performances de récursion.

Plus d'information sur la récursivité dans ce livre: \myref{Recursion_and_tail_call}.
