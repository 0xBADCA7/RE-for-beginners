\section{関数のプロローグとエピローグ}
\label{sec:prologepilog}
\myindex{Function epilogue}
\myindex{Function prologue}

関数プロローグは、関数の先頭にある一連の命令です。 それはしばしば以下のコード断片のように見えます。

\begin{lstlisting}[style=customasmx86]
    push    ebp
    mov     ebp, esp
    sub     esp, X
\end{lstlisting}

これらの命令が行うこと: \EBP レジスタに値を保存し、
\EBP レジスタの値をESPの値に設定し、
ローカル変数のためにスタック上に領域を割り当てます。

\EBP の値は、関数実行の期間にわたって同じままであり、ローカル変数および引数アクセスに使用されます。 
同じ目的のために \ESP を使うことができますが、時間の経過とともに変化するので、この方法はあまり便利ではありません。

関数のエピローグは、スタック内の割り当てられた領域を解放し、 \EBP レジスタの値を初期状態に戻し、制御フローを\gls{caller}に返します。

\begin{lstlisting}[style=customasmx86]
    mov    esp, ebp
    pop    ebp
    ret    0
\end{lstlisting}

% what about calling convention?
関数のプロローグとエピローグは、通常、逆アセンブラで関数の区切りとして検出されます。

\subsection{\Recursion}

\myindex{\Recursion}
エピローグとプロローグは、再帰のパフォーマンスに悪影響を及ぼします。

この本の再帰の詳細は下記を参照: \myref{Recursion_and_tail_call}
