\myparagraph{ARM 32-bit}

\mysubparagraph{\NonOptimizingXcodeIV (\ARMMode)}

\lstinputlisting[caption=\NonOptimizingXcodeIV (\ARMMode),label=ARM_leaf_example7,style=customasmARM]{patterns/10_strings/1_strlen/ARM/xcode_ARM_O0_FR.asm}

LLVM sans optimisation génère beaucoup trop de code, toutefois, ici nous pouvons
voir comment la fonction travaille avec les variables locales.
Il y a seulement deux variables locales dans notre fonction: \IT{eos} et \IT{str}.
Dans ce listing, généré par \IDA, nous avons renommé manuellement \IT{var\_8} et
\IT{var\_4} en \IT{eos} et \IT{str}.

La première instruction sauve simplement les valeurs d'entrée dans \IT{str} et \IT{eos}.

Le corps de la boucle démarre au label \IT{loc\_2CB8}.

Les trois première instructions du corps de la boucle (\TT{LDR}, \ADD, \TT{STR})
chargent la valeur de \IT{eos} dans \Reg{0}.
Puis la valeur est \glslink{increment}{incrémentée} et sauvée dans \IT{eos}, qui
se trouve sur la pile.

\myindex{ARM!\Instructions!LDRSB}
L'instruction suivante, \TT{LDRSB R0, [R0]} (\q{Load Register Signed Byte}), charge
un octet depuis la mémoire à l'adresse stockée dans \Reg{R0} et étend le signe à
32-bit\footnote{Le compilateur Keil considère le type \Tchar comme signé, tout
comme MSVC et GCC.}.
\myindex{x86!\Instructions!MOVSX}
Ceci est similaire à l'instruction \MOVSX en x86.

Le compilateur traite cet octet comme signé, puisque le type \Tchar est signé selon
la norme C.
Il a déjà été écrit à propos de cela~(\myref{MOVSX}) dans cette section, en relation
avec le x86.

\myindex{Intel!8086}
\myindex{Intel!8080}
\myindex{ARM}

Il est à noter qu'il est impossible en ARM d'utiliser séparément la partie 8- ou
16-bit d'un registre 32-bit complet, comme c'est le cas en x86.

Apparement, c'est parce que le x86 à une énorme histoire de rétro-compatibilité
avec ses ancêtres, jusqu'au 8086 16-bit et même 8080 8-bit, mais ARM a été développé
à partir de zéro comme un processeur RISC 32-bit.

Par conséquent, pour manipuler des octets séparés en ARM, on doit tout de même utiliser
des registres 32-bit.

Donc, \TT{LDRSB} charge des octets depuis la chaîne vers \Reg{0}, un par un.
Les intructions suivantes, \CMP et \ac{BEQ} vérifient si l'octet chargé est 0.
Si il n'est pas à 0, le contrôle passe au début du corps de la boucle.
Et si c'est 0, la boucle est terminée.

Á la fin de la fonction, la différence entre \IT{eos} et \IT{str} est calculée,
1 en est soustrait, et la valeur résultante est renvoyée via \Reg{0}.

N.B. Les registres n'ont pas été sauvés dans cette fonction.
\myindex{ARM!\Registers!scratch registers}

C'est parce que dans la convention d'appel ARM, les registres \Reg{0}-\Reg{3} sont
des \q{registres scratch}, destinés à passer les arguments, et il n'est pas requis
de restaurer leur valeur en sortant de la fonction, puisque  la fonction appelante
ne va plus les utiliser.
Par conséquent, ils peuvent être utilisés comme bien nous semble.

Il n'y a pas d'autres registres utilisés ici, c'est pourquoi nous n'avons rien à
sauvegarder sur la pile.

Ainsi, le contrôle peut être rendu à la fonction appelante par un simple saut (\TT{BX}),
à l'adresse contenue dans le registre \ac{LR}.

\mysubparagraph{\OptimizingXcodeIV (\ThumbMode)}

\lstinputlisting[caption=\OptimizingXcodeIV (\ThumbMode),style=customasmARM]{patterns/10_strings/1_strlen/ARM/xcode_thumb_O3.asm}

Comme le conclue LLVM avec l'optimisation, \IT{eos} et \IT{str} n'ont pas besoin
d'espace dans la pile, et peuvent toujours être stockés dans les registres.

Avant le début du corps de la boucle, \IT{str} est toujours dans \Reg{0}, et
\IT{eos}---dans \Reg{1}.

\myindex{ARM!\Instructions!LDRB.W}
L'instruction \TT{LDRB.W R2, [R1],\#1} charge, dans \Reg{2}, un octet de la mémoire
à l'adresse stockée dans \Reg{1}, en étendant le signe à une valeur 32-bit, mais
pas seulement cela.
\TT{\#1} à la fin de l'instruction indique un \q{Adressage post-indexé} (\q{Post-indexed addressing}),
qui signifie que 1 doit être ajouté à \Reg{1} après avoir chargé l'octet.
Pour en lire plus à ce propos: \myref{ARM_postindex_vs_preindex}.

Ensuite vous pouvez voir \CMP et \ac{BNE} dans le corps de la boucle, ces instructions
continuent de boucler jusqu'à ce que 0 soit trouvé dans la chaîne.

\myindex{ARM!\Instructions!MVNS}
\myindex{x86!\Instructions!NOT}
Les instructions \TT{MVNS}\footnote{MoVe Not} (inverse tous les bits, comme \NOT
en x86) et \ADD calculent $eos - str - 1$.
~(\myref{strlen_NOT_ADD}).
En fait, ces deux instructions calculent $R0 = ~str + eos$, qui est effectivement
équivalent à ce qui est dans le code source, et la raison de ceci à déjà été expliquée
ici ~(\myref{strlen_NOT_ADD}).

Apparemment, LLVM, tout comme GCC, conclu que ce code peut être plus court (ou plus
rapide).

\mysubparagraph{\OptimizingKeilVI (\ARMMode)}

\lstinputlisting[caption=\OptimizingKeilVI (\ARMMode),label=ARM_leaf_example6,style=customasmARM]{patterns/10_strings/1_strlen/ARM/Keil_ARM_O3.asm}

\myindex{ARM!\Instructions!SUBEQ}

Presque la même chose que ce que nous avions vu avant, à l'exception que l'expression
$str - eos - 1$ peut être calculée non pas à la fin de la fonction, mais dans le
corps de la boucle.
Le suffixe \TT{-EQ}, comme nous devrions nous en souvenir, implique que l'instruction
ne s'exécute que si les opérandes de la dernière instruction \CMP qui a été exécutée
avant étaient égaux.
Ainsi, si \Reg{0} contient 0, les deux instructions \TT{SUBEQ} sont exécutées et
le résultat est laissé dans le registre \Reg{0}.
