\myparagraph{ARM64}

\mysubparagraph{\Optimizing GCC (Linaro) 4.9}

\lstinputlisting{patterns/10_strings/1_strlen/ARM/ARM64_GCC_O3_DE.lst}

Der Algorithmus ist der gleiche wie in \myref{strlen_MSVC_Ox}: 
finde ein Nullbyte, berechne die Differenz zwischen den Pointern und subtrahiere
1 vom Ergebnis. 
Einige Kommentare wurden vom Autor hinzugef�gt.

Die einzig bemerkenswerte Sache ist, dass unser Beispiel in gewisser Weise
fehlerhaft ist:\\
\TT{my\_strlen()} liefert einen 32-bit \Tint, obwohl es \TT{size\_t} oder
einen anderen 64-bit Typ zur�ckliefern m�sste.

Der Grund daf�r ist, dass \TT{strlen()} theoretisch f�r einen sehr gro�en
Speicherblock, gr��er als 4GB, aufgerufen werden k�nnte und deshalb auf einer
64-bit-Plattform in der Lage sein muss, einen 64-bit-Wert zur�ckzuliefern.

Aufgrund meines Fehlers, arbeitet der letzte \SUB Befehl nur mit einem
32-bit-Teil des Registers, wohingegen der vorletzte \SUB Befehl mit dem
kompletten 64-bit-Register arbeitet (und die Differenz zwischen den Pointer
berechnet).

Es handelt sich um einen Fehler von mir, und es ist besser es so zu lassen, als
ein Lehrbeispiel wie Code in einem derartigen Fall aussehen kann.

\mysubparagraph{\NonOptimizing GCC (Linaro) 4.9}

\lstinputlisting{patterns/10_strings/1_strlen/ARM/ARM64_GCC_O0_DE.lst}

Es ist umfangreicher. 
Die Variablen werden hier viel im Speicher (lokaler Stack) herumgeschoben.
Der obige Fehler findet sich auch hier: das Dekrementieren geschieht nur in
einem 32-bit-Teil des Registers.
