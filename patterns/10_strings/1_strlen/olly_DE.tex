\clearpage
\myparagraph{\Optimizing MSVC + \olly}
\myindex{\olly}

Wir untersuchen das (optimierte) Beispiel in \olly. Hier ist der erste
Durchlauf:

\begin{figure}[H]
\centering
\myincludegraphics{patterns/10_strings/1_strlen/olly1.png}
\caption{\olly: Beginn erster Durchlauf}
\label{fig:strlen_olly_1}
\end{figure}

Wir sehen, dass \olly eine Schleife gefunden hat, und zur Verbesserung der
Lesbarkeit, diese in eckige Klammern \IT{eingeschlossen} hat.
Nach Rechtsklick auf \EAX w�hlen wir \q{Follow in Dump} und das Speicherfenster
scrollt an die passende Stelle. 
Hier sehen wir den String \q{hello!} im Speicher. 
Dahinter befindet sich mindestens ein Nullbyte und im Anschluss Zufallsbits.

Wenn \olly ein Register mit einer g�ltigen Adresse, die auf einen String zeigt,
findet, wird dieser String angezeigt.

\clearpage
Wir dr�cken einige Male F8 (\stepover) um zum Anfang der Schleifenk�rpers zu
gelangen:

\begin{figure}[H]
\centering
\myincludegraphics{patterns/10_strings/1_strlen/olly2.png}
\caption{\olly: Beginn zweiter Durchlauf}
\label{fig:strlen_olly_2}
\end{figure}

Wir sehen, dass \EAX nun die Adresse des zweiten Zeichens des Strings enth�lt.

\clearpage

Durch hinreichend h�ufiges Dr�cken von F8 verlassen wir schlie�lich die
Schleife:

\begin{figure}[H]
\centering
\myincludegraphics{patterns/10_strings/1_strlen/olly3.png}
\caption{\olly: Pointer Differenz wird berechnet}
\label{fig:strlen_olly_3}
\end{figure}

Wir sehen, dass \EAX jetzt die Adresse des Nullbytes direkt hinter dem String
enth�lt. In der Zwischenzeit hat sich \EDX nicht ver�ndert, es zeigt also immer
noch auf den Anfang des Strings. 

Die Differenz zwischen den beiden Adressen wird jetzt berechnet. 

\clearpage
Der \SUB Befehl wurde gerade ausgef�hrt:

\begin{figure}[H]
\centering
\myincludegraphics{patterns/10_strings/1_strlen/olly4.png}
\caption{\olly: \EAX muss dekrementiert werden}
\label{fig:strlen_olly_4}
\end{figure}

Die Differenz der Pointer im \EAX Register betr�gt nun--7.
Tats�chlich betr�gt die L�nge des \q{hello!} Strings 6 Zeichen, aber mit dem
Nullbyte am Ende dazugez�hlt sind es 7.
Die Funktion \TT{strlen()} soll aber die Anzahl der Nicht-Null-Zeichen im String
zur�ckliefert, also wird einmal dekrementiert und der Funktionsaufruf
anschlie�end beendet.
