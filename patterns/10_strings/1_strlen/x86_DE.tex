\subsubsection{x86}

\myparagraph{\NonOptimizing MSVC}

Kompilieren wir es:

\lstinputlisting[style=customasm]{patterns/10_strings/1_strlen/10_1_msvc_DE.asm}

\myindex{x86!\Instructions!MOVSX}
\myindex{x86!\Instructions!TEST}

Wir finden hier zwei neue Befehle: \MOVSX und \TEST.

\label{MOVSX}

Der erste --\MOVSX--nimmt ein Byte aus einer Speicheradresse und speichert den
Wert in einem 32-bit-Register.
\MOVSX steht f�r \IT{MOV with Sign-Extend}.
\MOVSX setzt die �brigen Bits vom 8. bis zum 31. auf 1, falls das Quellbyte
\IT{negativ} ist oder auf 0, falls es \IT{positiv} ist.

Und hier ist der Grund daf�r.

Standardm��ig ist der \Tchar Datentyp in MSVC und GCC vorzeichenbehaftet
(signed). Wenn wir zwei Werte haben, einen \Tchar und einen \Tint, (\Tint ist
ebenfalls vorzeichenbehaftet) und der erste Wert enth�lt -2 (kodiert als
\TT{0xFE}) und wir kopieren dieses Byte in den \Tint Container, erhalten wir
\TT{0x000000FE} und dies entspricht als signed \Tint 254, aber nicht -2. Der
signed \Tint -2 wird als \TT{0xFFFFFFFE} dargestellt. Wenn wir also \TT{0xFE}
vom Datentyp \Tchar nach \Tint �bertragen wollen, m�ssen wir das Vorzeichen
identifizieren und den Wert entsprechend erweitern. Genau dies tut der Befehl
\MOVSX.

Weitere Informationen dazu finden sich im Abschnitt
\q{\IT{\SignedNumbersSectionName}} ~(\myref{sec:signednumbers}).

Es ist schwer zu sagen, ob der Compiler tats�chlich eine \Tchar Variable in \EDX
speichern muss, er k�nnte auch einen 8-Bit-Registerteil (z.B. \DL) daf�r
verwenden . Offenbar arbeitet der \gls{register allocator} des Compilers auf
diese Art.

\myindex{ARM!\Instructions!TEST}

Wir finden im Weiteren den Befehl \TT{TEST EDX, EDX}. 
F�r mehr Informationen zum \TEST Befehl siehe auch den Abschnitt �ber
Bitfelder~(\myref{sec:bitfields}).
In unserem Fall �berpr�ft der Befehl lediglich, ob der Wert im Register \EDX
gleich 0 ist.

\myparagraph{\NonOptimizing GCC}

Schauen wir uns GCC 4.4.1 an:

\lstinputlisting[style=customasm]{patterns/10_strings/1_strlen/10_3_gcc.asm}

\label{movzx}
\myindex{x86!\Instructions!MOVZX}

Das Ergebnis ist fast identisch mit dem von MSVC, aber hier finden wir \MOVZX
anstelle von \MOVSX. 
\MOVZX steht f�r \IT{MOV with Zero-Extend}. 
Dieser Befehl kopiert einen 8-Bit- oder 16-Bit-Wert in ein 32-Bit-Register und
setzt die �brigen Bits auf 0.
Tats�chlich findet dieser Befehl vor allem deshalb Anwendung, weil er es uns
erlaubt, folgendes Befehlspaar zu ersetzen:\\
\TT{xor eax, eax / mov al, [...]}.

Andererseits ist offensichtlich, dass der Compiler folgenden Code erzeugen kann:
\\
\TT{mov al, byte ptr [eax] / test al, al}--es ist fast das gleiche, aber die
oberen Bits des \EAX Registers enthalten hier Zufallswerte bzw.
sogenanntes Zufallsrauschen.
Aber bedenken wir den Nachteil des Compilers--er kann nicht leichter
verst�ndlichen Code erzeugen. 
Genau genommen, ist der Compiler �berhaupt nicht daran gebunden, (Menschen)
verst�ndlichen Code zu erzeugen.

\myindex{x86!\Instructions!SETcc}

Der n�chste neue Befehl f�r uns ist \SETNZ.
In diesem Fall setzt \TT{test al,al} das \ZF flag auf 0, falls \AL nicht 0
enth�t, aber \SETNZ setzt \AL auf 1, falls \TT{ZF==0} (IT{NZ} steht f�r
\IT{non zero}).
In nat�rlicher Sprache, \IT{falls \AL ungleich 0, springe zu loc\_80483F0}. 
Der Compiler erzeugt leicht redundanten Code, aber bedenken wir, dass die
Optimierung hier deaktiviert ist.

\myparagraph{\Optimizing MSVC}
\label{strlen_MSVC_Ox}

Kompilieren wir nun alles in MSVC 2012 mit aktivierter Optimierung (\Ox):

\lstinputlisting[caption=\Optimizing MSVC 2012
/Ob0,style=customasm]{patterns/10_strings/1_strlen/10_2_DE.asm}

Jetzt ist alles einfacher.
Unn�tig zu erw�hnen, dass der Compiler Register mit solcher Effizienz nur in
kleinen Funktionen mit einigen wenigen lokalen Variablen verwenden kann.

\myindex{x86!\Instructions!INC}
\myindex{x86!\Instructions!DEC}
\INC/\DEC---sind \gls{inkrement}/\gls{dekrement} Befehle; mit anderen Worten:
addiere oder subtrahiere 1 zu bzw. von einer Variable. 

\clearpage
\mysubparagraph{\olly}
\myindex{\olly}

Let's try this example in \olly.
The input value of the function (2) is loaded into \EAX: 

\begin{figure}[H]
\centering
\myincludegraphics{patterns/08_switch/2_lot/olly1.png}
\caption{\olly: function's input value is loaded in \EAX}
\label{fig:switch_lot_olly1}
\end{figure}

\clearpage
The input value is checked, is it bigger than 4? 
If not, the \q{default} jump is not taken:
\begin{figure}[H]
\centering
\myincludegraphics{patterns/08_switch/2_lot/olly2.png}
\caption{\olly: 2 is no bigger than 4: no jump is taken}
\label{fig:switch_lot_olly2}
\end{figure}

\clearpage
Here we see a jumptable:

\begin{figure}[H]
\centering
\myincludegraphics{patterns/08_switch/2_lot/olly3.png}
\caption{\olly: calculating destination address using jumptable}
\label{fig:switch_lot_olly3}
\end{figure}

Here we've clicked \q{Follow in Dump} $\rightarrow$ \q{Address constant}, so now we see the \IT{jumptable} in the data window.
These are 5 32-bit values\footnote{They are underlined by \olly because
these are also FIXUPs: \myref{subsec:relocs}, we are going to come back to them later}.
\ECX is now 2, so the third element (can be indexed as 2\footnote{About indexing, see also: \ref{arrays_at_one}}) of the table is to be used.
It's also possible to click \q{Follow in Dump} $\rightarrow$ 
\q{Memory address} and \olly will show the element addressed by the \JMP instruction. 
That's \TT{0x010B103A}.

\clearpage
After the jump we are at \TT{0x010B103A}: the code printing \q{two} will now be executed:

\begin{figure}[H]
\centering
\myincludegraphics{patterns/08_switch/2_lot/olly4.png}
\caption{\olly: now we at the \IT{case:} label}
\label{fig:switch_lot_olly4}
\end{figure}


\myparagraph{\Optimizing GCC}

Schauen wir uns GCC 4.4.1 mit aktiverter Optimierung (\Othree key) an:

\lstinputlisting[style=customasm]{patterns/10_strings/1_strlen/10_3_gcc_O3.asm}
 
Hier erzeugt GCC fast identischen Code zu MSVC, au�er dass hier ein \MOVZX
auftritt. 
In der Tat k�nnte \MOVZX hier durch \TT{mov dl, byte ptr [eax]} ersetzt werden.
 
M�glicherweise ist es einfacher f�r den GCC Code Generator sich daran zu
\IT{erinnern}, dass das gesamte 32-bit-\EDX Register f�r eine \Tchar Variable
reserviert ist und so sicherzustellen, dass die oberen Bits zu keinem Zeitpunkt
Zufallsrauschen enthalten.

\label{strlen_NOT_ADD}
\myindex{x86!\Instructions!NOT}
\myindex{x86!\Instructions!XOR}

Danach finden wir also einen neuen Befehl--\NOT. Dieser Befehl kippt alle Bits
in seinem Operanden.\\
Man kann sagen, dass es sich um ein Synonym zum Befehl \TT{XOR ECX, 0ffffffffh}
handelt. 
\NOT und das darauf folgende \ADD berechnen die Differenz im Pointer und
subtrahieren 1, nur auf eine andere Art und Weise. 
Zu Beginn wird \ECX, in dem der Pointer auf \IT{str} gespeichert ist, invertiert
und vom Ergebnis wird 1 abgezogen.

Hierzu siehe auch: \q{\SignedNumbersSectionName}~(\myref{sec:signednumbers}).
 
Mit anderen Worten, am Ende der Funktion, direkt nach dem Schleifenk�rper,
werden die folgenden Befehle ausgef�hrt:

\begin{lstlisting}
ecx=str;
eax=eos;
ecx=(-ecx)-1; 
eax=eax+ecx
return eax
\end{lstlisting}

\dots~und das ist �quivalent zu:

\begin{lstlisting}
ecx=str;
eax=eos;
eax=eax-ecx;
eax=eax-1;
return eax
\end{lstlisting}

Warum GCC entschieden hat, dass das eine besser ist als das andere? Schwer zu
sagen.
M�glicherweise sind aber beide Variante gleicherma�en effizient.
