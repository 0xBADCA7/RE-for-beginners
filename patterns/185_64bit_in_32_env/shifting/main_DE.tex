\subsection{Verschiebung nach rechts}

\lstinputlisting[style=customc]{patterns/185_64bit_in_32_env/shifting/3.c}

\subsubsection{x86}

\lstinputlisting[caption=\Optimizing MSVC 2012 /Ob1,style=customasmx86]{patterns/185_64bit_in_32_env/shifting/3_MSVC.asm}

\lstinputlisting[caption=\Optimizing GCC 4.8.1 -fno-inline,style=customasmx86]{patterns/185_64bit_in_32_env/shifting/3_GCC.asm}

\myindex{x86!\Instructions!SHRD}
Das Verschieben geschieht ebenfalls zweigeteilt: zunächst wird der niedere Teil verschoben, danach der höhere.
Der niedere Teil wird mithilfe des Befehls \INS{SHRD} verschoben; er verschiebt den Wert in \EAX um 7 Bits, holt aber
die nachrutschenden Bits aus \EDX, d.h. aus dem höheren Teil.
Mit anderen Worten: der 64-Bit-Wert aus \TT{EDX:EAX} wird als ganzes um 7 Bits verschoben und die niederen 32 Bits des
Ergebnisses werden in \EAX abgelegt. Der höhere Teil wird mit dem häufig verwendeten \SHR Befehl verschoben, da die frei
werdenden Bits im höheren Teil mit Nullen aufgefüllt werden müssen.

\subsubsection{ARM}
ARM verfügt im Gegensatz zu x86 nicht über einen \IN{SHRD} Befehl, sodass der Keil Compiler die Aufgabe mit einer
Kombination aus einfachen Schiebebefehlen und \OR-Operationen durchführen muss:

\lstinputlisting[caption=\OptimizingKeilVI (\ARMMode),style=customasmARM]{patterns/185_64bit_in_32_env/shifting/Keil_ARM_O3.s}

\lstinputlisting[caption=\OptimizingKeilVI (\ThumbMode),style=customasmARM]{patterns/185_64bit_in_32_env/shifting/Keil_thumb_O3.s}
% TODO add explanation

\subsubsection{MIPS}
GCC für MIPS folgt dem gleichen Algorithmus wie Keil für Thumb mode:

\lstinputlisting[caption=\Optimizing GCC 4.4.5 (IDA),style=customasmMIPS]{patterns/185_64bit_in_32_env/shifting/MIPS_O3_IDA.lst}

% TODO add explanation

