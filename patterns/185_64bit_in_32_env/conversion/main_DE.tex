\subsection{32-Bit-Werte in 64-Bit-Werte umwandeln}
\label{subsec:sign_extending_32_to_64}

\lstinputlisting[style=customc]{patterns/185_64bit_in_32_env/conversion/4.c}

\subsubsection{x86}

\lstinputlisting[caption=\Optimizing MSVC 2012,style=customasmx86]{patterns/185_64bit_in_32_env/conversion/MSVC2012_Ox.asm}
Hier müssen wir einen vorzeichenbehafteten 32-Bit-Wert in einen 64-Bit-Wert mit Vorzeichen umwandeln.
Vorzeichenlose Werte werden ganz einfach umgewandelt: alle Bits im höheren Teil werden auf 0 gesetzt.
Bei Werten mit Vorzeichen funktioniert diese Methode nicht: das Vorzeichen muss in den höheren Teil des
32-Bit-Ergebnisses kopiert werden.
\myindex{x86!\Instructions!CDQ}
Der Befehl \INS{CDQ} übernimmt hier diese Aufgabe. Er nimmt seinen Eingabewert aus \EAX, erweitert ihn auf 64 Bit und
speichert das Ergebnis im Registerpaar \EDX/\EAX.
Mit anderen Worten: \INS{CDQ} extrahiert das Vorzeichen aus \EAX (dies ist das MSB in \EAX) und setzt je nach Wert des
Vorzeichens alle 32 Bits in \EDX auf 0 oder 1.
Dieser Befehl ähnelt dem Befehl \MOVSX.

\subsubsection{ARM}

\lstinputlisting[caption=\OptimizingKeilVI (\ARMMode),style=customasmARM]{patterns/185_64bit_in_32_env/conversion/Keil_ARM_O3.s}
Keil für ARM geht anders vor: er verschiebt den Eingabewert um 31 Bits nach rechts.
Wie wir wissen ist das Vorzeichenbit das \ac{MSB} und die arithmetische Verschiebung kopiert das Vorzeichenbit in die
ausgeschobenen Bits. Nach Ausführung von \q{ASR r1,r0,\#31} enthält \Reg{1} also 0xFFFFFFFF, falls der Eingabewert
negativ war und ansonsten 0.
\Reg{1} enthält den höheren Teil des resultierenden 64-Bit-Wertes.
Mit anderen Worten: dieser Code kopiert das \ac{MSB} (Vorzeichenbit) vom Eingabewert in \Reg{0} in alle Bits der höheren
32 Bits des Ergebnisses.

\subsubsection{MIPS}
GCC für MIPS erzeugt das gleich wie Keil für ARM mode:

\lstinputlisting[caption=\Optimizing GCC 4.4.5 (IDA),style=customasmMIPS]{patterns/185_64bit_in_32_env/conversion/MIPS_O3_IDA.lst}
