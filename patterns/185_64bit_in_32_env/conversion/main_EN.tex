\subsection{Converting 32-bit value into 64-bit one}
\label{subsec:sign_extending_32_to_64}

\lstinputlisting[style=customc]{patterns/185_64bit_in_32_env/conversion/4.c}

\subsubsection{x86}

\lstinputlisting[caption=\Optimizing MSVC 2012,style=customasm]{patterns/185_64bit_in_32_env/conversion/MSVC2012_Ox.asm}

Here we also run into necessity to extend a 32-bit signed value into a 64-bit signed one.
Unsigned values are converted straightforwardly: all bits in the higher part must be set to 0.
But this is not appropriate for signed data types: the sign has to be copied into the higher part of the resulting number.
\myindex{x86!\Instructions!CDQ}

The \INS{CDQ} instruction does that here, it takes its input value in \EAX{}, extends it to 64-bit and leaves it
in the \EDX{}:\EAX{} register pair.
In other words, \INS{CDQ} gets the number sign from \EAX{} (by getting the
most significant bit in \EAX{}), and depending of it, sets all 32 bits in \EDX{} to 0 or 1.
Its operation is somewhat
similar to the \MOVSX{} instruction.

\subsubsection{ARM}

\lstinputlisting[caption=\OptimizingKeilVI (\ARMMode),style=customasm]{patterns/185_64bit_in_32_env/conversion/Keil_ARM_O3.s}

Keil for ARM is different: it just arithmetically shifts right the input value by 31 bits.
As we know, the sign bit is \ac{MSB}, and the arithmetical shift copies the sign bit into the \q{emerged} bits.
So after \q{ASR r1,r0,\#31}, \Reg{1} containing 0xFFFFFFFF if the input value has been negative and 0 otherwise.
\Reg{1} contains the high part of the resulting 64-bit value.
In other words, this code just copies the \ac{MSB} (sign bit) from the input value in \Reg{0} to all bits of the high 32-bit part of the resulting 64-bit value.

\subsubsection{MIPS}

GCC for MIPS does the same as Keil did for ARM mode:

\lstinputlisting[caption=\Optimizing GCC 4.4.5 (IDA),style=customasm]{patterns/185_64bit_in_32_env/conversion/MIPS_O3_IDA.lst}
