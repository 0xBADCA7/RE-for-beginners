\ifdefined\ENGLISH
\section{64-bit values in 32-bit environment}
\label{sec:64bit_in_32_env}

In a 32-bit environment, \ac{GPR}'s are 32-bit, so 64-bit values are stored and passed as 32-bit value pairs
\footnote{By the way, 32-bit values are passed as pairs in 16-bit environment in the same way: \myref{win16_32bit_values}}.
\fi

\ifdefined\RUSSIAN
\section{64-битные значения в 32-битной среде}
\label{sec:64bit_in_32_env}

В среде, где \ac{GPR}-ы 32-битные, 64-битные значения хранятся и передаются как пары 32-битных значений
\footnote{Кстати, в 16-битной среде, 32-битные значения передаются 16-битными парами точно так же: \myref{win16_32bit_values}}.
\fi

\ifdefined\GERMAN
\section{64-Bit-Werte in 32-Bit-Umgebungen}
\label{sec:64bit_in_32_env}

In einer 32-Bit-Umgebung sind \ac{GPR} 32 Bit groß. Also werden 64-Bit-Werte in
32-Bit-Wertepaaren gespeichert und übergeben\footnote{Übrigens, 32-Bit-Werte werden
als Paare in 16--Bit-Umgebungen auf der gleiche Art übergeben: \myref{win16_32bit_values}}.
\fi

\ifdefined\FRENCH
\section{Valeurs 64-bit dans un environnement 32-bit}
\label{sec:64bit_in_32_env}

Dans un environnement 32-bit, les \ac{GPR} sont 32-bit, donc les valeurs 64-bit sont
stockées et passées comme une paire de registres 32-bit\footnote{Á propos, les valeurs
32-bit sont passées en tant que paire dans les environnements 16-bit de la même manière:
\myref{win16_32bit_values}}.
\fi

\EN{\subsection{Returning of 64-bit value}

\lstinputlisting[style=customc]{patterns/185_64bit_in_32_env/ret/0.c}

\subsubsection{x86}

In a 32-bit environment, 64-bit values are returned from functions in the \EDX{}:\EAX{} register pair.

\lstinputlisting[caption=\Optimizing MSVC 2010,style=customasmx86]{patterns/185_64bit_in_32_env/ret/0_MSVC_2010_Ox.asm}

\subsubsection{ARM}

A 64-bit value is returned in the \Reg{0}-\Reg{1} register pair (\Reg{1} is for the high part and \Reg{0} for the low part):

\lstinputlisting[caption=\OptimizingKeilVI (\ARMMode),style=customasmARM]{patterns/185_64bit_in_32_env/ret/Keil_ARM_O3.s}

\subsubsection{MIPS}

A 64-bit value is returned in the \TT{V0}-\TT{V1} (\$2-\$3) register pair (\TT{V0} (\$2) is for the high part and \TT{V1} (\$3) for the low part):

\lstinputlisting[caption=\Optimizing GCC 4.4.5 (assembly listing),style=customasmMIPS]{patterns/185_64bit_in_32_env/ret/0_MIPS.s}

\lstinputlisting[caption=\Optimizing GCC 4.4.5 (IDA),style=customasmMIPS]{patterns/185_64bit_in_32_env/ret/0_MIPS_IDA.lst}
}
\RU{\subsection{Возврат 64-битного значения}

\lstinputlisting[style=customc]{patterns/185_64bit_in_32_env/ret/0.c}

\subsubsection{x86}

64-битные значения в 32-битной среде возвращаются из функций в паре регистров \EDX{}:\EAX{}.

\lstinputlisting[caption=\Optimizing MSVC 2010,style=customasmx86]{patterns/185_64bit_in_32_env/ret/0_MSVC_2010_Ox.asm}

\subsubsection{ARM}

64-битное значение возвращается в паре регистров \Reg{0}-\Reg{1} --- (\Reg{1} это старшая часть и \Reg{0} --- младшая часть):

\lstinputlisting[caption=\OptimizingKeilVI (\ARMMode),style=customasmARM]{patterns/185_64bit_in_32_env/ret/Keil_ARM_O3.s}

\subsubsection{MIPS}

64-битное значение возвращается в паре регистров \TT{V0}-\TT{V1} (\$2-\$3) --- (\TT{V0} (\$2) это старшая часть и \TT{V1} (\$3) --- младшая часть):

\lstinputlisting[caption=\Optimizing GCC 4.4.5 (assembly listing),style=customasmMIPS]{patterns/185_64bit_in_32_env/ret/0_MIPS.s}

\lstinputlisting[caption=\Optimizing GCC 4.4.5 (IDA),style=customasmMIPS]{patterns/185_64bit_in_32_env/ret/0_MIPS_IDA.lst}

}
\DE{\subsection{Rückgabe von 64-Bit-Werten}

\lstinputlisting[style=customc]{patterns/185_64bit_in_32_env/ret/0.c}

\subsubsection{x86}
In einer 32-Bit-Umgebung werden 64-Bit-Werte von Funktionen über das Registerpaar \EDX:\EAX zurückgegeben:

\lstinputlisting[caption=\Optimizing MSVC 2010,style=customasmx86]{patterns/185_64bit_in_32_env/ret/0_MSVC_2010_Ox.asm}

\subsubsection{ARM}
Ein 64-Bit-Wert wird über das \Reg{0}-\Reg{1} Registerpaar zurückgegeben (\Reg{1} enthält dabei den höheren und \Reg{0}
den niederen Teil):

\lstinputlisting[caption=\OptimizingKeilVI (\ARMMode),style=customasmARM]{patterns/185_64bit_in_32_env/ret/Keil_ARM_O3.s}

\subsubsection{MIPS}

Ein 64-Bit-Wert wird über das \TT{V0}-\TT{V1} (\$2-\$3) Registerpaar zurückgegeben (\TT{V0} (\$2) enthält dabei den
höheren und \TT{V1} (\$3) den niederen Teil):

\lstinputlisting[caption=\Optimizing GCC 4.4.5 (assembly listing),style=customasmMIPS]{patterns/185_64bit_in_32_env/ret/0_MIPS.s}

\lstinputlisting[caption=\Optimizing GCC 4.4.5 (IDA),style=customasmMIPS]{patterns/185_64bit_in_32_env/ret/0_MIPS_IDA.lst}
}
\FR{\subsection{Renvoyer une valeur 64-bit}

\lstinputlisting[style=customc]{patterns/185_64bit_in_32_env/ret/0.c}

\subsubsection{x86}

Dans un environnement 32-bit, les valeurs 64-bit sont renvoyées des fonctions dans
la paire de registres \EDX{}:\EAX{}.

\lstinputlisting[caption=MSVC 2010 \Optimizing,style=customasmx86]{patterns/185_64bit_in_32_env/ret/0_MSVC_2010_Ox.asm}

\subsubsection{ARM}

Une valeur 64-bit est renvoyée dans la paire de registres \Reg{0}-\Reg{1} (\Reg{1}
est pour la partie haute et \Reg{0} pour la partie basse):

\lstinputlisting[caption=\OptimizingKeilVI (\ARMMode),style=customasmARM]{patterns/185_64bit_in_32_env/ret/Keil_ARM_O3.s}

\subsubsection{MIPS}

Une valeur 64-bit est renvoyée dans la paire de registres \TT{V0}-\TT{V1} (\$2-\$3)
(\TT{V0} (\$2) est pour la partie haute et \TT{V1} (\$3) pour la partie basse):

\lstinputlisting[caption=GCC 4.4.5 \Optimizing (assembly listing),style=customasmMIPS]{patterns/185_64bit_in_32_env/ret/0_MIPS.s}

\lstinputlisting[caption=GCC 4.4.5 \Optimizing (IDA),style=customasmMIPS]{patterns/185_64bit_in_32_env/ret/0_MIPS_IDA.lst}
}

\EN{\subsection{Arguments passing, addition, subtraction}

\lstinputlisting[style=customc]{patterns/185_64bit_in_32_env/passing_add_sub/1.c}

\subsubsection{x86}

\lstinputlisting[caption=\Optimizing MSVC 2012 /Ob1,style=customasmx86]{patterns/185_64bit_in_32_env/passing_add_sub/1_MSVC.asm}

We can see in the \GTT{f\_add\_test()} function that each 64-bit value is passed using two 32-bit values,
high part first, then low part. 

Addition and subtraction occur in pairs as well.

\myindex{x86!\Instructions!ADC}
In addition, the low 32-bit part are added first.
If carry has been occurred while adding, the \TT{CF} flag is set.

The following \INS{ADC} instruction adds the high parts of the values, and also adds 1 if $CF=1$.

\myindex{x86!\Instructions!SBB}
Subtraction also occurs in pairs.
The first \SUB may also turn on the CF flag, which is to be checked in the subsequent \INS{SBB} instruction:
if the carry flag is on, then 1 is also to be subtracted from the result.

It is easy to see how the \GTT{f\_add()} function result is then passed to \printf{}.

\lstinputlisting[caption=GCC 4.8.1 -O1 -fno-inline,style=customasmx86]{patterns/185_64bit_in_32_env/passing_add_sub/1_GCC.asm}

GCC code is the same.

\subsubsection{ARM}

\lstinputlisting[caption=\OptimizingKeilVI (\ARMMode),style=customasmARM]{patterns/185_64bit_in_32_env/passing_add_sub/Keil_ARM_O3.s}

\myindex{ARM!\Instructions!ADDS}
\myindex{ARM!\Instructions!SUBS}
\myindex{ARM!\Instructions!ADC}
\myindex{ARM!\Instructions!SBC}

The first 64-bit value is passed in \Reg{0} and \Reg{1} register pair, the second in \Reg{2} and \Reg{3} register pair.
ARM has the \INS{ADC} instruction as well (which counts carry flag) and \INS{SBC} (\q{subtract with carry}).
Important thing: when the low parts are added/subtracted, \INS{ADDS} and \INS{SUBS} instructions with -S suffix are used.
The -S suffix stands for \q{set flags}, and flags (esp. carry flag) is what consequent \INS{ADC}/\INS{SBC} instructions definitely need.
Otherwise, instructions without the -S suffix would do the job (\ADD and \SUB).

\subsubsection{MIPS}

\lstinputlisting[caption=\Optimizing GCC 4.4.5 (IDA),style=customasmMIPS]{patterns/185_64bit_in_32_env/passing_add_sub/MIPS_O3_IDA_EN.lst}

MIPS has no flags register, so there is no such information present after the execution of arithmetic operations.
So there are no instructions like x86's \INS{ADC} and \INS{SBB}.
To know if the carry flag would be set, a comparison (using \INS{SLTU} instruction) also occurs, 
which sets the destination register to 1 or 0.
This 1 or 0 is then added or subtracted to/from the final result.

}
\RU{\subsection{Передача аргументов, сложение, вычитание}

\lstinputlisting[style=customc]{patterns/185_64bit_in_32_env/passing_add_sub/1.c}

\subsubsection{x86}

\lstinputlisting[caption=\Optimizing MSVC 2012 /Ob1,style=customasm]{patterns/185_64bit_in_32_env/passing_add_sub/1_MSVC.asm}

В \GTT{f\_add\_test()} видно, как каждое 64-битное число передается двумя 32-битными значениями,
сначала старшая часть, затем младшая.

Сложение и вычитание происходит также парами. 

\myindex{x86!\Instructions!ADC}
При сложении, в начале складываются младшие 32 бита.
Если при сложении был перенос, выставляется флаг CF.
Следующая инструкция \INS{ADC} складывает старшие части чисел, но также прибавляет единицу если $CF=1$.

\myindex{x86!\Instructions!SBB}
Вычитание также происходит парами.
Первый \SUB может также включить флаг переноса CF, который затем будет проверяться в \INS{SBB}:
если флаг переноса включен, то от результата отнимется единица.

Легко увидеть, как результат работы \GTT{f\_add()} затем передается в \printf{}.

\lstinputlisting[caption=GCC 4.8.1 -O1 -fno-inline,style=customasm]{patterns/185_64bit_in_32_env/passing_add_sub/1_GCC.asm}

Код GCC почти такой же.

\subsubsection{ARM}

\lstinputlisting[caption=\OptimizingKeilVI (\ARMMode),style=customasm]{patterns/185_64bit_in_32_env/passing_add_sub/Keil_ARM_O3.s}

\myindex{ARM!\Instructions!ADDS}
\myindex{ARM!\Instructions!SUBS}
\myindex{ARM!\Instructions!ADC}
\myindex{ARM!\Instructions!SBC}
Первое 64-битное значение передается в паре регистров \Reg{0} и \Reg{1}, второе --- в паре \Reg{2} и \Reg{3}.
В ARM также есть инструкция \INS{ADC} (учитывающая флаг переноса) и \INS{SBC} (\q{subtract with carry} --- вычесть с переносом).
Важная вещь: когда младшие части слагаются/вычитаются, используются инструкции \INS{ADDS} и \INS{SUBS} с суффиксом -S.
Суффикс -S означает \q{set flags} (установить флаги), а флаги (особенно флаг переноса) это то что однозначно нужно последующим инструкциями \INS{ADC}/\INS{SBC}.
А иначе инструкции без суффикса -S здесь вполне бы подошли (\ADD и \SUB).

\subsubsection{MIPS}

\lstinputlisting[caption=\Optimizing GCC 4.4.5 (IDA),style=customasm]{patterns/185_64bit_in_32_env/passing_add_sub/MIPS_O3_IDA_RU.lst}

В MIPS нет регистра флагов, так что эта информация не присутствует после исполнения арифметических операций.

Так что здесь нет инструкций как \INS{ADC} или \INS{SBB} в x86.
Чтобы получить информацию о том, был бы выставлен флаг переноса, происходит сравнение (используя инструкцию
\INS{SLTU}), которая выставляет целевой регистр в 1 или 0.

Эта 1 или 0 затем прибавляется к итоговому результату, или вычитается.

}
\DE{\subsection{Übergabe von Argumenten bei Addition und Subtraktion}

\lstinputlisting[style=customc]{patterns/185_64bit_in_32_env/passing_add_sub/1.c}

\subsubsection{x86}

\lstinputlisting[caption=\Optimizing MSVC 2012 /Ob1,style=customasmx86]{patterns/185_64bit_in_32_env/passing_add_sub/1_MSVC.asm}
Wir sehen, dass in der Funktion \GTT{f\_add\_test()} jeder 64-Bit-Wert über zwei 32-Bit-Werten übergeben wird: zuerst
der höhere Teil, dann der niedere Teil.

Addition und Subtraktion werden auch mit Paaren ausgeführt.

\myindex{x86!\Instructions!ADC}
Bei einer Addition werden die niederen 32-Bit-Teile zuerst addiert.
Tritt hierbei ein Übertrag auf, wird das \CF Flag gesetzt.

Der folgende \INS{ADC} Befehl addiert die höheren Teile der Operanden und addiert 1, falls $CF=1$.

\myindex{x86!\Instructions!SBB}
Subtraktion wird auch mit den Wertepaaren durchgeführt.
Das erste \SUB setzt ggf. das \CF Flag, das von dem folgenden \INS{SBB} Befehl geprüft wird:
wenn das Carryflag gesetzt ist, wird am Ende 1 vom Ergebnis abgezogen.

Man erkennt im Code leicht, wie das Ergebnis der Funktion \GTT{f\_add()} an \printf übergeben wird.

\lstinputlisting[caption=GCC 4.8.1 -O1 -fno-inline,style=customasmx86]{patterns/185_64bit_in_32_env/passing_add_sub/1_GCC.asm}

Der Code von GCC ist identisch.

\subsubsection{ARM}

\lstinputlisting[caption=\OptimizingKeilVI (\ARMMode),style=customasmARM]{patterns/185_64bit_in_32_env/passing_add_sub/Keil_ARM_O3.s}

\myindex{ARM!\Instructions!ADDS}
\myindex{ARM!\Instructions!SUBS}
\myindex{ARM!\Instructions!ADC}
\myindex{ARM!\Instructions!SBC}
Der erste 64-Bit-Wert wird über das Registerpaar \Reg{0} und \Reg{1} übergeben, der zweite über \Reg{2} und \Reg{3}.
ARM verfügt ebenfalls über die Befehle \INS{ADC} und \INS{SBC} (die das Carryflag beachten).
Man beachte: wenn die niederen Teile addiert bzw. subtrahiert werden, werden \INS{ADDS} und \INS{SUBS} Befehle mit dem
Suffy -S verwendet. Dieser Suffix steht für \q{set flags} (dt. setze Flags) und wird von den folgenden
\INS{ADC}/\INS{SBC} Befehlen unbedingt benötigt. Würden die Flags nicht weiter beachtet werden, könnten hier \ADD und
\SUB verwendet werden.

\subsubsection{MIPS}

\lstinputlisting[caption=\Optimizing GCC 4.4.5
(IDA),style=customasmMIPS]{patterns/185_64bit_in_32_env/passing_add_sub/MIPS_O3_IDA_DE.lst}
MIPS besitzt kein Register für die Flags, sodass keine derartige Information nach der Ausführung von arithmetischen
Operationen verfügbar ist.
Es gibt also keine Befehle wie \INS{ADC} oder \INS{SBB} in x86.
Um zu prüfen, ob das Carryflag gesetzt werden muss, wird ein \INS{SLTU} Befehl verwendet, der das Zielregister auf 1
oder 0 setzt. Diese 1 oder 0 wird dann zum Ergebnis addiert bzw. davon subtrahiert.

}

\EN{\subsection{Multiplication, division}

\lstinputlisting[style=customc]{patterns/185_64bit_in_32_env/multdiv/2.c}

\subsubsection{x86}

\lstinputlisting[caption=\Optimizing MSVC 2013 /Ob1,style=customasmx86]{patterns/185_64bit_in_32_env/multdiv/2_MSVC_EN.asm}

Multiplication and division are more complex operations, so usually the compiler embeds calls to
a library functions doing that.

These functions are described here: \myref{sec:MSVC_library_func}.

\lstinputlisting[caption=\Optimizing GCC 4.8.1 -fno-inline,style=customasmx86]{patterns/185_64bit_in_32_env/multdiv/2_GCC_EN.asm}

GCC does the expected, but the multiplication
code is inlined right in the function, thinking it could be more efficient.
GCC has different library function names: \myref{sec:GCC_library_func}.

\subsubsection{ARM}

Keil for Thumb mode inserts library subroutine calls:

\lstinputlisting[caption=\OptimizingKeilVI (\ThumbMode),style=customasmARM]{patterns/185_64bit_in_32_env/multdiv/Keil_thumb_O3.s}

Keil for ARM mode, on the other hand, is able to produce 64-bit multiplication code:

\lstinputlisting[caption=\OptimizingKeilVI (\ARMMode),style=customasmARM]{patterns/185_64bit_in_32_env/multdiv/Keil_ARM_O3.s}
% TODO add explanation

\subsubsection{MIPS}

\Optimizing GCC for MIPS can generate 64-bit multiplication code, but has to call a library routine for 64-bit division:

\lstinputlisting[caption=\Optimizing GCC 4.4.5 (IDA),style=customasmMIPS]{patterns/185_64bit_in_32_env/multdiv/MIPS_O3_IDA.lst}

There are a lot of \ac{NOP}s, probably delay slots filled after the multiplication instruction (it's slower
than other instructions, after all).

% TODO add explanation
}
\RU{\subsection{Умножение, деление}

\lstinputlisting[style=customc]{patterns/185_64bit_in_32_env/multdiv/2.c}

\subsubsection{x86}

\lstinputlisting[caption=\Optimizing MSVC 2013 /Ob1,style=customasmx86]{patterns/185_64bit_in_32_env/multdiv/2_MSVC_RU.asm}

Умножение и деление --- это более сложная операция, так что обычно, компилятор встраивает вызовы библиотечных функций,
делающих это.

Значение этих библиотечных функций, здесь: \myref{sec:MSVC_library_func}.

\lstinputlisting[caption=\Optimizing GCC 4.8.1 -fno-inline,style=customasmx86]{patterns/185_64bit_in_32_env/multdiv/2_GCC_RU.asm}

GCC делает почти то же самое, тем не менее,
встраивает код умножения прямо в функцию, посчитав что так будет эффективнее.
У GCC другие имена библиотечных функций: \myref{sec:GCC_library_func}.

\subsubsection{ARM}

Keil для режима Thumb вставляет вызовы библиотечных функций:

\lstinputlisting[caption=\OptimizingKeilVI (\ThumbMode),style=customasmARM]{patterns/185_64bit_in_32_env/multdiv/Keil_thumb_O3.s}

Keil для режима ARM, тем не менее, может сгенерировать код для умножения 64-битных чисел:


\lstinputlisting[caption=\OptimizingKeilVI (\ARMMode),style=customasmARM]{patterns/185_64bit_in_32_env/multdiv/Keil_ARM_O3.s}
% TODO add explanation

\subsubsection{MIPS}

\Optimizing GCC для MIPS может генерировать код для 64-битного умножения, но для 64-битного деления приходится вызывать библиотечную функцию:

\lstinputlisting[caption=\Optimizing GCC 4.4.5 (IDA),style=customasmMIPS]{patterns/185_64bit_in_32_env/multdiv/MIPS_O3_IDA.lst}

Тут также много \ac{NOP}-ов, это возможно заполнение delay slot-ов после инструкции умножения (она ведь работает
медленнее прочих инструкций).

% TODO add explanation
}
\DE{\subsection{Multiplikation und Division}

\lstinputlisting[style=customc]{patterns/185_64bit_in_32_env/multdiv/2.c}

\subsubsection{x86}

\lstinputlisting[caption=\Optimizing MSVC 2013
/Ob1,style=customasmx86]{patterns/185_64bit_in_32_env/multdiv/2_MSVC_DE.asm}
Da Multiplikation und Division komplexere Operationen sind, verwendet der Compiler für deren Ausführung in der Regel
Bibliotheksfunktionen.

Diese Funktionen werden hier beschrieben:\myref{sec:MSVC_library_func}.

\lstinputlisting[caption=\Optimizing GCC 4.8.1
-fno-inline,style=customasmx86]{patterns/185_64bit_in_32_env/multdiv/2_GCC_DE.asm}
GCC verhält sich wie erwartet, aber der Code für die Multiplikation wird direkt in der Funktion platziert und ist so
effizienter. In GCC haben die Bibliotheksfunktionen andere Namen: \myref{sec:GCC_library_func}.

\subsubsection{ARM}
Keil für Thumb mode fügt Aufrufe von Routinen aus Bibliotheken ein:

\lstinputlisting[caption=\OptimizingKeilVI (\ThumbMode),style=customasmARM]{patterns/185_64bit_in_32_env/multdiv/Keil_thumb_O3.s}
Keil für ARM mode ist wiederum in der Lage Code für 64-Bit-Multiplikation zu erzeugen:

\lstinputlisting[caption=\OptimizingKeilVI (\ARMMode),style=customasmARM]{patterns/185_64bit_in_32_env/multdiv/Keil_ARM_O3.s}
% TODO add explanation

\subsubsection{MIPS}
Der optimierende GCC für MIPS kann Code für 64-Bit-Multiplikation erzeugen, muss aber für die Division auf eine
Programmbibliothek zurückgreifen:

\lstinputlisting[caption=\Optimizing GCC 4.4.5 (IDA),style=customasmMIPS]{patterns/185_64bit_in_32_env/multdiv/MIPS_O3_IDA.lst}
Hier gibt es eine Menge \ac{NOP}s; möglicherweise sind dies delay slots, die nach dem Multiplikationsbefehl eingefügt
wurden (diese Vorgehensweise ist jedoch langsamer als andere Befehle).

% TODO add explanation
}
\FR{\subsection{Multiplication, division}

\lstinputlisting[style=customc]{patterns/185_64bit_in_32_env/multdiv/2.c}

\subsubsection{x86}

\lstinputlisting[caption=MSVC 2013 /Ob1 \Optimizing,style=customasmx86]{patterns/185_64bit_in_32_env/multdiv/2_MSVC_FR.asm}

La multiplication et la division sont des opérations plus complexes, donc en général
le compilateur embarque des appels à des fonctions de bibliothèque les effectuant.

Ces fonctions sont décrites ici: \myref{sec:MSVC_library_func}.

\lstinputlisting[caption=GCC 4.8.1 -fno-inline \Optimizing,style=customasmx86]{patterns/185_64bit_in_32_env/multdiv/2_GCC_FR.asm}

GCC fait ce que l'on attend, mais le code multiplication est mis en ligne (inlined)
directement dans la fonction, pensant que ça peut être plus efficace.
GCC a des noms de fonctions de bibliothèque différents: \myref{sec:GCC_library_func}.

\subsubsection{ARM}

Keil pour mode Thumb insère des appels à des sous-routines de bibliothèque:

\lstinputlisting[caption=\OptimizingKeilVI (\ThumbMode),style=customasmARM]{patterns/185_64bit_in_32_env/multdiv/Keil_thumb_O3.s}

Keil pour mode ARM, d'un autre côté, est capable de produire le code de la multiplication
64-bit:

\lstinputlisting[caption=\OptimizingKeilVI (\ARMMode),style=customasmARM]{patterns/185_64bit_in_32_env/multdiv/Keil_ARM_O3.s}
% TODO add explanation

\subsubsection{MIPS}

GCC \Optimizing pour MIPS peut générer du code pour la multiplication 64-bit, mais
doit appeler une routine de bibliothèque pour la division 64-bit:

\lstinputlisting[caption=GCC 4.4.5 \Optimizing (IDA),style=customasmMIPS]{patterns/185_64bit_in_32_env/multdiv/MIPS_O3_IDA.lst}

Il y a beacuoup de \ac{NOP}s, sans doute des slots de délai de remplissage après
l'instruction de multiplication (c'est plus lent que les autres instructions après
tout).

% TODO add explanation
}

\EN{\subsection{Shifting right}

\lstinputlisting[style=customc]{patterns/185_64bit_in_32_env/shifting/3.c}

\subsubsection{x86}

\lstinputlisting[caption=\Optimizing MSVC 2012 /Ob1,style=customasm]{patterns/185_64bit_in_32_env/shifting/3_MSVC.asm}

\lstinputlisting[caption=\Optimizing GCC 4.8.1 -fno-inline,style=customasm]{patterns/185_64bit_in_32_env/shifting/3_GCC.asm}

\myindex{x86!\Instructions!SHRD}

Shifting also occurs in two passes: first the lower part is shifted, then the higher part.
But the lower part is shifted with the help of the \INS{SHRD} instruction, it shifts the value of \EAX{} by 7 bits, but pulls new bits
from \EDX{}, i.e., from the higher part.
In other words, 64-bit value from \TT{EDX:EAX} register's pair, as a whole, is shifted by 7 bits and lowest 32 bits of result are placed into \EAX{}.
The higher part is shifted using the much more popular \SHR{} instruction: indeed, the freed bits in the higher part
must be filled with zeros.

\subsubsection{ARM}

ARM doesn't have such instruction as \INS{SHRD} in x86, so the Keil compiler ought to do this using simple shifts and \INS{OR} operations:

\lstinputlisting[caption=\OptimizingKeilVI (\ARMMode),style=customasm]{patterns/185_64bit_in_32_env/shifting/Keil_ARM_O3.s}

\lstinputlisting[caption=\OptimizingKeilVI (\ThumbMode),style=customasm]{patterns/185_64bit_in_32_env/shifting/Keil_thumb_O3.s}
% TODO add explanation

\subsubsection{MIPS}

GCC for MIPS follows the same algorithm as Keil does for Thumb mode:

\lstinputlisting[caption=\Optimizing GCC 4.4.5 (IDA)]{patterns/185_64bit_in_32_env/shifting/MIPS_O3_IDA.lst}

% TODO add explanation

}
\RU{\subsection{Сдвиг вправо}

\lstinputlisting[style=customc]{patterns/185_64bit_in_32_env/shifting/3.c}

\subsubsection{x86}

\lstinputlisting[caption=\Optimizing MSVC 2012 /Ob1,style=customasm]{patterns/185_64bit_in_32_env/shifting/3_MSVC.asm}

\lstinputlisting[caption=\Optimizing GCC 4.8.1 -fno-inline,style=customasm]{patterns/185_64bit_in_32_env/shifting/3_GCC.asm}

\myindex{x86!\Instructions!SHRD}
Сдвиг происходит также в две операции: в начале сдвигается младшая часть, затем старшая.
Но младшая часть сдвигается при помощи инструкции \INS{SHRD}, она сдвигает значение в \EAX{} на 7 бит, но подтягивает новые биты из \EDX{}, т.е. из старшей части.
Другими словами, 64-битное значение из пары регистров \TT{EDX:EAX}, как одно целое, сдвигается на 7 бит и младшие
32 бита результата сохраняются в \EAX{}.
Старшая часть сдвигается куда более популярной инструкцией \SHR{}: действительно, ведь освободившиеся биты в старшей части нужно
просто заполнить нулями.

\subsubsection{ARM}

В ARM нет такой инструкции как \INS{SHRD} в x86, так что компилятору Keil приходится всё это делать,
используя простые сдвиги и операции \q{ИЛИ}:

\lstinputlisting[caption=\OptimizingKeilVI (\ARMMode),style=customasm]{patterns/185_64bit_in_32_env/shifting/Keil_ARM_O3.s}

\lstinputlisting[caption=\OptimizingKeilVI (\ThumbMode),style=customasm]{patterns/185_64bit_in_32_env/shifting/Keil_thumb_O3.s}
% TODO add explanation

\subsubsection{MIPS}

GCC для MIPS реализует тот же алгоритм, что сделал Keil для режима Thumb:

\lstinputlisting[caption=\Optimizing GCC 4.4.5 (IDA),style=customasm]{patterns/185_64bit_in_32_env/shifting/MIPS_O3_IDA.lst}

% TODO add explanation

}
\DE{\subsection{Verschiebung nach rechts}

\lstinputlisting[style=customc]{patterns/185_64bit_in_32_env/shifting/3.c}

\subsubsection{x86}

\lstinputlisting[caption=\Optimizing MSVC 2012 /Ob1,style=customasmx86]{patterns/185_64bit_in_32_env/shifting/3_MSVC.asm}

\lstinputlisting[caption=\Optimizing GCC 4.8.1 -fno-inline,style=customasmx86]{patterns/185_64bit_in_32_env/shifting/3_GCC.asm}

\myindex{x86!\Instructions!SHRD}
Das Verschieben geschieht ebenfalls zweigeteilt: zunächst wird der niedere Teil verschoben, danach der höhere.
Der niedere Teil wird mithilfe des Befehls \INS{SHRD} verschoben; er verschiebt den Wert in \EAX um 7 Bits, holt aber
die nachrutschenden Bits aus \EDX, d.h. aus dem höheren Teil.
Mit anderen Worten: der 64-Bit-Wert aus \TT{EDX:EAX} wird als ganzes um 7 Bits verschoben und die niederen 32 Bits des
Ergebnisses werden in \EAX abgelegt. Der höhere Teil wird mit dem häufig verwendeten \SHR Befehl verschoben, da die frei
werdenden Bits im höheren Teil mit Nullen aufgefüllt werden müssen.

\subsubsection{ARM}
ARM verfügt im Gegensatz zu x86 nicht über einen \IN{SHRD} Befehl, sodass der Keil Compiler die Aufgabe mit einer
Kombination aus einfachen Schiebebefehlen und \OR-Operationen durchführen muss:

\lstinputlisting[caption=\OptimizingKeilVI (\ARMMode),style=customasmARM]{patterns/185_64bit_in_32_env/shifting/Keil_ARM_O3.s}

\lstinputlisting[caption=\OptimizingKeilVI (\ThumbMode),style=customasmARM]{patterns/185_64bit_in_32_env/shifting/Keil_thumb_O3.s}
% TODO add explanation

\subsubsection{MIPS}
GCC für MIPS folgt dem gleichen Algorithmus wie Keil für Thumb mode:

\lstinputlisting[caption=\Optimizing GCC 4.4.5 (IDA),style=customasmMIPS]{patterns/185_64bit_in_32_env/shifting/MIPS_O3_IDA.lst}

% TODO add explanation

}
\EN{\subsection{Converting 32-bit value into 64-bit one}
\label{subsec:sign_extending_32_to_64}

\lstinputlisting[style=customc]{patterns/185_64bit_in_32_env/conversion/4.c}

\subsubsection{x86}

\lstinputlisting[caption=\Optimizing MSVC 2012,style=customasm]{patterns/185_64bit_in_32_env/conversion/MSVC2012_Ox.asm}

Here we also run into necessity to extend a 32-bit signed value into a 64-bit signed one.
Unsigned values are converted straightforwardly: all bits in the higher part must be set to 0.
But this is not appropriate for signed data types: the sign has to be copied into the higher part of the resulting number.
\myindex{x86!\Instructions!CDQ}

The \INS{CDQ} instruction does that here, it takes its input value in \EAX{}, extends it to 64-bit and leaves it
in the \EDX{}:\EAX{} register pair.
In other words, \INS{CDQ} gets the number sign from \EAX{} (by getting the
most significant bit in \EAX{}), and depending of it, sets all 32 bits in \EDX{} to 0 or 1.
Its operation is somewhat
similar to the \MOVSX{} instruction.

\subsubsection{ARM}

\lstinputlisting[caption=\OptimizingKeilVI (\ARMMode),style=customasm]{patterns/185_64bit_in_32_env/conversion/Keil_ARM_O3.s}

Keil for ARM is different: it just arithmetically shifts right the input value by 31 bits.
As we know, the sign bit is \ac{MSB}, and the arithmetical shift copies the sign bit into the \q{emerged} bits.
So after \q{ASR r1,r0,\#31}, \Reg{1} containing 0xFFFFFFFF if the input value has been negative and 0 otherwise.
\Reg{1} contains the high part of the resulting 64-bit value.
In other words, this code just copies the \ac{MSB} (sign bit) from the input value in \Reg{0} to all bits of the high 32-bit part of the resulting 64-bit value.

\subsubsection{MIPS}

GCC for MIPS does the same as Keil did for ARM mode:

\lstinputlisting[caption=\Optimizing GCC 4.4.5 (IDA)]{patterns/185_64bit_in_32_env/conversion/MIPS_O3_IDA.lst}
}
\RU{\subsection{Конвертирование 32-битного значения в 64-битное}
\label{subsec:sign_extending_32_to_64}

\lstinputlisting[style=customc]{patterns/185_64bit_in_32_env/conversion/4.c}

\subsubsection{x86}

\lstinputlisting[caption=\Optimizing MSVC 2012,,style=customasm]{patterns/185_64bit_in_32_env/conversion/MSVC2012_Ox.asm}

Здесь появляется необходимость расширить 32-битное знаковое значение в 64-битное знаковое.

Конвертировать беззнаковые значения очень просто: нужно просто выставить в 0 все биты в старшей части.
Но для знаковых типов это не подходит: знак числа должен быть скопирован в старшую часть числа-результата.
\myindex{x86!\Instructions!CDQ}
Здесь это делает инструкция \INS{CDQ}, она берет входное значение в \EAX{}, расширяет его до 64-битного,
и оставляет его в паре регистров \EDX{}:\EAX{}.
Иными словами, инструкция \INS{CDQ} узнает знак числа в \EAX{} (просто берет самый старший бит в \EAX{}) и в зависимости от этого,
выставляет все 32 бита в \EDX{} в 0 или в 1.
Её работа в каком-то смысле напоминает работу инструкции \MOVSX{}.

\subsubsection{ARM}

\lstinputlisting[caption=\OptimizingKeilVI (\ARMMode),style=customasm]{patterns/185_64bit_in_32_env/conversion/Keil_ARM_O3.s}

Keil для ARM работает иначе: он просто сдвигает (арифметически) входное значение на 31 бит вправо.
Как мы знаем, бит знака это \ac{MSB}, и арифметический сдвиг копирует бит знака в \q{появляющихся} битах.

Так что после инструкции \INS{ASR r1,r0,\#31}, \Reg{1} будет содержать 0xFFFFFFFF если входное значение
было отрицательным, или 0 в противном случае.
\Reg{1} содержит старшую часть возвращаемого 64-битного значения.
Другими словами, этот код просто копирует \ac{MSB} (бит знака) из входного значения в \Reg{0} во все
биты старшей 32-битной части итогового 64-битного значения.

\subsubsection{MIPS}

GCC для MIPS делает то же, что сделал Keil для режима ARM:

\lstinputlisting[caption=\Optimizing GCC 4.4.5 (IDA),style=customasm]{patterns/185_64bit_in_32_env/conversion/MIPS_O3_IDA.lst}

}
\DE{\subsection{32-Bit-Werte in 64-Bit-Werte umwandeln}
\label{subsec:sign_extending_32_to_64}

\lstinputlisting[style=customc]{patterns/185_64bit_in_32_env/conversion/4.c}

\subsubsection{x86}

\lstinputlisting[caption=\Optimizing MSVC 2012,style=customasmx86]{patterns/185_64bit_in_32_env/conversion/MSVC2012_Ox.asm}
Hier müssen wir einen vorzeichenbehafteten 32-Bit-Wert in einen 64-Bit-Wert mit Vorzeichen umwandeln.
Vorzeichenlose Werte werden ganz einfach umgewandelt: alle Bits im höheren Teil werden auf 0 gesetzt.
Bei Werten mit Vorzeichen funktioniert diese Methode nicht: das Vorzeichen muss in den höheren Teil des
32-Bit-Ergebnisses kopiert werden.
\myindex{x86!\Instructions!CDQ}
Der Befehl \INS{CDQ} übernimmt hier diese Aufgabe. Er nimmt seinen Eingabewert aus \EAX, erweitert ihn auf 64 Bit und
speichert das Ergebnis im Registerpaar \EDX/\EAX.
Mit anderen Worten: \INS{CDQ} extrahiert das Vorzeichen aus \EAX (dies ist das MSB in \EAX) und setzt je nach Wert des
Vorzeichens alle 32 Bits in \EDX auf 0 oder 1.
Dieser Befehl ähnelt dem Befehl \MOVSX.

\subsubsection{ARM}

\lstinputlisting[caption=\OptimizingKeilVI (\ARMMode),style=customasmARM]{patterns/185_64bit_in_32_env/conversion/Keil_ARM_O3.s}
Keil für ARM geht anders vor: er verschiebt den Eingabewert um 31 Bits nach rechts.
Wie wir wissen ist das Vorzeichenbit das \ac{MSB} und die arithmetische Verschiebung kopiert das Vorzeichenbit in die
ausgeschobenen Bits. Nach Ausführung von \q{ASR r1,r0,\#31} enthält \Reg{1} also 0xFFFFFFFF, falls der Eingabewert
negativ war und ansonsten 0.
\Reg{1} enthält den höheren Teil des resultierenden 64-Bit-Wertes.
Mit anderen Worten: dieser Code kopiert das \ac{MSB} (Vorzeichenbit) vom Eingabewert in \Reg{0} in alle Bits der höheren
32 Bits des Ergebnisses.

\subsubsection{MIPS}
GCC für MIPS erzeugt das gleich wie Keil für ARM mode:

\lstinputlisting[caption=\Optimizing GCC 4.4.5 (IDA),style=customasmMIPS]{patterns/185_64bit_in_32_env/conversion/MIPS_O3_IDA.lst}
}


