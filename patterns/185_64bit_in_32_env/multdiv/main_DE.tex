\subsection{Multiplikation und Division}

\lstinputlisting[style=customc]{patterns/185_64bit_in_32_env/multdiv/2.c}

\subsubsection{x86}

\lstinputlisting[caption=\Optimizing MSVC 2013
/Ob1,style=customasmx86]{patterns/185_64bit_in_32_env/multdiv/2_MSVC_DE.asm}
Da Multiplikation und Division komplexere Operationen sind, verwendet der Compiler für deren Ausführung in der Regel
Bibliotheksfunktionen.

Diese Funktionen werden hier beschrieben:\myref{sec:MSVC_library_func}.

\lstinputlisting[caption=\Optimizing GCC 4.8.1
-fno-inline,style=customasmx86]{patterns/185_64bit_in_32_env/multdiv/2_GCC_DE.asm}
GCC verhält sich wie erwartet, aber der Code für die Multiplikation wird direkt in der Funktion platziert und ist so
effizienter. In GCC haben die Bibliotheksfunktionen andere Namen: \myref{sec:GCC_library_func}.

\subsubsection{ARM}
Keil für Thumb mode fügt Aufrufe von Routinen aus Bibliotheken ein:

\lstinputlisting[caption=\OptimizingKeilVI (\ThumbMode),style=customasmARM]{patterns/185_64bit_in_32_env/multdiv/Keil_thumb_O3.s}
Keil für ARM mode ist wiederum in der Lage Code für 64-Bit-Multiplikation zu erzeugen:

\lstinputlisting[caption=\OptimizingKeilVI (\ARMMode),style=customasmARM]{patterns/185_64bit_in_32_env/multdiv/Keil_ARM_O3.s}
% TODO add explanation

\subsubsection{MIPS}
Der optimierende GCC für MIPS kann Code für 64-Bit-Multiplikation erzeugen, muss aber für die Division auf eine
Programmbibliothek zurückgreifen:

\lstinputlisting[caption=\Optimizing GCC 4.4.5 (IDA),style=customasmMIPS]{patterns/185_64bit_in_32_env/multdiv/MIPS_O3_IDA.lst}
Hier gibt es eine Menge \ac{NOP}s; möglicherweise sind dies delay slots, die nach dem Multiplikationsbefehl eingefügt
wurden (diese Vorgehensweise ist jedoch langsamer als andere Befehle).

% TODO add explanation
