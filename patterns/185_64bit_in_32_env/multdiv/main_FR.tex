\subsection{Multiplication, division}

\lstinputlisting[style=customc]{patterns/185_64bit_in_32_env/multdiv/2.c}

\subsubsection{x86}

\lstinputlisting[caption=MSVC 2013 /Ob1 \Optimizing,style=customasmx86]{patterns/185_64bit_in_32_env/multdiv/2_MSVC_FR.asm}

La multiplication et la division sont des opérations plus complexes, donc en général
le compilateur embarque des appels à des fonctions de bibliothèque les effectuant.

Ces fonctions sont décrites ici: \myref{sec:MSVC_library_func}.

\lstinputlisting[caption=GCC 4.8.1 -fno-inline \Optimizing,style=customasmx86]{patterns/185_64bit_in_32_env/multdiv/2_GCC_FR.asm}

GCC fait ce que l'on attend, mais le code multiplication est mis en ligne (inlined)
directement dans la fonction, pensant que ça peut être plus efficace.
GCC a des noms de fonctions de bibliothèque différents: \myref{sec:GCC_library_func}.

\subsubsection{ARM}

Keil pour mode Thumb insère des appels à des sous-routines de bibliothèque:

\lstinputlisting[caption=\OptimizingKeilVI (\ThumbMode),style=customasmARM]{patterns/185_64bit_in_32_env/multdiv/Keil_thumb_O3.s}

Keil pour mode ARM, d'un autre côté, est capable de produire le code de la multiplication
64-bit:

\lstinputlisting[caption=\OptimizingKeilVI (\ARMMode),style=customasmARM]{patterns/185_64bit_in_32_env/multdiv/Keil_ARM_O3.s}
% TODO add explanation

\subsubsection{MIPS}

GCC \Optimizing pour MIPS peut générer du code pour la multiplication 64-bit, mais
doit appeler une routine de bibliothèque pour la division 64-bit:

\lstinputlisting[caption=GCC 4.4.5 \Optimizing (IDA),style=customasmMIPS]{patterns/185_64bit_in_32_env/multdiv/MIPS_O3_IDA.lst}

Il y a beaucoup de \ac{NOP}s, sans doute des slots de délai de remplissage après
l'instruction de multiplication (c'est plus lent que les autres instructions après
tout).

% TODO add explanation
