\subsection{Arguments passing, addition, subtraction}

\lstinputlisting[style=customc]{patterns/185_64bit_in_32_env/passing_add_sub/1.c}

\subsubsection{x86}

\lstinputlisting[caption=\Optimizing MSVC 2012 /Ob1,style=customasmx86]{patterns/185_64bit_in_32_env/passing_add_sub/1_MSVC.asm}

We can see in the \GTT{f\_add\_test()} function that each 64-bit value is passed using two 32-bit values,
high part first, then low part. 

Addition and subtraction occur in pairs as well.

\myindex{x86!\Instructions!ADC}
In addition, the low 32-bit part are added first.
If carry has been occurred while adding, the \TT{CF} flag is set.

The following \INS{ADC} instruction adds the high parts of the values, and also adds 1 if $CF=1$.

\myindex{x86!\Instructions!SBB}
Subtraction also occurs in pairs.
The first \SUB may also turn on the CF flag, which is to be checked in the subsequent \INS{SBB} instruction:
if the carry flag is on, then 1 is also to be subtracted from the result.

It is easy to see how the \GTT{f\_add()} function result is then passed to \printf{}.

\lstinputlisting[caption=GCC 4.8.1 -O1 -fno-inline,style=customasmx86]{patterns/185_64bit_in_32_env/passing_add_sub/1_GCC.asm}

GCC code is the same.

\subsubsection{ARM}

\lstinputlisting[caption=\OptimizingKeilVI (\ARMMode),style=customasmARM]{patterns/185_64bit_in_32_env/passing_add_sub/Keil_ARM_O3.s}

\myindex{ARM!\Instructions!ADDS}
\myindex{ARM!\Instructions!SUBS}
\myindex{ARM!\Instructions!ADC}
\myindex{ARM!\Instructions!SBC}

The first 64-bit value is passed in \Reg{0} and \Reg{1} register pair, the second in \Reg{2} and \Reg{3} register pair.
ARM has the \INS{ADC} instruction as well (which counts carry flag) and \INS{SBC} (\q{subtract with carry}).
Important thing: when the low parts are added/subtracted, \INS{ADDS} and \INS{SUBS} instructions with -S suffix are used.
The -S suffix stands for \q{set flags}, and flags (esp. carry flag) is what consequent \INS{ADC}/\INS{SBC} instructions definitely need.
Otherwise, instructions without the -S suffix would do the job (\ADD and \SUB).

\subsubsection{MIPS}

\lstinputlisting[caption=\Optimizing GCC 4.4.5 (IDA),style=customasmMIPS]{patterns/185_64bit_in_32_env/passing_add_sub/MIPS_O3_IDA_EN.lst}

MIPS has no flags register, so there is no such information present after the execution of arithmetic operations.
So there are no instructions like x86's \INS{ADC} and \INS{SBB}.
To know if the carry flag would be set, a comparison (using \INS{SLTU} instruction) also occurs, 
which sets the destination register to 1 or 0.
This 1 or 0 is then added or subtracted to/from the final result.

