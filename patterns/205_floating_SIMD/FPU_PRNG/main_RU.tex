\subsection{И снова пример генератора случайных чисел}
\label{FPU_PRNG_SIMD}

Вернемся к примеру \q{пример генератора случайных чисел} \lstref{FPU_PRNG}.

Если скомпилировать это в MSVC 2012, компилятор будет использовать SIMD-инструкции для FPU.

\lstinputlisting[caption=\Optimizing MSVC 2012,style=customasmx86]{patterns/205_floating_SIMD/FPU_PRNG/MSVC2012_Ox_Ob0_RU.asm}

% FIXME1 rewrite!
У всех инструкций суффикс -SS, это означает \q{Scalar Single}.

\q{Scalar} означает, что только одно значение хранится в регистре.

\q{Single}\footnote{Т.е., single precision.} означает, что это тип \Tfloat.

