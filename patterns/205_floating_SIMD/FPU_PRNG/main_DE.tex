\subsection{Erneute Betrachtung des Beispiels zum Pseudozufallszahlengenerator}
\label{FPU_PRNG_SIMD}

Betrachten wir erneut das Beispiel zum Pseudozufallszahlengenerator \lstref{FPU_PRNG}.

Wenn wir es in MSVC 2012 kompilieren, werden SIMD Befehle für die FPU benutzt.

\lstinputlisting[caption=\Optimizing MSVC
2012,style=customasmx86]{patterns/205_floating_SIMD/FPU_PRNG/MSVC2012_Ox_Ob0_DE.asm}

% FIXME1 rewrite!
Alle Befehle haben den Suffix \TT{-SS}, der für \q{Scalar Single} steht.

\q{Scalar} bedeutet, dass nur ein Wert im Register gespeichert ist.
\q{Single} steht für den Datentyp \Tfloat.

