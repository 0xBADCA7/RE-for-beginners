\subsubsection{Modification d'adresse (Win64)}

Lorsque notre exemple est compilé sous MSVC2013 avec l'option \TT{\textbackslash{}MD}
(générant un exécutable plus petit à cause du lien avec \TT{MSVCR*.DLL}), la fonction \main vient en premier et
est trouvée facilement:

\begin{figure}[H]
\centering
\myincludegraphics{patterns/01_helloworld/hiew_incr1.png}
\caption{Hiew}
\label{}
\end{figure}

A titre expérimental, nous pouvons \glslink{increment}{incrémenter} l'adresse du pointeur de 1:

\begin{figure}[H]
\centering
\myincludegraphics{patterns/01_helloworld/hiew_incr2.png}
\caption{Hiew}
\label{}
\end{figure}

Hiew montre la chaîne \q{ello, world}.
Et lorsque nous lançons l'exécutable modifié, la chaîne raccourcie est affichée.

\subsubsection{Utiliser une autre chaîne d'un binaire (Linux x64)}

Le fichier binaire que j'obtiens en compilant notre exemple avec GCC 5.4.0 sur un système Linux x64 contient de
nombreuses autres chaînes:
la plupart sont des noms de fonction et de bibliotèque importées.

Je lance \IT{objdump} pour voir le contenu de toutes les sections du fichier compilé:

\begin{lstlisting}[basicstyle=\ttfamily, mathescape]
$\$$ objdump -s a.out

a.out:     file format elf64-x86-64

Contents of section .interp:
 400238 2f6c6962 36342f6c 642d6c69 6e75782d  /lib64/ld-linux-
 400248 7838362d 36342e73 6f2e3200           x86-64.so.2.
Contents of section .note.ABI-tag:
 400254 04000000 10000000 01000000 474e5500  ............GNU.
 400264 00000000 02000000 06000000 20000000  ............ ...
Contents of section .note.gnu.build-id:
 400274 04000000 14000000 03000000 474e5500  ............GNU.
 400284 fe461178 5bb710b4 bbf2aca8 5ec1ec10  .F.x[.......^...
 400294 cf3f7ae4                             .?z.

...
\end{lstlisting}

Ce n'est pas un problème de passer l'adresse de la chaîne \q{/lib64/ld-linux-x86-64.so.2} à l'appel de \TT{printf()}:

\begin{lstlisting}[style=customc]
#include <stdio.h>

int main()
{
    printf(0x400238);
    return 0;
}
\end{lstlisting}

Difficile à croire, ce code affiche la chaîne mentionnée.

%%On your system, executable may be slightly different, and all addresses will also be different.
%%Also, adding/removing code to/from this source code will probably shift all addresses back and forth.
Changez l'adresse en \TT{0x400260}, et la chaîne \q{GNU} sera affichée.
L'adresse est valable pour cette version spécifique de GCC, outils GNU, etc.
Sur votre système, l'exécutable peut être légèrement différent, et toutes les adresses seront différentes.
Ainsi, ajouter/supprimer du code à/de ce code source va probablement décaler les adresses en arrière et avant.

