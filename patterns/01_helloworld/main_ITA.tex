\section{\HelloWorldSectionName}
\label{sec:helloworld}

Utilizziamo il famoso esempio dal libro \KRBook:

\lstinputlisting[style=customc]{patterns/01_helloworld/hw.c}

\subsection{x86}

\EN{\subsubsection{MSVC}

Let's compile it in MSVC 2010:

\begin{lstlisting}
cl 1.cpp /Fa1.asm
\end{lstlisting}

(\TT{/Fa} option instructs the compiler to generate assembly listing file)

\begin{lstlisting}[caption=MSVC 2010,style=customasm]
CONST	SEGMENT
$SG3830	DB	'hello, world', 0AH, 00H
CONST	ENDS
PUBLIC	_main
EXTRN	_printf:PROC
; Function compile flags: /Odtp
_TEXT	SEGMENT
_main	PROC
	push	ebp
	mov	ebp, esp
	push	OFFSET $SG3830
	call	_printf
	add	esp, 4
	xor	eax, eax
	pop	ebp
	ret	0
_main	ENDP
_TEXT	ENDS
\end{lstlisting}

MSVC produces assembly listings in Intel-syntax.
The difference between Intel-syntax and AT\&T-syntax will be discussed in \myref{ATT_syntax}.

The compiler generated the file, \TT{1.obj}, which is to be linked into \TT{1.exe}.
In our case, the file contains two segments: \TT{CONST} (for data constants) and \TT{\_TEXT} (for code).

\myindex{\CLanguageElements!const}
\label{string_is_const_char}
The string \TT{hello, world} in \CCpp has type \TT{const char[]}\InSqBrackets{\TCPPPL p176, 7.3.2}, but it does not have its own name.
The compiler needs to deal with the string somehow so it defines the internal name \TT{\$SG3830} for it.

That is why the example may be rewritten as follows:

\lstinputlisting[style=customc]{patterns/01_helloworld/hw_2.c}

Let's go back to the assembly listing. As we can see, the string is terminated by a zero byte, which is standard for \CCpp strings.
More about \CCpp strings: \myref{C_strings}.

In the code segment, \TT{\_TEXT}, there is only one function so far: \main{}.
The function \main starts with prologue code and ends with epilogue code (like almost any function)
\footnote{You can read more about it in the section about function prologues and epilogues ~(\myref{sec:prologepilog}).}.

\myindex{x86!\Instructions!CALL}
After the function prologue we see the call to the \printf{} function:\\
\INS{CALL \_printf}.
\myindex{x86!\Instructions!PUSH}
Before the call the string address (or a pointer to it) containing our greeting is placed on the stack with the help of the \PUSH instruction.

When the \printf function returns the control to the \main function, the string address (or a pointer to it) is still on the stack.
Since we do not need it anymore, the \gls{stack pointer} (the \ESP register) needs to be corrected.

\myindex{x86!\Instructions!ADD}
\INS{ADD ESP, 4} means add 4 to the \ESP register value.

Why 4? Since this is a 32-bit program, we need exactly 4 bytes for address passing through the stack. If it was x64 code we would need 8 bytes.
\INS{ADD ESP, 4} is effectively equivalent to \INS{POP register} but without using any register\footnote{CPU flags, however, are modified}.

\myindex{Intel C++}
\myindex{\oracle}
\myindex{x86!\Instructions!POP}

For the same purpose, some compilers (like the Intel C++ Compiler) may emit \TT{POP ECX} 
instead of \ADD (e.g., such a pattern can be observed in the \oracle{} code as it is compiled with the Intel C++ compiler).
This instruction has almost the same effect but the \ECX register contents will be overwritten.
The Intel C++ compiler supposedly uses \TT{POP ECX} since this instruction's opcode is shorter than \TT{ADD ESP, x} (1 byte for \TT{POP} against 3 for \TT{ADD}).

Here is an example of using \POP instead of \ADD from \oracle{}:

\begin{lstlisting}[caption=\oracle 10.2 Linux (app.o file),style=customasm]
.text:0800029A                 push    ebx
.text:0800029B                 call    qksfroChild
.text:080002A0                 pop     ecx
\end{lstlisting}

%Read more about the stack in section
% ~(\myref{sec:stack}).
\myindex{\CLanguageElements!return}
After calling \printf, the original \CCpp code contains the statement \TT{return 0}~---return 0 as the result of the \main function.

\myindex{x86!\Instructions!XOR}
In the generated code this is implemented by the instruction \INS{XOR EAX, EAX}.

\myindex{x86!\Instructions!MOV}

\XOR is in fact just \q{eXclusive OR}\footnote{\href{http://go.yurichev.com/17118}{wikipedia}} but the compilers often use it instead of
\INS{MOV EAX, 0}---again because it is a slightly shorter opcode (2 bytes for \XOR against 5 for \MOV).

\myindex{x86!\Instructions!SUB}
Some compilers emit \INS{SUB EAX, EAX}, which means \IT{SUBtract the value in the} \EAX \IT{from the value in} \EAX, which, in any case, results in zero.

\myindex{x86!\Instructions!RET}
The last instruction \RET returns the control to the \gls{caller}. Usually, this is \CCpp \ac{CRT} code, which, in turn, returns control to the \ac{OS}.

}
\FR{\subsubsection{MSVC}

Compilons-le avec MSVC 2010:

\begin{lstlisting}
cl 1.cpp /Fa1.asm
\end{lstlisting}

(L'option \TT{/Fa} indique au compilateur de générer un fichier avec le listing en assembleur)

\begin{lstlisting}[caption=MSVC 2010,style=customasmx86]
CONST	SEGMENT
$SG3830	DB	'hello, world', 0AH, 00H
CONST	ENDS
PUBLIC	_main
EXTRN	_printf:PROC
; Function compile flags: /Odtp
_TEXT	SEGMENT
_main	PROC
	push	ebp
	mov	ebp, esp
	push	OFFSET $SG3830
	call	_printf
	add	esp, 4
	xor	eax, eax
	pop	ebp
	ret	0
_main	ENDP
_TEXT	ENDS
\end{lstlisting}

MSVC génère des listings assembleur avec la syntaxe Intel.
La différence entre la syntaxe Intel et la syntaxe AT\&T sera discutée dans \myref{ATT_syntax}.

Le compilateur a généré le fichier object \TT{1.obj}, qui sera lié dans l'exécutable \TT{1.exe}.
Dans notre cas, le fichier contient deux segments: \TT{CONST} (pour les données constantes)
 et \TT{\_TEXT} (pour le code).

\myindex{\CLanguageElements!const}
\label{string_is_const_char}
La chaîne \TT{hello, world} en \CCpp a le type \TT{const char[]}\InSqBrackets{\TCPPPL p176, 7.3.2}, mais
elle n'a pas de nom en propre.
Le compilateur doit pouvoir l'utiliser et lui défini donc un nom interne \TT{\$SG3830} à cette fin.

C'est pourquoi l'exemple pourrait être ré-écrit comme suit:

\lstinputlisting[style=customc]{patterns/01_helloworld/hw_2.c}

Retournons au listing assembleur. Comme nous le voyons, la chaîne est terminée avec un octet à zéro, ce qui
est le standard pour les chaînes \CC.

Dans le segment de code, \TT{\_TEXT}, il n'y a qu'une seule fonction: \main{}.
La fonction \main débute par le code du prologue et se termine par le code de l'épilogue 
(comme presque toutes les fonctions)
\footnote{Vous pouvez en lire plus dans la section concerant les prologues et épilogues de
fonctions ~(\myref{sec:prologepilog}).}.

\myindex{x86!\Instructions!CALL}
Après le prologue de la fonction nous voyons l'appel à la fonction \printf{}:\\
\INS{CALL \_printf}.
\myindex{x86!\Instructions!PUSH}
Avant l'appel, l'adresse de la chaîne (ou un pointeur sur elle) contenant notre message
 est placée sur la pile avec l'aide de l'instruction \PUSH.

Lorsque la fonction \printf rend le contrôle à la fonction \main, l'adresse de la chaîne (ou un pointeur sur elle)
est toujours sur la pile.
Comme nous n'en avons plus besoin, le pointeur de pile (\gls{stack pointer} le registre \ESP) doit être corrigé.

\myindex{x86!\Instructions!ADD}
\INS{ADD ESP, 4} signifie ajouter 4 à la valeur du registre \ESP.

Pourquoi 4? puisqu'il s'agit d'un programme 32-bit, nous avons besoin d'exactement 4 octets pour passer une adresse
par la pile. S'il s'agissait d'un code x64, nous aurions besoin de 8 octets.
\INS{ADD ESP, 4} est effectivement équivalent à \INS{POP register} mais sans utiliser de registre\footnote{Les
flags du CPU, toutefois, sont modifiés}.

\myindex{Intel C++}
\myindex{\oracle}
\myindex{x86!\Instructions!POP}

Pour la même raison, certains compilateurs (comme le compilateur C++ d'Intel) peuvent générer \TT{POP ECX}
à la place de \ADD (e.g., ce comportement peut être observé dans le code d'\oracle{} car il est
compilé avec le compilateur C++ d'Intel.
Cette instruction a presque le même effet mais le contenu du registre \ECX sera écrasé.
Le compilateur C++ d'Intel utilise probablement \TT{POP ECX} car l'opcode de cette instruction est plus
 court que celui de \TT{ADD ESP, x} (1 octet pour \TT{POP} contre 3 pour \TT{ADD}).

Voici un exemple d'utilisation de \POP à la place de \ADD dans \oracle{}:

\begin{lstlisting}[caption=\oracle 10.2 Linux (app.o file),style=customasmx86]
.text:0800029A                 push    ebx
.text:0800029B                 call    qksfroChild
.text:080002A0                 pop     ecx
\end{lstlisting}

%Lisez en plus sur la pile dans la section
% ~(\myref{sec:stack}).

\myindex{\CLanguageElements!return}
Après l'appel de \printf, le code \CCpp original contient la déclaration \TT{return 0}~---renvoie 0 comme valeur de retour de la fonction \main.

\myindex{x86!\Instructions!XOR}
Dans le code généré cela est implémenté par l'instruction \INS{XOR EAX, EAX}.

\myindex{x86!\Instructions!MOV}

\XOR est en fait un simple \q{OU exclusif (eXclusive OR}\footnote{\href{http://go.yurichev.com/17118}{wikipedia}} mais
les compilateurs l'utilisent souvent à la place de \INS{MOV EAX, 0}---à nouveau parce que l'opcode est légèrement plus
court (2 octets pour \XOR contre 5 pour \MOV).

\myindex{x86!\Instructions!SUB}
Certains compilateurs génèrent \INS{SUB EAX, EAX}, qui signifie \IT{Soustraire la valeur dans} \EAX \IT{de la valeur dans} \EAX,
 ce qui, dans tous les cas, donne zéro.

\myindex{x86!\Instructions!RET}
La dernière instruction \RET redonne le contrôle à l'appelant (\gls{caller}). c'est du code \CCpp \ac{CRT}, qui, à son tour, redonne le contrôle à l'\ac{OS}.

}
\ITA{\subsubsection{MSVC}

Compiliamolo in MSVC 2010:

\begin{lstlisting}
cl 1.cpp /Fa1.asm
\end{lstlisting}

(l'opzione \TT{/Fa} indica al compilatore di generare un file con il listato assembly)

\begin{lstlisting}[caption=MSVC 2010,style=customasm]
CONST	SEGMENT
$SG3830	DB	'hello, world', 0AH, 00H
CONST	ENDS
PUBLIC	_main
EXTRN	_printf:PROC
; Function compile flags: /Odtp
_TEXT	SEGMENT
_main	PROC
	push	ebp
	mov	ebp, esp
	push	OFFSET $SG3830
	call	_printf
	add	esp, 4
	xor	eax, eax
	pop	ebp
	ret	0
_main	ENDP
_TEXT	ENDS
\end{lstlisting}

\ITAph{}
La differenza tra le sintassi Intel e AT\&T-syntax sarà discussa al \myref{ATT_syntax}.

Il compilatore ha generato il file, \TT{1.obj}, che deve essere linkato (collegato) in \TT{1.exe}.
Nel nostro caso, il file contiene due segmentu: \TT{CONST} (per i dati constanti) e \TT{\_TEXT} (per il codice).

\myindex{\CLanguageElements!const}
\label{string_is_const_char}
La stringa \TT{hello, world} in \CCpp ha tipo \TT{const char[]}\InSqBrackets{\TCPPPL p176, 7.3.2}, ma non ha un nome proprio.
Il compilatore deve in qualche modo aver a che fare con la stringa, e la definisce quindi con il nome interno \TT{\$SG3830}.

Questo è il motivo per cui l'esempio potrebbe essere riscritto nel modo seguente:

\lstinputlisting[style=customc]{patterns/01_helloworld/hw_2.c}

Torniamo al listato assembly. Come possiamo vedere, la stringa è terminata con un byte zero, che è lo standard per la terminazione delle stringhe \CCpp.
\ITAph{}: \myref{C_strings}.

Nel code segment, \TT{\_TEXT}, esiste fino ad ora solo una funzione: \main{}.
La funzione \main inizia con il codice di prologo (prologue code) e termina con il codice di epilogo (epilogue code) (come quasi qualunque funzione)
\footnote{Maggiori informazioni si trovano nella sezione su prologo ed epilogo delle funzioni ~(\myref{sec:prologepilog}).}.

\myindex{x86!\Instructions!CALL}
Dopo il prologo della funzione, notiamo la chiamata alla funzione \printf{} : \INS{CALL \_printf}.
\myindex{x86!\Instructions!PUSH}
Prima della chiamata, l'indirizzo della stringa (o un puntatore ad essa) contenente il saluto viene messo sullo stack con l'aiuto dell'istruzione \PUSH.

Quando la funzione \printf restituisce il controllo alla funzione \main , l'indirizzo della stringa (o il puntatore) si trova ancora sullo stack.
Poichè non ne abbiamo più bisogno, lo \gls{stack pointer} (il registro \ESP ) deve essere corretto.

\myindex{x86!\Instructions!ADD}
\INS{ADD ESP, 4} significa aggiungi 4 al valore del registro \ESP.

Perchè 4? Essendo questo un programma a 32-bit, abbiamo esattamente bisogno di 4 bytes per passare un indirizzo attraverso lo stack. Se fosse stato codice x64 ne sarebbero serviti 8.
\INS{ADD ESP, 4} è a tutti gli effetti equivalente a \TT{POP register} ma senza usare alcun registro\footnote{i flag CPU vengono comunque modificati}.

\myindex{Intel C++}
\myindex{\oracle}
\myindex{x86!\Instructions!POP}

Per lo stesso scopo, alcuni compilatori (come l'Intel C++ Compiler) potrebbero emettere l'istruzione \TT{POP ECX} 
invece di \ADD (ad esempio, queto tipo di pattern può essere nel codice di \oracle{} che è compilato con l' Intel C++ compiler).
Questa istruzione ha pressoché lo stesso effetto ma il contenuto del registro \ECX sarà sovrascritto.
Il compilatore Intel C++ usa probabilmente \TT{POP ECX} poichè l'opcode di questa istruzione è più corto di \TT{ADD ESP, x} (1 byte per \TT{POP} contro 3 per \TT{ADD}).

Ecco un esempio dell'uso di \POP al posto di \ADD da \oracle{}:

\begin{lstlisting}[caption=\oracle 10.2 Linux (\ITAph{}),style=customasm]
.text:0800029A                 push    ebx
.text:0800029B                 call    qksfroChild
.text:080002A0                 pop     ecx
\end{lstlisting}

\myindex{\CLanguageElements!return}
Dopo la chiamata a \printf, il codice \CCpp originale contiene la direttiva \TT{return 0}~---restituisci 0 come risultato dalla funzione \main.

\myindex{x86!\Instructions!XOR}
Nel codice generato, questa è implementata dall'istruzione \INS{XOR EAX, EAX}.

\myindex{x86!\Instructions!MOV}

\XOR è infatti semplicemente \q{eXclusive OR, ovvero OR esclusivo}\footnote{\href{http://go.yurichev.com/17118}{wikipedia}} ma i compilatori lo usano spesso al posto di 
\INS{MOV EAX, 0}---ancora una volta poichè è un opcode leggermente più corto (2 byte per \XOR contro 5 per \MOV).

\myindex{x86!\Instructions!SUB}
Alcuni compilatori emettono l'istruzione \INS{SUB EAX, EAX}, che significa \IT{sottrai (SUBtract) il valore nel registro} \EAX \IT{dal valore nel registro} \EAX, che, in ogni caso, risulta uguale a zero.

\myindex{x86!\Instructions!RET}
L'ultima istruzione \RET restituisce il controllo al chiamante (\gls{caller}). Solitamente, questo è codice \CCpp \ac{CRT} , che, a sua volta, restituisce il controllo all' \ac{OS}.

}
\NL{\subsubsection{MSVC}

We compileren het in MSVC 2010:

\begin{lstlisting}
cl 1.cpp /Fa1.asm
\end{lstlisting}

(\TT{/Fa} optie zorgt ervoor dat de compiler het bestand met assembly listing genereert)

\begin{lstlisting}[caption=MSVC 2010,style=customasmx86]
CONST	SEGMENT
$SG3830	DB	'hello, world', 0AH, 00H
CONST	ENDS
PUBLIC	_main
EXTRN	_printf:PROC
; Function compile flags: /Odtp
_TEXT	SEGMENT
_main	PROC
	push	ebp
	mov	ebp, esp
	push	OFFSET $SG3830
	call	_printf
	add	esp, 4
	xor	eax, eax
	pop	ebp
	ret	0
_main	ENDP
_TEXT	ENDS
\end{lstlisting}

\NLph{}
Het verschil tussen Intel-syntax en AT\&T-syntax zal besproken worden in: \myref{ATT_syntax}.

De compiler heeft het bestand, \TT{1.obj} gegenereerd, hetwelk gelinkt wordt tot \TT{1.exe}.
In ons geval bevat het bestand twee segmenten: \TT{CONST} (voor data constanten) en \TT{\_TEXT}(voor code).

\myindex{\CLanguageElements!const}
\label{string_is_const_char}
De string \TT{hello, world} in \CCpp is van het type \TT{const char[]}\InSqBrackets{\TCPPPL p176, 7.3.2}, maar heeft geen eigen naam.
De compiler moet een manier hebben om met de string om te kunnen, en definieert er daarom de interne naam \TT{\$SG3830} voor.

Daarom kan het voorbeeld herschreven worden als volgt:

\lstinputlisting[style=customc]{patterns/01_helloworld/hw_2.c}

Laten we terug gaan naar de assembly listing. Zoals je kan zien, wordt de string beeindigd door een nul-byte. Dit is standaard voor \CCpp strings.
Meer over \CCpp strings: \myref{C_strings}.

In het code segment, \TT{\_TEXT}, is er slechts een functie tot nu toe: \main{}.
De functie \main begint met een proloog code en eindigt met een epiloog code (zoals bijna elke functie)
\footnote{Je kan hier meer over lezen in de sectie over functieprologen en epilogen ~(\myref{sec:prologepilog}).}.

\myindex{x86!\Instructions!CALL}
Na de functie proloog zien we de call naar de \printf{} functie: \INS{CALL \_printf}.
\myindex{x86!\Instructions!PUSH}
Voor de call wordt het adres van de string (of een pointer ernaar) die onze begroeting bevat, op de stack geplaatsd met de hulp van de \PUSH instructie.

Wanneer de \printf functie de controle teruggeeft aan de \main functie, staat het string adres (of de pointer ernaar) nog steeds op de stack.
Aangezien we dit niet meer nodig hebben, moet de \gls{stack pointer} (het \ESP register) gecorrigeerd worden.

\myindex{x86!\Instructions!ADD}
\INS{ADD ESP, 4} betekent dat er 4 wordt opgeteld bij de \ESP registerwaarde.

Waarom 4? Aangezien dit een 32-bit programma is, hebben we exact 4 bytes nodig om een adres door te geven via de stack. als het x64 code was, zouden we 8 bytes nodig gehad hebben.
\INS{ADD ESP, 4} is een effectief equivalent voor \TT{POP register} maar zonder gebruik van een register\footnote{CPU flags worden echter wel aangepast}.

\myindex{Intel C++}
\myindex{\oracle}
\myindex{x86!\Instructions!POP}

Met dezelfde reden zullen sommige compilers (zoals de Intel C++ Compiler) gebruik maken van \TT{POP ECX}
in plaats van \ADD (een dergelijk patroon kan waargenomen worden in de \oracle{} code aangezien deze gecompileerd is met de Intel C++ compiler).
Deze instructie heeft bijna hetzelfde effect, maar de inhoud van het \ECX register zal overschreven worden.
De Intel C++ Compiler gebruikt waarschijnlijk \TT{POP ECX} aangezien de opcode van deze instructie korter is dan \TT{ADD ESP, x} (1 byte voor \TT{POP} tegen 3 voor \TT{ADD}).

Hier is een voorbeeld van het gebruik van \POP in plaats van \ADD van \oracle{}:

\begin{lstlisting}[caption=\oracle 10.2 Linux (app.o bestand),style=customasmx86]
.text:0800029A                 push    ebx
.text:0800029B                 call    qksfroChild
.text:080002A0                 pop     ecx
\end{lstlisting}

%Lees meer over de stack in de sectie ~(\myref{sec:stack}).
\myindex{\CLanguageElements!return}
Na \printf aan te roepen, bevat de originele \CCpp code het statement \TT{return 0}~---return 0 als resultaat van de \main functie.

\myindex{x86!\Instructions!XOR}
In de gegenereerde code wordt dit geimplementeerd door de instructie \INS{XOR EAX, EAX}.

\myindex{x86!\Instructions!MOV}

\XOR is feitelijk simpelweg \q{eXclusive OR}\footnote{\href{http://go.yurichev.com/17118}{wikipedia}} maar de compilers gebruiken het vaak in plaats van
\INS{MOV EAX, 0} --- wederom omdat de opcode hiervoor iets korter is (2 bytes voor \XOR tegenover 5 voor \MOV).

\myindex{x86!\Instructions!SUB}
Sommige compilers gebruiken \INS{SUB EAX, EAX}, wat staat voor \IT{verminder de waarde in} \EAX \IT{met de waarde in} \EAX, wat in elke situatie resulteert in nul.

\myindex{x86!\Instructions!RET}
De laatste instructie \RET geeft de controle terug aan de \gls{caller}. Gewoonlijk is dit \CCpp \ac{CRT} code, die op zijn beurt de controle teruggeeft aan het \ac{OS}.

}
\RU{\subsubsection{MSVC}

Компилируем в MSVC 2010:

\begin{lstlisting}
cl 1.cpp /Fa1.asm
\end{lstlisting}

(Ключ \TT{/Fa} означает сгенерировать листинг на ассемблере)

\begin{lstlisting}[caption=MSVC 2010]
CONST	SEGMENT
$SG3830	DB	'hello, world', 0AH, 00H
CONST	ENDS
PUBLIC	_main
EXTRN	_printf:PROC
; Function compile flags: /Odtp
_TEXT	SEGMENT
_main	PROC
	push	ebp
	mov	ebp, esp
	push	OFFSET $SG3830
	call	_printf
	add	esp, 4
	xor	eax, eax
	pop	ebp
	ret	0
_main	ENDP
_TEXT	ENDS
\end{lstlisting}

MSVC выдает листинги в синтаксисе Intel.
Разница между синтаксисом Intel и AT\&T будет рассмотрена немного позже:

Компилятор сгенерировал файл \TT{1.obj}, который впоследствии будет слинкован линкером в \TT{1.exe}.
В нашем случае этот файл состоит из двух сегментов: \TT{CONST} (для данных-констант) и \TT{\_TEXT} (для кода).

\myindex{\CLanguageElements!const}
\label{string_is_const_char}
Строка \TT{hello, world} в \CCpp имеет тип \TT{const char[]}\InSqBrackets{\TCPPPL p176, 7.3.2}, однако не имеет имени.
Но компилятору нужно как-то с ней работать, поэтому он дает ей внутреннее имя \TT{\$SG3830}.

Поэтому пример можно было бы переписать вот так:

\lstinputlisting[style=customc]{patterns/01_helloworld/hw_2.c}

Вернемся к листингу на ассемблере. Как видно, строка заканчивается нулевым байтом~--- это требования стандарта \CCpp для строк.
Больше о строках в \CCpp: \myref{C_strings}.

В сегменте кода \TT{\_TEXT} находится пока только одна функция: \main{}.
Функция \main, как и практически все функции, начинается с пролога и заканчивается эпилогом
\footnote{Об этом смотрите подробнее в разделе о прологе и эпилоге функции ~(\myref{sec:prologepilog}).}.

\myindex{x86!\Instructions!CALL}
Далее следует вызов функции \printf{}: \INS{CALL \_printf}.
\myindex{x86!\Instructions!PUSH}
Перед этим вызовом адрес строки (или указатель на неё) с нашим приветствием (``Hello, world!'') при помощи инструкции \PUSH помещается в стек.

После того, как функция \printf возвращает управление в функцию \main, адрес строки (или указатель на неё) всё ещё лежит в стеке.
Так как он больше не нужен, то \glslink{stack pointer}{указатель стека} (регистр \ESP) корректируется.

\myindex{x86!\Instructions!ADD}
\INS{ADD ESP, 4} означает прибавить 4 к значению в регистре \ESP.

Почему 4? Так как это 32-битный код, для передачи адреса нужно 4 байта. В x64-коде это 8 байт.\\
\INS{ADD ESP, 4} эквивалентно \TT{POP регистр}, но без использования какого-либо регистра\footnote{Флаги процессора, впрочем, модифицируются}.

\myindex{Intel C++}
\myindex{\oracle}
\myindex{x86!\Instructions!POP}

Некоторые компиляторы, например, Intel C++ Compiler, в этой же ситуации могут вместо 
\ADD сгенерировать \TT{POP ECX} (подобное можно встретить, например, в коде \oracle{}, им скомпилированном),
что почти то же самое, только портится значение в регистре \ECX.
Возможно, компилятор применяет \TT{POP ECX}, потому что эта инструкция короче (1 байт у \TT{POP} против 3 у \TT{ADD}).

Вот пример использования \POP вместо \ADD из \oracle{}:

\begin{lstlisting}[caption=\oracle 10.2 Linux (файл app.o)]
.text:0800029A                 push    ebx
.text:0800029B                 call    qksfroChild
.text:080002A0                 pop     ecx
\end{lstlisting}

%О стеке можно прочитать в соответствующем разделе
% ~(\myref{sec:stack}).
\myindex{\CLanguageElements!return}
После вызова \printf в оригинальном коде на \CCpp указано \TT{return 0}~--- вернуть 0 в качестве результата функции \main.

\myindex{x86!\Instructions!XOR}
В сгенерированном коде это обеспечивается инструкцией \\
\INS{XOR EAX, EAX}.

\myindex{x86!\Instructions!MOV}

\XOR, как легко догадаться~--- \q{исключающее ИЛИ}\footnote{\href{http://go.yurichev.com/17118}{wikipedia}}, но компиляторы часто используют его вместо простого
\INS{MOV EAX, 0} --- снова потому, что опкод короче (2 байта у \XOR против 5 у \MOV).

\myindex{x86!\Instructions!SUB}
Некоторые компиляторы генерируют \INS{SUB EAX, EAX}, что значит \IT{отнять значение в} \EAX \IT{от значения в }\EAX, что в любом случае даст 0 в результате.

\myindex{x86!\Instructions!RET}
Самая последняя инструкция \RET возвращает управление в вызывающую функцию. Обычно это код \CCpp \ac{CRT}, который, в свою очередь, вернёт управление в \ac{OS}.

}
\PTBR{\subsubsection{MSVC}

Vamos compilar esse código no MSVC 2010:

\begin{lstlisting}
cl 1.cpp /Fa1.asm
\end{lstlisting}

(A opção \TT{/Fa} instrui o compilador para gerar o arquivo de listagem em assembly)

\begin{lstlisting}[caption=MSVC 2010]
CONST	SEGMENT
$SG3830	DB	'hello, world', 0AH, 00H
CONST	ENDS
PUBLIC	_main
EXTRN	_printf:PROC
; Function compile flags: /Odtp
_TEXT	SEGMENT
_main	PROC
	push	ebp
	mov	ebp, esp
	push	OFFSET $SG3830
	call	_printf
	add	esp, 4
	xor	eax, eax
	pop	ebp
	ret	0
_main	ENDP
_TEXT	ENDS
\end{lstlisting}

\PTBRph{} \myref{ATT_syntax}.

O compilador gerou o arquivo \TT{1.obj}, que está ligado a \TT{1.exe}.
No nosso caso, o arquivo contém dois segmentos: \TT{CONST} (para informações que são constantes) e \TT{\_TEXT} (para o código).

\myindex{\CLanguageElements!const}
\label{string_is_const_char}
A string \TT{hello, word} em \CCpp tem seu tipo const \TT{const char[]} \InSqBrackets{\TCPPPL p176, 7.3.2}, mas não tem um nome.
O compilador precisa lidar com essa string de alguma maneira, definindo então o nome de \TT{\$SG3830} para ela.

Assim, o código pode ser reescrito da seguinte maneira:

\lstinputlisting[style=customc]{patterns/01_helloworld/hw_2.c}

Vamos voltar para a listagem em assembly. Como podemos ver, a string é delimitada por um byte de valor zero, o que é padrão para strings em \CCpp.
Mais sobre strings em \CCpp: \myref{C_strings}.

No segmento de código \TT{\_TEXT}, só há uma função por enquanto: \main{}.
A função \main{} começa com um código como cabeçalho e termina com outro como rodapé (quase como qualquer outra função)
\footnote{\PTBRph{} ~(\myref{sec:prologepilog}).}.

\myindex{x86!\Instructions!CALL}
Depois do cabeçalho da função, podemos ver a chamada para a função \printf{}: \INS{CALL \_printf}.
\myindex{x86!\Instructions!PUSH}
Antes da chamada, o endereço da string (ou um ponteiro para o mesmo) contendo nossa saudação (``Hello, world!'') é colocado na stack com a ajuda a instrução \PUSH.

Quando o a função printf() retorna o controle para a função main(), o endereço da string (ou o ponteiro para a mesma) ainda está na stack.
Como não precisamos mais dela, o apontador da stack (registrador \ESP) precisa ser corrigido.

\myindex{x86!\Instructions!ADD}
\INS{ADD ESP, 4} significa adicionar 4 para o valor do registrador \ESP.

Mas por que 4? Como esse é um programa de 32-bits, nós precisamos exatamente 4 bytes para endereço passando pela stack.
\INS{ADD ESP, 4} é equivalente a um POP mas sem precisar de nenhum registrador\footnote{\ac{TBT}: CPU flags worden echter wel aangepast}.

\myindex{Intel C++}
\myindex{\oracle}
\myindex{x86!\Instructions!POP}

Pelos mesmos motivos, alguns compiladores (como o Intel C++ Compiler) podem emitir \TT{POP ECX} ao invés de \ADD (esse padrão pode ser observado no código do \oracle{} pois ele é compilado com o Intel C++ Compiler).
Essa instrução tem quase o mesmo efeito mas o conteúdo de ECX seria apagado.
O Intel C++ provavelmente usa \TT{POP ECX} pois o opcode é menor do que \TT{ADD ESP, x} (1 byte para \POP ao invés de 3 para \ADD).

Aqui está um exemplo do uso de \POP ao invés de \ADD do \oracle{}:

\begin{lstlisting}[caption=\oracle 10.2 Linux (app.o file)]
.text:0800029A                 push    ebx
.text:0800029B                 call    qksfroChild
.text:080002A0                 pop     ecx
\end{lstlisting}

\myindex{\CLanguageElements!return}
Depois de chamar \printf{}, o código original em \CCpp contém a declaração \TT{return 0} --- return 0 como o resultado da função \main{}.

\myindex{x86!\Instructions!XOR}
No código gerado, isso é implementado pela instrução \INS{XOR EAX, EAX}.

\myindex{x86!\Instructions!MOV}

\XOR é a condição lógica ``ou exclusivo''\footnote{\href{http://go.yurichev.com/17118}{wikipedia}} que os compiladores geralmente usam ao invés de 
\INS{MOV EAX, 0} --- de novo por causa de um pequeno decréscimo no número de bytes necessários (2 bytes para \XOR contra 5 para a instrução \MOV).

\myindex{x86!\Instructions!SUB}
Alguns compiladores também usam \INS{SUB EAX, EAX}, que significa, SUBtrair o valor contido em \EAX do valor em \EAX, que, em qualquer caso, resultará em zero.

\myindex{x86!\Instructions!RET}
A última instrução \RET retorna o controle para onde a função foi chamada. Geralmente, isso é código \CCpp \ac{CRT}, que retorna o controle para o sistema operacional.

}
\DE{\subsubsection{MSVC}

Das Beispiel wird jezt in MSVC 2010 kompiliert:

\begin{lstlisting}
cl 1.cpp /Fa1.asm
\end{lstlisting}

(Die \TT{/Fa}-Option weist den Compiler an, Assembler-Code auszugeben.)

\begin{lstlisting}[caption=MSVC 2010,style=customasmx86]
CONST	SEGMENT
$SG3830	DB	'hello, world', 0AH, 00H
CONST	ENDS
PUBLIC	_main
EXTRN	_printf:PROC
; Function compile flags: /Odtp
_TEXT	SEGMENT
_main	PROC
	push	ebp
	mov	ebp, esp
	push	OFFSET $SG3830
	call	_printf
	add	esp, 4
	xor	eax, eax
	pop	ebp
	ret	0
_main	ENDP
_TEXT	ENDS
\end{lstlisting}

MSVC erstellt Assembler-Code im Intel-Syntax.
Der Unterschied zum AT\&T-Syntax wird später in \myref{ATT_syntax} behandelt.

Der Compiler generiert die Datei \TT{1.obj}, die anschließend zu \TT{1.exe} gelinkt wird.
In diesem Fall besteht die Datei aus zwei Segmenten: \TT{CONST} (für konstante Daten) und \TT{\_TEXT} (für Quellcode).

\myindex{\CLanguageElements!const}
\label{string_is_const_char}
Die Zeichenkette \TT{hello, world} hat in \CCpp den Typ \TT{const char[]}\InSqBrackets{\TCPPPL p176, 7.3.2}, aber keinen eigenen Bezeichner.
Da der Compiler jedoch irgendwie auf diese Zeichenkette zugreifen muss, definiert er den internen Namen \TT{\$SG3830}.

Aus diesem Grund kann das Beispiel auch wie folgt geschrieben werden:

\lstinputlisting[style=customc]{patterns/01_helloworld/hw_2.c}

Nochmal zurück zum Assembler-Listing: wie man sehen kann ist die Zeichenkette gemäß dem \CCpp-Standard mit einem 0-Byte abgeschlossen.
Mehr über \CCpp-Zeichenketten ist im Abschnitt \myref{C_strings} zu finden.

% Not sure what the technically precise translation of prologue and epilogue is
In dem Code-Segment \TT{\_TEXT} ist lediglich eine Funktion: \main{}.
Diese startet mit einem Prolog-Teil und endet mit einem Epilog-Teil (wie fast alle Funktionen)
\footnote{Mehr darüber in dem Abschnitt über Funktions-Prologe und -Epiloge ~(\myref{sec:prologepilog}).}.

\myindex{x86!\Instructions!CALL}
Nach dem Funktions-Prolog ist der Aufruf der \printf{}-Funktion zu sehen:\\
\INS{CALL \_printf}.
\myindex{x86!\Instructions!PUSH}
Vor dem Aufruf wird die Adresse der Zeichenkette (oder ein Zeiger darauf) mit dem Inhalt unserer Begrüßung
auf dem Stack gespeichert. Dies geschieht durch die \PUSH-Anweisung.

Wenn \printf die Ausführung wieder an \main übergibt, befindet sich die Adresse der Zeichenkette (oder ein Zeiger darauf) immer noch auf dem Stack.
Da diese jedoch nicht mehr benötigt wird, muss der \gls{stack pointer} (das \ESP-Register) korrigiert werden.

\myindex{x86!\Instructions!ADD}
\INS{ADD ESP, 4} bedeutet, dass der Wert 4 zu dem \ESP-Rregister-Wert addiert wird.

Warum 4? Da dies ein 32-Bit-Programm ist, werden exakt 4 Byte benötigt um Adressen auf dem Stack abzulegen. Wenn dies x64-Code wäre, würden 8 Byte benötigt.
\INS{ADD ESP, 4} ist quasi gleichbedeutend mit \INS{POP Register} jedoch ohne die Verwendung von Registern\footnote{Statusregister der CPU können sich jedoch ändern}.

\myindex{Intel C++}
\myindex{\oracle}
\myindex{x86!\Instructions!POP}

Aus dem gleichen Grund generieren einige Compiler (wie der Intel C++-Compiler) die Anweisung \TT{POP ECX}
anstatt \ADD (dieses Muster kann zum Beispiel im \oracle{}-Code gefunden werden, da dieser mit dem Intel-Compiler erstellt wurde).
Diese Anweisung hat nahezu den gleichen Effekt, nur dass die Inhalte des \ECX-Registers überschrieben werden.
Der Intel C++-Compiler nutzt \TT{POP ECX} vermutlich, da der OpCode für diese Anweisung kürzer ist als \TT{ADD ESP, x} (1 Byte für \TT{POP} und 3 Byte für \TT{ADD}),

Nachfolgend ein Beispiel unter der Verwendung von \POP anstatt \ADD aus \oracle{}:

\begin{lstlisting}[caption=\oracle 10.2 Linux (app.o file),style=customasmx86]
.text:0800029A                 push    ebx
.text:0800029B                 call    qksfroChild
.text:080002A0                 pop     ecx
\end{lstlisting}

%Mehr über den Stack gibt es im Abschnitt
% ~(\myref{sec:stack}).
\myindex{\CLanguageElements!return}
Nachdem \printf aufgerufen wurde, enthält der Original-\CCpp-Code die Anweisung \TT{return 0} als Rückgabewert der \main-Funktion.

\myindex{x86!\Instructions!XOR}
In dem hier gezeigten Code ist dies durch die Anweisung \INS{XOR EAX, EAX} realisiert.

\myindex{x86!\Instructions!MOV}

XOR ist lediglich ein \q{exklusives Oder}\footnote{\href{http://go.yurichev.com/17118}{wikipedia}} aber der Compiler nutzt dies oft anstatt
\INS{MOV EAX, 0}---auch hier wieder aufgrund des leicht kürzeren OpCodes (2 Byte für \XOR und 5 Byte für \MOV).

\myindex{x86!\Instructions!SUB}
Einige Compiler erzeugen \INS{SUB EAX, EAX}, was \IT{Subtrahiere den Wert in} \EAX \IT{vom Wert in} \EAX bedeutet.
In jedem Fall erzeugt dies einen Wert von Null.

\myindex{x86!\Instructions!RET}
Die letzte Anweisung \RET gibt die Ausführungskontrolle wieder an die aufrufende Funktion \gls{caller}.
Üblicherweise ist dies \CCpp \ac{CRT}-Code welcher wiederum die Kontrolle an das Betriebssystem (\ac{OS}) übergibt.
}
\PL{\subsubsection{MSVC}

Skompilujmy kod w MSVC 2010:

\begin{lstlisting}
cl 1.cpp /Fa1.asm
\end{lstlisting}

(Klucz \TT{/Fa} oznacza wygenerowanie listingu w asemblerze)

\begin{lstlisting}[caption=MSVC 2010,style=customasmx86]
CONST	SEGMENT
$SG3830	DB	'hello, world', 0AH, 00H
CONST	ENDS
PUBLIC	_main
EXTRN	_printf:PROC
; Function compile flags: /Odtp
_TEXT	SEGMENT
_main	PROC
	push	ebp
	mov	ebp, esp
	push	OFFSET $SG3830
	call	_printf
	add	esp, 4
	xor	eax, eax
	pop	ebp
	ret	0
_main	ENDP
_TEXT	ENDS
\end{lstlisting}

MSVC generuje listingi w syntaksie Intel.
Różnica między syntaksą Intel a AT\&T będzie omówiona trochę później:

Kompilator wygenerował plik \TT{1.obj}, który następnie będzie połączony konsolidatorem ( ang. linker) w \TT{1.exe}.
W naszym przypadku ten plik składa się z dwóch segmentów: \TT{CONST} (dla danych-stałych) i \TT{\_TEXT} (dla kodu).

\myindex{\CLanguageElements!const}
\label{string_is_const_char}
Linia \TT{hello, world} w \CCpp ma typ \TT{const char[]}\InSqBrackets{\TCPPPL p176, 7.3.2}, jednak nie posiada nazwy.
Ale kompilator potrzebuje jakiejś nazwy, żeby z tą linią pracować, dlatego nadaje jej własną nazwę \TT{\$SG3830}.

Dlatego ten przykład można by było zapisać w ten sposób:

\lstinputlisting[style=customc]{patterns/01_helloworld/hw_2.c}

Wróćmy do przykładu w asemblerze. Jak widać, linia się kończy bajtem zerowym~--- są to wymagania standardu \CCpp do linii.
Więcej o liniach w \CCpp: \myref{C_strings}.

W segmencie kodu \TT{\_TEXT} znajduje się na razie tylko jedna f-cja: \main{}.
Funkcja \main, jak i prawie wszystkie funkcje, zaczyna się od prologu i kończy się epilogiem

\myindex{x86!\Instructions!CALL}
Dalej następuje wywołanie funkcji \printf{}: \INS{CALL \_printf}.
\myindex{x86!\Instructions!PUSH}
Przed tym wywołaniem adres linii (lub wskaźnik na nią) z naszym powiadomieniem (``Hello, world!'') za pomocą instrukcji \PUSH odkłada się na stos.

Po tym jak funkcja \printf zwaraca zarządzanie do funkcji \main, adres linii (lub wskaźnik na nią) dalej jest na stosie.
Z racji tego, że nie jest on już więcej potrzebny \glslink{stack pointer}{wskaźnik stosu} (rejestr \ESP) zostaje skorygowany.

\myindex{x86!\Instructions!ADD}
\INS{ADD ESP, 4} oznacza dodawanie wartości 4 do zawartości rejestru \ESP.

Dlaczego 4? Z racji tego, że jest to kod 32-bitowy, do przekazania adresu potrzebujemy 4 bajtów. W x64 są to 8 bajtów.\\
\INS{ADD ESP, 4} ekwiwalentnie \TT{POP rejestr}, ale bez wykorzystania jakiegokolwiek rejestru.

\myindex{Intel C++}
\myindex{\oracle}
\myindex{x86!\Instructions!POP}

Niektóre kompilatory, naprzykład, Intel C++ Compiler, w tej samej sytuacji mogą zamiast 
\ADD wygenerować \TT{POP ECX} (podobne rzeczy można również spotkać, np, w kodzie \oracle{}, skompilowanym w nim samym),
co jest w zasadzie tym samym, o tyle że wartość w rejestrze \ECX się psuje.
Możliwe, że kompilator stosuje tu \TT{POP ECX}, dlatego że ta instrukcja jest krótsza (1 bajt w \TT{POP} kontra 3 w \TT{ADD}).

Oto przykład wykorzystania \POP zamiast \ADD z \oracle{}:

\begin{lstlisting}[caption=\oracle 10.2 Linux (plik app.o),style=customasmx86]
.text:0800029A                 push    ebx
.text:0800029B                 call    qksfroChild
.text:080002A0                 pop     ecx
\end{lstlisting}

%Więcej informacji o stosie można znaleźć w odpowiednim rozdziale
% ~(\myref{sec:stack}).
\myindex{\CLanguageElements!return}
Po wywołaniu \printf w kodzie w \CCpp wskazane \TT{return 0}~--- zwrócić 0 jako wynik f-cji \main.

\myindex{x86!\Instructions!XOR}
W kodzie wygenerowanym robi to instrukcja \\
\INS{XOR EAX, EAX}.

\myindex{x86!\Instructions!MOV}

\XOR, jak można się domyśleć~--- \q{alternatywa wykluczająca}\footnote{\href{http://go.yurichev.com/17118}{wikipedia}}, ale kompilatory często korzystają z niego zamiast prostego
\INS{MOV EAX, 0} --- znowu dlatego, że opcode jest krótszy (2 bajty w \XOR kontra 5 w \MOV).

\myindex{x86!\Instructions!SUB}
Niektóre kompilatory genrują \INS{SUB EAX, EAX}, co oznacza \IT{odjąć wartość w} \EAX \IT{od wartości w }\EAX, co w każdym przypadku da w wyniku końcowym 0.

\myindex{x86!\Instructions!RET}
Ostatnia instrukcja \RET zwraca zarządzanie do funkcji wywołującej. Zwykle jest to kod \CCpp \ac{CRT}, który zwróci zarządzanie do \ac{OS}.


}

\EN{\subsubsection{GCC}

Now let's try to compile the same \CCpp code in the GCC 4.4.1 compiler in Linux: \TT{gcc 1.c -o 1}.
Next, with the assistance of the \IDA disassembler, let's see how the \main function was created.
\IDA, like MSVC, uses Intel-syntax\footnote{We could also have GCC produce assembly listings in Intel-syntax by applying the options \TT{-S -masm=intel}.}.

\begin{lstlisting}[caption=code in \IDA,style=customasm]
main            proc near

var_10          = dword ptr -10h

                push    ebp
                mov     ebp, esp
                and     esp, 0FFFFFFF0h
                sub     esp, 10h
                mov     eax, offset aHelloWorld ; "hello, world\n"
                mov     [esp+10h+var_10], eax
                call    _printf
                mov     eax, 0
                leave
                retn
main            endp
\end{lstlisting}

\myindex{Function prologue}
\myindex{x86!\Instructions!AND}
The result is almost the same.
The address of the \TT{hello, world} string (stored in the data segment) is loaded in the \EAX register first and then it is saved onto the stack. \\
In addition, the function prologue has \INS{AND ESP, 0FFFFFFF0h}~---this 
instruction aligns the \ESP register value on a 16-byte boundary.
This results in all values in the stack being aligned the same way (The CPU performs better if the values it is dealing with are located in memory at addresses aligned 
on a 4-byte or 16-byte boundary)\footnote{\URLWPDA}.

\myindex{x86!\Instructions!SUB}
\INS{SUB ESP, 10h} allocates 16 bytes on the stack. Although, as we can see hereafter, only 4 are necessary here.

This is because the size of the allocated stack is also aligned on a 16-byte boundary.

% TODO1: rewrite.
\myindex{x86!\Instructions!PUSH}
The string address (or a pointer to the string) is then stored directly onto the stack without using the \PUSH instruction.
\IT{var\_10}~---is a local variable and is also an argument for \printf{}.
Read about it below.

Then the \printf function is called.

Unlike MSVC, when GCC is compiling without optimization turned on, it emits \TT{MOV EAX, 0} instead of a shorter opcode.

\myindex{x86!\Instructions!LEAVE}
The last instruction, \LEAVE~---is the equivalent of the \TT{MOV ESP, EBP} and \TT{POP EBP} instruction pair~---in other words, this instruction sets the \gls{stack pointer} (\ESP) back and restores the \EBP register to its initial state.
This is necessary since we modified these register values (\ESP and \EBP) at the beginning of the function (by executing \INS{MOV EBP, ESP} / \INS{AND ESP, \ldots}).

\subsubsection{GCC: \ATTSyntax}
\label{ATT_syntax}

Let's see how this can be represented in assembly language AT\&T syntax.
This syntax is much more popular in the UNIX-world.

\begin{lstlisting}[caption=let's compile in GCC 4.7.3]
gcc -S 1_1.c
\end{lstlisting}

We get this:

\lstinputlisting[caption=GCC 4.7.3,style=customasm]{patterns/01_helloworld/GCC.s}

The listing contains many macros (beginning with dot). These are not interesting for us at the moment.

For now, for the sake of simplification, we can ignore them (except the \IT{.string} macro which
encodes a null-terminated character sequence just like a C-string). Then we'll see this
\footnote{This GCC option can be used to eliminate \q{unnecessary} macros: \IT{-fno-asynchronous-unwind-tables}}:

\lstinputlisting[caption=GCC 4.7.3,style=customasm]{patterns/01_helloworld/GCC_refined.s}

\myindex{\ATTSyntax}
\myindex{\IntelSyntax}
Some of the major differences between Intel and AT\&T syntax are:

\begin{itemize}

\item
Source and destination operands are written in opposite order.

In Intel-syntax: <instruction> <destination operand> <source operand>.

In AT\&T syntax: <instruction> <source operand> <destination operand>.

\myindex{\CStandardLibrary!memcpy()}
\myindex{\CStandardLibrary!strcpy()}
Here is an easy way to memorize the difference:
when you deal with Intel-syntax, you can imagine that there is an equality sign ($=$) between operands
and when you deal with AT\&T-syntax imagine there is a right arrow ($\rightarrow$)
\footnote{By the way, in some C standard functions (e.g., memcpy(), strcpy()) the arguments
are listed in the same way as in Intel-syntax: first the pointer to the destination memory block, and then
the pointer to the source memory block.}.

\item
AT\&T: Before register names, a percent sign must be written (\%) and before numbers a dollar sign (\$).
Parentheses are used instead of brackets.

\item
AT\&T: A suffix is added to instructions to define the operand size:

\begin{itemize}
\item q --- quad (64 bits)
\item l --- long (32 bits)
\item w --- word (16 bits)
\item b --- byte (8 bits)
\end{itemize}

% TODO1 simple example may be? \RU{Например mov\textbf{l}, movb, movw представляют различые версии инсструкция mov} \EN {For example: movl, movb, movw are variations of the mov instruciton}

\end{itemize}

Let's go back to the compiled result: it is identical to what we saw in \IDA.
With one subtle difference: \TT{0FFFFFFF0h} is presented as \TT{\$-16}.
It is the same thing: \TT{16} in the decimal system is \TT{0x10} in hexadecimal.
\TT{-0x10} is equal to \TT{0xFFFFFFF0} (for a 32-bit data type).

\myindex{x86!\Instructions!MOV}
One more thing: the return value is to be set to 0 by using the usual \MOV, not \XOR.
\MOV just loads a value to a register.
Its name is a misnomer (data is not moved but rather copied). In other architectures, this instruction is named \q{LOAD} or \q{STORE} or something similar.

}
\FR{\subsubsection{GCC}

Maintenant essayons de compiler le même code \CCpp avec le compilateur GCC 4.4.1 sur Linux: \TT{gcc 1.c -o 1}.
Ensuite, avec l'aide du désassembleur \IDA, regardons comment la fonction \main a été créée.
\IDA, comme MSVC, utilise la syntaxe Intel\footnote{GCC peut aussi produire un listing assembleur utilisant la syntaxe Intel en lui passant les options \TT{-S -masm=intel}.}.

\begin{lstlisting}[caption=code in \IDA,style=customasmx86]
main            proc near

var_10          = dword ptr -10h

                push    ebp
                mov     ebp, esp
                and     esp, 0FFFFFFF0h
                sub     esp, 10h
                mov     eax, offset aHelloWorld ; "hello, world\n"
                mov     [esp+10h+var_10], eax
                call    _printf
                mov     eax, 0
                leave
                retn
main            endp
\end{lstlisting}

\myindex{Prologue de fonction}
\myindex{x86!\Instructions!AND}
Le résultat est presque le même.
L'adresse de la chaîne \TT{hello, world} (stockée dans le segment de donnée) est d'abord chargée dans
 le registre \EAX puis sauvée dans la pile. \\
En plus, le prologue de la fonction a l'instruction \INS{AND ESP, 0FFFFFFF0h}~---cette
instruction aligne le registre \ESP sur une limite de 16-octet.
Ainsi, toutes les valeurs sur la pile seront alignées de la même manière (Le CPU
est plus performant si les adresses avec lesquelles il travaille en mémoire sont
alignées sur des limites de 4-octet ou 16-octet)\footnote{\URLWPDA}.

\myindex{x86!\Instructions!SUB}
\INS{SUB ESP, 10h} réserve 16 octets sur la pile. Pourtant, comme nous allons le voir, seuls 4 sont nécessaires ici.

C'est parce que la taille de la pile allouée est alignée sur une limite de 16-octet.

% TODO1: rewrite.
\myindex{x86!\Instructions!PUSH}
L'adresse de la chaîne est (ou un pointeur vers la chaîne) est stockée directement sur la pile sans utiliser
l'instruction \PUSH.
\IT{var\_10}~---est une variable locale et est aussi un argument pour \printf{}.
Lisez à ce propos en dessous.

Ensuite la fonction \printf est appelée.

Contrairement à MSVC, lorsque GCC compile sans optimisation, il génère \TT{MOV EAX, 0} au lieu d'un opcode plus court.

\myindex{x86!\Instructions!LEAVE}
La dernière instruction, \LEAVE~---est équivalente à la paire d'instruction \TT{MOV ESP, EBP} et \TT{POP EBP}~---en d'autres mots,
cette instruction déplace le pointeur de pile (\gls{stack pointer}) (\ESP) et remet le registre \EBP dans son état initial.
Cela est nécessaire puisque nous avons modifié les valeurs de ces registres (\ESP et \EBP) au début de la fonction (en exécutant \INS{MOV EBP, ESP} / \INS{AND ESP, \ldots}).

\subsubsection{GCC: \ATTSyntax}
\label{ATT_syntax}

Voyons comment cela peut-être représenté en langage d'assemblage avec la syntaxe AT\&T.
Cette syntaxe est bien plus populaire dans le monde UNIX.

\begin{lstlisting}[caption=compilons avec GCC 4.7.3]
gcc -S 1_1.c
\end{lstlisting}

Nous obtenons ceci:

\lstinputlisting[caption=GCC 4.7.3,style=customasmx86]{patterns/01_helloworld/GCC.s}

Le listing contient beaucoup de macros (qui commencent avec un point). Cela ne nous intéresse pas pour le moment.

Pour l'instant, dans un soucis de simplification, nous pouvons les ignorer (excepté la macro \IT{.string}
qui encode une séquence de caractères terminée par un octet nul, comme une chaîne C).
Ensuite nous verrons cela\footnote{Cette option de GCC peut être utilisée pour éliminer les macros \q{non necéssaire}:
\IT{-fno-asynchronous-unwind-tables}}:

\lstinputlisting[caption=GCC 4.7.3,style=customasmx86]{patterns/01_helloworld/GCC_refined.s}

\myindex{\ATTSyntax}
\myindex{\IntelSyntax}
Quelques-une des différences majeures entre la syntax Intel et AT\&T sont:

\begin{itemize}

\item
Opérandes source et destination sont écrites dnas l'ordre inverse.

En syntax Intel: <instruction> <destination operand> <source operand>.

En syntaxe AT\&T: <instruction> <source operand> <destination operand>.

\myindex{\CStandardLibrary!memcpy()}
\myindex{\CStandardLibrary!strcpy()}
Voici un moyen simple de mémoriser la différence:
lorsque vous avez affaire avec la syntaxe Intel, vous pouvez imaginer qu'il y a un signe
égal ($=$) entre les opérandes et lorsque vous avez affaire avec la syntaxe AT\&T imaginez
qu'il y a un flèche droite ($\rightarrow$)
\footnote{A propos, dans certaine fonction C standard (e.g., memcpy(), strcpy()) les arguments
sont listés de la même manière que dans la syntaxe Intel: en premier se trouve le pointeur du
bloc mémoire de destination, et ensuite le pointeur sur le bloc mémoire source.}.

\item
AT\&T: Avant les noms de registres, un signe pourcent doit être écrit (\%) et avant les nombres, un signe dollar (\$).
Les parenthèses sont utilisées à la place des crochets.

\item
AT\&T: un suffixe est ajouté à l'instruction pour définir la taille de l'opérande:

\begin{itemize}
\item q --- quad (64 bits)
\item l --- long (32 bits)
\item w --- word (16 bits)
\item b --- byte (8 bits)
\end{itemize}

% TODO1 simple example may be? \RU{Например mov\textbf{l}, movb, movw представляют различые версии инсструкция mov} \EN {For example: movl, movb, movw are variations of the mov instruction}
% \FR {Par exemple: movl, movb, movw sont des variantes de l'instruction mov}

\end{itemize}

Retournons au résultat compilé: il est identique à ce que l'on voit dans \IDA.
Avec une différence subtile: \TT{0FFFFFFF0h} est représenté avec \TT{\$-16}.
C'est la même chose: \TT{16} dans le système décimal est \TT{0x10} en hexadécimal.
\TT{-0x10} est équivalent à \TT{0xFFFFFFF0} (pour un type de donnée sur 32-bit).

\myindex{x86!\Instructions!MOV}
Encore une chose: la valeur de retour est mise à 0 en utilisant un \MOV usuel, pas un \XOR.
\MOV charge seulement la valeur dans le registre.
Le nom est mal choisi (la donnée n'est pas déplacée, mais plutôt copiée). Dans d'autres architectures, cette instruction
est nommée \q{LOAD} ou \q{STORE} ou quelque chose de similaire.

}
\RU{\subsubsection{GCC}

Теперь скомпилируем то же самое компилятором GCC 4.4.1 в Linux: \TT{gcc 1.c -o 1}.
Затем при помощи \IDA посмотрим как скомпилировалась функция \main.
\IDA, как и MSVC, показывает код в синтаксисе Intel\footnote{Мы также можем заставить GCC генерировать листинги в этом формате при помощи ключей \TT{-S -masm=intel}.}.

\begin{lstlisting}[caption=код в \IDA,style=customasmx86]
main            proc near

var_10          = dword ptr -10h

                push    ebp
                mov     ebp, esp
                and     esp, 0FFFFFFF0h
                sub     esp, 10h
                mov     eax, offset aHelloWorld ; "hello, world\n"
                mov     [esp+10h+var_10], eax
                call    _printf
                mov     eax, 0
                leave
                retn
main            endp
\end{lstlisting}

\myindex{Function prologue}
\myindex{x86!\Instructions!AND}
Почти то же самое. 
Адрес строки \TT{hello, world}, лежащей в сегменте данных, вначале сохраняется в \EAX, затем записывается в стек.
А ещё в прологе функции мы видим \TT{AND ESP, 0FFFFFFF0h}~--- 
эта инструкция выравнивает значение в \ESP по 16-байтной границе, делая все значения 
в стеке также выровненными по этой границе (процессор более эффективно работает с переменными, расположенными
в памяти по адресам кратным 4 или 16)\footnote{\URLWPDA}.

\myindex{x86!\Instructions!SUB}
\INS{SUB ESP, 10h} выделяет в стеке 16 байт. Хотя, как будет видно далее, здесь достаточно только 4.

Это происходит потому, что количество выделяемого места в локальном стеке тоже выровнено по 16-байтной границе.

% TODO1: rewrite.
\myindex{x86!\Instructions!PUSH}
Адрес строки (или указатель на строку) затем записывается прямо в стек без помощи инструкции \PUSH.
\IT{var\_10} одновременно и локальная переменная и аргумент для \printf{}. Подробнее об этом будет ниже.

Затем вызывается \printf.

В отличие от MSVC, GCC в компиляции без включенной оптимизации генерирует \TT{MOV EAX, 0} вместо более короткого опкода.

\myindex{x86!\Instructions!LEAVE}
Последняя инструкция \LEAVE~--- это аналог команд \TT{MOV ESP, EBP} и \TT{POP EBP}~--- то есть возврат \glslink{stack pointer}{указателя стека} и регистра \EBP в первоначальное состояние.
Это необходимо, т.к. в начале функции мы модифицировали регистры \ESP и \EBP{}\\
(при помощи \INS{MOV EBP, ESP} / \INS{AND ESP, \ldots}).

\subsubsection{GCC: \ATTSyntax}
\label{ATT_syntax}

Попробуем посмотреть, как выглядит то же самое в синтаксисе AT\&T языка ассемблера.
Этот синтаксис больше распространен в UNIX-мире.

\begin{lstlisting}[caption=компилируем в GCC 4.7.3]
gcc -S 1_1.c
\end{lstlisting}

Получим такой файл:

\lstinputlisting[caption=GCC 4.7.3,style=customasmx86]{patterns/01_helloworld/GCC.s}

Здесь много макросов (начинающихся с точки). Они нам пока не интересны.

Пока что, ради упрощения, мы можем 
их игнорировать (кроме макроса \IT{.string}, при помощи которого кодируется последовательность символов, 
оканчивающихся нулем~--- такие же строки как в Си). И тогда получится следующее
\footnote{Кстати, для уменьшения генерации \q{лишних} макросов, можно использовать такой ключ GCC: \IT{-fno-asynchronous-unwind-tables}}:

\lstinputlisting[caption=GCC 4.7.3,style=customasmx86]{patterns/01_helloworld/GCC_refined.s}

\myindex{\ATTSyntax}
\myindex{\IntelSyntax}
Основные отличия синтаксиса Intel и AT\&T следующие:

\begin{itemize}

\item
Операнды записываются наоборот.

В Intel-синтаксисе: \\
<инструкция> <операнд назначения> <операнд-источник>.

В AT\&T-синтаксисе: \\
<инструкция> <операнд-источник> <операнд назначения>.

\myindex{\CStandardLibrary!memcpy()}
\myindex{\CStandardLibrary!strcpy()}
Чтобы легче понимать разницу, можно запомнить следующее:
когда вы работаете с синтаксисом Intel~--- можете в уме ставить знак равенства ($=$) между операндами,
а когда с синтаксисом AT\&T~--- мысленно ставьте стрелку направо ($\rightarrow$)
\footnote{Кстати, в некоторых стандартных функциях библиотеки Си (например, memcpy(), strcpy()) также применяется 
расстановка аргументов как в синтаксисе Intel: вначале указатель в памяти на блок назначения, 
затем указатель на блок-источник.}.

\item
AT\&T: Перед именами регистров ставится символ процента (\%), а перед числами символ доллара (\$).
Вместо квадратных скобок используются круглые.

\item
AT\&T: К каждой инструкции добавляется специальный символ, определяющий тип данных:

\begin{itemize}
\item q --- quad (64 бита)
\item l --- long (32 бита)
\item w --- word (16 бит)
\item b --- byte (8 бит)
\end{itemize}

% TODO1 simple example may be? \RU{Например mov\textbf{l}, movb, movw представляют различые версии инсструкция mov} \EN {For example: movl, movb, movw are variations of the mov instruction}

\end{itemize}

Возвращаясь к результату компиляции: он идентичен тому, который мы посмотрели в \IDA.
Одна мелочь: \TT{0FFFFFFF0h} записывается как \TT{\$-16}.
Это то же самое: \TT{16} в десятичной системе это \TT{0x10} в шестнадцатеричной.
\TT{-0x10} будет как раз \TT{0xFFFFFFF0} (в рамках 32-битных чисел).

\myindex{x86!\Instructions!MOV}
Возвращаемый результат устанавливается в 0 обычной инструкцией \MOV, а не \XOR.
\MOV просто загружает значение в регистр.
Её название не очень удачное (данные не перемещаются, а копируются). В других архитектурах подобная инструкция обычно носит название \q{LOAD} или \q{STORE} или что-то в этом роде.

}
\NL{\subsubsection{GCC}

Nu zullen we dezelfde \CCpp code compileren in de GCC 4.4.1 compiler in Linux: \TT{gcc 1.c -o 1}.
Vervolgens, met de assistentie van de \IDA disassembler, zullen we kijken hoe de \main functie gemaakt is.
\IDA, maakt net als MSVC gebruik van de Intel-syntax\footnote{We hadden GCC ook assembly listings kunnen laten gereren in Intel-syntax door gebruik te maken van de opties \TT{-S -masm=intel}.}.

\begin{lstlisting}[caption=code in \IDA,style=customasmx86]
main            proc near

var_10          = dword ptr -10h

                push    ebp
                mov     ebp, esp
                and     esp, 0FFFFFFF0h
                sub     esp, 10h
                mov     eax, offset aHelloWorld ; "hello, world\n"
                mov     [esp+10h+var_10], eax
                call    _printf
                mov     eax, 0
                leave
                retn
main            endp
\end{lstlisting}

\myindex{Function prologue}
\myindex{x86!\Instructions!AND}
Het resultaat is bijna hetzelfde.
Het adres van de \TT{hello, world} string (opgeslagen in het data segment) wordt eerst ingeladen in het \EAX register en wordt daarna opgeslagen op de stack.
Daarbovenop vind je in de functie proloog hetvolgende terug: \TT{AND ESP, 0FFFFFFF0h}~---
deze instructie lijnt de \ESP registerwaarde uit op een 16-byte begrenzing.
Dit resulteert in het feit dat alle waarden op de stack op dezelfde manier uitgelijnd worden.
De CPU presteert beter als de waarden die hij moet behandelen gelokaliseerd zijn in het geheugen op adressen die gealigneerd zijn op een 4-byte of 16-byte begrenzing.\footnote{URLWPDA}.

\myindex{x86!\Instructions!SUB}
\INS{SUB ESP, 10h} reserveert 16 bytes op de stack. Zoals we hierna echter kunnen zijn, zijn er in dit geval slechts 4 nodig.

Dit komt doordat de grootte van de gereserveerde stack ook uitgelijnd is op een 16-byte begrenzing.

% TODO1: rewrite.
\myindex{x86!\Instructions!PUSH}
Het string adres (of een pointer naar de string) wordt dan rechtstreeks op de stack geplaatst zonder gebruik te maken van de \PUSH instructie.
\IT{var\_10}~---is een lokale variabele en is ook een argument voor \printf{}.
Lees er hieronder meer over.

\NLph{}

In tegenstelling tot MSVC, wanneer GCC compileert zonder optimizatie, maakt het gebruik van \TT{MOV EAX, 0} in plaats van kortere opcodes.

\myindex{x86!\Instructions!LEAVE}
De laatste instructie, \LEAVE~---is het equivalent van het \TT{MOV ESP, EBP} en \TT{POP EBP} instructiepaar.
Met andere woorden, deze instructie zet de \gls{stack pointer} (\ESP) terug, en herstelt het \EBP register
terug tot zijn oorspronkelijke staat.
Dit is nodig aangezien we deze registerwaarden hebben gewijzigd (\ESP en \EBP) in het begin van de functie (door het uitvoeren van \INS{MOV EBP, ESP} / \INS{AND ESP, \ldots}).

\subsubsection{GCC: \ATTSyntax}
\label{ATT_syntax}

Laat ons eens kijken hoe dit kan weergegeven worden in assembly in de AT\&T syntax.
Deze syntax is veel populairder in de UNIX-wereld.

\begin{lstlisting}[caption=\NLph{} GCC 4.7.3]
gcc -S 1_1.c
\end{lstlisting}

We krijgen dit resultaat:

\lstinputlisting[caption=GCC 4.7.3,style=customasmx86]{patterns/01_helloworld/GCC.s}

De lijst bevat vele macros (die beginnen met een punt). Maar deze zijn niet interessant voor ons momenteel.

Voorlopig, om het simpel te houden, kunnen we deze negeren (buiten de \IT{.string} macro, dewelke
een null-terminated karakter reeks encodeert net als een C-string). Daarna zien we dit
\footnote{Deze GCC optie kan gebruikt worden om alle \q{onnodige} macros te elimineren: \IT{-fno-asynchronous-unwind-tables}}:

\lstinputlisting[caption=GCC 4.7.3,style=customasmx86]{patterns/01_helloworld/GCC_refined.s}

\myindex{\ATTSyntax}
\myindex{\IntelSyntax}
Sommige grote verschillen tussen de Intel en AT\&T syntax zijn:

\begin{itemize}

\item
\NLph{}

In Intel-syntax: <instructie> <doel> <bron>.

In AT\&T syntax: <instructie> <bron> <doel>.

\myindex{\CStandardLibrary!memcpy()}
\myindex{\CStandardLibrary!strcpy()}
Een gemakkelijke manier om dit verschil te onthouden is: 
Wanneer je met Intel-syntax te doen krijgt, kan je je inbeelden dat er een gelijkheidsteken ($=$) staat tussen de operands
en met AT\&T-syntax beeld je je in dat er een pijl naar rechts staat ($\rightarrow$)
\footnote{Trouwens, in sommige C standaard functies (bv. memcpy(), strcpy()) worden
de argumenten opgelijst op dezelfde manier als in Intel-syntax: eerst een pointer naar het bestemmings geheugen block, 
gevolgd door een pointer naar de bron.}.

\item
AT\&T: Voor registernamen moet een percentteken geschreven worden (\%) en voor cijfers een dollarteken (\$).
Ronde haakjes worden gebruikt in plaats van haakjes.

\item
AT\&T: Een suffix wordt toegevoegd aan de instructies om de operand grootte te bepalen:

\begin{itemize}
\item q --- quad (64 bits)
\item l --- long (32 bits)
\item w --- word (16 bits)
\item b --- byte (8 bits)
\end{itemize}

\end{itemize}

Laten we even terugblikken op het gecompileerde resultaat: dit is identiek als wat we gezien hebben in \IDA.
Met een klein verschil: \TT{0FFFFFFF0h} wordt weergegeven als \TT{\$-16}.
Dit is hetzelfde: \TT{16} in het decimaalsysteem is \TT{0x10} in hexadecimal.
\TT{-0x10} is gelijk aan \TT{0xFFFFFFF0} (voor een 32-bit data type).

\myindex{x86!\Instructions!MOV}
Nog een ding: de return value wordt best op 0 gezet door gebruik te maken van \MOV, niet van \XOR.
\MOV laadt gewoon een waarde in het register.
De naam is een foute noemer (data wordt niet verplaatst, maar eerder gekopieerd). In andere architecturen wordt deze instructie \q{LOAD} of \q{STORE} of iets soortgelijks genoemd.

}
\ITA{\subsubsection{GCC}

Proviamo adesso a compilare lo stesso codice \CCpp con il compilatore GCC 4.4.1 su Linux: \TT{gcc 1.c -o 1}.
Successivamente, con l'aiuto del disassembler \IDA, vediamo come è stata creata la funzione \main .
\IDA, come MSVC, utilizza la sintassi Intel\footnote{Possiamo anche fare in modo che GCC produca un listato assembly con la sintassi Intel tramite l'opzione \TT{-S -masm=intel}.}.

\begin{lstlisting}[caption=codice in \IDA,style=customasm]
main            proc near

var_10          = dword ptr -10h

                push    ebp
                mov     ebp, esp
                and     esp, 0FFFFFFF0h
                sub     esp, 10h
                mov     eax, offset aHelloWorld ; "hello, world\n"
                mov     [esp+10h+var_10], eax
                call    _printf
                mov     eax, 0
                leave
                retn
main            endp
\end{lstlisting}

\myindex{Function prologue}
\myindex{x86!\Instructions!AND}
Il risultato è pressoché lo stesso.
L'indirizzo della stringa \TT{hello, world} (memorizzato nel data segment) è caricato prima nel registro \EAX e successivamente salvato sullo stack.
Inoltre, il prologo della funzione contiene \TT{AND ESP, 0FFFFFFF0h}~---questa 
istruzione allinea il valore del registro \ESP a 16-byte.
Ciò fa sì che tutti i valori sullo stack siano allineati allo stesso modo (la CPU è più efficiente se i valori che tratta sono collocati in memoria ad indirizzi allineati a, ovvero multipli di, 4 o 16 byte)\footnote{\URLWPDA}.

\myindex{x86!\Instructions!SUB}
\INS{SUB ESP, 10h} alloca 16 byte sullo stack. Tuttavia, come vedremo a breve, solo 4 sono necessari in questo caso.

Ciò è dovuto al fatto che la dimensione dello stack allocato è anch'essa allineata a 16 byte.

% TODO1: rewrite.
\myindex{x86!\Instructions!PUSH}
L'indirizzo della stringa (o un puntatore alla stringa) è quindi memorizzato direttamente sullo stack senza utilizzare l'istruzione \PUSH .
\IT{var\_10}~--- è una variabile locale ed è anche un argomento di \printf{}.
Maggiori dettagli in seguito.

Infine viene chiamata la funzione \printf.

Diversamente da MSVC, quando GCC compila senza ottimizzazione emette \TT{MOV EAX, 0} invece di un opcode più breve.

\myindex{x86!\Instructions!LEAVE}
L'ultima istruzione, \LEAVE~---è l'equivalente della coppia di istruzioni \TT{MOV ESP, EBP} e \TT{POP EBP} ~---in altre parole, questa istruzione riporta indietro lo \gls{stack pointer} (\ESP) e ripristina il registro \EBP al suo stato iniziale.
Ciò è necessario poiché abbiamo modificato i valori di questi registri (\ESP and \EBP) all'inizio della funzione ( eseguendo \INS{MOV EBP, ESP} / \INS{AND ESP, \ldots}).

\subsubsection{GCC: \ATTSyntax}
\label{ATT_syntax}

Vediamo come tutto questo può essere rappresentato nella sintassi assembly AT\&T.
Questa sintassi è molto più popolare nel mondo UNIX.

\begin{lstlisting}[caption=compiliamo in GCC 4.7.3]
gcc -S 1_1.c
\end{lstlisting}

Otteniamo questo:

\lstinputlisting[caption=GCC 4.7.3,style=customasm]{patterns/01_helloworld/GCC.s}

Il listato contiene molte macro (iniziano con il punto). Per il momento non ci interessano.

Per il momento, e solo per una questione di semplificazione, possiamo ignorarle (fatta eccezione per la macro \IT{.string} che codifica una sequenza di caratteri che termina con il null-byte (zero) proprio come una stringa C). Consideriamo soltanto questo
\footnote{Questa opzione di GCC può essere usata per eliminare le macro \q{superflue}: \IT{-fno-asynchronous-unwind-tables}}:

\lstinputlisting[caption=GCC 4.7.3,style=customasm]{patterns/01_helloworld/GCC_refined.s}

\myindex{\ATTSyntax}
\myindex{\IntelSyntax}
Alcune delle differenze maggiori tra la sintassi Intel e quella AT\&T sono:

\begin{itemize}

\item
\ITAph{}

Sintassi Intel: <istruzione> <operando di destinazione> <operando di origine>.

Sintassi AT\&T: <istruzione> <operando di origine> <operando di destinazione>.

\myindex{\CStandardLibrary!memcpy()}
\myindex{\CStandardLibrary!strcpy()}
Ecco un modo facile per memorizzare la differenza:
quando si tratta di sintassi Intel immagina che ci sia un segno di uguaglianza ($=$) tra i due operandi, quando si tratta di sintassi AT\&T immagina una freccia da sinistra a destra ($\rightarrow$)
\footnote{A proposito, in alcune funzioni standard C(es., memcpy(), strcpy()) gli argomenti sono elencati nello stesso modo della sintassi Intel: prima il puntatore al blocco di memoria di destinazione, e poi il puntatore al blocco di memoria di origine.}.

\item
AT\&T: Il simbolo di percentuale (\%) deve essere scritto prima del nome di un registro, e il dollaro (\$) prima dei numeri.

\item
AT\&T: All'istruzione si aggiunge un suffisso che definisce le dimensioni dell'operando:

\begin{itemize}
\item q --- quad (64 bit)
\item l --- long (32 bit)
\item w --- word (16 bit)
\item b --- byte (8 bit)
\end{itemize}

\end{itemize}

Torniamo al risultato compilato: è identico a quello che abbiamo visto in \IDA.
Con una piccola differenza: \TT{0FFFFFFF0h} è presentato come \TT{\$-16}.
E' la stessa cosa: \TT{16} nel sistema decimale è \TT{0x10} in esadecimale.
\TT{-0x10} è uguale a \TT{0xFFFFFFF0} (per un tipo di dato a 32-bit).

\myindex{x86!\Instructions!MOV}
Ancora una cosa: il valore di ritorno viene settato a 0 usando \MOV, non \XOR.
\MOV semplicemente carica un valore in un registro.
Il suo nome è fuorviante (il dato non viene spostato, bensì copiato). In altre architectures questa istruzione è chiamata \q{LOAD} o \q{STORE} o qualcosa di simile.

}
\DE{\subsubsection{GCC}

Als nächstes wird der gleiche \CCpp-Code mit GCC 4.4.1 unter Linux kompiliert: \TT{gcc 1.c -o 1}.
Mithilfe des \IDA-Disassemblers wird untersucht, wie die \main-Funktion erzeugt wurde.
\IDA nutzt, genau wie MSVX den Intel-Syntax\footnote{GCC kann Assembler-Ausgaben im Intel-Syntax erzeugen mit der Options \TT{-S -masm=intel}.}.

\begin{lstlisting}[caption=Code in \IDA,style=customasm]
main            proc near

var_10          = dword ptr -10h

                push    ebp
                mov     ebp, esp
                and     esp, 0FFFFFFF0h
                sub     esp, 10h
                mov     eax, offset aHelloWorld ; "hello, world\n"
                mov     [esp+10h+var_10], eax
                call    _printf
                mov     eax, 0
                leave
                retn
main            endp
\end{lstlisting}

\myindex{Function prologue}
\myindex{x86!\Instructions!AND}
Das Ergebnis ist fast das gleiche.
Die Adresse der \TT{hello, world}-Zeichenkette (im Daten-Segment) wird zunächst in das \EAX-Register geladen und anschließend auf dem Stack gesichert.\\
Zusätzlich beinhaltet der Funktions-Prolog \INS{AND ESP, 0FFFFFFF0h}~---diese
Anweisung richtet den \ESP-Register-Wert an eine 16-Byte-Grenze aus.
Dies führt dazu, dass alle Werte im Stack auf die gleiche Weise ausgerichtet sind.
Die CPU kann Anweisungen schneller ausführen, wenn die zu verarbeitenden Daten auf einer an 4- oder 16-Byte-Grenzen ausgerichteten Adresse liegen\footnote{\URLWPDA}.

\myindex{x86!\Instructions!SUB}
\INS{SUB ESP, 10h} reserviert 16 Byte auf dem Stack, auch wenn - wie später gezeigt wird - nur 4 Byte benötigt werden.

Der Grund liegt darin, dass auch die Größe des Stacks an eine 16-Byte-Grenze ausgerichtet ist.

% TODO1: rewrite.
\myindex{x86!\Instructions!PUSH}
Die Adresse der Zeichenkette (oder ein Zeiger darauf) wird anschließend direkt ohne die \PUSH-Anweisung auf dem Stack gespeichert.
IT{var\_10}~---ist eine lokale Variable und ein Argument für \printf{}.
Mehr dazu später.

Anschließend wird die \printf-Funktion aufgerufen.

Anders als MSVC erzeugt GCC ohne Optimierung Die Anweisung \TT{MOV EAX, 0} anstatt des kürzeren OpCodes.

\myindex{x86!\Instructions!LEAVE}
Die letzte Anweisung \LEAVE ist ein Äquivalent zu der Kombination aus \TT{MOV ESP, EBP} und \TT{POP EBP}.
Mit anderen Worten: diese Anweisung setzt den \gls{stack pointer} (\ESP) zurück und stellt die initalen Werte des \EBP-Registers wieder her.
Dies ist notwendig weil die Registerwerte (\ESP und \EBP) zu Beginn der Funktion (durch \INS{MOV EBP, ESP} / \INS{AND ESP, \ldots}).

\subsubsection{GCC: \ATTSyntax}
\label{ATT_syntax}

Im nächsten Beispiel ist sichtbar, wie dies im AT\%T-Syntax dargestellt werden kann.
Dieser Syntax ist sehr viel populärer in der UNIX-Welt.

\begin{lstlisting}[caption=Das Beispiel kompiliert mit GCC 4.7.3]
gcc -S 1_1.c
\end{lstlisting}

Das Ergebnis ist wie folgt:

\lstinputlisting[caption=GCC 4.7.3,style=customasm]{patterns/01_helloworld/GCC.s}

Der Quellcode beinhaltet Makros (beginnend mit einem Punkt), die hier aber nicht von Belang sind.

An dieser Stelle werden aus Gründen der Übersichtlichkeit alle Makros au0er \IT{.string}
ignoriert. Letzeres kodiert eine Null-terminierte Zeichenkette, die einem C-String entspricht.

Die resultierende Ausgabe ist diese
\footnote{Um die \q{unnötigen} Makros zu unterdrücen kann die GCC-Option \IT{-fno-asynchronous-unwind-tables} genutzt werden}:

\lstinputlisting[caption=GCC 4.7.3,style=customasm]{patterns/01_helloworld/GCC_refined.s}

\myindex{\ATTSyntax}
\myindex{\IntelSyntax}
Einige der Hauptunterschiede zwischen Intel und AT\&T-Syntax sin:

\begin{itemize}

\item
Quell- und Zieloperanden sind in umgekehrter Reihenfolge angegeben.

Im Intel-Syntax: <Anweisung> <Ziel-Operand> <Quell-Operand>.

Im AT\&T-Syntax: <Anweisung> <Quell-Operand> <Ziel-Operand>.

\myindex{\CStandardLibrary!memcpy()}
\myindex{\CStandardLibrary!strcpy()}
Hier ist eine einfache Möglichkeit um sich den Unterschied zu merken:
Beim Umgang mit dem Intel-Syntax, kann man sich ein Gleichheitszeichen ($=$) zwischen den Operanden vorstellen
und beim AT\&T-Syntax einen Pfeil nach rechts ($\rightarrow$)
\footnote{Einige C-Standard-Funktionen (z.B. memcpy(), strcpy()) sind die Parameter ebenfalls wie im
Intel-Syntax aufgelistet: erst der Zeiger zum Ziel, dann der Zeiger auf die Speicher-Quelle)}.

\item
AT\&T: Vor einem Register-Namen muss ein Prozentzeichen (\%) und vor Zahlen ein Dollarzeichen (\$) stehen.
Statt eckigen werden runde Klammern genutzt.

\item
AT\&T: An eine Anweisung ist ein Suffix angehängt, der die Operandengröße angibt:

\begin{itemize}
\item q --- quad (64 bits)
\item l --- long (32 bits)
\item w --- word (16 bits)
\item b --- byte (8 bits)
\end{itemize}

% TODO1 simple example may be? \RU{Например mov\textbf{l}, movb, movw представляют различые версии инсструкция mov} \EN {For example: movl, movb, movw are variations of the mov instruciton} \DE {Zum Beispiel sind movl, movb und movw Variationen der mov-Anweisung}

\end{itemize}

Nochmals zu dem kompilierten Ergebnis: Dieses ist identisch mit der Anzeige in \IDA,
jedoch mit einem kleinen Unterschied: \TT{0FFFFFFF0h} wird als \TT{\$-16} angezeigt.
Der eigentliche Wert ist der selbe: \TT{16} im Dezimalsystem ist \TT{0x10} im Hexadezimalsystem.
Für 32-Bit-Datentypen ist \TT{-0x10} identisch mit \TT{0xFFFFFFF0}.

\myindex{x86!\Instructions!MOV}
Eine weitere Sache: der Rückgabewert ist mittels \MOV auf Null gesetzt, nicht mit \XOR.
\MOV läd lediglich einen Wert in ein Register.
Der Name ist irreführend, da die Daten nicht verschoben, sondern kopiert werden.
In anderen Architekturen ist wird dieser Befehl \q{LOAD} oder \q{STORE} oder ähnlich genannt.
}


\subsection{x86-64}
\EN{\subsubsection{MSVC: x86-64}

\myindex{x86-64}
Let's also try 64-bit MSVC:

\lstinputlisting[caption=MSVC 2012 x64,style=customasmx86]{patterns/01_helloworld/MSVC_x64.asm}

\myindex{fastcall}

In x86-64, all registers were extended to 64-bit, and now their names have an \TT{R-} prefix.
In order to use the stack less often (in other words, to access external memory/cache less often), there is
a popular way to pass function arguments via registers (\IT{fastcall}) \myref{fastcall}.
I.e., a part of the function's arguments are passed in registers, and the rest---via the stack.
In Win64, 4 function arguments are passed in the \RCX, \RDX, \Reg{8}, and \Reg{9} registers.
That is what we see here: a pointer to the string for \printf is now passed not in the stack, but rather in the \RCX register.
The pointers are 64-bit now, so they are passed in the 64-bit registers (which have the \TT{R-} prefix).
However, for backward compatibility, it is still possible to access the 32-bit parts, using the \TT{E-} prefix.
This is how the \RAX/\EAX/\AX/\AL register looks like in x86-64:

\RegTableOne{RAX}{EAX}{AX}{AH}{AL}

The \main function returns an \Tint{}-typed value, which in \CCpp is still 32-bit, for better backward compatibility
and portability, so that is why the \EAX register is cleared at the function end (i.e., the 32-bit
part of the register) instead of with \RAX{}.
There are also 40 bytes allocated in the local stack.
This is called the \q{shadow space}, which we'll talk about later: \myref{shadow_space}.
}
\FR{\subsubsection{MSVC: x86-64}

\myindex{x86-64}
Essayons MSVC 64-bit:

\lstinputlisting[caption=MSVC 2012 x64,style=customasmx86]{patterns/01_helloworld/MSVC_x64.asm}

\myindex{fastcall}

En x86-64, tous les registres ont été étendus à 64-bit et leurs noms ont maintenant le préfixe \TT{R-}.
Afin d'utiliser la pile moins souvent (en d'autres termes, pour accéder moins souvent à la mémoire externe/au cache),
il existe un moyen répandu de passer les arguments aux fonctions par les registres (\IT{fastcall}) \myref{fastcall}.
I.e., une partie des arguments de la fonction est passée par les registres, le reste---par la pile.
En Win64, 4 arguments de fonction sont passés dans les registres \RCX, \RDX, \Reg{8}, \Reg{9}.
C'est ce que l'on voit ci-dessus: un pointeur sur la chaîne pour \printf est passé non pas par la pile,
mais par le registre \RCX.
Les pointeurs font maintenant 64-bit, ils sont donc passés dans les registres 64-bit (qui ont le préfixe \TT{R-}).
Toutefois, pour la rétrocompatibilité, il est toujours possible d'accéder à la partie 32-bits des registres,
en utilisant le préfixe \TT{E-}.
Voici à quoi ressemblent les registres \RAX/\EAX/\AX/\AL en x86-64:

\RegTableOne{RAX}{EAX}{AX}{AH}{AL}

La fonction \main renvoie un type \Tint{}, qui est, en \CCpp, pour une meilleure rétrocompatibilité
et portabilité, toujours 32-bit, c'est pourquoi le registre \EAX est mis à zéro à la fin de la fonction (i.e., la
partie 32-bit du registre) au lieu de \RAX{}.
Il y aussi 40 octets alloués sur la pile locale.
Cela est appelé le \q{shadow space}, dont nous parlerons plus tard: \myref{shadow_space}.
}
\ITA{\subsubsection{MSVC: x86-64}

\myindex{x86-64}
Proviamo anche con MSVC a 64-bit:

\lstinputlisting[caption=MSVC 2012 x64,style=customasm]{patterns/01_helloworld/MSVC_x64.asm}

\myindex{fastcall}

In x86-64, tutti i registri sono stati estesi a 64-bit ed il loro nome ha il prefisso \TT{R-}.
Per usare lo stack meno spesso (in altre parole, per accedere meno spesso alla memoria esterna/cache), esiste un metodo molto diffuso per passare gli argomenti delle funzioni tramite i registri (\IT{fastcall})
\myref{fastcall}.
Ovvero, una parte degli argomenti è passata attraverso i registri, il resto ---attraverso lo stack.
In Win64, 4 argpmenti di funzione sono passati nei registri \RCX, \RDX, \Reg{8}, \Reg{9}.
Questo è ciò che vediamo qui: un puntatore alla stringa per \printf è adesso passato nel registro \RCX anziché tramite lo stack.
I puntatori adesso sono a 64-bit , quindi sono passati nei registri a 64-bit (aventi il prefisso \TT{R-}).
E' comunque possibile, per retrocompatibilità, accedere alle parti a 32-bit parts, usando il prefisso \TT{E-}.
I registri \RAX/\EAX/\AX/\AL in x86-64 appaiono così:

\RegTableOne{RAX}{EAX}{AX}{AH}{AL}

La funzione \main restituisce un valore di tipo \Tint{}, che in \CCpp, per migliore retrocompatibilità e portabilità, resta ancora a 32-bit, motivo per cui il registro \EAX viene svuotato invece di \RAX{} alla fine della funzione (i.e., la parte a 32-bit
del registro).
Ci sono anche 40 byte allocati nello stack locale.
Questo spazio è detto \q{shadow space}, di cui parleremo più avanti: \myref{shadow_space}.

}
\NL{\subsubsection{MSVC: x86-64}

\myindex{x86-64}
Laat ons ook eens kijken naar 64-bit MSVC:

\lstinputlisting[caption=MSVC 2012 x64,style=customasmx86]{patterns/01_helloworld/MSVC_x64.asm}

\myindex{fastcall}

In x86-64 zijn alle registers uitgebreid tot 64-bit, en hebben hun namen een \TT{R-} prefix gekregen.
Om de stack minder te gebruiken (met andere woorden, om het externe geheugen/cache minder vaak te benaderen), bestaat
er een populaire manier om functies parameters door te geven via registers (\IT{fastcall}) \myref{fastcall}.
Bv., een deel van de parameters wordt doorgegeven via het register, de rest --- via de stack.
In Win64, worden 4 functie parameters doorgegeven via de \RCX, \RDX, \Reg{8}, \Reg{9} registers.
Dat is wat we hier zien: een pointer naar de string voor \printf wordt doorgegeven, niet via de stack, maar via het \RCX register.
De pointers zijn 64-bit nu, dus worden ze doorgegeven in de 64-bit registers (dewelke de \TT{R-} prefix hebben).
Voor backward compatibility is het echter nog steeds mogelijk om de 32-bit gedeelten aan te spreken, door gebruik te maken van de \TT{E-} prefix.
Dit is hoe de \RAX/\EAX/\AX/\AL registers eruit zien in x86-64:

\RegTableOne{RAX}{EAX}{AX}{AH}{AL}

De \main functie geeft een \Tint{}-typed waarde terug, hetwelk, in \CCpp, voor betere backward compatibiliteit
en portabiliteit, nog steeds 32-bit is. Daarom wordt het \EAX register ook leeggemaakt bij het einde van de functie
(het 32-bit gedeelte van het register) in plaats van \RAX{}.
Er zijn ook 40 bytes gealloceerd op de lokale stack.
Dit wordt de \q{shadow space} genoemd, waarover we het later nog gaan hebben: \myref{shadow_space}.

}
\RU{\subsubsection{MSVC: x86-64}

\myindex{x86-64}
Попробуем также 64-битный MSVC:

\lstinputlisting[caption=MSVC 2012 x64,style=customasm]{patterns/01_helloworld/MSVC_x64.asm}

\myindex{fastcall}

В x86-64 все регистры были расширены до 64-х бит и теперь имеют префикс \TT{R-}.
Чтобы поменьше задействовать стек (иными словами, поменьше обращаться кэшу и внешней памяти), уже давно имелся
довольно популярный метод передачи аргументов функции через регистры (\IT{fastcall}) \myref{fastcall}.
Т.е. часть аргументов функции передается через регистры и часть ---через стек.
В Win64 первые 4 аргумента функции передаются через регистры \RCX, \RDX, \Reg{8}, \Reg{9}.
Это мы здесь и видим: указатель на строку в \printf теперь передается не через стек, а через регистр \RCX.
Указатели теперь 64-битные, так что они передаются через 64-битные части регистров (имеющие префикс \TT{R-}).
Но для обратной совместимости можно обращаться и к нижним 32 битам регистров используя префикс \TT{E-}.
Вот как выглядит регистр \RAX/\EAX/\AX/\AL в x86-64:

\RegTableOne{RAX}{EAX}{AX}{AH}{AL}

Функция \main возвращает значение типа \Tint, который в \CCpp, надо полагать, для лучшей совместимости и переносимости,
оставили 32-битным. Вот почему в конце функции \main обнуляется не \RAX, а \EAX, т.е. 32-битная часть регистра.
Также видно, что 40 байт выделяются в локальном стеке.
Это \q{shadow space} которое мы будем рассматривать позже: \myref{shadow_space}.
}
\PTBR{\subsubsection{MSVC: x86-64}

\myindex{x86-64}
Vamos tentar também o MSVC 64-bits:

\lstinputlisting[caption=MSVC 2012 x64,style=customasmx86]{patterns/01_helloworld/MSVC_x64.asm}

\myindex{fastcall}

No x86-64, todos os registradores foram extendidos para 64-bits e agora seus nomes contém um \TT{R-} no prefixo.
A fim de diminuir a frequência com que a stack (pilha) é usada (em outras palavras, para acessar memória externa/cache menos vezes),
existe uma maneira popular de passar argumentos para funções através dos registradores (\IT{fastcall}) \myref{fastcall}.
Por exemplo, uma parte dos argumentos da função é passada nos registradores, o resto pela stack.
No Win64, 4 argumentos de funções são passados através dos registradores \RCX, \RDX, \Reg{8}, \Reg{9}.
Que é o que nós vemos, um ponteiro para a string para o printf() não é passado pela stack, mas no registrador \RCX.
Os ponteiros são 64-bits agora, então, eles são passados através dos registradores de 64-bits (que tem prefixo \TT{R-}).
Entretanto, para compatibilidade, ainda é possível acessar partes de 32-bits, usando o prefixo \TT{E-}.
É assim que os registradores \RAX/\EAX/\AX/\AL se parecem no x86-64:

\RegTableOne{RAX}{EAX}{AX}{AH}{AL}

A função \main retorna um valor do tipo inteiro, que em \CCpp é melhor para compatibilidade com versões anteriores e portabilidade,
de 32-bits, por isso o registrador \EAX é limpo no final da função (a parte de 32-bits do registrador) ao invés de \RAX.
Há também 40 bytes alocados na pilha local.
Que é chamado de ``shadow space'', o qual falaremos mais tarde: \myref{shadow_space}.

}
\DE{\subsubsection{MSVC: x86-64}

\myindex{x86-64}
Hier das gleiche Beispiel mit der 64-Bit-Variante von MSVC kompiliert:

\lstinputlisting[caption=MSVC 2012 x64,style=customasm]{patterns/01_helloworld/MSVC_x64.asm}

\myindex{fastcall}

In x86-64 wurden alle Regeister auf 64-Bit erweitert und die Registernamen mit einem \TT{R-}Prefix versehen.
Um den Stack weniger oft zu nutzen (also um auf externen Speicher / Cache selterner zuzugreifen), existiert
ein verbreiteter Weg um Funktionsargumente per Register (\IT{fastcall}) \myref{fastcall} zu übergeben.
Das heißt ein Teil der Funktionsargumente wird in Registern übergeben, der Rest---über den Stack.
In Win64 werden vier Funktionsargumente in den Registern \RCX, \RDX, \Reg{8} und \Reg{9} übergeben.
Das ist was hier sichtbar ist: der Zeiger zu der Zeichenkette für \printf ist jetzt nicht im Stack übergeben sondern im \RCX-Register.
Die Zeiger sind nun 64-Bit breit, also werden sie in den 64-Bit-Registern übergeben (die jetz den \TT{R-}Prefix haben).
Aus Gründen der Rückwärtskompatibilität ist es aber immer noch möglich mit dem \TT{E-}Prefix auf 32-Bit-Teile zuzugrifen.
Nachfolgend, der Aufbau der \RAX/\EAX/\AX/\AL-Register in x86-64:

\RegTableOne{RAX}{EAX}{AX}{AH}{AL}

Die \main-Funktion gibt einen Wert vom Typ \Tint{} zurück, der in \CCpp aus Gründen der Kompatibilität und
Portabilität immernoch 32 Bit breit ist. Daher wird am Ende der Funktion das \EAX-Register auf Null gesetzt
(das heißt der 32-Bit-Part des Registers) anstatt \RAX{}.
Auf dem lokalen Stack sind zusätzliche 40 Byte reserviert.
Dieser Bereich wird \q{shadow space} genannt und wird in Abschnitt \myref{shadow_space} noch genauer betrachtet.
}

\EN{\subsubsection{GCC: x86-64}

\myindex{x86-64}
Let's also try GCC in 64-bit Linux:

\lstinputlisting[caption=GCC 4.4.6 x64,style=customasmx86]{patterns/01_helloworld/GCC_x64_EN.s}

% I think I got the intent right on the following line - Renaissance
Linux, *BSD and \MacOSX also use a method to pass function arguments in registers. \SysVABI{}.

The first 6 arguments are passed in the \RDI, \RSI, \RDX, \RCX, \Reg{8}, and \Reg{9}  registers, and the rest---via the stack.

So the pointer to the string is passed in \EDI (the 32-bit part of the register).
Why doesn't it use the 64-bit part, \RDI?

It is important to keep in mind that all \MOV instructions in 64-bit mode that write something into the lower 32-bit register part also clear the higher 32-bits (as stated in Intel manuals: \myref{x86_manuals}).\\
I.e., the \INS{MOV EAX, 011223344h} writes a value into \RAX correctly, since the higher bits will be cleared.

If we open the compiled object file (.o), we can also see all the instructions' opcodes
\footnote{This must be enabled in \textbf{Options $\rightarrow$ Disassembly $\rightarrow$ Number of opcode bytes}}:

\lstinputlisting[caption=GCC 4.4.6 x64,style=customasmx86]{patterns/01_helloworld/GCC_x64.lst}

\label{hw_EDI_instead_of_RDI}
As we can see, the instruction that writes into \EDI at \TT{0x4004D4} occupies 5 bytes.
The same instruction writing a 64-bit value into \RDI occupies 7 bytes.
Apparently, GCC is trying to save some space.
Besides, it can be sure that the data segment containing the string will not be allocated at the addresses higher than 4\gls{GiB}.

\label{SysVABI_input_EAX}
% There isn't an ABI acronym in acronyms.tex - I figure the intent is to Application Binary Interface,
% so I put it in there (in the EN section, commented out)
We also see that the \EAX register has been cleared before the \printf function call.
This is done because according to \ac{ABI} standard mentioned above,
the number of used vector registers is to be passed in \EAX in *NIX systems on x86-64.
}
\FR{\subsubsection{GCC: x86-64}

\myindex{x86-64}
Essayons GCC sur un Linux 64-bit:

\lstinputlisting[caption=GCC 4.4.6 x64,style=customasmx86]{patterns/01_helloworld/GCC_x64_FR.s}

Une méthode de passage des arguments à la fonction dans des registres est aussi utilisée sur Linux, *BSD et
\MacOSX est \SysVABI.

Les 6 premiers arguments sont passés dans les registres \RDI, \RSI, \RDX, \RCX, \Reg{8}, \Reg{9} et les autres---par
la pile.

Donc le pointeur sur la chaîne est passé dans \EDI (la partie 32-bit du registre).
Mais pourquoi ne pas utiliser la partie 64-bit, \RDI?

Il est important de garder à l'esprit que toutes les instructions \MOV en mode 64-bit qui écrivent quelque chose
dans la partie 32-bit inférieur du registre efface également les 32-bit supérieur (comme indiqué dans les manuels Intel:
\myref{x86_manuals}).\\
I.e., l'instruction \INS{MOV EAX, 011223344h} écrit correctement une valeur dans \RAX, puisque que les bits supérieurs
sont mis à zéro.

Si nous ouvrons le fichier objet compilé (.o), nous pouvons voir tous les opcodes des instructions
\footnote{Ceci doit être activé dans \textbf{Options $\rightarrow$ Disassembly $\rightarrow$ Number of opcode bytes}}:

\lstinputlisting[caption=GCC 4.4.6 x64,style=customasmx86]{patterns/01_helloworld/GCC_x64.lst}

\label{hw_EDI_instead_of_RDI}
Comme on le voit, l'instruction qui écrit dans \EDI en \TT{0x4004D4} occupe 5 octets.
La même instruction qui écrit une valeur sur 64-bit dans \RDI occupe 7 octets.
Il semble que GCC essaye d'économiser un peu d'espace.
En outre, cela permet d'être sûr que le segment de données contenant la chaîne ne sera pas alloué à une adresse supérieure
à 4 \gls{GiB}.

\label{SysVABI_input_EAX}
Nous voyons aussi que le registre \EAX est mis à zéro avant l'appel à la fonction \printf.
Ceci, car conformément à l' \ac{ABI} standard mentionnée plus haut,
le nombre de registres vectoriel utilisés est passé dans \EAX sur les systèmes *NIX en x86-64.

}
\RU{\subsubsection{GCC: x86-64}

\myindex{x86-64}
Попробуем GCC в 64-битном Linux:

\lstinputlisting[caption=GCC 4.4.6 x64,style=customasm]{patterns/01_helloworld/GCC_x64_RU.s}

В Linux, *BSD и \MacOSX для x86-64 также принят способ передачи аргументов функции через регистры \SysVABI.

6 первых аргументов передаются через регистры \RDI, \RSI, \RDX, \RCX, \Reg{8}, \Reg{9}, а остальные --- через стек.

Так что указатель на строку передается через \EDI (32-битную часть регистра).
Но почему не через 64-битную часть, \RDI?

Важно запомнить, что в 64-битном режиме все инструкции \MOV, записывающие что-либо в младшую 32-битную часть регистра, обнуляют старшие 32-бита (это можно найти в документации от Intel: \myref{x86_manuals}).
То есть, инструкция \INS{MOV EAX, 011223344h} корректно запишет это значение в \RAX, старшие биты сбросятся в ноль.

Если посмотреть в \IDA скомпилированный объектный файл (.o), увидим также опкоды всех инструкций
\footnote{Это нужно задать в \textbf{Options $\rightarrow$ Disassembly $\rightarrow$ Number of opcode bytes}}:

\lstinputlisting[caption=GCC 4.4.6 x64]{patterns/01_helloworld/GCC_x64.lst}

\label{hw_EDI_instead_of_RDI}
Как видно, инструкция, записывающая в \EDI по адресу \TT{0x4004D4}, занимает 5 байт.
Та же инструкция, записывающая 64-битное значение в \RDI, занимает 7 байт.
Возможно, GCC решил немного сэкономить.
К тому же, вероятно, он уверен, что сегмент данных, где хранится строка, никогда не будет расположен в адресах выше 4\gls{GiB}.

\label{SysVABI_input_EAX}
Здесь мы также видим обнуление регистра \EAX перед вызовом \printf.
Это делается потому что по упомянутому выше стандарту передачи аргументов в *NIX для x86-64 в \EAX передается количество задействованных векторных регистров.

}
\NL{\subsubsection{GCC: x86-64}

\myindex{x86-64}
Laat ons ook eens kijken naar GCC in 64-bit Linux:

% TODO translate:
\lstinputlisting[caption=GCC 4.4.6 x64,style=customasmx86]{patterns/01_helloworld/GCC_x64_EN.s}

Een methode om functieargumenten door te geven in registers wordt ook gebruikt in Linux, *BSD en \MacOSX \SysVABI.

De eerste 6 argumenten worden doorgegeven in de \RDI, \RSI, \RDX, \RCX, \Reg{8}, \Reg{9} registers, en de rest --- via de stack.

De pointer naar de string wordt dus doorgegeven via \EDI (het 32-bit gedeelte van het register).
Maar waarom gebruikt men niet het 64-bit gedeelte, \RDI?

Het is belangrijk te onthouden dat alle \MOV instructies in 64-bit modus, die iets schrijven in het onderste 32-bit gedeelte van het register, ook het bovenste 32-bit gedeelte leegmaken.
\INS{MOV EAX, 011223344h} schrijft een waarde correct weg in \RAX, aangezien de bovenste bits zullen worden leeggemaakt.

Als we het gecompileerde object-bestand (.o) openen, kunnen we ook de opcodes zien van alle instructies
\footnote{Dit moet ook geactiveerd worden in \textbf{Options $\rightarrow$ Disassembly $\rightarrow$ Number of opcode bytes}}:

\lstinputlisting[caption=GCC 4.4.6 x64,style=customasmx86]{patterns/01_helloworld/GCC_x64.lst}

\label{hw_EDI_instead_of_RDI}
Zoals je kan zien, bezet de instructie die in \EDI schrijft op \TT{0x4004D4} 5 bytes.
Dezelfde instructie die een 64-bit waarde in \RDI schrijft, bezet 7 bytes.
Blijkbaar probeert GCC wat plaats te besparen.
Daarnaast kunnen we met zekerheid zeggen dat het data segment dat de string bevat, niet zal gealloceerd worden op de adressen hoger dan 4\gls{GiB}.

\label{SysVABI_input_EAX}
We zien ook dat het \EAX register leeggemaakt is voor de \printf functie call.
Dit wordt gedaan omdat het aantal gebruikte vector registers wordt doorgegeven in \EAX in *NIX systemen op x86-64.

}
\ITA{\subsubsection{GCC: x86-64}

\myindex{x86-64}
\ITAph{}:

% TODO: translate:
\lstinputlisting[caption=GCC 4.4.6 x64,,style=customasm]{patterns/01_helloworld/GCC_x64_EN.s}

Un metodo per passare argomenti di funzione nei registri usato anche in Linux, *BSD and \MacOSX è \SysVABI.

I primi 6 argomenti sono passati nei registri \RDI, \RSI, \RDX, \RCX, \Reg{8}, \Reg{9}  , ed il resto---tramite lo stack.

Quindi il puntatore alla stringa viene passato in \EDI (la parte a 32-bit del registro).
Ma perchè no nusare la parte a 64-bit \RDI?

E' importante ricordare che tutte le istruzioni \MOV in modalità 64-bit che scrivono qualcosa nella parte bassa a 32-bit di un registro, azzera anche la parte alta a 32-bits.
Ad esempio, \INS{MOV EAX, 011223344h} scrive un valore in \RAX correttamente, poichè i bit della parte alta saranno azzerati.

Se apriamo il file oggetto compilato (.o), possiamo anche vedere gli opcode di tutte le istruzioni
\footnote{Deve essere abilitato in \textbf{Options $\rightarrow$ Disassembly $\rightarrow$ Number of opcode bytes}}:

\lstinputlisting[caption=GCC 4.4.6 x64]{patterns/01_helloworld/GCC_x64.lst}

\label{hw_EDI_instead_of_RDI}
Come possiamo notare, l'istruzione che scrive dentro \EDI a \TT{0x4004D4} occupa 5 byte.
La stessa istruzione che scrive un valore a 64-bit dentro \RDI occupa 7 bytes.
Apparentemente, GCC sta cercando di risparmiare un po' di spazio.
Inoltre, può essere sicuro che il segmento dati contenente la stringa non sarà allocato ad indirizzi maggiori di 4\gls{GiB}.

\label{SysVABI_input_EAX}
Notiamo anche che il registro \EAX è stato azzerato prima della chiamata alla funzione \printf .
Ciò avviene perché il numbero dei registri vettore usati viene passato in \EAX nei sistemi *NIX x86-64.

}
\DE{\subsubsection{GCC: x86-64}

\myindex{x86-64}
Nachfolgend das Beispiel unter einem 64 Bit-Linux-System mit GCC kompoliert:

\lstinputlisting[caption=GCC 4.4.6 x64,style=customasm]{patterns/01_helloworld/GCC_x64_DE.s}

Eine Methode im Funktionsargumente in Registern zu übergeben, wird auch in Linux, *BSD und \MacOSX genutzt und heißt \SysVABI.

Die ersten sechs Argumente sind in den Registern \RDI, \RSI, \RDX, \RCX,\Reg{8} und \Reg{9} übergeben und der Rest---über den Stack.

Der Zeiger zu der Zeichenkette ist in \EDI (also, dem 32-Bit-Teil) gesichert.
Warum wird nicht der 64-Bit-Teil \RDI genutz?

Es ist wichtig sich zu vergegenwertigen, dass alle \MOV-Anweisungen im 64-Bit-Modus, die etwas in den niederwertigen 32-Bit-Teil eines Registers schreiben,
auch den höherwertigen 32-Bit-Teil des Registers löschen (siehe Intel-Handbücher: \myref{x86_manuals}).\\
Die Anweisung \INS{MOV EAX, 011223344h} schreibt also den richtigen Wert in \RAX, weil die höherwetigen Bits auf Null gesetzt werden.

In der Objekt-Datei (.o) eines Kompilats sind ebenfalls alles OpCodes der verwendeten Anweisungen zu sehen.
\footnote{Dies muss aktiviert werden: \textbf{Optionen $\rightarrow$ Disassembly $\rightarrow$ Number of opcode bytes}}:

\lstinputlisting[caption=GCC 4.4.6 x64]{patterns/01_helloworld/GCC_x64.lst}

\label{hw_EDI_instead_of_RDI}
Wie man sehen kann verändert die Anweisung zum Schreiben in \EDI an der Adresse \TT{0x4004D4} fünf Byte.
Dieselbe Anweisung die einen 64-Bit-Wert in \RDI schreibt, verändert 7 Bytes.
Offenstichtlich versucht GCC etwas Speicherplatz zu sparen.
Nebenbei ist es sicher, dass das Datensegment, welches die Zeichenkette entählt niemals an Adressen höher 4\gls{GiB} reserviert wird.

\label{SysVABI_input_EAX}
Es ist auch erkennbar, dass das \EAX-Register vor dem Aufruf von \printf zurückgesetzt wurde.
Dies geschieht, aufgrund der Konvention in der oben genannten \ac{ABI}, dass in *NIX-Systemen auf x86-64-Architektur
die Anzahl der genutzten Vektor-Register in \EAX übergeben wird.
}
\PL{\subsubsection{GCC: x86-64}

\myindex{x86-64}
Spróbujmy GCC na 64-bitowym Linux:

\lstinputlisting[caption=GCC 4.4.6 x64,style=customasmx86]{patterns/01_helloworld/GCC_x64_RU.s}

W Linux, *BSD i \MacOSX dla x86-64 także przekazuje się argumenty funkcji poprzez rejestry \SysVABI.

6 pierwszych argumentów jest przekazywane przez rejestry \RDI, \RSI, \RDX, \RCX, \Reg{8}, \Reg{9}, a reszta --- przez stos.

Także wskaźnik na linię jest przekazywany przez \EDI (32-bitową część rejestru).
Ale dlaczego nie przez 64-bitową część, \RDI?

Warto zapamiętać, że w 64-bitowym trybie wszystkie instrukcje \MOV, zapisujące cokolwiek do młodszej 32-bitowej części rejestru, zerują starsze 32-bity (Jest to opisane w dokumentacji Intel: \myref{x86_manuals}).
Z tego powodu instrukcja \INS{MOV EAX, 011223344h} poprawnie zapisze tę wartość do \RAX, a starsze bity się wyzerują.

Jeśli podejrzeć w \IDA skompilowany plik (.o), to można zobaczyć również opcode wszystkich instrukcji
\footnote{To trzeba wskazać w \textbf{Options $\rightarrow$ Disassembly $\rightarrow$ Number of opcode bytes}}:

\lstinputlisting[caption=GCC 4.4.6 x64,style=customasmx86]{patterns/01_helloworld/GCC_x64.lst}

\label{hw_EDI_instead_of_RDI}
Jak widać, instrukcja, która zapisuje do \EDI wg adresu \TT{0x4004D4}, zajmuje 5 bajtów.
Ta sama instrukcja, zapisująca 64-bitową wartość do \RDI, zajmuje 7 bajtów.
Możliwe, że GCC stwierdził, że trochę zaoszczędzi.
Do tego, on jest pewien, że segment danych, gdzie są przechowywane linie, nigdy nie będzie usytuowany powyżej 4\gls{GiB}.

\label{SysVABI_input_EAX}
Tutaj również widzimy wyzerowanie rejestru \EAX przed wywołaniem \printf.
Z powodu standardu opisanego wyżej, w *NIX dla x86-64 w \EAX jest przekazywana ilość wykorzystywanych rejestrów wektorowych.


}


\EN{\subsection{GCC---one more thing}
\label{use_parts_of_C_strings}

The fact that an \IT{anonymous} C-string has \IT{const} type (\myref{string_is_const_char}), 
and that C-strings allocated in constants segment are guaranteed to be immutable, has an interesting consequence:
the compiler may use a specific part of the string.

Let's try this example:

\begin{lstlisting}[style=customc]
#include <stdio.h>

int f1()
{
	printf ("world\n");
}

int f2()
{
	printf ("hello world\n");
}

int main()
{
	f1();
	f2();
}
\end{lstlisting}

Common \CCpp{}-compilers (including MSVC) allocate two strings, but let's see what GCC 4.8.1 does:

\begin{lstlisting}[caption=GCC 4.8.1 + IDA listing,style=customasmx86]
f1              proc near

s               = dword ptr -1Ch

                sub     esp, 1Ch
                mov     [esp+1Ch+s], offset s ; "world\n"
                call    _puts
                add     esp, 1Ch
                retn
f1              endp

f2              proc near

s               = dword ptr -1Ch

                sub     esp, 1Ch
                mov     [esp+1Ch+s], offset aHello ; "hello "
                call    _puts
                add     esp, 1Ch
                retn
f2              endp

aHello          db 'hello '
s               db 'world',0xa,0
\end{lstlisting}

Indeed: when we print the \q{hello world} string
these two words are positioned in memory adjacently and \puts called from \GTT{f2()}
function is not aware that this string is divided. 
In fact, it's not divided; it's divided only \q{virtually}, in this listing.

When \puts is called from \GTT{f1()}, it uses the \q{world} string plus a zero byte. \puts is not aware that there is something before this string!

This clever trick is often used by at least GCC and can save some memory.
This is close to \IT{string interning}.

Another related example is here: \myref{strstr_example}.

}
\FR{\subsection{GCC---encore une chose}
\label{use_parts_of_C_strings}

Le fait est qu'une chaîne C \IT{anonyme} a un type \IT{const} (\myref{string_is_const_char}),
et que les chaînes C allouées dans le segment des constantes sont garanties d'être immuables, ce qui a pour
conséquence:
Le compilateur peut utiliser une partie spécifique de la chaîne.

Voyons cela avec un exemple:

\begin{lstlisting}[style=customc]
#include <stdio.h>

int f1()
{
	printf ("world\n");
}

int f2()
{
	printf ("hello world\n");
}

int main()
{
	f1();
	f2();
}
\end{lstlisting}

La plupart des compilateurs \CCpp{} (incluant MSVC) allouent deux chaînes, mais voyons ce que fait GCC 4.8.1:

\begin{lstlisting}[caption=GCC 4.8.1 + IDA listing,style=customasmx86]
f1              proc near

s               = dword ptr -1Ch

                sub     esp, 1Ch
                mov     [esp+1Ch+s], offset s ; "world\n"
                call    _puts
                add     esp, 1Ch
                retn
f1              endp

f2              proc near

s               = dword ptr -1Ch

                sub     esp, 1Ch
                mov     [esp+1Ch+s], offset aHello ; "hello "
                call    _puts
                add     esp, 1Ch
                retn
f2              endp

aHello          db 'hello '
s               db 'world',0xa,0
\end{lstlisting}

Effectivement: lorsque nous affichons la chaîne \q{hello world} ses deux mots sont positionnés
consécutivement en mémoire et l'appel à \puts depuis la fonction \GTT{f2()}
n'est pas au courant que la chaîne est divisée.
En fait, elle n'est pas divisée; elle l'est \q{virtuellement}, dans ce listing.

Lorsque \puts est appelé depuis \GTT{f1()}, il utilise la chaîne \q{world} ainsi qu octet à zéro. \puts ne sait pas
qu'il y a quelque chose avant cette chaîne!

Cette astuce est souvent utilisée, au moins par GCC, et permet d'économiser de la mémoire.
C'est proche de \IT{string interning}. % TODO clarify imbrication de chaînes ?

Un autre exemple concernant ceci est là: \myref{strstr_example}.

}
\ITA{% TODO to be resynced with English version
\subsection{GCC---Ancora un'altra cosa}
\label{use_parts_of_C_strings}

Il fatto che una stringa C \IT{anonima} ha \IT{const} tipo (\myref{string_is_const_char}), 
e che le stringhe C allocate nel constants segment siano garantite essere immutabili, hanno una conseguenza interessante:
il compilatore potrebbe utilizzare una parte specifica della stringa.

Un esempio:

\begin{lstlisting}[style=customc]
#include <stdio.h>

int f1()
{
	printf ("world\n");
}

int f2()
{
	printf ("hello world\n");
}

int main()
{
	f1();
	f2();
}
\end{lstlisting}

I comuni compilatori \CCpp{}- (incluso MSVC) allocano due stringhe, ma vediamo cosa fa GCC 4.8.1:

\begin{lstlisting}[caption=GCC 4.8.1 + IDA,style=customasmx86]
f1              proc near

s               = dword ptr -1Ch

                sub     esp, 1Ch
                mov     [esp+1Ch+s], offset s ; "world\n"
                call    _puts
                add     esp, 1Ch
                retn
f1              endp

f2              proc near

s               = dword ptr -1Ch

                sub     esp, 1Ch
                mov     [esp+1Ch+s], offset aHello ; "hello "
                call    _puts
                add     esp, 1Ch
                retn
f2              endp

aHello          db 'hello '
s               db 'world',0xa,0
\end{lstlisting}

Quando stampiamo la stringa \q{hello world} le due parole sono posizionate adiacenti in memoria e \puts chiamata dalla funzione \GTT{f2(}) non è al corrente che la stringa è in divisa. Infatti non lo è; è divisa soltanto \q{virtualmente}, in questo listato.

Quando \puts è chiamata dalla funzione \GTT{f1()}, usa la stringa \q{world} più un byte zero. \puts non sa che c'è qualcos'altro prima di questa stringa!

Questo trucco intelligente è usato spesso, almeno da GCC, e può far risparmiare un po' di memoria.

}
\NL{% TODO to be resynced with English version
\subsection{GCC---Nog een ding}
\label{use_parts_of_C_strings}

Het feit dat een \IT{anonieme} C-string het type \IT{const} heeft (\myref{string_is_const_char}), 
en dat C-string gealloceerd in het segment met de constanten gegarandeerd immuteerbaar zijn, heeft een interessant gevolg:
De compiler kan een specifiek gedeelte van de string gebruiken.

Laten we dit voorbeeld proberen:

\begin{lstlisting}[style=customc]
#include <stdio.h>

int f1()
{
	printf ("world\n");
}

int f2()
{
	printf ("hello world\n");
}

int main()
{
	f1();
	f2();
}
\end{lstlisting}

Veelgebruikte \CCpp{}-compilers (inclusief MSVC) alloceren twee strings, maar laat ons eens kijken wat GCC 4.8.1 doet:

\begin{lstlisting}[caption=GCC 4.8.1 + IDA,style=customasm]
f1              proc near

s               = dword ptr -1Ch

                sub     esp, 1Ch
                mov     [esp+1Ch+s], offset s ; "world\n"
                call    _puts
                add     esp, 1Ch
                retn
f1              endp

f2              proc near

s               = dword ptr -1Ch

                sub     esp, 1Ch
                mov     [esp+1Ch+s], offset aHello ; "hello "
                call    _puts
                add     esp, 1Ch
                retn
f2              endp

aHello          db 'hello '
s               db 'world',0xa,0
\end{lstlisting}

Inderdaad: wanneer we de \q{hello world} string printen, 
Worden deze twee woorden aangrenzend in het geheugen geplaatsd, en \puts, die geroepen wordt van de \GTT{f2()}
functie, is er zicht niet van bewust dat deze string opgedeeld is. 
In feite is hij ook niet opgedeeld, hij is slechts \q{virtueel} opgedeeld in deze listing.

Wanneer \puts aangeroepen wordt vanuit \GTT{f1()}, gebruikt het de \q{world} string plus een nulbyte. \puts is er zich niet van bewust dat er nog iets voor deze string staat!

Dit slimme truukje wordt vaak gebruikt door op zijn minst GCC, en kan geheugen besparen.

}
\RU{\subsection{GCC --- ещё кое-что}
\label{use_parts_of_C_strings}

Тот факт, что \IT{анонимная} Си-строка имеет тип \IT{const} (\myref{string_is_const_char}), 
и тот факт, что выделенные в сегменте констант Си-строки гарантировано неизменяемые (immutable), 
ведет к интересному следствию: компилятор может использовать определенную часть строки.

Вот простой пример:

\begin{lstlisting}[style=customc]

#include <stdio.h>

int f1()
{
	printf ("world\n");
}

int f2()
{
	printf ("hello world\n");
}

int main()
{
	f1();
	f2();
}
\end{lstlisting}

Среднестатистический компилятор с \CCpp (включая MSVC) выделит место для двух строк, но вот что делает GCC 4.8.1:

\begin{lstlisting}[caption=GCC 4.8.1 + листинг в IDA,style=customasmx86]
f1              proc near

s               = dword ptr -1Ch

                sub     esp, 1Ch
                mov     [esp+1Ch+s], offset s ; "world\n"
                call    _puts
                add     esp, 1Ch
                retn
f1              endp

f2              proc near

s               = dword ptr -1Ch

                sub     esp, 1Ch
                mov     [esp+1Ch+s], offset aHello ; "hello "
                call    _puts
                add     esp, 1Ch
                retn
f2              endp

aHello          db 'hello '
s               db 'world',0xa,0
\end{lstlisting}

Действительно, когда мы выводим строку \q{hello world}, 
эти два слова расположены в памяти впритык друг к другу и \puts, вызываясь из функции \GTT{f2()}, вообще не знает,
что эти строки разделены. Они и не разделены на самом деле, они разделены
только \q{виртуально}, в нашем листинге.

Когда \puts вызывается из \GTT{f1()}, он использует строку \q{world} плюс нулевой байт. \puts не знает, что там ещё есть какая-то строка перед этой!

Этот трюк часто используется (по крайней мере в GCC) и может сэкономить немного памяти.
Это близко к \IT{string interning}.

Еще один связанный с этим пример находится здесь: \myref{strstr_example}.

}
\DE{% TODO to be resynced with EN version
\subsection{GCC---eine weitere Sache}
\label{use_parts_of_C_strings}

Die Tatsache, dass eine \IT{anonyme} C-Zeichenkette den Typ \IT{const} hat (\myref{string_is_const_char}), 
und dass C-Zeichenketten im Segment für konstante Daten angelegt sind (was dafür sorgt, dass sie unveränderbar sind),
hat eine interessante Auswirkung: der Compiler kann spezifische Teile der Zeichenkette verwenden.

Probieren wir das folgende Beispiel:

\begin{lstlisting}
#include <stdio.h>

int f1()
{
	printf ("world\n");
}

int f2()
{
	printf ("hello world\n");
}

int main()
{
	f1();
	f2();
}
\end{lstlisting}

Gebräuchliche \CCpp{}-Compiler (inklusive MSVC) allozieren zwei Strings.
Im Folgenden jedoch ist abgebildet, was GCC 4.8.1 erzeugt:

\begin{lstlisting}[caption=GCC 4.8.1 + IDA listing]
f1              proc near

s               = dword ptr -1Ch

                sub     esp, 1Ch
                mov     [esp+1Ch+s], offset s ; "world\n"
                call    _puts
                add     esp, 1Ch
                retn
f1              endp

f2              proc near

s               = dword ptr -1Ch

                sub     esp, 1Ch
                mov     [esp+1Ch+s], offset aHello ; "hello "
                call    _puts
                add     esp, 1Ch
                retn
f2              endp

aHello          db 'hello '
s               db 'world',0xa,0
\end{lstlisting}

Wenn die Zeichenkette \q{hello world} ausgegeben wird, werden die beiden Worte im Speicher nebeneinander
positioniert und die Funktion \puts in \GTT{f2()} merkt nicht, dass die Zeichenkette geteilt ist.
Tatsächlich ist sie lediglich \q{virtuell} in diesem Listing geteilt.

Wenn \puts aus der Funktion \GTT{f1()} aufgerufen wird, wir die \q{world}-Zeichenkette plus einem Null-Byte
genutzt. \puts merkt nicht, dass sich davor noch etwas befindet.

Dieser clevere Trick wird von GCC oft genutzt und ermöglicht das Einsparen von etwas Speicher.

Ein weiteres Beispiel ist hier: \myref{strstr_example}.

}
\PL{\subsection{GCC --- jeszcze co nieco}
\label{use_parts_of_C_strings}

To, że \IT{anonimowa} linia w C ma typ \IT{const} (\myref{string_is_const_char}), 
i to, że zaznaczone w segmencie stałych C-linie są gwarantowanie nie do zmiany (immutable), 
prowadzi do ciekawych konsekwencji: kompilator może korzystać tylko z pewnej części linii.

Prosty przykład:

\begin{lstlisting}[style=customc]

#include <stdio.h>

int f1()
{
	printf ("world\n");
}

int f2()
{
	printf ("hello world\n");
}

int main()
{
	f1();
	f2();
}
\end{lstlisting}

Typowy kompilator \CCpp (w tym MSVC) przydzieli miejsce dla 2 linii, ale o to jest to co robi GCC 4.8.1:

\begin{lstlisting}[caption=GCC 4.8.1 + listing w IDA,style=customasmx86]
f1              proc near

s               = dword ptr -1Ch

                sub     esp, 1Ch
                mov     [esp+1Ch+s], offset s ; "world\n"
                call    _puts
                add     esp, 1Ch
                retn
f1              endp

f2              proc near

s               = dword ptr -1Ch

                sub     esp, 1Ch
                mov     [esp+1Ch+s], offset aHello ; "hello "
                call    _puts
                add     esp, 1Ch
                retn
f2              endp

aHello          db 'hello '
s               db 'world',0xa,0
\end{lstlisting}

Naprawdę, kiedy printujemy linię \q{hello world}, 
te dwa słowa są usytuowane w pamięci tuż obok siebie i \puts, będąc wywołane z funkcji \GTT{f2()}, w ogóle nie wie,
że te linię są rozdzielone. Tak naprawdę to one i nie są rozdzielone, są rozdzielone 
tylko \q{wirtualnie}, w naszym listingu.

Kiedy \puts zostaje wywołane z \GTT{f1()}, ono wykorzystuje linie \q{world} + zerowy bajt. \puts w ogóle nie wie, że tam jest jeszcze jakaś linia przed tą!

Ten trik jest często wykorzystywany (przynajmniej w GCC) i może zaoszczędzić trochę pamięci.
Jest to podobny mechanizm do mechanizmu \IT{internowania łańcuchów}.

Jeszcze jeden przykład można znaleźć tutaj: \myref{strstr_example}.


}
\JPN{\subsection{GCC---もう1つ}
\label{use_parts_of_C_strings}

\IT{匿名の}C文字列が\IT{const}型(\myref{string_is_const_char})を持ち、
定数セグメントに割り当てられたC文字列が不変であることが保証されているという事実は興味深い結果をもたらします:
コンパイラは文字列の特定の部分を使用するかもしれません。

下記の例で試してみましょう。

\begin{lstlisting}[style=customc]
#include <stdio.h>

int f1()
{
	printf ("world\n");
}

int f2()
{
	printf ("hello world\n");
}

int main()
{
	f1();
	f2();
}
\end{lstlisting}

一般的な \CCpp{} コンパイラ(MSVCを含む)は2つの文字列を割り当てますが、GCC 4.8.1の動作を見てみましょう。

\begin{lstlisting}[caption=GCC 4.8.1 + IDA listing,style=customasmx86]
f1              proc near

s               = dword ptr -1Ch

                sub     esp, 1Ch
                mov     [esp+1Ch+s], offset s ; "world\n"
                call    _puts
                add     esp, 1Ch
                retn
f1              endp

f2              proc near

s               = dword ptr -1Ch

                sub     esp, 1Ch
                mov     [esp+1Ch+s], offset aHello ; "hello "
                call    _puts
                add     esp, 1Ch
                retn
f2              endp

aHello          db 'hello '
s               db 'world',0xa,0
\end{lstlisting}

確かに、\q{hello world}という文字列を印刷すると、これらの2つの単語はメモリに隣接して配置され、\GTT{f2()}関数から呼び出される \puts 関数はこの文字列が分割されていることを認識しません。 
実際、分割されていません。 このリストには、\q{仮想的に}分けられています。

\puts が\GTT{f1()}から呼び出されると、\q{world}文字列とNULLバイトを使用します。 \puts はこの文字列の前に何かがあることを認識していません!

この巧妙なトリックは、少なくともGCCでよく使用され、メモリを節約できます。 これは\IT{文字列インターン}に似ています。

別の関連する例がここにあります:\myref{strstr_example}
}


\EN{\section{Network address calculation example}

As we know, a TCP/IP address (IPv4) consists of four numbers in the $0 \ldots 255$ range, i.e., four bytes.

Four bytes can be fit in a 32-bit variable easily, so an IPv4 host address, network mask or network address
can all be 32-bit integers.

From the user's point of view, the network mask is defined as four numbers and is formatted like 255.255.255.0 or so,
but network engineers (sysadmins) use a more compact notation (\ac{CIDR}), like \q{/8}, \q{/16}, etc.

This notation just defines the number of bits the mask has, starting at the \ac{MSB}.

\small
\begin{center}
\begin{tabular}{ | l | l | l | l | l | l | }
\hline
\HeaderColor Mask & 
\HeaderColor Hosts & 
\HeaderColor Usable &
\HeaderColor Netmask &
\HeaderColor Hex mask &
\HeaderColor \\
\hline
/30  & 4        & 2        & 255.255.255.252  & 0xfffffffc  & \\
\hline
/29  & 8        & 6        & 255.255.255.248  & 0xfffffff8  & \\
\hline
/28  & 16       & 14       & 255.255.255.240  & 0xfffffff0  & \\
\hline
/27  & 32       & 30       & 255.255.255.224  & 0xffffffe0  & \\
\hline
/26  & 64       & 62       & 255.255.255.192  & 0xffffffc0  & \\
\hline
/24  & 256      & 254      & 255.255.255.0    & 0xffffff00  & class C network \\
\hline
/23  & 512      & 510      & 255.255.254.0    & 0xfffffe00  & \\
\hline
/22  & 1024     & 1022     & 255.255.252.0    & 0xfffffc00  & \\
\hline
/21  & 2048     & 2046     & 255.255.248.0    & 0xfffff800  & \\
\hline
/20  & 4096     & 4094     & 255.255.240.0    & 0xfffff000  & \\
\hline
/19  & 8192     & 8190     & 255.255.224.0    & 0xffffe000  & \\
\hline
/18  & 16384    & 16382    & 255.255.192.0    & 0xffffc000  & \\
\hline
/17  & 32768    & 32766    & 255.255.128.0    & 0xffff8000  & \\
\hline
/16  & 65536    & 65534    & 255.255.0.0      & 0xffff0000  & class B network \\
\hline
/8   & 16777216 & 16777214 & 255.0.0.0        & 0xff000000  & class A network \\
\hline
\end{tabular}
\end{center}
\normalsize

Here is a small example, which calculates the network address by applying the network mask to the host address.

\lstinputlisting{\CURPATH/netmask.c}

\subsection{calc\_network\_address()}

\TT{calc\_network\_address()} function is simplest one: 
it just ANDs the host address with the network mask, resulting in the network address.

\lstinputlisting[caption=\Optimizing MSVC 2012 /Ob0,numbers=left]{\CURPATH/calc_network_address_MSVC_2012_Ox.asm}

At line 22 we see the most important \AND---here the network address is calculated.

\subsection{form\_IP()}

The \TT{form\_IP()} function just puts all 4 bytes into a 32-bit value.

Here is how it is usually done:

\begin{itemize}
\item Allocate a variable for the return value.  Set it to 0.

\item Take the fourth (lowest) byte, apply OR operation to this byte and return the value.
The return value contain the 4th byte now.

\item Take the third byte, shift it left by 8 bits.
You'll get a value like \TT{0x0000bb00} where \TT{bb} is your third byte.
Apply the OR operation to the resulting value and it.
The return value has contained \TT{0x000000aa} so far, so ORing the values will produce a value 
like \TT{0x0000bbaa}.

\item Take the second byte, shift it left by 16 bits.
You'll get a value like \TT{0x00cc0000}, where \TT{cc} is your second byte.
Apply the OR operation to the resulting value and return it.
The return value has contained \TT{0x0000bbaa} so far, so ORing the values will produce
a value like \TT{0x00ccbbaa}.

\item Take the first byte, shift it left by 24 bits.
You'll get a value like \TT{0xdd000000}, where \TT{dd} is your first byte.
Apply the OR operation to the resulting value and return it.
The return value contain \TT{0x00ccbbaa} so far, so ORing the values will produce
a value like \TT{0xddccbbaa}.

\end{itemize}

And this is how it's done by non-optimizing MSVC 2012:

\lstinputlisting[caption=\NonOptimizing MSVC 2012]{\CURPATH/form_IP_MSVC_2012_EN.asm}

Well, the order is different, but, of course, the order of the operations doesn't matter.

\Optimizing MSVC 2012 does essentially the same, but in a different way:

\lstinputlisting[caption=\Optimizing MSVC 2012 /Ob0]{\CURPATH/form_IP_MSVC_2012_Ox_EN.asm}

We could say that each byte is written to the lowest 8 bits of the return value, 
and then the return value is shifted left by one byte at each step.

Repeat 4 times for each input byte.

\par
That's it! Unfortunately, there are probably no other ways to do it.

There are no popular \ac{CPU}s or \ac{ISA}s which has instruction for composing a value from
bits or bytes.

It's all usually done by bit shifting and ORing.

\subsection{print\_as\_IP()}

\TT{print\_as\_IP()} does the inverse: splitting a 32-bit value into 4 bytes.

Slicing works somewhat simpler: just shift input value by 24, 16, 8 or 0 bits, take the 
bits from zeroth to seventh (lowest byte), and that's it:

\lstinputlisting[caption=\NonOptimizing MSVC 2012]{\CURPATH/print_as_IP_MSVC_2012_EN.asm}

\Optimizing MSVC 2012 does almost the same, but without unnecessary reloading of the input value:

\lstinputlisting[caption=\Optimizing MSVC 2012 /Ob0]{\CURPATH/print_as_IP_MSVC_2012_Ox.asm}

\subsection{form\_netmask() and set\_bit()}

\TT{form\_netmask()} makes a network mask value from \ac{CIDR} notation.
Of course, it would be much effective to use here some kind of a precalculated table, but we consider it in this
way intentionally, to demonstrate bit shifts.

We will also write a separate function \TT{set\_bit()}. 
It's a not very good idea to create a function
for such primitive operation, but it would be easy to understand how it all works.

\lstinputlisting[caption=\Optimizing MSVC 2012 /Ob0]{\CURPATH/form_netmask_MSVC_2012_Ox.asm}

\TT{set\_bit()} is primitive: it just shift left 1 to number of bits we need and then 
ORs it with the \q{input} value.
\TT{form\_netmask()} has a loop: it will set as many bits (starting from the \ac{MSB}) as 
passed in the \TT{netmask\_bits} argument

\subsection{Summary}

That's it!
We run it and getting:

\begin{lstlisting}
netmask=255.255.255.0
network address=10.1.2.0
netmask=255.0.0.0
network address=10.0.0.0
netmask=255.255.255.128
network address=10.1.2.0
netmask=255.255.255.192
network address=10.1.2.64
\end{lstlisting}
}
\FR{\subsection{ARM}
\label{sec:hw_ARM}

\myindex{\idevices}
\myindex{Raspberry Pi}
\myindex{Xcode}
\myindex{LLVM}
\myindex{Keil}
Pour mes expérimentations avec les processeurs ARM, différents compilateurs ont été utilisés:

\begin{itemize}
\item Populaire dans le monde de l'embarqué: Keil Release 6/2013.

\item Apple Xcode 4.6.3 IDE avec le compilateur LLVM-GCC 4.2
\footnote{C'est ainsi: Apple Xcode 4.6.3 utilise les composants open-source GCC comme front-end et LLVM
comme générateur de code} % TODO clarify

\item GCC 4.9 (Linaro) (pour ARM64), disponible comme exécutable win32 ici \url{http://go.yurichev.com/17325}.

\end{itemize}

C'est du code ARM 32-bit qui est utilisé (également pour les modes Thumb et Thumb-2) dans tous les
cas dans ce livre, sauf mention contraire.

% subsections
\subsubsection{\NonOptimizingKeilVI (\ARMMode)}

Commençons par compiler notre exemple avec Keil:

\begin{lstlisting}
armcc.exe --arm --c90 -O0 1.c 
\end{lstlisting}

\myindex{\IntelSyntax}
Le compilateur \IT{armcc} produit un listing assembleur en syntaxe Intel, mais il dispose de macros
de haut niveau liées au processeur ARM\footnote{e.g. les instructions \PUSH/\POP manquent en mode
ARM}. Comme il est plus important pour nous de voir les instructions \q{telles quelles}, nous
regardons le résultat compilé dans \IDA.

\begin{lstlisting}[caption=\NonOptimizingKeilVI (\ARMMode) \IDA,style=customasmARM]
.text:00000000             main
.text:00000000 10 40 2D E9    STMFD   SP!, {R4,LR}
.text:00000004 1E 0E 8F E2    ADR     R0, aHelloWorld ; "hello, world"
.text:00000008 15 19 00 EB    BL      __2printf
.text:0000000C 00 00 A0 E3    MOV     R0, #0
.text:00000010 10 80 BD E8    LDMFD   SP!, {R4,PC}

.text:000001EC 68 65 6C 6C+aHelloWorld  DCB "hello, world",0    ; DATA XREF: main+4
\end{lstlisting}

Dans l'exemple, nous voyons facilement que chaque instruction a une taille de 4 octets.
En effet, nous avons compilé notre code en mode ARM, pas pour Thumb.

\myindex{ARM!\Instructions!STMFD}
\myindex{ARM!\Instructions!POP}
La toute première instruction, \INS{STMFD SP!, \{R4,LR\}}\footnote{\ac{STMFD}},
fonctionne comme une instruction \PUSH en x86, écrivant la valeur de deux registres
(\Reg{4} et \ac{LR}) sur la pile.

En effet, dans le listing de la sortie du compilateur \IT{armcc}, dans un souci
de simplification, il montre l'instruction \INS{PUSH \{r4,lr\}}.
Mais ce n'est pas très précis. L'instruction \PUSH est seulement disponible dans
le mode Thumb.  Donc, pour rendre les choses moins confuses, nous faisons cela
dans \IDA.

Cette instruction \glslink{decrement}{décrémente} d'abord le pointeur de pile \ac{SP}
pour qu'il pointe sur de l'espace libre pour de nouvelles entrées, ensuite elle
sauve les valeurs des registres \Reg{4} et \ac{LR} à cette adresse.

Cette instruction (comme l'instruction \PUSH en mode Thumb) est capable de
sauvegarder plusieurs valeurs de registre à la fois, ce qui peut être très utile.
À propos, elle n'a pas d'équivalent en x86.
On peut noter que l'instruction \TT{STMFD} est une généralisation de l'instruction
\PUSH (étendant ses fonctionnalités), puisqu'elle peut travailler avec n'importe
quel registre, pas seulement avec \ac{SP}.
En d'autres mots, l'instruction \TT{STMFD} peut être utilisée pour stocker un
ensemble de registres à une adresse donnée.

\myindex{\PICcode}
\myindex{ARM!\Instructions!ADR}
L'instruction \INS{ADR R0, aHelloWorld}
ajoute ou soustrait la valeur dans le registre \ac{PC} à l'offset où la chaîne
\TT{hello, world} se trouve.
On peut se demander comment le registre \TT{PC} est utilisé ici ?
C'est appelé du \q{\PICcode}\footnote{Lire à ce propos la section(\myref{sec:PIC})}.

Un tel code peut être exécuté à n'importe quelle adresse en mémoire.
En d'autres mots, c'est un adressage \ac{PC}-relatif. %TODO relatif au \ac{PC} ?
L'instruction \INS{ADR} prend en compte la différence entre l'adresse de cette
instruction et l'adresse où est située la chaîne.
Cette différence (offset) est toujours la même, peu importe à quelle adresse
notre code est chargé par l'\ac{OS}.
C'est pourquoi tout ce dont nous avons besoin est d'ajouter l'adresse de l'instruction
courante (du \ac{PC}) pour obtenir l'adresse absolue en mémoire de notre chaîne C.

\myindex{ARM!\Registers!Link Register}
\myindex{ARM!\Instructions!BL}
L'instruction \INS{BL \_\_2printf}\footnote{Branch with Link} appelle la fonction \printf.
Voici comment fonctionne cette instruction:

\begin{itemize}
\item sauve l'adresse suivant l'instruction \INS{BL} (\TT{0xC}) dans \ac{LR};
\item puis passe le contrôle à \printf en écrivant son adresse dans le registre \ac{PC}.
\end{itemize}

Lorsque la fonction \printf termine son exécution elle doit avoir savoir où elle
doit redonner le contrôle.
C'est pourquoi chaque fonction passe le contrôle à l'adresse se trouvant dans le registre \ac{LR}.

C'est une différence entre un processeur \ac{RISC} \q{pur} comme ARM et un
processeur \ac{CISC} comme x86, où l'adresse de retour est en général sauvée
sur la pile.
Pour aller plus loin, lire la section~(\myref{sec:stack}) suivante.

À propos, une adresse absolue ou un offset de 32-bit ne peuvent être encodés
dans l'instruction 32-bit \TT{BL} car il n'y a qu'un espace de 24 bits.
Comme nous devons nous en souvenir, toutes les instructions ont une taille de
4 octets (32 bits).
Par conséquent, elles ne peuvent se trouver qu'à des adresses alignées dur des
limites de 4 octets.
Cela implique que les 2 derniers bits de l'adresse d'une instruction (qui sont
toujours des bits à zéro) peuvent être omis.
En résumé, nous avons 26 bits pour encoder l'offset. C'est assez pour encoder
$current\_PC \pm{} \approx{}32M$.

\myindex{ARM!\Instructions!MOV}
Ensuite, l'instruction \INS{MOV R0, \#0}\footnote{Signifiant MOVe} écrit juste
0 dans le registre \Reg{0}.
C'est parce que notre fonction C renvoie 0 et la valeur de retour doit être
mise dans le registre \Reg{0}.

\myindex{ARM!\Registers!Link Register}
\myindex{ARM!\Instructions!LDMFD}
\myindex{ARM!\Instructions!POP}
La dernière instruction est \INS{LDMFD SP!, {R4,PC}}\footnote{\ac{LDMFD} est
l'instruction inverse de \ac{STMFD}}.
Elle prend des valeurs sur la pile (ou de toute autre endroit en mémoire)
afin de les sauver dans \Reg{4} et \ac{PC}, et \glslink{increment}{incrémente}
le \glslink{stack pointer}{pointeur de pile} \ac{SP}.
Cela fonctionne ici comme \POP.\\
N.B. La toute première instruction \TT{STMFD} a sauvé la paire de registres
\Reg{4} et \ac{LR} sur la pile, mais \Reg{4} et \ac{PC} sont \IT{restaurés}
pendant l'exécution de \TT{LDMFD}.

Comme nous le savons déjà, l'adresse où chaque fonction doit redonner le
contrôle est usuellement sauvée dans le registre \ac{LR}.
La toute première instruction sauve sa valeur sur la pile car le même
registre va être utilisé par notre fonction \main lors de l'appel à \printf.
A la fin de la fonction, cette valeur peut être écrite directement dans le
registre \ac{PC}, passant ainsi le contrôle là où notre fonction a été appelée.
Comme \main est en général la première fonction en \CCpp, le contrôle sera
redonné au chargeur de l'\ac{OS} ou a un point dans un \ac{CRT}, ou quelque
chose comme ça.

Tout cela permet d'omettre l'instruction \INS{BX LR} à la fin de la fonction.

\myindex{ARM!DCB}
\TT{DCB} est une directive du langage d'assemblage définissant un tableau d'octets
ou des chaînes ASCII, proche de la directive DB dans le langage d'assemblage x86.


\subsubsection{\NonOptimizingKeilVI (\ThumbMode)}

Compilons le même exemple en utilisant keil en mode Thumb:

\begin{lstlisting}
armcc.exe --thumb --c90 -O0 1.c 
\end{lstlisting}

Nous obtenons (dans \IDA):

\begin{lstlisting}[caption=\NonOptimizingKeilVI (\ThumbMode) + \IDA,style=customasmARM]
.text:00000000             main
.text:00000000 10 B5          PUSH    {R4,LR}
.text:00000002 C0 A0          ADR     R0, aHelloWorld ; "hello, world"
.text:00000004 06 F0 2E F9    BL      __2printf
.text:00000008 00 20          MOVS    R0, #0
.text:0000000A 10 BD          POP     {R4,PC}

.text:00000304 68 65 6C 6C+aHelloWorld  DCB "hello, world",0    ; DATA XREF: main+2
\end{lstlisting}

Nous pouvons repérer facilement les opcodes sur 2 octets (16-bit). C'est, comme déjà noté, Thumb.
\myindex{ARM!\Instructions!BL}
L'instruction \TT{BL}, toutefois, consiste en deux instructions 16-bit.
C'est parce qu'il est impossible de charger un offset pour la fonction \printf
en utilisant seulement le petit espace dans un opcode 16-bit.
Donc, la première instruction 16-bit charge les 10 bits supérieurs de l'offset
et la seconde instruction les 11 bits inférieurs de l'offset.

% TODO:
% BL has space for 11 bits, so if we don't encode the lowest bit,
% then we should get 11 bits for the upper half, and 12 bits for the lower half.
% And the highest bit encodes the sign, so the destination has to be within
% \pm 4M of current_PC.
% This may be less if adding the lower half does not carry over,
% but I'm not sure --all my programs have 0 for the upper half,
% and don't carry over for the lower half.
% It would be interesting to check where __2printf is located relative to 0x8
% (I think the program counter is the next instruction on a multiple of 4
% for THUMB).
% The lower 11 bytes of the BL instructions and the even bit are
% 000 0000 0110 | 001 0010 1110 0 = 000 0000 0110 0010 0101 1100 = 0x00625c,
% so __2printf should be at 0x006264.
% But if we only have 10 and 11 bits, then the offset would be:
% 00 0000 0110 | 01 0010 1110 0 = 0 0000 0011 0010 0101 1100 = 0x00325c,
% so __2printf should be at 0x003264.
% In this case, though, the new program counter can only be 1M away,
% because of the highest bit is used for the sign.

Comme il a été écrit, toutes les instructions en mode Thumb ont une taille de 2
octets (ou 16 bits).
Cela implique qu'il impossible pour une instruction Thumb d'être à une adresse
impaire, quelle qu'elle soit.
En tenant compte de cela, le dernier bit de l'adresse peut être omis lors de
l'encodage des instructions.

En résumé, l'instruction Thumb \TT{BL} peut encoder une adresse en $current\_PC \pm{}\approx{}2M$.

\myindex{ARM!\Instructions!PUSH}
\myindex{ARM!\Instructions!POP}
Comme pour les autres instructions dans la fonction: \PUSH et \POP fonctionnent
ici comme les instructions décrites \TT{STMFD}/\TT{LDMFD} seul le
registre \ac{SP} n'est pas mentionné explicitement ici.
\TT{ADR} fonctionne comme dans l'exemple précédent.
\TT{MOVS} écrit 0 dans le registre \Reg{0} afin de renvoyer zéro.


\subsubsection{\OptimizingXcodeIV (\ARMMode)}

Xcode 4.6.3 sans l'option d'optimisation produit beaucoup de code redondant c'est
pourquoi nous allons étudier le code généré avec optimisation, où le nombre
d'instruction est aussi petit que possible, en mettant l'option \Othree du
compilateur.

\begin{lstlisting}[caption=\OptimizingXcodeIV (\ARMMode),style=customasmARM]
__text:000028C4             _hello_world
__text:000028C4 80 40 2D E9   STMFD           SP!, {R7,LR}
__text:000028C8 86 06 01 E3   MOV             R0, #0x1686
__text:000028CC 0D 70 A0 E1   MOV             R7, SP
__text:000028D0 00 00 40 E3   MOVT            R0, #0
__text:000028D4 00 00 8F E0   ADD             R0, PC, R0
__text:000028D8 C3 05 00 EB   BL              _puts
__text:000028DC 00 00 A0 E3   MOV             R0, #0
__text:000028E0 80 80 BD E8   LDMFD           SP!, {R7,PC}

__cstring:00003F62 48 65 6C 6C+aHelloWorld_0  DCB "Hello world!",0
\end{lstlisting}

Les instructions \TT{STMFD} et \TT{LDMFD} nous sont déjà familières.

\myindex{ARM!\Instructions!MOV}

L'instruction \MOV écrit simplement le nombre \TT{0x1686} dans le registre \Reg{0}.
C'est l'offset pointant sur la chaîne \q{Hello world!}.

Le registre \TT{R7} (tel qu'il est standardisé dans \IOSABI) est un pointeur de frame. Voir plus loin.

\myindex{ARM!\Instructions!MOVT}
L'instruction \TT{MOVT R0, \#0} (MOVe Top) écrit 0 dans les 16 bits de poids
fort du registre.
Le problème ici est que l'instruction générique \MOV en mode ARM peut n'écrire
que dans les 16 bits de poids faible du registre.

Il faut garder à l'esprit que tout les opcodes d'instruction en mode ARM sont
limités en taille à 32 bits. Bien sûr, cette limitation n'est pas relative
au déplacement de données entre registres.
C'est pourquoi une instruction supplémentaire existe \TT{MOVT} pour écrire dans
les bits de la partie haute (de 16 à 31 inclus).
Son usage ici, toutefois, est redondant car l'instruction \TT{MOV R0, \#0x1686}
ci dessus a éffacé la partie haute du registre.
C'est soi-disant un défaut du compilateur.
% TODO:
% I think, more specifically, the string is not put in the text section,
% ie. the compiler is actually not using position-independent code,
% as mentioned in the next paragraph.
% MOVT is used because the assembly code is generated before the relocation,
% so the location of the string is not yet known,
% and the high bits may still be needed.

\myindex{ARM!\Instructions!ADD}
L'instruction \TT{ADD R0, PC, R0} ajoute la valeur dans \ac{PC} à celle de
\Reg{0}, pour calculer l'adresse absolue de la chaîne \q{Hello world!}.
Comme nous l'avons déjà vu, il s'agit de \q{\PICcode} donc la correction
est essentielle ici.

L'instruction \INS{BL} appelle la fonction \puts au lieu de \printf.

\label{puts}
\myindex{\CStandardLibrary!puts()}
\myindex{puts() instead of printf()}

GCC a remplacé le premier appel à \printf par un à \puts.
Effectivement: \printf avec un unique argument est presque analogue à \puts.

\IT{Presque}, car les deux fonctions produisent le même résultat uniquement dans
le cas où la chaîne ne contient pas d'identifiants de format débutant par \IT{\%}.
Dans le cas où elle en contient, l'effet de ces deux fonctions est différent.
\footnote{Il est à noter que \puts ne nécessite pas un `\textbackslash{}n'
symbole de retour à la ligne à la fin de la chaîne, donc nous ne le voyons pas ici.}.

Pourquoi est-ce que le compilateur a remplacé \printf par \puts? Probablement car
\puts est plus rapide.
\footnote{\href{http://go.yurichev.com/17063}{ciselant.de/projects/gcc\_printf/gcc\_printf.html}}. 

Car il envoie seulement les caractères dans \glslink{stdout}{sortie standard}
sans comparer chacun d'entre eux avec le symbole \IT{\%}.

Ensuite, nous voyons l'instruction familière \TT{MOV R0, \#0} pour mettre le
registre \Reg{0} à 0.

\subsubsection{\OptimizingXcodeIV (\ThumbTwoMode)}

By default Xcode 4.6.3 generates code for Thumb-2 in this manner:

\begin{lstlisting}[caption=\OptimizingXcodeIV (\ThumbTwoMode),style=customasm]
__text:00002B6C                   _hello_world
__text:00002B6C 80 B5          PUSH            {R7,LR}
__text:00002B6E 41 F2 D8 30    MOVW            R0, #0x13D8
__text:00002B72 6F 46          MOV             R7, SP
__text:00002B74 C0 F2 00 00    MOVT.W          R0, #0
__text:00002B78 78 44          ADD             R0, PC
__text:00002B7A 01 F0 38 EA    BLX             _puts
__text:00002B7E 00 20          MOVS            R0, #0
__text:00002B80 80 BD          POP             {R7,PC}

...

__cstring:00003E70 48 65 6C 6C 6F 20+aHelloWorld  DCB "Hello world!",0xA,0
\end{lstlisting}

% Q: If you subtract 0x13D8 from 0x3E70,
% you actually get a location that is not in this function, or in _puts.
% How is PC-relative addressing done in THUMB2?
% A: it's not Thumb-related. there are just mess with two different segments. TODO: rework this listing.

\myindex{\ThumbTwoMode}
\myindex{ARM!\Instructions!BL}
\myindex{ARM!\Instructions!BLX}

The \TT{BL} and \TT{BLX} instructions in Thumb mode, as we recall, are encoded as a pair of 16-bit instructions.
In Thumb-2 these \IT{surrogate} opcodes are extended in such a way so that new instructions may be encoded here as 32-bit instructions.

That is obvious considering that the opcodes of the Thumb-2 instructions always begin with \TT{0xFx} or \TT{0xEx}.

But in the \IDA listing
the opcode bytes are swapped because for ARM processor the instructions are encoded as follows: 
last byte comes first and after that comes the first one (for Thumb and Thumb-2 modes) 
or for instructions in ARM mode the fourth byte comes first, then the third,
then the second and finally the first (due to different \gls{endianness}).

So that is how bytes are located in IDA listings:
\begin{itemize}
\item for ARM and ARM64 modes: 4-3-2-1;
\item for Thumb mode: 2-1;
\item for 16-bit instructions pair in Thumb-2 mode: 2-1-4-3.
\end{itemize}

\myindex{ARM!\Instructions!MOVW}
\myindex{ARM!\Instructions!MOVT.W}
\myindex{ARM!\Instructions!BLX}

So as we can see, the \TT{MOVW}, \TT{MOVT.W} and \TT{BLX} instructions begin with \TT{0xFx}.

One of the Thumb-2 instructions is \TT{MOVW R0, \#0x13D8} ~---it stores a 16-bit value into the lower part of the \Reg{0} register, clearing the higher bits.

Also, \TT{MOVT.W R0, \#0} ~works just like \TT{MOVT} from the previous example only it works in Thumb-2.

\myindex{ARM!mode switching}
\myindex{ARM!\Instructions!BLX}

Among the other differences, the \TT{BLX} instruction is used in this case instead of the \TT{BL}.

The difference is that, besides saving the \ac{RA} in the \ac{LR} register and passing control 
to the \puts function, the processor is also switching from Thumb/Thumb-2 mode to ARM mode (or back).

This instruction is placed here since the instruction to which control is passed looks like (it is encoded in ARM mode):

\begin{lstlisting}[style=customasm]
__symbolstub1:00003FEC _puts           ; CODE XREF: _hello_world+E
__symbolstub1:00003FEC 44 F0 9F E5     LDR  PC, =__imp__puts
\end{lstlisting}

This is essentially a jump to the place where the address of \puts is written in the imports' section.

So, the observant reader may ask: why not call \puts right at the point in the code where it is needed?

Because it is not very space-efficient.

\myindex{Dynamically loaded libraries}
Almost any program uses external dynamic libraries (like DLL in Windows, .so in *NIX or .dylib in \MacOSX).
The dynamic libraries contain frequently used library functions, including the standard C-function \puts.

\myindex{Relocation}
In an executable binary file (Windows PE .exe, ELF or Mach-O) an import section is present.
This is a list of symbols (functions or global variables) imported from external modules along with the names of the modules themselves.

The \ac{OS} loader loads all modules it needs and, while enumerating import symbols in the primary module, determines the correct addresses of each symbol.

In our case, \IT{\_\_imp\_\_puts} is a 32-bit variable used by the \ac{OS} loader to store the correct address of the function in an external library. 
Then the \TT{LDR} instruction just reads the 32-bit value from this variable and writes it into the \ac{PC} register, passing control to it.

So, in order to reduce the time the \ac{OS} loader needs for completing this procedure, 
it is good idea to write the address of each symbol only once, to a dedicated place.

\myindex{thunk-functions}
Besides, as we have already figured out, it is impossible to load a 32-bit value into a register 
while using only one instruction without a memory access.

Therefore, the optimal solution is to allocate a separate function working in ARM mode with the sole 
goal of passing control to the dynamic library and then to jump to this short one-instruction function (the so-called \gls{thunk function}) from the Thumb-code.

\myindex{ARM!\Instructions!BL}
By the way, in the previous example (compiled for ARM mode) the control is passed by the \TT{BL} to the 
same \gls{thunk function}.
The processor mode, however, is not being switched (hence the absence of an \q{X} in the instruction mnemonic).

\myparagraph{More about thunk-functions}
\myindex{thunk-functions}

Thunk-functions are hard to understand, apparently, because of a misnomer.
The simplest way to understand it as adaptors or convertors of one type of jack to another.
For example, an adaptor allowing the insertion of a British power plug into an American wall socket, or vice-versa. 
Thunk functions are also sometimes called \IT{wrappers}.

Here are a couple more descriptions of these functions:

\begin{framed}
\begin{quotation}
“A piece of coding which provides an address:”, according to P. Z. Ingerman, 
who invented thunks in 1961 as a way of binding actual parameters to their formal 
definitions in Algol-60 procedure calls. If a procedure is called with an expression 
in the place of a formal parameter, the compiler generates a thunk which computes 
the expression and leaves the address of the result in some standard location.

\dots

Microsoft and IBM have both defined, in their Intel-based systems, a “16-bit environment” 
(with bletcherous segment registers and 64K address limits) and a “32-bit environment” 
(with flat addressing and semi-real memory management). The two environments can both be 
running on the same computer and OS (thanks to what is called, in the Microsoft world, 
WOW which stands for Windows On Windows). MS and IBM have both decided that the process 
of getting from 16- to 32-bit and vice versa is called a “thunk”; for Windows 95, 
there is even a tool, THUNK.EXE, called a “thunk compiler”.
\end{quotation}
\end{framed}
% TODO FIXME move to bibliography and quote properly above the quote
( \href{http://go.yurichev.com/17362}{The Jargon File} )

\myindex{LAPACK}
\myindex{FORTRAN}
Another example we can find in LAPACK library---a ``Linear Algebra PACKage'' written in FORTRAN.
\CCpp developers also want to use LAPACK, but it's insane to rewrite it to \CCpp and then maintain several versions.
So there are short C functions callable from \CCpp environment, which are, in turn, call FORTRAN functions,
and do almost anything else:

\begin{lstlisting}[style=customc]
double Blas_Dot_Prod(const LaVectorDouble &dx, const LaVectorDouble &dy)
{
    assert(dx.size()==dy.size());
    integer n = dx.size();
    integer incx = dx.inc(), incy = dy.inc();

    return F77NAME(ddot)(&n, &dx(0), &incx, &dy(0), &incy);
}
\end{lstlisting}

Also, functions like that are called ``wrappers''.


\subsubsection{ARM64}

\myparagraph{GCC}

Compilons l'exemple en utilisant GCC 4.8.1 en ARM64:

\lstinputlisting[numbers=left,label=hw_ARM64_GCC,caption=GCC 4.8.1 \NonOptimizing + objdump,style=customasmARM]{patterns/01_helloworld/ARM/hw.lst}

Il n'y a pas de mode Thumb ou Thumb-2 en ARM64, seulement en ARM, donc il n'y a que des
instructions 32-bit.
Le nombre de registres a doublé: \myref{ARM64_GPRs}.
Les registres 64-bit ont le préfixe \TT{X-}, tandis que leurs partie 32-bit basse---\TT{W-}.

\myindex{ARM!\Instructions!STP}
L'instruction \TT{STP} (\IT{Store Pair} stocke une paire)
sauve deux registres sur la pile simultanément: \RegX{29} et \RegX{30}.

Bien sûr, cette instruction peut sauvegarder cette paire à n'importe quelle endroit en mémoire,
mais le registre \ac{SP} est spécifié ici, donc la paire est sauvé sur le pile.

Les registres ARM64 font 64-bit, chacun a une taille de 8 octets, donc il faut 16 octets pour sauver
deux registres.

Le point d'exclamation (``!'') après l'opérande signifie que 16 octets doivent d'abord être soustrait de \ac{SP},
et ensuite les valeurs de la paire de registres peuvent être écrites sur la pile.
Ceci est appelé le \IT{pre-index}.
À propos de la différence entre \IT{post-index} et \IT{pre-index}
lisez ceci: \myref{ARM_postindex_vs_preindex}.

Dans la gamme plus connue du x86, la première instruction est analogue à la paire
\TT{PUSH X29} et \TT{PUSH X30}.
En ARM64, \RegX{29} est utilisé comme \ac{FP} et \RegX{30} comme \ac{LR}, c'est pourquoi ils sont
sauvegardés dans le prologue de la fonction et remis dans l'épilogue.

La seconde instruction copie \ac{SP} dans \RegX{29} (ou \ac{FP}).
Cela sert à préparer la pile de la fonction.

\label{pointers_ADRP_and_ADD}
\myindex{ARM!\Instructions!ADRP/ADD pair}
Les instructions \TT{ADRP} et \ADD sont utilisées pour remplir l'adresse de
la chaîne \q{Hello!} dans le registre \RegX{0},
car le premier argument de la fonction est passé dans ce registre.
Il n'y a pas d'instruction, quelqu'elle soit, en ARM qui puisse stocker un nombre large
dans un registre (car la longueur des instructions est limitée à 4 octets, cf: \myref{ARM_big_constants_loading}).
Plusieurs instructions doivent donc être utilisées. La première instruction (\TT{ADRP}) écrit l'adresse de
la page de 4KiB, où se trouve la chaîne, dans \RegX{0}, et la seconde (\ADD) ajoute simplement
le reste de l'adresse.
Plus d'information ici: \myref{ARM64_relocs}.

\TT{0x400000 + 0x648 = 0x400648}, et nous voyons notre chaîne C \q{Hello!} dans le \TT{.rodata} segment
des données à cette adresse.

\myindex{ARM!\Instructions!BL}

\puts est appelée après en utilisant l'instruction \TT{BL}. Cela a déjà été discuté: \myref{puts}.

\MOV écrit 0 dans \RegW{0}.
\RegW{0} est la partie basse 32 bits du registre 64-bit \RegX{0}:

\input{ARM_X0_register}

Le résultat de la fonction est retourné via \RegX{0} et main renvoie 0, donc c'est ainsi que la valeur
de retour est préparée.
Mais pourquoi utiliser la partie 32-bit?

Parce que ls type de donnée \Tint en ARM64, tout comme en x86-64, est toujours 32-bit, pour une
meilleure compatibilité.

Donc si la fonction renvoie un \Tint 32-bit, seul les 32 premiers bits du registre \RegX{0} doivent
être rempli.

Pour vérifier ceci, changer un peu cet exemple et recompilons le.
Maintenant, \main renvoie une valeur sur 64-bit:

\begin{lstlisting}[caption=\main renvoie une valeur de type \TT{uint64\_t} type,style=customc]
#include <stdio.h>
#include <stdint.h>

uint64_t main()
{
        printf ("Hello!\n");
        return 0;
}
\end{lstlisting}

%%The result is the same, but that's how \MOV at that line looks like now:
Le résultat est le même, mais voilà à quoi ressemble \MOV à cette ligne maintenant:

\begin{lstlisting}[caption=GCC 4.8.1 \NonOptimizing + objdump]
  4005a4:       d2800000        mov     x0, #0x0      // #0
\end{lstlisting}

\myindex{ARM!\Instructions!LDP}

\INS{LDP} (\IT{Load Pair}) remet les registres \RegX{29} et \RegX{30}.

Il n'y a pas de point d'exclamation après l'instruction: celui signifie que les valeurs sont
d'abord chargées depuis la pile, et ensuite \ac{SP} est incrémenté de 16.
Cela est appelé \IT{post-index}.

\myindex{ARM!\Instructions!RET}
Une nouvelle instruction est apparue en ARM64: \RET.
Elle fonctionne comme \TT{BX LR}, un \IT{hint} bit particulier est ajouté, qui informe le \ac{CPU}
qu'il s'agit d'un retour de fonction, et pas d'une autre instruction de saut, et il peut l'exécuter
de manière plus optimale. 

À cause de la simplicité de la fonction, GCC avec l'option d'optimisation génère le même code.


}
\RU{\subsection{ARM}
\label{sec:hw_ARM}

\myindex{\idevices}
\myindex{Raspberry Pi}
\myindex{Xcode}
\myindex{LLVM}
\myindex{Keil}
Для экспериментов с процессором ARM было использовано несколько компиляторов:

\begin{itemize}
\item Популярный в embedded-среде Keil Release 6/2013.

\item Apple Xcode 4.6.3 с компилятором LLVM-GCC 4.2
\footnote{Это действительно так: Apple Xcode 4.6.3 использует опен-сорсный GCC как компилятор переднего плана и кодогенератор LLVM}.

\item GCC 4.9 (Linaro) (для ARM64), 
доступный в виде исполняемого файла для win32 на \url{http://go.yurichev.com/17325}.

\end{itemize}

Везде в этой книге, если не указано иное, идет речь о 32-битном ARM (включая режимы Thumb и Thumb-2).
Когда речь идет о 64-битном ARM, он называется здесь ARM64.

% subsections
\subsubsection{\NonOptimizingKeilVI (\ARMMode)}

Для начала скомпилируем наш пример в Keil:

\begin{lstlisting}
armcc.exe --arm --c90 -O0 1.c 
\end{lstlisting}

\myindex{\IntelSyntax}
Компилятор \IT{armcc} генерирует листинг на ассемблере в формате Intel.
Этот листинг содержит некоторые высокоуровневые макросы, связанные с ARM
\footnote{например, он показывает инструкции \PUSH/\POP, отсутствующие в режиме ARM},
а нам важнее увидеть инструкции \q{как есть}, так что посмотрим скомпилированный результат в \IDA.

\begin{lstlisting}[caption=\NonOptimizingKeilVI (\ARMMode) \IDA,style=customasmARM]
.text:00000000             main
.text:00000000 10 40 2D E9    STMFD   SP!, {R4,LR}
.text:00000004 1E 0E 8F E2    ADR     R0, aHelloWorld ; "hello, world"
.text:00000008 15 19 00 EB    BL      __2printf
.text:0000000C 00 00 A0 E3    MOV     R0, #0
.text:00000010 10 80 BD E8    LDMFD   SP!, {R4,PC}

.text:000001EC 68 65 6C 6C+aHelloWorld  DCB "hello, world",0    ; DATA XREF: main+4
\end{lstlisting}

В вышеприведённом примере можно легко увидеть, что каждая инструкция имеет размер 4 байта.
Действительно, ведь мы же компилировали наш код для режима ARM, а не Thumb.

\myindex{ARM!\Instructions!STMFD}
\myindex{ARM!\Instructions!POP}
Самая первая инструкция, \INS{STMFD SP!, \{R4,LR\}}\footnote{\ac{STMFD}},
работает как инструкция \PUSH в x86, записывая значения двух регистров (\Reg{4} и \ac{LR}) в стек.
Действительно, в выдаваемом листинге на ассемблере компилятор \IT{armcc} для упрощения указывает здесь инструкцию
\INS{PUSH \{r4,lr\}}.
Но это не совсем точно, инструкция \PUSH доступна только в режиме Thumb, поэтому,
во избежание путаницы, я предложил работать в \IDA.

Итак, эта инструкция уменьшает \ac{SP}, чтобы он указывал на место в стеке, свободное для записи
новых значений, затем записывает значения регистров \Reg{4} и \ac{LR} 
по адресу в памяти, на который указывает измененный регистр \ac{SP}.

Эта инструкция, как и инструкция \PUSH в режиме Thumb, может сохранить в стеке одновременно несколько значений регистров, что может быть очень удобно.
Кстати, такого в x86 нет.
Также следует заметить, что \TT{STMFD}~--- генерализация инструкции \PUSH (то есть расширяет её возможности), потому что может работать с любым регистром, а не только с \ac{SP}.
Другими словами, \TT{STMFD} можно использовать для записи набора регистров в указанном месте памяти.

\myindex{\PICcode}
\myindex{ARM!\Instructions!ADR}
Инструкция \INS{ADR R0, aHelloWorld} прибавляет или отнимает значение регистра \ac{PC} к смещению, где хранится строка
\TT{hello, world}.
Причем здесь \ac{PC}, можно спросить? Притом, что это так называемый \q{\PICcode}
\footnote{Читайте больше об этом в соответствующем разделе ~(\myref{sec:PIC})}.
Он предназначен для исполнения будучи не привязанным к каким-либо адресам в памяти.
Другими словами, это относительная от \ac{PC} адресация.
В опкоде инструкции \TT{ADR} указывается разница между адресом этой инструкции и местом, где хранится строка.
Эта разница всегда будет постоянной, вне зависимости от того, куда был загружен \ac{OS} наш код.
Поэтому всё, что нужно~--- это прибавить адрес текущей инструкции (из \ac{PC}), чтобы получить текущий абсолютный адрес нашей Си-строки.

\myindex{ARM!\Registers!Link Register}
\myindex{ARM!\Instructions!BL}
Инструкция \INS{BL \_\_2printf}\footnote{Branch with Link} вызывает функцию \printf.
Работа этой инструкции состоит из двух фаз:

\begin{itemize}
\item записать адрес после инструкции \INS{BL} (\TT{0xC}) в регистр \ac{LR};
\item передать управление в \printf, записав адрес этой функции в регистр \ac{PC}.
\end{itemize}

Ведь когда функция \printf закончит работу, нужно знать, куда вернуть управление, поэтому закончив работу, всякая функция передает управление по адресу, записанному в регистре \ac{LR}.

В этом разница между \q{чистыми} \ac{RISC}-процессорами вроде ARM и \ac{CISC}-процессорами как x86,
где адрес возврата обычно записывается в стек ~(\myref{sec:stack}).

Кстати, 32-битный абсолютный адрес (либо смещение) невозможно закодировать в 32-битной инструкции \INS{BL}, в ней есть место только для 24-х бит.
Поскольку все инструкции в режиме ARM имеют длину 4 байта (32 бита) и инструкции могут находится только по адресам кратным 4, то последние 2 бита (всегда нулевых) можно не кодировать.
В итоге имеем 26 бит, при помощи которых можно закодировать $current\_PC \pm{} \approx{}32M$.

\myindex{ARM!\Instructions!MOV}
Следующая инструкция \INS{MOV R0, \#0}\footnote{Означает MOVe}
просто записывает 0 в регистр \Reg{0}.
Ведь наша Си-функция возвращает 0, а возвращаемое значение всякая функция оставляет в \Reg{0}.

\myindex{ARM!\Registers!Link Register}
\myindex{ARM!\Instructions!LDMFD}
\myindex{ARM!\Instructions!POP}
Последняя инструкция \INS{LDMFD SP!, {R4,PC}}\footnote{\ac{LDMFD}~--- это инструкция, обратная \ac{STMFD}}.
Она загружает из стека (или любого другого места в памяти) значения для сохранения их в \Reg{4} и \ac{PC}, увеличивая \glslink{stack pointer}{указатель стека} \ac{SP}.
Здесь она работает как аналог \POP.\\
N.B. Самая первая инструкция \TT{STMFD} сохранила в стеке \Reg{4} и \ac{LR}, а \IT{восстанавливаются} во время исполнения \TT{LDMFD} регистры \Reg{4} и \ac{PC}.

Как мы уже знаем, в регистре \ac{LR} обычно сохраняется адрес места, куда нужно всякой функции вернуть управление.
Самая первая инструкция сохраняет это значение в стеке, потому что наша функция \main позже будет сама пользоваться этим регистром в момент вызова \printf.
А затем, в конце функции, это значение можно сразу записать прямо в \ac{PC}, таким образом, передав управление туда, откуда была вызвана наша функция.

Так как функция \main обычно самая главная в \CCpp, управление будет возвращено в загрузчик \ac{OS}, либо куда-то в \ac{CRT} 
или что-то в этом роде.

Всё это позволяет избавиться от инструкции \INS{BX LR} в самом конце функции.

\myindex{ARM!DCB}
\TT{DCB}~--- директива ассемблера, описывающая массивы байт или ASCII-строк, аналог директивы DB в x86-ассемблере.


\subsubsection{\NonOptimizingKeilVI (\ThumbMode)}

Скомпилируем тот же пример в Keil для режима Thumb:

\begin{lstlisting}
armcc.exe --thumb --c90 -O0 1.c 
\end{lstlisting}

Получим (в \IDA):

\begin{lstlisting}[caption=\NonOptimizingKeilVI (\ThumbMode) + \IDA,style=customasmARM]
.text:00000000             main
.text:00000000 10 B5          PUSH    {R4,LR}
.text:00000002 C0 A0          ADR     R0, aHelloWorld ; "hello, world"
.text:00000004 06 F0 2E F9    BL      __2printf
.text:00000008 00 20          MOVS    R0, #0
.text:0000000A 10 BD          POP     {R4,PC}

.text:00000304 68 65 6C 6C+aHelloWorld  DCB "hello, world",0    ; DATA XREF: main+2
\end{lstlisting}

Сразу бросаются в глаза двухбайтные (16-битные) опкоды --- это, как уже было отмечено, Thumb.

\myindex{ARM!\Instructions!BL}
Кроме инструкции \TT{BL}.
Но на самом деле она состоит из двух 16-битных инструкций.
Это потому что в одном 16-битном опкоде слишком мало места для задания смещения, по которому находится функция \printf.
Так что первая 16-битная инструкция загружает старшие 10 бит смещения, а вторая~--- младшие 11 бит смещения.

% TODO:
% BL has space for 11 bits, so if we don't encode the lowest bit,
% then we should get 11 bits for the upper half, and 12 bits for the lower half.
% And the highest bit encodes the sign, so the destination has to be within
% \pm 4M of current_PC.
% This may be less if adding the lower half does not carry over,
% but I'm not sure --all my programs have 0 for the upper half,
% and don't carry over for the lower half.
% It would be interesting to check where __2printf is located relative to 0x8
% (I think the program counter is the next instruction on a multiple of 4
% for THUMB).
% The lower 11 bytes of the BL instructions and the even bit are
% 000 0000 0110 | 001 0010 1110 0 = 000 0000 0110 0010 0101 1100 = 0x00625c,
% so __2printf should be at 0x006264.
% But if we only have 10 and 11 bits, then the offset would be:
% 00 0000 0110 | 01 0010 1110 0 = 0 0000 0011 0010 0101 1100 = 0x00325c,
% so __2printf should be at 0x003264.
% In this case, though, the new program counter can only be 1M away,
% because of the highest bit is used for the sign.

Как уже было упомянуто, все инструкции в Thumb-режиме имеют длину 2 байта (или 16 бит).
Поэтому невозможна такая ситуация, когда Thumb-инструкция начинается по нечетному адресу.

Учитывая сказанное, последний бит адреса можно не кодировать.
Таким образом, в Thumb-инструкции \TT{BL} можно закодировать адрес $current\_PC \pm{}\approx{}2M$.

\myindex{ARM!\Instructions!PUSH}
\myindex{ARM!\Instructions!POP}
Остальные инструкции в функции (\PUSH и \POP) здесь работают почти так же, как и описанные \TT{STMFD}/\TT{LDMFD}, только регистр \ac{SP} здесь не указывается явно.
\INS{ADR} работает так же, как и в предыдущем примере.
\INS{MOVS} записывает 0 в регистр \Reg{0} для возврата нуля.


\subsubsection{\OptimizingXcodeIV (\ARMMode)}

Xcode 4.6.3 без включенной оптимизации выдает слишком много лишнего кода, поэтому включим оптимизацию компилятора (ключ \Othree), потому что там меньше инструкций.

\begin{lstlisting}[caption=\OptimizingXcodeIV (\ARMMode),style=customasm]
__text:000028C4             _hello_world
__text:000028C4 80 40 2D E9   STMFD           SP!, {R7,LR}
__text:000028C8 86 06 01 E3   MOV             R0, #0x1686
__text:000028CC 0D 70 A0 E1   MOV             R7, SP
__text:000028D0 00 00 40 E3   MOVT            R0, #0
__text:000028D4 00 00 8F E0   ADD             R0, PC, R0
__text:000028D8 C3 05 00 EB   BL              _puts
__text:000028DC 00 00 A0 E3   MOV             R0, #0
__text:000028E0 80 80 BD E8   LDMFD           SP!, {R7,PC}

__cstring:00003F62 48 65 6C 6C+aHelloWorld_0  DCB "Hello world!",0
\end{lstlisting}

Инструкции \TT{STMFD} и \TT{LDMFD} нам уже знакомы.

\myindex{ARM!\Instructions!MOV}
Инструкция \MOV просто записывает число \TT{0x1686} в регистр \Reg{0}~--- это смещение, указывающее на строку \q{Hello world!}.

Регистр \Reg{7} (по стандарту, принятому в \IOSABI) это frame pointer, о нем будет рассказано позже.

\myindex{ARM!\Instructions!MOVT}
Инструкция \TT{MOVT R0, \#0} (MOVe Top) записывает 0 в старшие 16 бит регистра.
Дело в том, что обычная инструкция \MOV в режиме ARM может записывать какое-либо значение только в младшие 16 бит регистра, ведь в ней нельзя закодировать больше.
Помните, что в режиме ARM опкоды всех инструкций ограничены длиной в 32 бита. Конечно, это ограничение не касается перемещений данных между регистрами.

Поэтому для записи в старшие биты (с 16-го по 31-й включительно) существует дополнительная команда \INS{MOVT}.
Впрочем, здесь её использование избыточно, потому что инструкция \INS{MOV R0, \#0x1686} выше и так обнулила старшую часть регистра.
Возможно, это недочет компилятора.
% TODO:
% I think, more specifically, the string is not put in the text section,
% ie. the compiler is actually not using position-independent code,
% as mentioned in the next paragraph.
% MOVT is used because the assembly code is generated before the relocation,
% so the location of the string is not yet known,
% and the high bits may still be needed.

\myindex{ARM!\Instructions!ADD}
Инструкция \TT{ADD R0, PC, R0} прибавляет \ac{PC} к \Reg{0} для вычисления действительного адреса строки \q{Hello world!}. Как нам уже известно, это \q{\PICcode}, поэтому такая корректива необходима.

Инструкция \TT{BL} вызывает \puts вместо \printf.

\label{puts}
\myindex{\CStandardLibrary!puts()}
\myindex{puts() вместо printf()}
Компилятор заменил вызов \printf на \puts. 
Действительно, \printf с одним аргументом это почти аналог \puts.
 
\IT{Почти}, если принять условие, что в строке не будет управляющих символов \printf, 
начинающихся со знака процента. Тогда эффект от работы этих двух функций будет разным
\footnote{Также нужно заметить, что \puts не требует символа перевода строки `\textbackslash{}n' в конце строки,
поэтому его здесь нет.}.

Зачем компилятор заменил один вызов на другой? Наверное потому что \puts работает быстрее
\footnote{\href{http://go.yurichev.com/17063}{ciselant.de/projects/gcc\_printf/gcc\_printf.html}}. 
Видимо потому что \puts проталкивает символы в \gls{stdout} не сравнивая каждый со знаком процента.

Далее уже знакомая инструкция \TT{MOV R0, \#0}, служащая для установки в 0 возвращаемого значения функции.


\subsubsection{\OptimizingXcodeIV (\ThumbTwoMode)}

По умолчанию Xcode 4.6.3 генерирует код для режима Thumb-2 примерно в такой манере:

\begin{lstlisting}[caption=\OptimizingXcodeIV (\ThumbTwoMode)]
__text:00002B6C                   _hello_world
__text:00002B6C 80 B5          PUSH            {R7,LR}
__text:00002B6E 41 F2 D8 30    MOVW            R0, #0x13D8
__text:00002B72 6F 46          MOV             R7, SP
__text:00002B74 C0 F2 00 00    MOVT.W          R0, #0
__text:00002B78 78 44          ADD             R0, PC
__text:00002B7A 01 F0 38 EA    BLX             _puts
__text:00002B7E 00 20          MOVS            R0, #0
__text:00002B80 80 BD          POP             {R7,PC}

...

__cstring:00003E70 48 65 6C 6C 6F 20+aHelloWorld  DCB "Hello world!",0xA,0
\end{lstlisting}

% Q: If you subtract 0x13D8 from 0x3E70,
% you actually get a location that is not in this function, or in _puts.
% How is PC-relative addressing done in THUMB2?
% A: it's not Thumb-related. there are just mess with two different segments. TODO: rework this listing.

\myindex{\ThumbTwoMode}
\myindex{ARM!\Instructions!BL}
\myindex{ARM!\Instructions!BLX}
Инструкции \TT{BL} и \TT{BLX} в Thumb, как мы помним, кодируются как пара 16-битных инструкций, 
а в Thumb-2 эти \IT{суррогатные} опкоды расширены так, что новые инструкции кодируются здесь как 
32-битные инструкции.
Это можно заметить по тому что опкоды Thumb-2 инструкций всегда начинаются с \TT{0xFx} либо с \TT{0xEx}.
Но в листинге \IDA байты опкода переставлены местами.
Это из-за того, что в процессоре ARM инструкции кодируются так:
в начале последний байт, потом первый (для Thumb и Thumb-2 режима), либо, 
(для инструкций в режиме ARM) в начале четвертый байт, затем третий, второй и первый 
(т.е. другой \gls{endianness}).

Вот так байты следуют в листингах IDA:

\begin{itemize}
\item для режимов ARM и ARM64: 4-3-2-1;
\item для режима Thumb: 2-1;
\item для пары 16-битных инструкций в режиме Thumb-2: 2-1-4-3.
\end{itemize}

\myindex{ARM!\Instructions!MOVW}
\myindex{ARM!\Instructions!MOVT.W}
\myindex{ARM!\Instructions!BLX}
Так что мы видим здесь что инструкции \TT{MOVW}, \TT{MOVT.W} и \TT{BLX} начинаются с \TT{0xFx}.

Одна из Thumb-2 инструкций это
\TT{MOVW R0, \#0x13D8}~--- она записывает 16-битное число в младшую часть регистра \Reg{0}, очищая старшие биты.

Ещё \TT{MOVT.W R0, \#0}~--- эта инструкция работает так же, как и \TT{MOVT} из предыдущего примера, но она работает в Thumb-2.

\myindex{ARM!переключение режимов}
\myindex{ARM!\Instructions!BLX}
Помимо прочих отличий, здесь используется инструкция \TT{BLX} вместо \TT{BL}.
Отличие в том, что помимо сохранения адреса возврата в регистре \ac{LR} и передаче управления 
в функцию \puts, происходит смена режима процессора с Thumb/Thumb-2 на режим ARM (либо назад).
Здесь это нужно потому, что инструкция, куда ведет переход, выглядит так (она закодирована в режиме ARM):

\begin{lstlisting}[style=customasm]
__symbolstub1:00003FEC _puts           ; CODE XREF: _hello_world+E
__symbolstub1:00003FEC 44 F0 9F E5     LDR  PC, =__imp__puts
\end{lstlisting}

Это просто переход на место, где записан адрес \puts в секции импортов.
Итак, внимательный читатель может задать справедливый вопрос: почему бы не вызывать \puts сразу в 
том же месте кода, где он нужен?
Но это не очень выгодно из-за экономии места и вот почему.

\myindex{Динамически подгружаемые библиотеки}
Практически любая программа использует внешние динамические библиотеки (будь то DLL в Windows, .so в *NIX 
либо .dylib в \MacOSX).
В динамических библиотеках находятся часто используемые библиотечные функции, в том числе стандартная функция Си \puts.

\myindex{Relocation}
В исполняемом бинарном файле 
(Windows PE .exe, ELF либо Mach-O) имеется секция импортов, список символов (функций либо глобальных переменных) импортируемых из внешних модулей, а также названия самих модулей.
Загрузчик \ac{OS} загружает необходимые модули и, перебирая импортируемые символы в основном модуле, проставляет правильные адреса каждого символа.
В нашем случае, \IT{\_\_imp\_\_puts} 
это 32-битная переменная, куда загрузчик \ac{OS} запишет правильный адрес этой же функции во внешней библиотеке. 
Так что инструкция \TT{LDR} просто берет 32-битное значение из этой переменной, и, записывая его в регистр \ac{PC}, просто передает туда управление.
Чтобы уменьшить время работы загрузчика \ac{OS}, нужно чтобы ему пришлось записать адрес каждого символа только один раз, в соответствующее, выделенное для них, место.

\myindex{thunk-функции}
К тому же, как мы уже убедились, нельзя одной инструкцией загрузить в регистр 32-битное число без обращений к памяти.
Так что наиболее оптимально выделить отдельную функцию, работающую в режиме ARM, 
чья единственная цель~--- передавать управление дальше, в динамическую библиотеку.
И затем ссылаться на эту короткую функцию из одной инструкции (так называемую \glslink{thunk function}{thunk-функцию}) из Thumb-кода.

\myindex{ARM!\Instructions!BL}
Кстати, в предыдущем примере (скомпилированном для режима ARM), переход при помощи инструкции \TT{BL} ведет 
на такую же \glslink{thunk function}{thunk-функцию}, однако режим процессора не переключается (отсюда отсутствие \q{X} в мнемонике инструкции).

\myparagraph{Еще о thunk-функциях}
\myindex{thunk-функции}

Thunk-функции трудновато понять, скорее всего, из-за путаницы в терминах.
Проще всего представлять их как адаптеры-переходники из одного типа разъемов в другой.
Например, адаптер позволяющее вставить в американскую розетку британскую вилку, или наоборот.
Thunk-функции также иногда называются \IT{wrapper-ами}. \IT{Wrap} в английском языке это \IT{обертывать}, \IT{завертывать}.
Вот еще несколько описаний этих функций:

\begin{framed}
\begin{quotation}
“A piece of coding which provides an address:”, according to P. Z. Ingerman, 
who invented thunks in 1961 as a way of binding actual parameters to their formal 
definitions in Algol-60 procedure calls. If a procedure is called with an expression 
in the place of a formal parameter, the compiler generates a thunk which computes 
the expression and leaves the address of the result in some standard location.

\dots

Microsoft and IBM have both defined, in their Intel-based systems, a “16-bit environment” 
(with bletcherous segment registers and 64K address limits) and a “32-bit environment” 
(with flat addressing and semi-real memory management). The two environments can both be 
running on the same computer and OS (thanks to what is called, in the Microsoft world, 
WOW which stands for Windows On Windows). MS and IBM have both decided that the process 
of getting from 16- to 32-bit and vice versa is called a “thunk”; for Windows 95, 
there is even a tool, THUNK.EXE, called a “thunk compiler”.
\end{quotation}
\end{framed}
% TODO FIXME move to bibliography and quote properly above the quote
( \href{http://go.yurichev.com/17362}{The Jargon File} )

\myindex{LAPACK}
\myindex{FORTRAN}
Еще один пример мы можем найти в библиотеке LAPACK --- (``Linear Algebra PACKage'') написанная на FORTRAN.
Разработчики на \CCpp также хотят использовать LAPACK, но переписывать её на \CCpp, а затем поддерживать несколько версий,
это безумие.
Так что имеются короткие функции на Си вызываемые из \CCpp{}-среды, которые, в свою очередь, вызывают функции на FORTRAN,
и почти ничего больше не делают:

\begin{lstlisting}
double Blas_Dot_Prod(const LaVectorDouble &dx, const LaVectorDouble &dy)
{
    assert(dx.size()==dy.size());
    integer n = dx.size();
    integer incx = dx.inc(), incy = dy.inc();

    return F77NAME(ddot)(&n, &dx(0), &incx, &dy(0), &incy);
}
\end{lstlisting}

Такие ф-ции еще называют ``wrappers'' (т.е., ``обертка'').


\subsubsection{ARM64}

\myparagraph{GCC}

Компилируем пример в GCC 4.8.1 для ARM64:

\lstinputlisting[numbers=left,label=hw_ARM64_GCC,caption=\NonOptimizing GCC 4.8.1 + objdump,style=customasmARM]{patterns/01_helloworld/ARM/hw.lst}

В ARM64 нет режима Thumb и Thumb-2, только ARM, так что тут только 32-битные инструкции.

Регистров тут в 2 раза больше: \myref{ARM64_GPRs}.
64-битные регистры теперь имеют префикс 
\TT{X-}, а их 32-битные части --- \TT{W-}.

\myindex{ARM!\Instructions!STP}
Инструкция \TT{STP} (\IT{Store Pair}) 
сохраняет в стеке сразу два регистра: \RegX{29} и \RegX{30}.
Конечно, эта инструкция может сохранять эту пару где угодно в памяти, но здесь указан регистр \ac{SP}, так что
пара сохраняется именно в стеке.

Регистры в ARM64 64-битные, каждый имеет длину в 8 байт, так что для хранения двух регистров нужно именно 16 байт.

Восклицательный знак (``!'') после операнда означает, что сначала от \ac{SP} будет отнято 16 и только затем
значения из пары регистров будут записаны в стек.

Это называется \IT{pre-index}.
Больше о разнице между \IT{post-index} и \IT{pre-index} 
описано здесь: \myref{ARM_postindex_vs_preindex}.

Таким образом, в терминах более знакомого всем процессора x86, первая инструкция~--- это просто аналог 
пары инструкций \TT{PUSH X29} и \TT{PUSH X30}.
\RegX{29} в ARM64 используется как \ac{FP}, а \RegX{30} 
как \ac{LR}, поэтому они сохраняются в прологе функции и
восстанавливаются в эпилоге.

Вторая инструкция копирует \ac{SP} в \RegX{29} (или \ac{FP}).
Это нужно для установки стекового фрейма функции.

\label{pointers_ADRP_and_ADD}
\myindex{ARM!\Instructions!ADRP/ADD pair}
Инструкции \TT{ADRP} и \ADD нужны для формирования адреса строки \q{Hello!} в регистре \RegX{0}, 
ведь первый аргумент функции передается через этот регистр.
Но в ARM нет инструкций, при помощи которых можно записать в регистр длинное число 
(потому что сама длина инструкции ограничена 4-я байтами. Больше об этом здесь: \myref{ARM_big_constants_loading}).
Так что нужно использовать несколько инструкций.
Первая инструкция (\TT{ADRP}) записывает в \RegX{0} адрес 4-килобайтной страницы где находится строка, 
а вторая (\ADD) просто прибавляет к этому адресу остаток.
Читайте больше об этом: \myref{ARM64_relocs}.

\TT{0x400000 + 0x648 = 0x400648}, и мы видим, что в секции данных \TT{.rodata} по этому адресу как раз находится наша
Си-строка \q{Hello!}.

\myindex{ARM!\Instructions!BL}
Затем при помощи инструкции \TT{BL} вызывается \puts. Это уже рассматривалось ранее: \myref{puts}.

Инструкция \MOV записывает 0 в \RegW{0}. 
\RegW{0} это младшие 32 бита 64-битного регистра \RegX{0}:

\input{ARM_X0_register}

А результат функции возвращается через \RegX{0}, и \main возвращает 0, 
так что вот так готовится возвращаемый результат.

Почему именно 32-битная часть?
Потому что в ARM64, как и в x86-64, тип \Tint оставили 32-битным, для лучшей совместимости.

Следовательно, раз уж функция возвращает 32-битный \Tint, то нужно заполнить только 32 младших бита регистра \RegX{0}.

Для того, чтобы удостовериться в этом, немного отредактируем этот пример и перекомпилируем его.%

Теперь \main возвращает 64-битное значение:

\begin{lstlisting}[caption=\main возвращающая значение типа \TT{uint64\_t},style=customc]
#include <stdio.h>
#include <stdint.h>

uint64_t main()
{
        printf ("Hello!\n");
        return 0;
}
\end{lstlisting}

Результат точно такой же, только \MOV в той строке теперь выглядит так:

\begin{lstlisting}[caption=\NonOptimizing GCC 4.8.1 + objdump]
  4005a4:       d2800000        mov     x0, #0x0      // #0
\end{lstlisting}

\myindex{ARM!\Instructions!LDP}
Далее при помощи инструкции \INS{LDP} (\IT{Load Pair}) восстанавливаются регистры \RegX{29} и \RegX{30}.

Восклицательного знака после инструкции нет. Это означает, что сначала значения достаются из стека, и только потом \ac{SP} увеличивается на 16.

Это называется \IT{post-index}.

\myindex{ARM!\Instructions!RET}
В ARM64 есть новая инструкция: \RET. 
Она работает так же как и \INS{BX LR}, но там добавлен специальный бит,
подсказывающий процессору, что это именно выход из функции, а не просто переход, чтобы процессор
мог более оптимально исполнять эту инструкцию.

Из-за простоты этой функции оптимизирующий GCC генерирует точно такой же код.



}
\ITA{\subsection{MIPS}

\subsubsection{3 argomenti}

\myparagraph{\Optimizing GCC 4.4.5}

La differenza principale con l'esempio \q{\HelloWorldSectionName} e' che in questo caso \printf e' chiamata 
al posto di \puts, e 3 argomenti aggiuntivi sono passati attraverso i registri \$5\dots \$7 (o \$A0\dots \$A2).
Questo e' il motivo per cui questi registri hanno il prefisso A-, che implica il loro uso per il passaggio di argomenti di funzioni.

% TODO translate to Italian:
\lstinputlisting[caption=\Optimizing GCC 4.4.5 (\assemblyOutput),style=customasm]{patterns/03_printf/MIPS/printf3.O3_EN.s}

% TODO translate to Italian:
\lstinputlisting[caption=\Optimizing GCC 4.4.5 (IDA)]{patterns/03_printf/MIPS/printf3.O3.IDA_EN.lst}

\IDA ha fuso le coppie di istruzioni \INS{LUI} e \INS{ADDIU} in una unica pseudoistruzione \INS{LA}.
Questo e' il motivo per cui non c'e' nessuna istruzione all'indirizzo 0x1C: perche' \INS{LA} \IT{occupa} 8 byte.%

\myparagraph{\NonOptimizing GCC 4.4.5}

\NonOptimizing GCC e' piu' verboso:

% TODO translate to Italian:
\lstinputlisting[caption=\NonOptimizing GCC 4.4.5 (\assemblyOutput),style=customasm]{patterns/03_printf/MIPS/printf3.O0_EN.s}

% TODO translate to Italian:
\lstinputlisting[caption=\NonOptimizing GCC 4.4.5 (IDA)]{patterns/03_printf/MIPS/printf3.O0.IDA_EN.lst}

\subsubsection{8 argomenti}

Usiamo nuovamente l'esempio con 9 argomenti dalla sezione prcedente: \myref{example_printf8_x64}.

\lstinputlisting[style=customc]{patterns/03_printf/2.c}

\myparagraph{\Optimizing GCC 4.4.5}

Solo i primi 4 argomenti sono passati nei registri \$A0 \dots \$A3, gli altri sono passati tramite lo stack.
\myindex{MIPS!O32}

Questa e' la calling convention O32 (che e' la piu' comune nel mondo MIPS).
Altre calling conventions (come N32) possono usare i registri per scopi diversi.

\myindex{MIPS!\Instructions!SW}

\INS{SW} e' l'abbreviazione di \q{Store Word} (da un registro alla memoria).
MIPS manca di istruzioni per memorizzare un valore in memoria, e' quindi necessario usare una coppia di istruzioni (LI/SW).

% TODO translate to Italian:
\lstinputlisting[caption=\Optimizing GCC 4.4.5 (\assemblyOutput),style=customasm]{patterns/03_printf/MIPS/printf8.O3_EN.s}

% TODO translate to Italian:
\lstinputlisting[caption=\Optimizing GCC 4.4.5 (IDA)]{patterns/03_printf/MIPS/printf8.O3.IDA_EN.lst}

\myparagraph{\NonOptimizing GCC 4.4.5}

\NonOptimizing GCC e' piu' verboso:

% TODO translate to Italian:
\lstinputlisting[caption=\NonOptimizing GCC 4.4.5 (\assemblyOutput),style=customasm]{patterns/03_printf/MIPS/printf8.O0_EN.s}

% TODO translate to Italian:
\lstinputlisting[caption=\NonOptimizing GCC 4.4.5 (IDA)]{patterns/03_printf/MIPS/printf8.O0.IDA_EN.lst}
}
\DE{\subsection{ARM}
\label{sec:hw_ARM}

\myindex{\idevices}
\myindex{Raspberry Pi}
\myindex{Xcode}
\myindex{LLVM}
\myindex{Keil}
Für die Experimente mit ARM-Prozessoren wurden verschiedene Compiler genutzt:

\begin{itemize}
\item Verbreitet im Embedded-Bereich: Keil Release 6/2013.

\item Apple Xcode 4.6.3 IDE mit dem LLVM-GCC 4.2-Compiler
\footnote{Tatsächlich nutzt Apple Xcode 4.6.3 GCC als Front-End-Compiler und LLVM 
Code Generator}.

\item GCC 4.9 (Linaro) (für ARM64), verfügbar als Win32-Executable unter \url{http://go.yurichev.com/17325}.

\end{itemize}
Wenn nicht anders angegeben wird immer der 32-Bit ARM-Code (inklusive Thumb und Thumb-2-Mode) genutzt.
Wenn von 64-Bit ARM die Rede ist, dann wird ARM64 geschrieben.

% subsections
\subsubsection{\NonOptimizingKeilVI (\ARMMode)}

Beginnen wir mit dem Kompilieren des Beispiels mit Keil:

\begin{lstlisting}
armcc.exe --arm --c90 -O0 1.c 
\end{lstlisting}

\myindex{\IntelSyntax}
Der \IT{armcc}-Compiler erstellt Assembler-Quelltext im Intel-Syntax, hat aber High-Level-Makros
bezüglich der ARM-Prozessoren\footnote{d.h. der ARM-Mode hat keine \PUSH/\POP-Anweisungen}.
Es ist hier wichtig die \q{richtigen} Anweisungen zu sehen, deswegen ist hier das Ergebnis mit
\IDA kompiliert.

\begin{lstlisting}[caption=\NonOptimizingKeilVI (\ARMMode) \IDA,style=customasm]
.text:00000000             main
.text:00000000 10 40 2D E9    STMFD   SP!, {R4,LR}
.text:00000004 1E 0E 8F E2    ADR     R0, aHelloWorld ; "hello, world"
.text:00000008 15 19 00 EB    BL      __2printf
.text:0000000C 00 00 A0 E3    MOV     R0, #0
.text:00000010 10 80 BD E8    LDMFD   SP!, {R4,PC}

.text:000001EC 68 65 6C 6C+aHelloWorld  DCB "hello, world",0    ; DATA XREF: main+4
\end{lstlisting}

Im ersten Beispiel ist zu erkennen, dass jede Anweisung 4 Byte groß ist.
Tatsächlich wurde der Code für den ARM- und nicht den Thumb-Mode erstellt.

\myindex{ARM!\Instructions!STMFD}
\myindex{ARM!\Instructions!POP}
Die erste Anweisung, \INS{STMFD SP!, \{R4,LR\}}\footnote{\ac{STMFD}}, arbeitet wie eine x86-\PUSH-Anweisung
um die Werte der beiden Register (\Reg{4} and \ac{LR}) auf den Stack zu legen.

Die Ausgabe des \IT{armcc}-Compilers, zeigt, aus Gründen der Einfachheit, die \INS{PUSH \{r4,lr\}}-Anweisung.
Dies ist nicht vollständig präzise. Die \PUSH-Anweisung ist nur im Thumb-Mode verfügbar.
Um die Dinge nicht zu verwirrend zu machen, wird der Code in \IDA kompiliert.

Die Anweisung dekrementiert zunächst den \ac{SP}, so dass er auf den Bereich im Stack zeigt, der
für neue Einträge frei ist. Anschließend werden die Werte der Register \Reg{4} und \ac{LR} an der Adresse
gespeichert auf den der (modifizierte) \ac{SP} zeigt.

Diese Anweisungen (wie \PUSH im Thumb-Mode) ist in der Lage mehrere Register-Werte auf einmal zu speichern,
was sehr nützlich sein kann. Übrigens: in x86 gibt es dazu kein Äquivalent.
Außerdem ist erwähnenswert, dass die \TT{STMFD}-Anweisung eine Generalisierung der \PUSH-Anweisung
(ohne deren Eigenschaften) ist, weil sie auf jedes Register angewandt werden kann und nicht nur auf \ac{SP}.
Mit anderen Worten kann \TT{STMFD} genutzt werden um eine Reihen von Registern an einer angegebenen
Speicher-Adresse zu sichern.

\myindex{\PICcode}
\myindex{ARM!\Instructions!ADR}
Die \INS{ADR R0, aHelloWorld}-Anweisung addiert oder subtrahiert den Wert im \ac{PC}-Register zum Offset
an dem die \TT{hello, world}-Zeichenkette ist.
Man kann sich nun fragen, wie das \TT{PC}-Register hier genutzt wird.
Dies wird \q{\PICcode}\footnote{mehr darüber in der entsprechenden Sektion~(\myref{sec:PIC})} genannt.

Code dieser Art kann an nicht-festen Adressen im Speicher ausgeführt werden.
Mit anderen Worten: dies ist \ac{PC}-relative Adressierung.
Die \INS{ADR}-Anweisung berücksichtigt den Unterschied zwischen der Adresse dieser Anweisung und der Adresse
an dem die Zeichenkette gespeichert ist.
Der Unterschied (Offset) ist immer gleich, egal an welcher Adresse der Code vom \ac{OS} geladen wurden.
Dementsprechend ist alles was gemacht werden muss, die Adresse der aktuellen Anweisung (vom \ac{PC})
zu addieren um die absolute Speicheradresse der Zeichenkette zu bekommen.

\myindex{ARM!\Registers!Link Register}
\myindex{ARM!\Instructions!BL}
\INS{BL \_\_2printf}\footnote{Branch with Link}-Anweisung ruft die \printf-Funktion auf. 
Die Anweisung funktioniert wie folgt:

\begin{itemize}
\item Speichere die Adresse hinter der \INS{BL}-Anweisung (\TT{0xC}) in \ac{LR};
\item anschließend wird übergebe die Kontrolle an \printf indem dessen Adresse ins \ac{PC}-Register geschrieben wird.
\end{itemize}

Wenn \printf die Ausführung beendet, müssen Informationen vorliegen, wo die Ausführung weitergehen soll.
Das ist der Grund warum jede Funktion die Kontrolle an die Adresse, gespeichert im \ac{LR}-Register übergibt.

Dies ist ein Unterschied zwischen einem \q{reinem} \ac{RISC}-Prozessor wie ARM und \ac{CISC}-Prozessoren wie x86,
bei denen die Rücksprungadresse in der Regel auf dem Stack gespeichert wird.
Mehr dazu ist im nächsten Abschnitt zu lesen~(\myref{sec:stack}).

Übrigens eine absolute 32-Bit-Adresse oder -Offset kann nicht in einer 32-Bit-\TT{BL}-Anweisung kodiert werden,
weil diese nur für 24 Bit Platz bietet. Wie bereits erwähnt haben alle ARM-Mode-Anweisungen eine Größe
von 4 Byte (32 Bit). Aus diesem Grund können diese nur an 4-Byte-Grenzen des Speichers platziert werden.
Dies heißt auch, das die letzten zwei Bit der Anweisungsadresse (die immer Null sind) entfallen können.
Zusammenfassend, stehen 26 Bit für die Offset-Kodierung zur Verfügung. Dies ist genug für
$current\_PC \pm{} \approx{}32M$.

\myindex{ARM!\Instructions!MOV}
Als nächstes schreibt die Anweisung \INS{MOV R0, \#0}\footnote{das heißt MOVe} lediglich 0 in
das \Reg{0}-Register weil der Rückgabewert hier gespeichert wird und die gezeigte C-Funktion 0
als Argument für die return-Anweisung hat.

\myindex{ARM!\Registers!Link Register}
\myindex{ARM!\Instructions!LDMFD}
\myindex{ARM!\Instructions!POP}
Die letzte Anweisung \INS{LDMFD SP!, {R4,PC}}\footnote{\ac{LDMFD} ist eine inverse Anweisung von \ac{STMFD}}
lädt die Werte nacheinander vom Stack (oder eine andere Speicheradresse) um sie in die Register \Reg{4} und \ac{PC}
zu sichern. Außerdem wird der Stack Pointer \ac{SP} inkrementiert. Hier arbeitet der Befehl wie \POP.

Die erste Anweisung \TT{STMFD} sichert das Register-Paar \Reg{4} und \ac{LR} auf dem Stack, jedoch werden \Reg{4} und \ac{PC}
während der Ausführung von \TT{LDMFD} \IT{wiederhergestellt}.

Wie bereits bekannt, wird die Adresse die nach der Ausführung einer Funktion angesprungen wird in dem \ac{LR}-Register gesichert.
Die allererste Anweisung sichert diese Wert auf dem Stack weil das gleiche Register von der \main-Funktion genutzt wird,
wenn \printf aufgerufen wird.
Am Ende der Funktion kann dieser Wert direkt in das \ac{PC}-Register geschrieben werden und so die Ausführung an der
Stelle fortgesetzt werden an der die Funktion aufgerufen wurde.

Da \main in der Regel die erste Funktion in \CCpp ist, wird die Kontrolle an das \ac{OS} oder einen Punkt in der
\ac{CRT} übergeben.

All dies erlaubt das Auslassen der \INS{BX LR}-Anweisung am Ende der Funktion.

\myindex{ARM!DCB}
\TT{DCB} ist eine Assemblerdirektive die ein Array von Bytes oder ASCII anlegt, ähnlich der DB-Direktive
in der x86-Assembler-Sprache.

\subsubsection{\NonOptimizingKeilVI (\ThumbMode)}

Nachfolgend das gleiche Beispiel mit dem Keil-Compiler im Thumb-Mode erstellt:

\begin{lstlisting}
armcc.exe --thumb --c90 -O0 1.c 
\end{lstlisting}

In \IDA wird folgende Ausgabe erzeugt:

\begin{lstlisting}[caption=\NonOptimizingKeilVI (\ThumbMode) + \IDA,style=customasmARM]
.text:00000000             main
.text:00000000 10 B5          PUSH    {R4,LR}
.text:00000002 C0 A0          ADR     R0, aHelloWorld ; "hello, world"
.text:00000004 06 F0 2E F9    BL      __2printf
.text:00000008 00 20          MOVS    R0, #0
.text:0000000A 10 BD          POP     {R4,PC}

.text:00000304 68 65 6C 6C+aHelloWorld  DCB "hello, world",0    ; DATA XREF: main+2
\end{lstlisting}

Leicht zu erkennen sind die 2-Byte (16 Bit) OpCodes, die wie bereits erwähnt Thumb-Anweisungen sind.
\myindex{ARM!\Instructions!BL}
Die \TT{BL}-Anweisung besteht aus zwei 16-Bit-Anweisungen, weil es für die \printf-Funktion unmöglich ist
einen Offset zu laden, wenn der kleine Speicherbereich in einem 16-Bit-Opcode genutzt wird.
Aus diesem Grund lädt die erste 16-Bit-Anweisung die höherwertigen 10 Bit des Offsets und die zweite
Anweisung die niederwertigen 11 Bit.

% TODO:
% BL has space for 11 bits, so if we don't encode the lowest bit,
% then we should get 11 bits for the upper half, and 12 bits for the lower half.
% And the highest bit encodes the sign, so the destination has to be within
% \pm 4M of current_PC.
% This may be less if adding the lower half does not carry over,
% but I'm not sure --all my programs have 0 for the upper half,
% and don't carry over for the lower half.
% It would be interesting to check where __2printf is located relative to 0x8
% (I think the program counter is the next instruction on a multiple of 4
% for THUMB).
% The lower 11 bytes of the BL instructions and the even bit are
% 000 0000 0110 | 001 0010 1110 0 = 000 0000 0110 0010 0101 1100 = 0x00625c,
% so __2printf should be at 0x006264.
% But if we only have 10 and 11 bits, then the offset would be:
% 00 0000 0110 | 01 0010 1110 0 = 0 0000 0011 0010 0101 1100 = 0x00325c,
% so __2printf should be at 0x003264.
% In this case, though, the new program counter can only be 1M away,
% because of the highest bit is used for the sign.

Wie erwähnt haben alle Anweisungen im Thumb-Mode eine Größe von 2 Byte (16 Bit).
Dies bedeutet, dass es unmöglich ist an einer ungeraden Adresse einen Anweisung unterzubringen.
Das hat auch zur Folge, dass das letzte Bit der Adresse bei der Kodierung der
Anweisungen weggelassen werden kann.

Zusammenfassend kann die \TT{BL}-Thumb-Anweisung eine Adresse bis $current\_PC \pm{}\approx{}2M$ kodieren.

\myindex{ARM!\Instructions!PUSH}
\myindex{ARM!\Instructions!POP}
Wie für die anderen Anweisungen in dieser Funktion arbeiten \PUSH und \POP wie die beschriebenden \TT{STMFD}/\TT{LDMFD},
nur dass das \ac{SP}-Register hier nicht explizit genannt wird.
\TT{ADR} arbeitet genau wie in dem vorherigen Beispiel.
\TT{MOVS} schreibt 0 in das Register \Reg{0} um 0 zurückzugeben.

\subsubsection{\OptimizingXcodeIV (\ARMMode)}

Xcode 4.6.3 ohne Optimierung produziert eine Menge redundanten Code, so dass im Folgenden die
optimierte Ausgabe gelistet ist bei der die Anzahl der Anweisungen so klein wie möglich ist.
Der Compiler-Schalter ist \Othree.

\begin{lstlisting}[caption=\OptimizingXcodeIV (\ARMMode),style=customasmARM]
__text:000028C4             _hello_world
__text:000028C4 80 40 2D E9   STMFD           SP!, {R7,LR}
__text:000028C8 86 06 01 E3   MOV             R0, #0x1686
__text:000028CC 0D 70 A0 E1   MOV             R7, SP
__text:000028D0 00 00 40 E3   MOVT            R0, #0
__text:000028D4 00 00 8F E0   ADD             R0, PC, R0
__text:000028D8 C3 05 00 EB   BL              _puts
__text:000028DC 00 00 A0 E3   MOV             R0, #0
__text:000028E0 80 80 BD E8   LDMFD           SP!, {R7,PC}

__cstring:00003F62 48 65 6C 6C+aHelloWorld_0  DCB "Hello world!",0
\end{lstlisting}

Die Anweisungen \TT{STMFD} und \TT{LDMFD} sind bereits bekannt.

\myindex{ARM!\Instructions!MOV}

Die \MOV-Anweisung schreibt lediglich die Nummer \TT{0x1686} in das Register \Reg{0}.
Dies ist der Offset der auf die Zeichenkette \q{Hello world!} zeigt.

Das Register \TT{R7} (spezifiziert in \IOSABI) ist ein Frame Pointer. Mehr darüber folgt später.

\myindex{ARM!\Instructions!MOVT}
Die \TT{MOVT R0, \#0} (MOVe Top)-Anweisung schreibt 0 in die höherwertigen 16 Bit des Registers.
Das Problem ist hier, dass die generische \MOV-Anweisung im ARM-Mode nur die niederwertigen 16 Bit
des Registers beschreibt.

Dran denken: alle Opcodes im ARM-Mode sind in der Größe auf 32 Bit begrenzt. Natürlich gilt diese
Begrenzung nicht für das Verschieben von Daten zwischen Registern.
Aus diesem Grund existiert die zusätzliche Anweisung  \TT{MOVT} um in die höherwertigen Bits
(von 16 bis einschließlich 31) zu beschreiben.
Die Benutzung ist in diesem Fall redundant, weil die Anweisung \TT{MOV R0, \#0x1686} darüpber
den höherwertigen Teil des Registers zurückgesetzt hat.
Dies ist vermutlich ein Mangel des Compilers.

% TODO:
% I think, more specifically, the string is not put in the text section,
% ie. the compiler is actually not using position-independent code,
% as mentioned in the next paragraph.
% MOVT is used because the assembly code is generated before the relocation,
% so the location of the string is not yet known,
% and the high bits may still be needed.

\myindex{ARM!\Instructions!ADD}
Die Anweisung \TT{ADD R0, PC, R0} addiert den Wert im \ac{PC} zum Wert im Register \Reg{0}
um die absolute Adresse der \q{Hello world!}-Zeichenkette zu berechnen.
Wie bereits bekannt ist dies \q{\PICcode}, so dass diese Korrektur hier unbedingt notwendig ist.

Die \INS{BL}-Anweisung ruft \puts anstatt \printf auf.

\label{puts}
\myindex{\CStandardLibrary!puts()}
\myindex{puts() anstatt printf()}

GCC ersetzt den ersten \printf-Aufruf mit \puts. In der Tat ist \printf mit nur einem
Argument identisch mit \puts.

Die beiden Funktionen produzieren lediglich das gleiche Ergebnis, weil printf keine
Formatkennzeichner, beginnend mit \IT{\%}, enhält.
Sollte dies jedoch der Fall sein, wäre die Auswirkung der beiden Funktionen
unterschiedlich\footnote{Des weiteren benötigt \puts kein '\textbackslash{}n'
für den Zeilenumbruch am Ende der Zeichenkette, weswegen wir dies hier nicht sehen.}.

Warum hat der Compiler diese Ersetzung durchgeführt? Vermutlich hat dies Vorteile bei
der Geschwindigkeit, weil \puts schneller ist
\footnote{\href{http://go.yurichev.com/17063}{ciselant.de/projects/gcc\_printf/gcc\_printf.html}}
und lediglich die Zeichen zu \gls{stdout} übergibt, anstatt jedes Zeichen mit \IT{\%} zu vergleichen.

Als nächstes ist die bekannte Anweisung \TT{MOV R0, \#0} zu sehen um das Register \Reg{0} auf 0 zu setzen.

\subsubsection{\OptimizingXcodeIV (\ThumbTwoMode)}

Standardmäßig generiert Xcode 4.6.3 den Thumb-2-Code auf folgende Weise:

\begin{lstlisting}[caption=\OptimizingXcodeIV (\ThumbTwoMode),style=customasm]
__text:00002B6C                   _hello_world
__text:00002B6C 80 B5          PUSH            {R7,LR}
__text:00002B6E 41 F2 D8 30    MOVW            R0, #0x13D8
__text:00002B72 6F 46          MOV             R7, SP
__text:00002B74 C0 F2 00 00    MOVT.W          R0, #0
__text:00002B78 78 44          ADD             R0, PC
__text:00002B7A 01 F0 38 EA    BLX             _puts
__text:00002B7E 00 20          MOVS            R0, #0
__text:00002B80 80 BD          POP             {R7,PC}

...

__cstring:00003E70 48 65 6C 6C 6F 20+aHelloWorld  DCB "Hello world!",0xA,0
\end{lstlisting}

% Q: If you subtract 0x13D8 from 0x3E70,
% you actually get a location that is not in this function, or in _puts.
% How is PC-relative addressing done in THUMB2?
% A: it's not Thumb-related. there are just mess with two different segments. TODO: rework this listing.

\myindex{\ThumbTwoMode}
\myindex{ARM!\Instructions!BL}
\myindex{ARM!\Instructions!BLX}

Die \TT{BL}- und \TT{BLX}-Anweisung im Thumb-Mode ist als Paar von 16-Bit-Anweisungen kodiert.
In Thumb-2 sind diese \IT{Ersatz}-Opcodes so erweitert, dass neue Anweisungen hier mit 32 Bit
kodiert werden können.

Offensichtlich beginnen die Opcodes der Thumb-2-Anweisungen immer mit \TT{0xFx} oder \TT{0xEx}.

Im \IDA-Listing jedoch sind die Bytes der Opcodes vertauscht weil für den ARM-Prozessor die
Anweisungen wie folgt kodiert werden:
Das letzte Byte kommt zuerst und danach das erste (für Thumb- und Thum-2-Mode) oder für
Anweisungen im ARM-Mode kommt das vierte Byte zuerst, dann das dritte, dann das zweite und
zum Schluss das erste (aufgrund des unterschiedlichen \gls{endianness}).

Die Bytes sind also im \IDA-Listing wie folgt angeordnet:
\begin{itemize}
\item für ARM und ARM64 Mode: 4-3-2-1;
\item für Thumb Mode: 2-1;
\item für 16-Bit-Anweisungspaar in Thumb-2 Mode: 2-1-4-3.
\end{itemize}

\myindex{ARM!\Instructions!MOVW}
\myindex{ARM!\Instructions!MOVT.W}
\myindex{ARM!\Instructions!BLX}

Wie zu sehen ist, beginnend die Anweisungen \TT{MOVW}, \TT{MOVT.W} und \TT{BLX} mit \TT{0xFx}.

Eine der Thumb-2-Anweisungen ist \TT{MOVW R0, \#0x13D8} ~---sie speichert einen 16-Bit-Wert in den
niederwertigeren Teil des \Reg{0}-Registers und setzt die höherwertigen Bits auf 0.

Des weiteren funktioniert \TT{MOVT.W R0, \#0} genau wie \TT{MOVT} aus dem vorherigen Beispiel,
jedoch nur für Thumb-2.

\myindex{ARM!mode switching}
\myindex{ARM!\Instructions!BLX}

Neben den anderen Unterschieden wird in diesem Fall die \TT{BLX}-Anweisung anstatt \TT{BL} genutzt.

Der Unterschied ist, dass, neben dem Speichern von \ac{RA} in das \ac{LR}-Register und die Übergabe
der Ausführungskontrolle an die \puts-Funktion, der Prozessor auch vom Thumb/Thumb-2-Mode in den
ARM-Mode (oder zurück) wechselt.

Diese Anweisung ist hier eingefügt weil die Anweisung mit der die Kontrolle abgegeben wird wie folgt
aussieht (im ARM-Mode kodiert):

\begin{lstlisting}[style=customasm]
__symbolstub1:00003FEC _puts           ; CODE XREF: _hello_world+E
__symbolstub1:00003FEC 44 F0 9F E5     LDR  PC, =__imp__puts
\end{lstlisting}

Dies ist im Endeffekt ein Sprung an die Stelle an der die Adresse von \puts in der import-Sektion geschriben wird.

Der aufmerksame Leser mag fragen: warum wird \puts nicht direkt an der Stelle im Code aufgerufen,
an der es benötigt wird? Dies wäre nicht sehr speicherplatzeffizient.

\myindex{Dynamically loaded libraries}
Fast jedes Programm nutzt externe, dynamische Bibliotheken (wie DLL in Windows, .so in *NIX oder.dylib in \MacOSX).
Diese Bibliotheken beinhalten häufig genutzte Funktion wie die Standard-C-Funktion \puts.

\myindex{Relocation}
In einer ausführbaren Binärdatei (Windows PE .exe, ELF oder Mach-O) existiert eine import-Sektion.
Dies ist eine Liste von Symbolen (Funktionen oder globale Variablen) die, zusammen mit den Namen, von
externen Modulen importiert werden.

Der \ac{OS}-Loader läd alle Module die gebraucht werden und bestimmt die korrekten Adressen von jedem Symbol,
während diese in dem primärem Modul aufgelistet werden.

In dem vorliegenden Fall ist \IT{\_\_imp\_\_puts} eine 32-Bit-Variable die vom \ac{OS}-Loader genutzt wird
um die korrekte Adresse der Funktion in der externen Bibliothek zu speichern.
Anschließend liest die \TT{LDR}-Anweisung den 32-Bit-Wert dieser Variable und schreibt ihn in das \ac{PC}-Register
bevor die Ausführkontrolle dorthin übergeben wird.

Um also die Zeit zu reduzieren die der \ac{OS}-Loader für dieses Vorgehen benötigt, ist es eine gute Idee
die Adressen für jedes Symbol einmalig an eine geeignete Stelle zu schreiben.

\myindex{thunk-functions}
Daneben wurde bereits erwähnt, dass es unmöglich ist einen 32-Bit-Wert in ein Register zu laden wenn
nur eine Anweisung ohne Speicher-Zugriff genutzt wird.

Aus diesem Grund ist die optimale Lösung, eine separate Funktion im ARM-Mode zu allozieren die lediglich
die Aufgabe hat die Ausführkontrolle an die dynamische Bibliothek zu übergeben und dann in diese kurze
Funktion mit einer Anweisung (so genannte \gls{thunk function}) aus dem Thumb-Code auszuführen.

\myindex{ARM!\Instructions!BL}
Übrigens: in dem vorherigen Beispiel (für ARM-Mode kompiliert) wird die Ausführkontrolle durch \TT{BL}
an die gleiche \gls{thunk function} übergeben.
Der Prozessor-Modus wird hier jedoch aufgrund des Fehlens eines \q{X} im Anweisungsnamen nicht gewechselt.

\myparagraph{Mehr über Thunk-Funktionen}
\myindex{thunk-functions}

Thunk-Funktionen sind aufgrund der irrtümlichen Bezeichnung schwierig zu verstehen.
Der einfachste Weg ist es sie als Adapter oder Konverter zwischen verschiedenen Anschlüssen aufzufassen.
Zum Beispiel wie einen Adapter zwischen einer britischen und einer amerikanischen Steckdose oder andersherum.
Thunk-Funktionen werden manchmal auch \IT{Wrapper} genannt.

Hier sind einige weitere Beschreibung dieser Funktionstypen:

\begin{framed}
\begin{quotation}
"Ein Teil der Software um Adressen zur Verfügung zu stellen:" nach P. Z. Ingerman,
der 1961 Thunk-Funktionen als Möglichkeit zum Binden von Aktualparametern zu deren
formalen Definitionen in Algol-60-Prozedur-Aufrufen. Wenn eine Prozedur mit einem Ausdruck anstatt
der formalen Parameter aufgerufen wird, generiert der Compiler eine Thunk-Funktion die den Ausdruck
errechnet und die Adresse des Ergebnisses an eine Standard-Stelle speichert.

\dots

% TODO: the english version is kind of misleading -- I could not find the word "bletcherous"
Microsoft und IBM haben beide in ihrem Intel-basierten System eine "16-Bit Umgebung"
und eine "32-Bit-Umgebung" definiert. Beide können auf dem selben Computer und demselben
Betriebssystem laufen (dank dem was Microsoft \q{ Windows On Windows} (WOW) nennt).
Sowohl MS als auch IBM haben entschieden, den Vorgang der zwischen 16- und 32-Bit wechselt
"Thunk" zu nennen; für Windows 95 existiert sogar ein Tool THUNK.EXE, das Thunk-Compiler
genannt wird.\end{quotation}
\end{framed}
% TODO FIXME move to bibliography and quote properly above the quote
( \href{http://go.yurichev.com/17362}{The Jargon File} )

\subsubsection{ARM64}

\myparagraph{GCC}

Das Beispiel wird im Folgenden mit GCC 4.1.8 in ARM64 kompiliert:

\lstinputlisting[numbers=left,label=hw_ARM64_GCC,caption=\NonOptimizing GCC 4.8.1 + objdump,style=customasm]{patterns/01_helloworld/ARM/hw.lst}

Es gibt keine Thumb- oder Thumb-2-Modes in ARM64, sondern nur ARM, also 32-Bit-Anweisungen.
Die Register-Anzahl ist verdoppelt: \myref{ARM64_GPRs}.
64-Bit-Register haben einen \TT{X-}Prefix, 32-Bit-Teile ein \TT{W-}.

\myindex{ARM!\Instructions!STP}
Die \TT{STP}-Anweisung (\IT{Store Pair}) speichert zwei Register auf dem Stack gleichzeitig:
\RegX{29} und \RegX{30}.

Natürlich kann diese Anweisung dieses Registerpaar an einer beliebigen Stelle im Speicher
sichern, aber da hier das \ac{SP}-Register angegeben ist, wird das Paar auf dem Stack gesichert.

ARM64-Register sind 64 Bit breit, jedes von ihnen ist 8 Byte groß. Dementsprechend werden 16 Byte
für das Speichern zweier Register benötigt.

Das Ausrufungszeichen (``!'')  nach dem Operanden bedeutet, dass zunächst der Wert 16 vom \ac{SP}
subtrahiert werden muss und erst dann die Werte vom Register-Paar auf den Stack geschrieben werden.
Dies wird auch \IT{pre-index} genannt.
Mehr über den Unterschied von \IT{post-index} und \IT{pre-index} ist im Abschnitt
\myref{ARM_postindex_vs_preindex} zu finden.

Im Sprachgebrauch des gebräuchlicheren x86, ist die erste Anweisung analog zu den Anweisungen
\TT{PUSH X29} und \TT{PUSH X30} zu verstehen.
\RegX{29} wird als \ac{FP} in ARM64 genutzt, und \RegX{30} als \ac{LR}, weswegen sie am Anfang der
Funktion gesichert und am Ende wiederhergestellt werden.

Die zweite Anweisung kopiert \ac{SP} in \RegX{29} (oder \ac{FP}) um den Stack Frame vorzubereiten.

\label{pointers_ADRP_and_ADD}
\myindex{ARM!\Instructions!ADRP/ADD pair}
\TT{ADRP} und \ADD-Anweisungen werden genutzt um die Adresse der Zeichenkette \q{Hello!} in das
Register \RegX{0} zu schreiben, da das erste Funktionsargument in an dieser Stelle übergeben wird. 

Es gibt in ARM keine Anweisung, die eine große Zahl in einem Register sichern kann, weil die Länge der
Anweisungen auf 4 Byte begrenzt ist. Siehe dazu auch \myref{ARM_big_constants_loading}).
Aus diesem Grund müssen mehrere Anweisungen genutzt werden. Die erste (\TT{ADRP}) schreibt die Adresse
der 4KiB-Page in der die Zeichenkette sich befindet in das Register \RegX{0}.
Die zweite (\ADD) addiert lediglich den Rest der Adresse.
Siehe dazu auch \myref{ARM64_relocs}.

\TT{0x400000 + 0x648 = 0x400648}, und die Zeichenkette \q{Hello!} ist im \TT{.rodata} Daten-Segmet
an dieser Adresse zu sehen.

\myindex{ARM!\Instructions!BL}

\puts wird anschließend mit der \TT{BL}-Anweisung aufgerufen. Dies wurde bereits diskutiert: \myref{puts}.

\MOV schreibt 0 in \RegW{0}.
\RegW{0} sind die niederwertigeren 32 Bit des 64-Bit-Registers \RegX{0}:

\input{ARM_X0_register}

Das Ergebnis der Funktion wird über \RegX{0} zurückgegeben und \main gibt 0 zurück.
Dies ist also der Weg wie das Ergebnis vorbereitet wird.
Der 32-Bit-Teil wird genutzt,weil der \Tint-Datentyp in ARM64 aus Kompatibilitätsgründen,
wie in x86-64, 32 Bit breit ist.

Da die Funktion einen 32-Bit \Tint-Wert zurück gibt, müssen lediglich die unteren 32 Bits des
\RegX{0}-Registers gefüllt werden.

Um dies zu überprüfen wird das Beispiel leicht verändert und neu kompiliert.
\main soll nun einen 64-Bit-Wert zurück geben:

\begin{lstlisting}[caption=\main gibt einen \TT{uint64\_t}-Datentyp zurück,style=customc]
#include <stdio.h>
#include <stdint.h>

uint64_t main()
{
        printf ("Hello!\n");
        return 0;
}
\end{lstlisting}

Das Ergebnis ist das gleiche, allerdings sieht \MOV nun wie folgt aus:

\begin{lstlisting}[caption=\NonOptimizing GCC 4.8.1 + objdump]
  4005a4:       d2800000        mov     x0, #0x0      // #0
\end{lstlisting}

\myindex{ARM!\Instructions!LDP}

\INS{LDP} (\IT{Load Pair}) stellt anschließend die Register \RegX{29} und \RegX{30} wieder her.

An dieser Stelle steht kein Ausrufungszeichen nach der Anweisung: dies impliziert, dass der Wert zunächst
vom Stack gelesen wird und erst dann wird \ac{SP} um den Wert 16 verringert.
Dies wird \IT{post-index} genannt.

\myindex{ARM!\Instructions!RET}
Eine neue Anweisung taucht hier in ARM64 auf \RET.
Diese arbeitet wie \TT{BX LR}, jedoch wird ein spezielles \IT{Hinweis-}Bit hinzugefügt, welches die \ac{CPU}
darüber informiert, dass dies ein Rücksprung aus einer Funktion ist und kein anderer Sprung, so dass die
Ausführung optimiert werden kann.

Aufgrund der Einfachheit dieser Funktion, erstellt der optimierende GCC den gleichen Code.

}
\PL{\subsection{ARM}
\label{sec:hw_ARM}

\myindex{\idevices}
\myindex{Raspberry Pi}
\myindex{Xcode}
\myindex{LLVM}
\myindex{Keil}
Do eksperymentów z ARM skorzystamy z kilku kompilatorów:

\begin{itemize}
\item Popularny w embedded-środowisku Keil Release 6/2013.

\item Apple Xcode 4.6.3 z kompilatorem LLVM-GCC 4.2
\footnote{To jest naprawdę tak: Apple Xcode 4.6.3 korzysta z open-source kompilatora GCC jako pierwszoplanowego kompilatora i generatora kodu LLVM}.

\item GCC 4.9 (Linaro) (dla ARM64), 
jest dostępny w postaci pliku wykonywalnego dla win32 na \url{http://go.yurichev.com/17325}.

\end{itemize}

Wszędzie w tej książce, jeżeli nie jest napisane co innego, mówimy o 32-bitowym ARM (w tym w trybach Thumb i Thumb-2).
Kiedy mówimy o 64-bitowym ARM, to on będzie tu oznaczony jako ARM64.

% subsections
\subsubsection{\NonOptimizingKeilVI (\ARMMode)}

Na początek skompilujmy nasz przykład w Keil:

\begin{lstlisting}
armcc.exe --arm --c90 -O0 1.c 
\end{lstlisting}

\myindex{\IntelSyntax}
Kompilator \IT{armcc} generuje listing w asemblerze w formacie Intel.
Ten listing zawiera niektóre wysokopoziomowe makro, powiązane z ARM
\footnote{naprzykład, on korzysta z instrukcji \PUSH/\POP, których nie ma w trybie ARM},
dlatego zobaczmy jak wygląda skompilowany kod w \IDA.

\begin{lstlisting}[caption=\NonOptimizingKeilVI (\ARMMode) \IDA,style=customasmARM]
.text:00000000             main
.text:00000000 10 40 2D E9    STMFD   SP!, {R4,LR}
.text:00000004 1E 0E 8F E2    ADR     R0, aHelloWorld ; "hello, world"
.text:00000008 15 19 00 EB    BL      __2printf
.text:0000000C 00 00 A0 E3    MOV     R0, #0
.text:00000010 10 80 BD E8    LDMFD   SP!, {R4,PC}

.text:000001EC 68 65 6C 6C+aHelloWorld  DCB "hello, world",0    ; DATA XREF: main+4
\end{lstlisting}

W przykładzie wyżej można łatwo dostrzec, że każda instrukcja ma rozmiar 4 bajty.
Rzeczywiście, przecież kompilowaliśmy nasz kod dla trybu ARM, a nie Thumb.

\myindex{ARM!\Instructions!STMFD}
\myindex{ARM!\Instructions!POP}
Pierwsza instrukcja, \INS{STMFD SP!, \{R4,LR\}}\footnote{\ac{STMFD}},
działa jak instrukcja \PUSH w x86: odkłada wartości dwóch rejestrów (\Reg{4} i \ac{LR}) na stos.
W listingu w asemblerze kompilator \IT{armcc}, żeby uprościć sprawę pokazuje tu instrukcję
\INS{PUSH \{r4,lr\}}.
Ale nie jest to dokładnie to co chcieliibyśmy zobaczyć, jako że instrukcja \PUSH jest dostępna tylko w trybie Thumb, dlatego,
 zaproponowałem pracować z kompilatorem \IDA.

Ta instrukcja zmniejsza \ac{SP}, żeby on wskazywał na miejsce na stosie, dostępne do zapisu nowych wartości, następnie zapisuje wartości rejestrów \Reg{4} i \ac{LR} 
pod adres w pamięci, na który wskazuje zmodyfikowany rejestr \ac{SP}.

Ta instrukcja, tak samo jak i instrukcja \PUSH w trybie Thumb, pozwala na odkładanie na stos kilku wartości rejestrów na raz, co może się okazać bardzo wygodne.
A propos, w x86 czegoś takiego nie ma.
Należy zauważyć, że \TT{STMFD}~--- jest generalizacją instrukcji \PUSH (czyli rozszerza jej możliwości), dlatego że może operować na różnych rejestrach, a nie tylko na \ac{SP}.
Inaczej mówiąc, z \TT{STMFD} można korzystać przy zapisie zestawu rejestrów we wskazanym miejscu w pamięci.

\myindex{\PICcode}
\myindex{ARM!\Instructions!ADR}
Instrukcja \INS{ADR R0, aHelloWorld} dodaje lub odejmuje zawartość rejestru \ac{PC} do przesunięcia, w którym jest przechowywana linia
\TT{hello, world}.
Co ma do tego \ac{PC}? Jest to tzw \q{\PICcode}
\footnote{będzie to szerzej omówione w kolejnym rozdziale ~(\myref{sec:PIC})}.
Nie jest on przywiązany do jakiegokolwiek miejsca w pamięci.
Inaczej mówiąc, jest to \ac{PC}-względne adresowanie.
W opcode instrukcji \TT{ADR} jest zapisana różnica między adresem tej instrukcji a miejscem, w którym jest przechowywana linia.
Ta różnica zawsze jest stała, niezależnie od tego w którym miejscu został załadowany \ac{OS} nasz kod.
Dlatego, wszystko czego potrzebujemy~--- to dodanie adresu bieżącej instrukcji (z \ac{PC}), żeby otrzymać bieżący adres bezwględny linii.

\myindex{ARM!\Registers!Link Register}
\myindex{ARM!\Instructions!BL}
Instrukcja \INS{BL \_\_2printf}\footnote{Branch with Link} wywołuje f-cje \printf.
Działanie tej f-cji składa się z 2 kroków:

\begin{itemize}
\item zapisać adres występujący po instrukcji \INS{BL} (\TT{0xC}) do rejestru \ac{LR};
\item przekazać zarządzanie do \printf, zapisując adres tej f-cji do \ac{PC}.
\end{itemize}

Kiedy f-cja \printf skończy pracę, procesor musi wiedzieć, gdzie zwrócić zarządzanie, dlatego kończąc pracę, każda funkcja przekazuje zarządzanie pod adres zapisany w rejestrze \ac{LR}.

Na tym polega główna różnica między \ac{RISC}-procesorami typu ARM i \ac{CISC}-procesorami typu x86,
gdzie adres powrotu zwykle jest odkładany na stos ~(\myref{sec:stack}).

A propos tego,  nie jest możliwe zakodowanie 32-bitowego adresu bezwzględnego (lub przesunięcia) w 32-bitowej instrukcji \INS{BL}, jako że ona ma miejsce tylko dla 24-ch bitów.
Jako że wszystkie instrukcje w trybie ARM mają długość 4 bajty (32 bity) i instrukcje mogą się znajdować tylko pod adresem wielokrotnym 4, to ostatnie 2 bity (które są zawsze zerowe) można nie kodować.
W wyniku mamy 26 bitów, za pomocą których można zakodować $current\_PC \pm{} \approx{}32M$.

\myindex{ARM!\Instructions!MOV}
Następna instrukcja \INS{MOV R0, \#0}\footnote{oznacza MOV}
po prostu zapisuje 0 do rejestru \Reg{0}.
To dlatego, że nasza f-cja zwraca 0, a wartość zwróconą każda f-cja zostawia w \Reg{0}.

\myindex{ARM!\Registers!Link Register}
\myindex{ARM!\Instructions!LDMFD}
\myindex{ARM!\Instructions!POP}
Ostatnia instrukcja \INS{LDMFD SP!, {R4,PC}}\footnote{\ac{LDMFD}~--- jest instrukcją względem działania odwrotną do \ac{STMFD}}.
Ona ładuje ze stosu (lub każdego innego miejsca w pamięci) wartości do zapisu w \Reg{4} i \ac{PC}, zwiększając \glslink{stack pointer}{wskaźnik stosu} \ac{SP}.
Tutaj ona działa analogicznie do \POP.\\
N.B. Pierwsza instrukcja \TT{STMFD} odłożyła na stos \Reg{4} i \ac{LR}, a \IT{przywracane} w wyniku działania \TT{LDMFD} są rejestry \Reg{4} i \ac{PC}.

Jak już wiemy, do rejestru \ac{LR} zwykle jest zapisywany adres miejsca w pamięci, pod który f-cja będzie zwracała zarządzanie.
Pierwsza isntrukcja odkłada tę wartość na stos, dlatego że nasza f-cja \main później sama będzie korzystała z tego rejstru w momencie wywołania f-cji \printf.
Czyli, na końcu f-cji, tę wartość można było od razu zapisać do \ac{PC}, w ten sposób przekazując zarządzanie tam, skąd nasza f-cja była wywołana.

Jako że, z reguły, f-cja \main jest funkcją główną w \CCpp, zarządzanie zostanie zwróconę do \ac{OS}, lub gdzieś do \ac{CRT} 
lub coś w tym stylu.

Всё это позволяет избавиться от инструкции \INS{BX LR} в самом конце функции.

\myindex{ARM!DCB}
\TT{DCB}~--- директива ассемблера, описывающая массивы байт или ASCII-строк, аналог директивы DB в x86-ассемблере.



\subsubsection{\NonOptimizingKeilVI (\ThumbMode)}

Skompilujmy ten sam przykład w Keil dla trybu Thumb:

\begin{lstlisting}
armcc.exe --thumb --c90 -O0 1.c 
\end{lstlisting}

Otrzymamy (w \IDA):

\begin{lstlisting}[caption=\NonOptimizingKeilVI (\ThumbMode) + \IDA,style=customasmARM]
.text:00000000             main
.text:00000000 10 B5          PUSH    {R4,LR}
.text:00000002 C0 A0          ADR     R0, aHelloWorld ; "hello, world"
.text:00000004 06 F0 2E F9    BL      __2printf
.text:00000008 00 20          MOVS    R0, #0
.text:0000000A 10 BD          POP     {R4,PC}

.text:00000304 68 65 6C 6C+aHelloWorld  DCB "hello, world",0    ; DATA XREF: main+2
\end{lstlisting}

Od razu można zauważyć (16-bitowe) opcode --- jest to, jak już było powiedziane, Thumb.

\myindex{ARM!\Instructions!BL}
Oprócz instrukcji \TT{BL}.
Ale tak naprawdę to ona się składa z dwóch 16-bitowych instrukcji.
Tak się dzieje dlatego że w jednym 16-bitowym opcode jest za mało miejsca dla przesunięcia, po którym znajduje się funkcja \printf.
Dlatego pierwsza 16-bitowa instrukcja ładuje starsze 10 bitów przesunięcia, a druga~--- młodsze 11 bitów przesunięcia.

% TODO:
% BL has space for 11 bits, so if we don't encode the lowest bit,
% then we should get 11 bits for the upper half, and 12 bits for the lower half.
% And the highest bit encodes the sign, so the destination has to be within
% \pm 4M of current_PC.
% This may be less if adding the lower half does not carry over,
% but I'm not sure --all my programs have 0 for the upper half,
% and don't carry over for the lower half.
% It would be interesting to check where __2printf is located relative to 0x8
% (I think the program counter is the next instruction on a multiple of 4
% for THUMB).
% The lower 11 bytes of the BL instructions and the even bit are
% 000 0000 0110 | 001 0010 1110 0 = 000 0000 0110 0010 0101 1100 = 0x00625c,
% so __2printf should be at 0x006264.
% But if we only have 10 and 11 bits, then the offset would be:
% 00 0000 0110 | 01 0010 1110 0 = 0 0000 0011 0010 0101 1100 = 0x00325c,
% so __2printf should be at 0x003264.
% In this case, though, the new program counter can only be 1M away,
% because of the highest bit is used for the sign.

jak już było powiedziane, wszystkie instrukcje w trybie Thumb są 2 bajtowe (lub 16 bitowe).
Dlatego sytuacja, w której Thumb-instrukcja zaczyna się pod adresem nieparzystym, jest niemożliwa.

Wniosek z tego jest taki, że ostatniego bitu adresu można nie kodować.
W ten sposób, w Thumb-instrukcji \TT{BL} można zakodować adres $current\_PC \pm{}\approx{}2M$.

\myindex{ARM!\Instructions!PUSH}
\myindex{ARM!\Instructions!POP}
Reszta instrukcji w funkcji (\PUSH i \POP) działa prawie tak samo, jak i opisane wyżej \TT{STMFD}/\TT{LDMFD}, tyle że rejestr \ac{SP} nie jest tu wskazany w sposób jawny.
\INS{ADR} działa dokładnie tak samo, jak i w poprzednim przykładzie.
\INS{MOVS} zapisuje wartość 0 do rejestru \Reg{0}, żeby f-cja zwróciła zero.



\subsubsection{\OptimizingXcodeIV (\ARMMode)}

Xcode 4.6.3 bez włączonego trybu optymalizacji generuje za dużo zbędnego kodu, dlatego włączymy optymalizacje kodu (flaga \Othree).

\begin{lstlisting}[caption=\OptimizingXcodeIV (\ARMMode),style=customasmARM]
__text:000028C4             _hello_world
__text:000028C4 80 40 2D E9   STMFD           SP!, {R7,LR}
__text:000028C8 86 06 01 E3   MOV             R0, #0x1686
__text:000028CC 0D 70 A0 E1   MOV             R7, SP
__text:000028D0 00 00 40 E3   MOVT            R0, #0
__text:000028D4 00 00 8F E0   ADD             R0, PC, R0
__text:000028D8 C3 05 00 EB   BL              _puts
__text:000028DC 00 00 A0 E3   MOV             R0, #0
__text:000028E0 80 80 BD E8   LDMFD           SP!, {R7,PC}

__cstring:00003F62 48 65 6C 6C+aHelloWorld_0  DCB "Hello world!",0
\end{lstlisting}

Instrukcje \TT{STMFD} i \TT{LDMFD} już są nam znane.

\myindex{ARM!\Instructions!MOV}
Instrukcja \MOV po prostu zapisuje liczbę \TT{0x1686} do rejestru \Reg{0}~--- jest to przesunięcie, wskazujące na linię \q{Hello world!}.

Rejestr \Reg{7} (wg standardu \IOSABI) jest frame pointer, będzie to omawiane później.

\myindex{ARM!\Instructions!MOVT}
Instrukcja \TT{MOVT R0, \#0} (MOVe Top) zapisuje 0 do starszych 16 bitów rejestru.
Rzecz polega na tym, że zwykła instrukcja \MOV w trybie ARM może zapisywać jakąkolwiek wartość tylko do młodszych 16 bitów rejestru, dlatego że nie można w niej zakodować więcej.
Warto pamiętać, że w trybie ARM opcode wszystkich instrukcji mogą wynosić maksymalnie 32 bity. Oczywiście, nie jest to prawda dla przemieszczenia danych między rejestrami.

Dlatego do zapisywania do starszych bitów (bity 16-31) istnieje dodatkowa instrukcja \INS{MOVT}.
Tutaj, natomiast, jej wykorzystanie jest zbędne, jako że \INS{MOV R0, \#0x1686} i bez tego wyzerowało starszą część rejestru.
Możliwe, że jest to niedociągnięcie kompilatora.
% TODO:
% I think, more specifically, the string is not put in the text section,
% ie. the compiler is actually not using position-independent code,
% as mentioned in the next paragraph.
% MOVT is used because the assembly code is generated before the relocation,
% so the location of the string is not yet known,
% and the high bits may still be needed.

\myindex{ARM!\Instructions!ADD}
Instrukcja \TT{ADD R0, PC, R0} dodaje \ac{PC} do \Reg{0} żeby wyliczyć adres rzeczywisty linii \q{Hello world!}. Jak już wiemy, jest to \q{\PICcode}, dlatego taka korektywa jest niezbędna.

Instrukcja \TT{BL} wywołuje \puts zamiast \printf.

\label{puts}
\myindex{\CStandardLibrary!puts()}
\myindex{puts() zamiast printf()}
Kompilator zamienia wywołanie \printf na \puts. 
Rzeczywiście, \printf z jednym argumentem jest prawie analogiczne do \puts.
 
\IT{Prawie}, jeśli założyć, że w linii nie będzie znaków sterujących \printf, 
zaczynających się od znaku procentu. Wtedy wyniki pracy tych dwóch f-cji będą różne
\footnote{Należy również zauważyć, że \puts nie potrzebuje znaku nowej linii `\textbackslash{}n',
dlatego jego tu nie ma.}.

Po co kompilator zamienił wywołanie jednej f-cji na inną? Prawdopodobnie, dlatego że \puts jest szybsze
\footnote{\href{http://go.yurichev.com/17063}{ciselant.de/projects/gcc\_printf/gcc\_printf.html}}. 
Najwidoczniej, dlatego że \puts wyprowadza symbole na \gls{stdout} nie porównując każdy ze znakiem procentu.

Dalej jest już znana instrukcja \TT{MOV R0, \#0}, ustawiająca 0 jako wartość zwracaną.



\subsubsection{\OptimizingXcodeIV (\ThumbTwoMode)}

Domyślnie Xcode 4.6.3 generuje kod dla Thumb-2 wyglądający mniej więcej tak:

\begin{lstlisting}[caption=\OptimizingXcodeIV (\ThumbTwoMode),style=customasmARM]
__text:00002B6C                   _hello_world
__text:00002B6C 80 B5          PUSH            {R7,LR}
__text:00002B6E 41 F2 D8 30    MOVW            R0, #0x13D8
__text:00002B72 6F 46          MOV             R7, SP
__text:00002B74 C0 F2 00 00    MOVT.W          R0, #0
__text:00002B78 78 44          ADD             R0, PC
__text:00002B7A 01 F0 38 EA    BLX             _puts
__text:00002B7E 00 20          MOVS            R0, #0
__text:00002B80 80 BD          POP             {R7,PC}

...

__cstring:00003E70 48 65 6C 6C 6F 20+aHelloWorld  DCB "Hello world!",0xA,0
\end{lstlisting}

% Q: If you subtract 0x13D8 from 0x3E70,
% you actually get a location that is not in this function, or in _puts.
% How is PC-relative addressing done in THUMB2?
% A: it's not Thumb-related. there are just mess with two different segments. TODO: rework this listing.

\myindex{\ThumbTwoMode}
\myindex{ARM!\Instructions!BL}
\myindex{ARM!\Instructions!BLX}
Instrukcje \TT{BL} i \TT{BLX} w Thumb, są kodowane jako para 16-bitowych instrukcji, 
a w Thumb-2 te opcode są rozszerzone w ten sposób, że nowe instrukcje tutaj są kodowane jako 32-bitowe instrukcje.
Można to łatwo zauważyc, jako że w Thumb-2 instrukcje zaczynają się od \TT{0xFx} lub od \TT{0xEx}.
Ale w listingu \IDA bajty opcode są zamienione miejscami.
Dzieje się tak dlatego, że w procesorze ARM instrukcje są kodowane w ten sposób:
najpierw podaje się ostatni bajt, potem pierwszy (dla trybów Thumb i Thumb-2), lub, 
(dla trybu ARM) najpierw czwarty bajt, następnie trzeci, drugi i pierwszy 
(tzn \gls{endianness}).

W ten sposób bajty są wypisywane w listingach IDA:

\begin{itemize}
\item dla trybów ARM i ARM64: 4-3-2-1;
\item dla trybu Thumb: 2-1;
\item dla pary 16-bitowych instrukcji w trybi Thumb-2: 2-1-4-3.
\end{itemize}

\myindex{ARM!\Instructions!MOVW}
\myindex{ARM!\Instructions!MOVT.W}
\myindex{ARM!\Instructions!BLX}
Także widzimy, że tutaj instrukcje \TT{MOVW}, \TT{MOVT.W} i \TT{BLX} zaczynają się od \TT{0xFx}.

Jedna z Thumb-2 instrukcji to
\TT{MOVW R0, \#0x13D8}~--- ona zapisuje 16-bitową liczbę do młodszych bitów rejestru \Reg{0}, zerując starsze bity.

jeszcze jest \TT{MOVT.W R0, \#0}~--- ta instrukcja działa tak samo, jak i \TT{MOVT} z poprzedniego przykładu, ale ona nie zadziała w trybie Thumb-2.

\myindex{ARM!przełączanie trybów}
\myindex{ARM!\Instructions!BLX}
Tutaj jest wykorzystywana instrukcja \TT{BLX} zamiast \TT{BL}.
Różnica polega na tym, że oprócz zapisania adresu powrotu do rejestru \ac{LR} i przekazania zarządzania 
do f-cji \puts, odbywa się zmiana trybu procesora z Thumb/Thumb-2 na tryb ARM (lub wstecz).
Tutaj jest to niezbędne dlatego, że instrukcja (ona jest zakodowana w trybie ARM) wygląda tak:

\begin{lstlisting}[style=customasmARM]
__symbolstub1:00003FEC _puts           ; CODE XREF: _hello_world+E
__symbolstub1:00003FEC 44 F0 9F E5     LDR  PC, =__imp__puts
\end{lstlisting}

Jest to zwykłe przejście na miejsce, w którym jest zapisany adres \puts w segmencie importów.
Można zadać pytanie: dlaczego nie można wywołać \puts od razu w tym miejscu kodu w którym jest to potrzebne?
Jest to niewygodne, z racji oszczędzania miejsca.

\myindex{Biblioteki łączone dynamicznie}
Praktycznie każdy program korzysta z zewnętrznych bibliotek łączonych dynamicznie (czy to będzie DLL w Windows, .so w *NIX 
czy .dylib w \MacOSX).
W bibliotekach dynamicznych znajdują się często wykorzystywane funkcje biblioteczne, w tym standardowa f-cja biblioteki C \puts.

\myindex{Relocation}
W wykonywalnym pliku binarnym 
(Windows PE .exe, ELF lub Mach-O) istnieje segment importów, lista symboli (funkcji lub zmiennych globalnych) importowanych z modułów zewnętrznych i, również, nazwy tych modułów.
Program rozruchowy ładuje niezbędne moduły i, iterując po symbolach zaimportowanych w module głównym, ustawia rzeczywiste adresy każdego z symboli.
W naszym przypadku, \IT{\_\_imp\_\_puts} 
jest zmienną 32-bitową, gdzie program rozruchowy wpiszę adres rzeczywisty tej f-cji w bibliotece zewnętrznej. 
Czyli instrukcja \TT{LDR} po prostu bierze 32-bitową wartość z tej zmiennej, i, zapisując ją do rejestru \ac{PC}, przekazuje tam zarządzanie.
Żeby zmniejszyć czas działania programu rozruchowego, trzeba, żeby adres się zapisywał tylko raz do specjalnie przydzielonego miejsca.

\myindex{thunk-функции}
Do tego, jak się już upewniliśmy, zapisywanie 32-bitowej liczby do rejestru jest niemożliwe bez odwoływań się do pamięci.
Także najbardziej optymalnym rozwiązaniem jest przydzielenie osobnej f-cji, pracującej w trybie ARM, 
jedynym celem której jest przekazywanie zarządzania dalej, do bbilioteki dynamicznie łączonej, i, następnie, odwoływanie się do tej któtkiej f-cji (tzw \glslink{thunk function}{thunk-funkcja}) z Thumb-kodu.

\myindex{ARM!\Instructions!BL}
A propos, w poprzednim przykładzie (skompilowanym dla trybu ARM), przejście za pomocą instrukcji \TT{BL} prowadzi 
do właśnie takiej \glslink{thunk function}{thunk-funkcji}, lecz procesor się nie przestawia na inny tryb (stąd wynika brak \q{X} w mnemoniku instrukcji).

\myparagraph{Jeszcze o thunk-funkcjach}
\myindex{thunk-функции}

Thunk-funckje są trudne do zrozumienia przede wszystkim przez brak spójności w terminologii.
Najprościej będzie traktować je jako adaptery-przejściówki z jednego typu gniazdek na drugi.
Na przykład, adapter pozwalający wstawić do gniazdka amerykańskiego wtyczkę brytyjską i na odwrót. Thunk-funkcje również są czasami nazywane \IT{wrapper-ami}. \IT{Wrap} w języku angielskim to \IT{owinąć}, \IT{zawinąć}.
Oto jeszcze kilka definicji tych funkcji:

\begin{framed}
\begin{quotation}
“A piece of coding which provides an address:”, according to P. Z. Ingerman, 
who invented thunks in 1961 as a way of binding actual parameters to their formal 
definitions in Algol-60 procedure calls. If a procedure is called with an expression 
in the place of a formal parameter, the compiler generates a thunk which computes 
the expression and leaves the address of the result in some standard location.

\dots

Microsoft and IBM have both defined, in their Intel-based systems, a “16-bit environment” 
(with bletcherous segment registers and 64K address limits) and a “32-bit environment” 
(with flat addressing and semi-real memory management). The two environments can both be 
running on the same computer and OS (thanks to what is called, in the Microsoft world, 
WOW which stands for Windows On Windows). MS and IBM have both decided that the process 
of getting from 16- to 32-bit and vice versa is called a “thunk”; for Windows 95, 
there is even a tool, THUNK.EXE, called a “thunk compiler”.
\end{quotation}
\end{framed}
% TODO FIXME move to bibliography and quote properly above the quote
( \href{http://go.yurichev.com/17362}{The Jargon File} )

\myindex{LAPACK}
\myindex{FORTRAN}
Jeszcze jeden przykład możemy znaleźć w bibliotece LAPACK --- (``Linear Algebra PACKage'') napisanej w języku FORTRAN.
Deweloperzy \CCpp również chcą korzystać z LAPACK, ale przepisywać ją na \CCpp, a następnie wspierać kilka wersji,
jest szaleństwem.
Także istnieją krótkie f-cje w C, które są wywoływane z \CCpp{}-środowiska, które, z kolei, wywołują funkcje FORTRAN,
i prawie nic oprócz tego nie robią:

\begin{lstlisting}[style=customc]
double Blas_Dot_Prod(const LaVectorDouble &dx, const LaVectorDouble &dy)
{
    assert(dx.size()==dy.size());
    integer n = dx.size();
    integer incx = dx.inc(), incy = dy.inc();

    return F77NAME(ddot)(&n, &dx(0), &incx, &dy(0), &incy);
}
\end{lstlisting}

Takie f-cje również są nazywane ``wrappers'' (tzn ``opakowanie'').



\subsubsection{ARM64}

\myparagraph{GCC}

Skompilujmy przykład w GCC 4.8.1 dla ARM64:

\lstinputlisting[numbers=left,label=hw_ARM64_GCC,caption=\NonOptimizing GCC 4.8.1 + objdump,style=customasmARM]{patterns/01_helloworld/ARM/hw.lst}

W ARM64 nie ma trybów Thumb i Thumb-2, tylko ARM, dlatego tu korzystamy tylko z 32-bitowych instrukcji.

Jest tu dwa razy więcej rejestrów: \myref{ARM64_GPRs}.
64-bitowe rejestry mają prefiks 
\TT{X-}, a ich 32-bitowe części --- \TT{W-}.

\myindex{ARM!\Instructions!STP}
Instrukcja \TT{STP} (\IT{Store Pair}) 
odkłada na stos od razu 2 rejestry: \RegX{29} i \RegX{30}.
Oczywiście, ta instrukcja może zapisać tę parę rejestrów gdziekolwiek w pamięci, ale tu jest wskazany rejestr \ac{SP}, także ta para jest odkładana na stos.

Rejestry w ARM64 są 64-bitowe, każdy o długości 8 bajtów, dlatego do przechowywania 2 rejestrów potrzeba 16 bajtów.

Wykrzyknik (``!'') po operandzie oznacza, że najpierw od \ac{SP} będzie odjęte 16 i dopiero po tej czynności wartości z obu rejestrów będą odłożone na stos.

Jest to nazywane \IT{pre-index}.
Więcej o róznicy między \IT{post-index} a \IT{pre-index} 
można znaleźć tu: \myref{ARM_postindex_vs_preindex}.

W ten sposób, posługując się terminologią x86, pierwsza instrukcja~--- jest analogiczna do \TT{PUSH X29} i \TT{PUSH X30}.
\RegX{29} w ARM64 jest wykorzystywane jako \ac{FP}, a \RegX{30} 
jako \ac{LR}, dlatego są one zapisywane w prologu funkcji.

Druga instrukcja kopiuje \ac{SP} do \RegX{29} (lub \ac{FP}).
Jest to niezbędne do ustawienia stack frame funkcji.

\label{pointers_ADRP_and_ADD}
\myindex{ARM!\Instructions!ADRP/ADD pair}
Instrukcje \TT{ADRP} i \ADD są potrzebne do skonstruowania adresu linii \q{Hello!} w rejestrze \RegX{0}, 
jako że pierwszy argument f-cji jest przekazywany przez ten rejestr.
Jednakże w ARM nie ma instrukcji, za pomocą których można zapisać do rejestru dużą liczbę 
(dlatego że długość instrukcji wynosi maksymalnie 4 bajty. Więcej informacji o tym można znaleźć tutaj: \myref{ARM_big_constants_loading}).
Dlatego trzeba skorzystać z kilku instrukcji.
Pierwsza instrukcja (\TT{ADRP}) zapisuje do \RegX{0} adres 4-kB strony, na której się znajduje linia, 
a druga (\ADD) dodaje do tego adresu resztę.
Więcej o tym tutaj: \myref{ARM64_relocs}.

\TT{0x400000 + 0x648 = 0x400648}, i możemy zobaczyć, że w segmencie danych \TT{.rodata} pod tym adresem znajduje się nasza
linia \q{Hello!}.

\myindex{ARM!\Instructions!BL}
Następnie, za pomocą instrukcji \TT{BL} jest wywoływane \puts. Zostało to omówione wcześniej: \myref{puts}.

Instrukcja \MOV zapisuje 0 do \RegW{0}. 
\RegW{0} to młodsze 32 bity 64-bitowego rejestru \RegX{0}:

\input{ARM_X0_register}

Wynik funkcji jest zwracany przez \RegX{0}, i \main zwraca 0.

Dlaczego 32-bitowa część?
Dlatego że w ARM64, jak i w x86-64, typ \Tint zostawili 32-bitowym, dla kompatybilności.

Odpowiednio, jako że funkcja zwraca 32-bitowy \Tint, to trzeba wypełnić tylko młodsze 32 bity 64-bitowego rejestru \RegX{0}.

Żeby mieć pewność, trochę zmienimy przykład i skompilujemy go ponownie.%

Teraz \main zwraca 64-bitową wartość:

\begin{lstlisting}[caption=\main zwracające wartość typu \TT{uint64\_t},style=customc]
#include <stdio.h>
#include <stdint.h>

uint64_t main()
{
        printf ("Hello!\n");
        return 0;
}
\end{lstlisting}

Wynik jest taki sam, tylko \MOV w tej linii wygląda teraz w ten sposób:

\begin{lstlisting}[caption=\NonOptimizing GCC 4.8.1 + objdump]
  4005a4:       d2800000        mov     x0, #0x0      // #0
\end{lstlisting}

\myindex{ARM!\Instructions!LDP}
Następnie za pomocą instrukcji \INS{LDP} (\IT{Load Pair}) są przywracane rejestry \RegX{29} i \RegX{30}.

Wykrzyknika po instrukcji brak. To oznacza, że najpierw wartości są zdejmowane ze stosu, i dopiero po tej czynności \ac{SP} jest zwiększane o 16.

Jest to nazywane \IT{post-index}.

\myindex{ARM!\Instructions!RET}
W ARM64 pojawia się nowa instrukcja: \RET. 
Ona działa tak samo jak i \INS{BX LR}, ale zawiera bit ,
który podpowiada procesorowi, że jest to wyjście z f-cji, a nie zwykłe przejście, żeby procesor mógł zoptymalizować tę instrukcję.

Jako że ta funkcja jest bardzo prosta, optymalizujący GCC generuje dokładnie taki sam kod.





}

\EN{\subsection{MIPS}

\subsubsection{A word about the \q{global pointer}}
\label{MIPS_GP}

\myindex{MIPS!\GlobalPointer}

One important MIPS concept is the \q{global pointer}.
As we may already know, each MIPS instruction has a size of 32 bits, so it's impossible to embed a 32-bit
address into one instruction: a pair has to be used for this 
(like GCC did in our example for the text string address loading).
It's possible, however, to load data from the address in the range of $register-32768...register+32767$ using one
single instruction (because 16 bits of signed offset could be encoded in a single instruction).
So we can allocate some register for this purpose and also allocate a 64KiB area of most used data.
This allocated register is called a \q{global pointer} and it points to the middle of the 64KiB area.
This area usually contains global variables and addresses of imported functions like \printf, 
because the GCC developers decided that getting the address of some function must be as fast as a single instruction
execution instead of two.
In an ELF file this 64KiB area is located partly in sections .sbss (\q{small \ac{BSS}}) for uninitialized data and 
.sdata (\q{small data}) for initialized data.
This implies that the programmer may choose what data he/she wants to be accessed fast and place it into 
.sdata/.sbss.
Some old-school programmers may recall the MS-DOS memory model \myref{8086_memory_model} 
or the MS-DOS memory managers like XMS/EMS where all memory was divided in 64KiB blocks.

\myindex{PowerPC}

This concept is not unique to MIPS. At least PowerPC uses this technique as well.

\subsubsection{\Optimizing GCC}

Lets consider the following example, which illustrates the \q{global pointer} concept.

\lstinputlisting[caption=\Optimizing GCC 4.4.5 (\assemblyOutput),numbers=left,style=customasm]{patterns/01_helloworld/MIPS/hw_O3_EN.s}

As we see, the \$GP register is set in the function prologue to point to the middle of this area.
The \ac{RA} register is also saved in the local stack.
\puts is also used here instead of \printf.
\myindex{MIPS!\Instructions!LW}
The address of the \puts function is loaded into \$25 using \INS{LW} the instruction (\q{Load Word}).
\myindex{MIPS!\Instructions!LUI}
\myindex{MIPS!\Instructions!ADDIU}
Then the address of the text string is loaded to \$4 using \INS{LUI} (\q{Load Upper Immediate}) and 
\INS{ADDIU} (\q{Add Immediate Unsigned Word}) instruction pair.
\INS{LUI} sets the high 16 bits of the register (hence \q{upper} word in instruction name) and \INS{ADDIU} adds
the lower 16 bits of the address.

\INS{ADDIU} follows \INS{JALR} (haven't you forgot \IT{branch delay slots} yet?).
The register \$4 is also called \$A0, which is used for passing the first function argument
\footnote{The MIPS registers table is available in appendix \myref{MIPS_registers_ref}}.

\myindex{MIPS!\Instructions!JALR}

\INS{JALR} (\q{Jump and Link Register}) jumps to the address stored in the \$25 register (address of \puts) 
while saving the address of the next instruction (LW) in \ac{RA}.
This is very similar to ARM.
Oh, and one important thing is that the address saved in \ac{RA} is not the address of the next instruction (because
it's in a \IT{delay slot} and is executed before the jump instruction),
but the address of the instruction after the next one (after the \IT{delay slot}).
Hence, $PC + 8$ is written to \ac{RA} during the execution of \TT{JALR}, in our case, this is the address of the \INS{LW}
instruction next to \INS{ADDIU}.

\INS{LW} (\q{Load Word}) at line 20 restores \ac{RA} from the local stack (this instruction is actually part of the function epilogue).

\myindex{MIPS!\Pseudoinstructions!MOVE}

\INS{MOVE} at line 22 copies the value from the \$0 (\$ZERO) register to \$2 (\$V0).
\label{MIPS_zero_register}

MIPS has a \IT{constant} register, which always holds zero.
Apparently, the MIPS developers came up with the idea that zero is in fact the busiest constant in the computer programming,
so let's just use the \$0 register every time zero is needed.

Another interesting fact is that MIPS lacks an instruction that transfers data between registers.
In fact, \TT{MOVE DST, SRC} is \TT{ADD DST, SRC, \$ZERO} ($DST=SRC+0$), which does the same.
Apparently, the MIPS developers wanted to have a compact opcode table.
This does not mean an actual addition happens at each \INS{MOVE} instruction.
Most likely, the \ac{CPU} optimizes these pseudo instructions and the \ac{ALU} is never used.

\myindex{MIPS!\Instructions!J}

\INS{J} at line 24 jumps to the address in \ac{RA}, which is effectively performing a return from the function.
\INS{ADDIU} after J is in fact executed before \INS{J} (remember \IT{branch delay slots}?) and is part of the function epilogue.
Here is also a listing generated by \IDA. Each register here has its own pseudo name:

\lstinputlisting[caption=\Optimizing GCC 4.4.5 (\IDA),numbers=left,style=customasm]{patterns/01_helloworld/MIPS/hw_O3_IDA_EN.lst}

The instruction at line 15 saves the GP value into the local stack, and this instruction is missing mysteriously from the GCC output listing, maybe by a GCC error
\footnote{Apparently, functions generating listings are not so critical to GCC users, so some unfixed errors may still exist.}.
The GP value has to be saved indeed, because each function can use its own 64KiB data window.
The register containing the \puts address is called \$T9, because registers prefixed with T- are called
\q{temporaries} and their contents may not be preserved.

\subsubsection{\NonOptimizing GCC}

\NonOptimizing GCC is more verbose.

\lstinputlisting[caption=\NonOptimizing GCC 4.4.5 (\assemblyOutput),numbers=left,style=customasm]{patterns/01_helloworld/MIPS/hw_O0_EN.s}

We see here that register FP is used as a pointer to the stack frame.
We also see 3 \ac{NOP}s.
The second and third of which follow the branch instructions.
Perhaps the GCC compiler always adds \ac{NOP}s (because of \IT{branch delay slots}) after branch
instructions and then, if optimization is turned on, maybe eliminates them.
So in this case they are left here.

Here is also \IDA listing:

\lstinputlisting[caption=\NonOptimizing GCC 4.4.5 (\IDA),numbers=left,style=customasm]{patterns/01_helloworld/MIPS/hw_O0_IDA_EN.lst}

\myindex{MIPS!\Pseudoinstructions!LA}

Interestingly, \IDA recognized the \INS{LUI}/\INS{ADDIU} instructions pair and coalesced them into one 
\INS{LA} (\q{Load Address}) pseudo instruction at line 15.
We may also see that this pseudo instruction has a size of 8 bytes!
This is a pseudo instruction (or \IT{macro}) because it's not a real MIPS instruction, but rather
a handy name for an instruction pair.

\myindex{MIPS!\Pseudoinstructions!NOP}
\myindex{MIPS!\Instructions!OR}

Another thing is that \IDA doesn't recognize \ac{NOP} instructions, so here they are at lines 22, 26 and 41.
It is \TT{OR \$AT, \$ZERO}.
Essentially, this instruction applies the OR operation to the contents of the \$AT register
with zero, which is, of course, an idle instruction.
MIPS, like many other \ac{ISA}s, doesn't have a separate \ac{NOP} instruction.

\subsubsection{Role of the stack frame in this example}

The address of the text string is passed in the register.
Why setup a local stack anyway?
The reason for this lies in the fact that the values of registers \ac{RA} and GP have to be saved somewhere 
(because \printf is called), and the local stack is used for this purpose.
If this was a \gls{leaf function}, it would have been possible to get rid of the function prologue and epilogue,
for example: \myref{MIPS_leaf_function_ex1}.

\subsubsection{\Optimizing GCC: load it into GDB}

\myindex{GDB}
\lstinputlisting[caption=sample GDB session]{patterns/01_helloworld/MIPS/O3_GDB.txt}

}
\RU{\subsection{MIPS}

\subsubsection{О \q{глобальном указателе} (\q{global pointer})}
\label{MIPS_GP}

\myindex{MIPS!\GlobalPointer}
\q{Глобальный указатель} (\q{global pointer})~--- это важная концепция в MIPS.
Как мы уже возможно знаем, каждая инструкция в MIPS имеет размер 32 бита, поэтому невозможно
закодировать 32-битный адрес внутри одной инструкции. Вместо этого нужно использовать пару инструкций
(как это сделал GCC для загрузки адреса текстовой строки в нашем примере).
С другой стороны, используя только одну инструкцию, 
возможно загружать данные по адресам в пределах $register-32768...register+32767$, потому что 16 бит
знакового смещения можно закодировать в одной инструкции).
Так мы можем выделить какой-то регистр для этих целей и ещё выделить буфер в 64KiB для самых 
часто используемых данных.
Выделенный регистр называется \q{глобальный указатель} (\q{global pointer}) и он указывает на середину
области 64KiB.
Эта область обычно содержит глобальные переменные и адреса импортированных функций вроде \printf,
потому что разработчики GCC решили, что получение адреса функции должно быть как можно более быстрой операцией,
исполняющейся за одну инструкцию вместо двух.
В ELF-файле эта 64KiB-область находится частично в секции .sbss (\q{small \ac{BSS}}) для неинициализированных
данных и в секции .sdata (\q{small data}) для инициализированных данных.
Это значит что программист может выбирать, к чему нужен как можно более быстрый доступ, и затем расположить
это в секциях .sdata/.sbss.
Некоторые программисты \q{старой школы} могут вспомнить модель памяти в MS-DOS \myref{8086_memory_model} 
или в менеджерах памяти вроде XMS/EMS, где вся память делилась на блоки по 64KiB.

\myindex{PowerPC}
Эта концепция применяется не только в MIPS. По крайней мере PowerPC также использует эту технику.

\subsubsection{\Optimizing GCC}

Рассмотрим следующий пример, иллюстрирующий концепцию \q{глобального указателя}.

\lstinputlisting[caption=\Optimizing GCC 4.4.5 (\assemblyOutput),numbers=left,style=customasm]{patterns/01_helloworld/MIPS/hw_O3_RU.s}

Как видно, регистр \$GP в прологе функции выставляется в середину этой области.
Регистр \ac{RA} сохраняется в локальном стеке.
Здесь также используется \puts вместо \printf.
\myindex{MIPS!\Instructions!LW}
Адрес функции \puts загружается в \$25 инструкцией \INS{LW} (\q{Load Word}).
\myindex{MIPS!\Instructions!LUI}
\myindex{MIPS!\Instructions!ADDIU}
Затем адрес текстовой строки загружается в \$4 парой инструкций \INS{LUI} (\q{Load Upper Immediate}) и
\INS{ADDIU} (\q{Add Immediate Unsigned Word}).
\INS{LUI} устанавливает старшие 16 бит регистра (поэтому в имени инструкции присутствует \q{upper}) и \INS{ADDIU}
прибавляет младшие 16 бит к адресу.
\INS{ADDIU} следует за \INS{JALR} (помните о \IT{branch delay slots}?).
Регистр \$4 также называется \$A0, который используется для передачи первого аргумента функции
\footnote{Таблица регистров в MIPS доступна в приложении \myref{MIPS_registers_ref}}.
\myindex{MIPS!\Instructions!JALR}
\INS{JALR} (\q{Jump and Link Register}) делает переход по адресу в регистре \$25 (там адрес \puts) 
при этом сохраняя адрес следующей инструкции (\INS{LW}) в \ac{RA}.
Это так же как и в ARM.
И ещё одна важная вещь: адрес сохраняемый в \ac{RA} это адрес не следующей инструкции (потому что
это \IT{delay slot} и исполняется перед инструкцией перехода),
а инструкции после неё (после \IT{delay slot}).
Таким образом во время исполнения \INS{JALR} в \ac{RA} записывается $PC + 8$.
В нашем случае это адрес инструкции \INS{LW} следующей после \INS{ADDIU}.

\INS{LW} (\q{Load Word}) в строке 20 восстанавливает \ac{RA} из локального стека (эта инструкция скорее часть эпилога функции).

\myindex{MIPS!\Pseudoinstructions!MOVE}
\INS{MOVE} в строке 22 копирует значение из регистра \$0 (\$ZERO) в \$2 (\$V0).

\label{MIPS_zero_register}
В MIPS есть \IT{константный} регистр, всегда содержащий ноль.
Должно быть, разработчики MIPS решили, что 0 это самая востребованная константа в программировании,
так что пусть будет использоваться регистр \$0, всякий раз, когда будет нужен 0.
Другой интересный факт: в MIPS нет инструкции, копирующей значения из регистра в регистр.
На самом деле, \TT{MOVE DST, SRC} это \TT{ADD DST, SRC, \$ZERO} ($DST=SRC+0$), которая делает тоже самое.
Очевидно, разработчики MIPS хотели сделать как можно более компактную таблицу опкодов.
Это не значит, что сложение происходит во время каждой инструкции \INS{MOVE}.
Скорее всего, эти псевдоинструкции оптимизируются в \ac{CPU} и \ac{ALU} никогда не используется.

\myindex{MIPS!\Instructions!J}
\INS{J} в строке 24 делает переход по адресу в \ac{RA}, и это работает как выход из функции.
\INS{ADDIU} после \INS{J} на самом деле исполняется перед \INS{J} (помните о \IT{branch delay slots}?) 
и это часть эпилога функции.

Вот листинг сгенерированный \IDA. Каждый регистр имеет свой псевдоним:

\lstinputlisting[caption=\Optimizing GCC 4.4.5 (\IDA),numbers=left]{patterns/01_helloworld/MIPS/hw_O3_IDA_RU.lst}

Инструкция в строке 15 сохраняет GP в локальном стеке. Эта инструкция мистическим образом отсутствует
в листинге от GCC, может быть из-за ошибки в самом GCC\footnote{Очевидно, функция вывода листингов не так критична
для пользователей GCC, поэтому там вполне могут быть неисправленные ошибки.}.
Значение GP должно быть сохранено, потому что всякая функция может работать со своим собственным окном данных
размером 64KiB.
Регистр, содержащий адрес функции \puts называется \$T9, потому что регистры с префиксом T- называются
\q{temporaries} и их содержимое можно не сохранять.

\subsubsection{\NonOptimizing GCC}

\NonOptimizing GCC более многословный.

\lstinputlisting[caption=\NonOptimizing GCC 4.4.5 (\assemblyOutput),numbers=left,style=customasm]{patterns/01_helloworld/MIPS/hw_O0_RU.s}

Мы видим, что регистр FP используется как указатель на фрейм стека.
Мы также видим 3 \ac{NOP}-а.
Второй и третий следуют за инструкциями перехода.
Видимо, компилятор GCC всегда добавляет \ac{NOP}-ы (из-за \IT{branch delay slots})
после инструкций переходов и затем, если включена оптимизация, от них может избавляться.
Так что они остались здесь.

Вот также листинг от \IDA:

\lstinputlisting[caption=\NonOptimizing GCC 4.4.5 (\IDA),numbers=left]{patterns/01_helloworld/MIPS/hw_O0_IDA_RU.lst}

\myindex{MIPS!\Pseudoinstructions!LA}
Интересно что \IDA распознала пару инструкций \INS{LUI}/\INS{ADDIU} и собрала их в одну псевдоинструкцию 
\INS{LA} (\q{Load Address}) в строке 15.
Мы также видим, что размер этой псевдоинструкции 8 байт!
Это псевдоинструкция (или \IT{макрос}), потому что это не настоящая инструкция MIPS, а скорее
просто удобное имя для пары инструкций.

\myindex{MIPS!\Pseudoinstructions!NOP}
\myindex{MIPS!\Instructions!OR}
Ещё кое что: \IDA не распознала \ac{NOP}-инструкции в строках 22, 26 и 41.

Это \TT{OR \$AT, \$ZERO}.
По своей сути это инструкция, применяющая операцию \IT{ИЛИ} к содержимому регистра \$AT с нулем,
что, конечно же, холостая операция.
MIPS, как и многие другие \ac{ISA}, не имеет отдельной \ac{NOP}-инструкции.

\subsubsection{Роль стекового фрейма в этом примере}

Адрес текстовой строки передается в регистре.
Так зачем устанавливать локальный стек?
Причина в том, что значения регистров \ac{RA} и GP должны быть сохранены где-то
(потому что вызывается \printf) и для этого используется локальный стек.

Если бы это была \gls{leaf function}, тогда можно было бы избавиться от пролога и эпилога функции. Например:
 \myref{MIPS_leaf_function_ex1}.

\subsubsection{\Optimizing GCC: загрузим в GDB}

\myindex{GDB}
\lstinputlisting[caption=пример сессии в GDB]{patterns/01_helloworld/MIPS/O3_GDB.txt}

}
\ITA{\subsection{MIPS}

\subsubsection{Qualche parola sul \q{global pointer}}
\label{MIPS_GP}

\myindex{MIPS!\GlobalPointer}

Un importante concetto MIPS e' il \q{global pointer}.
Come potremmo gia' sapere, ogni ustrizione MIPS ha lunghezza pari a 32 bit, quindi e' impossibile inserire un indirizzo a 32-bit
in una sola istruzione: deve essere usata una coppia (come ha fatto GCC nell'esempio per il caricamento dell'indirizzo della stringa).
E' comunque possibile caricare dati da un indirizzo nell'intervallo $register-32768...register+32767$ utilizzando una singola istruzione
(perche' 16 bit di un signed offset possono essere codificati in una singola istruzione).
Possiamo quindi allocare per questo scopo un registro e allocare anche un'area a 64KiB per i dati piu' usati.
Questo registro dedicato e' detto \q{global pointer} e punta in mezzo dall'area di 64KiB.
Questa area solitamente contiene variabili globali e indirizzi di funzioni importate come \printf, perche' gli sviluppatori di GCC hanno deciso
che il recupero dell'indirizzo di una funzione deve essere veloce tanto quanto l'esecuzione di una singola istruzione invece di due.

In un file ELF questa area di 64KiB e' collocata parzialmente nelle sezioni .sbss (\q{small \ac{BSS}}) per dati non inizializzati
e .sdata (\q{small data}) per dati inizializzati.

Cio' implica che il programmatore puo' scegliere a quale dati si possa accedere piu' velocemente e piazzarli nelle sezioni .sdata/.sbss.
Alcuni programmatori old-school potrebbero ricordarsi del memory model MS-DOS \myref{8086_memory_model} 
o dei memory manger MS-DOS come XMS/EMS, in cui tutta la memoria era divisa in blocchi da 64KiB.

\myindex{PowerPC}

Questo concetto non e' unicamente di MIPS. Anche PowerPC usa la stessa tecnica.

\subsubsection{\Optimizing GCC}

Consideriamo il seguente esempio che illustra il concetto di \q{global pointer}.

\lstinputlisting[caption=\Optimizing GCC 4.4.5 (\assemblyOutput),numbers=left,style=customasmMIPS]{patterns/01_helloworld/MIPS/hw_O3_EN.s}

Come possiamo vedere, il registro \$GP e' settato nel prologo della funzione affinche' punti nel mezzo di questa area.
Il registro \ac{RA} viene anche salvato sullo stack locale.
\puts e' usata anche qui al posto di \printf.
\myindex{MIPS!\Instructions!LW}
L'indirizzo della funzione \puts e' caticato in \$25 usando \INS{LW} , l'istruzione (\q{Load Word}).
\myindex{MIPS!\Instructions!LUI}
\myindex{MIPS!\Instructions!ADDIU}
Successivamente l'indirizzo della stringa viene caricato in \$4 usando la coppia di istruzioni \INS{LUI} (\q{Load Upper Immediate}) e 
\INS{ADDIU} (\q{Add Immediate Unsigned Word}).
\INS{LUI} setta i 16 bit alti del registro (da cui la parola \q{upper} nel nome dell'istruzione) e \INS{ADDIU} aggiunge
i 16 bit piu' bassi dell'indirizzo.

\INS{ADDIU} segue \INS{JALR} (ti ricordi dei \IT{branch delay slots}?).
Il registro \$4 e' anche detto \$A0, e' usato per passare il primo argomento di una funzione
\footnote{La tabella dei registri MIPS e' riportata in appendice \myref{MIPS_registers_ref}}.

\myindex{MIPS!\Instructions!JALR}

\INS{JALR} (\q{Jump and Link Register}) salta all'indirizzo memorizzato nel registro \$25 register (indirizzo di \puts) 
salvando l'indirizzo della prossima istruzione (LW) in \ac{RA}.
Questo e' molto simile ad ARM.
Oh, e una cosa importate e' che l'indirizzo salvato in \ac{RA} non e' l'indirizzo della prossima istruzione (perche' e' in un 
\IT{delay slot} e viene eseguito prima prima dell'istruzione jump),
ma l'indirizzo dell'istruzione dopo la prossima (dopo il \IT{delay slot}).

Quindi, $PC + 8$ viene scritto in \ac{RA} durante l'esecuzione di \TT{JALR}, nel nostro caso, questo e' l'indirizzo dell'istruzione 
\INS{LW} successiva a \INS{ADDIU}.

\INS{LW} (\q{Load Word}) a riga 20 ripristina \ac{RA} dallo stack locale (questa istruzione e' in effetti parte
dell'epilogo della funzione).

\myindex{MIPS!\Pseudoinstructions!MOVE}

\INS{MOVE} a riga 22 copia il valore dal registro \$0 (\$ZERO) al \$2 (\$V0).
\label{MIPS_zero_register}

MIPS ha un registro \IT{costante}, il cui valore e' sempre zero.
Apparentemente, gli sviluppatori MIPS hanno pensato che zero e' la costante piu' usata in programmazione, quindi usiamo il registro \$0
 ogni volta che serve il valore zero.

Un altro fatto interessante e' che in MIPS non c'e' un'istruzione che trasferisce dati tra registri.
Infatti, \TT{MOVE DST, SRC} e' \TT{ADD DST, SRC, \$ZERO} ($DST=SRC+0$), che fa la stessa cosa.
Apparentemente gli sviluppatori MIPS desideravano avere una tabella di opcode compatta.
Questo non significa che un'addizione si verifichi per ogni istruzione \INS{MOVE}.
Molto probabilmente, la \ac{CPU} ottimizza queste pseudoistruzioni e la \ac{ALU} non viene mai usata.

\myindex{MIPS!\Instructions!J}

\INS{J} a riga 24 salta all'indirizzo in \ac{RA}, effettuando di fatti il ritorno dalla funzione.
\INS{ADDIU} dopo J e' in effetti eseguita prima di J (ricordi i \IT{branch delay slots}?) e fa parte dell'epilogo della funzione.
Ecco anche il listato generato da \IDA. Ogni registro qui ha il suo pseudonimo:

\lstinputlisting[caption=\Optimizing GCC 4.4.5 (\IDA),numbers=left,style=customasmMIPS]{patterns/01_helloworld/MIPS/hw_O3_IDA_EN.lst}

L'istruzione alla riga 15 salva il valore di GP sullo stack locale, e questa istruzione manca misteriosamente dal listato prodotto da GCC, forse per un errore di GCC
\footnote{Apparentemente, le funzioni che generano i listati non sono fondamentali per gli utenti GCC, quindi qualche errore
non ancora corretto puo' esserci.}.
Il valore di GP deve essere infatti salvato, perche' ogni funzione puo' usare la sua finestra dati da 64KiB.
Il registro contenente l'indirizzo di \puts e' chiamato \$T9, perche' i registri con il prefisso T- sono detti 
\q{temporaries} ed il loro contenuto puo' non essere preservato.

\subsubsection{\NonOptimizing GCC}

\NonOptimizing GCC e' piu' verboso.

\lstinputlisting[caption=\NonOptimizing GCC 4.4.5 (\assemblyOutput),numbers=left,style=customasmMIPS]{patterns/01_helloworld/MIPS/hw_O0_EN.s}

Qui vediamo che il registro FP e' usato come un puntatore allo stack frame.
Vediamo anche 3 \ac{NOP}s.
Di cui il secondo e terzo seguono all'istruzione branch.
Forse GCC aggiunge sempre \ac{NOP}s (a causa dei \IT{branch delay slots}) dopo le istruzioni branch
e successivamente, se le ottimizzazioni sono attivate, forse li elimina.
Quindi in questo caso sono rimasti.

Ecco anche il listato \IDA:

\lstinputlisting[caption=\NonOptimizing GCC 4.4.5 (\IDA),numbers=left,style=customasmMIPS]{patterns/01_helloworld/MIPS/hw_O0_IDA_EN.lst}

\myindex{MIPS!\Pseudoinstructions!LA}

E' interessante notare che \IDA ha riconosciuto la coppia di istruzioni \INS{LUI}/\INS{ADDIU} e le ha fusae in un'unica pseudoistruzione 
\INS{LA} (\q{Load Address}) a riga 15.
Possiamo anche vedere che questa pseudoistruzione e' lunga 8 bytes!
Questa e' una pseudoistruzione (o \IT{macro}) in quanto non e' una vera istruzione MIPS , ma soltanto un nome comodo per una coppia
di istruzioni.

\myindex{MIPS!\Pseudoinstructions!NOP}
\myindex{MIPS!\Instructions!OR}

Un'altra cosa e' che \IDA non ha riconosciuto le istruzioni \ac{NOP} , che sono alle righe 22, 26 e 41.
E' \TT{OR \$AT, \$ZERO}.
Essenzialmente, questa istruzione applica l'operazione OR al contenuto del registro \$AT 
con zero, che e', ovviamente, un'istruzione nulla/inutile.
MIPS, come molte altre \ac{ISA}, non ha un'istruzione \ac{NOP} propria.

\subsubsection{Ruolo dello the stack frame in questo esempio}

L'indirizzo della stringa e' passato nel registro.
Perche' impostare allora ugualmente uno stack locale?
La ragione sta nel fatto che i valori dei registri \ac{RA} e GP devono essere salvati da qualche parte 
(poiche' viene chiamata \printf ), e lo stack locale e' usato proprio per questo scopo.
Se fosse stata una \gls{leaf function}, sarebbe stato possibile fare a meno (disfarsi) del prologo e dell'epilogo,
ad esempio: \myref{MIPS_leaf_function_ex1}.

\subsubsection{\Optimizing GCC: carichiamolo in GDB}

\myindex{GDB}
\lstinputlisting[caption=sample GDB session]{patterns/01_helloworld/MIPS/O3_GDB.txt}
}
\DE{\subsection{MIPS}

\subsubsection{ein Wort über \q{globale Zeiger}}
\label{MIPS_GP}

\myindex{MIPS!\GlobalPointer}

Ein wichtiges Konzept bei MIPS ist der \q{globale Zeiger}.
Wie bereits bekannt, besteht besteht jede MIPS-Anweisung aus 32 Bit, so dass es
nicht möglich ist eine 32-Bit-Anweisung darin unterzubringen: ein Anweisungspaar
wird verwendet (wie GCC dies in dem Beispiel zum Laden der Zeichenkettenadresse
getan hat).

Es ist jedoch möglich, Daten aus dem Adressbereich $register-32768...register+32767$
mit nur einer Anweisung zuladen, weil ein 16 Bit vorzeichenbehafteter Offset
in einer einzelnen Anweisung kodiert werden kann.
Es können also einige Register zu diesen Zweck alloziert werden und 64KiB-Bereiche
für die am häufigsten genutzten Daten
Dieses allozierte Register wird \q{globaler Zeiger} genannt und zeigt in die Mitte
des 64KiB-Bereichs.

Dieser Bereich enthält in der Regel globale Variablen und Adressen von importierten
Funktionen wie \printf, weil die GCC-Entwickler entschieden, dass das Laden einiger
Funktionsadressen so schnell sein sollte wie eine einzelne Anweisung anstatt zwei.
In einer ELF-Datei ist dieser 64KiB-Bereich teils in der Sektion .sbss (\q{small \ac{BSS}})
für uninitialiserte Daten und teil in .sdata (\q{small data}) für initialisierte
Daten zu finden.

Dies impliziert, dass der Programmierer entscheiden kann, auf welche Daten ein schneller
Zugriff (durch das Platzieren in .sdata/.sbss) möglich sein soll.
Einige Programmierer \q{der alten Schule} erinnern sich vielleicht an das MS-DOS
Speichermodell \myref{8086_memory_model} oder MS-DOS Speicherverwaltungen wie XMS/EMS
bei denen der komplette Speicher in 64KiB-Blöcke unterteilt war.

\myindex{PowerPC}

Dieses Konzept ist nicht nur bei MIPS vorhanden. Zumindest der PowerPC nutzt es ebenfalls.

\subsubsection{\Optimizing GCC}

Nachfolgen ein Beispiel welches das Konzept der \q{globalen Zeiger} veranschaulichen soll.

\lstinputlisting[caption=\Optimizing GCC 4.4.5 (\assemblyOutput),numbers=left,style=customasm]{patterns/01_helloworld/MIPS/hw_O3_DE.s}

Wie zu sehen wird das \$GP-Register im Funktionsprolog so gesetzt, dass auf die Mitte
dieses Bereichs gezeigt wird.
Das \ac{RA}-Register wird ebenfalls auf dem lokalen Stack gesichert.
Anstelle von \printf wird wieder \puts aufgerufen.
\myindex{MIPS!\Instructions!LW}
Die Adresse der Funktion \puts wird mit der \INS{LW}-Anweisung (\q{Load Word}) in \$25 geladen.
\myindex{MIPS!\Instructions!LUI}
\myindex{MIPS!\Instructions!ADDIU}
Anschließend wird die Adressse der Zeichenkette mit dem Anweisungspaar \INS{LUI} (\q{Load Upper Immediate})
und \INS{ADDIU} (\q{Add Immediate Unsigned Word}) in \$4 geladen.
\INS{LUI} setzt die  oberen 16 Bit des Registers (deswegen \q{upper} im Anweisungsnamen)
und \INS{ADDIU} addiert die unteren 16 Bit der Adresse.

\INS{ADDIU} folgt \INS{JALR} (zur Erinnerung: \IT{branch delay slots}).
Das Register \$4 wird auch \$A0 genannt und für das Übergeben des ersten Funktionsarguments
genutzt\footnote{Die MIPS-Register-Tabelle ist im Anhang verfügbar \myref{MIPS_registers_ref}}.

\myindex{MIPS!\Instructions!JALR}

\INS{JALR} (\q{Jump and Link Register}) springt zu der Adresse die im Register \$25
gespeichert ist (Adresse von \puts) und speichert die Adresse der übernächsten Anweisung
(LW) in \ac{RA}. Dies ist sehr ähnlich zu ARM.
Eine wichtige Sache ist, dass die Adresse in \ac{RA} nicht die Adresse der nächsten
Anweisung ist (da dies ein \IT{delay slot} ist und vor der Sprunganweisung ausgeführt wird),
sondern die Adresse der darauf folgenden Anweisung (nach dem \IT{delay slot}).
Da in diesem Fall während der Ausführung von \TT{JALR} der Wert $PC + 8$ in \ac{RA}
geschrieben wird, ist dies die Adresse der \INS{LW}-Anweisung nach \INS{ADDIU}.

\INS{LW} (\q{Load Word}) in Zeile 20 stellt \ac{RA} wieder vom lokalen Stack her.
Diese Anweisung ist tatsächlich ein Teil des Funktionsepilogs.

\myindex{MIPS!\Pseudoinstructions!MOVE}

\INS{MOVE} in Zeile 22 kopiert der Wert vom \$0 (\$ZERO)-Register in \$2 (\$V0).
\label{MIPS_zero_register}

MIPS besitzt ein \IT{konstantes} Register, welches immer eine Null beinhaltet.
Anscheinend hatten die MIPS-Entwickler die Idee, dass eine Null die beliebteste
Konstante in der Programmierung ist, also wird in Zukunft immer das \$0-Register
genutzt wenn eine Null benötigt wird.

Eine weitere interessante Tatsache in MIPS ist das Fehlen einer Anweisung zum Transferieren
von Daten zwischen zwei Registern.
Die Anweisung \TT{MOVE DST, SRC} entspricht jedoch \TT{ADD DST, SRC, \$ZERO} ($DST=SRC+0$),
und bewirkt genau das gleiche.
Anscheinend wollten die MIPS-Entwickler eine kompakte Opcode-Tabelle haben.
Das bedeutet nicht, dass dies bei jeder \INS{MOVE}-Anweisung passiert.
Sehr wahrscheinlich optimiert die \ac{CPU} diese Pseudo-Anweisung und die \ac{ALU}
wird niemals genutzt.

\myindex{MIPS!\Instructions!J}

\INS{J} in Zeile 24 springt zu der Adresse in \ac{RA}, was im Endeffekt einem Sprung aus einer Funktion entspricht.
\INS{ADDIU} nach J wird tatsächlich bevor \INS{J} ausgeführt (siehe \IT{branch delay slots}) und ist
ein Teil des Funktions-Epilogs.
Hier ist die Ausgabe, die \IDA generiert. Jedes Register hat einen eigenen Pseudo-Namen:

\lstinputlisting[caption=\Optimizing GCC 4.4.5 (\IDA),numbers=left,style=customasm]{patterns/01_helloworld/MIPS/hw_O3_IDA_DE.lst}

Die Anweisung in Zeile 15 speichert den GP-Wert auf dem lokalen Stack. Diese Anweisung fällt seltsamerweise
beim GCC, was vielleicht auf einen Fehler des Compilers hinweist. \footnote{Anscheinend sind Funktionen die
Listings erzeugen nicht so kritisch für GCC-Nutzer, so dass vielleicht noch unbehobene Fehler existieren.}.
Der GP-Wert muss auch gespeichert werden weil jede Funktion ihren eigenen 64KiB-Datenbereich nutzen kann.
Das Register mit der Adresse von \puts wird \$T9, da Register mit Präfix T- temporäre Register sind deren
Inhalte nicht erhalten werden müssen.

\subsubsection{\NonOptimizing GCC}

\NonOptimizing GCC ist ausführlicher.

\lstinputlisting[caption=\NonOptimizing GCC 4.4.5 (\assemblyOutput),numbers=left,style=customasm]{patterns/01_helloworld/MIPS/hw_O0_DE.s}

Es ist zu sehen, dass das FP-Register als Zeiger zum Stack Frame genutzt wird.
Außerdem sind im Listing drei \ac{NOP}-Anweisungen.
Die zweite und dritte welche der Sprunganweisung folgt.
Möglicherweise fügt der GCC-Compiler immer \ac{NOP}-Anweisungen nach einer Sprung hinzu
(wegen der \IT{branch delay slots}) und entfernt diese wenn die Optimierung eingeschaltet ist.
In diesem Fall bleiben sie also bestehen.

Nachfolgend das \IDA-Listing:

\lstinputlisting[caption=\NonOptimizing GCC 4.4.5 (\IDA),numbers=left,style=customasm]{patterns/01_helloworld/MIPS/hw_O0_IDA_DE.lst}

\myindex{MIPS!\Pseudoinstructions!LA}

Interessanterweise kennt \IDA das Anweisungspaar \INS{LUI}/\INS{ADDIU} und fasst diese zu einer einzigen
Pseudoanweisung \INS{LA} (\q{Load Address}) zusammen (Zeile 15).
Es ist auch zu sehen, dass diese Pseudoanweisung eine Größe von 8 Byte hat!
Dies ist eine Pseudoanweisung (oder \IT{Makro}) weil es sich hier nicht um eine echte MIPS-Anweisung
handelt, sondern eher um einen handlichen Namen für ein Anweisungspaar.

\myindex{MIPS!\Pseudoinstructions!NOP}
\myindex{MIPS!\Instructions!OR}

Eine weitere Sache ist, dass \IDA keine \ac{NOP}-Anweisung kennt.
Also ist in den Zeilen 22, 26 und 41 \TT{OR \$AT, \$ZERO}.
Im Wesentlichen führt diese Anweisung eine ODER-Operation auf die Inhalte des \$AT-Register aus,
welche 0 ist. Dies entspricht natürlich einer Idle-Anweisung.
MIPS hat wie viele andere \ac{ISA} keine separate \ac{NOP}-Anweisung.

\subsubsection{Aufgabe des Stack Frames in diesem Beispiel}

Die Adresse dieser Zeichenkette ist in einem Register übergeben.
Warum wird dennoch der lokale Stack vorbereitet?
Der Grund dafür liegt in der Tatsache, dass die Werte der Register \ac{RA} und GP
wegen des Aufrugs von \printf irgendwo gesichert werden müssen und hier eben der
lokale Stack dafür genutzt wird.
Wenn dies eine \gls{leaf function} wäre, bestünde die Möglichkeit den Funktionsepilog
und -prolog wegzulassen, wie hier: \myref{MIPS_leaf_function_ex1}.

\subsubsection{\Optimizing GCC: in GDB laden}

\myindex{GDB}
\lstinputlisting[caption=sample GDB session]{patterns/01_helloworld/MIPS/O3_GDB.txt}
}
\FR{\subsection{MIPS}

\subsubsection{Un mot à propos du \q{pointeur global}}
\label{MIPS_GP}

\myindex{MIPS!\GlobalPointer}

Un concept MIPS important est le \q{pointeur global}.
Comme nous le savons déjà, chaque instruction MIPS a une taille de 32bits, donc
il est impossible d'avoir une adresse 32-bit dans une instruction: il faut pour
cela utiliser une paire.
(comme le fait GCC dans notre exemple pour le chargement de l'adresse de la chaîne
de texte).
Il est possible, toutefois, de charger des données depuis une adresse dans l'interval
$register-32768...register+32767$ en utilisant une seule instruction (car un offset
signé de 16 bits peut être encodé dans une seule instruction).
Nous pouvons alors allouer un registre dans ce but et dédier un block de 64KiB
pour les données les plus utilisées.
Ce registre dédié est appelé un \q{pointeur global} et il pointe au milieu du
block de 64 KiB.
Ce block contient en général les variables globales et les adresses des fonctions
importées, comme \printf, car les développeurs de GCC ont décidé qu'obtenir
l'adresse d'une fonction devait se faire en une instruction au lieu de deux.
Dans un fichier ELF ce block de 64KiB se trouve en partie dans une section .sbss
(\q{small \ac{BSS}}) pour les données non initialisées et .sdata (\q{small data})
pour celles initialisées.
Cela implique que le programmeur peut choisir quelle donnée il/elle souhaite rendre
accesible rapidement et doit les stocker dans .sdata/.sbss.
Certains programmeurs old-school peuvent se souvenir du modèle de mémoire MS-DOS
\myref{8086_memory_model} ou des gestionnaires de mémoire MS-DOS comme XMS/EMS
où toute la mémoire était divisée en bloc de 64KiB.

\myindex{PowerPC}

Ce concept n'est pas restreint à MIPS. Au moins les PowerPC utilisent aussi cette
technique.

\subsubsection{GCC \Optimizing}

Considérons l'exemple suivant, qui illustre le concept de \q{pointeur global}.

\lstinputlisting[caption=GCC 4.4.5 \Optimizing (\assemblyOutput),numbers=left,style=customasmMIPS]{patterns/01_helloworld/MIPS/hw_O3_FR.s}

Comme on le voit, le registre \$GP est défini dans le prologue de la fonction
pour pointer au milieu de ce block.
Le registre \ac{RA} est sauvé sur la pile locale.
\puts est utilisé ici au lieu de \printf.
\myindex{MIPS!\Instructions!LW}
L'adresse de la fonction \puts est chargée dans \$25 en utilisant l'instruction \INS{LW} (\q{Load Word}).
\myindex{MIPS!\Instructions!LUI}
\myindex{MIPS!\Instructions!ADDIU}
Ensuite l'adresse de la chaîne de texte est chargée dans \$4 avec la paire
d'instructions \INS{LUI} ((\q{Load Upper Immediate}) et \INS{ADDIU}
(\q{Add Immediate Unsigned Word}).
\INS{LUI} défini les 16 bits de poids fort du registre (d'où le mot \q{upper}
dans le nom de l'instruction) et \INS{ADDIU} ajoute les 16 bits de poids faible
de l'adresse.

\INS{ADDIU} suit \INS{JALR} (vous n'avez pas déjà oublié le \IT{slot de
retard de branchement} ?).
Le registre \$4 est aussi appelé \$A0, qui est utilisé pour passer le premier
argument d'une fonction \footnote{La table des registres MIPS est disponible en
appendice \myref{MIPS_registers_ref}}.

\myindex{MIPS!\Instructions!JALR}

\INS{JALR} (\q{Jump and Link Register}) saute à l'adresse stockée dans le registre
\$25 (adresse de \puts) en sauvant l'adresse de la prochaine instruction (LW)
dans \ac{RA}.
C'est très similaire à ARM.
Oh, encore une chose importante, l'adresse sauvée dans \ac{RA} n'est pas
l'adresse de l'instruction suivante (car c'est celle du \IT{slot de délai} et
elle est exécutée avant l'instruction de saut), mais l'adresse de l'instruction
après la suivante (après le \IT{slot de délai}).
Par conséquent, $PC + 8$ est écrit dans \ac{RA} pendant l'exécution de \TT{JALR},
dans notre cas, c'est l'adresse de l'instruction \INS{LW} après \INS{ADDIU}.

\INS{LW} (\q{Load Word}) à la ligne 20 restaure \ac{RA} depuis la pile locale
(cette instruction fait partie de l'épilogue de la fonction).

\myindex{MIPS!\Pseudoinstructions!MOVE}

\INS{MOVE} à la ligne 22 copie la valeur du registre \$0 (\$ZERO) dans \$2 (\$V0).
\label{MIPS_zero_register}

MIPS a un registre \IT{constant}, qui contient toujours zéro.
Apparemment, les développeurs de MIPS avaient à l'esprit que zéro est la constante
la plus utilisée en programmation, utilisons donc le registre \$0 à chaque fois
que zéro est requis.

Un autre fait intéressant est qu'il manque en MIPS une instruction qui transfère
des données entre des registres.
En fait,  \TT{MOVE DST, SRC} est \TT{ADD DST, SRC, \$ZERO} ($DST=SRC+0$), qui
fait la même chose.
Apparemment, les développeurs de MIPS voulaient une table des opcodes compacte.
Cela ne siginifie pas qu'il y a une addition à chaque instruction \INS{MOVE}.
Très probablement, le \ac{CPU} optimise ces pseudo instructions et l'\ac{ALU} n'est
jamais utilisé.

\myindex{MIPS!\Instructions!J}

\INS{J} à la ligne 24 saute à l'adresse dans \ac{RA}, qui effectue effectivement
un retour de la fonction.
\INS{ADDIU} après \INS{J} est en fait exécutée avant \INS{J} (vous vous rappeler
du \IT{slot de délai de branchement}?) et fait partie de l'épilogue de la fonction.
Voici un listing généré par \IDA. Chaque registre a son propre pseudo nom:

\lstinputlisting[caption=GCC 4.4.5 \Optimizing (\IDA),numbers=left,style=customasmMIPS]{patterns/01_helloworld/MIPS/hw_O3_IDA_FR.lst}

L'instruction à la ligne 15 sauve la valeur de GP sur la pile locale, et cette
instruction manque mystérieusement dans le listing de sortie de GCC, peut-être
une erreur de GCC
\footnote{Apparamment, les fonctions générant les listings ne sont pas si critique
pour les utilisateurs de GCC, donc des erreurs peuvent toujours subsister.}.
La valeur de GP doit effectivement être sauvée, car chaque fonction utilise sa
propre fenêtre de 64KiB.
Le registre contenant l'adresse de \puts est appelé \$T9, car les registres
préfixés avec T- sont appelés \q{temporaires} et leur contenu ne doit pas être
préservé. 

\subsubsection{GCC \NonOptimizing}

GCC \NonOptimizing est plus verbeux.

\lstinputlisting[caption=GCC 4.4.5 \NonOptimizing (\assemblyOutput),numbers=left,style=customasmMIPS]{patterns/01_helloworld/MIPS/hw_O0_FR.s}

Nous voyons ici que le registre FP est utilisé comme un pointeur sur la pile.
Nous voyons aussi 3 \ac{NOP}s.
Le second et le troisième suivent une instruction de branchement.
Peut-être que le compilateur GCC ajoute toujours des \ac{NOP}s (à cause du
\IT{slot de retard de branchement}) après les instructions de branchement, et
alors, si l'optimisation est demandée, peut-être qu'il les élimine.
Donc, dans ce cas, ils sont laissés en place.

Voici le listing \IDA:

\lstinputlisting[caption=GCC 4.4.5 \NonOptimizing (\IDA),numbers=left,style=customasmMIPS]{patterns/01_helloworld/MIPS/hw_O0_IDA_FR.lst}

\myindex{MIPS!\Pseudoinstructions!LA}

Intéressant, \IDA a reconnu les instructions \INS{LUI}/\INS{ADDIU} et les a concaténées
en une pseudo instruction \INS{LA} (\q{Load Address}) à la ligne 15.
Nous pouvons voir que cette pseudo instruction a une taille de 8 octets!
C'est une pseudo instruction (ou \IT{macro}) car ce n'est pas une instruction MIPS
réelle, mais plutôt un nom pratique pour une paire d'instructions.

\myindex{MIPS!\Pseudoinstructions!NOP}
\myindex{MIPS!\Instructions!OR}

Une autre chose est qu'\IDA ne reconnait pas les instructions \ac{NOP}, donc ici
elles se trouvent aux lignes 22, 26 et 41.
C'est \TT{OR \$AT, \$ZERO}.
Essentiellement, cette instruction applique l'opération OR au contenu du registre
\$AT avec zéro, ce qui, bien sûr, est une instruction sans effet.
MIPS, comme beaucoup d'autres \ac{ISA}s, n'a pas une instruction \ac{NOP}.

\subsubsection{Rôle de la pile dans cet exemple}

L'adresse de la chaîne de texte est passée dans le registre.
Pourquoi définir une pile locale quand même?
La raison de cela est que la valeur des registres \ac{RA} et GP doit être sauvée
quelque part (car \printf est appelée), et que la pile locale est utilisée pour cela.
Si cela avait été une \glslink{leaf function}{fonction leaf}, il aurait été
possible de se passer du prologue et de l'épilogue de la fonction, par
exemple: \myref{MIPS_leaf_function_ex1}.

\subsubsection{GCC \Optimizing: chargeons-le dans GDB}

\myindex{GDB}
\lstinputlisting[caption=extrait d'une session GDB]{patterns/01_helloworld/MIPS/O3_GDB.txt}

}
\PL{\subsection{MIPS}

\subsubsection{O \q{wskaźniku globalnym} (\q{global pointer})}
\label{MIPS_GP}

\myindex{MIPS!\GlobalPointer}
\q{Wskaźnik globalny} (\q{global pointer})~--- jest bardzo ważną koncepcją MIPS.
Jak już wiemy, każda instrukcja w MIPS ma długość 32 bity, dlatego kodowanie 32-bitowego adresu w jednej instrukcji nie jest możliwe. Zamiast tego trzeba wykorzystać parę instrukcji
(jak to zrobił GCC dla załadowania adresu linii tekstowej w naszym przykładzie).
Z innej strony, korzystając tylko z jednej instrukcji, 
można ładować dane do adresów w granicach $register-32768...register+32767$, dlatego że 16 bitów
przesunięcia można zakodować w jednej instrukcji).
Także możemy przydzielić jakiś rejestr do tych celów i jeszcze bufor 64KiB dla najczęściej wykorzystywanych danych.
Przydzielony rejestr nazywamy \q{wskaźnikiem globalnym} (\q{global pointer}) i wskazuje on na środek buforu 64KiB.
Ten bufor zwykle zawiera zmienne globalne oraz adresy funkcji importowanych typu \printf,
dlatego że deweloperzy GCC stwierdzili, że dostęp do adresu funkcji musi być jak najszybszy,
wykonywany jako jedna instrukcja zamiast dwóch.
W ELF-pliku ten 64KiB-bufor znajduje się po części w segmencie .sbss (\q{small \ac{BSS}}) dla danych niezainicjalizowanych i w segmencie .sdata (\q{small data}) dla danych zainicjalizowanych.
To oznacza, że programista może wybierać, do czego potrzebuje szybszego dostępu, i następnie rozmieścić to
w segmentach .sdata/.sbss.
Niektórzy \q{old-school} programiści mogą pamiętać model pamięci w MS-DOS \myref{8086_memory_model} 
lub w menadżerach pamięci typu XMS/EMS, gdzie cała pamięć była podzielona na bloki po 64KiB długości.

\myindex{PowerPC}
Ta koncepcja jest stosowana nie tylko w MIPS. Przynajmniej PowerPC również korzysta z tej techniki.

\subsubsection{\Optimizing GCC}

Popatrzmy na następujący przykład, realizujący koncepcję \q{wskaźnika globalnego}.

\lstinputlisting[caption=\Optimizing GCC 4.4.5 (\assemblyOutput),numbers=left,style=customasmMIPS]{patterns/01_helloworld/MIPS/hw_O3_PL.s}

Jak widać, rejestr \$GP w prologu funkcji wystawia się na środek tego obszaru.
Rejestr \ac{RA} jest odkładany na stos lokalny.
Tutaj kompilator również skorzystał z \puts zamiast \printf.
\myindex{MIPS!\Instructions!LW}
Adres funkcji \puts jest ładowany do \$25 za pomocą instrukcji \INS{LW} (\q{Load Word}).
\myindex{MIPS!\Instructions!LUI}
\myindex{MIPS!\Instructions!ADDIU}
Następnie adres linii tekstowej jest ładowany do \$4 za pomocą pary instrukcji \INS{LUI} (\q{Load Upper Immediate}) i
\INS{ADDIU} (\q{Add Immediate Unsigned Word}).
\INS{LUI} ustawia starsze 16 bitów rejestru (dlatego w nazwie jest obecne \q{upper}) i \INS{ADDIU}
dodaje młodsze 16 bitów do adresu.
\INS{ADDIU} idzie za \INS{JALR} (pamiętając o \IT{branch delay slots}).
Rejestr \$4 także jest nazywany \$A0, który jest wykorzystywany do przekazywania pierwszego argumentu funkcji
\footnote{Tablica rejestrów MIPS jest dostępna w dodatku \myref{MIPS_registers_ref}}.
\myindex{MIPS!\Instructions!JALR}
\INS{JALR} (\q{Jump and Link Register}) robi przejście wg adresu zawartego w \$25 (tam się znajduje adres \puts) 
jednocześnie zapisując adres następnej instrukcji (\INS{LW}) w \ac{RA}.
Wygląda to tak samo jak i w ARM.
I jeszcze jedna bardzo ważna rzecz: adres zapisywany do \ac{RA} nie jest adresem następnej instrukcji (dlatego, że jest to
\IT{delay slot} i wykonuje się przed instrukcją przejścia),
a instrukcji następującej po \IT{delay slot}.
W ten sposób podczas wykonania \INS{JALR} do \ac{RA} jest zapisywane $PC + 8$.
W naszym wypadku jest to adres instrukcji \INS{LW} kolejnej po \INS{ADDIU}.

\INS{LW} (\q{Load Word}) w linii 20 przywraca \ac{RA} ze stosu lokalnego (ta instrukcja jest raczej częścią epilogu f-cji).

\myindex{MIPS!\Pseudoinstructions!MOVE}
\INS{MOVE} w linii 22 kopiuje wartość z \$0 (\$ZERO) do \$2 (\$V0).

\label{MIPS_zero_register}
W MIPS istnieje rejestr dla stałych, zawsze zawierający zero.
Możliwe, że deweloperzy MIPS stwierdzili, że 0 jest najbardziej zapotrzebowaną stałą w programowaniu,
także niech rejestr \$0, będzie wykorzystywany za każdym razem, kiedy będzie potrzebne 0.
Inna ciekawostka: w MIPS nie ma instrukcji, kopiującej wartość z rejestru do rejestru.
Naprawdę, \TT{MOVE DST, SRC} to \TT{ADD DST, SRC, \$ZERO} ($DST=SRC+0$), która robi to samo.
Najwidoczniej, twórcy MIPS chcieli stworzyć jak najbardziej zwięzłą tablicę opcode'ów.
To wcale nie znaczy, że dodawanie jest wykonywane podczas każdej instrukcji \INS{MOVE}.
Prawdopodobnie, te pseudo-instrukcje są optymalizowane w \ac{CPU} i \ac{ALU} nigdy nie jest wykorzystywane.

\myindex{MIPS!\Instructions!J}
\INS{J} w linii 24 robi przejście pod adres w \ac{RA}, i to działa jako wyjście z funkcji.
\INS{ADDIU} po \INS{J} wprawdzie jest wykonywane przed \INS{J} (\IT{branch delay slots}?) 
jest to częścią epilogu funkcji.

To jest listing wygenerowany w \IDA. Każdy rejestr posiada swoją pseudonazwę:

\lstinputlisting[caption=\Optimizing GCC 4.4.5 (\IDA),numbers=left,style=customasmMIPS]{patterns/01_helloworld/MIPS/hw_O3_IDA_PL.lst}

Instrukcja w linii 15 odkłada GP na lokalny stos. Ta instrukcja w dziwny sposób nie jest obecna na listingu GCC, możliwe, że jest to spowodowane błędem w samym GCC\footnote{Najwidoczniej, f-cja wygenerowania listingów nie jest krytyczna
dla użytkowników GCC, dlatego tam mogą być obecne niepoprawione błędy.}.
Wartość GP musi być zapisana, dlatego że każda funkcjamoże pracować ze swoim własnym obszarem danych o długości 64KiB.
Rejestr, zawierający adres \puts to \$T9, dlatego że rejestry z prefiksem T- nazywają się
\q{temporaries} i ich zawartości można nie zapisywać.

\subsubsection{\NonOptimizing GCC}

\NonOptimizing GCC generuje większy objętościowo kod.

\lstinputlisting[caption=\NonOptimizing GCC 4.4.5 (\assemblyOutput),numbers=left,style=customasmMIPS]{patterns/01_helloworld/MIPS/hw_O0_PL.s}

Widzimy, że FP jest wykorzystywany jako frame pointer.
Również widzimy 3 instrukcje \ac{NOP}.
Drugi i trzeci \ac{NOP} występują po instrukcjach skoku.
Najwidoczniej, kompilator GCC zawsze dodaje \ac{NOP}-y (przez \IT{branch delay slots})
po instrukcjach skoku i następnie, jeśli optymalizacja jest włączona, może się od nich pozbywać.
Także one tutaj zostały.

Jeszcze jeden listing w \IDA:

\lstinputlisting[caption=\NonOptimizing GCC 4.4.5 (\IDA),numbers=left,style=customasmMIPS]{patterns/01_helloworld/MIPS/hw_O0_IDA_PL.lst}

\myindex{MIPS!\Pseudoinstructions!LA}
Ciekawe, że \IDA zebrała parę instrukcji \INS{LUI}/\INS{ADDIU} w jedną pseudoinstrukcję 
\INS{LA} (\q{Load Address}) w linii 15.
Również widzimy, że długość tej pseudoinstrukcji to 8 bajt!
Jest to pseudoinstrukcja (lub \IT{makro}), dlatego że nie jest to typowa instrukcja MIPS, a raczej wygodna nazwa zbiorowa dla tych dwóch instrukcji.

\myindex{MIPS!\Pseudoinstructions!NOP}
\myindex{MIPS!\Instructions!OR}
Jeszcze coś: \IDA nie rozpoznała \ac{NOP}-instrukcje w liniach 22, 26 i 41.

Jest to \TT{OR \$AT, \$ZERO}.
Generalnie, jest to instrukcja \IT{LUB} stosowana do zawartości rejestru \$AT z zerem,
co, oczywiście, jest pustą instrukcją.
MIPS, jak i inne \ac{ISA}, nie posiada oddzielnej \ac{NOP}-instrukcji.

\subsubsection{Rola stack frame w tym przykładzie}

Adres linii tekstowej jest przekazywany przez rejestr.
Po co w takim razie ustawiać stos lokalny?
Dzieje się tak dlatego, że wartości rejestrów \ac{RA} i GP muszą być gdzieś zapisane
(dlatego że jest wywoływane \printf) i po to się korzysta ze stosu lokalnego.

Gdyby to była \gls{leaf function}, to wtedy można by było pozbyć prologu i epilogu funkcji. Na przykład:
 \myref{MIPS_leaf_function_ex1}.

\subsubsection{\Optimizing GCC: załadujemy do GDB}

\myindex{GDB}
\lstinputlisting[caption=przykład sesji w GDB]{patterns/01_helloworld/MIPS/O3_GDB.txt}


}
\JPN{\subsection{MIPS}

\subsubsection{\q{グローバルポインタ}について少し}
\label{MIPS_GP}

\myindex{MIPS!\GlobalPointer}

1つの重要なMIPSコンセプトは、\q{グローバルポインタ}です。
既にわかっているように、各MIPS命令のサイズは32ビットなので、32ビットアドレスを1つの命令に組み込むことは不可能です
この例ではGCCのように対を使用しなければなりません読み込み)。
ただし、1つの命令を使用してレジスタ$register-32768...register+32767$の範囲のアドレスからデータをロードすることは可能です
(16ビットの符号付きオフセットを1つの命令でエンコードできるため)。
したがって、この目的のためにいくつかのレジスタを割り当てて、最も多く使用されているデータの64KiB領域を割り当てることができます。
この割り当てられたレジスタは "グローバルポインタ"と呼ばれ、64KiB領域の中央を指します。
この領域には通常、 \printf のようなインポートされた関数のグローバル変数とアドレスが含まれています。
なぜなら、GCCの開発者は、関数のアドレスを得ることは2つではなく1つの命令の実行と同じくらい速くなければならないと判断したからです。 
ELFファイルでは、この64KiB領域は初期化されていないデータの場合は.sbss(\q{small \ac{BSS}})、
初期化されたデータの場合は.sdata( \q{small data})のセクションに部分的に配置されています。
これはプログラマがどのデータを高速にアクセスしたいのかを選択して.sdata / .sbssに入れることを意味します。
いくつかの古い学校のプログラマは、MS-DOSメモリモデル\myref{8086_memory_model}、またはすべてのメモリが64KiBブロックに分割されたXMS / EMSのようなMS-DOSメモリマネージャ。

\myindex{PowerPC}

この概念はMIPS特有のものではありません。少なくともPowerPCはこの手法も使用しています。

\subsubsection{\Optimizing GCC}

グローバルポインタの概念を示す次の例を考えてみましょう。

\lstinputlisting[caption=\Optimizing GCC 4.4.5 (\assemblyOutput),numbers=left,style=customasmMIPS]{patterns/01_helloworld/MIPS/hw_O3_EN.s}

我々が見るように、\$GPレジスタは関数のプロローグでこの領域の中央を指すように設定されています。 
\ac{RA}レジスタもローカルスタックに保存されます。 
\printf の代わりに \puts もここで使用されます。 
\myindex{MIPS!\Instructions!LW}

\puts 関数のアドレスは、\INS{LW}命令(\q{Load Word})を使用して \$25 にロードされます。 
\myindex{MIPS!\Instructions!LUI}
\myindex{MIPS!\Instructions!ADDIU}
\INS{LUI}(\q{Load Upper Immediate})と\INS{ADDIU}(\q{Add Immediate Unsigned Word})命令のペアを使用して、テキスト文字列のアドレスが\$4にロードされます。 
\INS{LUI}はレジスタの上位16ビット(したがって\q{命令名の上位ワード})を設定し、\INS{ADDIU}はアドレスの下位16ビットを加算します。

\INS{ADDIU}は\INS{JALR}に従います(まだ\IT{ブランチ遅延スロット}を覚えていますか?)。
レジスタ \$4 は \$A0 とも呼ばれ、最初の関数引数を渡すために使用されます。
\footnote{MIPSレジスタの表はappendixで見られます:\myref{MIPS_registers_ref}}.

\myindex{MIPS!\Instructions!JALR}

\INS{JALR}(\q{Jump and Link Register})は、\ac{RA}の次の命令(LW)のアドレスを保存している間、
\$25 レジスタ( \puts のアドレス)に格納されているアドレスにジャンプします。
これはARMと非常によく似ています。
ああ、重要なことの1つは、RAに保存されたアドレスは、次の命令のアドレスではないことです。
(\IT{遅延スロット}にあり、ジャンプ命令の前に実行されるため)
したがって、 $PC + 8$ は\TT{JALR}の実行中に\ac{RA}に書き込まれます。私たちの場合、これは\INS{ADDIU}の次の\INS{LW}命令のアドレスです。

20行目の\INS{LW} (\q{Load Word})は、ローカルスタックから\ac{RA}を復元します(この命令は実際には関数のエピローグの一部です)。

22行目の\INS{MOVE}は、\$0(\$ZERO)レジスタから\$2(\$V0)までの値をコピーします。
\label{MIPS_zero_register}

MIPSは\IT{定数}レジスタを持ち、常に0を保持します。
どうやら、MIPSの開発者たちは、実際にはゼロがコンピュータプログラミングで最も忙しいという考えを思いついたので、
ゼロが必要なたびに\$0レジスタを使用しましょう。

もう1つの興味深い事実は、MIPSにレジスタ間でデータを転送する命令がないことです。 
実際、\TT{MOVE DST, SRC}は\TT{ADD DST, SRC, \$ZERO} ($DST=SRC+0$)です。これは同じです。 
明らかに、MIPS開発者はコンパクトなopcodeテーブルを用意したいと考えました。 
これは、各\INS{MOVE}命令で実際の加算が行われることを意味しません。 
ほとんどの場合、\ac{CPU}はこれらの疑似命令を最適化し、\ac{ALU}は決して使用されません。

\myindex{MIPS!\Instructions!J}

24行目の\INS{J}は、\ac{RA}のアドレスにジャンプします。これは、関数からの戻り値を効果的に実行しています。 
\INS{J}の後の\INS{ADDIU}は実際に\INS{J}の前に実行されます(\IT{分岐遅延スロット}を覚えていますか?)。そして関数のエピローグの一部です。 
ここに \IDA によって生成されたリストもあります。 ここの各レジスタには、独自の擬似名があります。

\lstinputlisting[caption=\Optimizing GCC 4.4.5 (\IDA),numbers=left,style=customasmMIPS]{patterns/01_helloworld/MIPS/hw_O3_IDA_EN.lst}

15行目の命令は、GPの値をローカルスタックに保存します。この命令は、GCCの出力リストから不思議に見えます.GCCのエラーがあります
\footnote{明らかに、リストを生成する関数はGCCユーザーにとってあまり重要ではないので、修正されていないエラーがまだ存在するかもしれません}
GPの値は実際に保存しなければなりません。 データウィンドウ。 \puts アドレスを含むレジスタは\$T9と呼ばれ、
T-が前に付いたレジスタは\q{一時的}と呼ばれ、その内容は保持されない可能性があるためです。

\subsubsection{\NonOptimizing GCC}

\NonOptimizing GCCはもっと冗長です。

\lstinputlisting[caption=\NonOptimizing GCC 4.4.5 (\assemblyOutput),numbers=left,style=customasmMIPS]{patterns/01_helloworld/MIPS/hw_O0_EN.s}

レジスタFPはスタックフレームへのポインタとして使用されることがわかります。 
3つの\ac{NOP}も見てみましょう。 
2番目と3番目は分岐命令に従います。 
おそらく、GCCコンパイラは分岐命令の後に常に\IT{分岐遅延スロット}のために\ac{NOP}を追加し、
最適化がオンになっていればそれらを削除するかもしれません。 
したがって、この場合、それらはここに残されます。

\IDA のリストもあります:

\lstinputlisting[caption=\NonOptimizing GCC 4.4.5 (\IDA),numbers=left,style=customasmMIPS]{patterns/01_helloworld/MIPS/hw_O0_IDA_EN.lst}

\myindex{MIPS!\Pseudoinstructions!LA}

興味深いことに、 \IDA は\INS{LUI}/\INS{ADDIU}命令のペアを認識し、15行目の1つの
\INS{LA}(\q{Load Address})疑似命令に統合しました。
この疑似命令のサイズは8バイトです。 
これは実際のMIPS命令ではなく、むしろ命令対のための便利な名前であるため、疑似命令(または\IT{マクロ})です。

\myindex{MIPS!\Pseudoinstructions!NOP}
\myindex{MIPS!\Instructions!OR}

もう1つのことは、 \IDA は\ac{NOP}命令を認識していないことです。22行目、26行目、41行目です。
\TT{OR \$AT, \$ZERO}です。 基本的に、この命令は、 \$AT レジスタの内容にOR演算を0(もちろんアイドル命令)で適用します。 
MIPSは他の多くの\ac{ISA}と同様に、独立した\ac{NOP}命令を持っていません。

\subsubsection{スタックフレームの役割}

テキスト文字列のアドレスはレジスタに渡されます。 
とにかくローカルスタックをセットアップする理由は? 
これは、 \printf が呼び出されるため、レジスタ\ac{RA}とGPの値をどこかに保存する必要があり、
ローカルスタックがこの目的のために使用されているという事実にあります。 
これが \gls{leaf function} であれば、関数のプロローグとエピローグを取り除くことができました。
例:\myref{MIPS_leaf_function_ex1}

\subsubsection{\Optimizing GCC: GDBにロードしてみる}

\myindex{GDB}
\lstinputlisting[caption=sample GDB session]{patterns/01_helloworld/MIPS/O3_GDB.txt}

}


\subsection{\Conclusion{}}

La differenza principale tra il codice x86/ARM e x64/ARM64 è che il puntatore alla stringa è adesso lungo 64 bit.
Infatti, le moderne \ac{CPU} sono ora a 64-bit grazie ai costi ridotti della memoria e alla sua grande richiesta da parte delle applicazioni moderne. 
Possiamo aggiungere ai nostri computer più memoria di quanto i puntatori a 32-bit siano in grado di indirizzare.  
Di conseguenza, tutti i puntatori sono oggi a 64-bit.

% sections
\subsection{\Exercise}

\begin{itemize}
	\item \url{http://challenges.re/27}
\end{itemize}


