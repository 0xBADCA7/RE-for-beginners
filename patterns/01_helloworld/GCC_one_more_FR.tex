\subsection{GCC---encore une chose}
\label{use_parts_of_C_strings}

Le fait est qu'une chaîne C \IT{anonyme} a un type \IT{const} (\myref{string_is_const_char}),
et que les chaînes C allouées dans le segment des constantes sont garanties d'être immuables, ce qui a pour
conséquence:
Le compilateur peut utiliser une partie spécifique de la chaîne.

Voyons cela avec un exemple:

\begin{lstlisting}[style=customc]
#include <stdio.h>

int f1()
{
	printf ("world\n");
}

int f2()
{
	printf ("hello world\n");
}

int main()
{
	f1();
	f2();
}
\end{lstlisting}

La plupart des compilateurs \CCpp{} (incluant MSVC) allouent deux chaînes, mais voyons ce que fait GCC 4.8.1:

\begin{lstlisting}[caption=GCC 4.8.1 + IDA listing,style=customasmx86]
f1              proc near

s               = dword ptr -1Ch

                sub     esp, 1Ch
                mov     [esp+1Ch+s], offset s ; "world\n"
                call    _puts
                add     esp, 1Ch
                retn
f1              endp

f2              proc near

s               = dword ptr -1Ch

                sub     esp, 1Ch
                mov     [esp+1Ch+s], offset aHello ; "hello "
                call    _puts
                add     esp, 1Ch
                retn
f2              endp

aHello          db 'hello '
s               db 'world',0xa,0
\end{lstlisting}

Effectivement: lorsque nous affichons la chaîne \q{hello world} ses deux mots sont positionnés
consécutivement en mémoire et l'appel à \puts depuis la fonction \GTT{f2()}
n'est pas au courant que la chaîne est divisée.
En fait, elle n'est pas divisée; elle l'est \q{virtuellement}, dans ce listing.

Lorsque \puts est appelé depuis \GTT{f1()}, il utilise la chaîne \q{world} ainsi qu octet à zéro. \puts ne sait pas
qu'il y a quelque chose avant cette chaîne!

Cette astuce est souvent utilisée, au moins par GCC, et permet d'économiser de la mémoire.
C'est proche de \IT{string interning}. % TODO clarify imbrication de chaînes ?

Un autre exemple concernant ceci est là: \myref{strstr_example}.

