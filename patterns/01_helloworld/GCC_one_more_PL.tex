\subsection{GCC --- jeszcze co nieco}
\label{use_parts_of_C_strings}

To, że \IT{anonimowa} linia w C ma typ \IT{const} (\myref{string_is_const_char}), 
i to, że zaznaczone w segmencie stałych C-linie są gwarantowanie nie do zmiany (immutable), 
prowadzi do ciekawych konsekwencji: kompilator może korzystać tylko z pewnej części linii.

Prosty przykład:

\begin{lstlisting}[style=customc]

#include <stdio.h>

int f1()
{
	printf ("world\n");
}

int f2()
{
	printf ("hello world\n");
}

int main()
{
	f1();
	f2();
}
\end{lstlisting}

Typowy kompilator \CCpp (w tym MSVC) przydzieli miejsce dla 2 linii, ale o to jest to co robi GCC 4.8.1:

\begin{lstlisting}[caption=GCC 4.8.1 + listing w IDA,style=customasmx86]
f1              proc near

s               = dword ptr -1Ch

                sub     esp, 1Ch
                mov     [esp+1Ch+s], offset s ; "world\n"
                call    _puts
                add     esp, 1Ch
                retn
f1              endp

f2              proc near

s               = dword ptr -1Ch

                sub     esp, 1Ch
                mov     [esp+1Ch+s], offset aHello ; "hello "
                call    _puts
                add     esp, 1Ch
                retn
f2              endp

aHello          db 'hello '
s               db 'world',0xa,0
\end{lstlisting}

Naprawdę, kiedy printujemy linię \q{hello world}, 
te dwa słowa są usytuowane w pamięci tuż obok siebie i \puts, będąc wywołane z funkcji \GTT{f2()}, w ogóle nie wie,
że te linię są rozdzielone. Tak naprawdę to one i nie są rozdzielone, są rozdzielone 
tylko \q{wirtualnie}, w naszym listingu.

Kiedy \puts zostaje wywołane z \GTT{f1()}, ono wykorzystuje linie \q{world} + zerowy bajt. \puts w ogóle nie wie, że tam jest jeszcze jakaś linia przed tą!

Ten trik jest często wykorzystywany (przynajmniej w GCC) i może zaoszczędzić trochę pamięci.
Jest to podobny mechanizm do mechanizmu \IT{internowania łańcuchów}.

Jeszcze jeden przykład można znaleźć tutaj: \myref{strstr_example}.


