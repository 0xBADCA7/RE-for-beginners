\subsubsection{Korekcja (patching) linii (Win32)}

Możemy w łatwy sposób znaleźć linię ``hello, world'' w pliku wykonywalnym za pomocą Hiew:

\begin{figure}[H]
\centering
\myincludegraphics{patterns/01_helloworld/hola_edit1.png}
\caption{Hiew}
\label{}
\end{figure}

Możemy przetłumaczyć naszą wiadomość na jeżyk hiszpański:

\begin{figure}[H]
\centering
\myincludegraphics{patterns/01_helloworld/hola_edit2.png}
\caption{Hiew}
\label{}
\end{figure}

Tekst w języku kiszpańskim jest o 1 bajt krótszy od tekstu w języku angielskim, dlatego dodajemy na koniec bajt 0x0A (\TT{\textbackslash{}n}) i bajt zerowy.
Działa.

A co jeśli chcemy wstawić dłuższe powiadomienie?
Po oryginalnym tekście w j. angielskim widzimy jakieś bajty zerowe.
Ciężko powiedzieć, czy można je nadpisać: mogą one być wykorzystywane gdzieś w \ac{CRT}-kodzie, a mogą i nie być.
Tak czy inaczej, możemy je nadpisywać tylko jeśli naprawdę wiemy co robimy.

\subsubsection{Korekcja linii (Linux x64)}

\myindex{\radare}
Spróbujmy spatchować plik wykonywalny dla Linux x64 korzystając z \radare{}:

\lstinputlisting[caption=Sesja w \radare{}]{patterns/01_helloworld/radare.lst}

Co się tu dzieje: szukam linii \q{hello} korzystając z komendy \TT{/}, 
zatem ustawiam \IT{kursor} (\IT{seek} w terminach \radare{}) pod ten adres.
Następnie chcę się upewnić, że jest to rzeczywiście potrzebne miejsce: \TT{px} wylistowuje bajty pod tym adresem.
\TT{oo+} przełącza \radare{} w tryb \IT{odczytu-zapisu}.
\TT{w} zapisuje ASCII-linię w miejscu kursora (\IT{seek}).
Warto zauważyć, iż \TT{\textbackslash{}00} na końcu jest bajtem zerowym.
\TT{q} kończy pracę.

\subsubsection{Lokalizacja oprogramowania za czasów MS-DOS}

Sposób przedstawiony wyżej był rozpowszechniony przy tłumaczeniu oprogramowania pod MS-DOS na język rosyjski w latach 80 i 90.
Rosyjskie słowa i zdania zwykle dą trochę dłuższe od odpowiedników w j. angielskim, dlatego \IT{lokalne} oprogramowanie zawierało 
sporo dziwnych akronimów i ciężkich w zrozumieniu skrótów.

Prawdopodobnie, sytuacja wyglądała podobnie z tłumaczeniem na inne języki.


