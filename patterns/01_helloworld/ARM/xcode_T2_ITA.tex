\subsubsection{\OptimizingXcodeIV (\ThumbTwoMode)}

Di default Xcode 4.6.3 genera codice per Thumb-2 in questo modo:

\begin{lstlisting}[caption=\OptimizingXcodeIV (\ThumbTwoMode),style=customasm]
__text:00002B6C                   _hello_world
__text:00002B6C 80 B5          PUSH            {R7,LR}
__text:00002B6E 41 F2 D8 30    MOVW            R0, #0x13D8
__text:00002B72 6F 46          MOV             R7, SP
__text:00002B74 C0 F2 00 00    MOVT.W          R0, #0
__text:00002B78 78 44          ADD             R0, PC
__text:00002B7A 01 F0 38 EA    BLX             _puts
__text:00002B7E 00 20          MOVS            R0, #0
__text:00002B80 80 BD          POP             {R7,PC}

...

__cstring:00003E70 48 65 6C 6C 6F 20+aHelloWorld  DCB "Hello world!",0xA,0
\end{lstlisting}

% Q: If you subtract 0x13D8 from 0x3E70,
% you actually get a location that is not in this function, or in _puts.
% How is PC-relative addressing done in THUMB2?
% A: it's not Thumb-related. there are just mess with two different segments. TODO: rework this listing.

\myindex{\ThumbTwoMode}
\myindex{ARM!\Instructions!BL}
\myindex{ARM!\Instructions!BLX}

Le istruzioni \TT{BL} e \TT{BLX} in Thumb mode, come ricordiamo, sono codificate con una coppia di istruzioni 16-bit.
In Thumb-2 questi opcode \IT{surrogati} sono estesi in modo tale che le nuove istruzioni possano essere codificate in istruzioni a 32-bit.

Cio' appare ovvio considerando che che gli opcodes delle istruzioni Thumb-2 iniziano sempre con \TT{0xFx} o \TT{0xEx}.

Ma nel listato \IDA 
i byte degli opcode sono invertiti poiche' per i processori ARM le istruzioni sono codificate secondo il seguente principio: 
l'ultimo byte viene prima ed e' seguito dal primo byte ( per le modalita' Thumb e Thumb-2 ) 
oppure, per istruzioni in ARM mode il quarto byte viene prima, seguito dal terzo, dal secondo ed infine dal primo (a causa
della diversa \gls{endianness}).

Quindi i byte nei listati IDA sono collocati cosi':
\begin{itemize}
\item per ARM and ARM64 modes: 4-3-2-1;
\item per Thumb mode: 2-1;
\item per coppie di istruzioni a 16-bit in Thumb-2 mode: 2-1-4-3.
\end{itemize}

\myindex{ARM!\Instructions!MOVW}
\myindex{ARM!\Instructions!MOVT.W}
\myindex{ARM!\Instructions!BLX}

Come possiamo vedere, le istruzioni \TT{MOVW}, \TT{MOVT.W} e \TT{BLX} iniziano con \TT{0xFx}.

Ona delle istruzioni Thumb-2 e' \TT{MOVW R0, \#0x13D8} ~---memorizza un valore a 16-bit nella parte bassa del registro \Reg{0} ,
azzerando i bit piu' alti.

Allo stesso modo, \TT{MOVT.W R0, \#0} ~funziona come \TT{MOVT} nel precedente esempio, ma in Thumb-2.

\myindex{ARM!mode switching}
\myindex{ARM!\Instructions!BLX}

Tra le altre differenze, l'istruzione \TT{BLX} in questo caso e' usata al posto di \TT{BL}.

La differenza sta nel fatto che, oltre a salvare \ac{RA} nel registro \ac{LR} e passare il controllo alla funzione \puts,
il processore passa dalla modalita' Thumb/Thumb-2 alla modalita' ARM mode (o viceversa).

Questa istruzione e' posta qui poiche' l'istruzione a cui il controllo viene passato appare cosi' (e' codificata in ARM mode):

\begin{lstlisting}[style=customasm]
__symbolstub1:00003FEC _puts           ; CODE XREF: _hello_world+E
__symbolstub1:00003FEC 44 F0 9F E5     LDR  PC, =__imp__puts
\end{lstlisting}

E' essenzialmente un jump alla zona dove l'indirizzo di \puts e' scritto nella imports section.

Il lettore attento potrebbe chiedere: perche' non chiamare \puts proprio nel punto del codice, dove serve effettivamente?

Perche' non e' efficiente in termini di spazio.

\myindex{Dynamically loaded libraries}
Quasi tutti i programmi utilizzano librerie esterne dinamiche (come le DLL in Windows, .so in *NIX o .dylib in \MacOSX).
Le librerie dinamiche contengono funzioni usate di frequente, inclusa la funzione C standard \puts.

\myindex{Relocation}
In un file eseguibile (Windows PE .exe, ELF o Mach-O) e' presente una import section.
E' una lista di simboli (symbols - funzioni o variabili globali) importata da moduli esterni insieme ai nomi dei moduli stessi.

Il loade dell' \ac{OS} carica tutti i moduli necessari e, mentre enumera gli import symbols nel modulo primario, determina gli indirizzi
corretti per ciascun simbolo.

Nel nostro caso, \IT{\_\_imp\_\_puts} e' una variabile a 32-bit usata dal loader dell'\ac{OS} per memorizzare l'indirizzo corretto
della funzione in una libreria esterna. 
Successivamente l'istruzione \TT{LDR} legge semplicemente il valore a 32-bit da questa variabile e lo scrive nel registro \ac{PC},
passando il controllo ad esso.

Quindi, per ridurre il tempo necessario al loader dell'\ac{OS} per completare questa procedura, e' una buona idea scrivere l'indirizzo
di ogni simbolo solo una volta, in un punto dedicato.

\myindex{thunk-functions}
Inoltre, come abbiamo gia' capito, e' impossibile caricare un valore a 32-bit value in un registro utilizzando solo una istruzione senza
accesso alla memoria.

Pertanto, la soluzione ottimale e' quella di allocare una funzione separata, che funziona in ARM mode, con il solo scopo di passare
il controllo alla libreria dinamica e quindi saltare dal codice Thumb a questa piccola funzione di una sola istruzione (and then to jump to this short one-instruction)
(la cosiddetta \gls{thunk function}).

\myindex{ARM!\Instructions!BL}
A proposito, nel precedente esempio (compilato per ARM mode) il controllo e' passato da \TT{BL} alla stessa \gls{thunk function}.
La modalita' del processore pero' non viene cambiata (da cui l'assenza di una \q{X} nella instruction mnemonic).

\myparagraph{More about thunk-functions}
\myindex{thunk-functions}

Le thunk-functions sono difficili da comprendere, apparentemente, a causa di una denominazione impropria.
Il modo migliore per capirle e' pensarle come adattatori o convertitori da un tipo di jack ad un altro.
Ad esempio, un adattatore che consente l'inserimento di una spina elettrica Inglese in una presa Americana, o viceversa.
Le thunk functions sono a volte anche dette \IT{wrappers}.

Seguono altre descrizioni di queste funzioni:

\begin{framed}
\begin{quotation}
“Un pezzo di codice che fornisce un indirizzo:”, secondo to P. Z. Ingerman, 
che ha inventato le thunks nel 1961 come un modo di legare i parameters alle loro definizioni formali 
nelle chiamate a procedura in Algol-60. Se una procedura e' chiamata con un'espressione al posto di un parametro formale,
il compilatore genera una thunk che calcola l'espressione e lascia l'indirizzo del risultato in una posizione standard.

\dots

Microsoft e IBM hanno definito, nei loro sistemi basati su Intel, un “ambiente a 16-bit” 
(con orrendi segment registers e limitazioni di indirizzi a 64K) e un “ambiente a 32-bit” 
(con indirizzamento piatto e gestione della memoria semi reale). I due ambienti possono girare contemporaneamente
sullo stesso computer e OS (grazie a quello che, nel mondo Microsoft, e' detto WOW, acronimo per Windows On Windows).
MS e IBM hanno entrambi deciso che il processo di di passare da 16- a 32-bit e viceversa e' detto un “thunk”; in Windows 95, 
esiste anche un tool, THUNK.EXE, detto “thunk compiler”.
\end{quotation}
\end{framed}
% TODO FIXME move to bibliography and quote properly above the quote
( \href{http://go.yurichev.com/17362}{The Jargon File} )
