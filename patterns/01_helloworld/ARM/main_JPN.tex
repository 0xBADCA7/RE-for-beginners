\subsection{ARM}
\label{sec:hw_ARM}

\myindex{\idevices}
\myindex{Raspberry Pi}
\myindex{Xcode}
\myindex{LLVM}
\myindex{Keil}
ARMプロセッサを使用した実験では、いくつかのコンパイラを使用しました。

\begin{itemize}
\item 組み込みの分野で人気があります:Keilリリース 2013/6

\item LLVM-GCC 4.2コンパイラを搭載したApple Xcode 4.6.3 IDE
\footnote{Apple Xcode 4.6.3は、オープンソースのGCCをフロントエンドコンパイラとLLVMコードジェネレータとして使用しています}

\item GCC 4.9(Linaro)(ARM64用)は\url{http://go.yurichev.com/17325}で入手可能です。

\end{itemize}

特に記載がない場合、このマニュアルのすべてのケースで32ビットARMコード(ThumbおよびThumb-2モードを含む)が使用されます。 
64ビットARMについて話すときは、ARM64と呼びます。

% subsections
\subsubsection{\NonOptimizingKeilVI (\ARMMode)}

Keilの例をコンパイルすることから始めましょう。

\begin{lstlisting}
armcc.exe --arm --c90 -O0 1.c 
\end{lstlisting}

\myindex{\IntelSyntax}
\IT{armcc}コンパイラはIntel構文でアセンブリリストを生成しますが、
\footnote{例えば ARMモードには \PUSH/\POP 命令がありません}
それには高レベルのARMプロセッサに関連するマクロがありますが、 \q{そのまま}の命令を見ることが重要ですので、 \IDA のコンパイル結果を見てみましょう。

\begin{lstlisting}[caption=\NonOptimizingKeilVI (\ARMMode) \IDA,style=customasmARM]
.text:00000000             main
.text:00000000 10 40 2D E9    STMFD   SP!, {R4,LR}
.text:00000004 1E 0E 8F E2    ADR     R0, aHelloWorld ; "hello, world"
.text:00000008 15 19 00 EB    BL      __2printf
.text:0000000C 00 00 A0 E3    MOV     R0, #0
.text:00000010 10 80 BD E8    LDMFD   SP!, {R4,PC}

.text:000001EC 68 65 6C 6C+aHelloWorld  DCB "hello, world",0    ; DATA XREF: main+4
\end{lstlisting}

この例では、各命令のサイズが4バイトであることを簡単に確認できます。
実際、Thumb用ではなくARMモード用のコードをコンパイルしました。

\myindex{ARM!\Instructions!STMFD}
\myindex{ARM!\Instructions!POP}
最初の命令\INS{STMFD SP!, \{R4,LR\}}\footnote{\ac{STMFD}}は、2つのレジスタ(\Reg{4} と \ac{LR})の値をスタックに書き込むx86 \PUSH 命令として機能します。

実際、 \IT{armcc} コンパイラの出力リストには、簡略化のために実際に \INS{PUSH \{r4,lr\}} 命令が示されています。しかしそれはかなり正確ではありません。 \PUSH 命令は、Thumbモードでのみ使用できます。したがって、物事をあまり混乱させないために、私たちは \IDA でこれをやっています。

この命令は、最初に \ac{SP} を \glspl{decrement} して、新しいエントリがないスタック内の場所をポイントし、 \Reg{4} および \ac{LR} レジスタの値を変更された \ac{SP} に格納されたアドレスに保存します。

この命令(Thumbモードの \PUSH 命令のような)は、一度にいくつかのレジスタ値を保存することができ、非常に便利です。
ところで、これはx86には同等の機能はありません。
また、\TT{STMFD}命令は、\ac{SP}だけでなく、どのレジスタでも動作できるため、\PUSH 命令の一般化(機能拡張)であることにも注意してください。
換言すれば、\TT{STMFD}は、指定されたメモリアドレスにレジスタのセットを格納するために使用することもできます。

\myindex{\PICcode}
\myindex{ARM!\Instructions!ADR}
\INS{ADR R0, aHelloWorld}命令は、\ac{PC}レジスタの値を\TT{hello, world}文字列が配置されているオフセットに加算または減算します。 
ここでPCレジスタはどのように使用されるのですか? これは\q{\PICcode}\footnote{関連セクションの詳細を読む:(\myref{sec:PIC})}と呼ばれます。

このようなコードは、メモリ内の固定されていないアドレスで実行することができます。 換言すれば、これは\ac{PC}相対アドレッシングである。

\INS{ADR}命令は、この命令のアドレスと文字列が配置されているアドレスとの間の差異を考慮する。 
この違い(オフセット)は、\ac{OS}によってコードがロードされるアドレスに関係なく、常に同じになります。 
だから私たちが必要とするのは、現在の命令のアドレス(\ac{PC}から)を追加して、C文字列の絶対メモリアドレスを取得することだけです。

\myindex{ARM!\Registers!Link Register}
\myindex{ARM!\Instructions!BL}
\INS{BL \_\_2printf}\footnote{Branch with Link}命令は \printf 関数を呼び出します。 
この命令の仕組みは次のとおりです。

\begin{itemize}
\item \INS{BL}命令(\TT{0xC})に続くアドレスを\ac{LR}に格納する
\item そのアドレスを\ac{PC}レジスタに書き込むことによって、コントロールを \printf に渡します。
\end{itemize}

\printf の実行が終了すると、コントロールを返す必要がある場所に関する情報が必要です。
各機能が\ac{LR}レジスタに格納されたアドレスに制御を渡す理由です。

これはARMのような\q{純粋な} \ac{RISC}プロセッサとx86のような\ac{CISC}プロセッサとの違いです。
リターンアドレスは通常スタックに格納されます。
これについての詳細は、次のセクション(\myref{sec:stack})を参照してください。

ところで、絶対32ビットのアドレスまたはオフセットは、24ビットのためのスペースしか有していないので、32ビット\TT{BL}命令では符号化することができない。
思い出されるように、すべてのARMモード命令は4バイト(32ビット)のサイズを持ちます。
したがって、それらは4バイトの境界アドレスにのみ配置することができます。
これは、命令アドレスの最後の2ビット(常にゼロビット)が省略されることを意味する。
要約すると、オフセットエンコーディングには26ビットがあります。これは$current\_PC \pm{} \approx{}32M$をエンコードするのに十分です。

\myindex{ARM!\Instructions!MOV}
次に、\INS{MOV R0, \#0}\footnote{MOVeの意味}命令は、\Reg{0}レジスタに0を書き込むだけです。
これは、C関数が0を返し、戻り値が\Reg{0}レジスタに格納されるためです。

\myindex{ARM!\Registers!Link Register}
\myindex{ARM!\Instructions!LDMFD}
\myindex{ARM!\Instructions!POP}
最後の命令\INS{LDMFD SP!, {R4,PC}}\footnote{\ac{LDMFD}は\ac{STMFD}とは逆の命令です}
スタック(または他のメモリ場所)から値をロードして\Reg{4}と\ac{PC}に保存し、\gls{stack pointer} \ac{SP}を\glslink{increment}{increments}します。
ここで \POP のように動作します。
注意:最初の命令\TT{STMFD}は\Reg{4}と\ac{LR}レジスタのペアをスタックに保存しましたが、\Reg{4}と\ac{PC}は\TT{LDMFD}の実行中に\IT{リストア}されます。

すでにわかっているように、各関数が制御を返さなければならない場所のアドレスは、通常、\ac{LR}レジスタに保存されます。
最初の命令は、 \printf を呼び出すときに \main 関数が同じレジスタを使用するため、その値をスタックに保存します。
関数の終わりでは、この値を直接\ac{PC}レジスタに書き込むことができ、したがって関数が呼び出された場所に制御を渡します。

\main は通常 \CCpp の主要な関数なので、コントロールは\ac{OS}ローダーや\ac{CRT}のような点に返されます。

すべての機能を使用すると、関数の最後に\INS{BX LR}命令を省略できます。

\myindex{ARM!DCB}
\TT{DCB}は、x86アセンブリ言語のDBディレクティブと同様に、バイトまたはASCII文字列の配列を定義するアセンブリ言語ディレクティブです。

\subsubsection{\NonOptimizingKeilVI (\ThumbMode)}

ThumbモードでKeilを使って同じ例をコンパイルしましょう。

\begin{lstlisting}
armcc.exe --thumb --c90 -O0 1.c 
\end{lstlisting}

IDAに入ってみましょう。

\begin{lstlisting}[caption=\NonOptimizingKeilVI (\ThumbMode) + \IDA,style=customasmARM]
.text:00000000             main
.text:00000000 10 B5          PUSH    {R4,LR}
.text:00000002 C0 A0          ADR     R0, aHelloWorld ; "hello, world"
.text:00000004 06 F0 2E F9    BL      __2printf
.text:00000008 00 20          MOVS    R0, #0
.text:0000000A 10 BD          POP     {R4,PC}

.text:00000304 68 65 6C 6C+aHelloWorld  DCB "hello, world",0    ; DATA XREF: main+2
\end{lstlisting}

2バイト(16ビット)のオペコードを簡単に見つけることができます。これは既に述べたように、Thumbです。
\myindex{ARM!\Instructions!BL}
ただし、\TT{BL}命令は2つの16ビット命令で構成されています。
これは、1つの16ビットオペコードの小さなスペースを使用している間に \printf 関数のオフセットをロードすることが不可能なためです。
したがって、第1の16ビット命令はオフセットの上位10ビットをロードし、第2命令はオフセットの下位11ビットをロードする。

% TODO:
% BL has space for 11 bits, so if we don't encode the lowest bit,
% then we should get 11 bits for the upper half, and 12 bits for the lower half.
% And the highest bit encodes the sign, so the destination has to be within
% \pm 4M of current_PC.
% This may be less if adding the lower half does not carry over,
% but I'm not sure --all my programs have 0 for the upper half,
% and don't carry over for the lower half.
% It would be interesting to check where __2printf is located relative to 0x8
% (I think the program counter is the next instruction on a multiple of 4
% for THUMB).
% The lower 11 bytes of the BL instructions and the even bit are
% 000 0000 0110 | 001 0010 1110 0 = 000 0000 0110 0010 0101 1100 = 0x00625c,
% so __2printf should be at 0x006264.
% But if we only have 10 and 11 bits, then the offset would be:
% 00 0000 0110 | 01 0010 1110 0 = 0 0000 0011 0010 0101 1100 = 0x00325c,
% so __2printf should be at 0x003264.
% In this case, though, the new program counter can only be 1M away,
% because of the highest bit is used for the sign.

前述したように、Thumbモードの命令はすべて2バイト(または16ビット)のサイズです。
これは、Thumb命令が奇妙なアドレスにあることはまったく不可能であることを意味します。
上記を前提として、命令を符号化する間に最後のアドレスビットを省略することができる。

まとめると、\TT{BL}Thumb命令は$current\_PC \pm{}\approx{}2M$のアドレスを符号化することができます。

\myindex{ARM!\Instructions!PUSH}
\myindex{ARM!\Instructions!POP}
関数内の他の命令については、 \PUSH と \POP はここで説明した\TT{STMFD}/\TT{LDMFD}のように動作しますが、ここでは\ac{SP}レジスタのみが明示的に言及されていません。 
\TT{ADR}は前の例と同様に動作します。 
\TT{MOVS}は、0を返すために\Reg{0}レジスタに0を書き込みます。


\subsubsection{\OptimizingXcodeIV (\ARMMode)}

最適化を有効にしない場合のXcode 4.6.3では、冗長なコードが多数生成されるため、命令カウントができるだけ小さい最適化された出力を検討し、コンパイラスイッチ \Othree を設定します。

\begin{lstlisting}[caption=\OptimizingXcodeIV (\ARMMode),style=customasmARM]
__text:000028C4             _hello_world
__text:000028C4 80 40 2D E9   STMFD           SP!, {R7,LR}
__text:000028C8 86 06 01 E3   MOV             R0, #0x1686
__text:000028CC 0D 70 A0 E1   MOV             R7, SP
__text:000028D0 00 00 40 E3   MOVT            R0, #0
__text:000028D4 00 00 8F E0   ADD             R0, PC, R0
__text:000028D8 C3 05 00 EB   BL              _puts
__text:000028DC 00 00 A0 E3   MOV             R0, #0
__text:000028E0 80 80 BD E8   LDMFD           SP!, {R7,PC}

__cstring:00003F62 48 65 6C 6C+aHelloWorld_0  DCB "Hello world!",0
\end{lstlisting}

命令\TT{STMFD} と \TT{LDMFD}はもうよく知っていますね。

\myindex{ARM!\Instructions!MOV}

\MOV 命令は、\Reg{0}レジスタに数値\TT{0x1686}を書き込むだけです。 
これは「Hello world!」文字列を指すオフセットです。

\TT{R7}レジスタ( \IOSABI で標準化されている)はフレームポインタです。 以下でもっとみてみましょう。

MOVT R0、#0(MOVe Top)命令は、レジスタの上位16ビットに0を書き込みます。 ここでの問題点は、ARMモードの汎用MOV命令がレジスタの下位16ビットだけを書き込むことができることです。

ARMモードの命令オペコードはすべて32ビットに制限されています。 もちろん、この制限はレジスタ間でのデータの移動には関係しません。 
そのため、上位ビット(16から31まで)に書き込むための命令\TT{MOVT}が追加されています。 
ただし、ここでの使用方法は冗長です。これは、\TT{MOV R0, \#0x1686}命令がレジスタの上位部分をクリアしたためです。 
これはおそらくコンパイラの欠点です。
% TODO:
% I think, more specifically, the string is not put in the text section,
% ie. the compiler is actually not using position-independent code,
% as mentioned in the next paragraph.
% MOVT is used because the assembly code is generated before the relocation,
% so the location of the string is not yet known,
% and the high bits may still be needed.

\myindex{ARM!\Instructions!ADD}
\TT{ADD R0, PC, R0}命令は、\ac{PC}の値を\Reg{0}の値に加算し、\q{Hello world!}文字列の絶対アドレスを計算します。 
すでにわかっているように、それは\q{位置独立コード}なので、この修正はここでは必須です。

\INS{BL}命令は \printf の代わりに \puts 関数を呼び出します。

\label{puts}
\myindex{\CStandardLibrary!puts()}
\myindex{puts() instead of printf()}

GCCは最初の \printf 呼び出しを \puts に置き換えました。 
確かに、唯一の引数を持つ \printf は、 \puts とほぼ同じです。

\IT{ほとんどの場合}、文字列に\IT{\%}で始まるprintf形式識別子が含まれていない場合にのみ、
2つの関数が同じ結果を生成するためです。 その場合、これらの2つの機能の効果は異なります
\footnote{ \puts は文字列の最後に改行記号`\textbackslash{}n'を必要としないので、ここでは見られません}

なぜコンパイラは \printf を \puts に置き換えたのでしょうか?おそらく \puts が高速であるためです。
\footnote{\href{http://go.yurichev.com/17063}{ciselant.de/projects/gcc\_printf/gcc\_printf.html}}. 

これは、文字を\IT{\%}と一緒に比較することなく、文字を\gls{stdout}に渡すだけです。

次に、\Reg{0}レジスタを0に設定するための使い慣れた\TT{MOV R0, \#0}命令があります。

\subsubsection{\OptimizingXcodeIV (\ThumbTwoMode)}

Xcode 4.6.3ではThumb-2のコードがデフォルトでは次のように生成されます。

\begin{lstlisting}[caption=\OptimizingXcodeIV (\ThumbTwoMode),style=customasmARM]
__text:00002B6C                   _hello_world
__text:00002B6C 80 B5          PUSH            {R7,LR}
__text:00002B6E 41 F2 D8 30    MOVW            R0, #0x13D8
__text:00002B72 6F 46          MOV             R7, SP
__text:00002B74 C0 F2 00 00    MOVT.W          R0, #0
__text:00002B78 78 44          ADD             R0, PC
__text:00002B7A 01 F0 38 EA    BLX             _puts
__text:00002B7E 00 20          MOVS            R0, #0
__text:00002B80 80 BD          POP             {R7,PC}

...

__cstring:00003E70 48 65 6C 6C 6F 20+aHelloWorld  DCB "Hello world!",0xA,0
\end{lstlisting}

% Q: If you subtract 0x13D8 from 0x3E70,
% you actually get a location that is not in this function, or in _puts.
% How is PC-relative addressing done in THUMB2?
% A: it's not Thumb-related. there are just mess with two different segments. TODO: rework this listing.

\myindex{\ThumbTwoMode}
\myindex{ARM!\Instructions!BL}
\myindex{ARM!\Instructions!BLX}

Thumbモードの\TT{BL} と \TT{BLX}命令は、16ビット命令のペアとしてエンコードされています。 
Thumb-2では、これらの\IT{代理}オペコードは、新しい命令がここで32ビット命令として符号化されるように拡張される。

これは、Thumb-2命令のオペコードが常に\TT{0xFx} または \TT{0xEx}で始まることを考慮すると明らかです

しかし、 \IDA のリストでは、opcodeバイトはスワップされます。これは、ARMプロセッサの場合、命令は次のようにエンコードされるためです。
最後のバイトが最初に来て、最初のバイトが来ると(ThumbおよびThumb-2モードの場合)、
ARMモードの命令の場合、 第1、第3、第2、そして最後に第1(異なるエンディアンのため)です。

つまり、バイトがIDAリストにどのように配置されているかです。

\begin{itemize}
\item ARMおよびARM64モードの場合:4-3-2-1;
\item Thumbモードの場合:2-1;
\item Thumb-2モードの16ビット命令の場合は2-1-4-3になります。
\end{itemize}

\myindex{ARM!\Instructions!MOVW}
\myindex{ARM!\Instructions!MOVT.W}
\myindex{ARM!\Instructions!BLX}

したがって、\TT{MOVW}、 \TT{MOVT.W} および \TT{BLX}X命令は\TT{0xFx}で始まります。

Thumb-2命令の1つは\TT{MOVW R0, \#0x13D8}です。16ビット値を\Reg{0}レジスタの下部に格納し、上位ビットをクリアします。

また、\TT{MOVT.W R0, \#0}は、前の例の\TT{MOVT}と同様に動作し、Thumb-2でのみ動作します。

\myindex{ARM!mode switching}
\myindex{ARM!\Instructions!BLX}

\TT{BLX}命令は、\TT{BL}の代わりにこの場合に使用されます。

違いは、\ac{RA}を\ac{LR}レジスタに保存し、 \puts 関数に制御を渡すことに加えて、
プロセッサはThumb/Thumb-2モードからARMモード(またはその逆)にも切り替わります。

この命令は、制御が渡される命令が次のようになっているため、ここに配置されています(ARMモードでエンコードされています)。

\begin{lstlisting}[style=customasmARM]
__symbolstub1:00003FEC _puts           ; CODE XREF: _hello_world+E
__symbolstub1:00003FEC 44 F0 9F E5     LDR  PC, =__imp__puts
\end{lstlisting}

これは本質的に、importsセクションに \puts のアドレスが書き込まれる場所へのジャンプです。

したがって、注意深い読者が質問するかもしれません:コードのどこに必要なところにputs()を呼び出すのはなぜですか?

非常にスペース効率が良いわけではないからです。

\myindex{Dynamically loaded libraries}
ほぼすべてのプログラムは外部のダイナミックライブラリ(WindowsではDLL、*NIXでは.so、 \MacOSX では.dylib)を使用します。
動的ライブラリには、標準のC関数puts()を含む、頻繁に使用されるライブラリ関数が含まれています。

\myindex{Relocation}
実行可能バイナリファイル(Windows PE .exe、ELFまたはMach-O)には、インポートセクションが存在します。
これは、外部モジュールからインポートされたシンボル(関数またはグローバル変数)のリストと、モジュール自体の名前です。

\ac{OS}ローダは、必要なすべてのモジュールをロードし、プライマリモジュールのインポートシンボルを列挙しながら、各シンボルの正しいアドレスを決定します。

私たちの場合、\IT{\_\_imp\_\_puts}は、\ac{OS}ローダーが外部ライブラリに関数の正しいアドレスを格納するために使用する32ビットの変数です。
次に、\TT{LDR}命令はこの変数から32ビットの値を読み込み、それを制御に渡して\ac{PC}レジスタに書き込みます。

\myindex{thunk-functions}
したがって、この手順を完了するために\ac{OS}ローダが必要とする時間を短縮するには、各シンボルのアドレスを専用の場所に1回だけ書き込むことをお勧めします。

さらに、すでにわかっているように、メモリアクセスなしで1つの命令だけを使用している間は、32ビットの値をレジスタにロードすることは不可能です。

したがって、最適な解決策は、ダイナミックライブラリに制御を渡し、
次にThumbコードからこの短い1命令関数(いわゆる\gls{thunk function})にジャンプするという唯一の目的で、ARMモードで動作する別の関数を割り当てることです。

\myindex{ARM!\Instructions!BL}
ところで、(ARMモード用にコンパイルされた)前の例では、コントロールは\\TT{BL}によって同じ\gls{thunk function}に渡されます。
ただし、プロセッサモードは切り替えられていません(したがって、命令ニーモニックに \q{X}がありません)。

\myparagraph{thunk-functionsの追加情報}
\myindex{thunk-functions}

サンク関数は、誤った名前のために、明らかに理解するのが難しいです。 
1つのタイプのジャックのアダプターまたはコンバーターとして別のタイプのジャックに理解する最も簡単な方法です。 
たとえば、イギリスの電源プラグをアメリカのコンセントに差し込むことができるアダプタ、またはその逆。 
サンク関数はラッパーと呼ばれることもあります。

これらの関数についてもう少し詳しく説明します:

\begin{framed}
\begin{quotation}
1961年にAlgol-60プロシージャコールの正式な定義に実際のパラメータをバインドする手段としてThunksを発明したP.Z. Ingermanによると、
「アドレスを提供するコーディング」:仮パラメータの代わりに式を使用してプロシージャーを呼び出すと、
コンパイラーは式を計算するサンクを生成し、結果のアドレスを何らかの標準の場所に残します。

\dots

マイクロソフトとIBMは、Intelベースのシステムでは、(ブレティックセグメントレジスタと64Kアドレス制限付きの)
「16ビット環境」とフラットアドレッシングとセミリアルメモリ管理を備えた「32ビット環境」を定義しています。
この2つの環境は、同じコンピュータとOS上で動作することができます(Microsoftの世界では、Windows on Windowsの略です)。 
MSとIBMはどちらも、16ビットから32ビットへの変換プロセスを「サンク」と呼んでいます。 
Windows 95には、 THUNK.EXEというツールがあります。これは "サンクコンパイラ"と呼ばれています。
\end{quotation}
\end{framed}
% TODO FIXME move to bibliography and quote properly above the quote
( \href{http://go.yurichev.com/17362}{The Jargon File} )

\myindex{LAPACK}
\myindex{FORTRAN}
他の例としてLAPACK libraryがあります。FORTRANで書かれた ``Linear Algebra PACKage''です。
\CCpp 開発者もLAPACKを使いたいと思っていますが、 \CCpp に書き直していくつかのバージョンを維持するのは難しいことです。 
ですから、 \CCpp 環境から呼び出し可能な短いC関数があります。これは、順番にFORTRAN関数を呼び出し、他の何かを実行します。

\begin{lstlisting}[style=customc]
double Blas_Dot_Prod(const LaVectorDouble &dx, const LaVectorDouble &dy)
{
    assert(dx.size()==dy.size());
    integer n = dx.size();
    integer incx = dx.inc(), incy = dy.inc();

    return F77NAME(ddot)(&n, &dx(0), &incx, &dy(0), &incy);
}
\end{lstlisting}

また、そのような関数は ``ラッパー''と呼ばれます。

\subsubsection{ARM64}

\myparagraph{GCC}

ARM64環境で、GCC 4.8.1を使用してサンプルをコンパイルしましょう。

\lstinputlisting[numbers=left,label=hw_ARM64_GCC,caption=\NonOptimizing GCC 4.8.1 + objdump,style=customasmARM]{patterns/01_helloworld/ARM/hw.lst}

ARM64にはThumbモードとThumb-2モードはなく、ARMのみであるため、32ビット命令のみがあります。
レジスタ数は2倍になります:\myref{ARM64_GPRs}
64ビットレジスタは\TT{X-}プレフィックスを持ち、32ビット部分は\TT{W-}です。

\myindex{ARM!\Instructions!STP}
\TT{STP}命令(\IT{ストアペア})は、スタック内の2つのレジスタ\RegX{29}と\RegX{30}を同時に保存します。

もちろん、この命令はメモリ内の任意の場所にこのペアを保存できますが、
ここで\ac{SP}レジスタが指定されているため、ペアはスタックに保存されます。

ARM64レジスタは64ビットのレジスタで、それぞれ8バイトのサイズを持つため、2つのレジスタを保存するために16バイト必要です。

オペランドの後の感嘆符(``!'')は、最初に16が\ac{SP}から減算され、
次にスタックに書き込まれるレジスタ・ペアの値であることを意味します。
これは\IT{事前インデックス}とも呼ばれます。\IT{事後インデックス}と\IT{事前インデックス}の違いについては、
\myref{ARM_postindex_vs_preindex} を読んでください。

したがって、より使い慣れたx86では、最初の命令は\TT{PUSH X29} と \TT{PUSH X30}のペアのアナログに過ぎません。
\RegX{29}はARM64では\ac{FP}として、\ac{LR}では\RegX{30}として使用されているため、関数プロローグに保存され、関数エピローグで復元されます。

2番目の命令は\RegX{29}(または\ac{FP})の\ac{SP}をコピーします。
これは、関数スタックフレームを設定するために行われます。

\label{pointers_ADRP_and_ADD}
\myindex{ARM!\Instructions!ADRP/ADD pair}
\TT{ADRP}命令と \ADD 命令は、最初の関数引数がこのレジスタに渡されるため、
文字列\q{Hello!}のアドレスを\RegX{0}レジスタに入力するために使用されます。
命令長は4バイトに制限されているため、レジスタに多数の命令を格納できる命令はありません。
詳細は\myref{ARM_big_constants_loading}参照してください。
したがって、いくつかの命令を利用する必要があります。最初の命令(\TT{ADRP})は、文字列が配置されている4KiBページのアドレスを\RegX{0}に書き込み、
2番目の命令(\ADD)は残りのアドレスをアドレスに追加するだけです。
詳細については、\myref{ARM64_relocs}を参照してください。

\TT{0x400000 + 0x648 = 0x400648}であり、このアドレスの\TT{.rodata}データセグメントにある\q{Hello!} C文字列を参照してください。

\myindex{ARM!\Instructions!BL}

\TT{BL}命令を使用して \puts を呼び出します。 これについては既に説明しました:\myref{puts}

\MOV は\RegW{0}に0を書き込みます。 
\RegW{0}は64ビット\RegX{0}レジスタの下位32ビットです。

\begin{center}
\begin{tabular}{ | l | l | }
\hline
\RU{Старшие 32 бита}\EN{High 32-bit part}\ES{Parte alta de 32 bits}\PTBRph{}\PLph{}\ITAph{}\DEph{}\THAph{}\NLph{}\FR{Partie 32 bits haute} & \RU{младшие 32 бита}\EN{low 32-bit part}\ES{parte baja de 32 bits}\PTBRph{}\PLph{Starsze 32 bity}\ITAph{}\DEph{}\THAph{}\NLph{}\FR{Partie 32 bits basse} \\
\hline
\multicolumn{2}{ | c | }{X0} \\
\hline
\multicolumn{1}{ | c | }{} & \multicolumn{1}{ c | }{W0} \\
\hline
\end{tabular}
\end{center}


関数の結果は\RegX{0}を介して返され、 \main は0を返します。これで、リターンされる結果がどのように準備されるのかがわかります。
しかし、なぜ32ビットの部分を使用するのでしょうか?

ARM64の \Tint データ型はx86-64の場合と同じように、互換性を高めるため、32ビットとなっています。

関数が32ビット \Tint を返す場合は、\RegX{0}レジスタの下位32ビットのみを埋めなければなりません。

これを確認するために、この例を少し変更して再コンパイルしましょう。 \main は64ビット値を返します:

\begin{lstlisting}[caption=\main returning a value of \TT{uint64\_t} type,style=customc]
#include <stdio.h>
#include <stdint.h>

uint64_t main()
{
        printf ("Hello!\n");
        return 0;
}
\end{lstlisting}

結果は同じですが、その行の \MOV は次のようになります:

\begin{lstlisting}[caption=\NonOptimizing GCC 4.8.1 + objdump]
  4005a4:       d2800000        mov     x0, #0x0      // #0
\end{lstlisting}

\myindex{ARM!\Instructions!LDP}

\INS{LDP} (\IT{Load Pair}) は\RegX{29}と\RegX{30}レジスタを復元します。

命令の後には感嘆符はありません。これは、値が最初にスタックからロードされ、
次に\ac{SP}が16だけ増加したことを意味します。
これは\IT{事後インデックス}と呼ばれます。

\myindex{ARM!\Instructions!RET}
ARM64: \RET という新しい命令が登場しました。 
これは\TT{BX LR}と同様に機能し、特別な\IT{ヒント}ビットのみが追加され、
これが別のジャンプ命令ではなく関数からの戻りであることを\ac{CPU}に通知するので、より最適に実行できます。

関数の単純さのために、GCCの最適化はまさに同じコードを生成します。


