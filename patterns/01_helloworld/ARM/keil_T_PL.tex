\subsubsection{\NonOptimizingKeilVI (\ThumbMode)}

Skompilujmy ten sam przykład w Keil dla trybu Thumb:

\begin{lstlisting}
armcc.exe --thumb --c90 -O0 1.c 
\end{lstlisting}

Otrzymamy (w \IDA):

\begin{lstlisting}[caption=\NonOptimizingKeilVI (\ThumbMode) + \IDA,style=customasmARM]
.text:00000000             main
.text:00000000 10 B5          PUSH    {R4,LR}
.text:00000002 C0 A0          ADR     R0, aHelloWorld ; "hello, world"
.text:00000004 06 F0 2E F9    BL      __2printf
.text:00000008 00 20          MOVS    R0, #0
.text:0000000A 10 BD          POP     {R4,PC}

.text:00000304 68 65 6C 6C+aHelloWorld  DCB "hello, world",0    ; DATA XREF: main+2
\end{lstlisting}

Od razu można zauważyć (16-bitowe) opcode --- jest to, jak już było powiedziane, Thumb.

\myindex{ARM!\Instructions!BL}
Oprócz instrukcji \TT{BL}.
Ale tak naprawdę to ona się składa z dwóch 16-bitowych instrukcji.
Tak się dzieje dlatego że w jednym 16-bitowym opcode jest za mało miejsca dla przesunięcia, po którym znajduje się funkcja \printf.
Dlatego pierwsza 16-bitowa instrukcja ładuje starsze 10 bitów przesunięcia, a druga~--- młodsze 11 bitów przesunięcia.

% TODO:
% BL has space for 11 bits, so if we don't encode the lowest bit,
% then we should get 11 bits for the upper half, and 12 bits for the lower half.
% And the highest bit encodes the sign, so the destination has to be within
% \pm 4M of current_PC.
% This may be less if adding the lower half does not carry over,
% but I'm not sure --all my programs have 0 for the upper half,
% and don't carry over for the lower half.
% It would be interesting to check where __2printf is located relative to 0x8
% (I think the program counter is the next instruction on a multiple of 4
% for THUMB).
% The lower 11 bytes of the BL instructions and the even bit are
% 000 0000 0110 | 001 0010 1110 0 = 000 0000 0110 0010 0101 1100 = 0x00625c,
% so __2printf should be at 0x006264.
% But if we only have 10 and 11 bits, then the offset would be:
% 00 0000 0110 | 01 0010 1110 0 = 0 0000 0011 0010 0101 1100 = 0x00325c,
% so __2printf should be at 0x003264.
% In this case, though, the new program counter can only be 1M away,
% because of the highest bit is used for the sign.

jak już było powiedziane, wszystkie instrukcje w trybie Thumb są 2 bajtowe (lub 16 bitowe).
Dlatego sytuacja, w której Thumb-instrukcja zaczyna się pod adresem nieparzystym, jest niemożliwa.

Wniosek z tego jest taki, że ostatniego bitu adresu można nie kodować.
W ten sposób, w Thumb-instrukcji \TT{BL} można zakodować adres $current\_PC \pm{}\approx{}2M$.

\myindex{ARM!\Instructions!PUSH}
\myindex{ARM!\Instructions!POP}
Reszta instrukcji w funkcji (\PUSH i \POP) działa prawie tak samo, jak i opisane wyżej \TT{STMFD}/\TT{LDMFD}, tyle że rejestr \ac{SP} nie jest tu wskazany w sposób jawny.
\INS{ADR} działa dokładnie tak samo, jak i w poprzednim przykładzie.
\INS{MOVS} zapisuje wartość 0 do rejestru \Reg{0}, żeby f-cja zwróciła zero.


