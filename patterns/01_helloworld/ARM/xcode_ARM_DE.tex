\subsubsection{\OptimizingXcodeIV (\ARMMode)}

Xcode 4.6.3 ohne Optimierung produziert eine Menge redundanten Code, so dass im Folgenden die
optimierte Ausgabe gelistet ist bei der die Anzahl der Anweisungen so klein wie möglich ist.
Der Compiler-Schalter ist \Othree.

\begin{lstlisting}[caption=\OptimizingXcodeIV (\ARMMode),style=customasmARM]
__text:000028C4             _hello_world
__text:000028C4 80 40 2D E9   STMFD           SP!, {R7,LR}
__text:000028C8 86 06 01 E3   MOV             R0, #0x1686
__text:000028CC 0D 70 A0 E1   MOV             R7, SP
__text:000028D0 00 00 40 E3   MOVT            R0, #0
__text:000028D4 00 00 8F E0   ADD             R0, PC, R0
__text:000028D8 C3 05 00 EB   BL              _puts
__text:000028DC 00 00 A0 E3   MOV             R0, #0
__text:000028E0 80 80 BD E8   LDMFD           SP!, {R7,PC}

__cstring:00003F62 48 65 6C 6C+aHelloWorld_0  DCB "Hello world!",0
\end{lstlisting}

Die Anweisungen \TT{STMFD} und \TT{LDMFD} sind bereits bekannt.

\myindex{ARM!\Instructions!MOV}

Die \MOV-Anweisung schreibt lediglich die Nummer \TT{0x1686} in das Register \Reg{0}.
Dies ist der Offset der auf die Zeichenkette \q{Hello world!} zeigt.

Das Register \TT{R7} (spezifiziert in \IOSABI) ist ein Frame Pointer. Mehr darüber folgt später.

\myindex{ARM!\Instructions!MOVT}
Die \TT{MOVT R0, \#0} (MOVe Top)-Anweisung schreibt 0 in die höherwertigen 16 Bit des Registers.
Das Problem ist hier, dass die generische \MOV-Anweisung im ARM-Mode nur die niederwertigen 16 Bit
des Registers beschreibt.

Dran denken: alle Opcodes im ARM-Mode sind in der Größe auf 32 Bit begrenzt. Natürlich gilt diese
Begrenzung nicht für das Verschieben von Daten zwischen Registern.
Aus diesem Grund existiert die zusätzliche Anweisung  \TT{MOVT} um in die höherwertigen Bits
(von 16 bis einschließlich 31) zu beschreiben.
Die Benutzung ist in diesem Fall redundant, weil die Anweisung \TT{MOV R0, \#0x1686} darüpber
den höherwertigen Teil des Registers zurückgesetzt hat.
Dies ist vermutlich ein Mangel des Compilers.

% TODO:
% I think, more specifically, the string is not put in the text section,
% ie. the compiler is actually not using position-independent code,
% as mentioned in the next paragraph.
% MOVT is used because the assembly code is generated before the relocation,
% so the location of the string is not yet known,
% and the high bits may still be needed.

\myindex{ARM!\Instructions!ADD}
Die Anweisung \TT{ADD R0, PC, R0} addiert den Wert im \ac{PC} zum Wert im Register \Reg{0}
um die absolute Adresse der \q{Hello world!}-Zeichenkette zu berechnen.
Wie bereits bekannt ist dies \q{\PICcode}, so dass diese Korrektur hier unbedingt notwendig ist.

Die \INS{BL}-Anweisung ruft \puts anstatt \printf auf.

\label{puts}
\myindex{\CStandardLibrary!puts()}
\myindex{puts() anstatt printf()}

GCC ersetzt den ersten \printf-Aufruf mit \puts. In der Tat ist \printf mit nur einem
Argument identisch mit \puts.

Die beiden Funktionen produzieren lediglich das gleiche Ergebnis, weil printf keine
Formatkennzeichner, beginnend mit \IT{\%}, enhält.
Sollte dies jedoch der Fall sein, wäre die Auswirkung der beiden Funktionen
unterschiedlich\footnote{Des weiteren benötigt \puts kein '\textbackslash{}n'
für den Zeilenumbruch am Ende der Zeichenkette, weswegen wir dies hier nicht sehen.}.

Warum hat der Compiler diese Ersetzung durchgeführt? Vermutlich hat dies Vorteile bei
der Geschwindigkeit, weil \puts schneller ist
\footnote{\href{http://go.yurichev.com/17063}{ciselant.de/projects/gcc\_printf/gcc\_printf.html}}
und lediglich die Zeichen zu \gls{stdout} übergibt, anstatt jedes Zeichen mit \IT{\%} zu vergleichen.

Als nächstes ist die bekannte Anweisung \TT{MOV R0, \#0} zu sehen um das Register \Reg{0} auf 0 zu setzen.
