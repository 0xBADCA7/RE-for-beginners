\subsubsection{\OptimizingXcodeIV (\ARMMode)}

Xcode 4.6.3 bez włączonego trybu optymalizacji generuje za dużo zbędnego kodu, dlatego włączymy optymalizacje kodu (flaga \Othree).

\begin{lstlisting}[caption=\OptimizingXcodeIV (\ARMMode),style=customasmARM]
__text:000028C4             _hello_world
__text:000028C4 80 40 2D E9   STMFD           SP!, {R7,LR}
__text:000028C8 86 06 01 E3   MOV             R0, #0x1686
__text:000028CC 0D 70 A0 E1   MOV             R7, SP
__text:000028D0 00 00 40 E3   MOVT            R0, #0
__text:000028D4 00 00 8F E0   ADD             R0, PC, R0
__text:000028D8 C3 05 00 EB   BL              _puts
__text:000028DC 00 00 A0 E3   MOV             R0, #0
__text:000028E0 80 80 BD E8   LDMFD           SP!, {R7,PC}

__cstring:00003F62 48 65 6C 6C+aHelloWorld_0  DCB "Hello world!",0
\end{lstlisting}

Instrukcje \TT{STMFD} i \TT{LDMFD} już są nam znane.

\myindex{ARM!\Instructions!MOV}
Instrukcja \MOV po prostu zapisuje liczbę \TT{0x1686} do rejestru \Reg{0}~--- jest to przesunięcie, wskazujące na linię \q{Hello world!}.

Rejestr \Reg{7} (wg standardu \IOSABI) jest frame pointer, będzie to omawiane później.

\myindex{ARM!\Instructions!MOVT}
Instrukcja \TT{MOVT R0, \#0} (MOVe Top) zapisuje 0 do starszych 16 bitów rejestru.
Rzecz polega na tym, że zwykła instrukcja \MOV w trybie ARM może zapisywać jakąkolwiek wartość tylko do młodszych 16 bitów rejestru, dlatego że nie można w niej zakodować więcej.
Warto pamiętać, że w trybie ARM opcode wszystkich instrukcji mogą wynosić maksymalnie 32 bity. Oczywiście, nie jest to prawda dla przemieszczenia danych między rejestrami.

Dlatego do zapisywania do starszych bitów (bity 16-31) istnieje dodatkowa instrukcja \INS{MOVT}.
Tutaj, natomiast, jej wykorzystanie jest zbędne, jako że \INS{MOV R0, \#0x1686} i bez tego wyzerowało starszą część rejestru.
Możliwe, że jest to niedociągnięcie kompilatora.
% TODO:
% I think, more specifically, the string is not put in the text section,
% ie. the compiler is actually not using position-independent code,
% as mentioned in the next paragraph.
% MOVT is used because the assembly code is generated before the relocation,
% so the location of the string is not yet known,
% and the high bits may still be needed.

\myindex{ARM!\Instructions!ADD}
Instrukcja \TT{ADD R0, PC, R0} dodaje \ac{PC} do \Reg{0} żeby wyliczyć adres rzeczywisty linii \q{Hello world!}. Jak już wiemy, jest to \q{\PICcode}, dlatego taka korekta jest niezbędna.

Instrukcja \TT{BL} wywołuje \puts zamiast \printf.

\label{puts}
\myindex{\CStandardLibrary!puts()}
\myindex{puts() zamiast printf()}
Kompilator zamienia wywołanie \printf na \puts. 
Rzeczywiście, \printf z jednym argumentem jest prawie analogiczne do \puts.
 
\IT{Prawie}, jeśli założyć, że w linii nie będzie znaków sterujących \printf, 
zaczynających się od znaku procentu. Wtedy wyniki pracy tych dwóch f-cji będą różne
\footnote{Należy również zauważyć, że \puts nie potrzebuje znaku nowej linii `\textbackslash{}n',
dlatego jego tu nie ma.}.

Po co kompilator zamienił wywołanie jednej f-cji na inną? Prawdopodobnie, dlatego że \puts jest szybsze
\footnote{\href{http://go.yurichev.com/17063}{ciselant.de/projects/gcc\_printf/gcc\_printf.html}}. 
Najwidoczniej, dlatego że \puts wyprowadza symbole na \gls{stdout} nie porównując każdego ze znakiem procentu.

Dalej jest już znana instrukcja \TT{MOV R0, \#0}, ustawiająca 0 jako wartość zwracaną.


