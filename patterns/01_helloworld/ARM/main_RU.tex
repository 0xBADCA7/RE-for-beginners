\subsection{ARM}
\label{sec:hw_ARM}

\myindex{\idevices}
\myindex{Raspberry Pi}
\myindex{Xcode}
\myindex{LLVM}
\myindex{Keil}
Для экспериментов с процессором ARM было использовано несколько компиляторов:

\begin{itemize}
\item Популярный в embedded-среде Keil Release 6/2013.

\item Apple Xcode 4.6.3 с компилятором LLVM-GCC 4.2
\footnote{Это действительно так: Apple Xcode 4.6.3 использует опен-сорсный GCC как компилятор переднего плана и кодогенератор LLVM}.

\item GCC 4.9 (Linaro) (для ARM64), 
доступный в виде исполняемого файла для win32 на \url{http://go.yurichev.com/17325}.

\end{itemize}

Везде в этой книге, если не указано иное, идет речь о 32-битном ARM (включая режимы Thumb и Thumb-2).
Когда речь идет о 64-битном ARM, он называется здесь ARM64.

% subsections
\subsubsection{\NonOptimizingKeilVI (\ARMMode)}

Для начала скомпилируем наш пример в Keil:

\begin{lstlisting}
armcc.exe --arm --c90 -O0 1.c 
\end{lstlisting}

\myindex{\IntelSyntax}
Компилятор \IT{armcc} генерирует листинг на ассемблере в формате Intel.
Этот листинг содержит некоторые высокоуровневые макросы, связанные с ARM
\footnote{например, он показывает инструкции \PUSH/\POP, отсутствующие в режиме ARM},
а нам важнее увидеть инструкции \q{как есть}, так что посмотрим скомпилированный результат в \IDA.

\begin{lstlisting}[caption=\NonOptimizingKeilVI (\ARMMode) \IDA,style=customasmARM]
.text:00000000             main
.text:00000000 10 40 2D E9    STMFD   SP!, {R4,LR}
.text:00000004 1E 0E 8F E2    ADR     R0, aHelloWorld ; "hello, world"
.text:00000008 15 19 00 EB    BL      __2printf
.text:0000000C 00 00 A0 E3    MOV     R0, #0
.text:00000010 10 80 BD E8    LDMFD   SP!, {R4,PC}

.text:000001EC 68 65 6C 6C+aHelloWorld  DCB "hello, world",0    ; DATA XREF: main+4
\end{lstlisting}

В вышеприведённом примере можно легко увидеть, что каждая инструкция имеет размер 4 байта.
Действительно, ведь мы же компилировали наш код для режима ARM, а не Thumb.

\myindex{ARM!\Instructions!STMFD}
\myindex{ARM!\Instructions!POP}
Самая первая инструкция, \INS{STMFD SP!, \{R4,LR\}}\footnote{\ac{STMFD}},
работает как инструкция \PUSH в x86, записывая значения двух регистров (\Reg{4} и \ac{LR}) в стек.
Действительно, в выдаваемом листинге на ассемблере компилятор \IT{armcc} для упрощения указывает здесь инструкцию
\INS{PUSH \{r4,lr\}}.
Но это не совсем точно, инструкция \PUSH доступна только в режиме Thumb, поэтому,
во избежание путаницы, я предложил работать в \IDA.

Итак, эта инструкция уменьшает \ac{SP}, чтобы он указывал на место в стеке, свободное для записи
новых значений, затем записывает значения регистров \Reg{4} и \ac{LR} 
по адресу в памяти, на который указывает измененный регистр \ac{SP}.

Эта инструкция, как и инструкция \PUSH в режиме Thumb, может сохранить в стеке одновременно несколько значений регистров, что может быть очень удобно.
Кстати, такого в x86 нет.
Также следует заметить, что \TT{STMFD}~--- генерализация инструкции \PUSH (то есть расширяет её возможности), потому что может работать с любым регистром, а не только с \ac{SP}.
Другими словами, \TT{STMFD} можно использовать для записи набора регистров в указанном месте памяти.

\myindex{\PICcode}
\myindex{ARM!\Instructions!ADR}
Инструкция \INS{ADR R0, aHelloWorld} прибавляет или отнимает значение регистра \ac{PC} к смещению, где хранится строка
\TT{hello, world}.
Причем здесь \ac{PC}, можно спросить? Притом, что это так называемый \q{\PICcode}
\footnote{Читайте больше об этом в соответствующем разделе ~(\myref{sec:PIC})}.
Он предназначен для исполнения будучи не привязанным к каким-либо адресам в памяти.
Другими словами, это относительная от \ac{PC} адресация.
В опкоде инструкции \TT{ADR} указывается разница между адресом этой инструкции и местом, где хранится строка.
Эта разница всегда будет постоянной, вне зависимости от того, куда был загружен \ac{OS} наш код.
Поэтому всё, что нужно~--- это прибавить адрес текущей инструкции (из \ac{PC}), чтобы получить текущий абсолютный адрес нашей Си-строки.

\myindex{ARM!\Registers!Link Register}
\myindex{ARM!\Instructions!BL}
Инструкция \INS{BL \_\_2printf}\footnote{Branch with Link} вызывает функцию \printf.
Работа этой инструкции состоит из двух фаз:

\begin{itemize}
\item записать адрес после инструкции \INS{BL} (\TT{0xC}) в регистр \ac{LR};
\item передать управление в \printf, записав адрес этой функции в регистр \ac{PC}.
\end{itemize}

Ведь когда функция \printf закончит работу, нужно знать, куда вернуть управление, поэтому закончив работу, всякая функция передает управление по адресу, записанному в регистре \ac{LR}.

В этом разница между \q{чистыми} \ac{RISC}-процессорами вроде ARM и \ac{CISC}-процессорами как x86,
где адрес возврата обычно записывается в стек ~(\myref{sec:stack}).

Кстати, 32-битный абсолютный адрес (либо смещение) невозможно закодировать в 32-битной инструкции \INS{BL}, в ней есть место только для 24-х бит.
Поскольку все инструкции в режиме ARM имеют длину 4 байта (32 бита) и инструкции могут находится только по адресам кратным 4, то последние 2 бита (всегда нулевых) можно не кодировать.
В итоге имеем 26 бит, при помощи которых можно закодировать $current\_PC \pm{} \approx{}32M$.

\myindex{ARM!\Instructions!MOV}
Следующая инструкция \INS{MOV R0, \#0}\footnote{Означает MOVe}
просто записывает 0 в регистр \Reg{0}.
Ведь наша Си-функция возвращает 0, а возвращаемое значение всякая функция оставляет в \Reg{0}.

\myindex{ARM!\Registers!Link Register}
\myindex{ARM!\Instructions!LDMFD}
\myindex{ARM!\Instructions!POP}
Последняя инструкция \INS{LDMFD SP!, {R4,PC}}\footnote{\ac{LDMFD}~--- это инструкция, обратная \ac{STMFD}}.
Она загружает из стека (или любого другого места в памяти) значения для сохранения их в \Reg{4} и \ac{PC}, увеличивая \glslink{stack pointer}{указатель стека} \ac{SP}.
Здесь она работает как аналог \POP.\\
N.B. Самая первая инструкция \TT{STMFD} сохранила в стеке \Reg{4} и \ac{LR}, а \IT{восстанавливаются} во время исполнения \TT{LDMFD} регистры \Reg{4} и \ac{PC}.

Как мы уже знаем, в регистре \ac{LR} обычно сохраняется адрес места, куда нужно всякой функции вернуть управление.
Самая первая инструкция сохраняет это значение в стеке, потому что наша функция \main позже будет сама пользоваться этим регистром в момент вызова \printf.
А затем, в конце функции, это значение можно сразу записать прямо в \ac{PC}, таким образом, передав управление туда, откуда была вызвана наша функция.

Так как функция \main обычно самая главная в \CCpp, управление будет возвращено в загрузчик \ac{OS}, либо куда-то в \ac{CRT} 
или что-то в этом роде.

Всё это позволяет избавиться от инструкции \INS{BX LR} в самом конце функции.

\myindex{ARM!DCB}
\TT{DCB}~--- директива ассемблера, описывающая массивы байт или ASCII-строк, аналог директивы DB в x86-ассемблере.


\subsubsection{\NonOptimizingKeilVI (\ThumbMode)}

Скомпилируем тот же пример в Keil для режима Thumb:

\begin{lstlisting}
armcc.exe --thumb --c90 -O0 1.c 
\end{lstlisting}

Получим (в \IDA):

\begin{lstlisting}[caption=\NonOptimizingKeilVI (\ThumbMode) + \IDA,style=customasmARM]
.text:00000000             main
.text:00000000 10 B5          PUSH    {R4,LR}
.text:00000002 C0 A0          ADR     R0, aHelloWorld ; "hello, world"
.text:00000004 06 F0 2E F9    BL      __2printf
.text:00000008 00 20          MOVS    R0, #0
.text:0000000A 10 BD          POP     {R4,PC}

.text:00000304 68 65 6C 6C+aHelloWorld  DCB "hello, world",0    ; DATA XREF: main+2
\end{lstlisting}

Сразу бросаются в глаза двухбайтные (16-битные) опкоды --- это, как уже было отмечено, Thumb.

\myindex{ARM!\Instructions!BL}
Кроме инструкции \TT{BL}.
Но на самом деле она состоит из двух 16-битных инструкций.
Это потому что в одном 16-битном опкоде слишком мало места для задания смещения, по которому находится функция \printf.
Так что первая 16-битная инструкция загружает старшие 10 бит смещения, а вторая~--- младшие 11 бит смещения.

% TODO:
% BL has space for 11 bits, so if we don't encode the lowest bit,
% then we should get 11 bits for the upper half, and 12 bits for the lower half.
% And the highest bit encodes the sign, so the destination has to be within
% \pm 4M of current_PC.
% This may be less if adding the lower half does not carry over,
% but I'm not sure --all my programs have 0 for the upper half,
% and don't carry over for the lower half.
% It would be interesting to check where __2printf is located relative to 0x8
% (I think the program counter is the next instruction on a multiple of 4
% for THUMB).
% The lower 11 bytes of the BL instructions and the even bit are
% 000 0000 0110 | 001 0010 1110 0 = 000 0000 0110 0010 0101 1100 = 0x00625c,
% so __2printf should be at 0x006264.
% But if we only have 10 and 11 bits, then the offset would be:
% 00 0000 0110 | 01 0010 1110 0 = 0 0000 0011 0010 0101 1100 = 0x00325c,
% so __2printf should be at 0x003264.
% In this case, though, the new program counter can only be 1M away,
% because of the highest bit is used for the sign.

Как уже было упомянуто, все инструкции в Thumb-режиме имеют длину 2 байта (или 16 бит).
Поэтому невозможна такая ситуация, когда Thumb-инструкция начинается по нечетному адресу.

Учитывая сказанное, последний бит адреса можно не кодировать.
Таким образом, в Thumb-инструкции \TT{BL} можно закодировать адрес $current\_PC \pm{}\approx{}2M$.

\myindex{ARM!\Instructions!PUSH}
\myindex{ARM!\Instructions!POP}
Остальные инструкции в функции (\PUSH и \POP) здесь работают почти так же, как и описанные \TT{STMFD}/\TT{LDMFD}, только регистр \ac{SP} здесь не указывается явно.
\INS{ADR} работает так же, как и в предыдущем примере.
\INS{MOVS} записывает 0 в регистр \Reg{0} для возврата нуля.


\subsubsection{\OptimizingXcodeIV (\ARMMode)}

Xcode 4.6.3 без включенной оптимизации выдает слишком много лишнего кода, поэтому включим оптимизацию компилятора (ключ \Othree), потому что там меньше инструкций.

\begin{lstlisting}[caption=\OptimizingXcodeIV (\ARMMode),style=customasm]
__text:000028C4             _hello_world
__text:000028C4 80 40 2D E9   STMFD           SP!, {R7,LR}
__text:000028C8 86 06 01 E3   MOV             R0, #0x1686
__text:000028CC 0D 70 A0 E1   MOV             R7, SP
__text:000028D0 00 00 40 E3   MOVT            R0, #0
__text:000028D4 00 00 8F E0   ADD             R0, PC, R0
__text:000028D8 C3 05 00 EB   BL              _puts
__text:000028DC 00 00 A0 E3   MOV             R0, #0
__text:000028E0 80 80 BD E8   LDMFD           SP!, {R7,PC}

__cstring:00003F62 48 65 6C 6C+aHelloWorld_0  DCB "Hello world!",0
\end{lstlisting}

Инструкции \TT{STMFD} и \TT{LDMFD} нам уже знакомы.

\myindex{ARM!\Instructions!MOV}
Инструкция \MOV просто записывает число \TT{0x1686} в регистр \Reg{0}~--- это смещение, указывающее на строку \q{Hello world!}.

Регистр \Reg{7} (по стандарту, принятому в \IOSABI) это frame pointer, о нем будет рассказано позже.

\myindex{ARM!\Instructions!MOVT}
Инструкция \TT{MOVT R0, \#0} (MOVe Top) записывает 0 в старшие 16 бит регистра.
Дело в том, что обычная инструкция \MOV в режиме ARM может записывать какое-либо значение только в младшие 16 бит регистра, ведь в ней нельзя закодировать больше.
Помните, что в режиме ARM опкоды всех инструкций ограничены длиной в 32 бита. Конечно, это ограничение не касается перемещений данных между регистрами.

Поэтому для записи в старшие биты (с 16-го по 31-й включительно) существует дополнительная команда \INS{MOVT}.
Впрочем, здесь её использование избыточно, потому что инструкция \INS{MOV R0, \#0x1686} выше и так обнулила старшую часть регистра.
Возможно, это недочет компилятора.
% TODO:
% I think, more specifically, the string is not put in the text section,
% ie. the compiler is actually not using position-independent code,
% as mentioned in the next paragraph.
% MOVT is used because the assembly code is generated before the relocation,
% so the location of the string is not yet known,
% and the high bits may still be needed.

\myindex{ARM!\Instructions!ADD}
Инструкция \TT{ADD R0, PC, R0} прибавляет \ac{PC} к \Reg{0} для вычисления действительного адреса строки \q{Hello world!}. Как нам уже известно, это \q{\PICcode}, поэтому такая корректива необходима.

Инструкция \TT{BL} вызывает \puts вместо \printf.

\label{puts}
\myindex{\CStandardLibrary!puts()}
\myindex{puts() вместо printf()}
Компилятор заменил вызов \printf на \puts. 
Действительно, \printf с одним аргументом это почти аналог \puts.
 
\IT{Почти}, если принять условие, что в строке не будет управляющих символов \printf, 
начинающихся со знака процента. Тогда эффект от работы этих двух функций будет разным
\footnote{Также нужно заметить, что \puts не требует символа перевода строки `\textbackslash{}n' в конце строки,
поэтому его здесь нет.}.

Зачем компилятор заменил один вызов на другой? Наверное потому что \puts работает быстрее
\footnote{\href{http://go.yurichev.com/17063}{ciselant.de/projects/gcc\_printf/gcc\_printf.html}}. 
Видимо потому что \puts проталкивает символы в \gls{stdout} не сравнивая каждый со знаком процента.

Далее уже знакомая инструкция \TT{MOV R0, \#0}, служащая для установки в 0 возвращаемого значения функции.


\subsubsection{\OptimizingXcodeIV (\ThumbTwoMode)}

По умолчанию Xcode 4.6.3 генерирует код для режима Thumb-2 примерно в такой манере:

\begin{lstlisting}[caption=\OptimizingXcodeIV (\ThumbTwoMode)]
__text:00002B6C                   _hello_world
__text:00002B6C 80 B5          PUSH            {R7,LR}
__text:00002B6E 41 F2 D8 30    MOVW            R0, #0x13D8
__text:00002B72 6F 46          MOV             R7, SP
__text:00002B74 C0 F2 00 00    MOVT.W          R0, #0
__text:00002B78 78 44          ADD             R0, PC
__text:00002B7A 01 F0 38 EA    BLX             _puts
__text:00002B7E 00 20          MOVS            R0, #0
__text:00002B80 80 BD          POP             {R7,PC}

...

__cstring:00003E70 48 65 6C 6C 6F 20+aHelloWorld  DCB "Hello world!",0xA,0
\end{lstlisting}

% Q: If you subtract 0x13D8 from 0x3E70,
% you actually get a location that is not in this function, or in _puts.
% How is PC-relative addressing done in THUMB2?
% A: it's not Thumb-related. there are just mess with two different segments. TODO: rework this listing.

\myindex{\ThumbTwoMode}
\myindex{ARM!\Instructions!BL}
\myindex{ARM!\Instructions!BLX}
Инструкции \TT{BL} и \TT{BLX} в Thumb, как мы помним, кодируются как пара 16-битных инструкций, 
а в Thumb-2 эти \IT{суррогатные} опкоды расширены так, что новые инструкции кодируются здесь как 
32-битные инструкции.
Это можно заметить по тому что опкоды Thumb-2 инструкций всегда начинаются с \TT{0xFx} либо с \TT{0xEx}.
Но в листинге \IDA байты опкода переставлены местами.
Это из-за того, что в процессоре ARM инструкции кодируются так:
в начале последний байт, потом первый (для Thumb и Thumb-2 режима), либо, 
(для инструкций в режиме ARM) в начале четвертый байт, затем третий, второй и первый 
(т.е. другой \gls{endianness}).

Вот так байты следуют в листингах IDA:

\begin{itemize}
\item для режимов ARM и ARM64: 4-3-2-1;
\item для режима Thumb: 2-1;
\item для пары 16-битных инструкций в режиме Thumb-2: 2-1-4-3.
\end{itemize}

\myindex{ARM!\Instructions!MOVW}
\myindex{ARM!\Instructions!MOVT.W}
\myindex{ARM!\Instructions!BLX}
Так что мы видим здесь что инструкции \TT{MOVW}, \TT{MOVT.W} и \TT{BLX} начинаются с \TT{0xFx}.

Одна из Thumb-2 инструкций это
\TT{MOVW R0, \#0x13D8}~--- она записывает 16-битное число в младшую часть регистра \Reg{0}, очищая старшие биты.

Ещё \TT{MOVT.W R0, \#0}~--- эта инструкция работает так же, как и \TT{MOVT} из предыдущего примера, но она работает в Thumb-2.

\myindex{ARM!переключение режимов}
\myindex{ARM!\Instructions!BLX}
Помимо прочих отличий, здесь используется инструкция \TT{BLX} вместо \TT{BL}.
Отличие в том, что помимо сохранения адреса возврата в регистре \ac{LR} и передаче управления 
в функцию \puts, происходит смена режима процессора с Thumb/Thumb-2 на режим ARM (либо назад).
Здесь это нужно потому, что инструкция, куда ведет переход, выглядит так (она закодирована в режиме ARM):

\begin{lstlisting}[style=customasm]
__symbolstub1:00003FEC _puts           ; CODE XREF: _hello_world+E
__symbolstub1:00003FEC 44 F0 9F E5     LDR  PC, =__imp__puts
\end{lstlisting}

Это просто переход на место, где записан адрес \puts в секции импортов.
Итак, внимательный читатель может задать справедливый вопрос: почему бы не вызывать \puts сразу в 
том же месте кода, где он нужен?
Но это не очень выгодно из-за экономии места и вот почему.

\myindex{Динамически подгружаемые библиотеки}
Практически любая программа использует внешние динамические библиотеки (будь то DLL в Windows, .so в *NIX 
либо .dylib в \MacOSX).
В динамических библиотеках находятся часто используемые библиотечные функции, в том числе стандартная функция Си \puts.

\myindex{Relocation}
В исполняемом бинарном файле 
(Windows PE .exe, ELF либо Mach-O) имеется секция импортов, список символов (функций либо глобальных переменных) импортируемых из внешних модулей, а также названия самих модулей.
Загрузчик \ac{OS} загружает необходимые модули и, перебирая импортируемые символы в основном модуле, проставляет правильные адреса каждого символа.
В нашем случае, \IT{\_\_imp\_\_puts} 
это 32-битная переменная, куда загрузчик \ac{OS} запишет правильный адрес этой же функции во внешней библиотеке. 
Так что инструкция \TT{LDR} просто берет 32-битное значение из этой переменной, и, записывая его в регистр \ac{PC}, просто передает туда управление.
Чтобы уменьшить время работы загрузчика \ac{OS}, нужно чтобы ему пришлось записать адрес каждого символа только один раз, в соответствующее, выделенное для них, место.

\myindex{thunk-функции}
К тому же, как мы уже убедились, нельзя одной инструкцией загрузить в регистр 32-битное число без обращений к памяти.
Так что наиболее оптимально выделить отдельную функцию, работающую в режиме ARM, 
чья единственная цель~--- передавать управление дальше, в динамическую библиотеку.
И затем ссылаться на эту короткую функцию из одной инструкции (так называемую \glslink{thunk function}{thunk-функцию}) из Thumb-кода.

\myindex{ARM!\Instructions!BL}
Кстати, в предыдущем примере (скомпилированном для режима ARM), переход при помощи инструкции \TT{BL} ведет 
на такую же \glslink{thunk function}{thunk-функцию}, однако режим процессора не переключается (отсюда отсутствие \q{X} в мнемонике инструкции).

\myparagraph{Еще о thunk-функциях}
\myindex{thunk-функции}

Thunk-функции трудновато понять, скорее всего, из-за путаницы в терминах.
Проще всего представлять их как адаптеры-переходники из одного типа разъемов в другой.
Например, адаптер позволяющее вставить в американскую розетку британскую вилку, или наоборот.
Thunk-функции также иногда называются \IT{wrapper-ами}. \IT{Wrap} в английском языке это \IT{обертывать}, \IT{завертывать}.
Вот еще несколько описаний этих функций:

\begin{framed}
\begin{quotation}
“A piece of coding which provides an address:”, according to P. Z. Ingerman, 
who invented thunks in 1961 as a way of binding actual parameters to their formal 
definitions in Algol-60 procedure calls. If a procedure is called with an expression 
in the place of a formal parameter, the compiler generates a thunk which computes 
the expression and leaves the address of the result in some standard location.

\dots

Microsoft and IBM have both defined, in their Intel-based systems, a “16-bit environment” 
(with bletcherous segment registers and 64K address limits) and a “32-bit environment” 
(with flat addressing and semi-real memory management). The two environments can both be 
running on the same computer and OS (thanks to what is called, in the Microsoft world, 
WOW which stands for Windows On Windows). MS and IBM have both decided that the process 
of getting from 16- to 32-bit and vice versa is called a “thunk”; for Windows 95, 
there is even a tool, THUNK.EXE, called a “thunk compiler”.
\end{quotation}
\end{framed}
% TODO FIXME move to bibliography and quote properly above the quote
( \href{http://go.yurichev.com/17362}{The Jargon File} )

\myindex{LAPACK}
\myindex{FORTRAN}
Еще один пример мы можем найти в библиотеке LAPACK --- (``Linear Algebra PACKage'') написанная на FORTRAN.
Разработчики на \CCpp также хотят использовать LAPACK, но переписывать её на \CCpp, а затем поддерживать несколько версий,
это безумие.
Так что имеются короткие функции на Си вызываемые из \CCpp{}-среды, которые, в свою очередь, вызывают функции на FORTRAN,
и почти ничего больше не делают:

\begin{lstlisting}
double Blas_Dot_Prod(const LaVectorDouble &dx, const LaVectorDouble &dy)
{
    assert(dx.size()==dy.size());
    integer n = dx.size();
    integer incx = dx.inc(), incy = dy.inc();

    return F77NAME(ddot)(&n, &dx(0), &incx, &dy(0), &incy);
}
\end{lstlisting}

Такие ф-ции еще называют ``wrappers'' (т.е., ``обертка'').


\subsubsection{ARM64}

\myparagraph{GCC}

Компилируем пример в GCC 4.8.1 для ARM64:

\lstinputlisting[numbers=left,label=hw_ARM64_GCC,caption=\NonOptimizing GCC 4.8.1 + objdump,style=customasmARM]{patterns/01_helloworld/ARM/hw.lst}

В ARM64 нет режима Thumb и Thumb-2, только ARM, так что тут только 32-битные инструкции.

Регистров тут в 2 раза больше: \myref{ARM64_GPRs}.
64-битные регистры теперь имеют префикс 
\TT{X-}, а их 32-битные части --- \TT{W-}.

\myindex{ARM!\Instructions!STP}
Инструкция \TT{STP} (\IT{Store Pair}) 
сохраняет в стеке сразу два регистра: \RegX{29} и \RegX{30}.
Конечно, эта инструкция может сохранять эту пару где угодно в памяти, но здесь указан регистр \ac{SP}, так что
пара сохраняется именно в стеке.

Регистры в ARM64 64-битные, каждый имеет длину в 8 байт, так что для хранения двух регистров нужно именно 16 байт.

Восклицательный знак (``!'') после операнда означает, что сначала от \ac{SP} будет отнято 16 и только затем
значения из пары регистров будут записаны в стек.

Это называется \IT{pre-index}.
Больше о разнице между \IT{post-index} и \IT{pre-index} 
описано здесь: \myref{ARM_postindex_vs_preindex}.

Таким образом, в терминах более знакомого всем процессора x86, первая инструкция~--- это просто аналог 
пары инструкций \TT{PUSH X29} и \TT{PUSH X30}.
\RegX{29} в ARM64 используется как \ac{FP}, а \RegX{30} 
как \ac{LR}, поэтому они сохраняются в прологе функции и
восстанавливаются в эпилоге.

Вторая инструкция копирует \ac{SP} в \RegX{29} (или \ac{FP}).
Это нужно для установки стекового фрейма функции.

\label{pointers_ADRP_and_ADD}
\myindex{ARM!\Instructions!ADRP/ADD pair}
Инструкции \TT{ADRP} и \ADD нужны для формирования адреса строки \q{Hello!} в регистре \RegX{0}, 
ведь первый аргумент функции передается через этот регистр.
Но в ARM нет инструкций, при помощи которых можно записать в регистр длинное число 
(потому что сама длина инструкции ограничена 4-я байтами. Больше об этом здесь: \myref{ARM_big_constants_loading}).
Так что нужно использовать несколько инструкций.
Первая инструкция (\TT{ADRP}) записывает в \RegX{0} адрес 4-килобайтной страницы где находится строка, 
а вторая (\ADD) просто прибавляет к этому адресу остаток.
Читайте больше об этом: \myref{ARM64_relocs}.

\TT{0x400000 + 0x648 = 0x400648}, и мы видим, что в секции данных \TT{.rodata} по этому адресу как раз находится наша
Си-строка \q{Hello!}.

\myindex{ARM!\Instructions!BL}
Затем при помощи инструкции \TT{BL} вызывается \puts. Это уже рассматривалось ранее: \myref{puts}.

Инструкция \MOV записывает 0 в \RegW{0}. 
\RegW{0} это младшие 32 бита 64-битного регистра \RegX{0}:

\begin{center}
\begin{tabular}{ | l | l | }
\hline
\RU{Старшие 32 бита}\EN{High 32-bit part}\ES{Parte alta de 32 bits}\PTBRph{}\PLph{}\ITAph{}\DEph{}\THAph{}\NLph{}\FR{Partie 32 bits haute} & \RU{младшие 32 бита}\EN{low 32-bit part}\ES{parte baja de 32 bits}\PTBRph{}\PLph{Starsze 32 bity}\ITAph{}\DEph{}\THAph{}\NLph{}\FR{Partie 32 bits basse} \\
\hline
\multicolumn{2}{ | c | }{X0} \\
\hline
\multicolumn{1}{ | c | }{} & \multicolumn{1}{ c | }{W0} \\
\hline
\end{tabular}
\end{center}


А результат функции возвращается через \RegX{0}, и \main возвращает 0, 
так что вот так готовится возвращаемый результат.

Почему именно 32-битная часть?
Потому что в ARM64, как и в x86-64, тип \Tint оставили 32-битным, для лучшей совместимости.

Следовательно, раз уж функция возвращает 32-битный \Tint, то нужно заполнить только 32 младших бита регистра \RegX{0}.

Для того, чтобы удостовериться в этом, немного отредактируем этот пример и перекомпилируем его.%

Теперь \main возвращает 64-битное значение:

\begin{lstlisting}[caption=\main возвращающая значение типа \TT{uint64\_t},style=customc]
#include <stdio.h>
#include <stdint.h>

uint64_t main()
{
        printf ("Hello!\n");
        return 0;
}
\end{lstlisting}

Результат точно такой же, только \MOV в той строке теперь выглядит так:

\begin{lstlisting}[caption=\NonOptimizing GCC 4.8.1 + objdump]
  4005a4:       d2800000        mov     x0, #0x0      // #0
\end{lstlisting}

\myindex{ARM!\Instructions!LDP}
Далее при помощи инструкции \INS{LDP} (\IT{Load Pair}) восстанавливаются регистры \RegX{29} и \RegX{30}.

Восклицательного знака после инструкции нет. Это означает, что сначала значения достаются из стека, и только потом \ac{SP} увеличивается на 16.

Это называется \IT{post-index}.

\myindex{ARM!\Instructions!RET}
В ARM64 есть новая инструкция: \RET. 
Она работает так же как и \INS{BX LR}, но там добавлен специальный бит,
подсказывающий процессору, что это именно выход из функции, а не просто переход, чтобы процессор
мог более оптимально исполнять эту инструкцию.

Из-за простоты этой функции оптимизирующий GCC генерирует точно такой же код.



