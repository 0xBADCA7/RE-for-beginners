\subsection{ARM}
\label{sec:hw_ARM}

\myindex{\idevices}
\myindex{Raspberry Pi}
\myindex{Xcode}
\myindex{LLVM}
\myindex{Keil}
Do eksperymentów z ARM skorzystamy z kilku kompilatorów:

\begin{itemize}
\item Popularny w embedded-środowisku Keil Release 6/2013.

\item Apple Xcode 4.6.3 z kompilatorem LLVM-GCC 4.2
\footnote{W rzeczywistości Apple Xcode 4.6.3 korzysta z open-source kompilatora GCC jako pierwszoplanowego kompilatora i generatora kodu LLVM}.

\item GCC 4.9 (Linaro) (dla ARM64), 
jest dostępny w postaci pliku wykonywalnego dla win32 na \url{http://go.yurichev.com/17325}.

\end{itemize}

Wszędzie w tej książce, jeżeli nie jest napisane co innego, mówimy o 32-bitowym ARM (w tym w trybach Thumb i Thumb-2).
Kiedy mówimy o 64-bitowym ARM, to on będzie tu oznaczony jako ARM64.

% subsections
\subsubsection{\NonOptimizingKeilVI (\ARMMode)}

Na początek skompilujmy nasz przykład w Keil:

\begin{lstlisting}
armcc.exe --arm --c90 -O0 1.c 
\end{lstlisting}

\myindex{\IntelSyntax}
Kompilator \IT{armcc} generuje listing w asemblerze w formacie Intel.
Ten listing zawiera niektóre wysokopoziomowe makro, powiązane z ARM
\footnote{naprzykład, on korzysta z instrukcji \PUSH/\POP, których nie ma w trybie ARM},
dlatego zobaczmy jak wygląda skompilowany kod w \IDA.

\begin{lstlisting}[caption=\NonOptimizingKeilVI (\ARMMode) \IDA,style=customasmARM]
.text:00000000             main
.text:00000000 10 40 2D E9    STMFD   SP!, {R4,LR}
.text:00000004 1E 0E 8F E2    ADR     R0, aHelloWorld ; "hello, world"
.text:00000008 15 19 00 EB    BL      __2printf
.text:0000000C 00 00 A0 E3    MOV     R0, #0
.text:00000010 10 80 BD E8    LDMFD   SP!, {R4,PC}

.text:000001EC 68 65 6C 6C+aHelloWorld  DCB "hello, world",0    ; DATA XREF: main+4
\end{lstlisting}

W przykładzie wyżej można łatwo dostrzec, że każda instrukcja ma rozmiar 4 bajty.
Rzeczywiście, przecież kompilowaliśmy nasz kod dla trybu ARM, a nie Thumb.

\myindex{ARM!\Instructions!STMFD}
\myindex{ARM!\Instructions!POP}
Pierwsza instrukcja, \INS{STMFD SP!, \{R4,LR\}}\footnote{\ac{STMFD}},
działa jak instrukcja \PUSH w x86: odkłada wartości dwóch rejestrów (\Reg{4} i \ac{LR}) na stos.
W listingu w asemblerze kompilator \IT{armcc}, żeby uprościć sprawę pokazuje tu instrukcję
\INS{PUSH \{r4,lr\}}.
Ale nie jest to dokładnie to co chcieliibyśmy zobaczyć, jako że instrukcja \PUSH jest dostępna tylko w trybie Thumb, dlatego,
 zaproponowałem pracować z kompilatorem \IDA.

Ta instrukcja zmniejsza \ac{SP}, żeby on wskazywał na miejsce na stosie, dostępne do zapisu nowych wartości, następnie zapisuje wartości rejestrów \Reg{4} i \ac{LR} 
pod adres w pamięci, na który wskazuje zmodyfikowany rejestr \ac{SP}.

Ta instrukcja, tak samo jak i instrukcja \PUSH w trybie Thumb, pozwala na odkładanie na stos kilku wartości rejestrów na raz, co może się okazać bardzo wygodne.
A propos, w x86 czegoś takiego nie ma.
Należy zauważyć, że \TT{STMFD}~--- jest generalizacją instrukcji \PUSH (czyli rozszerza jej możliwości), dlatego że może operować na różnych rejestrach, a nie tylko na \ac{SP}.
Inaczej mówiąc, z \TT{STMFD} można korzystać przy zapisie zestawu rejestrów we wskazanym miejscu w pamięci.

\myindex{\PICcode}
\myindex{ARM!\Instructions!ADR}
Instrukcja \INS{ADR R0, aHelloWorld} dodaje lub odejmuje zawartość rejestru \ac{PC} do przesunięcia, w którym jest przechowywana linia
\TT{hello, world}.
Co ma do tego \ac{PC}? Jest to tzw \q{\PICcode}
\footnote{będzie to szerzej omówione w kolejnym rozdziale ~(\myref{sec:PIC})}.
Nie jest on przywiązany do jakiegokolwiek miejsca w pamięci.
Inaczej mówiąc, jest to \ac{PC}-względne adresowanie.
W opcode instrukcji \TT{ADR} jest zapisana różnica między adresem tej instrukcji a miejscem, w którym jest przechowywana linia.
Ta różnica zawsze jest stała, niezależnie od tego w którym miejscu został załadowany \ac{OS} nasz kod.
Dlatego, wszystko czego potrzebujemy~--- to dodanie adresu bieżącej instrukcji (z \ac{PC}), żeby otrzymać bieżący adres bezwględny linii.

\myindex{ARM!\Registers!Link Register}
\myindex{ARM!\Instructions!BL}
Instrukcja \INS{BL \_\_2printf}\footnote{Branch with Link} wywołuje f-cje \printf.
Działanie tej f-cji składa się z 2 kroków:

\begin{itemize}
\item zapisać adres występujący po instrukcji \INS{BL} (\TT{0xC}) do rejestru \ac{LR};
\item przekazać zarządzanie do \printf, zapisując adres tej f-cji do \ac{PC}.
\end{itemize}

Kiedy f-cja \printf skończy pracę, procesor musi wiedzieć, gdzie zwrócić zarządzanie, dlatego kończąc pracę, każda funkcja przekazuje zarządzanie pod adres zapisany w rejestrze \ac{LR}.

Na tym polega główna różnica między \ac{RISC}-procesorami typu ARM i \ac{CISC}-procesorami typu x86,
gdzie adres powrotu zwykle jest odkładany na stos ~(\myref{sec:stack}).

A propos tego,  nie jest możliwe zakodowanie 32-bitowego adresu bezwzględnego (lub przesunięcia) w 32-bitowej instrukcji \INS{BL}, jako że ona ma miejsce tylko dla 24-ch bitów.
Jako że wszystkie instrukcje w trybie ARM mają długość 4 bajty (32 bity) i instrukcje mogą się znajdować tylko pod adresem wielokrotnym 4, to ostatnie 2 bity (które są zawsze zerowe) można nie kodować.
W wyniku mamy 26 bitów, za pomocą których można zakodować $current\_PC \pm{} \approx{}32M$.

\myindex{ARM!\Instructions!MOV}
Następna instrukcja \INS{MOV R0, \#0}\footnote{oznacza MOV}
po prostu zapisuje 0 do rejestru \Reg{0}.
To dlatego, że nasza f-cja zwraca 0, a wartość zwróconą każda f-cja zostawia w \Reg{0}.

\myindex{ARM!\Registers!Link Register}
\myindex{ARM!\Instructions!LDMFD}
\myindex{ARM!\Instructions!POP}
Ostatnia instrukcja \INS{LDMFD SP!, {R4,PC}}\footnote{\ac{LDMFD}~--- jest instrukcją względem działania odwrotną do \ac{STMFD}}.
Ona ładuje ze stosu (lub każdego innego miejsca w pamięci) wartości do zapisu w \Reg{4} i \ac{PC}, zwiększając \glslink{stack pointer}{wskaźnik stosu} \ac{SP}.
Tutaj ona działa analogicznie do \POP.\\
N.B. Pierwsza instrukcja \TT{STMFD} odłożyła na stos \Reg{4} i \ac{LR}, a \IT{przywracane} w wyniku działania \TT{LDMFD} są rejestry \Reg{4} i \ac{PC}.

Jak już wiemy, do rejestru \ac{LR} zwykle jest zapisywany adres miejsca w pamięci, pod który f-cja będzie zwracała zarządzanie.
Pierwsza isntrukcja odkłada tę wartość na stos, dlatego że nasza f-cja \main później sama będzie korzystała z tego rejstru w momencie wywołania f-cji \printf.
Czyli, na końcu f-cji, tę wartość można było od razu zapisać do \ac{PC}, w ten sposób przekazując zarządzanie tam, skąd nasza f-cja była wywołana.

Jako że, z reguły, f-cja \main jest funkcją główną w \CCpp, zarządzanie zostanie zwróconę do \ac{OS}, lub gdzieś do \ac{CRT} 
lub coś w tym stylu.

Всё это позволяет избавиться от инструкции \INS{BX LR} в самом конце функции.

\myindex{ARM!DCB}
\TT{DCB}~--- директива ассемблера, описывающая массивы байт или ASCII-строк, аналог директивы DB в x86-ассемблере.



\subsubsection{\NonOptimizingKeilVI (\ThumbMode)}

Skompilujmy ten sam przykład w Keil dla trybu Thumb:

\begin{lstlisting}
armcc.exe --thumb --c90 -O0 1.c 
\end{lstlisting}

Otrzymamy (w \IDA):

\begin{lstlisting}[caption=\NonOptimizingKeilVI (\ThumbMode) + \IDA,style=customasmARM]
.text:00000000             main
.text:00000000 10 B5          PUSH    {R4,LR}
.text:00000002 C0 A0          ADR     R0, aHelloWorld ; "hello, world"
.text:00000004 06 F0 2E F9    BL      __2printf
.text:00000008 00 20          MOVS    R0, #0
.text:0000000A 10 BD          POP     {R4,PC}

.text:00000304 68 65 6C 6C+aHelloWorld  DCB "hello, world",0    ; DATA XREF: main+2
\end{lstlisting}

Od razu można zauważyć (16-bitowe) opcode --- jest to, jak już było powiedziane, Thumb.

\myindex{ARM!\Instructions!BL}
Oprócz instrukcji \TT{BL}.
Ale tak naprawdę to ona się składa z dwóch 16-bitowych instrukcji.
Tak się dzieje dlatego że w jednym 16-bitowym opcode jest za mało miejsca dla przesunięcia, po którym znajduje się funkcja \printf.
Dlatego pierwsza 16-bitowa instrukcja ładuje starsze 10 bitów przesunięcia, a druga~--- młodsze 11 bitów przesunięcia.

% TODO:
% BL has space for 11 bits, so if we don't encode the lowest bit,
% then we should get 11 bits for the upper half, and 12 bits for the lower half.
% And the highest bit encodes the sign, so the destination has to be within
% \pm 4M of current_PC.
% This may be less if adding the lower half does not carry over,
% but I'm not sure --all my programs have 0 for the upper half,
% and don't carry over for the lower half.
% It would be interesting to check where __2printf is located relative to 0x8
% (I think the program counter is the next instruction on a multiple of 4
% for THUMB).
% The lower 11 bytes of the BL instructions and the even bit are
% 000 0000 0110 | 001 0010 1110 0 = 000 0000 0110 0010 0101 1100 = 0x00625c,
% so __2printf should be at 0x006264.
% But if we only have 10 and 11 bits, then the offset would be:
% 00 0000 0110 | 01 0010 1110 0 = 0 0000 0011 0010 0101 1100 = 0x00325c,
% so __2printf should be at 0x003264.
% In this case, though, the new program counter can only be 1M away,
% because of the highest bit is used for the sign.

jak już było powiedziane, wszystkie instrukcje w trybie Thumb są 2 bajtowe (lub 16 bitowe).
Dlatego sytuacja, w której Thumb-instrukcja zaczyna się pod adresem nieparzystym, jest niemożliwa.

Wniosek z tego jest taki, że ostatniego bitu adresu można nie kodować.
W ten sposób, w Thumb-instrukcji \TT{BL} można zakodować adres $current\_PC \pm{}\approx{}2M$.

\myindex{ARM!\Instructions!PUSH}
\myindex{ARM!\Instructions!POP}
Reszta instrukcji w funkcji (\PUSH i \POP) działa prawie tak samo, jak i opisane wyżej \TT{STMFD}/\TT{LDMFD}, tyle że rejestr \ac{SP} nie jest tu wskazany w sposób jawny.
\INS{ADR} działa dokładnie tak samo, jak i w poprzednim przykładzie.
\INS{MOVS} zapisuje wartość 0 do rejestru \Reg{0}, żeby f-cja zwróciła zero.



\subsubsection{\OptimizingXcodeIV (\ARMMode)}

Xcode 4.6.3 bez włączonego trybu optymalizacji generuje za dużo zbędnego kodu, dlatego włączymy optymalizacje kodu (flaga \Othree).

\begin{lstlisting}[caption=\OptimizingXcodeIV (\ARMMode),style=customasmARM]
__text:000028C4             _hello_world
__text:000028C4 80 40 2D E9   STMFD           SP!, {R7,LR}
__text:000028C8 86 06 01 E3   MOV             R0, #0x1686
__text:000028CC 0D 70 A0 E1   MOV             R7, SP
__text:000028D0 00 00 40 E3   MOVT            R0, #0
__text:000028D4 00 00 8F E0   ADD             R0, PC, R0
__text:000028D8 C3 05 00 EB   BL              _puts
__text:000028DC 00 00 A0 E3   MOV             R0, #0
__text:000028E0 80 80 BD E8   LDMFD           SP!, {R7,PC}

__cstring:00003F62 48 65 6C 6C+aHelloWorld_0  DCB "Hello world!",0
\end{lstlisting}

Instrukcje \TT{STMFD} i \TT{LDMFD} już są nam znane.

\myindex{ARM!\Instructions!MOV}
Instrukcja \MOV po prostu zapisuje liczbę \TT{0x1686} do rejestru \Reg{0}~--- jest to przesunięcie, wskazujące na linię \q{Hello world!}.

Rejestr \Reg{7} (wg standardu \IOSABI) jest frame pointer, będzie to omawiane później.

\myindex{ARM!\Instructions!MOVT}
Instrukcja \TT{MOVT R0, \#0} (MOVe Top) zapisuje 0 do starszych 16 bitów rejestru.
Rzecz polega na tym, że zwykła instrukcja \MOV w trybie ARM może zapisywać jakąkolwiek wartość tylko do młodszych 16 bitów rejestru, dlatego że nie można w niej zakodować więcej.
Warto pamiętać, że w trybie ARM opcode wszystkich instrukcji mogą wynosić maksymalnie 32 bity. Oczywiście, nie jest to prawda dla przemieszczenia danych między rejestrami.

Dlatego do zapisywania do starszych bitów (bity 16-31) istnieje dodatkowa instrukcja \INS{MOVT}.
Tutaj, natomiast, jej wykorzystanie jest zbędne, jako że \INS{MOV R0, \#0x1686} i bez tego wyzerowało starszą część rejestru.
Możliwe, że jest to niedociągnięcie kompilatora.
% TODO:
% I think, more specifically, the string is not put in the text section,
% ie. the compiler is actually not using position-independent code,
% as mentioned in the next paragraph.
% MOVT is used because the assembly code is generated before the relocation,
% so the location of the string is not yet known,
% and the high bits may still be needed.

\myindex{ARM!\Instructions!ADD}
Instrukcja \TT{ADD R0, PC, R0} dodaje \ac{PC} do \Reg{0} żeby wyliczyć adres rzeczywisty linii \q{Hello world!}. Jak już wiemy, jest to \q{\PICcode}, dlatego taka korektywa jest niezbędna.

Instrukcja \TT{BL} wywołuje \puts zamiast \printf.

\label{puts}
\myindex{\CStandardLibrary!puts()}
\myindex{puts() zamiast printf()}
Kompilator zamienia wywołanie \printf na \puts. 
Rzeczywiście, \printf z jednym argumentem jest prawie analogiczne do \puts.
 
\IT{Prawie}, jeśli założyć, że w linii nie będzie znaków sterujących \printf, 
zaczynających się od znaku procentu. Wtedy wyniki pracy tych dwóch f-cji będą różne
\footnote{Należy również zauważyć, że \puts nie potrzebuje znaku nowej linii `\textbackslash{}n',
dlatego jego tu nie ma.}.

Po co kompilator zamienił wywołanie jednej f-cji na inną? Prawdopodobnie, dlatego że \puts jest szybsze
\footnote{\href{http://go.yurichev.com/17063}{ciselant.de/projects/gcc\_printf/gcc\_printf.html}}. 
Najwidoczniej, dlatego że \puts wyprowadza symbole na \gls{stdout} nie porównując każdy ze znakiem procentu.

Dalej jest już znana instrukcja \TT{MOV R0, \#0}, ustawiająca 0 jako wartość zwracaną.



\subsubsection{\OptimizingXcodeIV (\ThumbTwoMode)}

Domyślnie Xcode 4.6.3 generuje kod dla Thumb-2 wyglądający mniej więcej tak:

\begin{lstlisting}[caption=\OptimizingXcodeIV (\ThumbTwoMode),style=customasmARM]
__text:00002B6C                   _hello_world
__text:00002B6C 80 B5          PUSH            {R7,LR}
__text:00002B6E 41 F2 D8 30    MOVW            R0, #0x13D8
__text:00002B72 6F 46          MOV             R7, SP
__text:00002B74 C0 F2 00 00    MOVT.W          R0, #0
__text:00002B78 78 44          ADD             R0, PC
__text:00002B7A 01 F0 38 EA    BLX             _puts
__text:00002B7E 00 20          MOVS            R0, #0
__text:00002B80 80 BD          POP             {R7,PC}

...

__cstring:00003E70 48 65 6C 6C 6F 20+aHelloWorld  DCB "Hello world!",0xA,0
\end{lstlisting}

% Q: If you subtract 0x13D8 from 0x3E70,
% you actually get a location that is not in this function, or in _puts.
% How is PC-relative addressing done in THUMB2?
% A: it's not Thumb-related. there are just mess with two different segments. TODO: rework this listing.

\myindex{\ThumbTwoMode}
\myindex{ARM!\Instructions!BL}
\myindex{ARM!\Instructions!BLX}
Instrukcje \TT{BL} i \TT{BLX} w Thumb, są kodowane jako para 16-bitowych instrukcji, 
a w Thumb-2 te opcode są rozszerzone w ten sposób, że nowe instrukcje tutaj są kodowane jako 32-bitowe instrukcje.
Można to łatwo zauważyc, jako że w Thumb-2 instrukcje zaczynają się od \TT{0xFx} lub od \TT{0xEx}.
Ale w listingu \IDA bajty opcode są zamienione miejscami.
Dzieje się tak dlatego, że w procesorze ARM instrukcje są kodowane w ten sposób:
najpierw podaje się ostatni bajt, potem pierwszy (dla trybów Thumb i Thumb-2), lub, 
(dla trybu ARM) najpierw czwarty bajt, następnie trzeci, drugi i pierwszy 
(tzn \gls{endianness}).

W ten sposób bajty są wypisywane w listingach IDA:

\begin{itemize}
\item dla trybów ARM i ARM64: 4-3-2-1;
\item dla trybu Thumb: 2-1;
\item dla pary 16-bitowych instrukcji w trybi Thumb-2: 2-1-4-3.
\end{itemize}

\myindex{ARM!\Instructions!MOVW}
\myindex{ARM!\Instructions!MOVT.W}
\myindex{ARM!\Instructions!BLX}
Także widzimy, że tutaj instrukcje \TT{MOVW}, \TT{MOVT.W} i \TT{BLX} zaczynają się od \TT{0xFx}.

Jedna z Thumb-2 instrukcji to
\TT{MOVW R0, \#0x13D8}~--- ona zapisuje 16-bitową liczbę do młodszych bitów rejestru \Reg{0}, zerując starsze bity.

jeszcze jest \TT{MOVT.W R0, \#0}~--- ta instrukcja działa tak samo, jak i \TT{MOVT} z poprzedniego przykładu, ale ona nie zadziała w trybie Thumb-2.

\myindex{ARM!przełączanie trybów}
\myindex{ARM!\Instructions!BLX}
Tutaj jest wykorzystywana instrukcja \TT{BLX} zamiast \TT{BL}.
Różnica polega na tym, że oprócz zapisania adresu powrotu do rejestru \ac{LR} i przekazania zarządzania 
do f-cji \puts, odbywa się zmiana trybu procesora z Thumb/Thumb-2 na tryb ARM (lub wstecz).
Tutaj jest to niezbędne dlatego, że instrukcja (ona jest zakodowana w trybie ARM) wygląda tak:

\begin{lstlisting}[style=customasmARM]
__symbolstub1:00003FEC _puts           ; CODE XREF: _hello_world+E
__symbolstub1:00003FEC 44 F0 9F E5     LDR  PC, =__imp__puts
\end{lstlisting}

Jest to zwykłe przejście na miejsce, w którym jest zapisany adres \puts w segmencie importów.
Można zadać pytanie: dlaczego nie można wywołać \puts od razu w tym miejscu kodu w którym jest to potrzebne?
Jest to niewygodne, z racji oszczędzania miejsca.

\myindex{Biblioteki łączone dynamicznie}
Praktycznie każdy program korzysta z zewnętrznych bibliotek łączonych dynamicznie (czy to będzie DLL w Windows, .so w *NIX 
czy .dylib w \MacOSX).
W bibliotekach dynamicznych znajdują się często wykorzystywane funkcje biblioteczne, w tym standardowa f-cja biblioteki C \puts.

\myindex{Relocation}
W wykonywalnym pliku binarnym 
(Windows PE .exe, ELF lub Mach-O) istnieje segment importów, lista symboli (funkcji lub zmiennych globalnych) importowanych z modułów zewnętrznych i, również, nazwy tych modułów.
Program rozruchowy ładuje niezbędne moduły i, iterując po symbolach zaimportowanych w module głównym, ustawia rzeczywiste adresy każdego z symboli.
W naszym przypadku, \IT{\_\_imp\_\_puts} 
jest zmienną 32-bitową, gdzie program rozruchowy wpiszę adres rzeczywisty tej f-cji w bibliotece zewnętrznej. 
Czyli instrukcja \TT{LDR} po prostu bierze 32-bitową wartość z tej zmiennej, i, zapisując ją do rejestru \ac{PC}, przekazuje tam zarządzanie.
Żeby zmniejszyć czas działania programu rozruchowego, trzeba, żeby adres się zapisywał tylko raz do specjalnie przydzielonego miejsca.

\myindex{thunk-функции}
Do tego, jak się już upewniliśmy, zapisywanie 32-bitowej liczby do rejestru jest niemożliwe bez odwoływań się do pamięci.
Także najbardziej optymalnym rozwiązaniem jest przydzielenie osobnej f-cji, pracującej w trybie ARM, 
jedynym celem której jest przekazywanie zarządzania dalej, do bbilioteki dynamicznie łączonej, i, następnie, odwoływanie się do tej któtkiej f-cji (tzw \glslink{thunk function}{thunk-funkcja}) z Thumb-kodu.

\myindex{ARM!\Instructions!BL}
A propos, w poprzednim przykładzie (skompilowanym dla trybu ARM), przejście za pomocą instrukcji \TT{BL} prowadzi 
do właśnie takiej \glslink{thunk function}{thunk-funkcji}, lecz procesor się nie przestawia na inny tryb (stąd wynika brak \q{X} w mnemoniku instrukcji).

\myparagraph{Jeszcze o thunk-funkcjach}
\myindex{thunk-функции}

Thunk-funckje są trudne do zrozumienia przede wszystkim przez brak spójności w terminologii.
Najprościej będzie traktować je jako adaptery-przejściówki z jednego typu gniazdek na drugi.
Na przykład, adapter pozwalający wstawić do gniazdka amerykańskiego wtyczkę brytyjską i na odwrót. Thunk-funkcje również są czasami nazywane \IT{wrapper-ami}. \IT{Wrap} w języku angielskim to \IT{owinąć}, \IT{zawinąć}.
Oto jeszcze kilka definicji tych funkcji:

\begin{framed}
\begin{quotation}
“A piece of coding which provides an address:”, according to P. Z. Ingerman, 
who invented thunks in 1961 as a way of binding actual parameters to their formal 
definitions in Algol-60 procedure calls. If a procedure is called with an expression 
in the place of a formal parameter, the compiler generates a thunk which computes 
the expression and leaves the address of the result in some standard location.

\dots

Microsoft and IBM have both defined, in their Intel-based systems, a “16-bit environment” 
(with bletcherous segment registers and 64K address limits) and a “32-bit environment” 
(with flat addressing and semi-real memory management). The two environments can both be 
running on the same computer and OS (thanks to what is called, in the Microsoft world, 
WOW which stands for Windows On Windows). MS and IBM have both decided that the process 
of getting from 16- to 32-bit and vice versa is called a “thunk”; for Windows 95, 
there is even a tool, THUNK.EXE, called a “thunk compiler”.
\end{quotation}
\end{framed}
% TODO FIXME move to bibliography and quote properly above the quote
( \href{http://go.yurichev.com/17362}{The Jargon File} )

\myindex{LAPACK}
\myindex{FORTRAN}
Jeszcze jeden przykład możemy znaleźć w bibliotece LAPACK --- (``Linear Algebra PACKage'') napisanej w języku FORTRAN.
Deweloperzy \CCpp również chcą korzystać z LAPACK, ale przepisywać ją na \CCpp, a następnie wspierać kilka wersji,
jest szaleństwem.
Także istnieją krótkie f-cje w C, które są wywoływane z \CCpp{}-środowiska, które, z kolei, wywołują funkcje FORTRAN,
i prawie nic oprócz tego nie robią:

\begin{lstlisting}[style=customc]
double Blas_Dot_Prod(const LaVectorDouble &dx, const LaVectorDouble &dy)
{
    assert(dx.size()==dy.size());
    integer n = dx.size();
    integer incx = dx.inc(), incy = dy.inc();

    return F77NAME(ddot)(&n, &dx(0), &incx, &dy(0), &incy);
}
\end{lstlisting}

Takie f-cje również są nazywane ``wrappers'' (tzn ``opakowanie'').



\subsubsection{ARM64}

\myparagraph{GCC}

Skompilujmy przykład w GCC 4.8.1 dla ARM64:

\lstinputlisting[numbers=left,label=hw_ARM64_GCC,caption=\NonOptimizing GCC 4.8.1 + objdump,style=customasmARM]{patterns/01_helloworld/ARM/hw.lst}

W ARM64 nie ma trybów Thumb i Thumb-2, tylko ARM, dlatego tu korzystamy tylko z 32-bitowych instrukcji.

Jest tu dwa razy więcej rejestrów: \myref{ARM64_GPRs}.
64-bitowe rejestry mają prefiks 
\TT{X-}, a ich 32-bitowe części --- \TT{W-}.

\myindex{ARM!\Instructions!STP}
Instrukcja \TT{STP} (\IT{Store Pair}) 
odkłada na stos od razu 2 rejestry: \RegX{29} i \RegX{30}.
Oczywiście, ta instrukcja może zapisać tę parę rejestrów gdziekolwiek w pamięci, ale tu jest wskazany rejestr \ac{SP}, także ta para jest odkładana na stos.

Rejestry w ARM64 są 64-bitowe, każdy o długości 8 bajtów, dlatego do przechowywania 2 rejestrów potrzeba 16 bajtów.

Wykrzyknik (``!'') po operandzie oznacza, że najpierw od \ac{SP} będzie odjęte 16 i dopiero po tej czynności wartości z obu rejestrów będą odłożone na stos.

Jest to nazywane \IT{pre-index}.
Więcej o róznicy między \IT{post-index} a \IT{pre-index} 
można znaleźć tu: \myref{ARM_postindex_vs_preindex}.

W ten sposób, posługując się terminologią x86, pierwsza instrukcja~--- jest analogiczna do \TT{PUSH X29} i \TT{PUSH X30}.
\RegX{29} w ARM64 jest wykorzystywane jako \ac{FP}, a \RegX{30} 
jako \ac{LR}, dlatego są one zapisywane w prologu funkcji.

Druga instrukcja kopiuje \ac{SP} do \RegX{29} (lub \ac{FP}).
Jest to niezbędne do ustawienia stack frame funkcji.

\label{pointers_ADRP_and_ADD}
\myindex{ARM!\Instructions!ADRP/ADD pair}
Instrukcje \TT{ADRP} i \ADD są potrzebne do skonstruowania adresu linii \q{Hello!} w rejestrze \RegX{0}, 
jako że pierwszy argument f-cji jest przekazywany przez ten rejestr.
Jednakże w ARM nie ma instrukcji, za pomocą których można zapisać do rejestru dużą liczbę 
(dlatego że długość instrukcji wynosi maksymalnie 4 bajty. Więcej informacji o tym można znaleźć tutaj: \myref{ARM_big_constants_loading}).
Dlatego trzeba skorzystać z kilku instrukcji.
Pierwsza instrukcja (\TT{ADRP}) zapisuje do \RegX{0} adres 4-kB strony, na której się znajduje linia, 
a druga (\ADD) dodaje do tego adresu resztę.
Więcej o tym tutaj: \myref{ARM64_relocs}.

\TT{0x400000 + 0x648 = 0x400648}, i możemy zobaczyć, że w segmencie danych \TT{.rodata} pod tym adresem znajduje się nasza
linia \q{Hello!}.

\myindex{ARM!\Instructions!BL}
Następnie, za pomocą instrukcji \TT{BL} jest wywoływane \puts. Zostało to omówione wcześniej: \myref{puts}.

Instrukcja \MOV zapisuje 0 do \RegW{0}. 
\RegW{0} to młodsze 32 bity 64-bitowego rejestru \RegX{0}:

\begin{center}
\begin{tabular}{ | l | l | }
\hline
\RU{Старшие 32 бита}\EN{High 32-bit part}\ES{Parte alta de 32 bits}\PTBRph{}\PLph{}\ITAph{}\DEph{}\THAph{}\NLph{}\FR{Partie 32 bits haute} & \RU{младшие 32 бита}\EN{low 32-bit part}\ES{parte baja de 32 bits}\PTBRph{}\PLph{Starsze 32 bity}\ITAph{}\DEph{}\THAph{}\NLph{}\FR{Partie 32 bits basse} \\
\hline
\multicolumn{2}{ | c | }{X0} \\
\hline
\multicolumn{1}{ | c | }{} & \multicolumn{1}{ c | }{W0} \\
\hline
\end{tabular}
\end{center}


Wynik funkcji jest zwracany przez \RegX{0}, i \main zwraca 0.

Dlaczego 32-bitowa część?
Dlatego że w ARM64, jak i w x86-64, typ \Tint zostawili 32-bitowym, dla kompatybilności.

Odpowiednio, jako że funkcja zwraca 32-bitowy \Tint, to trzeba wypełnić tylko młodsze 32 bity 64-bitowego rejestru \RegX{0}.

Żeby mieć pewność, trochę zmienimy przykład i skompilujemy go ponownie.%

Teraz \main zwraca 64-bitową wartość:

\begin{lstlisting}[caption=\main zwracające wartość typu \TT{uint64\_t},style=customc]
#include <stdio.h>
#include <stdint.h>

uint64_t main()
{
        printf ("Hello!\n");
        return 0;
}
\end{lstlisting}

Wynik jest taki sam, tylko \MOV w tej linii wygląda teraz w ten sposób:

\begin{lstlisting}[caption=\NonOptimizing GCC 4.8.1 + objdump]
  4005a4:       d2800000        mov     x0, #0x0      // #0
\end{lstlisting}

\myindex{ARM!\Instructions!LDP}
Następnie za pomocą instrukcji \INS{LDP} (\IT{Load Pair}) są przywracane rejestry \RegX{29} i \RegX{30}.

Wykrzyknika po instrukcji brak. To oznacza, że najpierw wartości są zdejmowane ze stosu, i dopiero po tej czynności \ac{SP} jest zwiększane o 16.

Jest to nazywane \IT{post-index}.

\myindex{ARM!\Instructions!RET}
W ARM64 pojawia się nowa instrukcja: \RET. 
Ona działa tak samo jak i \INS{BX LR}, ale zawiera bit ,
który podpowiada procesorowi, że jest to wyjście z f-cji, a nie zwykłe przejście, żeby procesor mógł zoptymalizować tę instrukcję.

Jako że ta funkcja jest bardzo prosta, optymalizujący GCC generuje dokładnie taki sam kod.





