\subsubsection{\NonOptimizingKeilVI (\ThumbMode)}

ThumbモードでKeilを使って同じ例をコンパイルしましょう。

\begin{lstlisting}
armcc.exe --thumb --c90 -O0 1.c 
\end{lstlisting}

IDAに入ってみましょう。

\begin{lstlisting}[caption=\NonOptimizingKeilVI (\ThumbMode) + \IDA,style=customasmARM]
.text:00000000             main
.text:00000000 10 B5          PUSH    {R4,LR}
.text:00000002 C0 A0          ADR     R0, aHelloWorld ; "hello, world"
.text:00000004 06 F0 2E F9    BL      __2printf
.text:00000008 00 20          MOVS    R0, #0
.text:0000000A 10 BD          POP     {R4,PC}

.text:00000304 68 65 6C 6C+aHelloWorld  DCB "hello, world",0    ; DATA XREF: main+2
\end{lstlisting}

2バイト(16ビット)のオペコードを簡単に見つけることができます。これは既に述べたように、Thumbです。
\myindex{ARM!\Instructions!BL}
ただし、\TT{BL}命令は2つの16ビット命令で構成されています。
これは、1つの16ビットオペコードの小さなスペースを使用している間に \printf 関数のオフセットをロードすることが不可能なためです。
したがって、第1の16ビット命令はオフセットの上位10ビットをロードし、第2命令はオフセットの下位11ビットをロードする。

% TODO:
% BL has space for 11 bits, so if we don't encode the lowest bit,
% then we should get 11 bits for the upper half, and 12 bits for the lower half.
% And the highest bit encodes the sign, so the destination has to be within
% \pm 4M of current_PC.
% This may be less if adding the lower half does not carry over,
% but I'm not sure --all my programs have 0 for the upper half,
% and don't carry over for the lower half.
% It would be interesting to check where __2printf is located relative to 0x8
% (I think the program counter is the next instruction on a multiple of 4
% for THUMB).
% The lower 11 bytes of the BL instructions and the even bit are
% 000 0000 0110 | 001 0010 1110 0 = 000 0000 0110 0010 0101 1100 = 0x00625c,
% so __2printf should be at 0x006264.
% But if we only have 10 and 11 bits, then the offset would be:
% 00 0000 0110 | 01 0010 1110 0 = 0 0000 0011 0010 0101 1100 = 0x00325c,
% so __2printf should be at 0x003264.
% In this case, though, the new program counter can only be 1M away,
% because of the highest bit is used for the sign.

前述したように、Thumbモードの命令はすべて2バイト(または16ビット)のサイズです。
これは、Thumb命令が奇妙なアドレスにあることはまったく不可能であることを意味します。
上記を前提として、命令を符号化する間に最後のアドレスビットを省略することができる。

まとめると、\TT{BL}Thumb命令は$current\_PC \pm{}\approx{}2M$のアドレスを符号化することができます。

\myindex{ARM!\Instructions!PUSH}
\myindex{ARM!\Instructions!POP}
関数内の他の命令については、 \PUSH と \POP はここで説明した\TT{STMFD}/\TT{LDMFD}のように動作しますが、ここでは\ac{SP}レジスタのみが明示的に言及されていません。 
\TT{ADR}は前の例と同様に動作します。 
\TT{MOVS}は、0を返すために\Reg{0}レジスタに0を書き込みます。

