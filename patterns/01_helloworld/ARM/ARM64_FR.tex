\subsubsection{ARM64}

\myparagraph{GCC}

Compilons l'exemple en utilisant GCC 4.8.1 en ARM64:

\lstinputlisting[numbers=left,label=hw_ARM64_GCC,caption=\NonOptimizing GCC 4.8.1 + objdump,style=customasmARM]{patterns/01_helloworld/ARM/hw.lst}

Il n'y a pas de mode Thumb ou Thumb-2 en ARM64, seulement en ARM, donc il n'y a que des
instructions 32-bit.
Le nombre de registres a doublé: \myref{ARM64_GPRs}.
Les registres 64-bit ont le préfixe \TT{X-}, tandis que leurs partie 32-bit basse---\TT{W-}.

\myindex{ARM!\Instructions!STP}
L'instruction \TT{STP} (\IT{Store Pair} stocke une paire)
sauve deux registres sur la pile simultanément: \RegX{29} et \RegX{30}.

Bien sûr, cette instruction peut sauvegarder cette paire à n'importe quelle endroit en mémoire,
mais le registre \ac{SP} est spécifié ici, donc la paire est sauvé sur le pile.

Les registres ARM64 font 64-bit, chacun a une taille de 8 octets, donc il faut 16 octets pour sauver
deux registres.

Le point d'exclamation (``!'') après l'opérande signifie que 16 octets doivent d'abord être soustrait de \ac{SP},
et ensuite les valeurs de la paire de registres peuvent être écrites sur la pile.
Ceci est appelé le \IT{pre-index}.
A propos de la différence entre \IT{post-index} et \IT{pre-index}
lisez ceci: \myref{ARM_postindex_vs_preindex}.

Dans la gamme plus connue du x86, la première instruction est analogue à la paire
\TT{PUSH X29} et \TT{PUSH X30}.
En ARM64, \RegX{29} est utilisé comme \ac{FP} et \RegX{30} comme \ac{LR}, c'est pourquoi ils sont
sauvegardés dans le prologue de la fonction et remis dans l'épilogue.

La seconde instruction copie \ac{SP} dans \RegX{29} (ou \ac{FP}).
Cela sert à préparer la pile de la fonction.

\label{pointers_ADRP_and_ADD}
\myindex{ARM!\Instructions!ADRP/ADD pair}
Les instructions \TT{ADRP} et \ADD sont utilisées pour remplir l'adresse de
la chaîne \q{Hello!} dans le registre \RegX{0},
car le premier argument de la fonction est passé dans ce registre.
Il n'y a pas d'instruction, quelqu'elle soit, en ARM qui puisse stocker un nombre large
dans un registre (car la longueur des instructions est limitée à 4 octets, cf: \myref{ARM_big_constants_loading}).
Plusieurs instructions doivent donc être utilisées. La première instruction (\TT{ADRP}) écrit l'adresse de
la page de 4KiB, où se trouve la chaîne, dans \RegX{0}, et la seconde (\ADD) ajoute simplement
le reste de l'adresse.
Plus d'information ici: \myref{ARM64_relocs}.

\TT{0x400000 + 0x648 = 0x400648}, et nous voyons notre chaîne C \q{Hello!} dans le \TT{.rodata} segment
des données à cette adresse.

\myindex{ARM!\Instructions!BL}

\puts est appelée après en utilisant l'instruction \TT{BL}. Cela a déjà été discuté: \myref{puts}.

\MOV écrit 0 dans \RegW{0}.
\RegW{0} est la partie basse 32 bits du registre 64-bit \RegX{0}:

\begin{center}
\begin{tabular}{ | l | l | }
\hline
\RU{Старшие 32 бита}\EN{High 32-bit part}\ES{Parte alta de 32 bits}\PTBRph{}\PLph{}\ITAph{}\DEph{}\THAph{}\NLph{}\FR{Partie 32 bits haute} & \RU{младшие 32 бита}\EN{low 32-bit part}\ES{parte baja de 32 bits}\PTBRph{}\PLph{Starsze 32 bity}\ITAph{}\DEph{}\THAph{}\NLph{}\FR{Partie 32 bits basse} \\
\hline
\multicolumn{2}{ | c | }{X0} \\
\hline
\multicolumn{1}{ | c | }{} & \multicolumn{1}{ c | }{W0} \\
\hline
\end{tabular}
\end{center}


Le résultat de la fonction est retourné via \RegX{0} et main renvoie 0, donc c'est ainsi que la valeur
de retour est préparée.
Mais pourquoi utiliser la partie 32-bit?

Parce que ls type de donnée \Tint en ARM64, tout comme en x86-64, est toujours 32-bit, pour une
meilleure compatibilité.

Donc si la fonction renvoie un \Tint 32-bit, seul les 32 premiers bits du registre \RegX{0} doivent
être rempli.

Pour vérifier ceci, changer un peu cet exemple et recompilons le.
Maintenant, \main renvoie une valeur sur 64-bit:

\begin{lstlisting}[caption=\main renvoie une valeur de type \TT{uint64\_t} type,style=customc]
#include <stdio.h>
#include <stdint.h>

uint64_t main()
{
        printf ("Hello!\n");
        return 0;
}
\end{lstlisting}

%%The result is the same, but that's how \MOV at that line looks like now:
Le résultat est le même, mais voilà à quoi ressemble \MOV à cette ligne maintenant:

\begin{lstlisting}[caption=\NonOptimizing GCC 4.8.1 + objdump]
  4005a4:       d2800000        mov     x0, #0x0      // #0
\end{lstlisting}

\myindex{ARM!\Instructions!LDP}

\INS{LDP} (\IT{Load Pair}) remet les registres \RegX{29} et \RegX{30}.

Il n'y a pas de point d'exclamation après l'instruction: celui signifie que les valeurs sont
d'abord chargées depuis la pile, et ensuite \ac{SP} est incrémenté de 16.
Cela est appelé \IT{post-index}.

\myindex{ARM!\Instructions!RET}
Une nouvelle instruction est apparue en ARM64: \RET.
Elle fonctionne comme \TT{BX LR}, un \IT{hint} bit particulier est ajouté, qui informe le \ac{CPU}
qu'il s'agit d'un retour de fonction, et pas d'une autre instruction de saut, et il peut l'exécuter
de manière plus optimale. 

A cause de la simplicité de la fonction, GCC avec l'option d'optimisation génère le même code.
