\subsection{ARM}
\label{sec:hw_ARM}

\myindex{\idevices}
\myindex{Raspberry Pi}
\myindex{Xcode}
\myindex{LLVM}
\myindex{Keil}
Für die Experimente mit ARM-Prozessoren wurden verschiedene Compiler genutzt:

\begin{itemize}
\item Verbreitet im Embedded-Bereich: Keil Release 6/2013.

\item Apple Xcode 4.6.3 IDE mit dem LLVM-GCC 4.2-Compiler
\footnote{Tatsächlich nutzt Apple Xcode 4.6.3 GCC als Front-End-Compiler und LLVM 
Code Generator}.

\item GCC 4.9 (Linaro) (für ARM64), verfügbar als Win32-Executable unter \url{http://go.yurichev.com/17325}.

\end{itemize}
Wenn nicht anders angegeben wird immer der 32-Bit ARM-Code (inklusive Thumb und Thumb-2-Mode) genutzt.
Wenn von 64-Bit ARM die Rede ist, dann wird ARM64 geschrieben.

% subsections
\subsubsection{\NonOptimizingKeilVI (\ARMMode)}

Beginnen wir mit dem Kompilieren des Beispiels mit Keil:

\begin{lstlisting}
armcc.exe --arm --c90 -O0 1.c 
\end{lstlisting}

\myindex{\IntelSyntax}
Der \IT{armcc}-Compiler erstellt Assembler-Quelltext im Intel-Syntax, hat aber High-Level-Makros
bezüglich der ARM-Prozessoren\footnote{d.h. der ARM-Mode hat keine \PUSH/\POP-Anweisungen}.
Es ist hier wichtig die \q{richtigen} Anweisungen zu sehen, deswegen ist hier das Ergebnis mit
\IDA kompiliert.

\begin{lstlisting}[caption=\NonOptimizingKeilVI (\ARMMode) \IDA,style=customasm]
.text:00000000             main
.text:00000000 10 40 2D E9    STMFD   SP!, {R4,LR}
.text:00000004 1E 0E 8F E2    ADR     R0, aHelloWorld ; "hello, world"
.text:00000008 15 19 00 EB    BL      __2printf
.text:0000000C 00 00 A0 E3    MOV     R0, #0
.text:00000010 10 80 BD E8    LDMFD   SP!, {R4,PC}

.text:000001EC 68 65 6C 6C+aHelloWorld  DCB "hello, world",0    ; DATA XREF: main+4
\end{lstlisting}

Im ersten Beispiel ist zu erkennen, dass jede Anweisung 4 Byte groß ist.
Tatsächlich wurde der Code für den ARM- und nicht den Thumb-Mode erstellt.

\myindex{ARM!\Instructions!STMFD}
\myindex{ARM!\Instructions!POP}
Die erste Anweisung, \INS{STMFD SP!, \{R4,LR\}}\footnote{\ac{STMFD}}, arbeitet wie eine x86-\PUSH-Anweisung
um die Werte der beiden Register (\Reg{4} and \ac{LR}) auf den Stack zu legen.

Die Ausgabe des \IT{armcc}-Compilers, zeigt, aus Gründen der Einfachheit, die \INS{PUSH \{r4,lr\}}-Anweisung.
Dies ist nicht vollständig präzise. Die \PUSH-Anweisung ist nur im Thumb-Mode verfügbar.
Um die Dinge nicht zu verwirrend zu machen, wird der Code in \IDA kompiliert.

Die Anweisung dekrementiert zunächst den \ac{SP}, so dass er auf den Bereich im Stack zeigt, der
für neue Einträge frei ist. Anschließend werden die Werte der Register \Reg{4} und \ac{LR} an der Adresse
gespeichert auf den der (modifizierte) \ac{SP} zeigt.

Diese Anweisungen (wie \PUSH im Thumb-Mode) ist in der Lage mehrere Register-Werte auf einmal zu speichern,
was sehr nützlich sein kann. Übrigens: in x86 gibt es dazu kein Äquivalent.
Außerdem ist erwähnenswert, dass die \TT{STMFD}-Anweisung eine Generalisierung der \PUSH-Anweisung
(ohne deren Eigenschaften) ist, weil sie auf jedes Register angewandt werden kann und nicht nur auf \ac{SP}.
Mit anderen Worten kann \TT{STMFD} genutzt werden um eine Reihen von Registern an einer angegebenen
Speicher-Adresse zu sichern.

\myindex{\PICcode}
\myindex{ARM!\Instructions!ADR}
Die \INS{ADR R0, aHelloWorld}-Anweisung addiert oder subtrahiert den Wert im \ac{PC}-Register zum Offset
an dem die \TT{hello, world}-Zeichenkette ist.
Man kann sich nun fragen, wie das \TT{PC}-Register hier genutzt wird.
Dies wird \q{\PICcode}\footnote{mehr darüber in der entsprechenden Sektion~(\myref{sec:PIC})} genannt.

Code dieser Art kann an nicht-festen Adressen im Speicher ausgeführt werden.
Mit anderen Worten: dies ist \ac{PC}-relative Adressierung.
Die \INS{ADR}-Anweisung berücksichtigt den Unterschied zwischen der Adresse dieser Anweisung und der Adresse
an dem die Zeichenkette gespeichert ist.
Der Unterschied (Offset) ist immer gleich, egal an welcher Adresse der Code vom \ac{OS} geladen wurden.
Dementsprechend ist alles was gemacht werden muss, die Adresse der aktuellen Anweisung (vom \ac{PC})
zu addieren um die absolute Speicheradresse der Zeichenkette zu bekommen.

\myindex{ARM!\Registers!Link Register}
\myindex{ARM!\Instructions!BL}
\INS{BL \_\_2printf}\footnote{Branch with Link}-Anweisung ruft die \printf-Funktion auf. 
Die Anweisung funktioniert wie folgt:

\begin{itemize}
\item Speichere die Adresse hinter der \INS{BL}-Anweisung (\TT{0xC}) in \ac{LR};
\item anschließend wird übergebe die Kontrolle an \printf indem dessen Adresse ins \ac{PC}-Register geschrieben wird.
\end{itemize}

Wenn \printf die Ausführung beendet, müssen Informationen vorliegen, wo die Ausführung weitergehen soll.
Das ist der Grund warum jede Funktion die Kontrolle an die Adresse, gespeichert im \ac{LR}-Register übergibt.

Dies ist ein Unterschied zwischen einem \q{reinem} \ac{RISC}-Prozessor wie ARM und \ac{CISC}-Prozessoren wie x86,
bei denen die Rücksprungadresse in der Regel auf dem Stack gespeichert wird.
Mehr dazu ist im nächsten Abschnitt zu lesen~(\myref{sec:stack}).

Übrigens eine absolute 32-Bit-Adresse oder -Offset kann nicht in einer 32-Bit-\TT{BL}-Anweisung kodiert werden,
weil diese nur für 24 Bit Platz bietet. Wie bereits erwähnt haben alle ARM-Mode-Anweisungen eine Größe
von 4 Byte (32 Bit). Aus diesem Grund können diese nur an 4-Byte-Grenzen des Speichers platziert werden.
Dies heißt auch, das die letzten zwei Bit der Anweisungsadresse (die immer Null sind) entfallen können.
Zusammenfassend, stehen 26 Bit für die Offset-Kodierung zur Verfügung. Dies ist genug für
$current\_PC \pm{} \approx{}32M$.

\myindex{ARM!\Instructions!MOV}
Als nächstes schreibt die Anweisung \INS{MOV R0, \#0}\footnote{das heißt MOVe} lediglich 0 in
das \Reg{0}-Register weil der Rückgabewert hier gespeichert wird und die gezeigte C-Funktion 0
als Argument für die return-Anweisung hat.

\myindex{ARM!\Registers!Link Register}
\myindex{ARM!\Instructions!LDMFD}
\myindex{ARM!\Instructions!POP}
Die letzte Anweisung \INS{LDMFD SP!, {R4,PC}}\footnote{\ac{LDMFD} ist eine inverse Anweisung von \ac{STMFD}}
lädt die Werte nacheinander vom Stack (oder eine andere Speicheradresse) um sie in die Register \Reg{4} und \ac{PC}
zu sichern. Außerdem wird der Stack Pointer \ac{SP} inkrementiert. Hier arbeitet der Befehl wie \POP.

Die erste Anweisung \TT{STMFD} sichert das Register-Paar \Reg{4} und \ac{LR} auf dem Stack, jedoch werden \Reg{4} und \ac{PC}
während der Ausführung von \TT{LDMFD} \IT{wiederhergestellt}.

Wie bereits bekannt, wird die Adresse die nach der Ausführung einer Funktion angesprungen wird in dem \ac{LR}-Register gesichert.
Die allererste Anweisung sichert diese Wert auf dem Stack weil das gleiche Register von der \main-Funktion genutzt wird,
wenn \printf aufgerufen wird.
Am Ende der Funktion kann dieser Wert direkt in das \ac{PC}-Register geschrieben werden und so die Ausführung an der
Stelle fortgesetzt werden an der die Funktion aufgerufen wurde.

Da \main in der Regel die erste Funktion in \CCpp ist, wird die Kontrolle an das \ac{OS} oder einen Punkt in der
\ac{CRT} übergeben.

All dies erlaubt das Auslassen der \INS{BX LR}-Anweisung am Ende der Funktion.

\myindex{ARM!DCB}
\TT{DCB} ist eine Assemblerdirektive die ein Array von Bytes oder ASCII anlegt, ähnlich der DB-Direktive
in der x86-Assembler-Sprache.

\subsubsection{\NonOptimizingKeilVI (\ThumbMode)}

Nachfolgend das gleiche Beispiel mit dem Keil-Compiler im Thumb-Mode erstellt:

\begin{lstlisting}
armcc.exe --thumb --c90 -O0 1.c 
\end{lstlisting}

In \IDA wird folgende Ausgabe erzeugt:

\begin{lstlisting}[caption=\NonOptimizingKeilVI (\ThumbMode) + \IDA,style=customasmARM]
.text:00000000             main
.text:00000000 10 B5          PUSH    {R4,LR}
.text:00000002 C0 A0          ADR     R0, aHelloWorld ; "hello, world"
.text:00000004 06 F0 2E F9    BL      __2printf
.text:00000008 00 20          MOVS    R0, #0
.text:0000000A 10 BD          POP     {R4,PC}

.text:00000304 68 65 6C 6C+aHelloWorld  DCB "hello, world",0    ; DATA XREF: main+2
\end{lstlisting}

Leicht zu erkennen sind die 2-Byte (16 Bit) OpCodes, die wie bereits erwähnt Thumb-Anweisungen sind.
\myindex{ARM!\Instructions!BL}
Die \TT{BL}-Anweisung besteht aus zwei 16-Bit-Anweisungen, weil es für die \printf-Funktion unmöglich ist
einen Offset zu laden, wenn der kleine Speicherbereich in einem 16-Bit-Opcode genutzt wird.
Aus diesem Grund lädt die erste 16-Bit-Anweisung die höherwertigen 10 Bit des Offsets und die zweite
Anweisung die niederwertigen 11 Bit.

% TODO:
% BL has space for 11 bits, so if we don't encode the lowest bit,
% then we should get 11 bits for the upper half, and 12 bits for the lower half.
% And the highest bit encodes the sign, so the destination has to be within
% \pm 4M of current_PC.
% This may be less if adding the lower half does not carry over,
% but I'm not sure --all my programs have 0 for the upper half,
% and don't carry over for the lower half.
% It would be interesting to check where __2printf is located relative to 0x8
% (I think the program counter is the next instruction on a multiple of 4
% for THUMB).
% The lower 11 bytes of the BL instructions and the even bit are
% 000 0000 0110 | 001 0010 1110 0 = 000 0000 0110 0010 0101 1100 = 0x00625c,
% so __2printf should be at 0x006264.
% But if we only have 10 and 11 bits, then the offset would be:
% 00 0000 0110 | 01 0010 1110 0 = 0 0000 0011 0010 0101 1100 = 0x00325c,
% so __2printf should be at 0x003264.
% In this case, though, the new program counter can only be 1M away,
% because of the highest bit is used for the sign.

Wie erwähnt haben alle Anweisungen im Thumb-Mode eine Größe von 2 Byte (16 Bit).
Dies bedeutet, dass es unmöglich ist an einer ungeraden Adresse einen Anweisung unterzubringen.
Das hat auch zur Folge, dass das letzte Bit der Adresse bei der Kodierung der
Anweisungen weggelassen werden kann.

Zusammenfassend kann die \TT{BL}-Thumb-Anweisung eine Adresse bis $current\_PC \pm{}\approx{}2M$ kodieren.

\myindex{ARM!\Instructions!PUSH}
\myindex{ARM!\Instructions!POP}
Wie für die anderen Anweisungen in dieser Funktion arbeiten \PUSH und \POP wie die beschriebenden \TT{STMFD}/\TT{LDMFD},
nur dass das \ac{SP}-Register hier nicht explizit genannt wird.
\TT{ADR} arbeitet genau wie in dem vorherigen Beispiel.
\TT{MOVS} schreibt 0 in das Register \Reg{0} um 0 zurückzugeben.

\subsubsection{\OptimizingXcodeIV (\ARMMode)}

Xcode 4.6.3 ohne Optimierung produziert eine Menge redundanten Code, so dass im Folgenden die
optimierte Ausgabe gelistet ist bei der die Anzahl der Anweisungen so klein wie möglich ist.
Der Compiler-Schalter ist \Othree.

\begin{lstlisting}[caption=\OptimizingXcodeIV (\ARMMode),style=customasmARM]
__text:000028C4             _hello_world
__text:000028C4 80 40 2D E9   STMFD           SP!, {R7,LR}
__text:000028C8 86 06 01 E3   MOV             R0, #0x1686
__text:000028CC 0D 70 A0 E1   MOV             R7, SP
__text:000028D0 00 00 40 E3   MOVT            R0, #0
__text:000028D4 00 00 8F E0   ADD             R0, PC, R0
__text:000028D8 C3 05 00 EB   BL              _puts
__text:000028DC 00 00 A0 E3   MOV             R0, #0
__text:000028E0 80 80 BD E8   LDMFD           SP!, {R7,PC}

__cstring:00003F62 48 65 6C 6C+aHelloWorld_0  DCB "Hello world!",0
\end{lstlisting}

Die Anweisungen \TT{STMFD} und \TT{LDMFD} sind bereits bekannt.

\myindex{ARM!\Instructions!MOV}

Die \MOV-Anweisung schreibt lediglich die Nummer \TT{0x1686} in das Register \Reg{0}.
Dies ist der Offset der auf die Zeichenkette \q{Hello world!} zeigt.

Das Register \TT{R7} (spezifiziert in \IOSABI) ist ein Frame Pointer. Mehr darüber folgt später.

\myindex{ARM!\Instructions!MOVT}
Die \TT{MOVT R0, \#0} (MOVe Top)-Anweisung schreibt 0 in die höherwertigen 16 Bit des Registers.
Das Problem ist hier, dass die generische \MOV-Anweisung im ARM-Mode nur die niederwertigen 16 Bit
des Registers beschreibt.

Dran denken: alle Opcodes im ARM-Mode sind in der Größe auf 32 Bit begrenzt. Natürlich gilt diese
Begrenzung nicht für das Verschieben von Daten zwischen Registern.
Aus diesem Grund existiert die zusätzliche Anweisung  \TT{MOVT} um in die höherwertigen Bits
(von 16 bis einschließlich 31) zu beschreiben.
Die Benutzung ist in diesem Fall redundant, weil die Anweisung \TT{MOV R0, \#0x1686} darüpber
den höherwertigen Teil des Registers zurückgesetzt hat.
Dies ist vermutlich ein Mangel des Compilers.

% TODO:
% I think, more specifically, the string is not put in the text section,
% ie. the compiler is actually not using position-independent code,
% as mentioned in the next paragraph.
% MOVT is used because the assembly code is generated before the relocation,
% so the location of the string is not yet known,
% and the high bits may still be needed.

\myindex{ARM!\Instructions!ADD}
Die Anweisung \TT{ADD R0, PC, R0} addiert den Wert im \ac{PC} zum Wert im Register \Reg{0}
um die absolute Adresse der \q{Hello world!}-Zeichenkette zu berechnen.
Wie bereits bekannt ist dies \q{\PICcode}, so dass diese Korrektur hier unbedingt notwendig ist.

Die \INS{BL}-Anweisung ruft \puts anstatt \printf auf.

\label{puts}
\myindex{\CStandardLibrary!puts()}
\myindex{puts() anstatt printf()}

GCC ersetzt den ersten \printf-Aufruf mit \puts. In der Tat ist \printf mit nur einem
Argument identisch mit \puts.

Die beiden Funktionen produzieren lediglich das gleiche Ergebnis, weil printf keine
Formatkennzeichner, beginnend mit \IT{\%}, enhält.
Sollte dies jedoch der Fall sein, wäre die Auswirkung der beiden Funktionen
unterschiedlich\footnote{Des weiteren benötigt \puts kein '\textbackslash{}n'
für den Zeilenumbruch am Ende der Zeichenkette, weswegen wir dies hier nicht sehen.}.

Warum hat der Compiler diese Ersetzung durchgeführt? Vermutlich hat dies Vorteile bei
der Geschwindigkeit, weil \puts schneller ist
\footnote{\href{http://go.yurichev.com/17063}{ciselant.de/projects/gcc\_printf/gcc\_printf.html}}
und lediglich die Zeichen zu \gls{stdout} übergibt, anstatt jedes Zeichen mit \IT{\%} zu vergleichen.

Als nächstes ist die bekannte Anweisung \TT{MOV R0, \#0} zu sehen um das Register \Reg{0} auf 0 zu setzen.

%\subsubsection{\OptimizingXcodeIV (\ThumbTwoMode)}

By default Xcode 4.6.3 generates code for Thumb-2 in this manner:

\begin{lstlisting}[caption=\OptimizingXcodeIV (\ThumbTwoMode),style=customasm]
__text:00002B6C                   _hello_world
__text:00002B6C 80 B5          PUSH            {R7,LR}
__text:00002B6E 41 F2 D8 30    MOVW            R0, #0x13D8
__text:00002B72 6F 46          MOV             R7, SP
__text:00002B74 C0 F2 00 00    MOVT.W          R0, #0
__text:00002B78 78 44          ADD             R0, PC
__text:00002B7A 01 F0 38 EA    BLX             _puts
__text:00002B7E 00 20          MOVS            R0, #0
__text:00002B80 80 BD          POP             {R7,PC}

...

__cstring:00003E70 48 65 6C 6C 6F 20+aHelloWorld  DCB "Hello world!",0xA,0
\end{lstlisting}

% Q: If you subtract 0x13D8 from 0x3E70,
% you actually get a location that is not in this function, or in _puts.
% How is PC-relative addressing done in THUMB2?
% A: it's not Thumb-related. there are just mess with two different segments. TODO: rework this listing.

\myindex{\ThumbTwoMode}
\myindex{ARM!\Instructions!BL}
\myindex{ARM!\Instructions!BLX}

The \TT{BL} and \TT{BLX} instructions in Thumb mode, as we recall, are encoded as a pair of 16-bit instructions.
In Thumb-2 these \IT{surrogate} opcodes are extended in such a way so that new instructions may be encoded here as 32-bit instructions.

That is obvious considering that the opcodes of the Thumb-2 instructions always begin with \TT{0xFx} or \TT{0xEx}.

But in the \IDA listing
the opcode bytes are swapped because for ARM processor the instructions are encoded as follows: 
last byte comes first and after that comes the first one (for Thumb and Thumb-2 modes) 
or for instructions in ARM mode the fourth byte comes first, then the third,
then the second and finally the first (due to different \gls{endianness}).

So that is how bytes are located in IDA listings:
\begin{itemize}
\item for ARM and ARM64 modes: 4-3-2-1;
\item for Thumb mode: 2-1;
\item for 16-bit instructions pair in Thumb-2 mode: 2-1-4-3.
\end{itemize}

\myindex{ARM!\Instructions!MOVW}
\myindex{ARM!\Instructions!MOVT.W}
\myindex{ARM!\Instructions!BLX}

So as we can see, the \TT{MOVW}, \TT{MOVT.W} and \TT{BLX} instructions begin with \TT{0xFx}.

One of the Thumb-2 instructions is \TT{MOVW R0, \#0x13D8} ~---it stores a 16-bit value into the lower part of the \Reg{0} register, clearing the higher bits.

Also, \TT{MOVT.W R0, \#0} ~works just like \TT{MOVT} from the previous example only it works in Thumb-2.

\myindex{ARM!mode switching}
\myindex{ARM!\Instructions!BLX}

Among the other differences, the \TT{BLX} instruction is used in this case instead of the \TT{BL}.

The difference is that, besides saving the \ac{RA} in the \ac{LR} register and passing control 
to the \puts function, the processor is also switching from Thumb/Thumb-2 mode to ARM mode (or back).

This instruction is placed here since the instruction to which control is passed looks like (it is encoded in ARM mode):

\begin{lstlisting}[style=customasm]
__symbolstub1:00003FEC _puts           ; CODE XREF: _hello_world+E
__symbolstub1:00003FEC 44 F0 9F E5     LDR  PC, =__imp__puts
\end{lstlisting}

This is essentially a jump to the place where the address of \puts is written in the imports' section.

So, the observant reader may ask: why not call \puts right at the point in the code where it is needed?

Because it is not very space-efficient.

\myindex{Dynamically loaded libraries}
Almost any program uses external dynamic libraries (like DLL in Windows, .so in *NIX or .dylib in \MacOSX).
The dynamic libraries contain frequently used library functions, including the standard C-function \puts.

\myindex{Relocation}
In an executable binary file (Windows PE .exe, ELF or Mach-O) an import section is present.
This is a list of symbols (functions or global variables) imported from external modules along with the names of the modules themselves.

The \ac{OS} loader loads all modules it needs and, while enumerating import symbols in the primary module, determines the correct addresses of each symbol.

In our case, \IT{\_\_imp\_\_puts} is a 32-bit variable used by the \ac{OS} loader to store the correct address of the function in an external library. 
Then the \TT{LDR} instruction just reads the 32-bit value from this variable and writes it into the \ac{PC} register, passing control to it.

So, in order to reduce the time the \ac{OS} loader needs for completing this procedure, 
it is good idea to write the address of each symbol only once, to a dedicated place.

\myindex{thunk-functions}
Besides, as we have already figured out, it is impossible to load a 32-bit value into a register 
while using only one instruction without a memory access.

Therefore, the optimal solution is to allocate a separate function working in ARM mode with the sole 
goal of passing control to the dynamic library and then to jump to this short one-instruction function (the so-called \gls{thunk function}) from the Thumb-code.

\myindex{ARM!\Instructions!BL}
By the way, in the previous example (compiled for ARM mode) the control is passed by the \TT{BL} to the 
same \gls{thunk function}.
The processor mode, however, is not being switched (hence the absence of an \q{X} in the instruction mnemonic).

\myparagraph{More about thunk-functions}
\myindex{thunk-functions}

Thunk-functions are hard to understand, apparently, because of a misnomer.
The simplest way to understand it as adaptors or convertors of one type of jack to another.
For example, an adaptor allowing the insertion of a British power plug into an American wall socket, or vice-versa. 
Thunk functions are also sometimes called \IT{wrappers}.

Here are a couple more descriptions of these functions:

\begin{framed}
\begin{quotation}
“A piece of coding which provides an address:”, according to P. Z. Ingerman, 
who invented thunks in 1961 as a way of binding actual parameters to their formal 
definitions in Algol-60 procedure calls. If a procedure is called with an expression 
in the place of a formal parameter, the compiler generates a thunk which computes 
the expression and leaves the address of the result in some standard location.

\dots

Microsoft and IBM have both defined, in their Intel-based systems, a “16-bit environment” 
(with bletcherous segment registers and 64K address limits) and a “32-bit environment” 
(with flat addressing and semi-real memory management). The two environments can both be 
running on the same computer and OS (thanks to what is called, in the Microsoft world, 
WOW which stands for Windows On Windows). MS and IBM have both decided that the process 
of getting from 16- to 32-bit and vice versa is called a “thunk”; for Windows 95, 
there is even a tool, THUNK.EXE, called a “thunk compiler”.
\end{quotation}
\end{framed}
% TODO FIXME move to bibliography and quote properly above the quote
( \href{http://go.yurichev.com/17362}{The Jargon File} )

\myindex{LAPACK}
\myindex{FORTRAN}
Another example we can find in LAPACK library---a ``Linear Algebra PACKage'' written in FORTRAN.
\CCpp developers also want to use LAPACK, but it's insane to rewrite it to \CCpp and then maintain several versions.
So there are short C functions callable from \CCpp environment, which are, in turn, call FORTRAN functions,
and do almost anything else:

\begin{lstlisting}[style=customc]
double Blas_Dot_Prod(const LaVectorDouble &dx, const LaVectorDouble &dy)
{
    assert(dx.size()==dy.size());
    integer n = dx.size();
    integer incx = dx.inc(), incy = dy.inc();

    return F77NAME(ddot)(&n, &dx(0), &incx, &dy(0), &incy);
}
\end{lstlisting}

Also, functions like that are called ``wrappers''.


%\subsubsection{ARM64}

\myparagraph{GCC}

Let's compile the example using GCC 4.8.1 in ARM64:

\lstinputlisting[numbers=left,label=hw_ARM64_GCC,caption=\NonOptimizing GCC 4.8.1 + objdump,style=customasmARM]{patterns/01_helloworld/ARM/hw.lst}

There are no Thumb and Thumb-2 modes in ARM64, only ARM, so there are 32-bit instructions only.
The Register count is doubled: \myref{ARM64_GPRs}.
64-bit registers have \TT{X-} prefixes, while its 32-bit parts---\TT{W-}.

\myindex{ARM!\Instructions!STP}
The \TT{STP} instruction (\IT{Store Pair}) 
saves two registers in the stack simultaneously: \RegX{29} and \RegX{30}.

Of course, this instruction is able to save this pair at an arbitrary place in memory, 
but the \ac{SP} register is specified here, so the pair is saved in the stack.

ARM64 registers are 64-bit ones, each has a size of 8 bytes, so one needs 16 bytes for saving two registers.

The exclamation mark (``!'') after the operand means that 16 is to be subtracted from \ac{SP} first, and only then
are values from register pair to be written into the stack.
This is also called \IT{pre-index}.
About the difference between \IT{post-index} and \IT{pre-index} 
read here: \myref{ARM_postindex_vs_preindex}.

Hence, in terms of the more familiar x86, the first instruction is just an analogue to a pair of
\TT{PUSH X29} and \TT{PUSH X30}.
\RegX{29} is used as \ac{FP} in ARM64, and \RegX{30} 
as \ac{LR}, so that's why they are saved in the function prologue and restored in the function epilogue.

The second instruction copies \ac{SP} in \RegX{29} (or \ac{FP}).
This is made so to set up the function stack frame.

\label{pointers_ADRP_and_ADD}
\myindex{ARM!\Instructions!ADRP/ADD pair}
\TT{ADRP} and \ADD instructions are used to fill the 
address of the string \q{Hello!} into the \RegX{0} register, 
because the first function argument is passed
in this register.
There are no instructions, whatsoever, in ARM that can store a large number into a register (because the instruction
length is limited to 4 bytes, read more about it here: \myref{ARM_big_constants_loading}).
So several instructions must be utilized. The first instruction (\TT{ADRP}) writes the address of the 4KiB page, where the string is
located, into \RegX{0}, 
and the second one (\ADD) just adds the remainder to the address.
More about that in: \myref{ARM64_relocs}.

\TT{0x400000 + 0x648 = 0x400648}, and we see our \q{Hello!} C-string in the \TT{.rodata} data segment at this address.

\myindex{ARM!\Instructions!BL}

\puts is called afterwards using the \TT{BL} instruction. This was already discussed: \myref{puts}.

\MOV writes 0 into \RegW{0}. 
\RegW{0} is the lower 32 bits of the 64-bit \RegX{0} register:

\begin{center}
\begin{tabular}{ | l | l | }
\hline
\RU{Старшие 32 бита}\EN{High 32-bit part}\ES{Parte alta de 32 bits}\PTBRph{}\PLph{}\ITAph{}\DEph{}\THAph{}\NLph{}\FR{Partie 32 bits haute} & \RU{младшие 32 бита}\EN{low 32-bit part}\ES{parte baja de 32 bits}\PTBRph{}\PLph{Starsze 32 bity}\ITAph{}\DEph{}\THAph{}\NLph{}\FR{Partie 32 bits basse} \\
\hline
\multicolumn{2}{ | c | }{X0} \\
\hline
\multicolumn{1}{ | c | }{} & \multicolumn{1}{ c | }{W0} \\
\hline
\end{tabular}
\end{center}


The function result is returned via \RegX{0} and \main returns 0, so that's how the return result is prepared.
But why use the 32-bit part?

Because the \Tint data type in ARM64, just like in x86-64, is still 32-bit, for better compatibility.

So if a function returns a 32-bit \Tint, only the lower 32 bits of \RegX{0} register have to be filled.

In order to verify this, let's change this example slightly and recompile it.
Now \main returns a 64-bit value:

\begin{lstlisting}[caption=\main returning a value of \TT{uint64\_t} type,style=customc]
#include <stdio.h>
#include <stdint.h>

uint64_t main()
{
        printf ("Hello!\n");
        return 0;
}
\end{lstlisting}

The result is the same, but that's how \MOV at that line looks like now:

\begin{lstlisting}[caption=\NonOptimizing GCC 4.8.1 + objdump]
  4005a4:       d2800000        mov     x0, #0x0      // #0
\end{lstlisting}

\myindex{ARM!\Instructions!LDP}

\INS{LDP} (\IT{Load Pair}) then restores the \RegX{29} and \RegX{30} registers.

There is no exclamation mark after the instruction: this implies that the values are first loaded from the stack,
and only then is \ac{SP} increased by 16.
This is called \IT{post-index}.

\myindex{ARM!\Instructions!RET}
A new instruction appeared in ARM64: \RET. 
It works just as \TT{BX LR}, only a special \IT{hint} bit is added, informing the \ac{CPU}
that this is a return from a function, not just another jump instruction, so it can execute it more optimally.

Due to the simplicity of the function, optimizing GCC generates the very same code.

