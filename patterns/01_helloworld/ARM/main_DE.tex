\subsection{ARM}
\label{sec:hw_ARM}

\myindex{\idevices}
\myindex{Raspberry Pi}
\myindex{Xcode}
\myindex{LLVM}
\myindex{Keil}
Für die Experimente mit ARM-Prozessoren wurden verschiedene Compiler genutzt:

\begin{itemize}
\item Verbreitet im Embedded-Bereich: Keil Release 6/2013.

\item Apple Xcode 4.6.3 IDE mit dem LLVM-GCC 4.2-Compiler
\footnote{Tatsächlich nutzt Apple Xcode 4.6.3 GCC als Front-End-Compiler und LLVM 
Code Generator}.

\item GCC 4.9 (Linaro) (für ARM64), verfügbar als Win32-Executable unter \url{http://go.yurichev.com/17325}.

\end{itemize}
Wenn nicht anders angegeben wird immer der 32-Bit ARM-Code (inklusive Thumb und Thumb-2-Mode) genutzt.
Wenn von 64-Bit ARM die Rede ist, dann wird ARM64 geschrieben.

% subsections
\subsubsection{\NonOptimizingKeilVI (\ARMMode)}

Beginnen wir mit dem Kompilieren des Beispiels mit Keil:

\begin{lstlisting}
armcc.exe --arm --c90 -O0 1.c 
\end{lstlisting}

\myindex{\IntelSyntax}
Der \IT{armcc}-Compiler erstellt Assembler-Quelltext im Intel-Syntax, hat aber High-Level-Makros
bezüglich der ARM-Prozessoren\footnote{d.h. der ARM-Mode hat keine \PUSH/\POP-Anweisungen}.
Es ist hier wichtig die \q{richtigen} Anweisungen zu sehen, deswegen ist hier das Ergebnis mit
\IDA kompiliert.

\begin{lstlisting}[caption=\NonOptimizingKeilVI (\ARMMode) \IDA,style=customasm]
.text:00000000             main
.text:00000000 10 40 2D E9    STMFD   SP!, {R4,LR}
.text:00000004 1E 0E 8F E2    ADR     R0, aHelloWorld ; "hello, world"
.text:00000008 15 19 00 EB    BL      __2printf
.text:0000000C 00 00 A0 E3    MOV     R0, #0
.text:00000010 10 80 BD E8    LDMFD   SP!, {R4,PC}

.text:000001EC 68 65 6C 6C+aHelloWorld  DCB "hello, world",0    ; DATA XREF: main+4
\end{lstlisting}

Im ersten Beispiel ist zu erkennen, dass jede Anweisung 4 Byte groß ist.
Tatsächlich wurde der Code für den ARM- und nicht den Thumb-Mode erstellt.

\myindex{ARM!\Instructions!STMFD}
\myindex{ARM!\Instructions!POP}
Die erste Anweisung, \INS{STMFD SP!, \{R4,LR\}}\footnote{\ac{STMFD}}, arbeitet wie eine x86-\PUSH-Anweisung
um die Werte der beiden Register (\Reg{4} and \ac{LR}) auf den Stack zu legen.

Die Ausgabe des \IT{armcc}-Compilers, zeigt, aus Gründen der Einfachheit, die \INS{PUSH \{r4,lr\}}-Anweisung.
Dies ist nicht vollständig präzise. Die \PUSH-Anweisung ist nur im Thumb-Mode verfügbar.
Um die Dinge nicht zu verwirrend zu machen, wird der Code in \IDA kompiliert.

Die Anweisung dekrementiert zunächst den \ac{SP}, so dass er auf den Bereich im Stack zeigt, der
für neue Einträge frei ist. Anschließend werden die Werte der Register \Reg{4} und \ac{LR} an der Adresse
gespeichert auf den der (modifizierte) \ac{SP} zeigt.

Diese Anweisungen (wie \PUSH im Thumb-Mode) ist in der Lage mehrere Register-Werte auf einmal zu speichern,
was sehr nützlich sein kann. Übrigens: in x86 gibt es dazu kein Äquivalent.
Außerdem ist erwähnenswert, dass die \TT{STMFD}-Anweisung eine Generalisierung der \PUSH-Anweisung
(ohne deren Eigenschaften) ist, weil sie auf jedes Register angewandt werden kann und nicht nur auf \ac{SP}.
Mit anderen Worten kann \TT{STMFD} genutzt werden um eine Reihen von Registern an einer angegebenen
Speicher-Adresse zu sichern.

\myindex{\PICcode}
\myindex{ARM!\Instructions!ADR}
Die \INS{ADR R0, aHelloWorld}-Anweisung addiert oder subtrahiert den Wert im \ac{PC}-Register zum Offset
an dem die \TT{hello, world}-Zeichenkette ist.
Man kann sich nun fragen, wie das \TT{PC}-Register hier genutzt wird.
Dies wird \q{\PICcode}\footnote{mehr darüber in der entsprechenden Sektion~(\myref{sec:PIC})} genannt.

Code dieser Art kann an nicht-festen Adressen im Speicher ausgeführt werden.
Mit anderen Worten: dies ist \ac{PC}-relative Adressierung.
Die \INS{ADR}-Anweisung berücksichtigt den Unterschied zwischen der Adresse dieser Anweisung und der Adresse
an dem die Zeichenkette gespeichert ist.
Der Unterschied (Offset) ist immer gleich, egal an welcher Adresse der Code vom \ac{OS} geladen wurden.
Dementsprechend ist alles was gemacht werden muss, die Adresse der aktuellen Anweisung (vom \ac{PC})
zu addieren um die absolute Speicheradresse der Zeichenkette zu bekommen.

\myindex{ARM!\Registers!Link Register}
\myindex{ARM!\Instructions!BL}
\INS{BL \_\_2printf}\footnote{Branch with Link}-Anweisung ruft die \printf-Funktion auf. 
Die Anweisung funktioniert wie folgt:

\begin{itemize}
\item Speichere die Adresse hinter der \INS{BL}-Anweisung (\TT{0xC}) in \ac{LR};
\item anschließend wird übergebe die Kontrolle an \printf indem dessen Adresse ins \ac{PC}-Register geschrieben wird.
\end{itemize}

Wenn \printf die Ausführung beendet, müssen Informationen vorliegen, wo die Ausführung weitergehen soll.
Das ist der Grund warum jede Funktion die Kontrolle an die Adresse, gespeichert im \ac{LR}-Register übergibt.

Dies ist ein Unterschied zwischen einem \q{reinem} \ac{RISC}-Prozessor wie ARM und \ac{CISC}-Prozessoren wie x86,
bei denen die Rücksprungadresse in der Regel auf dem Stack gespeichert wird.
Mehr dazu ist im nächsten Abschnitt zu lesen~(\myref{sec:stack}).

Übrigens eine absolute 32-Bit-Adresse oder -Offset kann nicht in einer 32-Bit-\TT{BL}-Anweisung kodiert werden,
weil diese nur für 24 Bit Platz bietet. Wie bereits erwähnt haben alle ARM-Mode-Anweisungen eine Größe
von 4 Byte (32 Bit). Aus diesem Grund können diese nur an 4-Byte-Grenzen des Speichers platziert werden.
Dies heißt auch, das die letzten zwei Bit der Anweisungsadresse (die immer Null sind) entfallen können.
Zusammenfassend, stehen 26 Bit für die Offset-Kodierung zur Verfügung. Dies ist genug für
$current\_PC \pm{} \approx{}32M$.

\myindex{ARM!\Instructions!MOV}
Als nächstes schreibt die Anweisung \INS{MOV R0, \#0}\footnote{das heißt MOVe} lediglich 0 in
das \Reg{0}-Register weil der Rückgabewert hier gespeichert wird und die gezeigte C-Funktion 0
als Argument für die return-Anweisung hat.

\myindex{ARM!\Registers!Link Register}
\myindex{ARM!\Instructions!LDMFD}
\myindex{ARM!\Instructions!POP}
Die letzte Anweisung \INS{LDMFD SP!, {R4,PC}}\footnote{\ac{LDMFD} ist eine inverse Anweisung von \ac{STMFD}}
lädt die Werte nacheinander vom Stack (oder eine andere Speicheradresse) um sie in die Register \Reg{4} und \ac{PC}
zu sichern. Außerdem wird der Stack Pointer \ac{SP} inkrementiert. Hier arbeitet der Befehl wie \POP.

Die erste Anweisung \TT{STMFD} sichert das Register-Paar \Reg{4} und \ac{LR} auf dem Stack, jedoch werden \Reg{4} und \ac{PC}
während der Ausführung von \TT{LDMFD} \IT{wiederhergestellt}.

Wie bereits bekannt, wird die Adresse die nach der Ausführung einer Funktion angesprungen wird in dem \ac{LR}-Register gesichert.
Die allererste Anweisung sichert diese Wert auf dem Stack weil das gleiche Register von der \main-Funktion genutzt wird,
wenn \printf aufgerufen wird.
Am Ende der Funktion kann dieser Wert direkt in das \ac{PC}-Register geschrieben werden und so die Ausführung an der
Stelle fortgesetzt werden an der die Funktion aufgerufen wurde.

Da \main in der Regel die erste Funktion in \CCpp ist, wird die Kontrolle an das \ac{OS} oder einen Punkt in der
\ac{CRT} übergeben.

All dies erlaubt das Auslassen der \INS{BX LR}-Anweisung am Ende der Funktion.

\myindex{ARM!DCB}
\TT{DCB} ist eine Assemblerdirektive die ein Array von Bytes oder ASCII anlegt, ähnlich der DB-Direktive
in der x86-Assembler-Sprache.

\subsubsection{\NonOptimizingKeilVI (\ThumbMode)}

Nachfolgend das gleiche Beispiel mit dem Keil-Compiler im Thumb-Mode erstellt:

\begin{lstlisting}
armcc.exe --thumb --c90 -O0 1.c 
\end{lstlisting}

In \IDA wird folgende Ausgabe erzeugt:

\begin{lstlisting}[caption=\NonOptimizingKeilVI (\ThumbMode) + \IDA,style=customasmARM]
.text:00000000             main
.text:00000000 10 B5          PUSH    {R4,LR}
.text:00000002 C0 A0          ADR     R0, aHelloWorld ; "hello, world"
.text:00000004 06 F0 2E F9    BL      __2printf
.text:00000008 00 20          MOVS    R0, #0
.text:0000000A 10 BD          POP     {R4,PC}

.text:00000304 68 65 6C 6C+aHelloWorld  DCB "hello, world",0    ; DATA XREF: main+2
\end{lstlisting}

Leicht zu erkennen sind die 2-Byte (16 Bit) OpCodes, die wie bereits erwähnt Thumb-Anweisungen sind.
\myindex{ARM!\Instructions!BL}
Die \TT{BL}-Anweisung besteht aus zwei 16-Bit-Anweisungen, weil es für die \printf-Funktion unmöglich ist
einen Offset zu laden, wenn der kleine Speicherbereich in einem 16-Bit-Opcode genutzt wird.
Aus diesem Grund lädt die erste 16-Bit-Anweisung die höherwertigen 10 Bit des Offsets und die zweite
Anweisung die niederwertigen 11 Bit.

% TODO:
% BL has space for 11 bits, so if we don't encode the lowest bit,
% then we should get 11 bits for the upper half, and 12 bits for the lower half.
% And the highest bit encodes the sign, so the destination has to be within
% \pm 4M of current_PC.
% This may be less if adding the lower half does not carry over,
% but I'm not sure --all my programs have 0 for the upper half,
% and don't carry over for the lower half.
% It would be interesting to check where __2printf is located relative to 0x8
% (I think the program counter is the next instruction on a multiple of 4
% for THUMB).
% The lower 11 bytes of the BL instructions and the even bit are
% 000 0000 0110 | 001 0010 1110 0 = 000 0000 0110 0010 0101 1100 = 0x00625c,
% so __2printf should be at 0x006264.
% But if we only have 10 and 11 bits, then the offset would be:
% 00 0000 0110 | 01 0010 1110 0 = 0 0000 0011 0010 0101 1100 = 0x00325c,
% so __2printf should be at 0x003264.
% In this case, though, the new program counter can only be 1M away,
% because of the highest bit is used for the sign.

Wie erwähnt haben alle Anweisungen im Thumb-Mode eine Größe von 2 Byte (16 Bit).
Dies bedeutet, dass es unmöglich ist an einer ungeraden Adresse einen Anweisung unterzubringen.
Das hat auch zur Folge, dass das letzte Bit der Adresse bei der Kodierung der
Anweisungen weggelassen werden kann.

Zusammenfassend kann die \TT{BL}-Thumb-Anweisung eine Adresse bis $current\_PC \pm{}\approx{}2M$ kodieren.

\myindex{ARM!\Instructions!PUSH}
\myindex{ARM!\Instructions!POP}
Wie für die anderen Anweisungen in dieser Funktion arbeiten \PUSH und \POP wie die beschriebenden \TT{STMFD}/\TT{LDMFD},
nur dass das \ac{SP}-Register hier nicht explizit genannt wird.
\TT{ADR} arbeitet genau wie in dem vorherigen Beispiel.
\TT{MOVS} schreibt 0 in das Register \Reg{0} um 0 zurückzugeben.

\subsubsection{\OptimizingXcodeIV (\ARMMode)}

Xcode 4.6.3 ohne Optimierung produziert eine Menge redundanten Code, so dass im Folgenden die
optimierte Ausgabe gelistet ist bei der die Anzahl der Anweisungen so klein wie möglich ist.
Der Compiler-Schalter ist \Othree.

\begin{lstlisting}[caption=\OptimizingXcodeIV (\ARMMode),style=customasmARM]
__text:000028C4             _hello_world
__text:000028C4 80 40 2D E9   STMFD           SP!, {R7,LR}
__text:000028C8 86 06 01 E3   MOV             R0, #0x1686
__text:000028CC 0D 70 A0 E1   MOV             R7, SP
__text:000028D0 00 00 40 E3   MOVT            R0, #0
__text:000028D4 00 00 8F E0   ADD             R0, PC, R0
__text:000028D8 C3 05 00 EB   BL              _puts
__text:000028DC 00 00 A0 E3   MOV             R0, #0
__text:000028E0 80 80 BD E8   LDMFD           SP!, {R7,PC}

__cstring:00003F62 48 65 6C 6C+aHelloWorld_0  DCB "Hello world!",0
\end{lstlisting}

Die Anweisungen \TT{STMFD} und \TT{LDMFD} sind bereits bekannt.

\myindex{ARM!\Instructions!MOV}

Die \MOV-Anweisung schreibt lediglich die Nummer \TT{0x1686} in das Register \Reg{0}.
Dies ist der Offset der auf die Zeichenkette \q{Hello world!} zeigt.

Das Register \TT{R7} (spezifiziert in \IOSABI) ist ein Frame Pointer. Mehr darüber folgt später.

\myindex{ARM!\Instructions!MOVT}
Die \TT{MOVT R0, \#0} (MOVe Top)-Anweisung schreibt 0 in die höherwertigen 16 Bit des Registers.
Das Problem ist hier, dass die generische \MOV-Anweisung im ARM-Mode nur die niederwertigen 16 Bit
des Registers beschreibt.

Dran denken: alle Opcodes im ARM-Mode sind in der Größe auf 32 Bit begrenzt. Natürlich gilt diese
Begrenzung nicht für das Verschieben von Daten zwischen Registern.
Aus diesem Grund existiert die zusätzliche Anweisung  \TT{MOVT} um in die höherwertigen Bits
(von 16 bis einschließlich 31) zu beschreiben.
Die Benutzung ist in diesem Fall redundant, weil die Anweisung \TT{MOV R0, \#0x1686} darüpber
den höherwertigen Teil des Registers zurückgesetzt hat.
Dies ist vermutlich ein Mangel des Compilers.

% TODO:
% I think, more specifically, the string is not put in the text section,
% ie. the compiler is actually not using position-independent code,
% as mentioned in the next paragraph.
% MOVT is used because the assembly code is generated before the relocation,
% so the location of the string is not yet known,
% and the high bits may still be needed.

\myindex{ARM!\Instructions!ADD}
Die Anweisung \TT{ADD R0, PC, R0} addiert den Wert im \ac{PC} zum Wert im Register \Reg{0}
um die absolute Adresse der \q{Hello world!}-Zeichenkette zu berechnen.
Wie bereits bekannt ist dies \q{\PICcode}, so dass diese Korrektur hier unbedingt notwendig ist.

Die \INS{BL}-Anweisung ruft \puts anstatt \printf auf.

\label{puts}
\myindex{\CStandardLibrary!puts()}
\myindex{puts() anstatt printf()}

GCC ersetzt den ersten \printf-Aufruf mit \puts. In der Tat ist \printf mit nur einem
Argument identisch mit \puts.

Die beiden Funktionen produzieren lediglich das gleiche Ergebnis, weil printf keine
Formatkennzeichner, beginnend mit \IT{\%}, enhält.
Sollte dies jedoch der Fall sein, wäre die Auswirkung der beiden Funktionen
unterschiedlich\footnote{Des weiteren benötigt \puts kein '\textbackslash{}n'
für den Zeilenumbruch am Ende der Zeichenkette, weswegen wir dies hier nicht sehen.}.

Warum hat der Compiler diese Ersetzung durchgeführt? Vermutlich hat dies Vorteile bei
der Geschwindigkeit, weil \puts schneller ist
\footnote{\href{http://go.yurichev.com/17063}{ciselant.de/projects/gcc\_printf/gcc\_printf.html}}
und lediglich die Zeichen zu \gls{stdout} übergibt, anstatt jedes Zeichen mit \IT{\%} zu vergleichen.

Als nächstes ist die bekannte Anweisung \TT{MOV R0, \#0} zu sehen um das Register \Reg{0} auf 0 zu setzen.

\subsubsection{\OptimizingXcodeIV (\ThumbTwoMode)}

Standardmäßig generiert Xcode 4.6.3 den Thumb-2-Code auf folgende Weise:

\begin{lstlisting}[caption=\OptimizingXcodeIV (\ThumbTwoMode),style=customasm]
__text:00002B6C                   _hello_world
__text:00002B6C 80 B5          PUSH            {R7,LR}
__text:00002B6E 41 F2 D8 30    MOVW            R0, #0x13D8
__text:00002B72 6F 46          MOV             R7, SP
__text:00002B74 C0 F2 00 00    MOVT.W          R0, #0
__text:00002B78 78 44          ADD             R0, PC
__text:00002B7A 01 F0 38 EA    BLX             _puts
__text:00002B7E 00 20          MOVS            R0, #0
__text:00002B80 80 BD          POP             {R7,PC}

...

__cstring:00003E70 48 65 6C 6C 6F 20+aHelloWorld  DCB "Hello world!",0xA,0
\end{lstlisting}

% Q: If you subtract 0x13D8 from 0x3E70,
% you actually get a location that is not in this function, or in _puts.
% How is PC-relative addressing done in THUMB2?
% A: it's not Thumb-related. there are just mess with two different segments. TODO: rework this listing.

\myindex{\ThumbTwoMode}
\myindex{ARM!\Instructions!BL}
\myindex{ARM!\Instructions!BLX}

Die \TT{BL}- und \TT{BLX}-Anweisung im Thumb-Mode ist als Paar von 16-Bit-Anweisungen kodiert.
In Thumb-2 sind diese \IT{Ersatz}-Opcodes so erweitert, dass neue Anweisungen hier mit 32 Bit
kodiert werden können.

Offensichtlich beginnen die Opcodes der Thumb-2-Anweisungen immer mit \TT{0xFx} oder \TT{0xEx}.

Im \IDA-Listing jedoch sind die Bytes der Opcodes vertauscht weil für den ARM-Prozessor die
Anweisungen wie folgt kodiert werden:
Das letzte Byte kommt zuerst und danach das erste (für Thumb- und Thum-2-Mode) oder für
Anweisungen im ARM-Mode kommt das vierte Byte zuerst, dann das dritte, dann das zweite und
zum Schluss das erste (aufgrund des unterschiedlichen \gls{endianness}).

Die Bytes sind also im \IDA-Listing wie folgt angeordnet:
\begin{itemize}
\item für ARM und ARM64 Mode: 4-3-2-1;
\item für Thumb Mode: 2-1;
\item für 16-Bit-Anweisungspaar in Thumb-2 Mode: 2-1-4-3.
\end{itemize}

\myindex{ARM!\Instructions!MOVW}
\myindex{ARM!\Instructions!MOVT.W}
\myindex{ARM!\Instructions!BLX}

Wie zu sehen ist, beginnend die Anweisungen \TT{MOVW}, \TT{MOVT.W} und \TT{BLX} mit \TT{0xFx}.

Eine der Thumb-2-Anweisungen ist \TT{MOVW R0, \#0x13D8} ~---sie speichert einen 16-Bit-Wert in den
niederwertigeren Teil des \Reg{0}-Registers und setzt die höherwertigen Bits auf 0.

Des weiteren funktioniert \TT{MOVT.W R0, \#0} genau wie \TT{MOVT} aus dem vorherigen Beispiel,
jedoch nur für Thumb-2.

\myindex{ARM!mode switching}
\myindex{ARM!\Instructions!BLX}

Neben den anderen Unterschieden wird in diesem Fall die \TT{BLX}-Anweisung anstatt \TT{BL} genutzt.

Der Unterschied ist, dass, neben dem Speichern von \ac{RA} in das \ac{LR}-Register und die Übergabe
der Ausführungskontrolle an die \puts-Funktion, der Prozessor auch vom Thumb/Thumb-2-Mode in den
ARM-Mode (oder zurück) wechselt.

Diese Anweisung ist hier eingefügt weil die Anweisung mit der die Kontrolle abgegeben wird wie folgt
aussieht (im ARM-Mode kodiert):

\begin{lstlisting}[style=customasm]
__symbolstub1:00003FEC _puts           ; CODE XREF: _hello_world+E
__symbolstub1:00003FEC 44 F0 9F E5     LDR  PC, =__imp__puts
\end{lstlisting}

Dies ist im Endeffekt ein Sprung an die Stelle an der die Adresse von \puts in der import-Sektion geschriben wird.

Der aufmerksame Leser mag fragen: warum wird \puts nicht direkt an der Stelle im Code aufgerufen,
an der es benötigt wird? Dies wäre nicht sehr speicherplatzeffizient.

\myindex{Dynamically loaded libraries}
Fast jedes Programm nutzt externe, dynamische Bibliotheken (wie DLL in Windows, .so in *NIX oder.dylib in \MacOSX).
Diese Bibliotheken beinhalten häufig genutzte Funktion wie die Standard-C-Funktion \puts.

\myindex{Relocation}
In einer ausführbaren Binärdatei (Windows PE .exe, ELF oder Mach-O) existiert eine import-Sektion.
Dies ist eine Liste von Symbolen (Funktionen oder globale Variablen) die, zusammen mit den Namen, von
externen Modulen importiert werden.

Der \ac{OS}-Loader läd alle Module die gebraucht werden und bestimmt die korrekten Adressen von jedem Symbol,
während diese in dem primärem Modul aufgelistet werden.

In dem vorliegenden Fall ist \IT{\_\_imp\_\_puts} eine 32-Bit-Variable die vom \ac{OS}-Loader genutzt wird
um die korrekte Adresse der Funktion in der externen Bibliothek zu speichern.
Anschließend liest die \TT{LDR}-Anweisung den 32-Bit-Wert dieser Variable und schreibt ihn in das \ac{PC}-Register
bevor die Ausführkontrolle dorthin übergeben wird.

Um also die Zeit zu reduzieren die der \ac{OS}-Loader für dieses Vorgehen benötigt, ist es eine gute Idee
die Adressen für jedes Symbol einmalig an eine geeignete Stelle zu schreiben.

\myindex{thunk-functions}
Daneben wurde bereits erwähnt, dass es unmöglich ist einen 32-Bit-Wert in ein Register zu laden wenn
nur eine Anweisung ohne Speicher-Zugriff genutzt wird.

Aus diesem Grund ist die optimale Lösung, eine separate Funktion im ARM-Mode zu allozieren die lediglich
die Aufgabe hat die Ausführkontrolle an die dynamische Bibliothek zu übergeben und dann in diese kurze
Funktion mit einer Anweisung (so genannte \gls{thunk function}) aus dem Thumb-Code auszuführen.

\myindex{ARM!\Instructions!BL}
Übrigens: in dem vorherigen Beispiel (für ARM-Mode kompiliert) wird die Ausführkontrolle durch \TT{BL}
an die gleiche \gls{thunk function} übergeben.
Der Prozessor-Modus wird hier jedoch aufgrund des Fehlens eines \q{X} im Anweisungsnamen nicht gewechselt.

\myparagraph{Mehr über Thunk-Funktionen}
\myindex{thunk-functions}

Thunk-Funktionen sind aufgrund der irrtümlichen Bezeichnung schwierig zu verstehen.
Der einfachste Weg ist es sie als Adapter oder Konverter zwischen verschiedenen Anschlüssen aufzufassen.
Zum Beispiel wie einen Adapter zwischen einer britischen und einer amerikanischen Steckdose oder andersherum.
Thunk-Funktionen werden manchmal auch \IT{Wrapper} genannt.

Hier sind einige weitere Beschreibung dieser Funktionstypen:

\begin{framed}
\begin{quotation}
"Ein Teil der Software um Adressen zur Verfügung zu stellen:" nach P. Z. Ingerman,
der 1961 Thunk-Funktionen als Möglichkeit zum Binden von Aktualparametern zu deren
formalen Definitionen in Algol-60-Prozedur-Aufrufen. Wenn eine Prozedur mit einem Ausdruck anstatt
der formalen Parameter aufgerufen wird, generiert der Compiler eine Thunk-Funktion die den Ausdruck
errechnet und die Adresse des Ergebnisses an eine Standard-Stelle speichert.

\dots

% TODO: the english version is kind of misleading -- I could not find the word "bletcherous"
Microsoft und IBM haben beide in ihrem Intel-basierten System eine "16-Bit Umgebung"
und eine "32-Bit-Umgebung" definiert. Beide können auf dem selben Computer und demselben
Betriebssystem laufen (dank dem was Microsoft \q{ Windows On Windows} (WOW) nennt).
Sowohl MS als auch IBM haben entschieden, den Vorgang der zwischen 16- und 32-Bit wechselt
"Thunk" zu nennen; für Windows 95 existiert sogar ein Tool THUNK.EXE, das Thunk-Compiler
genannt wird.\end{quotation}
\end{framed}
% TODO FIXME move to bibliography and quote properly above the quote
( \href{http://go.yurichev.com/17362}{The Jargon File} )

\subsubsection{ARM64}

\myparagraph{GCC}

Das Beispiel wird im Folgenden mit GCC 4.1.8 in ARM64 kompiliert:

\lstinputlisting[numbers=left,label=hw_ARM64_GCC,caption=\NonOptimizing GCC 4.8.1 + objdump,style=customasm]{patterns/01_helloworld/ARM/hw.lst}

Es gibt keine Thumb- oder Thumb-2-Modes in ARM64, sondern nur ARM, also 32-Bit-Anweisungen.
Die Register-Anzahl ist verdoppelt: \myref{ARM64_GPRs}.
64-Bit-Register haben einen \TT{X-}Prefix, 32-Bit-Teile ein \TT{W-}.

\myindex{ARM!\Instructions!STP}
Die \TT{STP}-Anweisung (\IT{Store Pair}) speichert zwei Register auf dem Stack gleichzeitig:
\RegX{29} und \RegX{30}.

Natürlich kann diese Anweisung dieses Registerpaar an einer beliebigen Stelle im Speicher
sichern, aber da hier das \ac{SP}-Register angegeben ist, wird das Paar auf dem Stack gesichert.

ARM64-Register sind 64 Bit breit, jedes von ihnen ist 8 Byte groß. Dementsprechend werden 16 Byte
für das Speichern zweier Register benötigt.

Das Ausrufungszeichen (``!'')  nach dem Operanden bedeutet, dass zunächst der Wert 16 vom \ac{SP}
subtrahiert werden muss und erst dann die Werte vom Register-Paar auf den Stack geschrieben werden.
Dies wird auch \IT{pre-index} genannt.
Mehr über den Unterschied von \IT{post-index} und \IT{pre-index} ist im Abschnitt
\myref{ARM_postindex_vs_preindex} zu finden.

Im Sprachgebrauch des gebräuchlicheren x86, ist die erste Anweisung analog zu den Anweisungen
\TT{PUSH X29} und \TT{PUSH X30} zu verstehen.
\RegX{29} wird als \ac{FP} in ARM64 genutzt, und \RegX{30} als \ac{LR}, weswegen sie am Anfang der
Funktion gesichert und am Ende wiederhergestellt werden.

Die zweite Anweisung kopiert \ac{SP} in \RegX{29} (oder \ac{FP}) um den Stack Frame vorzubereiten.

\label{pointers_ADRP_and_ADD}
\myindex{ARM!\Instructions!ADRP/ADD pair}
\TT{ADRP} und \ADD-Anweisungen werden genutzt um die Adresse der Zeichenkette \q{Hello!} in das
Register \RegX{0} zu schreiben, da das erste Funktionsargument in an dieser Stelle übergeben wird. 

Es gibt in ARM keine Anweisung, die eine große Zahl in einem Register sichern kann, weil die Länge der
Anweisungen auf 4 Byte begrenzt ist. Siehe dazu auch \myref{ARM_big_constants_loading}).
Aus diesem Grund müssen mehrere Anweisungen genutzt werden. Die erste (\TT{ADRP}) schreibt die Adresse
der 4KiB-Page in der die Zeichenkette sich befindet in das Register \RegX{0}.
Die zweite (\ADD) addiert lediglich den Rest der Adresse.
Siehe dazu auch \myref{ARM64_relocs}.

\TT{0x400000 + 0x648 = 0x400648}, und die Zeichenkette \q{Hello!} ist im \TT{.rodata} Daten-Segmet
an dieser Adresse zu sehen.

\myindex{ARM!\Instructions!BL}

\puts wird anschließend mit der \TT{BL}-Anweisung aufgerufen. Dies wurde bereits diskutiert: \myref{puts}.

\MOV schreibt 0 in \RegW{0}.
\RegW{0} sind die niederwertigeren 32 Bit des 64-Bit-Registers \RegX{0}:

\begin{center}
\begin{tabular}{ | l | l | }
\hline
\RU{Старшие 32 бита}\EN{High 32-bit part}\ES{Parte alta de 32 bits}\PTBRph{}\PLph{}\ITAph{}\DEph{}\THAph{}\NLph{}\FR{Partie 32 bits haute} & \RU{младшие 32 бита}\EN{low 32-bit part}\ES{parte baja de 32 bits}\PTBRph{}\PLph{Starsze 32 bity}\ITAph{}\DEph{}\THAph{}\NLph{}\FR{Partie 32 bits basse} \\
\hline
\multicolumn{2}{ | c | }{X0} \\
\hline
\multicolumn{1}{ | c | }{} & \multicolumn{1}{ c | }{W0} \\
\hline
\end{tabular}
\end{center}


Das Ergebnis der Funktion wird über \RegX{0} zurückgegeben und \main gibt 0 zurück.
Dies ist also der Weg wie das Ergebnis vorbereitet wird.
Der 32-Bit-Teil wird genutzt,weil der \Tint-Datentyp in ARM64 aus Kompatibilitätsgründen,
wie in x86-64, 32 Bit breit ist.

Da die Funktion einen 32-Bit \Tint-Wert zurück gibt, müssen lediglich die unteren 32 Bits des
\RegX{0}-Registers gefüllt werden.

Um dies zu überprüfen wird das Beispiel leicht verändert und neu kompiliert.
\main soll nun einen 64-Bit-Wert zurück geben:

\begin{lstlisting}[caption=\main gibt einen \TT{uint64\_t}-Datentyp zurück,style=customc]
#include <stdio.h>
#include <stdint.h>

uint64_t main()
{
        printf ("Hello!\n");
        return 0;
}
\end{lstlisting}

Das Ergebnis ist das gleiche, allerdings sieht \MOV nun wie folgt aus:

\begin{lstlisting}[caption=\NonOptimizing GCC 4.8.1 + objdump]
  4005a4:       d2800000        mov     x0, #0x0      // #0
\end{lstlisting}

\myindex{ARM!\Instructions!LDP}

\INS{LDP} (\IT{Load Pair}) stellt anschließend die Register \RegX{29} und \RegX{30} wieder her.

An dieser Stelle steht kein Ausrufungszeichen nach der Anweisung: dies impliziert, dass der Wert zunächst
vom Stack gelesen wird und erst dann wird \ac{SP} um den Wert 16 verringert.
Dies wird \IT{post-index} genannt.

\myindex{ARM!\Instructions!RET}
Eine neue Anweisung taucht hier in ARM64 auf \RET.
Diese arbeitet wie \TT{BX LR}, jedoch wird ein spezielles \IT{Hinweis-}Bit hinzugefügt, welches die \ac{CPU}
darüber informiert, dass dies ein Rücksprung aus einer Funktion ist und kein anderer Sprung, so dass die
Ausführung optimiert werden kann.

Aufgrund der Einfachheit dieser Funktion, erstellt der optimierende GCC den gleichen Code.

