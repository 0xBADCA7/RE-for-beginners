\subsubsection{\NonOptimizingKeilVI (\ThumbMode)}

Nachfolgend das gleiche Beispiel mit dem Keil-Compiler im Thumb-Mode erstellt:

\begin{lstlisting}
armcc.exe --thumb --c90 -O0 1.c 
\end{lstlisting}

In \IDA wird folgende Ausgabe erzeugt:

\begin{lstlisting}[caption=\NonOptimizingKeilVI (\ThumbMode) + \IDA]
.text:00000000             main
.text:00000000 10 B5          PUSH    {R4,LR}
.text:00000002 C0 A0          ADR     R0, aHelloWorld ; "hello, world"
.text:00000004 06 F0 2E F9    BL      __2printf
.text:00000008 00 20          MOVS    R0, #0
.text:0000000A 10 BD          POP     {R4,PC}

.text:00000304 68 65 6C 6C+aHelloWorld  DCB "hello, world",0    ; DATA XREF: main+2
\end{lstlisting}

Leicht zu erkennen sind die 2-Byte (16 Bit) OpCodes, die wie bereits erwähnt Thumb-Anweisungen sind.
\myindex{ARM!\Instructions!BL}
Die \TT{BL}-Anweisung besteht aus zwei 16-Bit-Anweisungen, weil es für die \printf-Funktion unmöglich ist
einen Offset zu laden, wenn der kleine Speicherbereich in einem 16-Bit-Opcode genutzt wird.
Aus diesem Grund lädt die erste 16-Bit-Anweisung die höherwertigen 10 Bit des Offsets und die zweite
Anweisung die niederwertigen 11 Bit.

% TODO:
% BL has space for 11 bits, so if we don't encode the lowest bit,
% then we should get 11 bits for the upper half, and 12 bits for the lower half.
% And the highest bit encodes the sign, so the destination has to be within
% \pm 4M of current_PC.
% This may be less if adding the lower half does not carry over,
% but I'm not sure --all my programs have 0 for the upper half,
% and don't carry over for the lower half.
% It would be interesting to check where __2printf is located relative to 0x8
% (I think the program counter is the next instruction on a multiple of 4
% for THUMB).
% The lower 11 bytes of the BL instructions and the even bit are
% 000 0000 0110 | 001 0010 1110 0 = 000 0000 0110 0010 0101 1100 = 0x00625c,
% so __2printf should be at 0x006264.
% But if we only have 10 and 11 bits, then the offset would be:
% 00 0000 0110 | 01 0010 1110 0 = 0 0000 0011 0010 0101 1100 = 0x00325c,
% so __2printf should be at 0x003264.
% In this case, though, the new program counter can only be 1M away,
% because of the highest bit is used for the sign.

Wie erwähnt haben alle Anweisungen im Thumb-Mode eine Größe von 2 Byte (16 Bit).
Dies bedeutet, dass es unmöglich ist an einer ungeraden Adresse einen Anweisung unterzubringen.
Das hat auch zur Folge, dass das letzte Bit der Adresse bei der Kodierung der
Anweisungen weggelassen werden kann.

Zusammenfassend kann die \TT{BL}-Thumb-Anweisung eine Adresse bis $current\_PC \pm{}\approx{}2M$ kodieren.

\myindex{ARM!\Instructions!PUSH}
\myindex{ARM!\Instructions!POP}
Wie für die anderen Anweisungen in dieser Funktion arbeiten \PUSH und \POP wie die beschriebenden \TT{STMFD}/\TT{LDMFD},
nur dass das \ac{SP}-Register hier nicht explizit genannt wird.
\TT{ADR} arbeitet genau wie in dem vorherigen Beispiel.
\TT{MOVS} schreibt 0 in das Register \Reg{0} um 0 zurückzugeben.
