\subsubsection{String patching (Win32)}

We can easily find ``hello, world'' string in executable file using Hiew:

\begin{figure}[H]
\centering
\myincludegraphics{patterns/01_helloworld/hola_edit1.png}
\caption{Hiew}
\label{}
\end{figure}

And we can try to translate our message to Spanish language:

\begin{figure}[H]
\centering
\myincludegraphics{patterns/01_helloworld/hola_edit2.png}
\caption{Hiew}
\label{}
\end{figure}

Spanish text is one byte shorter than English, so we also add 0x0A byte at the end (\TT{\textbackslash{}n}) and zero byte.

It works.

What if we want to insert longer message?
There are some zero bytes after original English text.
Hard to say if they can be overwritten: they may be used somewhere in \ac{CRT} code, or maybe not.
Anyway, you can overwrite them only if you really know what you are doing.

\subsubsection{String patching (Linux x64)}

\myindex{\radare}
Let's try to patch Linux x64 executable using \radare{}:

\lstinputlisting[caption=\radare{} session]{patterns/01_helloworld/radare.lst}

What I do here: I search for \q{hello} string using \TT{/} command, 
then I set \IT{cursor} (or \IT{seek} in \radare{} terms) to that address.
Then I want to be sure that this is really that place: \TT{px} dumps bytes there.
\TT{oo+} switches \radare{} to \IT{read-write} mode.
\TT{w} writes ASCII string at the current \IT{seek}.
Note \TT{\textbackslash{}00} at the end---this is zero byte.
\TT{q} quits.

\subsubsection{Software \IT{localization} of MS-DOS era}

The way I described was a common way to translate MS-DOS software to Russian language back to 1980's and 1990's.
Russian words and sentences are usually slightly longer than its English counterparts, so that is why \IT{localized}
software has a lot of weird acronyms and hardly readable abbreviations.

Perhaps, this also happened to other languages during that era, in other countries.

