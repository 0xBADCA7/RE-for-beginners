\subsection{MIPS}

\subsubsection{O \q{wskaźniku globalnym} (\q{global pointer})}
\label{MIPS_GP}

\myindex{MIPS!\GlobalPointer}
\q{Wskaźnik globalny} (\q{global pointer})~--- jest bardzo ważną koncepcją MIPS.
Jak już wiemy, każda instrukcja w MIPS ma długość 32 bity, dlatego kodowanie 32-bitowego adresu w jednej instrukcji nie jest możliwe. Zamiast tego trzeba wykorzystać parę instrukcji
(jak to zrobił GCC dla załadowania adresu linii tekstowej w naszym przykładzie).
Z innej strony, korzystając tylko z jednej instrukcji, 
można ładować dane do adresów w granicach $register-32768...register+32767$, dlatego że 16 bitów
przesunięcia można zakodować w jednej instrukcji).
Także możemy przydzielić jakiś rejestr do tych celów i jeszcze bufor 64KiB dla najczęściej wykorzystywanych danych.
Przydzielony rejestr nazywamy \q{wskaźnikiem globalnym} (\q{global pointer}) i wskazuje on na środek buforu 64KiB.
Ten bufor zwykle zawiera zmienne globalne oraz adresy funkcji importowanych typu \printf,
dlatego że deweloperzy GCC stwierdzili, że dostęp do adresu funkcji musi być jak najszybszy,
wykonywany jako jedna instrukcja zamiast dwóch.
W ELF-pliku ten 64KiB-bufor znajduje się po części w segmencie .sbss (\q{small \ac{BSS}}) dla danych niezainicjalizowanych i w segmencie .sdata (\q{small data}) dla danych zainicjalizowanych.
To oznacza, że programista może wybierać, do czego potrzebuje szybszego dostępu, i następnie rozmieścić to
w segmentach .sdata/.sbss.
Niektórzy \q{old-school} programiści mogą pamiętać model pamięci w MS-DOS \myref{8086_memory_model} 
lub w menadżerach pamięci typu XMS/EMS, gdzie cała pamięć była podzielona na bloki po 64KiB długości.

\myindex{PowerPC}
Ta koncepcja jest stosowana nie tylko w MIPS. Przynajmniej PowerPC również korzysta z tej techniki.

\subsubsection{\Optimizing GCC}

Popatrzmy na następujący przykład, realizujący koncepcję \q{wskaźnika globalnego}.

\lstinputlisting[caption=\Optimizing GCC 4.4.5 (\assemblyOutput),numbers=left,style=customasmMIPS]{patterns/01_helloworld/MIPS/hw_O3_PL.s}

Jak widać, rejestr \$GP w prologu funkcji wystawia się na środek tego obszaru.
Rejestr \ac{RA} jest odkładany na stos lokalny.
Tutaj kompilator również skorzystał z \puts zamiast \printf.
\myindex{MIPS!\Instructions!LW}
Adres funkcji \puts jest ładowany do \$25 za pomocą instrukcji \INS{LW} (\q{Load Word}).
\myindex{MIPS!\Instructions!LUI}
\myindex{MIPS!\Instructions!ADDIU}
Następnie adres linii tekstowej jest ładowany do \$4 za pomocą pary instrukcji \INS{LUI} (\q{Load Upper Immediate}) i
\INS{ADDIU} (\q{Add Immediate Unsigned Word}).
\INS{LUI} ustawia starsze 16 bitów rejestru (dlatego w nazwie jest obecne \q{upper}) i \INS{ADDIU}
dodaje młodsze 16 bitów do adresu.
\INS{ADDIU} idzie za \INS{JALR} (pamiętając o \IT{branch delay slots}).
Rejestr \$4 także jest nazywany \$A0, który jest wykorzystywany do przekazywania pierwszego argumentu funkcji
\footnote{Tablica rejestrów MIPS jest dostępna w dodatku \myref{MIPS_registers_ref}}.
\myindex{MIPS!\Instructions!JALR}
\INS{JALR} (\q{Jump and Link Register}) robi przejście wg adresu zawartego w \$25 (tam się znajduje adres \puts) 
jednocześnie zapisując adres następnej instrukcji (\INS{LW}) w \ac{RA}.
Wygląda to tak samo jak i w ARM.
I jeszcze jedna bardzo ważna rzecz: adres zapisywany do \ac{RA} nie jest adresem następnej instrukcji (dlatego, że jest to
\IT{delay slot} i wykonuje się przed instrukcją przejścia),
a instrukcji następującej po \IT{delay slot}.
W ten sposób podczas wykonania \INS{JALR} do \ac{RA} jest zapisywane $PC + 8$.
W naszym wypadku jest to adres instrukcji \INS{LW} kolejnej po \INS{ADDIU}.

\INS{LW} (\q{Load Word}) w linii 20 przywraca \ac{RA} ze stosu lokalnego (ta instrukcja jest raczej częścią epilogu f-cji).

\myindex{MIPS!\Pseudoinstructions!MOVE}
\INS{MOVE} w linii 22 kopiuje wartość z \$0 (\$ZERO) do \$2 (\$V0).

\label{MIPS_zero_register}
W MIPS istnieje rejestr dla stałych, zawsze zawierający zero.
Możliwe, że deweloperzy MIPS stwierdzili, że 0 jest najbardziej zapotrzebowaną stałą w programowaniu,
także niech rejestr \$0, będzie wykorzystywany za każdym razem, kiedy będzie potrzebne 0.
Inna ciekawostka: w MIPS nie ma instrukcji, kopiującej wartość z rejestru do rejestru.
Naprawdę, \TT{MOVE DST, SRC} to \TT{ADD DST, SRC, \$ZERO} ($DST=SRC+0$), która robi to samo.
Najwidoczniej, twórcy MIPS chcieli stworzyć jak najbardziej zwięzłą tablicę opcode'ów.
To wcale nie znaczy, że dodawanie jest wykonywane podczas każdej instrukcji \INS{MOVE}.
Prawdopodobnie, te pseudo-instrukcje są optymalizowane w \ac{CPU} i \ac{ALU} nigdy nie jest wykorzystywane.

\myindex{MIPS!\Instructions!J}
\INS{J} w linii 24 robi przejście pod adres w \ac{RA}, i to działa jako wyjście z funkcji.
\INS{ADDIU} po \INS{J} wprawdzie jest wykonywane przed \INS{J} (\IT{branch delay slots}?) 
jest to częścią epilogu funkcji.

To jest listing wygenerowany w \IDA. Każdy rejestr posiada swoją pseudonazwę:

\lstinputlisting[caption=\Optimizing GCC 4.4.5 (\IDA),numbers=left,style=customasmMIPS]{patterns/01_helloworld/MIPS/hw_O3_IDA_PL.lst}

Instrukcja w linii 15 odkłada GP na lokalny stos. Ta instrukcja w dziwny sposób nie jest obecna na listingu GCC, możliwe, że jest to spowodowane błędem w samym GCC\footnote{Najwidoczniej, f-cja wygenerowania listingów nie jest krytyczna
dla użytkowników GCC, dlatego tam mogą być obecne niepoprawione błędy.}.
Wartość GP musi być zapisana, dlatego że każda funkcjamoże pracować ze swoim własnym obszarem danych o długości 64KiB.
Rejestr, zawierający adres \puts to \$T9, dlatego że rejestry z prefiksem T- nazywają się
\q{temporaries} i ich zawartości można nie zapisywać.

\subsubsection{\NonOptimizing GCC}

\NonOptimizing GCC generuje większy objętościowo kod.

\lstinputlisting[caption=\NonOptimizing GCC 4.4.5 (\assemblyOutput),numbers=left,style=customasmMIPS]{patterns/01_helloworld/MIPS/hw_O0_PL.s}

Widzimy, że FP jest wykorzystywany jako frame pointer.
Również widzimy 3 instrukcje \ac{NOP}.
Drugi i trzeci \ac{NOP} występują po instrukcjach skoku.
Najwidoczniej, kompilator GCC zawsze dodaje \ac{NOP}-y (przez \IT{branch delay slots})
po instrukcjach skoku i następnie, jeśli optymalizacja jest włączona, może się od nich pozbywać.
Także one tutaj zostały.

Jeszcze jeden listing w \IDA:

\lstinputlisting[caption=\NonOptimizing GCC 4.4.5 (\IDA),numbers=left,style=customasmMIPS]{patterns/01_helloworld/MIPS/hw_O0_IDA_PL.lst}

\myindex{MIPS!\Pseudoinstructions!LA}
Ciekawe, że \IDA zebrała parę instrukcji \INS{LUI}/\INS{ADDIU} w jedną pseudoinstrukcję 
\INS{LA} (\q{Load Address}) w linii 15.
Również widzimy, że długość tej pseudoinstrukcji to 8 bajt!
Jest to pseudoinstrukcja (lub \IT{makro}), dlatego że nie jest to typowa instrukcja MIPS, a raczej wygodna nazwa zbiorowa dla tych dwóch instrukcji.

\myindex{MIPS!\Pseudoinstructions!NOP}
\myindex{MIPS!\Instructions!OR}
Jeszcze coś: \IDA nie rozpoznała \ac{NOP}-instrukcje w liniach 22, 26 i 41.

Jest to \TT{OR \$AT, \$ZERO}.
Generalnie, jest to instrukcja \IT{LUB} stosowana do zawartości rejestru \$AT z zerem,
co, oczywiście, jest pustą instrukcją.
MIPS, jak i inne \ac{ISA}, nie posiada oddzielnej \ac{NOP}-instrukcji.

\subsubsection{Rola stack frame w tym przykładzie}

Adres linii tekstowej jest przekazywany przez rejestr.
Po co w takim razie ustawiać stos lokalny?
Dzieje się tak dlatego, że wartości rejestrów \ac{RA} i GP muszą być gdzieś zapisane
(dlatego że jest wywoływane \printf) i po to się korzysta ze stosu lokalnego.

Gdyby to była \gls{leaf function}, to wtedy można by było pozbyć prologu i epilogu funkcji. Na przykład:
 \myref{MIPS_leaf_function_ex1}.

\subsubsection{\Optimizing GCC: załadujemy do GDB}

\myindex{GDB}
\lstinputlisting[caption=przykład sesji w GDB]{patterns/01_helloworld/MIPS/O3_GDB.txt}


