\subsubsection{GCC: x86-64}

\myindex{x86-64}
Spróbujmy GCC в 64-bitowym Linux:

\lstinputlisting[caption=GCC 4.4.6 x64,style=customasmx86]{patterns/01_helloworld/GCC_x64_RU.s}

W Linux, *BSD i \MacOSX dla x86-64 także przekazuje się argumenty funkcji poprzez rejestry \SysVABI.

6 pierwszych argumentów są przekazywane przez rejestry \RDI, \RSI, \RDX, \RCX, \Reg{8}, \Reg{9}, a reszta --- przez stos.

Także wskaźnik na linię jest przekazywany przez \EDI (32-bitową część rejestru).
Ale dlaczego nie przez 64-bitową część, \RDI?

Warto zapamiętać, że w 64-bitowym trybie wszystkie instrukcje \MOV, zapisujące cokolwiek do młodszej 32-bitowej części rejestru, zerują starsze 32-bity (Jest to opisane w dokumentacji Intel: \myref{x86_manuals}).
Czyli instrukcja \INS{MOV EAX, 011223344h} poprawnie zapisze tę wartość do \RAX, a starsze bity się wyzerują.

Jeśli podejrzeć w \IDA skompilowany plik (.o), to można zobaczyć również opcode wszystkich instrukcji
\footnote{To trzeba wskazać w \textbf{Options $\rightarrow$ Disassembly $\rightarrow$ Number of opcode bytes}}:

\lstinputlisting[caption=GCC 4.4.6 x64,style=customasmx86]{patterns/01_helloworld/GCC_x64.lst}

\label{hw_EDI_instead_of_RDI}
Jak widać, instrukcja, która zapisuje do \EDI wg adresu \TT{0x4004D4}, zajmuje 5 bajt.
Ta sama instrukcja, zapisująca 64-bitową wartość do \RDI, zajmuje 7 bajt.
Możliwe, że GCC stwierdził, że trochę zaoszczędzi.
Do tego, on jest pewien, że segment danych, gdzie są przechowywane linie, nigdy nie będzie usytuowany powyżej 4\gls{GiB}.

\label{SysVABI_input_EAX}
Tutaj my również widzimy wyzerowanie rejestru \EAX przed wywołaniem \printf.
To się robi dlatego że według standardu opisanego wyżej w *NIX dla x86-64 w \EAX jest przekazywana ilość wykorzystywanych rejestrów wektorowych.


