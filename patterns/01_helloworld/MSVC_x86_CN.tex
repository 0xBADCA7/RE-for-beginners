\subsubsection{MSVC}

我们在MSVC 2010上编译一下看看:

\begin{lstlisting}
cl 1.cpp /Fa1.asm
\end{lstlisting}

(\TT{/Fa}参数通知编译器生成汇编语句列表)

\begin{lstlisting}[caption=MSVC 2010,style=customasmx86]
CONST	SEGMENT
$SG3830	DB	'hello, world', 0AH, 00H
CONST	ENDS
PUBLIC	_main
EXTRN	_printf:PROC
; Function compile flags: /Odtp
_TEXT	SEGMENT
_main	PROC
	push	ebp
	mov	ebp, esp
	push	OFFSET $SG3830
	call	_printf
	add	esp, 4
	xor	eax, eax
	pop	ebp
	ret	0
_main	ENDP
_TEXT	ENDS
\end{lstlisting}

MSVC生成Intel语法的汇编语句列表。
Intel语法和AT-T语法的区别将在\myref{ATT_syntax}讨论。

编译器生成名为\TT{1.obj}的文件,该文件随后被链接成\TT{1.exe}。
在我这边,生成的文件包含两部分: \TT{CONST}段(用于存放常量)和\TT{\_TEXT}段(用于存放代码)。

\myindex{\CLanguageElements!const}
\label{string_is_const_char}
在\CCpp中,字符串\TT{hello, world}具有\TT{const char[]}类型\InSqBrackets{\TCPPPL p176, 7.3.2},但这个字符串没有名称。
编译器为了对字符串进行操作,为其指定了一个内部名\TT{\$SG3830}。

所以这个样例的代码可以被重写为如下形式:

\lstinputlisting[style=customc]{patterns/01_helloworld/hw_2.c}

现在我们继续看汇编代码。如我们所见,字符串由一个字节'0'结尾,这是\CCpp中字符串标准规定的。
关于\CCpp字符串的更多细节参考这一节: \myref{C_strings}。

在代码段\TT{\_TEXT}中,目前只有一个函数\main{}。

像大部分函数一样,\main函数以prologue代码开始,以epilogue代码结束(没有找到比较合适的词翻译)。
\footnote{在这一节你可以阅读到更多关于prologue和epilogue的内容~(\myref{sec:prologepilog})。}

\myindex{x86!\Instructions!CALL}
在prologue代码之后,我们能看到对\printf{}函数的调用:\\
\INS{CALL \_printf}.
\myindex{x86!\Instructions!PUSH}
在调用函数前,字符串(或指向字符串的指针)的地址通过\PUSH指令被传递到栈上。

当\printf函数将控制权返回给\main函数的时候,字符串(或指向字符串的指针)的地址仍留存在栈上。
然而我们已经不再需要它,\gls{栈指针}(\ESP寄存器)需要被修正。

\myindex{x86!\Instructions!ADD}
\INS{ADD ESP, 4}表示将\ESP寄存器中的值加4。

为什么是4呢?因为这是一个32位程序,我们通过栈来传递地址时需要4个字节,如果它是64位程序,我们则需要8个字节。
\INS{ADD ESP, 4}是\INS{POP register}的有效替代品,并且不需要任何额外的寄存器。\footnote{然而CPU的标志位被修改了}

\myindex{Intel C++}
\myindex{\oracle}
\myindex{x86!\Instructions!POP}

出于同样的原因,一些编译器(例如Intel C++编译器)会生成\TT{POP ECX}
而不是\ADD(例如,在\oracle{}的代码中可以观察到这个现象,而它正是由Intel C++编译器所编译的)。
这两个指令的效果几乎完全相同,区别只在于\ECX寄存器的值是否会被覆盖掉。
Intel C++编译器使用\TT{POP ECX}是因为它的机器码比\TT{ADD ESP, x}少(前者占用1字节,后者占用3字节)。

下面就是\oracle{}中一个使用\POP代替\ADD的例子:

\begin{lstlisting}[caption=\oracle 10.2 Linux (app.o file),style=customasmx86]
.text:0800029A                 push    ebx
.text:0800029B                 call    qksfroChild
.text:080002A0                 pop     ecx
\end{lstlisting}

%Read more about the stack in section
% ~(\myref{sec:stack}).
\myindex{\CLanguageElements!return}
在调用\printf后,原始的\CCpp代码包含了语句\TT{return 0}~---将0作为\main函数的返回值。

\myindex{x86!\Instructions!XOR}
而在生成的代码中,这一语句通过\INS{XOR EAX, EAX}指令实现。

\myindex{x86!\Instructions!MOV}

\XOR即\q{eXclusive OR},\footnote{\href{http://go.yurichev.com/17118}{wikipedia}}但是编译器通常使用它来代替
\INS{MOV EAX, 0}---原因还是它的机器码更少(\XOR占用2字节而\MOV占用5字节)。

\myindex{x86!\Instructions!SUB}
一些编译器会生成\INS{SUB EAX, EAX},即从\EAX中的值减去\EAX的值,意即,将EAX清零。

\myindex{x86!\Instructions!RET}
最后一条指令\RET将控制权返回给\gls{调用者}。通常情况下,这是\CCpp的\ac{CRT}代码,即会将控制权返回给\ac{OS}。

