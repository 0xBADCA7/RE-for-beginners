\subsubsection{GCC}

Als nächstes wird der gleiche \CCpp-Code mit GCC 4.4.1 unter Linux kompiliert: \TT{gcc 1.c -o 1}.
Mithilfe des \IDA-Disassemblers wird untersucht, wie die \main-Funktion erzeugt wurde.
\IDA nutzt, genau wie MSVX den Intel-Syntax\footnote{GCC kann Assembler-Ausgaben im Intel-Syntax erzeugen mit der Options \TT{-S -masm=intel}.}.

\begin{lstlisting}[caption=Code in \IDA,style=customasmx86]
main            proc near

var_10          = dword ptr -10h

                push    ebp
                mov     ebp, esp
                and     esp, 0FFFFFFF0h
                sub     esp, 10h
                mov     eax, offset aHelloWorld ; "hello, world\n"
                mov     [esp+10h+var_10], eax
                call    _printf
                mov     eax, 0
                leave
                retn
main            endp
\end{lstlisting}

\myindex{Function prologue}
\myindex{x86!\Instructions!AND}
Das Ergebnis ist fast das gleiche.
Die Adresse der \TT{hello, world}-Zeichenkette (im Daten-Segment) wird zunächst in das \EAX-Register geladen und anschließend auf dem Stack gesichert.\\
Zusätzlich beinhaltet der Funktions-Prolog \INS{AND ESP, 0FFFFFFF0h}~---diese
Anweisung richtet den \ESP-Register-Wert an eine 16-Byte-Grenze aus.
Dies führt dazu, dass alle Werte im Stack auf die gleiche Weise ausgerichtet sind.
Die CPU kann Anweisungen schneller ausführen, wenn die zu verarbeitenden Daten auf einer an 4- oder 16-Byte-Grenzen ausgerichteten Adresse liegen\footnote{\URLWPDA}.

\myindex{x86!\Instructions!SUB}
\INS{SUB ESP, 10h} reserviert 16 Byte auf dem Stack, auch wenn - wie später gezeigt wird - nur 4 Byte benötigt werden.

Der Grund liegt darin, dass auch die Größe des Stacks an eine 16-Byte-Grenze ausgerichtet ist.

% TODO1: rewrite.
\myindex{x86!\Instructions!PUSH}
Die Adresse der Zeichenkette (oder ein Zeiger darauf) wird anschließend direkt ohne die \PUSH-Anweisung auf dem Stack gespeichert.
IT{var\_10}~---ist eine lokale Variable und ein Argument für \printf{}.
Mehr dazu später.

Anschließend wird die \printf-Funktion aufgerufen.

Anders als MSVC erzeugt GCC ohne Optimierung Die Anweisung \TT{MOV EAX, 0} anstatt des kürzeren OpCodes.

\myindex{x86!\Instructions!LEAVE}
Die letzte Anweisung \LEAVE ist ein Äquivalent zu der Kombination aus \TT{MOV ESP, EBP} und \TT{POP EBP}.
Mit anderen Worten: diese Anweisung setzt den \gls{stack pointer} (\ESP) zurück und stellt die initalen Werte des \EBP-Registers wieder her.
Dies ist notwendig weil die Registerwerte (\ESP und \EBP) zu Beginn der Funktion (durch \INS{MOV EBP, ESP} / \INS{AND ESP, \ldots}).

\subsubsection{GCC: \ATTSyntax}
\label{ATT_syntax}

Im nächsten Beispiel ist sichtbar, wie dies im AT\%T-Syntax dargestellt werden kann.
Dieser Syntax ist sehr viel populärer in der UNIX-Welt.

\begin{lstlisting}[caption=Das Beispiel kompiliert mit GCC 4.7.3]
gcc -S 1_1.c
\end{lstlisting}

Das Ergebnis ist wie folgt:

\lstinputlisting[caption=GCC 4.7.3,style=customasmx86]{patterns/01_helloworld/GCC.s}

Der Quellcode beinhaltet Makros (beginnend mit einem Punkt), die hier aber nicht von Belang sind.

An dieser Stelle werden aus Gründen der Übersichtlichkeit alle Makros au0er \IT{.string}
ignoriert. Letzeres kodiert eine Null-terminierte Zeichenkette, die einem C-String entspricht.

Die resultierende Ausgabe ist diese
\footnote{Um die \q{unnötigen} Makros zu unterdrücen kann die GCC-Option \IT{-fno-asynchronous-unwind-tables} genutzt werden}:

\lstinputlisting[caption=GCC 4.7.3,style=customasmx86]{patterns/01_helloworld/GCC_refined.s}

\myindex{\ATTSyntax}
\myindex{\IntelSyntax}
Einige der Hauptunterschiede zwischen Intel und AT\&T-Syntax sin:

\begin{itemize}

\item
Quell- und Zieloperanden sind in umgekehrter Reihenfolge angegeben.

Im Intel-Syntax: <Anweisung> <Ziel-Operand> <Quell-Operand>.

Im AT\&T-Syntax: <Anweisung> <Quell-Operand> <Ziel-Operand>.

\myindex{\CStandardLibrary!memcpy()}
\myindex{\CStandardLibrary!strcpy()}
Hier ist eine einfache Möglichkeit um sich den Unterschied zu merken:
Beim Umgang mit dem Intel-Syntax, kann man sich ein Gleichheitszeichen ($=$) zwischen den Operanden vorstellen
und beim AT\&T-Syntax einen Pfeil nach rechts ($\rightarrow$)
\footnote{Einige C-Standard-Funktionen (z.B. memcpy(), strcpy()) sind die Parameter ebenfalls wie im
Intel-Syntax aufgelistet: erst der Zeiger zum Ziel, dann der Zeiger auf die Speicher-Quelle)}.

\item
AT\&T: Vor einem Register-Namen muss ein Prozentzeichen (\%) und vor Zahlen ein Dollarzeichen (\$) stehen.
Statt eckigen werden runde Klammern genutzt.

\item
AT\&T: An eine Anweisung ist ein Suffix angehängt, der die Operandengröße angibt:

\begin{itemize}
\item q --- quad (64 bits)
\item l --- long (32 bits)
\item w --- word (16 bits)
\item b --- byte (8 bits)
\end{itemize}

% TODO1 simple example may be? \RU{Например mov\textbf{l}, movb, movw представляют различые версии инсструкция mov} \EN {For example: movl, movb, movw are variations of the mov instruciton} \DE {Zum Beispiel sind movl, movb und movw Variationen der mov-Anweisung}

\end{itemize}

Nochmals zu dem kompilierten Ergebnis: Dieses ist identisch mit der Anzeige in \IDA,
jedoch mit einem kleinen Unterschied: \TT{0FFFFFFF0h} wird als \TT{\$-16} angezeigt.
Der eigentliche Wert ist der selbe: \TT{16} im Dezimalsystem ist \TT{0x10} im Hexadezimalsystem.
Für 32-Bit-Datentypen ist \TT{-0x10} identisch mit \TT{0xFFFFFFF0}.

\myindex{x86!\Instructions!MOV}
Eine weitere Sache: der Rückgabewert ist mittels \MOV auf Null gesetzt, nicht mit \XOR.
\MOV läd lediglich einen Wert in ein Register.
Der Name ist irreführend, da die Daten nicht verschoben, sondern kopiert werden.
In anderen Architekturen ist wird dieser Befehl \q{LOAD} oder \q{STORE} oder ähnlich genannt.
