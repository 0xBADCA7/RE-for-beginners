\section{Metodyka}

Kiedy autor tej książki uczył się C, a później \Cpp, on tylko pisał niewielkie kawałki kodu, kompilował i patrzył jak to wygląda w asemblerze. Tak było o wiele łatwiej zrozumieć co się dzieje w programie.%
\footnote{Szczerze mówiąc, on ddalej tak robi, kiedy nie rozumie jak jakiś kod działa.}.
On robił to wystarczająco dużą ilość razy, żeby związek między kodem w \CCpp a tym co generuje kompilator wbił się w jego podświadomość bardzo głęboko. Po tym, patrząc na kod w asemblerze od razu ogólnikowo się rozumiało to, co było napisane w C. Możliwe, że to pomoże komuś jeszcze.

%Tu jest dużo przykładów do x86/x64 zarówno jak i do ARM.
%Ci którzy są wystarczająco dobrze obeznani z jedną z architektur, mogą pomijać te fragmenty.

Czasami tu się będzie korzystało ze starych kompilatorów, żeby otrzymać bardzo krótki lub prosty kawałek kodu.

A proppos, jest jedna dobra strona, gdzie można robić wszystko to samo z różnymi kompilatorami bez konieczności installowania ich u siebie na maszynie: \url{http://gcc.beta.godbolt.org/}.

\section*{\Exercises}

Kiedy autor tej książki uczył się asemblera, on również często kompilował krótkie funkcje w C i powoli przepisywał je na asembler, mając na celu otzrymać jak najkrótszy kod. Możliwe, że nie jest to obecnie niezbędne jako że współczesne kompilatory optymalizują kod najbardziej jak się da, ale to jest bardzo dobry sposób, żeby oswoić się z asemblerem . Także możecie wziąć dowolny fragment kodu w asemblerze z tej książki i postarać się zrobić go któtszym. Ale warto nie zapominać o testowaniu wyników.

\section*{Poziomy optymalizajci i debuggowanie}

Kod źródłowy można kompilować różnymi kompilatorami z różnym poziomami optymalizacji. W typowym kompilatorze jest tych poziomów około trzech, gdzi poziom zerowy - wyłączona optymalizacja. Rozróżnia się także kierunki optymalizacji według rozmiaru i poziomu prędkości. Niezoptymalizujący kompilator dzuała szybciej i generuje bardziej przejrzysty kod (choć i większy objętościowo). Optymalizujący kompilator działa wolniej i stara się wygenerować jak najszybszy kod (co nie zawsze znaczy, że kod będzi krótszy). Obok poziomów i kierunków optymalizacji kompilator może załączać do pliku wyniki debuggowania, tworząc w ten sposób kod, który jest prostszy w debuggowaniu. Bardzo ważną cechą kodu debuggowanego jest to, że on zawiera związki między każdą linią kodu źródłowego i adresem w kodzie maszynowym. Optymalizujące kompilatory zwykle generują kod, gdzie całe linie kodu źródłowego mogą zostać pominięte. Praktykujący reverse engineer z reguły ma styczność z obiema wersjami, jako że niektórzy developerzy włączają optymalizację, a niektórzy - nie. Dlatego postaramy się ćwiczyć rozważając obie możliwości.

