\subsection{x86}

\subsubsection{MSVC}

コンパイルして得られるものを次に示します(MSVC 2010 Express)。

\lstinputlisting[label=src:passing_arguments_ex_MSVC_cdecl,caption=MSVC 2010 Express,style=customasmx86]{patterns/05_passing_arguments/msvc_JPN.asm}

\myindex{x86!\Registers!EBP}

\main 関数は3つの数値をスタックにプッシュし、\TT{f(int,int,int)}を呼び出すことがわかります。

\ttf 内の引数アクセスは、ローカル変数と同じ方法で
\TT{\_a\$ = 8}
のようなマクロの助けを借りて構成されますが、正のオフセット(\IT{プラス}で扱われます)を持ちます。 
したがって、\TT{\_a\$}マクロを \EBP レジスタの値に追加することによって\gls{stack frame}の\IT{外側}を処理しています。

\myindex{x86!\Instructions!IMUL}
\myindex{x86!\Instructions!ADD}

次に、 $a$ の値が \EAX に格納されます。 \IMUL 命令実行後、 
\EAX の値は \EAX の値と\TT{\_b}の内容の\gls{product}です。

その後、 \ADD は\TT{\_c}の値を \EAX に追加します。

\EAX の値は移動する必要はありません。すでに存在している必要があります。 
\gls{caller}に戻ると、 \EAX 値をとり、 \printf の引数として使用します。

\subsubsection{MSVC + \olly}
\myindex{\olly}
これを \olly で説明しましょう。 
最初の引数(最初の引数)を使用する \ttf の最初の命令をトレースすると、
\EBP が赤い四角でマークされた\gls{stack frame}を指していることがわかります。

\gls{stack frame}の最初の要素は \EBP のセーブされた値であり、
2番目の要素は\ac{RA}であり、3番目の要素は最初の関数の引数であり、2番目と3番目の要素です。

最初の関数引数にアクセスするには、 \EBP にちょうど8(2つの32ビットワード)を追加する必要があります。

\olly はこれを知っているので、 

\q{RETURN from}や \q{Arg1 = \dots}などのスタック要素にコメントを追加しました。

注意:関数の引数は、関数のスタックフレームのメンバーではなく、むしろ\gls{caller}関数のスタックフレームのメンバーです。

したがって、 \olly は別のスタックフレームのメンバーとして \q{Arg}要素をマークしました。

\begin{figure}[H]
\centering
\myincludegraphics{patterns/05_passing_arguments/olly.png}
\caption{\olly: inside of \ttf{} function}
\label{fig:passing_arguments_olly}
\end{figure}



\subsubsection{GCC}

GCC 4.4.1で同じものをコンパイルし、 \IDA の結果を見てみましょう。

\lstinputlisting[caption=GCC 4.4.1,style=customasmx86]{patterns/05_passing_arguments/gcc_JPN.asm}

結果はほぼ同じで、以前に説明したいくつかの小さな違いがあります。

\gls{stack pointer}は2つの関数呼び出し(fとprintf)の後にセットバックされません。
最後から2番目の\TT{LEAVE}命令(\myref{x86_ins:LEAVE})
命令が最後にこれを処理するためです。
