\subsubsection{ARM64}

\myparagraph{\Optimizing GCC (Linaro) 4.9}

\myindex{Fused multiply–add}
\myindex{ARM!\Instructions!MADD}
ここでのすべては簡単です。 
\TT{MADD}は乗算/加算を一緒に行う命令です(既に見た\TT{MLA}に似ています)。 
3つの引数はすべて、Xレジスタの32ビット部分に渡されます。 
実際、引数の型は32ビット\IT{int}です。 結果は\TT{W0}に返されます。

\lstinputlisting[caption=\Optimizing GCC (Linaro) 4.9,style=customasmARM]{patterns/05_passing_arguments/ARM/ARM64_O3_JPN.s}

すべてのデータ型を64ビット\TT{uint64\_t}に拡張してテストしましょう:

\lstinputlisting[style=customc]{patterns/05_passing_arguments/ex64.c}

\begin{lstlisting}[style=customasmARM]
f:
	madd	x0, x0, x1, x2
	ret
main:
	mov	x1, 13396
	adrp	x0, .LC8
	stp	x29, x30, [sp, -16]!
	movk	x1, 0x27d0, lsl 16
	add	x0, x0, :lo12:.LC8
	movk	x1, 0x122, lsl 32
	add	x29, sp, 0
	movk	x1, 0x58be, lsl 48
	bl	printf
	mov	w0, 0
	ldp	x29, x30, [sp], 16
	ret

.LC8:
	.string	"%lld\n"
\end{lstlisting}

\ttf 関数は同じで、64ビットのXレジスタ全体が使用されます。 
長い64ビットの値は、レジスタごとにロードされます。これについては、以下で説明します:\myref{ARM_big_constants_loading}

\myparagraph{\NonOptimizing GCC (Linaro) 4.9}

非最適化コンパイラはより冗長です。

\begin{lstlisting}[style=customasmARM]
f:
	sub	sp, sp, #16
	str	w0, [sp,12]
	str	w1, [sp,8]
	str	w2, [sp,4]
	ldr	w1, [sp,12]
	ldr	w0, [sp,8]
	mul	w1, w1, w0
	ldr	w0, [sp,4]
	add	w0, w1, w0
	add	sp, sp, 16
	ret
\end{lstlisting}

コードは、この関数の誰か(または何か)が
\TT{W0...W2}レジスタを使用する必要がある場合に、その入力引数をローカルスタックに保存します。 
これにより、元の関数の引数を上書きすることが防止されます。
これは、将来必要になる可能性があります。

これは、\IT{レジスタセーブエリア}と呼ばれます。((\ARMPCS))
しかし、呼び出し先関数はそれらを保存する義務はありません。 
これは、\q{シャドースペース}(\myref{shadow_space})に多少似ています。

GCC 4.9を最適化すると、なぜこの引数はコードを保存しなくなったのでしょうか? 
これはいくつかの追加の最適化作業を行い、
関数の引数がこの先に必要ではなく、
レジスタ\TT{W0...W2}が使用されないと結論付けたためです。

\myindex{ARM!\Instructions!MUL}
\myindex{ARM!\Instructions!ADD}

また、単一の\TT{MADD}の代わりに\TT{MUL}/\TT{ADD}命令ペアがあります。
