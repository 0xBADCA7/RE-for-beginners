\subsubsection{\NonOptimizingKeilVI (\ARMMode)}

\begin{lstlisting}[style=customasmARM]
.text:000000A4 00 30 A0 E1      MOV     R3, R0
.text:000000A8 93 21 20 E0      MLA     R0, R3, R1, R2
.text:000000AC 1E FF 2F E1      BX      LR
...
.text:000000B0             main
.text:000000B0 10 40 2D E9      STMFD   SP!, {R4,LR}
.text:000000B4 03 20 A0 E3      MOV     R2, #3
.text:000000B8 02 10 A0 E3      MOV     R1, #2
.text:000000BC 01 00 A0 E3      MOV     R0, #1
.text:000000C0 F7 FF FF EB      BL      f
.text:000000C4 00 40 A0 E1      MOV     R4, R0
.text:000000C8 04 10 A0 E1      MOV     R1, R4
.text:000000CC 5A 0F 8F E2      ADR     R0, aD_0        ; "%d\n"
.text:000000D0 E3 18 00 EB      BL      __2printf
.text:000000D4 00 00 A0 E3      MOV     R0, #0
.text:000000D8 10 80 BD E8      LDMFD   SP!, {R4,PC}
\end{lstlisting}

\main 関数は他の2つの関数を呼び出し、3つの値が最初の関数に渡されます(\ttf)。

前述のように、ARMでは最初の4つの値が通常最初の4つのレジスタ(\Reg{0}-\Reg{3})に渡されます。

\ttf 関数は、最初の3つのレジスタ(\Reg{0}-\Reg{2})を引数として使用します。

\myindex{ARM!\Instructions!MLA}
\TT{MLA}(\IT{Multiply Accumulate})
命令は最初の2つのオペランド(\Reg{3}と\Reg{1})を乗算し、3番目のオペランド(\Reg{2})を積に加算し、
その結果をゼロ関数(\Reg{0})に格納します。

\myindex{Fused multiply–add}
一度に乗算と加算を同時に行うの(\IT{Fused multiply?add})は非常に便利な操作です。
ところで、SIMD\footnote{\href{http://go.yurichev.com/17103}{wikipedia}}
にFMA命令が登場する前に、x86にそのような命令はありませんでした。

最初の\TT{MOV R3, R0}命令は
明らかに冗長です(ここでは単一の\TT{MLA}命令を代わりに使用できます)。
これは最適化されないコンパイルであるため、コンパイラは最適化していません。

\myindex{ARM!Mode switching}
\myindex{ARM!\Instructions!BX}

\TT{BX}命令は、制御を \ac{LR}レジスタに格納されているアドレスに戻し、
必要に応じてプロセッサモードをThumbからARMに、またはその逆に切り替えます。
これは、関数 \ttf がどのような種類のコード(ARMまたはThumb)から認識されていないため、必要な場合があります。
したがって、Thumbコードから呼び出された場合、
\TT{BX}は呼び出し元の関数に制御を戻すだけでなく、
プロセッサーモードをThumbに切り替えます。 
ARMコード(\InSqBrackets{\ARMSevenRef A2.3.2})から関数が呼び出されているかどうかを切り替えます。
% look for "BXWritePC()" in manual
