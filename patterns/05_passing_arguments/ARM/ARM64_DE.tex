\subsubsection{ARM64}

\myparagraph{\Optimizing GCC (Linaro) 4.9}

\myindex{Fused multiply–add}
\myindex{ARM!\Instructions!MADD}

Hier ist alles ganz einfach. 
\TT{MADD} ist einfach eine Instruktion die Multiplikation/Addition verschmelzt ( ähnlich wie wir es bei \TT{MLA}
gesehen haben)

Alle drei Argumente werden über den 32-Bit Part des X-Registers übergeben.
In der tat, die Argumente sind 32-Bit \IT{int}'s.
Das Ergebnis wird in \TT{W0} gespeichert.

\lstinputlisting[caption=\Optimizing GCC (Linaro) 4.9]{patterns/05_passing_arguments/ARM/ARM64_O3_EN.s}

Lasst uns nun alle Datentypen nach 64-Bit \TT{uint64\_t} konvertieren und testen:

\lstinputlisting{patterns/05_passing_arguments/ex64.c}

\begin{lstlisting}
f:
	madd	x0, x0, x1, x2
	ret
main:
	mov	x1, 13396
	adrp	x0, .LC8
	stp	x29, x30, [sp, -16]!
	movk	x1, 0x27d0, lsl 16
	add	x0, x0, :lo12:.LC8
	movk	x1, 0x122, lsl 32
	add	x29, sp, 0
	movk	x1, 0x58be, lsl 48
	bl	printf
	mov	w0, 0
	ldp	x29, x30, [sp], 16
	ret

.LC8:
	.string	"%lld\n"
\end{lstlisting}

Die \ttf{} Funktion ist die gleiche, nur das jetzt alle 64-Bit X-Register benutzt werden.
Lange 64-Bit Werte werden Stückweise in die Register geladen, genauer beschrieben hier: \myref{ARM_big_constants_loading}.

\myparagraph{\NonOptimizing GCC (Linaro) 4.9}

Der nicht optimierte Compiler lauf ist redundanter:

\begin{lstlisting}
f:
	sub	sp, sp, #16
	str	w0, [sp,12]
	str	w1, [sp,8]
	str	w2, [sp,4]
	ldr	w1, [sp,12]
	ldr	w0, [sp,8]
	mul	w1, w1, w0
	ldr	w0, [sp,4]
	add	w0, w1, w0
	add	sp, sp, 16
	ret
\end{lstlisting}

Der Code speichert seine Eingabe Argumente auf dem lokalen Stack,
für den Fall das die Funktion die \TT{W0..W2} Register benutzen muss
das verhindert das überschreiben der Original Argumente, die vielleicht
noch in Zukunft gebraucht werden.

Das bezeichnet man auch als \IT{Register Save Area.} (\ARMPCS).
Der Callee, ist hier nicht in der Pflicht die Werte zu speichern.
So Ähnlich wie beim \q{Shadow Space}: \myref{shadow_space}.


Warum hat der optimierte GCC 4.9 Aufruf dieses Argument weg gelassen?
Weil der Compiler in dem Fall zusätzliche Optimierungen gemacht hat. Und
erkannt hat das die zusätzlichen Argumente in der weiteren Ausführung des 
Codes nicht mehr benötigt werden. Und auch das die Register \TT{W0..W2} auch 
nicht weiter benötigt werden.

\myindex{ARM!\Instructions!MUL}
\myindex{ARM!\Instructions!ADD}

Wir können auch ein \TT{MUL}/\TT{ADD} Instruktions paar sehen anstatt einem einzelnen
\TT{MADD}.

