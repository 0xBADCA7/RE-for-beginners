\subsubsection{ARM64}

\myparagraph{\Optimizing GCC (Linaro) 4.9}

\myindex{Fused multiply–add}
\myindex{ARM!\Instructions!MADD}
Everything here is simple.
\TT{MADD} is just an instruction doing fused multiply/add (similar to the \TT{MLA} we already saw).
All 3 arguments are passed in the 32-bit parts of X-registers.
Indeed, the argument types are 32-bit \IT{int}'s.
The result is returned in \TT{W0}.

\lstinputlisting[caption=\Optimizing GCC (Linaro) 4.9,style=customasm]{patterns/05_passing_arguments/ARM/ARM64_O3_EN.s}

Let's also extend all data types to 64-bit \TT{uint64\_t} and test:

\lstinputlisting[style=customc]{patterns/05_passing_arguments/ex64.c}

\begin{lstlisting}
f:
	madd	x0, x0, x1, x2
	ret
main:
	mov	x1, 13396
	adrp	x0, .LC8
	stp	x29, x30, [sp, -16]!
	movk	x1, 0x27d0, lsl 16
	add	x0, x0, :lo12:.LC8
	movk	x1, 0x122, lsl 32
	add	x29, sp, 0
	movk	x1, 0x58be, lsl 48
	bl	printf
	mov	w0, 0
	ldp	x29, x30, [sp], 16
	ret

.LC8:
	.string	"%lld\n"
\end{lstlisting}

The \ttf{} function is the same, only the whole 64-bit X-registers are now used.
Long 64-bit values are loaded into the registers by parts, this is also described here: \myref{ARM_big_constants_loading}.

\myparagraph{\NonOptimizing GCC (Linaro) 4.9}

The non-optimizing compiler is more redundant:

\begin{lstlisting}
f:
	sub	sp, sp, #16
	str	w0, [sp,12]
	str	w1, [sp,8]
	str	w2, [sp,4]
	ldr	w1, [sp,12]
	ldr	w0, [sp,8]
	mul	w1, w1, w0
	ldr	w0, [sp,4]
	add	w0, w1, w0
	add	sp, sp, 16
	ret
\end{lstlisting}

The code saves its input arguments in the local stack, 
in case someone (or something) in this function needs using the \TT{W0...W2} 
registers. This prevents overwriting the original
function arguments, which may be needed again in the future.

This is called \IT{Register Save Area.} (\ARMPCS).
The callee, however, is not obliged to save them.
This is somewhat similar to \q{Shadow Space}: \myref{shadow_space}.

Why did the optimizing GCC 4.9 drop this argument saving code?
Because it did some additional optimizing work and concluded
that the function arguments will not be needed in the future 
and also that the registers \TT{W0...W2} will not be used.

\myindex{ARM!\Instructions!MUL}
\myindex{ARM!\Instructions!ADD}

We also see a \TT{MUL}/\TT{ADD} instruction pair instead of single a \TT{MADD}.
