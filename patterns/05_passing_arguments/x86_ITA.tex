\subsection{x86}

\subsubsection{MSVC}

Ecco il risultato della compilazione ocn MSVC 2010 Express:

\lstinputlisting[label=src:passing_arguments_ex_MSVC_cdecl,caption=MSVC 2010 Express,style=customasm]{patterns/05_passing_arguments/msvc_EN.asm}

\myindex{x86!\Registers!EBP}

Vediamo che la funzione \main fa il push di 3 numeri sullo stack e chiama \TT{f(int,int,int).} 

L'accesso agli argomenti all'interno della funzione \ttf e' gestito con l'aiuto di macro come: \TT{\_a\$ = 8}, 
allo stesso modo delle variabili locali, ma con offset positivi.
Si sta quindi indirizzando il lato \IT{esterno} dello \gls{stack frame} sommando la macro \TT{\_a\$} al valore contenuto nel registro \EBP.

\myindex{x86!\Instructions!IMUL}
\myindex{x86!\Instructions!ADD}

Successivamente il valore di $a$ e' memorizzato in \EAX. A seguito dell'esecuzione dell'istruzione \IMUL, il valore in \EAX e' 
il \gls{prodotto} del valore in \EAX e del contenuto di \TT{\_b}.

Infine, \ADD aggiunge il valore in \TT{\_c} a \EAX.

Il valore \EAX non necessita di essere spostato: si trova gia' nel posto giusto.
Al termine, la funzione chiamante (\gls{caller}) prende il valore di \EAX e lo usa come argomento di \printf.

\clearpage
\mysubparagraph{\olly}
\myindex{\olly}

Esaminiamo questo esempio con \olly.
Il valore di input della funzione (2) viene caricato \EAX: 

\begin{figure}[H]
\centering
\myincludegraphics{patterns/08_switch/2_lot/olly1.png}
\caption{\olly: il valore di input è caricato in \EAX}
\label{fig:switch_lot_olly1}
\end{figure}

\clearpage
Il valore viene controllato, è maggiore di 4?
Se no, il \q{default} jump non viene innescato:
\begin{figure}[H]
\centering
\myincludegraphics{patterns/08_switch/2_lot/olly2.png}
\caption{\olly: 2 non è maggiore di 4: il salto non viene fatto}
\label{fig:switch_lot_olly2}
\end{figure}

\clearpage
Qui vediamo un jumptable:

\begin{figure}[H]
\centering
\myincludegraphics{patterns/08_switch/2_lot/olly3.png}
\caption{\olly: calcolo dell'indirizzo di destinazione mediante jumptable}
\label{fig:switch_lot_olly3}
\end{figure}

Qui abbiamo cliccato \q{Follow in Dump} $\rightarrow$ \q{Address constant}, così da vedere la \IT{jumptable} nella data window.
Sono 5 valori a 32-bit \footnote{Sono sottolineati da \olly poiché
sono anche FIXUPs: \myref{subsec:relocs}, torneremo su questo argomento più avanti}.
\ECX adesso è 2, quindi il terzo elemento (avente indice 2\footnote{Per l'indicizzazione, vedi anche: \ref{arrays_at_one}}) della tabella.
E' anche possibile cliccare su \q{Follow in Dump} $\rightarrow$ 
\q{Memory address} e \olly mostrerà l'elemento a cui punta l'istruzione \JMP. 
In questo caso è \TT{0x010B103A}.

\clearpage
Dopo il salto ci troviamo a \TT{0x010B103A}: il codice che stampa \q{two} sarà ora eseguito:

\begin{figure}[H]
\centering
\myincludegraphics{patterns/08_switch/2_lot/olly4.png}
\caption{\olly: ora ci troviamo alla label \IT{case:}}
\label{fig:switch_lot_olly4}
\end{figure}


\subsubsection{GCC}

Compiliamo lo stesso esempio con GCC 4.4.1 ed osserviamo il risultato con \IDA:

\lstinputlisting[caption=GCC 4.4.1,style=customasm]{patterns/05_passing_arguments/gcc_EN.asm}

Il risultato e' pressoche' identico, a meno di piccole differenze gia' discusse in precedenza.

Lo \gls{stack pointer} non viene ripristinato dopo le due chiamate a funzione(f and printf), 
poiche' se ne occupa la penultima istruzione \TT{LEAVE} (\myref{x86_ins:LEAVE}) alla fine della funizone.
