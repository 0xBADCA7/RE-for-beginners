\subsection{x64}

\myindex{x86-64}

この話はx86-64では少し違っています。関数の引数(最初の4つまたは最初の6つ)
はレジスタに渡されます。つまり、\gls{callee}はレジスタからレジスタを読み込みます。

\subsubsection{MSVC}

\Optimizing MSVC:

\lstinputlisting[caption=\Optimizing MSVC 2012 x64,style=customasmx86]{patterns/05_passing_arguments/x64_MSVC_Ox_JPN.asm}

見てわかるように、コンパクトな関数 \ttf はすべての引数をレジスタから取ります。

ここでの \LEA 命令は加算に使用され、
明らかにコンパイラは \TT{ADD} よりも速いと考えました。
\myindex{x86!\Instructions!LEA}

\LEA は、第1および第3の \ttf 引数を準備するために \main 関数でも使用されます。
コンパイラは、 \MOV 命令を使用してレジスタに値をロードする通常の方法よりも速く動作すると判断する必要があります。

非最適化MSVCの出力を見てみましょう。

\lstinputlisting[caption=MSVC 2012 x64,style=customasmx86]{patterns/05_passing_arguments/x64_MSVC_IDA_JPN.asm}

レジスタからの3つの引数は何らかの理由でスタックに保存されるため、ややこしいことになっています。

\myindex{Shadow space}
\label{shadow_space}
これは "シャドウスペース"と呼ばれます。
\footnote{\href{http://go.yurichev.com/17256}{MSDN}}
すべてのWin64は、そこにある4つのレジスタ値をすべて保存することができます(必須ではありません)。
これは2つの理由で行われます。
1)入力引数にレジスタ全体(または4つのレジスタ)を割り当てるのはあまりにも贅沢なので、スタック経由でアクセスされます。 
2)デバッガはブレークで関数の引数をどこに見つけるか常に認識しています。
\footnote{\href{http://go.yurichev.com/17257}{MSDN}}

だから、大規模な関数の中には、実行中にそれらを使用したい場合、入力引数を \q{シャドウスペース}に保存することができますが、
私たちのような小さな関数ではそうでないかもしれません。

スタックに\q{シャドウスペース}を割り当てるのは\gls{caller}の責任です。

\subsubsection{GCC}

\Optimizing GCCはまあまあわかりやすいコードを生成します。

\lstinputlisting[caption=\Optimizing GCC 4.4.6 x64,style=customasmx86]{patterns/05_passing_arguments/x64_GCC_O3_JPN.s}

\NonOptimizing GCC:

\lstinputlisting[caption=GCC 4.4.6 x64,style=customasmx86]{patterns/05_passing_arguments/x64_GCC_JPN.s}

\myindex{Shadow space}

System V *NIX (\SysVABI)には\q{シャドースペース}の要件はありませんが、 \gls{callee}は
レジスタが不足している場合には引数をどこかに保存します。

\subsubsection{GCC: intの代わりのuint64\_t}

私たちの例は32ビットintで動作するため、32ビットのレジスタが使用されています(\TT{E-}が前に付いています)。 

64ビット値を使用するためには少し変更する必要があります。

\lstinputlisting[style=customc]{patterns/05_passing_arguments/ex64.c}

\lstinputlisting[caption=\Optimizing GCC 4.4.6 x64,style=customasmx86]{patterns/05_passing_arguments/ex64_GCC_O3_IDA_JPN.asm}

コードは同じですが、今回は\IT{フルサイズ}のレジスタ(\TT{R-}が前に付いています)が使用されています。
