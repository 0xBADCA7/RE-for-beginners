\subsection{MIPS}

\lstinputlisting[caption=\Optimizing GCC 4.4.5,style=customasm]{patterns/05_passing_arguments/MIPS_O3_IDA_EN.lst}

Die ersten vier Funktoins Argumente werden in vier Register übergeben die das A- Präfix haben.

\myindex{MIPS!\Instructions!MULT}

Es gibt zwei spezial Register in MIPS: HI und LO die das 64-Bit Multiplikation Ergebnis der
Ausführung der \TT{MULT} Instruktion enthalten.

\myindex{MIPS!\Instructions!MFLO}
\myindex{MIPS!\Instructions!MFHI}

Auf diese Register sind nur zugreifbar durch die \TT{MFLO} und die \TT{MFHI} Instruktionen.
\TT{MFLO} enthält hier die niedrigen Bits aus dem Multiplikations Ergebnis und speichert diese in \$V0.
Also wird der höhere Wert des 32-Bit Teils der multiplikation einfach verworfen ( der HI Register in halt 
wird nicht verwendet ) .
In der Tat: Wir operieren hier auf 32-Bit \Tint Daten Typen.

\myindex{MIPS!\Instructions!ADDU}

Zum Schluss addiert \TT{ADDU} (\q{Add Unsigned}) den Wert des dritten Argumentes zum Ergebnis.

\myindex{MIPS!\Instructions!ADD}
\myindex{MIPS!\Instructions!ADDU}
\myindex{Ada}
\myindex{Integer overflow}

Es gibt zwei unterschiedliche Addition Instruktionen auf der MIPS Plattform: \TT{ADD} und \TT{ADDU}.
Der unterschied zwischen den beiden Instruktionen bezieht sich nicht auf das Vorzeichen (+/-) sondern
auf die exceptions. \TT{ADD} kann eine exception werfen bei einem overflow, was manchmal nützlich\footnote{\url{http://go.yurichev.com/17326}} sein kann und wird auch bei Ada \ac{PL} unterstützt, zum Beispiel:

\TT{ADDU} wirft keine exception bei einem overflow.

Da \CCpp keine Unterstützung hierfür bietet, sehen wir in unserem Beispiel \TT{ADDU} statt \TT{ADD}.


Das 32-Bit Ergebnis bleibt übrig in \$V0.

\myindex{MIPS!\Instructions!JAL}
\myindex{MIPS!\Instructions!JALR}

In \main existiert nun eine neue Instruktion, die interessant für uns ist: \TT{JAL} {\q{Jump an Link}).

Der unterschied zwischen \INS{JAL} und \INS{JALR} ist das in der ersten Instruktion ein relatives offset
hart codiert ist, während \INS{JALR} zur absoluten Adresse gespeichert in einem Register springt (\q{Jump und Link Register}).

Beide \ttf und die  \main Funktionen liegen innerhalb der gleichen Objekt Datei, also ist die 
relative Adresse von \ttf bekannt und fix.

