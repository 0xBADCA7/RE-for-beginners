\subsection{x86}

\subsubsection{MSVC}

Voici ce que l'on obtient après compilation (MSVC 2010 Express) :

\lstinputlisting[label=src:passing_arguments_ex_MSVC_cdecl,caption=MSVC 2010 Express,style=customasmx86]{patterns/05_passing_arguments/msvc_FR.asm}

\myindex{x86!\Registers!EBP}

Ce que l'on voit, c'est que la fonction \main pousse 3 nombres sur la pile et appelle
\TT{f(int,int,int)}.

L'accès aux arguments à l'intérieur de \ttf est organisé à l'aide de macros
comme:\\
\TT{\_a\$ = 8},
de la même façon que pour les variables locales, mais avec des offsets positifs
(adressés avec \IT{plus}).
Donc, nous accèdons la partie \IT{hors} de la \glslink{stack frame}{structure locale de pile}
en ajoutant la macro \TT{\_a\$} à la valeur dans le registre \EBP.

\myindex{x86!\Instructions!IMUL}
\myindex{x86!\Instructions!ADD}

Ensuite, la valeur de $a$ est stockée dans \EAX. Après l'exécution de l'instruction
\IMUL, la valeur dans \EAX est le \glslink{product}{produit} de la valeur dans \EAX
et du contenu de \TT{\_b}.

Après cela, \ADD ajoute la valeur dans \TT{\_c} à \EAX.

La valeur dans \EAX n'a pas besoin d'être déplacée/copiée : elle est déjà là
oú elle doit être.
Lors du retour dans la fonction \glslink{caller}{appelante}, elle prend la valeur dans
\EAX et l'utilise comme argument pour \printf.

\subsubsection{MSVC + \olly}
\myindex{\olly}
Illustrons ceci dans \olly.
Lorsque nous traçons jusqu'à la première instruction de \ttf qui utilise un des
arguments (le premier), nous voyons qu'\EBP pointe sur la \glslink{stack frame}{structure de pile locale},
qui est entourée par un rectangle rouge. 

Le premier élément de la \glslink{stack frame}{structure de pile locale} est la
valeur sauvegardée de \EBP, le second est \ac{RA}, le troisième est le premier
argument de la fonction, puis le second et le troisième.

Pour accéder au premier argument de la fonction, on doit ajouter exactement 8 (2
mots de 32-bit) à \EBP.

\olly est au courant de cela, c'est pourquoi il a ajouté des commentaires aux éléments
de la pile comme

\q{RETURN from} et \q{Arg1 = \dots}, etc.

N.B.: Les arguments de la fonction ne font pas partie de la structure de pile de
la fonction, ils font plutôt partie de celle de la fonction \glslink{caller}{appelante}.

Par conséquent, \olly a marqué les éléments comme appartenant à une autre structure
de pile.

\begin{figure}[H]
\centering
\myincludegraphics{patterns/05_passing_arguments/olly.png}
\caption{\olly: à l'intérieur de la fonction \ttf{}}
\label{fig:passing_arguments_olly}
\end{figure}



\subsubsection{GCC}

Compilons le même code avec GCC 4.4.1 et regardons le résultat dans \IDA :

\lstinputlisting[caption=GCC 4.4.1,style=customasmx86]{patterns/05_passing_arguments/gcc_FR.asm}

Le résultat est presque le même, avec quelques différences mineures discutées
précédemment.

Le \glslink{stack pointer}{pointeur de pile} n'est pas remis après les deux appels
de fonctions (f and printf), car la pénultième instruction \TT{LEAVE} (\myref{x86_ins:LEAVE})
s'en occupe à la fin.
