\subsection{MIPS}

\lstinputlisting[caption=\Optimizing GCC 4.4.5,style=customasmMIPS]{patterns/05_passing_arguments/MIPS_O3_IDA_RU.lst}

Первые 4 аргумента функции передаются в четырех регистрах с префиксами A-.

\myindex{MIPS!\Instructions!MULT}
В MIPS есть два специальных регистра: HI и LO, которые выставляются в 64-битный результат умножения
во время исполнения инструкции \TT{MULT}.

\myindex{MIPS!\Instructions!MFLO}
\myindex{MIPS!\Instructions!MFHI}
К регистрам можно обращаться только используя инструкции \TT{MFLO} и \TT{MFHI}.
Здесь \TT{MFLO} берет младшую часть результата умножения и записывает в \$V0.
Так что старшая 32-битная часть результата игнорируется (содержимое регистра HI не используется).
Действительно, мы ведь работаем с 32-битным типом \Tint.


\myindex{MIPS!\Instructions!ADDU}
И наконец, \TT{ADDU} (\q{Add Unsigned}~--- добавить беззнаковое) прибавляет значение третьего аргумента к результату.

\myindex{MIPS!\Instructions!ADD}
\myindex{MIPS!\Instructions!ADDU}
\myindex{Ada}
\myindex{Integer overflow}
В MIPS есть две разных инструкции сложения: \TT{ADD} и \TT{ADDU}.
На самом деле, дело не в знаковых числах, а в исключениях: \TT{ADD} может вызвать исключение
во время переполнения. Это иногда полезно\footnote{\url{http://go.yurichev.com/17326}} и поддерживается,
например, в \ac{PL} Ada.

\TT{ADDU} не вызывает исключения во время переполнения.
А так как \CCpp не поддерживает всё это, мы видим здесь \TT{ADDU} вместо \TT{ADD}.

32-битный результат оставляется в \$V0.

\myindex{MIPS!\Instructions!JAL}
\myindex{MIPS!\Instructions!JALR}
В \main есть новая для нас инструкция: \TT{JAL} (\q{Jump and Link}). 
Разница между \INS{JAL} и \INS{JALR} в том, что относительное смещение кодируется в первой инструкции,
а \INS{JALR} переходит по абсолютному адресу, записанному в регистр (\q{Jump and Link Register}).

Обе функции \ttf и \main расположены в одном объектном файле, так что относительный адрес \ttf известен и фиксирован.

