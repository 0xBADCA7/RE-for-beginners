\subsection{x86}

\subsubsection{MSVC}

Рассмотрим пример, скомпилированный в (MSVC 2010 Express):

\lstinputlisting[label=src:passing_arguments_ex_MSVC_cdecl,caption=MSVC 2010 Express,style=customasmx86]{patterns/05_passing_arguments/msvc_RU.asm}

\myindex{x86!\Registers!EBP}
Итак, здесь видно: в функции \main заталкиваются три числа в стек и вызывается функция \TT{f(int,int,int)}.
 
Внутри \ttf доступ к аргументам, также как и к локальным переменным, происходит через макросы: 
\TT{\_a\$ = 8}, но разница в том, что эти смещения со знаком \IT{плюс}, 
таким образом если прибавить макрос \TT{\_a\$} к указателю на \EBP, то адресуется \IT{внешняя} 
часть \glslink{stack frame}{фрейма} стека относительно \EBP.

\myindex{x86!\Instructions!IMUL}
\myindex{x86!\Instructions!ADD}
Далее всё более-менее просто: значение $a$ помещается в \EAX. 
Далее \EAX умножается при помощи инструкции \IMUL на то, что лежит в \TT{\_b}, 
и в \EAX остается \glslink{product}{произведение} этих двух значений.

Далее к регистру \EAX прибавляется то, что лежит в \TT{\_c}.

Значение из \EAX никуда не нужно перекладывать, оно уже лежит где надо. 
Возвращаем управление вызываемой функции~--- она возьмет значение из \EAX и отправит его в \printf.

\clearpage
\mysubparagraph{\olly}
\myindex{\olly}

Попробуем этот пример в \olly.
Входное значение функции (2) загружается в \EAX: 

\begin{figure}[H]
\centering
\myincludegraphics{patterns/08_switch/2_lot/olly1.png}
\caption{\olly: входное значение функции загружено в \EAX}
\label{fig:switch_lot_olly1}
\end{figure}

\clearpage
Входное значение проверяется, не больше ли оно чем 4? 
Нет, переход по умолчанию (\q{default}) не будет исполнен:

\begin{figure}[H]
\centering
\myincludegraphics{patterns/08_switch/2_lot/olly2.png}
\caption{\olly: 2 не больше чем 4: переход не сработает}
\label{fig:switch_lot_olly2}
\end{figure}

\clearpage
Здесь мы видим jumptable:

\begin{figure}[H]
\centering
\myincludegraphics{patterns/08_switch/2_lot/olly3.png}
\caption{\olly: вычисляем адрес для перехода используя jumptable}
\label{fig:switch_lot_olly3}
\end{figure}

Кстати, щелкнем по \q{Follow in Dump} $\rightarrow$ \q{Address constant}, так что теперь \IT{jumptable} видна в окне данных.

Это 5 32-битных значений\footnote{Они подчеркнуты в \olly, потому что это также и FIXUP-ы: \myref{subsec:relocs}, мы вернемся к ним позже}.
\ECX сейчас содержит 2, так что третий элемент (либо второй, если считать с нулевого) таблицы будет использован.
Кстати, можно также щелкнуть \q{Follow in Dump} $\rightarrow$ \q{Memory address} и \olly покажет элемент, который сейчас адресуется в инструкции \JMP. 
Это \TT{0x010B103A}.

\clearpage
Переход сработал и мы теперь на \TT{0x010B103A}: сейчас будет исполнен код, выводящий строку \q{two}:

\begin{figure}[H]
\centering
\myincludegraphics{patterns/08_switch/2_lot/olly4.png}
\caption{\olly: теперь мы на соответствующей метке \IT{case:}}
\label{fig:switch_lot_olly4}
\end{figure}


\subsubsection{GCC}

Скомпилируем то же в GCC 4.4.1 и посмотрим результат в \IDA:

\lstinputlisting[caption=GCC 4.4.1,style=customasmx86]{patterns/05_passing_arguments/gcc_RU.asm}

Практически то же самое, если не считать мелких отличий описанных ранее.

После вызова обоих функций \glslink{stack pointer}{указатель стека} не возвращается назад, 
потому что предпоследняя инструкция \TT{LEAVE} (\myref{x86_ins:LEAVE}) делает это за один раз, в конце исполнения.

