\subsection{MIPS}

\lstinputlisting[caption=\Optimizing GCC 4.4.5,style=customasmMIPS]{patterns/05_passing_arguments/MIPS_O3_IDA_EN.lst}

The first four function arguments are passed in four registers prefixed by A-.

\myindex{MIPS!\Instructions!MULT}

There are two special registers in MIPS: HI and LO which are filled with the 64-bit result of the multiplication during the execution of the \TT{MULT} instruction.
\myindex{MIPS!\Instructions!MFLO}
\myindex{MIPS!\Instructions!MFHI}

These registers are accessible only by using the \TT{MFLO} and \TT{MFHI} instructions.
\TT{MFLO} here takes the low-part of the multiplication result and stores it into \$V0.
So the high 32-bit part of the multiplication result is dropped (the HI register content is not used).
Indeed: we work with 32-bit \Tint data types here.

\myindex{MIPS!\Instructions!ADDU}

Finally, \TT{ADDU} (\q{Add Unsigned}) adds the value of the third argument to the result.

\myindex{MIPS!\Instructions!ADD}
\myindex{MIPS!\Instructions!ADDU}
\myindex{Ada}
\myindex{Integer overflow}

There are two different addition instructions in MIPS: \TT{ADD} and \TT{ADDU}.
The difference between them is not related to signedness, but to exceptions. \TT{ADD} can raise an exception on overflow, which is sometimes useful\footnote{\url{http://go.yurichev.com/17326}} and supported in Ada \ac{PL}, for instance.
\TT{ADDU} does not raise exceptions on overflow.

Since \CCpp does not support this, in our example we see \TT{ADDU} instead of \TT{ADD}.

The 32-bit result is left in \$V0.

\myindex{MIPS!\Instructions!JAL}
\myindex{MIPS!\Instructions!JALR}

There is a new instruction for us in \main: \TT{JAL} (\q{Jump and Link}). 

The difference between \INS{JAL} and \INS{JALR} is that a relative offset is encoded in the first instruction, 
while \INS{JALR} jumps to the absolute address stored in a register (\q{Jump and Link Register}).

Both \ttf and \main functions are located in the same object file, so the relative address of \ttf 
is known and fixed.
