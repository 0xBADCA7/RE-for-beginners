\subsubsection{MSVC + \olly}
\myindex{\olly}
Illustrons ceci dans \olly.
Lorsque nous traçons jusqu'à la première instruction de \ttf qui utilise un des
arguments (le premier), nous voyons qu'\EBP pointe sur la \glslink{stack frame}{structure de pile locale},
qui est entourée par un rectangle rouge. 

Le premier élément de la \glslink{stack frame}{structure de pile locale} est la
valeur sauvegardée de \EBP, le second est \ac{RA}, le troisième est le premier
argument de la fonction, puis le second et le troisième.

Pour accèder au premier argument de la fonction, on doit ajouter exactement 8 (2
mots de 32-bit) à \EBP.

\olly est au courant de cela, c'est pourquoi il a ajouté des commentaires aux éléments
de la pile comme

\q{RETURN from} et \q{Arg1 = \dots}, etc.

N.B.: Les arguments de la fonction ne font pas partie de la structure de pile de
la fonction, ils font plutôt partie de celle de la fonction \glslink{caller}{appelante}.

Par conséquent, \olly a marqué les éléments comme appartenant à une autre structure
de pile.

\begin{figure}[H]
\centering
\myincludegraphics{patterns/05_passing_arguments/olly.png}
\caption{\olly: à l'intérieur de la fonction \ttf{}}
\label{fig:passing_arguments_olly}
\end{figure}

