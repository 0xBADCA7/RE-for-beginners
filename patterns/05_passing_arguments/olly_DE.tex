\subsubsection{MSVC + \olly}
\myindex{\olly}
Lasst uns die Darstellung in \olly betrachten

Wenn wir die erste Instruktion tracen in \ttf das auf eines der Argumente
zugreift (das erste), können wir sehen das \EBP auf den \gls{stack frame} zeigt,
dieser Frame wird mit dem roten Rechteck markiert dargestellt.

Das erste Element des \gls{stack frame} ist der gespeicherte Wert von \EBP,
das zweite Element ist \ac{RA}, das dritte Element ist das erste Funktions Argument, dann
folgt das zweite und dritte Funktions Argument.

Um auf das erste Funktions Argument zu zugreifen, muss man lediglich exakt 8 (2 32-Bit Wörter) zu 
\EBP addieren.

\olly erkennt diesen Umstand, und Kommentare zu den entsprechenden Stack Elementen hinzugefügt zum Beispiel:

\q{RETURN from} und \q{Arg1 = \dots}, etc.

Beachte: Funktions Argumente sind keine Mitglieder des Funktions Stack Frame, sie sind eher
Mitglieder des Stack Frame der \gls{caller} Funktion.

Deswegen, hat \olly die \q{Arg} Elemente als Mitglied eines anderen Stackframes identifiziert.

\begin{figure}[H]
\centering
\myincludegraphics{patterns/05_passing_arguments/olly.png}
\caption{\olly: inside of \ttf{} function}
\label{fig:passing_arguments_olly}
\end{figure}

