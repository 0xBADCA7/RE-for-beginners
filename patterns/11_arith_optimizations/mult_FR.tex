\subsection{Multiplication}

\subsubsection{Multiplication en utilisant l'addition}

Voici un exemple simple:

\begin{lstlisting}[style=customc]
unsigned int f(unsigned int a)
{
	return a*8;
};
\end{lstlisting}

La multiplication par 8 a été remplacée par 3 instructions d'addition, qui font la
même chose.
Il semble que l'optimiseur de MSVC a décidé que ce code peut être plus rapide.

\begin{lstlisting}[caption=MSVC 2010 \Optimizing,style=customasmx86]
_TEXT	SEGMENT
_a$ = 8		; size = 4
_f	PROC
; File c:\polygon\c\2.c
	mov	eax, DWORD PTR _a$[esp-4]
	add	eax, eax
	add	eax, eax
	add	eax, eax
	ret	0
_f	ENDP
_TEXT	ENDS
END
\end{lstlisting}

\subsubsection{Multiplication en utilisant le décalage}
\label{subsec:mult_using_shifts}

Les instructions de multiplication et de division par un nombre qui est une puissance
de 2 sont souvent remplacées par des instructions de décalage.

\begin{lstlisting}[style=customc]
unsigned int f(unsigned int a)
{
	return a*4;
};
\end{lstlisting}

\begin{lstlisting}[caption=MSVC 2010 \NonOptimizing,style=customasmx86]
_a$ = 8		; size = 4
_f	PROC
	push	ebp
	mov	ebp, esp
	mov	eax, DWORD PTR _a$[ebp]
	shl	eax, 2
	pop	ebp
	ret	0
_f	ENDP
\end{lstlisting}


La multiplication par 4 consiste en un décalage du nombre de 2 bits vers la gauche
et l'insertion de deux bits à zéro sur la droite (les deux derniers bits).
C'est comme multiplier 3 par 100~---nous devons juste ajouter deux zéros sur la droite.

C'est ainsi que fonctionne l'instruction de décalage vers la gauche:

\myindex{x86!\Instructions!SHL}
\input{shift_left}

Les bits ajoutés à droite sont toujours des zéros.

Multiplication par 4 en ARM:

\begin{lstlisting}[caption=\NonOptimizingKeilVI (\ARMMode),style=customasmARM]
f PROC
        LSL      r0,r0,#2
        BX       lr
        ENDP
\end{lstlisting}

Multiplication par 4 en MIPS:

\lstinputlisting[caption=GCC 4.4.5 \Optimizing (IDA),style=customasmMIPS]{patterns/11_arith_optimizations/MIPS_SLL.lst}

\myindex{MIPS!\Instructions!SLL}
\INS{SLL} is \q{Shift Left Logical}.

\subsubsection{Multiplication en utilisant le décalage, la soustraction, et l'addition}
\label{multiplication_using_shifts_adds_subs}

Il est aussi possible de se passer des opérations de multiplication lorsque l'on
multiplie par des nombres comme 7 ou 17, toujours en utilisant le décalage.
Les mathématiques utilisées ici sont assez facile.

\myparagraph{32-bit}

\lstinputlisting[style=customc]{patterns/11_arith_optimizations/mult_shifts.c}

\mysubparagraph{x86}

\lstinputlisting[caption=MSVC 2012 \Optimizing,style=customasmx86]{patterns/11_arith_optimizations/mult_shifts_MSVC_2012_Ox.asm}

\mysubparagraph{ARM}

Keil pour le mode ARM tire partie du décalage de registre du second opérande:

\lstinputlisting[caption=\OptimizingKeilVI (\ARMMode),style=customasmARM]{patterns/11_arith_optimizations/mult_shifts_Keil_ARM_O3.s}

Mais ce n'est pas disponible en mode Thumb.
Il ne peut donc pas l'optimiser:

\lstinputlisting[caption=\OptimizingKeilVI (\ThumbMode),style=customasmARM]{patterns/11_arith_optimizations/mult_shifts_Keil_thumb_O3.s}

\mysubparagraph{MIPS}

\lstinputlisting[caption=GCC 4.4.5 \Optimizing (IDA),style=customasmMIPS]{patterns/11_arith_optimizations/mult_shifts_MIPS_O3_IDA.lst}

\myparagraph{64-bit}

\lstinputlisting[style=customc]{patterns/11_arith_optimizations/mult_shifts_64.c}

\mysubparagraph{x64}

\lstinputlisting[caption=MSVC 2012 \Optimizing,style=customasmx86]{patterns/11_arith_optimizations/mult_shifts_64_GCC49_x64_O3.s}

\mysubparagraph{ARM64}

GCC 4.9 pour ARM64 et aussi concis, grâce au modificateur de décalage:

\lstinputlisting[caption=GCC (Linaro) 4.9 \Optimizing ARM64,style=customasmARM]{patterns/11_arith_optimizations/mult_shifts_64_GCC49_ARM64.s}

\myparagraph{Algorithme de multiplication de Booth}

\myindex{Data general Nova}
\myindex{Booth's multiplication algorithm}
Il fût un temps oú les ordinateurs étaient si gros et cher, que certains d'entre
eux ne disposait pas de la multiplication dans le \ac{CPU}, comme le Data General Nova.
Et losrque l'on avait besoin de l'opérateur de multiplication, il pouvait être fourni
au niveau logiciel, par exemple, en utilisant l'algorithme de multiplication de Booth.
C'est un algorithme de multiplication qui utilise seulement des opérations d'addition
et de décalage.

Ce que les optimiseurs des compilateurs modernes font n'est pas la même chose, mais
le but (multiplication) et les ressources (des opérations plus rapide) sont les mêmes.
