\subsection{Multiplication}

\subsubsection{Multiplication using addition}

Here is a simple example:

\begin{lstlisting}[style=customc]
unsigned int f(unsigned int a)
{
	return a*8;
};
\end{lstlisting}

Multiplication by 8 is replaced by 3 addition instructions, which do the same.
Apparently, MSVC's optimizer decided that this code can be faster.

\begin{lstlisting}[caption=\Optimizing MSVC 2010,style=customasmx86]
_TEXT	SEGMENT
_a$ = 8		; size = 4
_f	PROC
; File c:\polygon\c\2.c
	mov	eax, DWORD PTR _a$[esp-4]
	add	eax, eax
	add	eax, eax
	add	eax, eax
	ret	0
_f	ENDP
_TEXT	ENDS
END
\end{lstlisting}

\subsubsection{Multiplication using shifting}
\label{subsec:mult_using_shifts}

Multiplication and division instructions by a numbers that's a power of 2 are often replaced by shift instructions.

\begin{lstlisting}[style=customc]
unsigned int f(unsigned int a)
{
	return a*4;
};
\end{lstlisting}

\begin{lstlisting}[caption=\NonOptimizing MSVC 2010,style=customasmx86]
_a$ = 8		; size = 4
_f	PROC
	push	ebp
	mov	ebp, esp
	mov	eax, DWORD PTR _a$[ebp]
	shl	eax, 2
	pop	ebp
	ret	0
_f	ENDP
\end{lstlisting}


Multiplication by 4 is just shifting the number to the left by 2 bits
and inserting 2 zero bits at the right (as the last two bits).
It is just like multiplying 3 by 100~---we just have to add two zeros at the right.

That's how the shift left instruction works:

\myindex{x86!\Instructions!SHL}
\input{shift_left}

The added bits at right are always zeros.

Multiplication by 4 in ARM:

\begin{lstlisting}[caption=\NonOptimizingKeilVI (\ARMMode),style=customasmARM]
f PROC
        LSL      r0,r0,#2
        BX       lr
        ENDP
\end{lstlisting}

Multiplication by 4 in MIPS:

\lstinputlisting[caption=\Optimizing GCC 4.4.5 (IDA),style=customasmMIPS]{patterns/11_arith_optimizations/MIPS_SLL.lst}

\myindex{MIPS!\Instructions!SLL}
\INS{SLL} is \q{Shift Left Logical}.

\subsubsection{Multiplication using shifting, subtracting, and adding}
\label{multiplication_using_shifts_adds_subs}

It's still possible to get rid of the multiplication operation when you multiply by numbers like
7 or 17 again by using shifting.
The mathematics used here is relatively easy.

\myparagraph{32-bit}

\lstinputlisting[style=customc]{patterns/11_arith_optimizations/mult_shifts.c}

\mysubparagraph{x86}

\lstinputlisting[caption=\Optimizing MSVC 2012,style=customasmx86]{patterns/11_arith_optimizations/mult_shifts_MSVC_2012_Ox.asm}

\mysubparagraph{ARM}

Keil for ARM mode takes advantage of the second operand's shift modifiers:

\lstinputlisting[caption=\OptimizingKeilVI (\ARMMode),style=customasmARM]{patterns/11_arith_optimizations/mult_shifts_Keil_ARM_O3.s}

But there are no such modifiers in Thumb mode.
It also can't optimize \TT{f2()}:

\lstinputlisting[caption=\OptimizingKeilVI (\ThumbMode),style=customasmARM]{patterns/11_arith_optimizations/mult_shifts_Keil_thumb_O3.s}

\mysubparagraph{MIPS}

\lstinputlisting[caption=\Optimizing GCC 4.4.5 (IDA),style=customasmMIPS]{patterns/11_arith_optimizations/mult_shifts_MIPS_O3_IDA.lst}

\myparagraph{64-bit}

\lstinputlisting[style=customc]{patterns/11_arith_optimizations/mult_shifts_64.c}

\mysubparagraph{x64}

\lstinputlisting[caption=\Optimizing MSVC 2012,style=customasmx86]{patterns/11_arith_optimizations/mult_shifts_64_GCC49_x64_O3.s}

\mysubparagraph{ARM64}

GCC 4.9 for ARM64 is also terse, thanks to the shift modifiers:

\lstinputlisting[caption=\Optimizing GCC (Linaro) 4.9 ARM64,style=customasmARM]{patterns/11_arith_optimizations/mult_shifts_64_GCC49_ARM64.s}

\myparagraph{Booth's multiplication algorithm}

\myindex{Data general Nova}
\myindex{Booth's multiplication algorithm}
There was a time when computers were big and that expensive, that some of them lacked hardware support of multiplication
operation in \ac{CPU}, like Data General Nova.
And when one need multiplication operation, it can be provided at software level, for example, using Booth's multiplication
algorithm.
This is a multiplication algorithm which uses only addition operation and shifts.

What modern optimizing compilers do, isn't the same,
but the goal (multiplication) and resources (faster operations) are the same.

