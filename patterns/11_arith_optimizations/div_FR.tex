\subsection{Division}

\subsubsection{Division en utilisant des décalages}
\label{division_by_shifting}

Exemple de division par 4:

\begin{lstlisting}[style=customc]
unsigned int f(unsigned int a)
{
	return a/4;
};
\end{lstlisting}

Nous obtenons (MSVC 2010):

\begin{lstlisting}[caption=MSVC 2010,style=customasmx86]
_a$ = 8		; size = 4
_f	PROC
	mov	eax, DWORD PTR _a$[esp-4]
	shr	eax, 2
	ret	0
_f	ENDP
\end{lstlisting}

\label{SHR}
\myindex{x86!\Instructions!SHR}

L'instruction \SHR (\IT{SHift Right} décalage à droite) dans cet exemple décale un
nombre de 2 bits vers la droite.
Les deux bits libérés à gauche (i.e., les deux bits les plus significatifs) sont
mis à zéro.
Les deux bits les moins significatifs sont perdus.
En fait, ces deux bits perdus sont le reste de la division.

\myindex{x86!\Instructions!SHR}

L'instruction \SHR fonctionne tout comme \SHL, mais dans l'autre direction.

\input{shift_right}

Il est facile de comprendre si vous imaginez le nombre 23 dans le système décimal.
23 peut être facilement divisé par 10, juste en supprimant le dernier chiffre (3---le
reste de la division).
Il reste 2 après l'opération, qui est le \gls{quotient}.

Donc, le reste est perdu, mais c'est OK, nous travaillons de toutes façons sur des
valeurs entières, ce sont sont pas des \glslink{real number}{nombres réels}!

Division par 4 en ARM:

\begin{lstlisting}[caption=\NonOptimizingKeilVI (\ARMMode),style=customasmARM]
f PROC
        LSR      r0,r0,#2
        BX       lr
        ENDP
\end{lstlisting}

Division par 4 en MIPS:

\begin{lstlisting}[caption=GCC 4.4.5 \Optimizing (IDA),style=customasmMIPS]
        jr      $ra
        srl     $v0, $a0, 2 ; slot de délai de branchement
\end{lstlisting}

\myindex{MIPS!\Instructions!SRL}
L'instruction SLR est \q{Shift Right Logical} (décalage logique à droite).
