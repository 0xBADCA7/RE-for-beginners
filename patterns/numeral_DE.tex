\subsection{Zahlensysteme}

Menschen sind an das Dezimal-System gewöhnt, möglicherweise weil sie zehn Finger haben.
Trotzdem hat die Zahl 10 keine besondere Bedeutung in der Wissenschaft und Mathematik.
Das natürliche Zahlensystem in der Digitaltechnik ist binär: 0 für die Abwesenheit und 1 für die Anwesenheit von
Strom in einer Leitung. Die binäre 10 ist 2 im Dezimalsystem; Die binäre 100 ist 4 im Dezimalsystem und so weiter.

Wenn ein Zahlensystem 10 Ziffern hat, spricht man von \IT{Basis} von 10.
Binäre Zahlensysteme haben die \IT{Basis} von 2.

Wichtige Dinge zum Merken:
\begin{enumerate}
\item \IT{Nummer} ist eine Nummer, während \IT{Ziffer} in der Regel eine einzelne Zahl ist;
\item Eine Zahl ändert sich nicht beim Konvertieren in ein anderes Zahlensystem: nur die Darstellung ist anders.
\end{enumerate}

Wie konvertiert man eine Zahl von einer Basis in eine andere?

Fast überall wird das Stellenwertsystem genutzt. Dies bedeutet, dass eine einzelne Ziffer je nach Position in der Zahl ein bestimmtes Gewicht hat.
Wenn 2 an der rechten Position steht, ist es eine 2.
Wenn sie jedoch eine Position weiter links steht, ist die Zahl eine 20.

Wofür steht $1234$?

$10^3 \cdot 1 + 10^2 \cdot 2 + 10^1 \cdot 3 + 1 \cdot 4$ = 1234 oder 
$1000 \cdot 1 + 100 \cdot 2 + 10 \cdot 3 + 4 = 1234$

Das Gleiche gilt für binäre Zahlen, nur zur Basis 2 statt 10.
Wofür steht 0b101011?

$2^5 \cdot 1 + 2^4 \cdot 0 + 2^3 \cdot 1 + 2^2 \cdot 0 + 2^1 \cdot 1 + 2^0 \cdot 1 = 43$ oder
$32 \cdot 1 + 16 \cdot 0 + 8 \cdot 1 + 4 \cdot 0 + 2 \cdot 1 + 1 = 43$

Stellenwertsysteme können den Additionssystemen wie den römischen Ziffern gegenübergestellt werden
\footnote{Zur Entwickling der Zahlensysteme, siehe \InSqBrackets{\TAOCPvolII{}, 195--213.}}.
Möglicherweise hat die Menschheit zu Stellenwertsystemen gewechselt, weil diese für einfache Operationen (Addition, Multiplikation, usw.)
per Hand oder auf Papier einfacher zu verwenden sind.

Tatsächlich können binäre Zahlen genauso addiert, subtrahiert und so weiter werden wie man es in der Schule gelernt hat,
es stehen allerdings nur zwei Ziffern zur Verfügung.

Binärzahlen sind sperrig, wenn sie im Quellcode oder Speicherauszügen auftreten. Hier bietet sich die Verwendung des Hexadezimalsystems an.
Die Basis besteht hier aus den Ziffen 0..9 und den sechs lateinischen Buchstaben A..F.
Jede hexadezimale Ziffer besteht aus vier Bits oder vier binären Ziffern, was die Konvertierung zwischen Hexadezimal und Binär,
selbst im Kopf, sehr einfach macht.

\begin{center}
\begin{longtable}{ | l | l | l | }
\hline
\HeaderColor Hexadezimal & \HeaderColor Binär & \HeaderColor Dezimal \\
\hline
0	&0000	&0 \\
1	&0001	&1 \\
2	&0010	&2 \\
3	&0011	&3 \\
4	&0100	&4 \\
5	&0101	&5 \\
6	&0110	&6 \\
7	&0111	&7 \\
8	&1000	&8 \\
9	&1001	&9 \\
A	&1010	&10 \\
B	&1011	&11 \\
C	&1100	&12 \\
D	&1101	&13 \\
E	&1110	&14 \\
F	&1111	&15 \\
\hline
\end{longtable}
\end{center}

Woran sieht man welche Basis gerade verwendet wird?

Dezimalzahlen werden für gewöhnlich ohne Zusatz geschrieben, zum Beispiel 1234. Einige Assembler betonen die Basis 10
jedoch mit dem Zusatz "d": 1234d.

Binärzahlen sind manchmal mit dem Präfix "0b" gekennzeichnet: 0b100110111 \ac{GCC} hat eine Nicht-standardisierte
Erweiterung hierfür\footnote{\url{https://gcc.gnu.org/onlinedocs/gcc/Binary-constants.html})}.
Ein weiterer Weg ist der "b" Suffix, zum Beispiel: 100110111b.
In diesem Buch wird versucht durchgängig die "0b"-Präfix-Variante zu benutzen.

Hexadezimalzahlen werden in \CCpp und anderen Hochsprachen mit dem "0x"-Präfix versehen: 0x1234ABCD,
oder sie haben einen "h"-Suffix: 1234ABCDh - dies ist eine weit verbreitete Variante in Assembler und Debuggern.
Wenn die Zahl mit A..F startet, muss eine 0 davor geschrieben werden: 0ABCDEFh.
In diesem Buch wird versucht durchgängig die "0x"-Präfix-Variante zu benutzen.

Ist es ratsam das Konvertieren von Zahlen im Kopf zu Üben? Die Tabelle der einstelligen Hexadezimalziffern kann
leicht auswendig gelernt werden. Für größere Zahlen lohnt sich der Aufwand vielleicht nicht wirklich.

\subsubsection{Oktalsystem}
Dies ist ein weiteres Zahlensystem, welches in der Computerprogrammierung sehr verbreitet ist: die 8 Ziffern (0..7)
sind jeweils drei Bits zugeordnet. Dies macht das Konvertieren in andere Zahlensysteme sehr einfach.
Das Oktalsystem wurde fast überall ersetzt, dennoch gibt es, überraschenderweise, ein *NIX-Tool, welches häufig genutzt wird
und eine Oktalzahl als Aufrufparameter hat:\TT{chmod}.
\myindex{UNIX!chmod}

Wie viele *NIX-Nutzer wissen, ist der Aufrufparameter von \TT{chmod} eine Zahl mit drei Ziffern.
Die erste beschreibt die Rechte des Besitzers einer Datei, die zweite der Gruppe zu der die Datei gehört
und die dritte ist für alle anderen.
Jede Ziffer kann in ihrer binären Form repräsentiert werden:

\begin{center}
\begin{longtable}{ | l | l | l | }
\hline
\HeaderColor Dezimal & \HeaderColor Binär & \HeaderColor Bedeutung \\
\hline
7	&111	&\textbf{rwx} \\
6	&110	&\textbf{rw-} \\
5	&101	&\textbf{r-x} \\
4	&100	&\textbf{r-{}-} \\
3	&011	&\textbf{-wx} \\
2	&010	&\textbf{-w-} \\
1	&001	&\textbf{-{}-x} \\
0	&000	&\textbf{-{}-{}-} \\
\hline
\end{longtable}
\end{center}

Jedes Bit wird also abgebildet auf die Flags: lesen/schreiben/ausführen.

Der Grund warum hier das \TT{chmod}-Kommando erwähnt wird, ist weil der Aufrufparameter auch in Oktalform angegeben werden kann.
Zum Beispiel die 644: wenn \TT{chmod 644 file} ausgeführt wird, werden Lese/Schreib-Rechte für den Besitzer, Lese-Rechte für
die Gruppe und ebenfalls Lese-Rechte für alle anderen gesetzt.

Die Oktalzahl 644 ist in binärer Form \TT{110100100}, oder (in Dreierbitgruppen) \TT{110 100 100}.

Man erkennt nun sehr schön, dass jedes dieser Dreiergruppen die Rechte für Besitzer/Gruppe/Andere beschreibt:
zuerst  \TT{rw-}, dann \TT{r--} und zuletzt nochmals \TT{r--}.

Das Oktalsystem war sehr populär auf alten Computerarchitekturen wie PDP-8, weil ein Word hier 12, 24 oder 36 Bit breit sein konnte,
und diese Zahlen alle durch drei teilbar sind. Daher war die Verwendung des Oktalsystems am Natürlichsten.
Heutzutage sind die Word- und Adressbreiten der Computer 16, 32 oder 64 Bit, und damit durch vier teilbar. Dementsprechend ist die
Verwendung des Hexadezimalsystems natürlicher und intuitiver.

Das Oktalsystem wird von allen standardkonformen \CCpp-Compilern unterstützt.
Dies kann gelegentlich zu Verwirrung führen, weil Oktalzahlen mit einer vorangestellten Null gekennzeichnet werden,
also zum Beispiel 0277, was 255 in der Dezimalschreibweise entspricht.
Ein versehentlicher Schreibfehler wie "09" statt "9" wird der Compiler erkennen, weil die 9 keine Ziffer des Oktalsystems ist.
GCC meldet einen Fehler wie:\\
\TT{error: invalid digit "9" in octal constant}.

\subsubsection{Teilbarkeit}

Wenn Sie eine Dezimalzahl wie 120 sehen, ist schnell ersichtlich, dass diese durch 10 teilbar ist, weil die letzte Ziffer eine Null ist.
Genauso ist 123400 durch 100 teilbar, weil die letzten zwei Ziffern Nullen sind.

Auf die gleiche Weise ist die Hexadezimalzahl 0x1230 durch 0x10 (oder 16) und 0x123000 durch 0x1000 (oder 4096) teilbar.

Die Binärzahl 0b1000101000 ist durch 0b1000 (8) teilbar und so weiter.
Diese Eigenschaft kann häufig dazu genutzt werden um schnell herauszufinden ob ein Speicherinhalt zu einer bestimmten Grenze
aufgefüllt ist. Beispielsweise starten Sektionen in \ac{PE}-Dateien fast immer an Adressen die in hexadezimaler Schreibweise
mit drei Nullen enden: 0x41000, 0x10001000, usw.
Der Grund dafür ist, dass fast alle diese Sektionen an Grenzen ausgerichtet sind, die Vielfache von 0x1000 (4096) Byte sind.

\subsubsection{Langzahlarithmetik und Basis}
\index{RSA}
Langzahlarithmetik nutzt sehr große Zahlen die in mehreren Bytes gespeichert sein können. Der öffentliche und private
Schlüssel im RSA-Algorithmus beispielsweise benötigt 4096 Bit und mehr.
\DEph{}
% TODO translate this part so that it is understandable for beginners
%In \InSqBrackets{\TAOCPvolII, 265} we can find the following idea: when you store multi-precision number in several bytes,
%the whole number can be represented as having a radix of $2^8=256$, and each digit goes to corresponding byte.
%Likewise, if you store multi-precision number in several 32-bit integer values, each digit goes to each 32-bit slot,
%and you may think about this number as stored in radix of $2^{32}$.

