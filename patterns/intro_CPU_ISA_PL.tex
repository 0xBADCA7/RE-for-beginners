\subsection{Krótkie wprowadzenie do CPU}

\ac{CPU} jest urządzeniem wykonującym wszystkie programy.

\textbf{Terminologia:}

\begin{description}
\item[Instrukcja]: prymitywny rozkaz \ac{CPU}.
Najprostsze przykłady: poruszanie się między rejestrami, zarządzanie pamięcią, prymitywne operacje arytmetyczne. 

Z reguły, każdy \ac{CPU} ma swój zestaw instrukcji (\ac{ISA}).

\item[Kod maszynowy]: kod, zrozumiały dla \ac{CPU}. 
Każda instrukcja zwykle jest kodowana za pomocą kilku bajtów.
\item[Język assemblera]: kod maszynowy plus niektóre rozszerzenia, stworzone po to, żeby ułatwić pracę programiście: makro, nazwy itd.
\item[Rejestr CPU]: Każde \ac{CPU} ma swój fiksowany zestaw rejestrów ogólnego przeznaczenia (\ac{GPR}).
$\approx 8$ w x86, $\approx 16$ w x86-64, $\approx 16$ w ARM.
Najłatwiej traktować rejestr jako zmieną tymczasową bez określonego typu. Można wyobrazić, że piszecie w wysokopoziomowym języku programowania i macie tylko 8 zmiennych o szerokości 32 (lub 64) bajty. Można zrobić naprawdę wiele rzeczy korzystając tylko z nich!
\end{description}

Skąd się wzięła różnica między kodem maszynowym a językiem programowania wysokiego poziomu? Odpowiedź polega na tym, że ludzie i \ac{CPU} różnią się między sobą - dla człowieka jest o wiele prostsze pisać w wysokopoziomowym języku programowania typu \CCpp, Java, Python, а dla \ac{CPU} łatwiej jest pracować na niższym poziomie abstrakcji.
Możliwe, że można by było wymyśleć \ac{CPU} la CPU łatwiej jest pracować na niższym poziomie abstrakcji. Możliwe, że można by było wymyśleć CPU wykonujące kod wysokopoziomowego języka programowania, ale było by to o wiele bardziej skomplikowane urządzenie, niż to co mamy teraz. I na odwrót, dla człowieka pisać w assemblerze jest bardzo niewygodne, do tego ciężko się obejść bez drobnych błędów. program translujący kod z wysokopoziomowego języka programowania na assembler nazywa się kompilatorem.

\myindex{ARM!\ARMMode}%
\myindex{ARM!\ThumbMode}%
\myindex{ARM!\ThumbTwoMode}%

\subsubsection{Kilka słów o różnicy między \ac{ISA}}
x86 od zawsze był architekturą z opcode zmiennej długości, tak że kiedy nadszedł czas 64-bitowej architektury, rozszerzenia x64 niezbyt silno wpłyneły na  \ac{ISA}.
ARM to \ac{RISC}-procesor wyprodukowany biorąc pod uwagę opcode o jednakowej długości, co było pewną korzyścią w przeszłości. Na początku wszystkie instrukcje ARM były kodowane w czterech bajtach.%
\footnote{
Instrukcje o rozmiarze stałym są wygodne dlatego że jest łątwo łatwo znaleźć adres następnej (lub poprzedniej) instrukcji. Będzie to dokładniej omawiane w sekcji o operatorze switch()~(\myref{sec:SwitchARMLot}).
}.
Jest to tym, na co obecnie się mówi \q{режим ARM}.
Później oni pomyśleli, że jest to niezbyt ekonomiczne. 
Wprawdzie, najbardziej wykorzystywane przez procesor instrukcje\footnote{Są nimi MOV/PUSH/CALL/Jcc} mogą być zakodowane z wykorzystaniem mniejszej ilości informacji. Także oni dodali inną \ac{ISA} o nazwie "Thumb", gdzie każda instrukcja koduję się tylko za pomocą 2 bajtów. Obecnie jest to nazywane \q{tryb Thumb}.
Ale nie wszystkie instrukcje ARM mogą być zakodowane w 2 bajtach, także zestaw instrukcji Thumb jest ograniczony. Kod, skompilowany dla trybu ARM i Thumb może współdziałać w jednym programie. Później producenci ARM stwierdzili, że Thumb można poszerzyć: tak się pojawił Thumb-2 (w ARMv7). Thumb-2 to wciąż dwubajtowe instrukcje, ale niektóre nowe instrukcje mają długość 4 bajty. Bardzo razpowszechnioną błęðną opinią jest to, że Thumb-2 - to mieszanina ARM i Thumb. Nie jest to prawdą. Tryb Thumb-2 był dopełniony, żeby lepiej wykorzystywać możliwości procesora i jest poważnym konkurentem trybu ARM. Aplikacje dla \idevices są głównie skompilowane dla zestawu instrukcji Thumb-2, ponieważ Xcode robi tak domyślnie. Później pojawił się 64-bitowy ARM. Jest to \ac{ISA} z znowu 4-bajtowymi opcode, bez dodatkowego trybu Thumb. Ale 64-bitowe wymagania wpłyneły na \ac{ISA}, tak że obecnie mamy 3 zestawy instrukcji ARM: tryb ARM, tryb Thumb/Thumb-2 i ARM64. Te zestawy instrukcji gdzieniegdzie się przecinają, ale można powiedzieć, że są to różne zestawy. Odpowiednio, w tej książce postaramy się zaprezentować fragmenty kodu we wszystkich trzech ARM  \ac{ISA}.
\myindex{PowerPC}%
\myindex{MIPS}%
\myindex{Alpha AXP}%
Istnieje jeszcze wiele \ac{RISC} \ac{ISA} z opcode fiksowanej 32-bitowej długości~--- co najmniej MIPS, PowerPC и Alpha AXP.

