\subsection{数値システム}

おそらくほとんどの人に10本の指があるので、人間は10進数字システムに慣れてきました。 それにもかかわらず、数\q{10}は科学と数学では重要な意味を持ちません。 デジタル電子機器の自然数システムはバイナリです.0は電線に電流が流れていないことを表し、1は存在を表します。 バイナリで10は10進数で2、バイナリで100は小数点で4などです。

% This sentence is a bit unweildy - maybe try 'Our ten-digit system would be described as having a radix...' - Renaissance
数値システムが10桁の場合、基数は10です。2進数字システムの基数は2です。

思い出すべき重要なこと:

1)\IT{数字}は数字であり、\IT{桁}は書記体系からの言葉であり、通常は1文字

2)数値の値は別の基数に変換されても変更されません。 その値に対する書き込み表記だけが変更されています(したがって、\ac{RAM}で表現する方法)。

\subsection{1つの基数から別の基数への変換}

位置表記はほぼすべての数値システムで使用されます。これは、数字が大きな数字の中に置かれている場所に対する相対的な重みを持つことを意味します。
2が右端に置かれている場合は2ですが、右端の前に1桁置かれている場合は20です。

1234は何を表しますか?

$10^3 \cdot 1 + 10^2 \cdot 2 + 10^1 \cdot 3 + 1 \cdot 4 = 1234$ or
$1000 \cdot 1 + 100 \cdot 2 + 10 \cdot 3 + 4 = 1234$

バイナリの数字は同じですが、ベースは10ではなく2です.0b101011は何を表していますか?

$2^5 \cdot 1 + 2^4 \cdot 0 + 2^3 \cdot 1 + 2^2 \cdot 0 + 2^1 \cdot 1 + 2^0 \cdot 1 = 43$ or
$32 \cdot 1 + 16 \cdot 0 + 8 \cdot 1 + 4 \cdot 0 + 2 \cdot 1 + 1 = 43$

ローマ数字のような非定位表記のようなものがあります。
\footnote{数値システムの進化については、195-213を参照してください。}
% Maybe add a sentence to fill in that X is always 10, and is therefore non-positional, even though putting an I before subtracts and after adds, and is in that sense positional
おそらく人類は紙で基本的な操作(加算、乗算など)を手作業で行う方が簡単であるため、位置表記法に切り替えました。

二進数は、学校で教えられたのと同じように追加、減算などが可能ですが、2桁しか利用できません。

2進数は、ソースコードとダンプで表現されているときにはかさばります。したがって、16進数表記が役立ちます。 16進の基数は、0..9の数字と6つのラテン文字A..Fを使用します。各16進数字は4ビットまたは4バイナリの数字を取るので、バイナリの数字から16進数に変換したり、手動でさえ戻したりするのは非常に簡単です。

\begin{center}
\begin{longtable}{ | l | l | l | }
\hline
\HeaderColor hexadecimal & \HeaderColor binary & \HeaderColor decimal \\
\hline
0	&0000	&0 \\
1	&0001	&1 \\
2	&0010	&2 \\
3	&0011	&3 \\
4	&0100	&4 \\
5	&0101	&5 \\
6	&0110	&6 \\
7	&0111	&7 \\
8	&1000	&8 \\
9	&1001	&9 \\
A	&1010	&10 \\
B	&1011	&11 \\
C	&1100	&12 \\
D	&1101	&13 \\
E	&1110	&14 \\
F	&1111	&15 \\
\hline
\end{longtable}
\end{center}

特定のインスタンスでどの基数が使用されているかをどのように知ることができますか?

小数点は通常1234のように書かれます。アセンブラの中には小数点以下の基数に識別子を付けることができますが、数字には接尾辞d(1234d)が付きます。

2進数には、接頭辞 "0b"が付いていることがあります:0b100110111(\ac{GCC}にはこのための非標準言語拡張があります\footnote{\url{https://gcc.gnu.org/onlinedocs/gcc/Binary-constants.html}})

もう1つの方法もあります。たとえば、 "b"接尾辞を使用します(例:100110111b)。この本では、バイナリ番号のために本の中で一貫して "0b"という接頭辞を使用しようとしています。

16進数の先頭には、Cppや他の\ac{PL}:0x1234ABCDの接頭辞「0x」が付加されています。あるいは、 "h"接尾辞1234ABCDhが与えられます。これはアセンブラとデバッガでそれらを表現する一般的な方法です。この規則では、数字がLatin(A..F)桁で始まる場合、先頭に0が追加されます(0ABCDEFh)。 ABCDのような\$接頭辞を使って8ビットの家庭用コンピュータ時代に普及した大会もありました。この本は16進数のために本の中に "0x"というプレフィックスを付けようとします。

数字を精神的に変換することを学ぶべきでしょうか? 1桁の16進数の表を簡単に記憶できます。大きな数字については、おそらく自分自身を苦しめる価値はありません。

おそらく最も目に見える16進数はURLにあります。これは非ラテン文字がコード化される方法です。
たとえば\url{https://en.wiktionary.org/wiki/na\%C3\%AFvet\%C3\%A9}は、\q{naïveté}という単語に関するWiktionaryの記事の\ac{URL}です。

\subsubsection{8進数}

コンピュータプログラミングの過去に多用された別の数字システムは8進数である。 8進数では8桁(0..7)であり、それぞれが3ビットにマッピングされるので、数値を前後に変換するのは簡単です。 ほぼすべての場所で16進法に取って代わられていますが、驚くべきことに、多くの人が頻繁に使う*NIXユーティリティがあります。これは引数として8進数をとります(\TT{chmod})。

\myindex{UNIX!chmod}
多くの* NIXユーザーが知っているように、\TT{chmod}引数は3桁の数字にすることができます。 最初の桁はファイル所有者の権利(読み込み、書き込み、実行)を表し、2番目はファイルが属するグループの権利で、3番目は他の人の権利です。 \TT{chmod}がとる各数字はバイナリ形式で表すことができます:

\begin{center}
\begin{longtable}{ | l | l | l | }
\hline
\HeaderColor decimal & \HeaderColor binary & \HeaderColor meaning \\
\hline
7	&111	&\textbf{rwx} \\
6	&110	&\textbf{rw-} \\
5	&101	&\textbf{r-x} \\
4	&100	&\textbf{r-{}-} \\
3	&011	&\textbf{-wx} \\
2	&010	&\textbf{-w-} \\
1	&001	&\textbf{-{}-x} \\
0	&000	&\textbf{-{}-{}-} \\
\hline
\end{longtable}
\end{center}

したがって、各ビットはフラグread / write / executeにマップされます。

ここで\TT{chmod}の重要性は、引数の整数全体を8進数で表現できることです。 \TT{chmod 644 file}を実行すると、所有者の読み取り/書き込み権限、グループの読み取り権限、他のユーザーの読み取り権限が再度設定されます。 8進数644を2進数に変換すると、\TT{110100100}、または3ビットのグループで\TT{110 100 100}になります。

各トリプレットは所有者/グループ/その他のパーミッションを記述しています。最初は\TT{rw-}、2番目は\TT{r--}、3番目は\TT{r--}です。

8進数字システムは、PDP-8のような古いコンピュータでも人気がありました。なぜなら、そこには12,24、または36ビットが存在する可能性があり、これらの数値はすべて3で割り切れるからです。今日、普及しているすべてのコンピュータは16,32,64ビットのワード/アドレスサイズを使用しており、これらの数値はすべて4で割り切れるため、16進数のシステムはより自然です。

すべての標準Cppコンパイラでサポートされています。これは混乱の原因となることがあります。なぜなら、8進数はゼロの前に付加されています(0377は255など)。時には、タイプミスをして9の代わりに "09"と書くことがあります。 GCCは次のようなことを報告するかもしれません:
\TT{error: 無効な数字 "9"は8進定数です}

また、8進数のシステムは、Javaではやや人気があります。 IDAが印刷不可能な文字を含むJava文字列を表示すると、16進数ではなく8進数でエンコードされます。 
\myindex{JAD}
JAD Javaデコンパイラも同じように動作します。

\subsubsection{除算能力}

120のような10進数を見ると、最後の桁がゼロであるため、itfsを10で割り切れるとすぐに推論することができます。 同様に、最後の2桁が0であるため、123400は100で割り切れる。 同様に、16進数の0x1230は0x10(または16)で割り切れ、0x123000は0x1000(または4096)で割り切れる

バイナリ番号0b1000101000は0b1000(8)などで割り切れます。このプロパティは、メモリ内の一部のブロックのサイズがある境界に埋め込まれているかどうかを素早く認識するためによく使用されます。 たとえば、PE12ファイルのセクションは、ほとんどの場合、0x41000、0x10001000など3つの16進ゼロで終わるアドレスで開始されます。これは、ほとんどすべての\ac{PE}セクションが0x1000(4096)バイトの境界にパディングされているためです。

\subsubsection{多精度算術演算と基数}

\index{RSA}
多精度算術演算では膨大な数を使用でき、それぞれが数バイトで格納されます。
たとえば、公開鍵と秘密鍵の両方のRSA鍵は、最大4096ビットに及んでいます。

\InSqBrackets{\TAOCPvolII, 265}において、我々は多バイト数で多倍精度の数値を格納するとき、整数を表すことができる28 = 256の基数を有するものとして、各桁は対応するバイトに進む。同様に、複数の精度の数値を複数の32ビットの整数値に格納すると、各桁は32ビットの各スロットに移動し、この数値は基数232に格納されていると考えることができます。

\subsubsection{非小数点の発音方法}

非小数点の基数の数字は、通常、桁で数字によって発音されます。イチ・ゼロ・ゼロ・イチ・イチ。 10や1000のような言葉は、小数点の基本システムとの混同を避けるために、通常は発音されません。

\subsubsection{浮動小数点数}

浮動小数点数を整数から区別するために、浮動小数点数は通常$0.0$, $123.0$などの末尾に.0で書かれています。

