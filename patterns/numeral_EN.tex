% TODO translate
\subsection{Numeral systems}

Humans accustomed to decimal numeral system probably because almost all ones has 10 fingers.
Nevertheless, number 10 has no significant meaning in science and mathematics.
Natural numeral system in digital electronics is binary: 0 is for absence of current in wire and 1 for presence.
10 in binary is 2 in decimal; 100 in binary is 4 in decimal and so on.

If the numeral system has 10 digits, it has \IT{radix} (or \IT{base}) of 10.
Binary numeral system has \IT{radix} of 2.

Important things to recall:
1) \IT{number} is a number, while \IT{digit} is a term of writing system and is usually one character;
2) number is not changed when converted to another radix: writing notation is.

How to convert a number from one radix to another?

Positional notation is used almost everywhere, this means, a digit (number placed in single character) has some weight depending on where it is placed.
If 2 is placed at the rightmost place, it's 2.
If it is placed at the place one digit before rightmost, it's 20.

What does $1234$ stand for?

$10^3 \cdot 1 + 10^2 \cdot 2 + 10^1 \cdot 3 + 1 \cdot 4$ = 1234 or 
$1000 \cdot 1 + 100 \cdot 2 + 10 \cdot 3 + 4 = 1234$

Same story for binary numbers, but base is 2 instead of 10.
What does 0b101011 stand for?

$2^5 \cdot 1 + 2^4 \cdot 0 + 2^3 \cdot 1 + 2^2 \cdot 0 + 2^1 \cdot 1 + 2^0 \cdot 1 = 43$ or
$32 \cdot 1 + 16 \cdot 0 + 8 \cdot 1 + 4 \cdot 0 + 2 \cdot 1 + 1 = 43$

Positional notation can be opposed to non-positional notation such as Roman numeric system
\footnote{About numeric system evolution, see \InSqBrackets{\TAOCPvolII{}, 195--213.}}.
Perhaps, humankind switched to positional notation because it's easier to do basic operations (addition, multiplication, etc.) on paper by hand.

Indeed, binary numbers can be added, subtracted and so on in the very same as taught in schools, but only 2 digits are available.

Binary numbers are bulky when represented in source code and dumps, so that is where hexadecimal numeral system can be used.
Hexadecimal radix uses 0..9 numbers and also 6 Latin characters: A..F.
Each hexadecimal digit takes 4 bits or 4 binary digits, so it's very easy to convert from binary number to hexadecimal and back, even manually, in one's mind.

\begin{center}
\begin{longtable}{ | l | l | l | }
\hline
\HeaderColor hexadecimal & \HeaderColor binary & \HeaderColor decimal \\
\hline
0	&0000	&0 \\
1	&0001	&1 \\
2	&0010	&2 \\
3	&0011	&3 \\
4	&0100	&4 \\
5	&0101	&5 \\
6	&0110	&6 \\
7	&0111	&7 \\
8	&1000	&8 \\
9	&1001	&9 \\
A	&1010	&10 \\
B	&1011	&11 \\
C	&1100	&12 \\
D	&1101	&13 \\
E	&1110	&14 \\
F	&1111	&15 \\
\hline
\end{longtable}
\end{center}

How to understand, which radix is currently used?

Decimal numbers are usually written as is, i.e., 1234. But some assemblers allows to make emphasis on decimal radix and this number can be written with "d" suffix: 1234d.

Binary numbers sometimes prepended with "0b" prefix: 0b100110111 (\ac{GCC} has non-standard language extension for this\footnote{\url{https://gcc.gnu.org/onlinedocs/gcc/Binary-constants.html})}.
There is also another way: "b" suffix, for example: 100110111b.
I'll try to stick to "0b" prefix throughout the book for binary numbers.

Hexadecimal numbers are prepended with "0x" prefix in \CCpp and other \ac{PL}s: 0x1234ABCD.
Or they are has "h" suffix: 1234ABCDh - this is common way of representing them in assemblers and debuggers.
If the number is started with A..F digit, 0 is to be added before: 0ABCDEFh.
I'll try to stick to "0x" prefix throughout the book for hexadecimal numbers.

Should one learn to convert numbers in mind? A table of 1-digit hexadecimal numbers can easily be memorized.
As of larger numbers, probably, it's not worth to torment yourself.

\subsubsection{Octal radix}

Another numeral system heavily used in past of computer programming is octal: there are 8 digits (0..7) and each is mapped to 3 bits, so it's easy to convert numbers back and forth.
It has been superseded by hexadecimal system almost everywhere, but surprisingly, there is *NIX utility used by many people often which takes octal number as argument: \TT{chmod}.

\myindex{UNIX!chmod}
As many *NIX users know, \TT{chmod} argument can be a number of 3 digits. The first one is rights for owner of file, second is rights for group (to which file belongs), third is for everyone else.
And each digit can be represented in binary form:

\begin{center}
\begin{longtable}{ | l | l | l | }
\hline
\HeaderColor decimal & \HeaderColor binary & \HeaderColor meaning \\
\hline
7	&111	&\textbf{rwx} \\
6	&110	&\textbf{rw-} \\
5	&101	&\textbf{r-x} \\
4	&100	&\textbf{r-{}-} \\
3	&011	&\textbf{-wx} \\
2	&010	&\textbf{-w-} \\
1	&001	&\textbf{-{}-x} \\
0	&000	&\textbf{-{}-{}-} \\
\hline
\end{longtable}
\end{center}

So each bit is mapped to a flag: read/write/execute.

Now the reason why I'm talking about \TT{chmod} here is that the whole number in argument can be represented as octal number.
Let's take for example, 644.
When you run \TT{chmod 644 file}, you set read/write permissions for owner, read permissions for group and again, read permissions for everyone else.
Let's convert 644 octal number to binary, this will be \TT{110100100}, or (in groups of 3 bits) \TT{110 100 100}.

Now we see that each triplet describe permissions for owner/group/others: first is \TT{rw-}, second is \TT{r--} and third is \TT{r--}.

Octal numeral system was also popular on old computers like PDP-8, because word there could be 12, 24 or 36 bits, and these numbers are divisible by 3, so octal system was natural on that environment.
Nowadays, all popular computers employs word/address size of 16, 32 or 64 bits, and these numbers are divisible by 4, so hexadecimal system is more natural here.

Octal numeral system is supported by all standard \CCpp compilers.
This is source of confusion sometimes, because octal numbers are encoded with zero prepended, for example, 0377 is 255.
And sometimes, you may make a typo and write "09" instead of 9, and the compiler wouldn't allow you.
GCC may report something like that:\\
\TT{error: invalid digit "9" in octal constant}.

\subsubsection{Divisibility}

When you see a decimal number like 120, you can quickly deduce that it's divisible by 10, because the last digit is zero.
In the same way, 123400 is divisible by 100, because two last digits are zeros.

Likewise, hexadecimal number 0x1230 is divisible by 0x10 (or 16), 0x123000 is divisible by 0x1000 (or 4096), etc.

Binary number 0b1000101000 is divisible by 0b1000 (8), etc.

This property can be used often to realize quickly if a size of some block in memory is padded to some boundary.
For example, sections in \ac{PE} files are almost always started at addresses ending with 3 hexadecimal zeros: 0x41000, 0x10001000, etc.
The reason behind this is in the fact that almost all \ac{PE} sections are padded to boundary of 0x1000 (4096) bytes.

\subsubsection{Multi-precision arithmetic and radix}

\index{RSA}
Multi-precision arithmetic uses huge numbers, and each one may be stored in several bytes.
For example, RSA keys, both public and private, are spanning up to 4096 bits and maybe even more.

In \InSqBrackets{\TAOCPvolII, 265} we can find the following idea: when you store multi-precision number in several bytes,
the whole number can be represented as having a radix of $2^8=256$, and each digit goes to corresponding byte.
Likewise, if you store multi-precision number in several 32-bit integer values, each digit goes to each 32-bit slot,
and you may think about this number as stored in radix of $2^{32}$.

\subsubsection{Pronouncement}

Numbers in non-decimal base are usually pronounced by one digit: ``one-zero-zero-one-one-...''.
Words like ``ten'', ``thousand'', etc, are usually not pronounced, because it will be confused with decimal base then.

