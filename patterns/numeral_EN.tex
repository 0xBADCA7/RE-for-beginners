\subsection{Numeral Systems}

Humans have become accustomed to a decimal numeral system, probably because almost everyone has 10 fingers.
Nevertheless, the number \q{10} has no significant meaning in science and mathematics.
The natural numeral system in digital electronics is binary: 0 is for an absence of current in the wire, and 1 for presence.
10 in binary is 2 in decimal, 100 in binary is 4 in decimal, and so on.

If the numeral system has 10 digits, it has a \IT{radix} (or \IT{base}) of 10.
The binary numeral system has a \IT{radix} of 2.

Important things to recall:

1) A \IT{number} is a number, while a \IT{digit} is a term from writing systems, and is usually one character

% The original is 'number' is not changed; I think the intent is value, and changed it - Renaissance
2) The value of a number does not change when converted to another radix; only the writing notation for that value has changed (and therefore the way of representing it in \ac{RAM}).

\subsection{Converting From One Radix To Another}

Positional notation is used almost every numerical system. This means that a digit has weight relative to where it is placed inside of the larger number.
If 2 is placed at the rightmost place, it's 2, but if it's placed one digit before rightmost, it's 20.

What does $1234$ stand for?

$10^3 \cdot 1 + 10^2 \cdot 2 + 10^1 \cdot 3 + 1 \cdot 4 = 1234$ or
$1000 \cdot 1 + 100 \cdot 2 + 10 \cdot 3 + 4 = 1234$

It's the same story for binary numbers, but the base is 2 instead of 10.
What does 0b101011 stand for?

$2^5 \cdot 1 + 2^4 \cdot 0 + 2^3 \cdot 1 + 2^2 \cdot 0 + 2^1 \cdot 1 + 2^0 \cdot 1 = 43$ or
$32 \cdot 1 + 16 \cdot 0 + 8 \cdot 1 + 4 \cdot 0 + 2 \cdot 1 + 1 = 43$

There is such a thing as non-positional notation, such as the Roman numeral system.
\footnote{About numeric system evolution, see \InSqBrackets{\TAOCPvolII{}, 195--213.}}.
% Maybe add a sentence to fill in that X is always 10, and is therefore non-positional, even though putting an I before subtracts and after adds, and is in that sense positional
Perhaps, humankind switched to positional notation because it's easier to do basic operations (addition, multiplication, etc.) on paper by hand.

Binary numbers can be added, subtracted and so on in the very same as taught in schools, but only 2 digits are available.

Binary numbers are bulky when represented in source code and dumps, so that is where the hexadecimal numeral system can be useful.
A hexadecimal radix uses the digits 0..9, and also 6 Latin characters: A..F.
Each hexadecimal digit takes 4 bits or 4 binary digits, so it's very easy to convert from binary number to hexadecimal and back, even manually, in one's mind.

\begin{center}
\begin{longtable}{ | l | l | l | }
\hline
\HeaderColor hexadecimal & \HeaderColor binary & \HeaderColor decimal \\
\hline
0	&0000	&0 \\
1	&0001	&1 \\
2	&0010	&2 \\
3	&0011	&3 \\
4	&0100	&4 \\
5	&0101	&5 \\
6	&0110	&6 \\
7	&0111	&7 \\
8	&1000	&8 \\
9	&1001	&9 \\
A	&1010	&10 \\
B	&1011	&11 \\
C	&1100	&12 \\
D	&1101	&13 \\
E	&1110	&14 \\
F	&1111	&15 \\
\hline
\end{longtable}
\end{center}

How does one understand, which radix is being used in a specific instance?

Decimal numbers are usually written as is, i.e., 1234. Some assemblers allow an identifier on decimal radix numbers, in which the number would be written with a "d" suffix: 1234d.

Binary numbers are sometimes prepended with the "0b" prefix: 0b100110111 (\ac{GCC} has a non-standard language extension for this\footnote{\url{https://gcc.gnu.org/onlinedocs/gcc/Binary-constants.html}}).
There is also another way: using a "b" suffix, for example: 100110111b.
This book tries to use the "0b" prefix consistently throughout the book for binary numbers.

Hexadecimal numbers are prepended with "0x" prefix in \CCpp and other \ac{PL}s: 0x1234ABCD.
Alternatively, they are given a "h" suffix: 1234ABCDh. This is common way of representing them in assemblers and debuggers.
In this convention, if the number is started with a Lating (A..F) digit, a 0 is added at the beginning: 0ABCDEFh.
There was also convention that was popular in 8-bit home computers era, using \$ prefix, like \$ABCD.
The book will try to stick to "0x" prefix throughout the book for hexadecimal numbers.

Should one learn to convert numbers mentally? A table of 1-digit hexadecimal numbers can easily be memorized.
As for larger numbers, it's probably not worth tormenting yourself.

Perhaps the most visible hexadecimal numbers are in \ac{URL}s.
This is the way that non-Latin characters are encoded.
For example:
\url{https://en.wiktionary.org/wiki/na\%C3\%AFvet\%C3\%A9} is the \ac{URL} of Wiktionary article about \q{naïveté} word.

\subsubsection{Octal Radix}

Another numeral system heavily used in the past of computer programming is octal. In octal there are 8 digits (0..7), and each is mapped to 3 bits, so it's easy to convert numbers back and forth.
It has been superseded by the hexadecimal system almost everywhere, but, surprisingly, there is a *NIX utility, used often by many people, which takes octal numbers as argument: \TT{chmod}.

\myindex{UNIX!chmod}
As many *NIX users know, \TT{chmod} argument can be a number of 3 digits. The first digit represents the rights of the owner of the file (read, write and/or execute), the second is the rights for the group to which the file belongs, and the third is for everyone else.
Each digit that \TT{chmod} takes can be represented in binary form:

\begin{center}
\begin{longtable}{ | l | l | l | }
\hline
\HeaderColor decimal & \HeaderColor binary & \HeaderColor meaning \\
\hline
7	&111	&\textbf{rwx} \\
6	&110	&\textbf{rw-} \\
5	&101	&\textbf{r-x} \\
4	&100	&\textbf{r-{}-} \\
3	&011	&\textbf{-wx} \\
2	&010	&\textbf{-w-} \\
1	&001	&\textbf{-{}-x} \\
0	&000	&\textbf{-{}-{}-} \\
\hline
\end{longtable}
\end{center}

So each bit is mapped to a flag: read/write/execute.

The importance of \TT{chmod} here is that the whole number in argument can be represented as octal number.
Let's take, for example, 644.
When you run \TT{chmod 644 file}, you set read/write permissions for owner, read permissions for group and again, read permissions for everyone else.
If we convert the octal number 644 to binary, it would be \TT{110100100}, or, in groups of 3 bits, \TT{110 100 100}.

Now we see that each triplet describe permissions for owner/group/others: first is \TT{rw-}, second is \TT{r--} and third is \TT{r--}.

The octal numeral system was also popular on old computers like PDP-8, because word there could be 12, 24 or 36 bits, and these numbers are all divisible by 3, so the octal system was natural in that environment.
Nowadays, all popular computers employ word/address sizes of 16, 32 or 64 bits, and these numbers are all divisible by 4, so the hexadecimal system is more natural there.

The octal numeral system is supported by all standard \CCpp compilers.
This is a source of confusion sometimes, because octal numbers are encoded with a zero prepended, for example, 0377 is 255.
Sometimes, you might make a typo and write "09" instead of 9, and the compiler would report an error.
GCC might report something like this:\\
\TT{error: invalid digit "9" in octal constant}.

Also, the octal system is somewhat popular in Java. When the IDA shows Java strings with non-printable characters,
they are encoded in the octal system instead of hexadecimal.
\myindex{JAD}
The JAD Java decompiler behaves the same way.

\subsubsection{Divisibility}

When you see a decimal number like 120, you can quickly deduce that it's divisible by 10, because the last digit is zero.
In the same way, 123400 is divisible by 100, because the two last digits are zeros.

Likewise, the hexadecimal number 0x1230 is divisible by 0x10 (or 16), 0x123000 is divisible by 0x1000 (or 4096), etc.

The binary number 0b1000101000 is divisible by 0b1000 (8), etc.

This property can often be used to quickly realize if the size of some block in memory is padded to some boundary.
For example, sections in \ac{PE} files are almost always started at addresses ending with 3 hexadecimal zeros: 0x41000, 0x10001000, etc.
The reason behind this is the fact that almost all \ac{PE} sections are padded to a boundary of 0x1000 (4096) bytes.

\subsubsection{Multi-Precision Arithmetic and Radix}

\index{RSA}
Multi-precision arithmetic can use huge numbers, and each one may be stored in several bytes.
For example, RSA keys, both public and private, span up to 4096 bits, and maybe even more.

% I'm not sure how to change this, but the normal format for quoting would be just to mention the suthor or book, and footnote to the full reference
In \InSqBrackets{\TAOCPvolII, 265} we find the following idea: when you store a multi-precision number in several bytes,
the whole number can be represented as having a radix of $2^8=256$, and each digit goes to the corresponding byte.
Likewise, if you store a multi-precision number in several 32-bit integer values, each digit goes to each 32-bit slot,
and you may think about this number as stored in radix of $2^{32}$.

\subsubsection{How to Pronounce Non-Decimal Numbers}

Numbers in a non-decimal base are usually pronounced by digit by digit: ``one-zero-zero-one-one-...''.
Words like ``ten'' and ``thousand'' are usually not pronounced, to prevent confusion with the decimal base system.

\subsubsection{Floating point numbers}

To distinguish floating point numbers from integers, they are usually written with ``.0'' at the end,
like $0.0$, $123.0$, etc.
