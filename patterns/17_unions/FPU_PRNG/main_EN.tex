\subsection{Pseudo-random number generator example}
\label{FPU_PRNG}

If we need float random numbers between 0 and 1, the simplest thing is to use a \ac{PRNG} like
the Mersenne twister. 
It produces random unsigned 32-bit values (in other words, it produces random 32 bits).
Then we can transform this value to \Tfloat and then
divide it by \GTT{RAND\_MAX} (\GTT{0xFFFFFFFF} in our case)---we getting a value in the 0..1 interval.

But as we know, division is slow.
Also, we would like to issue as few FPU operations as possible.
Can we get rid of the division?

\myindex{IEEE 754}

Let's recall what a floating point number consists of: sign bit, significand bits and exponent bits.
We just have to store random bits in all significand bits to get a random float number!

The exponent cannot be zero (the floating number is denormalized in this case),
so we are storing 0b01111111
to exponent---this means that the exponent is 1. 
Then we filling the significand with random bits, set the sign bit to
0 (which means a positive number) and voilà.
The generated numbers is to be between 1 and 2, so we must also subtract 1.

\newcommand{\URLXOR}{\url{http://go.yurichev.com/17308}}

A very simple linear congruential random numbers generator is used in my 
example\footnote{the idea was taken from: \URLXOR}, it produces 32-bit numbers. 
The \ac{PRNG} is initialized with the current time in UNIX timestamp format.

Here we represent the \Tfloat type as an \IT{union}---it is the \CCpp construction that enables us
to interpret a piece of memory as different types.
In our case, we are able to create a variable
of type \IT{union} and then access to it as it is \Tfloat or as it is \IT{uint32\_t}. 
It can be said, it is just a hack. A dirty one.

% WTF?

The integer \ac{PRNG} code is the same as we already considered: \myref{LCG_simple}.
So this code in compiled form is omitted.

\lstinputlisting[style=customc]{patterns/17_unions/FPU_PRNG/FPU_PRNG_EN.cpp}

\subsubsection{x86}

\lstinputlisting[caption=\Optimizing MSVC 2010,style=customasmx86]{patterns/17_unions/FPU_PRNG/MSVC2010_Ox_Ob0_EN.asm}

Function names are so strange here because this example was compiled as C++ and this is name mangling in C++,
we will talk about it later: \myref{namemangling}.
If we compile this in MSVC 2012, it uses the SIMD instructions for the FPU, read more about it here: \myref{FPU_PRNG_SIMD}.

\subsubsection{MIPS}

\lstinputlisting[caption=\Optimizing GCC 4.4.5,style=customasmMIPS]{patterns/17_unions/FPU_PRNG/MIPS_O3_IDA_EN.lst}

There is also an useless \INS{LUI} instruction added for some weird reason.
We considered this artifact earlier: \myref{MIPS_FPU_LUI}.

\subsubsection{ARM (\ARMMode)}

\lstinputlisting[caption=\Optimizing GCC 4.6.3 (IDA),style=customasmARM]{patterns/17_unions/FPU_PRNG/raspberry_GCC_O3_IDA_EN.lst}

\myindex{objdump}
\myindex{binutils}
\myindex{IDA}

We'll also make a dump in objdump and we'll see that the FPU instructions have different names than in \IDA.
Apparently, IDA and binutils developers used different manuals?
Perhaps it would be good to know both instruction name variants.

\lstinputlisting[caption=\Optimizing GCC 4.6.3 (objdump),style=customasmARM]{patterns/17_unions/FPU_PRNG/raspberry_GCC_O3_objdump.lst}

The instructions at 0x5c in \TT{float\_rand()} and at 0x38 in \main are (pseudo-)random noise.

