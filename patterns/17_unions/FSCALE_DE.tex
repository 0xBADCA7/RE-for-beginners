\section{FSCALE Ersatz}
\myindex{x86!\Instructions!FSCALE}
Agner Fog schreibt in seiner Abhandlung \IT{Optimizing subroutines in assembly language / An optimization guide for x86
platforms} \footnote{\url{http://www.agner.org/optimize/optimizing_assembly.pdf}} , dass der Befehl \INS{FSCALE}
\ac{FPU} (der $2^n$ berechnet) auf vielen CPUs langsam ist und bietet einen schnelleren Ersatz an.

Hier ist meine Übersetzung von seinem Assemblercode in \CCpp:

\lstinputlisting[style=customc]{patterns/17_unions/FSCALE.c}
Der Befehl \INS{FSCALE} kann zwar in bestimmten Umgebungen schneller sein, ist aber vor allem ein gutes Beispiel für
\IT{unions} und die Tatsache, dass der Exponent in der Form $2^n$ gespeichert wird, sodass ein Eingabewert $n$ zum
Exponenten nach IEEE 754 Standard verschoben wird.
Der Exponent wird dann durch Addition von 0x3f800000 oder 0x3ff0000000000000 korrigiert.

Das gleiche kann ohne Verschiebung durch ein \IT{struct} erreicht werden, aber intern werden stets Schiebebefehle
verwendet.
