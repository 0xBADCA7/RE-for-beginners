\section{\RU{Объединения (union)}\EN{Unions}\DE{Unions}\FR{Unions}}

\EN{\CCpp \IT{union} is mostly used for interpreting a variable (or memory block) of one data type as a variable of another data type.}
\DE{Die \\Cpp \IT{union} wird hauptsächlich verwendet um eine Variable (oder einen Speicherblock) eines Datentyps als
Variable eines anderen Datentyps zu interpretieren.}
\RU{\IT{union} в \CCpp используется в основном для интерпретации переменной (или блока памяти) одного типа как переменной другого типа.}
\FR{Les \IT{unions} en \CCpp sont utilisées principalement pour interprèter une variable
(ou un bloc de mémoire) d'un type de données comme une variable d'un autre type de données.}

% sections
\EN{\subsection{Pseudo-random number generator example}
\label{FPU_PRNG}

If we need float random numbers between 0 and 1, the simplest thing is to use a \ac{PRNG} like
the Mersenne twister. 
It produces random unsigned 32-bit values (in other words, it produces random 32 bits).
Then we can transform this value to \Tfloat and then
divide it by \GTT{RAND\_MAX} (\GTT{0xFFFFFFFF} in our case)---we getting a value in the 0..1 interval.

But as we know, division is slow.
Also, we would like to issue as few FPU operations as possible.
Can we get rid of the division?

\myindex{IEEE 754}

Let's recall what a floating point number consists of: sign bit, significand bits and exponent bits.
We just have to store random bits in all significand bits to get a random float number!

The exponent cannot be zero (the floating number is denormalized in this case),
so we are storing 0b01111111
to exponent---this means that the exponent is 1. 
Then we filling the significand with random bits, set the sign bit to
0 (which means a positive number) and voilà.
The generated numbers is to be between 1 and 2, so we must also subtract 1.

\newcommand{\URLXOR}{\url{http://go.yurichev.com/17308}}

A very simple linear congruential random numbers generator is used in my 
example\footnote{the idea was taken from: \URLXOR}, it produces 32-bit numbers. 
The \ac{PRNG} is initialized with the current time in UNIX timestamp format.

Here we represent the \Tfloat type as an \IT{union}---it is the \CCpp construction that enables us
to interpret a piece of memory as different types.
In our case, we are able to create a variable
of type \IT{union} and then access to it as it is \Tfloat or as it is \IT{uint32\_t}. 
It can be said, it is just a hack. A dirty one.

% WTF?

The integer \ac{PRNG} code is the same as we already considered: \myref{LCG_simple}.
So this code in compiled form is omitted.

\lstinputlisting[style=customc]{patterns/17_unions/FPU_PRNG/FPU_PRNG_EN.cpp}

\subsubsection{x86}

\lstinputlisting[caption=\Optimizing MSVC 2010,style=customasm]{patterns/17_unions/FPU_PRNG/MSVC2010_Ox_Ob0_EN.asm}

Function names are so strange here because this example was compiled as C++ and this is name mangling in C++,
we will talk about it later: \myref{namemangling}.
If we compile this in MSVC 2012, it uses the SIMD instructions for the FPU, read more about it here: \myref{FPU_PRNG_SIMD}.

\subsubsection{MIPS}

\lstinputlisting[caption=\Optimizing GCC 4.4.5,style=customasm]{patterns/17_unions/FPU_PRNG/MIPS_O3_IDA_EN.lst}

There is also an useless \INS{LUI} instruction added for some weird reason.
We considered this artifact earlier: \myref{MIPS_FPU_LUI}.

\subsubsection{ARM (\ARMMode)}

\lstinputlisting[caption=\Optimizing GCC 4.6.3 (IDA),style=customasm]{patterns/17_unions/FPU_PRNG/raspberry_GCC_O3_IDA_EN.lst}

\myindex{objdump}
\myindex{binutils}
\myindex{IDA}

We'll also make a dump in objdump and we'll see that the FPU instructions have different names than in \IDA.
Apparently, IDA and binutils developers used different manuals?
Perhaps it would be good to know both instruction name variants.

\lstinputlisting[caption=\Optimizing GCC 4.6.3 (objdump),style=customasm]{patterns/17_unions/FPU_PRNG/raspberry_GCC_O3_objdump.lst}

The instructions at 0x5c in \TT{float\_rand()} and at 0x38 in \main are (pseudo-)random noise.

}
\RU{\subsection{Пример генератора случайных чисел}
\label{FPU_PRNG}

Если нам нужны случайные значения с плавающей запятой в интервале от 0 до 1, самое простое это взять
\ac{PRNG} вроде Mersenne twister.
Он выдает случайные беззнаковые 32-битные числа (иными словами, он выдает 32 случайных бита).
Затем мы можем преобразовать это число в \Tfloat и затем разделить на \\
\GTT{RAND\_MAX} (\GTT{0xFFFFFFFF} в данном случае) --- полученное число будет в интервале от 0 до 1.

Но как известно, операция деления --- это медленная операция. 
Да и вообще хочется избежать лишних операций с FPU.
Сможем ли мы избежать деления?

\myindex{IEEE 754}
Вспомним состав числа с плавающей запятой: это бит знака, биты мантиссы и биты экспоненты. 
Для получения случайного числа, нам нужно просто заполнить случайными битами все биты мантиссы!

Экспонента не может быть нулевой (иначе число с плавающей точкой будет денормализованным), 
так что в эти биты мы запишем 0b01111111 --- это будет означать что экспонента равна единице.
Далее заполняем мантиссу случайными битами, знак оставляем в виде 0 (что значит наше число положительное), и вуаля.
Генерируемые числа будут в интервале от 1 до 2, так что нам еще нужно будет отнять единицу.

\newcommand{\URLXOR}{\url{http://go.yurichev.com/17308}}

В моем примере\footnote{идея взята здесь: \URLXOR} 
применяется очень простой линейный конгруэнтный генератор случайных чисел, выдающий 32-битные числа.
Генератор инициализируется текущим временем в стиле UNIX.

Далее, тип \Tfloat представляется в виде \IT{union} --- это конструкция \CCpp позволяющая 
интерпретировать часть памяти по-разному. В нашем случае, мы можем создать переменную типа \IT{union} 
и затем обращаться к ней как к \Tfloat или как к \IT{uint32\_t}. Можно сказать, что это хак, причем грязный.

% WTF?

Код целочисленного \ac{PRNG} точно такой же, как мы уже рассматривали ранее: \myref{LCG_simple}.
Так что и в скомпилированном виде этот код будет опущен.

\lstinputlisting[style=customc]{patterns/17_unions/FPU_PRNG/FPU_PRNG_RU.cpp}

\subsubsection{x86}

\lstinputlisting[caption=\Optimizing MSVC 2010,style=customasmx86]{patterns/17_unions/FPU_PRNG/MSVC2010_Ox_Ob0_RU.asm}

Имена функций такие странные, потому что этот пример был скомпилирован как Си++, и это манглинг имен в Си++, мы будем рассматривать это позже: \myref{namemangling}.

Если скомпилировать это в MSVC 2012, компилятор будет использовать SIMD-инструкции для FPU, читайте об этом здесь:

\myref{FPU_PRNG_SIMD}.

\subsubsection{MIPS}

\lstinputlisting[caption=\Optimizing GCC 4.4.5,style=customasmMIPS]{patterns/17_unions/FPU_PRNG/MIPS_O3_IDA_RU.lst}

Здесь снова зачем-то добавлена инструкция \INS{LUI}, которая ничего не делает.
Мы уже рассматривали этот артефакт ранее: \myref{MIPS_FPU_LUI}.

\subsubsection{ARM (\ARMMode)}

\lstinputlisting[caption=\Optimizing GCC 4.6.3 (IDA),style=customasmARM]{patterns/17_unions/FPU_PRNG/raspberry_GCC_O3_IDA_RU.lst}

\myindex{objdump}
\myindex{binutils}
\myindex{IDA}
Мы также сделаем дамп в objdump и увидим, что FPU-инструкции имеют немного другие имена чем в \IDA.
Наверное, разработчики IDA и binutils пользовались разной документацией?
Должно быть, будет полезно знать оба варианта названий инструкций.

\lstinputlisting[caption=\Optimizing GCC 4.6.3 (objdump),style=customasmARM]{patterns/17_unions/FPU_PRNG/raspberry_GCC_O3_objdump.lst}

Инструкции по адресам 0x5c в \TT{float\_rand()} и 0x38 в main() это (псевдо-)случайный мусор.

}
\DE{\subsection{Pseudozufallszahlengenerator Beispiel}
\label{FPU_PRNG}
Wenn wir Zufallszahlen vom Typ \Tfloat zwischen 0 und 1 brauchen, ist die einfachste Möglichkeit einen \ac{PRNG} wie den
Mersenne-Twister zu verwenden.
Er produziert vorzeichenlose 32-Bit-Werte (mit anderen Worten: er erzeugt 32 zufällige Bits).
Diesen Wert können wir in einen \Tfloat umwandeln und dann durch \GTT{RAND\_MAX} (\GTT{0xFFFFFFFF} in unserem Fall)
teilen---wir erhalten einen Wert im Intervall 0..1.

Wie wir jedoch wissen, ist die Division langsam.
Auch würden wir gerne so wenig FPU Operation wie möglich verwenden.
Deshalb fragen wir uns, ob wir die Division loswerden können.

\myindex{IEEE 754}
Erinnern wir uns an den Aufbau einer Fließkommazahl: sie besteht aus einem Vorzeichenbit, Bits im Signifikanden und Bits
im Exponenten.
Wir müssen also lediglich Zufallsbits in allen Bits des Signifikanden speichern um einen zufällige Fließkommazahl zu
erhalten!

Der Exponent kann nicht null sein (in diesem Fall ist die Fließkommazahl denormalisiert); also speichern wir 0b01111111
im Exponenten---das bedeutet, dass der Exponent 1 ist.
Dann füllen wir den Signifikanden mit Zufallsbits, setzen das Vorzeichenbit auf 0 (entspricht einer positiven Zahl) und
voilà.
Die erzeugte Zahl liegt zwischen 1 und 2; wir müssen also am Ende noch 1 abziehen.

\newcommand{\URLXOR}{\url{http://go.yurichev.com/17308}}
Ein sehr einfacher linearer Kongruenzgenerator für Zufallszahlen wird in meinem Beispiel\footnote{die Idee stammt von:
\URLXOR} vorgestellt: er erzeugt 32-Bit-Zahlen.
Der \ac{PRNG} wird mit der aktuellen Zeit um UNIX-Timestamp-Format initialisiert.

Wir stellen hier den Typ \Tfloat als \IT{union} dar--das ist die \CCpp Konstruktion, die es uns ermöglicht, einen
Speicherblock als unterschiedliche Typen aufzufassen.
In unserem Fall sind wir in der Lage eine Variable vom Typ \IT{union} zu erzeugen und dann auf diese wie auf einen
\Tfloat oder \IT{uint32\_t} zuzugreifen.
Man kann sagen, dass es sich dabei um einen Hack, d.h. Trick handelt. Sogar um einen sehr schmutzigen.

% WTF?
Der \ac{PRNG} Code für Integer ist der bereits betrachtete: \myref{LCG_simple}.
Deshalb verzichten wir an dieser Stelle auf ein erneutes Listing des kompilierten Codes.

\lstinputlisting[style=customc]{patterns/17_unions/FPU_PRNG/FPU_PRNG_EN.cpp}

\subsubsection{x86}

\lstinputlisting[caption=\Optimizing MSVC 2010,style=customasmx86]{patterns/17_unions/FPU_PRNG/MSVC2010_Ox_Ob0_DE.asm}
Die Namen der Funktionen sind hier so seltsam, weil das Beispiel als C++ kompiliert wurde, und hier name mangling in C++
vorliegt. Dies werden wir später besprechen: \myref{namemangling}.
Wenn wir das Beispiel mit MSVC 2012 kompilieren, verwendet es SIMD Befehle für die FPU; mehr dazu unter:
\myref{FPU_PRNG_SIMD}.

\subsubsection{MIPS}

\lstinputlisting[caption=\Optimizing GCC 4.4.5,style=customasmMIPS]{patterns/17_unions/FPU_PRNG/MIPS_O3_IDA_DE.lst}
Hier wurde auch ein unnützer \INS{LUI} Befehl aus unerfindlichen Gründen hinzugefügt.
Wir haben solch ein Artefakt schon früher betrachtet: \myref{MIPS_FPU_LUI}.

\subsubsection{ARM (\ARMMode)}

\lstinputlisting[caption=\Optimizing GCC 4.6.3
(IDA),style=customasmARM]{patterns/17_unions/FPU_PRNG/raspberry_GCC_O3_IDA_DE.lst}

\myindex{objdump}
\myindex{binutils}
\myindex{IDA}
Wir ziehen auch einen Dump in objdump und shen, dass die FPU Befehle andere Namen als in \IDA haben.
Offenbar haben die Entwickler von \IDA und binutils unterschiedliche Handbücher verwendet.
Möglicherweise ist es hilfreich, beide Varianten der Befehlsnamen zu kennen.

\lstinputlisting[caption=\Optimizing GCC 4.6.3 (objdump),style=customasmARM]{patterns/17_unions/FPU_PRNG/raspberry_GCC_O3_objdump.lst}

Die Befehle an den Stellen 0x5c in \TT{float\_rand()} und 0x38 in \main sind (Pseudo-)Zufallsrauschen.

}


\EN{\subsection{Calculating machine epsilon}

The machine epsilon is the smallest possible value the \ac{FPU} can work with.
The more bits allocated for floating point number, the smaller the machine epsilon.
It is $2^{-23} = 1.19e-07$ for \Tfloat and $2^{-52} = 2.22e-16$ for \Tdouble.
See also: \href{http://link.yurichev.com/17367}{Wikipedia article}.%
% TODO recheck values

It's interesting, how easy it's to calculate the machine epsilon:

\lstinputlisting[style=customc]{patterns/17_unions/epsilon/float.c}

What we do here is just treat the fraction part of the IEEE 754 number as integer and add 1 to it.
The resulting floating number is equal to $starting\_value+machine\_epsilon$, so we just have to subtract
the starting value (using floating point arithmetic) to measure, what difference one bit reflects
in the single precision (\Tfloat).
The \IT{union} serves here as a way to access IEEE 754 number as a regular integer.
Adding 1 to it in fact adds 1 to the \IT{fraction} part of the number, however, needless to say,
overflow is possible, which will add another 1 to the exponent part.

\subsubsection{x86}

\lstinputlisting[caption=\Optimizing MSVC 2010,style=customasm]{patterns/17_unions/epsilon/float_MSVC_2010_Ox_EN.asm}

The second \INS{FST} instruction is redundant: there is no necessity to store the input value in the same
place (the compiler decided to allocate the $v$ variable at the same point in the local stack as the input 
argument).
Then it is incremented with \INS{INC}, as it is a normal integer variable.
Then it is loaded into the FPU as a 32-bit IEEE 754 number, \INS{FSUBR} does the rest of job and the resulting
value is stored in \TT{ST0}.
The last \INS{FSTP}/\INS{FLD} instruction pair is redundant, but the compiler didn't optimize it out.

\subsubsection{ARM64}

Let's extend our example to 64-bit:

\lstinputlisting[label=machine_epsilon_double_c,style=customc]{patterns/17_unions/epsilon/double.c}

ARM64 has no instruction that can add a number to a FPU D-register, 
so the input value (that came in \TT{D0}) is first copied into \ac{GPR},
incremented, copied to FPU register \TT{D1}, and then subtraction occurs.

\lstinputlisting[caption=\Optimizing GCC 4.9 ARM64,style=customasm]{patterns/17_unions/epsilon/double_GCC49_ARM64_O3_EN.s}

See also this example compiled for x64 with SIMD instructions: \myref{machine_epsilon_x64_and_SIMD}.

\subsubsection{MIPS}

\myindex{MIPS!\Instructions!MTC1}

The new instruction here is \INS{MTC1} (\q{Move To Coprocessor 1}), it just transfers data from \ac{GPR} to the FPU's registers.

\lstinputlisting[caption=\Optimizing GCC 4.4.5 (IDA)]{patterns/17_unions/epsilon/MIPS_O3_IDA.lst}

\subsubsection{\Conclusion}

It's hard to say whether someone may need this trickery in real-world code, 
but as was mentioned many times in this book, this example serves well 
for explaining the IEEE 754 format and \IT{union}s in \CCpp.

}
\RU{\subsection{Вычисление машинного эпсилона}

Машинный эпсилон --- это самая маленькая гранула, с которой может работать \ac{FPU} 
\footnote{В русскоязычной литературе встречается также термин \q{машинный ноль}.}.
Чем больше бит выделено для числа с плавающей точкой, тем меньше машинный эпсилон.
Это $2^{-23} = 1.19e-07$ для \Tfloat и $2^{-52} = 2.22e-16$ для \Tdouble.
См.также: \href{http://link.yurichev.com/17368}{статью в Wikipedia}.
% TODO recheck values

Любопытно, что вычислить машинный эпсилон очень легко:

\lstinputlisting[style=customc]{patterns/17_unions/epsilon/float.c}

Что мы здесь делаем это обходимся с мантиссой числа в формате IEEE 754 как с целочисленным числом и прибавляем
единицу к нему.
Итоговое число с плавающей точкой будет равно $starting\_value+machine\_epsilon$, так что нам
нужно просто вычесть изначальное значение (используя арифметику с плавающей точкой) чтобы измерить, 
какое число отражает один бит в одинарной точности (\Tfloat).
\IT{union} здесь нужен чтобы мы могли обращаться к числу в формате IEEE 754 как к обычному целочисленному.
Прибавление 1 к нему на самом деле прибавляет 1 к \IT{мантиссе} числа, хотя, нужно сказать,
переполнение также возможно, что приведет к прибавлению единицы к экспоненте.

\subsubsection{x86}

\lstinputlisting[caption=\Optimizing MSVC 2010,style=customasm]{patterns/17_unions/epsilon/float_MSVC_2010_Ox_RU.asm}

Вторая инструкция \INS{FST} избыточная: нет необходимости сохранять входное значение в этом же месте
(компилятор решил выделить переменную $v$ в том же месте локального стека, где находится и 
входной аргумент).
Далее оно инкрементируется при помощи \INS{INC}, как если это обычная целочисленная переменная.
Затем оно загружается в FPU как если это 32-битное число в формате IEEE 754, \INS{FSUBR} делает остальную
часть работы и результат в \TT{ST0}.
Последняя пара инструкций \INS{FSTP}/\INS{FLD} избыточна, но компилятор не соптимизировал её.

\subsubsection{ARM64}

Расширим этот пример до 64-бит:

\lstinputlisting[label=machine_epsilon_double_c,style=customc]{patterns/17_unions/epsilon/double.c}

В ARM64 нет инструкции для добавления числа к D-регистру в FPU, так что входное значение
(пришедшее в D0) в начале копируется в \ac{GPR},
инкрементируется, копируется в регистр FPU \TT{D1}, затем происходит вычитание.

\lstinputlisting[caption=\Optimizing GCC 4.9 ARM64,style=customasm]{patterns/17_unions/epsilon/double_GCC49_ARM64_O3_RU.s}

Смотрите также этот пример скомпилированный под x64 с SIMD-инструкциями: \myref{machine_epsilon_x64_and_SIMD}.

\subsubsection{MIPS}

\myindex{MIPS!\Instructions!MTC1}
Новая для нас здесь инструкция это \INS{MTC1} (\q{Move To Coprocessor 1}), она просто переносит данные из \ac{GPR} в регистры FPU.

\lstinputlisting[caption=\Optimizing GCC 4.4.5 (IDA)]{patterns/17_unions/epsilon/MIPS_O3_IDA.lst}

\subsubsection{\Conclusion}

Трудно сказать, понадобится ли кому-то такая эквилибристика в реальном коде,
но как уже было упомянуто много раз в этой книге, этот пример хорошо подходит для объяснения формата
IEEE 754 и \IT{union} в \CCpp.

}
\DE{\subsection{Berechnung der Maschinengenauigkeit}
Die Maschinengenauigkeit ist der kleinstmögliche Wert, mit dem die \ac{FPU} arbeiten kann.
Je mehr Bits für eine Fließkommazahl verwendet werden, desto kleiner ist die Maschinengenauigkeit.
Sie beträgt $2^{-23} = 1.19e-07$ für \Tfloat und $2^{-52} = 2.22e-16$ für \Tdouble.
Siehe auch:\href{http://link.yurichev.com/17367}{Wikipedia article}.

% TODO recheck values
Interessant ist, wie einfach die Maschinengenauigkeit berechnet werden kann:

\lstinputlisting[style=customc]{patterns/17_unions/epsilon/float.c}
Was wir hier machen ist, den Bruch in der IEEE 754 Zahl als Integer zu behandeln und 1 hinzuzuaddieren.
Die resultierende Fließkommazahl ist gleich $starting\_value+machine\_epsilon$, sodass wir nur den Startwert (in der
Fließkommaarithmetik) abziehen müssen um zu messen, welchen Unterschied ein Bit in der einfachen Genauigkeit (\Tfloat)
ausmacht.
Die \IT{union} dient hier als Mittel, um auf die IEEE 754 Zahl als regulären Integer zuzugreifen.
Die Addition von 1 entspricht hier einer Addition von 1 zum Bruch in der Zahl, aber natürlich kann hier ein Overflow
auftreten, was eine Addition von 1 zum Exponenten nach sich zieht.

\subsubsection{x86}

\lstinputlisting[caption=\Optimizing MSVC 2010,style=customasmx86]{patterns/17_unions/epsilon/float_MSVC_2010_Ox_DE.asm}
Der zweite \INS{FST} Befehl ist redundant: es besteht keine Notwendigkeit, den Eingabewert an derselben Stelle zu
speichern (der Compiler hat entschieden, die Variable $v$ an der gleichen Stelle im lokalen Stack anzulegen wie den
Eingabewert).
Der Wert wird mit \INS{INC} um 1 erhöht als wäre es einen normale Integervariable.
Danach wir der Wert als 32-Bit IEEE 754 Zahl in die FPU geladen, \INS{FSUBR} erledigt den Rest und das Ergebnis wird in
\TT{ST0} gespeichert.
Das letzte \INS{FSTP}/\INS{FLD} Befehlspaar ist redundant, aber der Compiler hat es hier nicht wegoptimiert.

\subsubsection{ARM64}

Erweitern wir unser Beispiel auf 64 Bit:

\lstinputlisting[label=machine_epsilon_double_c,style=customc]{patterns/17_unions/epsilon/double.c}
ARM64 kennt keinen Befehl, der eine Zahl zu einem FPU D-Register addieren kann, sodass der Eingabewert (in \TT{D0})
zunächst nach \ac{GPR} kopiert wird, dann inkrementiert wird und schließlich in das FPU Register \TT{D1} kopiert wird,
bevor die Subtraktion ausgeführt wird.

\lstinputlisting[caption=\Optimizing GCC 4.9
ARM64,style=customasmARM]{patterns/17_unions/epsilon/double_GCC49_ARM64_O3_DE.s}
Schauen Sie sich dieses Beispiel auch für x64 kompiliert mit SIMB Befehlen an:\myref{machine_epsilon_x64_and_SIMD}.

\subsubsection{MIPS}

\myindex{MIPS!\Instructions!MTC1}
Der neue Befehl ist hier \INS{MTC1} (\q{Move To Coprocessor 1}): er überträgt Daten von \ac{GPR} in die Register der
FPU.

\lstinputlisting[caption=\Optimizing GCC 4.4.5 (IDA),style=customasmMIPS]{patterns/17_unions/epsilon/MIPS_O3_IDA.lst}

\subsubsection{\Conclusion}
Es ist schwer zu sagen, ob jemand eine solche Trickserei in echtem Produktivcode benötigt, aber wie bereits mehrfach
erwähnt, ist dieses Beispiel gut geeignet, um das IEEE 754 Format und \IT{unions} in \CCpp zu erklären.

}

\EN{\section{FSCALE replacement}
\myindex{x86!\Instructions!FSCALE}

Agner Fog in his \IT{Optimizing subroutines in assembly language / An optimization guide for x86 platforms} work
\footnote{\url{http://www.agner.org/optimize/optimizing_assembly.pdf}} states that \INS{FSCALE} \ac{FPU} instruction
(calculating $2^n$) may be slow on many CPUs, and he offers faster replacement.

Here is my translation of his assembly code to \CCpp:

\lstinputlisting[style=customc]{patterns/17_unions/FSCALE.c}

\INS{FSCALE} instruction may be faster in your environment, but still, it's a good example of \IT{union}'s and the fact
that exponent is stored in $2^n$ form,
so an input $n$ value is shifted to the exponent in IEEE 754 encoded number.
Then exponent is then corrected with addition of 0x3f800000 or 0x3ff0000000000000.

The same can be done without shift using \IT{struct}, but internally, shift operations still occurred.

}
\DE{\section{FSCALE Ersatz}
\myindex{x86!\Instructions!FSCALE}
Agner Fog schreibt in seiner Abhandlung \IT{Optimizing subroutines in assembly language / An optimization guide for x86
platforms} \footnote{\url{http://www.agner.org/optimize/optimizing_assembly.pdf}} , dass der Befehl \INS{FSCALE}
\ac{FPU} (der $2^n$ berechnet) auf vielen CPUs langsam ist und bietet einen schnelleren Ersatz an.

Hier ist meine Übersetzung von seinem Assemblercode in \CCpp:

\lstinputlisting[style=customc]{patterns/17_unions/FSCALE.c}
Der Befehl \INS{FSCALE} kann zwar in bestimmten Umgebungen schneller sein, ist aber vor allem ein gutes Beispiel für
\IT{unions} und die Tatsache, dass der Exponent in der Form $2^n$ gespeichert wird, sodass ein Eingabewert $n$ zum
Exponenten nach IEEE 754 Standard verschoben wird.
Der Exponent wird dann durch Addition von 0x3f800000 oder 0x3ff0000000000000 korrigiert.

Das gleiche kann ohne Verschiebung durch ein \IT{struct} erreicht werden, aber intern werden stets Schiebebefehle
verwendet.
}
\FR{\section{Remplacement de FSCALE}
\myindex{x86!\Instructions!FSCALE}

Agner Fog dans son travail\footnote{\url{http://www.agner.org/optimize/optimizing_assembly.pdf}}
\IT{Optimizing subroutines in assembly language / An optimization guide for x86 platforms}
indique que l'instruction \ac{FPU} \INS{FSCALE} (qui calcule $2^n$) peut être lente
sur de nombreux CPUs, et propose un remplacement plus rapide.

Voici ma conversion de son code assembleur en \CCpp:

\lstinputlisting[style=customc]{patterns/17_unions/FSCALE.c}

L'instruction\INS{FSCALE} peut être plus rapide dans votre environnement, mais néanmoins,
c'est un bon exemple d'\IT{union} et du fait que l'exposant est stocké sous la forme
$2^n$, donc une valeur $n$ en entrée est décalée à l'exposant dans le nombre encodé
en IEEE 754.
Ensuite, l'exposant est corrigé avec l'ajout de 0x3f800000 ou de 0x3ff0000000000000.

La même chose peut être faite sans décalage utilisant \IT{struct}, mais en interne,
l'opération de décalage aura toujours lieu.

}

\subsection{\RU{Быстрое вычисление квадратного корня}\EN{Fast square root calculation}\DE{Schnelle Berechnung der
Quadratwurzel}}

\RU{Вот где еще можно на практике применить трактовку типа \Tfloat как целочисленного, это быстрое вычисление квадратного корня.}%
\EN{Another well-known algorithm where \Tfloat is interpreted as integer is fast calculation of square root.}
\DE{Ein anderer bekannter Algorithmus, in dem \Tfloat als \Tint interpretiert wird, ist die schnelle Berechnung einer
Quadratwurzel.}
\FR{Un autre algorithme connu oú un \Tfloat est interprèté comme un entier est celui
de calcul rapide de racine carrée.}

\begin{lstlisting}[caption=\DE{Quellcode stammt aus der Wikipedia}\EN{The source code is taken from
Wikipedia}\RU{Исходный код взят из Wikipedia}\FR{Le code source provient de Wikipedia}:
\url{http://go.yurichev.com/17364},style=customc] /* Assumes that float is in the IEEE 754 single precision floating point format
 * and that int is 32 bits. */
float sqrt_approx(float z)
{
    int val_int = *(int*)&z; /* Same bits, but as an int */
    /*
     * To justify the following code, prove that
     *
     * ((((val_int / 2^m) - b) / 2) + b) * 2^m = ((val_int - 2^m) / 2) + ((b + 1) / 2) * 2^m)
     *
     * where
     *
     * b = exponent bias
     * m = number of mantissa bits
     *
     * .
     */
 
    val_int -= 1 << 23; /* Subtract 2^m. */
    val_int >>= 1; /* Divide by 2. */
    val_int += 1 << 29; /* Add ((b + 1) / 2) * 2^m. */
 
    return *(float*)&val_int; /* Interpret again as float */
}
\end{lstlisting}

\ifdefined\RUSSIAN
В качестве упражнения, вы можете попробовать скомпилировать эту функцию и разобраться, как она работает. \\
\\
Имеется также известный алгоритм быстрого вычисления $\frac{1}{\sqrt{x}}$.
\myindex{Quake III Arena}
Алгоритм стал известным, вероятно потому, что был применен в Quake III Arena.

Описание алгоритма есть в Wikipedia: \url{http://go.yurichev.com/17361}.
\fi % RUSSIAN

\ifdefined\ENGLISH
As an exercise, you can try to compile this function and to understand, how it works. \\
\\
There is also well-known algorithm of fast calculation of $\frac{1}{\sqrt{x}}$.
\myindex{Quake III Arena}
Algorithm became popular, supposedly, because it was used in Quake III Arena.

Algorithm description can be found in Wikipedia: \url{http://go.yurichev.com/17360}.
\fi % ENGLISH

\ifdefined\GERMAN
Versuchen Sie als Übung, diese Funktion zu kompilieren und zu verstehen wie sie funktioniert.\\\\
Es gibt auch einen bekannten Algorithmus zur schnellen Berechnung von $\frac{1}{\sqrt{x}}$.
\myindex{Quake III Arena}
Der Algorithmus wurde vermutlich so populär, weil er in Quake III Arena verwendet wurde.
Eine Beschreibung des Algorithmus' findet man bei Wikipedia: \url{http://go.yurichev.com/17360}.
\fi % GERMAN

\ifdefined\FRENCH
Á titre d'exercice, vous pouvez assayez de compiler cette fonction et de comprendre
comme elle fonctionne.\\
\\
C'est un algorithme connu de calcul rapide de $\frac{1}{\sqrt{x}}$.
\myindex{Quake III Arena}
L'algorithme devînt connu, supposément, car il a été utilisé dans Quake III Arena.

La description de l'alogrithme peut être trouvée sur Wikipédia: \url{http://go.yurichev.com/17360}.
\fi % FRENCH

