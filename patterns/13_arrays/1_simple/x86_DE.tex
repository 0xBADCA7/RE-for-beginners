\subsubsection{x86}

\myparagraph{MSVC}

Kompilieren wir das Beispiel:

\lstinputlisting[caption=MSVC 2008,style=customasmx86]{patterns/13_arrays/1_simple/simple_msvc.asm}

\myindex{x86!\Instructions!SHL}
Soweit nichts Außergewöhnliches, nur zwei Schleifen: die erste füllt mit Werten auf und die zweite gibt Werte aus.
Der Befehl \TT{shl ecx, 1} wird für die Multiplikation mit 2 in \ECX verwendet; mehr dazu unten~\myref{SHR}.

Auf dem Stack werden 80 Bytes für das Array reserviert: 20 Elemente von je 4 Byte.

\clearpage
Untersuchen wir dieses Beispiel in \olly.
\myindex{\olly}

Wir erkennen wie das Array befüllt wird:

jedes Element ist ein 32-Bit-Wort vom Typ \Tint und der Wert ist der Index multipliziert mit 2:

\begin{figure}[H]
\centering
\myincludegraphics{patterns/13_arrays/1_simple/olly.png}
\caption{\olly: nach dem Füllen des Arrays}
\label{fig:array_simple_olly}
\end{figure}
Da sich dieses Array auf dem Stack befindet, finden wir dort alle seine 20 Elemente.

\myparagraph{GCC}

Hier ist was GCC 4.4.1 erzeugt:

\lstinputlisting[caption=GCC 4.4.1,style=customasmx86]{patterns/13_arrays/1_simple/simple_gcc.asm}
Die Variable $a$ ist übrigens vom Typ \IT{int*} (Pointer auf \Tint{})--man kann einen Pointer auf ein Array an eine
andere Funktion übergeben, aber es ist richtiger zu sagen, dass der Pointer auf das erste Element des Arrays übergeben
wird. (Die Adressen der übrigen Elemente werden in bekannter Weise berechnet.)

Wenn man diesen Pointer mittels \IT{a[idx]} indiziert, wird \IT{idx} zum Pointer addiert und das dort abgelegte Element
(auf das der berechnete Pointer zeigt) wird zurückgegeben.

Ein interessantes Beispiel: ein String wie \IT{\q{string}} ist ein Array von Chars und hat den Typ \IT{const
char[]}.

Auch auf diesen Pointer kann ein Index angewendet werden.

Das ist der Grund warum es es möglich ist, Dinge wie \TT{\q{string}[i]} zu schreiben--es handelt sich dabei um einen
korrekten \CCpp Ausdruck!

