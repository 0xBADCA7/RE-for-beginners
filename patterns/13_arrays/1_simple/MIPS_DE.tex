\subsubsection{MIPS}
% FIXME better start at non-optimizing version?
Die Funktion verwendet eine Menge S-Register, die gesichert werden müssen. Das ist der Grund dafür, dass die Werte im
Funktionsprolog gespeichert und im Funktionsepilog wiederhergestellt werden.

\lstinputlisting[caption=\Optimizing GCC 4.4.5
(IDA),style=customasmMIPS]{patterns/13_arrays/1_simple/MIPS_O3_IDA_DE.lst}
Interessant: es gibt zwei Schleifen und die erste benötigt $i$ nicht; sie benötigt nur $i\cdot 2$ (erhöht um 2 bei
jedem Iterationsschritt) und die Adresse im Speicher (erhöht um 4 bei jedem Iterationsschritt).

Wir sehen hier also zwei Variablen: eine (in \$V0), die jedes Mal um 2 erhöht wird, und eine andere (in\$V1), die um 4
erhöht wird.

Die zweite Schleife ist der Ort, an dem \printf aufgerufen wird und dem Benutzer den Wert von $i$ zurückliefert, es gibt
also eine Variable die in \$S0 inkrementiert wird und eine Speicheradresse in \$S1, die jedes Mal um 4 erhöht wird.

Das erinnert uns an die Optimierung von Schleifen, die wir früher betrachtet haben: \myref{loop_iterators}.

Das Ziel der Optimierung ist es, die Multiplikationen loszuwerden.