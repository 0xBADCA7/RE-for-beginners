\subsubsection{ARM}

\myparagraph{\NonOptimizingKeilVI (\ARMMode)}

\lstinputlisting[style=customasmARM]{patterns/13_arrays/1_simple/simple_Keil_ARM_O0_RU.asm}

Тип \Tint требует 32 бита для хранения (или 4 байта),

так что для хранения 20 переменных типа \Tint, нужно 80 (\TT{0x50}) байт.

Поэтому инструкция \INS{SUB SP, SP, \#0x50} 
в прологе функции выделяет в локальном стеке под массив именно столько места.

И в первом и во втором цикле итератор цикла \var{i} будет постоянно находится в регистре \Reg{4}.

\myindex{ARM!Optional operators!LSL}
Число, которое нужно записать в массив, вычисляется так: $i*2$, и это эквивалентно 
сдвигу на 1 бит влево,\\
так что инструкция \INS{MOV R0, R4,LSL\#1} делает это.

\myindex{ARM!\Instructions!STR}
\INS{STR R0, [SP,R4,LSL\#2]} записывает содержимое \Reg{0} в массив.
Указатель на элемент массива вычисляется так: \ac{SP} указывает на начало массива, \Reg{4} это $i$.

Так что сдвигаем $i$ на 2 бита влево, что эквивалентно умножению на 4 
(ведь каждый элемент массива занимает 4 байта) и прибавляем это к адресу начала массива.

\myindex{ARM!\Instructions!LDR}
Во втором цикле используется обратная инструкция\\
\INS{LDR R2, [SP,R4,LSL\#2]}.
Она загружает из массива нужное значение и указатель на него вычисляется точно так же.

\myparagraph{\OptimizingKeilVI (\ThumbMode)}

\lstinputlisting[style=customasmARM]{patterns/13_arrays/1_simple/simple_Keil_thumb_O3_RU.asm}

Код для Thumb очень похожий.
\myindex{ARM!\Instructions!LSLS}
В Thumb имеются отдельные инструкции для битовых сдвигов (как \TT{LSLS}), 
вычисляющие и число для записи в массив и адрес каждого элемента массива.

Компилятор почему-то выделил в локальном стеке немного больше места, 
однако последние 4 байта не используются.

\myparagraph{\NonOptimizing GCC 4.9.1 (ARM64)}

\lstinputlisting[caption=\NonOptimizing GCC 4.9.1 (ARM64),style=customasmARM]{patterns/13_arrays/1_simple/ARM64_GCC491_O0_RU.s}

