\subsubsection{ARM}

\myparagraph{\NonOptimizingKeilVI (\ARMMode)}

\lstinputlisting[style=customasmARM]{patterns/13_arrays/1_simple/simple_Keil_ARM_O0_FR.asm}

Le type \Tint nécessite 32 bits pour le stockage (ou 4 octets).

donc pour stocker 20 variables, \Tint 80 (\TT{0x50}) octets sont nécessaires.

C'est pourquoi l'instruction \INS{SUB SP, SP, \#0x50} dans le prologue de la fonction
alloue exactement cet espace sur la pile.

Dans la première et la seconde boucle, la variable de boucle \var{i} se trouve dans
le registre \Reg{4}.

\myindex{ARM!Optional operators!LSL}

Le nombre qui doit être écrit dans le tableau est calculé comme $i*2$, ce qui est
effectivement équivalent à décaler d'un bit vers la gauche,
ce que fait l'instruction \INS{MOV R0, R4,LSL\#1}.

\myindex{ARM!\Instructions!STR}
\INS{STR R0, [SP,R4,LSL\#2]} écrit le contenu de \Reg{0} dans le tableau.

Voici comment le pointeur sur un élément du tableau est calculé: \ac{SP} pointe sur
le début du tableau, \Reg{4} est $i$.

Donc décaler $i$ de 2 bits vers la gauche est effectivement équivalent à la multiplication
par 4 (puisque chaque élément du tableau a une taille de 4 octets) et ensuite on
l'ajoute à l'adresse du début du tableau.

\myindex{ARM!\Instructions!LDR}

La seconde boucle a l'instruction inverse \INS{LDR R2, [SP,R4,LSL\#2]}. Elle charge
la valeur du tableau dont nous avons besoin, et le pointeur est calculé de même.

\myparagraph{\OptimizingKeilVI (\ThumbMode)}

\lstinputlisting[style=customasmARM]{patterns/13_arrays/1_simple/simple_Keil_thumb_O3_FR.asm}

Le code Thumb est très similaire.
\myindex{ARM!\Instructions!LSLS}

Le mode Thumb a des instructions spéciales pour le décalage (comme \TT{LSLS}), qui
calculent la valeur à écrire dans le tableau et l'adresse de chaque élément dans le
tableau.

Le compilateur alloue légèrement plus d'espace sur la pile locale, cependant, les
4 derniers octets ne sont pas utilisés.

\myparagraph{GCC 4.9.1 \NonOptimizing (ARM64)}

\lstinputlisting[caption=GCC 4.9.1 \NonOptimizing (ARM64),style=customasmARM]{patterns/13_arrays/1_simple/ARM64_GCC491_O0_FR.s}

