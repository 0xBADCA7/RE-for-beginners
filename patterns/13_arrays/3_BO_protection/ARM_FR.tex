\subsubsection{\OptimizingXcodeIV (\ThumbTwoMode)}

Reprenons notre exemple de simple tableau (\myref{arrays_simple}),

à nouveau, nous pouvons voir comment LLVM teste si le \q{canari} est correct:

% TODO shorten the listing a bit? is full display of unrolled loop necessary?
\lstinputlisting[style=customasmARM]{patterns/13_arrays/3_BO_protection/simple_Xcode_thumb_O3_FR.asm}

\myindex{Unrolled loop}

Tout d'abord, on voit que LLVM a \q{déroulé} la boucle et que toutes les valeurs
sont écrites une par une, pré-calculée, car LLVM a conclu que c'est plus rapide.
Á propos, des instructions en mode ARM peuvent aider à rendre cela encore plus rapide,
et les trouver peut être un exercice pour vous.

A la fin de la fonction, nous voyons la comparaison des \q{canaris}---celui dans
la pile locale et le correct.
\myindex{ARM!\Instructions!IT}

S'ils sont égaux, un bloc de 4 instructions est effectué par \INS{ITTTT EQ}, qui
contient l'écriture de 0 dans \Reg{0}, l'épilogue de la fonction et la sortie.
Si les \q{canaris} ne sont pas égaux, le bloc est passé,\\
et la fonction saute en \TT{\_\_\_stack\_chk\_fail}, qui, peut-être, stope l'exécution.
% TODO1 illustrate this!
