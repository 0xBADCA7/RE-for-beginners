\subsubsection{\OptimizingXcodeIV (\ThumbTwoMode)}
Betrachten wir nochmals unser einfaches Arraybeispiel(\myref{arrays_simple}), erkennen wir nun wie LLVM die Korrektheit
des \q{canary} überprüft:

% TODO shorten the listing a bit? is full display of unrolled loop necessary?
\lstinputlisting[style=customasmARM]{patterns/13_arrays/3_BO_protection/simple_Xcode_thumb_O3_DE.asm}

\myindex{Unrolled loop}
Zunächst hat LLVM die Schleife entwickelt und alle Werte werden nacheinander vorberechnet in ein Array geschrieben, da
LLVM dies für schneller hält. 
Befehle im ARM mode können helfen, das noch schneller auszuführen und dies herauszufinden könnte Ihre Aufgabe sein.

Am Ende der Funktion sehen wir den Vergleich der beiden \q{canaries}--dem im lokalen Stack und dem richtigen, auf den
\Reg{8} zeigt.
\myindex{ARM!\Instructions!IT}
Wenn sie gleich sind wird durch \INS{ITTTT EQ} ein Block aus vier Befehlen ausgeführt, der 0 nach \Reg{0} schreibt, den
Funktionsepilog durchführt und dann beendet.
Wenn die \q{canaries} ungleich sind, wird der Block übersprungen und es wird zu \\
\TT{\_\_\_stack\_chk\_fail} gesprungen und die Ausführung wird angehalten.
