\subsubsection{Exemple de tableau à trois dimensions}

C'est la même chose pour des tableaux multidimensionnels.

Nous allons travailler avec des tableaux de type \Tint: chaque élément nécessite
4 octets en mémoire.

Voyons ceci:

\lstinputlisting[caption=simple exemple,style=customc]{patterns/13_arrays/5_multidimensional/multi.c}

\myparagraph{x86}

Nous obtenons (MSVC 2010):

\lstinputlisting[caption=MSVC 2010,style=customasmx86]{patterns/13_arrays/5_multidimensional/multi_msvc_FR.asm}

Rien de particulier. Pour le calcul de l'index, trois arguments en entrée sont utilisés
dans la formule $address=600 \cdot 4 \cdot x + 30 \cdot 4 \cdot y + 4z$, pour représenter
le tableau comme multidimensionnel.
N'oubliez pas que le type \Tint est 32-bit (4 octets), donc tous les coéfficients
doivent être multipliés par 4.

\lstinputlisting[caption=GCC 4.4.1,style=customasmx86]{patterns/13_arrays/5_multidimensional/multi_gcc_FR.asm}

Le compilateur GCC fait cela différement.

Pour une des opérations du calcul ($30y$), GCC produit un code sans instruction de
multiplication. Voici comment il fait:
$(y+y) \ll 4 - (y+y) = (2y) \ll 4 - 2y = 2 \cdot 16 \cdot y - 2y = 32y - 2y = 30y$. 
Ainsi, pour le calcul de $30y$, seulement une addition, un décalage de bit et une
soustraction sont utilisés.
Ceci fonctionne plus vite.

\myparagraph{ARM + \NonOptimizingXcodeIV (\ThumbMode)}

\lstinputlisting[caption=\NonOptimizingXcodeIV (\ThumbMode),style=customasmARM]{patterns/13_arrays/5_multidimensional/multi_Xcode_thumb_O0_FR.asm}

LLVM \NonOptimizing sauve toutes les variables dans la pile locale, ce qui est redondant.

L'adresse de l'élément du tableau est calculée par la formule vue précédemment.

\myparagraph{ARM + \OptimizingXcodeIV (\ThumbMode)}

\lstinputlisting[caption=\OptimizingXcodeIV (\ThumbMode),style=customasmARM]{patterns/13_arrays/5_multidimensional/multi_Xcode_thumb_O3_FR.asm}

L'astuce de remplacer la multiplication par des décalage, addition et soustraction
que nous avons déjà vue est aussi utilisée ici.

\myindex{ARM!\Instructions!RSB}
\myindex{ARM!\Instructions!SUB}
Ici, nous voyons aussi une nouvelle instruction: \RSB (\IT{Reverse Subtract}).

Elle fonctionne comme \SUB, mais échange ses opérandes l'un avec l'autre avant l'exécution.
Pourquoi?
\myindex{ARM!Optional operators!LSL}
\SUB et \RSB sont des instructions auxquelles un coéfficient de décalage peut être
appliqué au second opérande: (\INS{LSL\#4}).

Mais ce coéfficient ne peut être appliqué qu'au second opérande.

C'est bien pour des opérations commutatives comme l'addition ou la multiplication
(les opérandes peuvent être échangés sans changer le résultat).

Mais la soustraction est une opération non commutative, donc \RSB existe pour ces
cas.

\myparagraph{MIPS}

\myindex{MIPS!Global Pointer}
Mon exemple est minuscule, donc le compilateur GCC a décidé de mettre le tableau
$a$ dans la zone de 64KiB adressable par le Global Pointer.

\lstinputlisting[caption=GCC 4.4.5 \Optimizing (IDA),style=customasmMIPS]{patterns/13_arrays/5_multidimensional/multi_MIPS_O3_IDA_FR.lst}

