\subsection{Tableau de pointeurs sur des chaînes}
\label{array_of_pointers_to_strings}

Voici un exemple de tableau de pointeurs.\footnote{NDT: attention à l'encodage des
fichiers, en ASCII ou en ISO-8859, un caractère occupe un octet, alors qu'en UTF-8,
notamment, il peut en occuper plusieurs. Par exemple, 'û' est codé \$fb (1 octet)
en ISO-8859 et \$c3\$bb (2 octets) en UTF-8. J'ai donc volontairement mis des caractères
non accentués dans le code.}

\lstinputlisting[caption=Get month name,label=get_month1,style=customc]{patterns/13_arrays/45_month_1D/month1_FR.c}

\subsubsection{x64}

\lstinputlisting[caption=MSVC 2013 \Optimizing x64,style=customasmx86]{patterns/13_arrays/45_month_1D/month1_MSVC_2013_x64_Ox.asm}

Le code est très simple:

\begin{itemize}

\item
\myindex{x86!\Instructions!MOVSXD}

La première instruction \INS{MOVSXD} copie une valeur 32-bit depuis \ECX (oú l'argument
$month$ est passé) dans \RAX avec extension du signe (car l'argument $month$ est
de type \Tint).

La raison de l'extension du signe est que cette valeur 32-bit va être utilisée dans
des calculs avec d'autres valeurs 64-bit.

C'est pourquoi il doit être étendu à 64-bit\footnote{C'est parfois bizarre, mais des
indices négatifs de tableau peuvent être passés par $month$ (les indices négatifs
de tableaux sont expliqués plus loin: \myref{negative_array_indices}).
Et si cela arrive, la valeur entrée négative de type \Tint est étendue correctement
et l'élément correspondant avant le tableau est sélectionné.
Ca ne fonctionnera pas correctement sans l'extension du signe.}.

\item
Ensuite l'adresse du pointeur de la table est chargée dans \RCX.

\item
Enfin, la valeur d'entrée ($month$) est multipliée par 8 et ajoutée à l'adresse.
Effectivement: nous sommes dans un environnement 64-bit et toutes les adresses (ou
pointeurs) nécessitent exactement 64 bits (ou 8 octets) pour être stockées.
C'est pourquoi chaque élément de la table a une taille de 8 octets.
Et c'est pourquoi pour prendre un élément spécifique, $month*8$ octets doivent être
passés depuis le début.
C'est ce que fait \MOV.
De plus, cette instruction charge également l'élément à cette adresse.
Pour 1, l'élément sera un pointeur sur la chaîne qui contient \q{février}, etc.

\end{itemize}

GCC 4.9 \Optimizing peut faire encore mieux\footnote{\q{0+} a été laissé dans le
listing car la sortie de l'assembleur GCC n'est pas assez soignée pour l'éliminer.
C'est un \IT{déplacement}, et il vaut zéro ici.}:

\begin{lstlisting}[caption=GCC 4.9 \Optimizing x64,style=customasmx86]
	movsx	rdi, edi
	mov	rax, QWORD PTR month1[0+rdi*8]
	ret
\end{lstlisting}

\myparagraph{MSVC 32-bit}

Compilons-le aussi avec le compilateur MSVC 32-bit:

\lstinputlisting[caption=MSVC 2013 \Optimizing x86,style=customasmx86]{patterns/13_arrays/45_month_1D/month1_MSVC_2013_x86_Ox.asm}

La valeur en entrée n'a pas besoin  d'être étendue sur 64-bit, donc elle est utilisée
telle quelle.

Et elle est multipliée par 4, car les éléments de la table sont larges de 32-bit
(ou 4 octets).

% FIXME1 move to another file
\subsubsection{ARM 32-bit}

\myparagraph{ARM en mode ARM}

\lstinputlisting[caption=\OptimizingKeilVI (\ARMMode),style=customasmARM]{patterns/13_arrays/45_month_1D/month1_Keil_ARM_O3.s}

% TODO Fix R1s

L'adresse de la table est chargée en R1.
\myindex{ARM!\Instructions!LDR}

Tout le reste est effectué en utilisant juste une instruction \LDR.

Puis la valeur en entrée est décalée de 2 vers la gauche (ce qui est la même chose
que multiplier par 4), puis ajoutée à R1 (oú se trouve l'adresse de la table) et
enfin un élément de la table est chargé depuis cette adresse.

L'élément 32-bit de la table est chargé dans R1 depuis la table.

\myparagraph{ARM en mode Thumb}

Le code est essentiellement le même, mais moins dense, car le suffixe \LSL ne peut
pas être spécifié dans l'instruction \LDR ici:

\begin{lstlisting}[style=customasmARM]
get_month1 PROC
        LSLS     r0,r0,#2
        LDR      r1,|L0.64|
        LDR      r0,[r1,r0]
        BX       lr
        ENDP
\end{lstlisting}

\subsubsection{ARM64}

\lstinputlisting[caption=GCC 4.9 \Optimizing ARM64,style=customasmARM]{patterns/13_arrays/45_month_1D/month1_GCC49_ARM64_O3.s}

\myindex{ARM!\Instructions!ADRP/ADD pair}

L'adresse de la table est chargée dans X1 en utilisant la paire \ADRP/\ADD.

Puis l'élément correspondant est choisi dans la table en utilisant seulement un \LDR,
qui prend W0 (le registre oú l'argument d'entrée $month$ se trouve), le décale de
3 bits vers la gauche (ce qui est la même chose que de le multiplier par 8), étend
son signe (c'est ce que le suffixe \q{sxtw} implique) et l'ajoute à X0.
Enfin la valeur 64-bit est chargée depuis la table dans X0.

\subsubsection{MIPS}

\lstinputlisting[caption=GCC 4.4.5 \Optimizing (IDA),style=customasmMIPS]{patterns/13_arrays/45_month_1D/MIPS_O3_IDA_FR.lst}

\subsubsection{Débordement de tableau}

Notre fonction accepte des valeurs dans l'intervalle 0..11, mais que se passe-t-il
si 12 est passé?
Il n'y a pas d'élément dans la table à cet endroit.

Donc la fonction va charger la valeur qui se trouve là, et la renvoyer.

Peu après, une autre fonction pourrait essayer de lire une chaîne de texte depuis
cette adresse et pourrait planter.

Compilons l'exemple dans MSVC pour win64 et ouvrons le dans \IDA pour voir ce que
l'éditeur de lien à stocker après la table:

\lstinputlisting[caption=Executable file in IDA,style=customasmx86]{patterns/13_arrays/45_month_1D/MSVC2012_win64_1.lst}

Les noms des mois se trouvent juste après.

Notre programme est minuscule, il n'y a donc pas beaucoup de données à mettre dans
le segment de données, donc c'est juste les noms des mois.
Mais il faut noter qu'il peut y avoir ici vraiment \IT{n'importe quoi} que l'éditeur
de lien aurait décidé d'y mettre.

Donc, que se passe-t-il si nous passons 12 à la fonction?
Le 13ème élément va être renvoyé.

Voyons comment le CPU traite les octets en une valeur 64-bit:

\lstinputlisting[caption=Executable file in IDA,style=customasmx86]{patterns/13_arrays/45_month_1D/MSVC2012_win64_2.lst}

Et c'est 0x797261756E614A.

Peu après, une autre fonction (supposons, une qui traite des chaînes) pourrait essayer
de lire des octets à cette adresse, y attendant une chaîne-C.

Plus probablement, ça planterait, car cette valeur ne ressemble pas à une adresse
valide.

\myparagraph{Protection contre les débordements de tableaux}

\epigraph{Si quelque chose peut mal tourner, ça tournera mal}{Loi de Murphy}

Il est un peu naïf de s'attendre à ce que chaque programmeur qui utilisera votre
fonction ou votre bibliothèque ne passera jamais un argument plus grand que 11.

Il existe une philosophie qui dit \q{échouer tôt et échouer bruyamment} ou \q{échouer rapidement},
qui enseigne de remonter les problèmes le plus tôt possible et d'arrêter.
\myindex{\CStandardLibrary!assert()}

Une telle méthode en \CCpp est les assertions.

Nous pouvons modifier notre programme pour qu'il échoue si une valeur incorrecte
est passée:

\lstinputlisting[caption=assert() added,style=customc]{patterns/13_arrays/45_month_1D/month1_assert.c}

La macro assertion vérifie que la validité des valeurs à chaque démarrage de fonction
et échoue si l'expression est fausse.

\lstinputlisting[caption=\Optimizing MSVC 2013 x64,style=customasmx86]{patterns/13_arrays/45_month_1D/MSVC2013_x64_Ox_checked.asm}

En fait, assert() n'est pas une fonction, mais une macro. Elle teste une condition,
puis passe le numéro de ligne et le nom du fichier à une autre fonction qui rapporte
cette information à l'utilisateur.

Ici nous voyons qu'à la fois le nom du fichier et la condition sont encodés en UTF-16.
Le numéro de ligne est aussi passé (c'est 29).

Le mécanisme est sans doute le même dans tous les compilateurs.
Voici ce que fait GCC:

\lstinputlisting[caption=\Optimizing GCC 4.9 x64,style=customasmx86]{patterns/13_arrays/45_month_1D/GCC491_x64_O3_checked.s}

Donc la macro dans GCC passe aussi le nom de la fonction par commodité.

Rien n'est vraiment gratuit, et c'est également vrai pour les tests de validité.

Ils rendent votre programme plus lent, en particulier si la macro assert() est utilisée
dans des petites fonctions à durée critiques.

Donc MSCV, par exemple, laisse les tests dans les compilations debug, mais ils disparaissent
dans celles de release.

Les noyaux de Microsoft \gls{Windows NT} existent en versions \q{checked} et \q{free}.
\footnote{\href{http://go.yurichev.com/17259}{msdn.microsoft.com/en-us/library/windows/hardware/ff543450(v=vs.85).aspx}}.

Le premier a des tests de validation (d'oú, \q{checked}), le second n'en a pas (d'oú, \q{free/libre} de tests).

Bien sûr, le noyau \q{checked} fonctionne plus lentement à cause de ces tests, donc
il n'est utilisé que pour des sessions de debug.

% FIXME: ARM? MIPS?

\subsubsection{Accèder à un caractère spécifique}

Un tableau de pointeurs sur des chaînes peut être accèdé comme ceci\footnote{Lisez
l'avertissement dans la NDT ici \myref{array_of_pointers_to_strings}}:

\lstinputlisting[style=customc]{patterns/13_arrays/45_month_1D/month2_FR.c}

\dots puisque l'expression \IT{month[3]} a un type \IT{const char*}.
Et donc, le 5ème caractère est extrait de cette expression en ajoutant 4 octets à
cette adresse.

Á propos, la liste d'arguments passée à la fonction \IT{main()} a le même type de
données:

\lstinputlisting[style=customc]{patterns/13_arrays/45_month_1D/argv_FR.c}

Il est très important de comprendre, que, malgrè la syntaxe similaire, c'est différent
d'un tableau à deux dimensions, dont nous allons parler plus tard.

