\subsection{Array von Stringpointern}
\label{array_of_pointers_to_strings}
Hier ist ein Beispiel für ein Array aus Pointern.

\lstinputlisting[caption=Get month name,label=get_month1,style=customc]{patterns/13_arrays/45_month_1D/month1_DE.c}

\subsubsection{x64}

\lstinputlisting[caption=\Optimizing MSVC 2013 x64,style=customasmx86]{patterns/13_arrays/45_month_1D/month1_MSVC_2013_x64_Ox.asm}

Der Code ist sehr einfach:

\begin{itemize}

\item
\myindex{x86!\Instructions!MOVSXD}

Der erste \INS{MOVSXD} Befehl kopiert einen 32-Bit-Wert aus \ECX (wohin das $month$ Argument übergeben wird) nach \RAX
und erweitert ihn um ein Vorzeichen (da $month$ vom Typ \Tint ist).

Der Grund für die Erweiterung ist, dass dieser 32-Bit-Wert in Berechnungen mit anderen 64-Bit-Werten zusammen verwendet
wird.
\footnote{Es ist seltsam, aber negative Arrayindizes für $month$ können hier verwendet werden
(negative Arrayindizes werden später erklärt: \myref{negative_array_indices}).
Wenn dies passiert, wird der negative Eingabewert vom Typ \Tint korrekt um ein Vorzeichen erweitert und das zugehörige
Element vor der Tabelle wird ausgewählt. Ohne Vorzeichenerweiterung würde es nicht korrekt funktionieren.}. 

\item
Danach wird die Adresse der Pointertabelle nach \RCX geladen.

\item
Schließlich wird der Eingabewert ($month$) mit 9 multipliziert und zur Adresse addiert.
Es gilt: wir befinden uns in einer 64-Bit-Umgebung und alle Adressen (oder Pointer) benötigen genau 64 Bit (oder 8 Byte)
zum Speichern.
Deshalb ist jedes Element der Tabelle 8 Byte breit.
Ebenfalls deshalb müssen, um ein spezifisches Element auszuwählen $month\cdot 8$ Bytes vom Start weg übersprungen
werden.
Genau das tut \MOV. Zusätzlich lädt dieser Befehl auch ein Element in diese Adresse.
Für 1 würde das Element ein Pointer auf den String \q{Februar} sein, etc.

\end{itemize}

\Optimizing GCC 4.9 ist noch effizienter:
\footnote{\q{0+} blieb im Listing,da der GCC Assembler-Output nicht sauber genug ist, um es zu eliminieren.}:

\begin{lstlisting}[caption=\Optimizing GCC 4.9 x64,style=customasmx86]
	movsx	rdi, edi
	mov	rax, QWORD PTR month1[0+rdi*8]
	ret
\end{lstlisting}

\myparagraph{32-bit MSVC}
Kompilieren wir den Code mit dem 32-Bit-MSVC-Compiler:

\lstinputlisting[caption=\Optimizing MSVC 2013 x86,style=customasmx86]{patterns/13_arrays/45_month_1D/month1_MSVC_2013_x86_Ox.asm}
Der Eingabewert darf nicht zu einem 64-Bit-Wert erweitert werden und wird so wie er ist verwendet.

Er wird mit 4 multipliziert, da die Tabellenelemente 32 Bit (oder 4 Byte) breit sind.

% FIXME1 move to another file
\subsubsection{32-bit ARM}

\myparagraph{ARM im ARM mode}

\lstinputlisting[caption=\OptimizingKeilVI (\ARMMode),style=customasmARM]{patterns/13_arrays/45_month_1D/month1_Keil_ARM_O3.s}

% TODO Fix R1s
Die Adresse der Tabelle wird nach R1 geladen.

\myindex{ARM!\Instructions!LDR}
Der ganze Rest wird mit lediglich einem \LDR Befehl erledigt.

Danach wird der Eingabewert $month$ um zwei Bit nach links verschoben (dies entspricht einer Multiplikation mit 4),
schließlich zu R1 addiert (wo sich die Adresse der Tabelle befindet) und schlußendlich wird ein Tabellenelement aus
dieser Adresse geladen.

Das 32-Bit-Tabellenelement wird aus der Tabelle nach R0 geladen.

\myparagraph{ARM im Thumb mode}
Der Code ist fast identisch, aber weniger dicht, denn der \LSL Suffix kann hier nicht an den \LDR Befehl angehängt
werden:

\begin{lstlisting}[style=customasmARM]
get_month1 PROC
        LSLS     r0,r0,#2
        LDR      r1,|L0.64|
        LDR      r0,[r1,r0]
        BX       lr
        ENDP
\end{lstlisting}

\subsubsection{ARM64}

\lstinputlisting[caption=\Optimizing GCC 4.9 ARM64,style=customasmARM]{patterns/13_arrays/45_month_1D/month1_GCC49_ARM64_O3.s}

\myindex{ARM!\Instructions!ADRP/ADD pair}
Die Adresse der Tabelle wird mit \ADRP/\ADD nach X1 geladen.

Dann wird das zugehörige Element mit einem \LDR ausgewählt, das W0 nimmt (das Register, in dem sich der Eingabewert
$month$ befindet), es um 3 Bit nach links verschiebt (was einer Multiplikation mit 8 entspricht), um ein Vorzeichen
erweitert (das bedeutet der Suffix \q{sxtw}) und dann zu X0 addiert.
Schließlich wird der 64-Bit-Wert aus der Tabelle nach X0 geladen.

\subsubsection{MIPS}

\lstinputlisting[caption=\Optimizing GCC 4.4.5
(IDA),style=customasmMIPS]{patterns/13_arrays/45_month_1D/MIPS_O3_IDA_DE.lst}

\subsubsection{Array Overflow}
Unsere Funktion akzeptiert Werte im Bereich von 0 bis 11, aber was, wenn 12 übergeben wird?
Es gibt an dieser Stelle in der Tabelle kein Element.

Die Funktion wird also irgendeinen dort befindlichen Wert laden und ihn zurückgeben.

Kurz danach kann eine andere Funktion versuchen, einen Textstring von dieser Adresse zu laden und könnte abstürzen.

Kompilieren wir das Beipsiel mit MSVC für win64 und öffnen es in \IDA, um zu sehen was der Linker hinter der Tabelle
angelegt hat:

\lstinputlisting[caption=Executable file in IDA,style=customasmx86]{patterns/13_arrays/45_month_1D/MSVC2012_win64_1.lst}
Die Monatsnamen befinden sich direkt dahinter.

Unser Programm ist winzig, sodass hier nicht viele Daten im Datensegment abgelegt werden müssen, nur die Monatsnamen,
Wir stellen aber fest, dass sich hier irgendetwas befinden könnte, was der Linker hier zufällig platziert hat.

Was also, falls 12 an die Funktion übergeben wird? Das 13. Element wird zurückgegeben.

Schauen wir uns an, wie die CPU die Bytes dort wie einen 64-Bit-Wert behandelt:

\lstinputlisting[caption=Executable file in IDA,style=customasmx86]{patterns/13_arrays/45_month_1D/MSVC2012_win64_2.lst}

Dieser Wert ist 0x797261756E614A.

Kurz danach könnte eine andere Funktion (möglicherweise eine, die Strings verarbeitet) versuchen, Bytes von dieser
Adresse zu lesen, weil sie hier einen C-String erwartet.

Höchstwahrscheinlich wird dies zu einem Absturz führen, da der Wert keine gültige Adresse sein wird.

\myparagraph{Schutz vor Array Overflows}

\epigraph{Alles, was schiefgehen kann, wird auch schiefgehen}{Murphy's Law}
Es ist ein wenig naiv zu erwarten, dass jeder Programmierer, der diese Funktion oder Bibliotek verwendet, nie ein
größeres Argument als 11 übergeben wird.

Es gibt einen Ansatz, der besagt \q{scheitere früh und laut} oder \q{scheitere schnell}, und dessen Aussage es ist, dass
Probleme so früh wie möglich gemeldet werden und das Programm angehalten werden sollte.

\myindex{\CStandardLibrary!assert()}
Eine solche Methode in \CCpp sind Assertions.

Wir modifizieren unser Programm, sodass es einen Fehler liefert, wenn ein falscher Wert übergeben wird:

\lstinputlisting[caption=assert() added,style=customc]{patterns/13_arrays/45_month_1D/month1_assert.c}

Das Assertionmakro prüft zu Beginn der Funktion auf valide Werte und liefert einen Fehler, wenn der Ausdruck falsch ist.

\lstinputlisting[caption=\Optimizing MSVC 2013 x64,style=customasmx86]{patterns/13_arrays/45_month_1D/MSVC2013_x64_Ox_checked.asm}
Tatsächlich ist assert() keine Funktion, sondern ein Makro. Es prüft auf eine Bedingung und übergibt dann auch die
Zeilennummer und den Dateinamen an eine andere Funktion, die diese Informationen an den Benutzer weiterleitet.

Wir erkennen hier, dass sowohl Dateiname als auch Bedingung in UTF-16 kodiert sind.
Die Zeilennummer wird ebenfalls übergeben (hier: 29).

Dieser Mechanismus ist möglicherweise in allen Compilern der gleiche. 
GCC erzeugt den folgenden Code:

\lstinputlisting[caption=\Optimizing GCC 4.9 x64,style=customasmx86]{patterns/13_arrays/45_month_1D/GCC491_x64_O3_checked.s}
Das Makro übergibt praktischerweise in GCC auch den Funktionsnamen.

Es gibt aber nichts umsonst und das gilt auch für solche Überprüfungen.

Sie machen das Programm langsamer, vor allem, wenn das assert() Makro in kleinen zeitkritischen Funktionen verwendet
wird.

MSVC zum Beispiel verwendet die Checks in den Debug Builds, aber lässt sie in den Release Builds weg.
 
Microsoft \gls{Windows NT} kernels gibt es als \q{geprüfte} und \q{freie} Builds
\footnote{\href{http://go.yurichev.com/17259}{msdn.microsoft.com/en-us/library/windows/hardware/ff543450(v=vs.85).aspx}}.
Die erste Variante enthält Validierungsprüfungen (daher \q{geprüft}), die zweite nicht (daher \q{frei} von
Überprüfungen).

Natürlich arbeitet der \q{geprüfte} Kernel wegen der vielen Überprüfungen langsamer, sodass es normalerweise nur zum
Debuggen verwendet wird.

% FIXME: ARM? MIPS?
