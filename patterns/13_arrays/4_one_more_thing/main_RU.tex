\subsection{Еще немного о массивах}

Теперь понятно, почему нельзя написать в исходном коде на \CCpp что-то вроде:


\begin{lstlisting}[style=customc]
void f(int size)
{
    int a[size];
...
};
\end{lstlisting}

Чтобы выделить место под массив в локальном стеке, 
компилятору нужно знать размер массива, чего он на стадии компиляции, 
разумеется, знать не может.


\myindex{\CLanguageElements!C99!variable length arrays}
\myindex{\CStandardLibrary!alloca()}
Если вам нужен массив произвольной длины, то выделите столько, сколько нужно, через \TT{malloc()}, 
а затем обращайтесь к выделенному блоку байт как к массиву того типа, который вам нужен.


Либо используйте возможность стандарта C99~ \InSqBrackets{\CNineNineStd 6.7.5/2},
и внутри это очень похоже на \IT{alloca()} (\myref{alloca}).


Для работы в с памятью, можно также воспользоваться библиотекой сборщика мусора в Си.

А для языка Си++ есть библиотеки с поддержкой умных указателей.


