\subsubsection{Lire en dehors des bornes du tableau}

Donc, indexer un tableau est juste \IT{array\lbrack{}index\rbrack}.
Si vous étudiez le code généré avec soin, vous remarquerez sans doute l'absence de
test sur les bornes de l'index, qui devrait vérifier \IT{si il est inférieur à 20}.
Que ce passe-t-il si l'index est supérieur à 20?
C'est une des caractèristiques de \CCpp qui est souvent critiquée.

Voici un code qui compile et fonctionne:

\lstinputlisting[style=customc]{patterns/13_arrays/2_BO/r.c}

Résultat de la compilation (MSVC 2008):

\lstinputlisting[caption=MSVC 2008 \NonOptimizing,style=customasmx86]{patterns/13_arrays/2_BO/r_msvc.asm}

Le code produit ce résultat:

\lstinputlisting[caption=\olly: sortie sur la console]{patterns/13_arrays/2_BO/console.txt}

C'est juste \IT{quelque chose} qui se trouvait sur la pile à côté du tableau, 80 octets
après le début de son premier élément.

\clearpage
\myindex{\olly}
Essayons de trouver d'oú vient cette valeur, en utilisant \olly.

Chargeons et trouvons la valeur située juste après le dernier élément du tableau:

\begin{figure}[H]
\centering
\myincludegraphics{patterns/13_arrays/2_BO/olly_r1.png}
\caption{\olly: lecture du 20ème élément et exécution de \printf}
\label{fig:array_BO_olly_r1}
\end{figure}

Qu'est-ce que c'est?
D'après le schéma de la pile, c'est la valeur sauvegardée du registre EBP.
\clearpage
Exécutons encore et voyons comment il est restauré:

\begin{figure}[H]
\centering
\myincludegraphics{patterns/13_arrays/2_BO/olly_r2.png}
\caption{\olly: restaurer la valeur de EBP}
\label{fig:array_BO_olly_r2}
\end{figure}

En effet, comment est-ce ça pourrait être différent?
Le compilateur pourrait générer du code supplémentaire pour vérifier que la valeur
de l'index est toujours entre les bornes du tableau (comme dans les langages de
programmation de plus haut-niveau\footnote{Java, Python, etc.}) mais cela rendrait
le code plus lent.

