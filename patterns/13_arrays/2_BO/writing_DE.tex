\subsubsection{Schreibzugriff außerhalb von Arraygrenzen}
Nehmen wir an, wir hätte ein paar Werte illegalerweise vom Stack gelesen, wie könnten wir etwas hineinschreiben?

Hier ist, was wir haben:

\lstinputlisting[style=customc]{patterns/13_arrays/2_BO/w.c}

\myparagraph{MSVC}

Wir erhalten das Folgende:

\lstinputlisting[caption=\NonOptimizing MSVC 2008,style=customasmx86]{patterns/13_arrays/2_BO/w_DE.asm}
Das kompilierte Programm stürzt nach der Ausführung ab. Das verwundert nicht. Schauen wir, was genau den Absturz
verursacht.

\clearpage
\myindex{\olly}
Laden wir das Programm in \olly und verfolgen den Ablauf, bis alle 30 Elemente geschrieben worden sind:

\begin{figure}[H]
\centering
\myincludegraphics{patterns/13_arrays/2_BO/olly_w1.png}
\caption{\olly: nach Wiederherstellung des Wertes von EBP}
\label{fig:array_BO_olly_w1}
\end{figure}

\clearpage
Nachverfolgen bis zum Ende der Funktion:

\begin{figure}[H]
\centering
\myincludegraphics{patterns/13_arrays/2_BO/olly_w2.png}
\caption{\olly: 
\TT{EIP} wurde wiederhergestellt, aber \olly kann an 0x15 nicht disassemblieren}
\label{fig:array_BO_olly_w2}
\end{figure}
Richten wir unser Augenmerk auf die Register.

\EIP ist jetzt gerade 0x15. Das ist keine gültige Adreses für Code---zumindest nicht für win32 Code!
Interessant ist auch, dass das \EBP Register 0x14 enthält und \ECX sowie \EDX jeweils 0x1D

Schauen wir uns das Stacklayout etwas genauer an.

Nachdem der Control Flow an \TT{\main} übergeben worde ist, wurde der Wert in \EBP auf dem Stack abgelegt.
Danach wurden 84 Byte für das Array und die Variable $i$ reserviert.
Das entspricht \TT{(20+1)*sizeof(int)}.
\ESP zeigt jetzt auf die Variable \TT{\_i} im lokalen Stack und nach der Ausführung von \TT{PUSH something} scheint sich
\TT{something} neben \TT{\_i} zu befinden.

Hier ist das Stacklayout während der Control Flow in der \main ist:

\begin{center}
\begin{tabular}{ | l | l | }
\hline
  \TT{ESP}    & 4 Byte reserviert für Variable $i$ \\
\hline
  \TT{ESP+4}  & 80 Byte reserviert für Array \TT{a[20]} \\
\hline
  \TT{ESP+84} & sichere Wert von \EBP \\
\hline
  \TT{ESP+88} & Rücksprungadresse \\
\hline
\end{tabular}
\end{center}
Der Befehl \TT{a[19]=something} schreibt den letzten \Tint innerhalb der Grenzen des Arrays (bis hierhin ist alles in
Ordnung!).
Der Befehl \TT{a[20]=something} schreibt \IT{something} an die Stelle, an der der \EBP gespeichert ist.

Sehen wir uns den Zustand der Register im Moment des Absturzes an. In unserem Fall wurde 20 in das zwanzigste Element
geschrieben. Am Ende der Funktion stellt der Funktionsepilog den originalen Wert von \EBP wieder her.
(20 dezimal entspricht \TT{0x14} hexadezimal).
Danach wird \RET ausgeführt, was äquivalent zum Befehl \TT{POP EIP} ist.

Der Befehl \RET nimmt die Rücksprungadresse vom Stack (das ist die Adresse in \ac{CRT}, die \main aufgerufen hat) und
speichert hier den Wert 21 (\TT{0x15} hexadezimal).
Die CPU springt an die Adresse \TT{0x15}, aber hier befindet sich kein ausführbarer Code, sodass eine Exception geworfen
wird.

\myindex{\BufferOverflow}
Dies nennt man einen \IT{Buffer Overflow}\footnote{\href{http://go.yurichev.com/17132}{wikipedia}}.

Ersetzt man das \Tint Array durch einen String (\Tchar Array) und erzeugt absichtlich einen langen String und übergibt
ihn im Programm an eine Funktion, die die Länge des Strings nicht prüft und ihn in einen kurzen Buffer kopiert, kann man
das Programm zwingen an eine bestimmte Adresse zu springen.
In der Realität ist dieses Verhalten nicht so einfach zu erzeugen, funktioniert aber von Prinzip her genau wie hier.
Ein klassischer Artikel dazu:\AlephOne.

\myparagraph{GCC}

Kompilieren wir denselben Code mit GCC 4.4.1, erhalten wir:

\lstinputlisting[style=customasmx86]{patterns/13_arrays/2_BO/w_gcc.asm}

Lässt man das Programm unter Linux laufen, lautet das Ergebnis: \TT{Segmentation fault}.

\myindex{GDB}
Wenn wir es mit dem GDB Debugger laufen lassen, erhalten wir das Folgende:


\begin{lstlisting}
(gdb) r
Starting program: /home/dennis/RE/1 

Program received signal SIGSEGV, Segmentation fault.
0x00000016 in ?? ()
(gdb) info registers
eax            0x0	0
ecx            0xd2f96388	-755407992
edx            0x1d	29
ebx            0x26eff4	2551796
esp            0xbffff4b0	0xbffff4b0
ebp            0x15	0x15
esi            0x0	0
edi            0x0	0
eip            0x16	0x16
eflags         0x10202	[ IF RF ]
cs             0x73	115
ss             0x7b	123
ds             0x7b	123
es             0x7b	123
fs             0x0	0
gs             0x33	51
(gdb) 
\end{lstlisting}
Die Registerwerte unterscheiden sich geringfügig vom win32 Beispiel, da auch das Stacklayout ein wenig anders ist.
