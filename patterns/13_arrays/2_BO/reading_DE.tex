\subsubsection{Lesezugriff außerhalb von Arraygrenzen}
Der indizierte Zugriff auf ein Array wird durch \IT{array\lbrack{}index\rbrack} realisiert.
Wenn man sich den erzeugten Code genau ansieht, bemerkt man, dass eine Prüfung der Indexgrenzen fehlt, welche die
Bedingung \IT{kleiner als 20} validiert.
Was also passiert, wenn der Index 20 oder größer ist? 
Hier haben wir es mit einem unschönen Feature von \CCpp zu tun

Hier ein Beipsielcode der erfolgreich kompiliert wurde und funktioniert:

\lstinputlisting[style=customc]{patterns/13_arrays/2_BO/r.c}

Ergebnis des Kompiliervorgangs (MSVC 2008):

\lstinputlisting[caption=\NonOptimizing MSVC 2008,style=customasmx86]{patterns/13_arrays/2_BO/r_msvc.asm}

Der Code produziert dieses Ergebnis:

\lstinputlisting[caption=\olly: console output]{patterns/13_arrays/2_BO/console.txt}
Es handelt sich um \IT{irgendetwas}, das auf dem Stack in der Nähe des Arrays gelegen hat, 80 Byte von dessen erstem
Element entfernt.

\clearpage
\myindex{\olly}
Versuchen wir mit \olly herauszufinden, woher dieser Wert kommt.

Laden und finden wir also den Wert, der sich direkt hinter dem letzten Arrayelement befindet:

\begin{figure}[H]
\centering
\myincludegraphics{patterns/13_arrays/2_BO/olly_r1.png}
\caption{\olly: das 20. Element lesen und \printf ausführen}
\label{fig:array_BO_olly_r1}
\end{figure}

Worum handelt es sich? 
Dem Stacklayout nach zu urteilen ist dies der gespeicherte Wert des EBP Registers.
\clearpage
Verfolgen wir das ganze weiter und schauen uns an, wie dieser wiederhergestellt wird:

\begin{figure}[H]
\centering
\myincludegraphics{patterns/13_arrays/2_BO/olly_r2.png}
\caption{\olly: Wert von EBP wiederherstellen}
\label{fig:array_BO_olly_r2}
\end{figure}
Wie könnte es anders gelöst werden?
Der Compiler könnte zusätzlichen Code erzeugen, der sicherstellt, dass der Index sich stets innerhalb der Arraygrenzen
befindet (wie in höheren Programmiersprachen\footnote{Java, Python, etc.}), aber das würde den Code langsamer machen.
