\clearpage
\myparagraph{x86 + MSVC + \olly}
\myindex{\olly}
\myindex{x86!\Registers!\Flags}

\olly でこの例を実行すると、フラグがどのように設定されているかを見ることができます。 
符号なしの数値で動作する\TT{f\_unsigned()}から始めましょう。

\CMP はここで3回実行されますが、同じ引数についてはフラグは毎回同じです。

最初の比較の結果は、

\begin{figure}[H]
\centering
\myincludegraphics{patterns/07_jcc/simple/olly_unsigned1.png}
\caption{\olly: \TT{f\_unsigned()}: 最初の条件付きジャンプ}
\label{fig:jcc_olly_unsigned_1}
\end{figure}

従って、フラグは、C=1、P=1、A=1、Z=0、S=1、T=0、D=0、O=0です。

これらは \olly では1文字の略号で命名されています。

\olly は、(\JBE)ジャンプがトリガーされることを示唆しています。 
実際に、インテルのマニュアル(\myref{x86_manuals})を調べると、
CF=1またはZF=1の場合、JBEが起動することがわかります。 
条件はここに当てはまるので、ジャンプが開始されます。

\clearpage
次の条件付きジャンプは、

\begin{figure}[H]
\centering
\myincludegraphics{patterns/07_jcc/simple/olly_unsigned2.png}
\caption{\olly: \TT{f\_unsigned()}: 2番目の条件付きジャンプ}
\label{fig:jcc_olly_unsigned_2}
\end{figure}

\olly は、 \JNZ がトリガーされることを示唆しています。 
実際、ZF=0(ゼロフラグ)の場合、JNZが起動します。

\clearpage
3番目の条件付きジャンプは、 \JNB です。

\begin{figure}[H]
\centering
\myincludegraphics{patterns/07_jcc/simple/olly_unsigned3.png}
\caption{\olly: \TT{f\_unsigned()}: 3番目の条件付きジャンプ}
\label{fig:jcc_olly_unsigned_3}
\end{figure}

インテルのマニュアル(\myref{x86_manuals})では、CF=0(キャリーフラグ)の場合に \JNB が起動することがわかります。 
今回は当てはまらないので、3番目の \printf が実行されます。

\clearpage
次に、\olly で、符号付きの値で動作する\TT{f\_signed()}関数を見てみましょう。 
フラグは、C=1、P=1、A=1、Z=0、S=1、T=0、D=0、O=0と同様に設定されます。 
最初の条件付きジャンプ \JLE が起動されます。

インテルマニュアル(166ページの7.1.4)では、ZF = 1またはSFxOFの場合にこの命令がトリガされることがわかりました。 SFxOF私たちの場合は、ジャンプがトリガするように。

\begin{figure}[H]
\centering
\myincludegraphics{patterns/07_jcc/simple/olly_signed1.png}
\caption{\olly: \TT{f\_signed()}: 最初の条件付きジャンプ}
\label{fig:jcc_olly_signed_1}
\end{figure}

インテルマニュアル(\myref{x86_manuals})では、ZF=1または SF$\neq$OF の場合にこの命令が起動されることがわかりました。 
私たちの場合では SF$\neq$OF が、ジャンプが起動されます。

\clearpage
2番目の \JNZ 条件付きジャンプはZF=0の場合(ゼロ・フラグ)に起動します。

\begin{figure}[H]
\centering
\myincludegraphics{patterns/07_jcc/simple/olly_signed2.png}
\caption{\olly: \TT{f\_signed()}: 2番目の条件付きジャンプ}
\label{fig:jcc_olly_signed_2}
\end{figure}

\clearpage
第3の条件付きジャンプ \JGE は、SF=OFの場合にのみ実行されるため、起動しません。今回は、当てはまりません。

\begin{figure}[H]
\centering
\myincludegraphics{patterns/07_jcc/simple/olly_signed3.png}
\caption{\olly: \TT{f\_signed()}: 3番目の条件付きジャンプ}
\label{fig:jcc_olly_signed_3}
\end{figure}
