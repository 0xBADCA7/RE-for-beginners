\clearpage
\myparagraph{x86 + MSVC + \olly}
\myindex{\olly}
\myindex{x86!\Registers!\Flags}

Nous pouvons voir comment les flags sont mis en lançant cet exemple dans \olly.
Commençons par \TT{f\_unsigned()}, qui utilise des entiers non signés.

\CMP est exécuté trois fois ici, mais avec les même arguments, donc les flags
sont les même à chaque fois.

Résultat de la première comparaison:

\begin{figure}[H]
\centering
\myincludegraphics{patterns/07_jcc/simple/olly_unsigned1.png}
\caption{\olly: \TT{f\_unsigned()}: premier saut conditionnel}
\label{fig:jcc_olly_unsigned_1}
\end{figure}

Donc, les flags sont: C=1, P=1, A=1, Z=0, S=1, T=0, D=0, O=0.

Ils sont nommés avec une seule lettre dans \olly par concision.

\olly indique que le saut (\JBE) va être suivi maintenant.
En effet, si nous regardons dans le manuel d'Intel (\myref{x86_manuals}), nous pouvons
lire que \JBE est déclenchée si CF=1 ou ZF=1.
La condition est vraie ici, donc le saut est effectué.

\clearpage
Le saut conditionnel suivant:

\begin{figure}[H]
\centering
\myincludegraphics{patterns/07_jcc/simple/olly_unsigned2.png}
\caption{\olly: \TT{f\_unsigned()}: second saut conditionnel}
\label{fig:jcc_olly_unsigned_2}
\end{figure}

\olly indique que le saut \JNZ va être effectué maintenant.
En effet, \JNZ est déclenché si ZF=0 (Zero Flag).

\clearpage
Le troisième saut conditionnel, \JNB:

\begin{figure}[H]
\centering
\myincludegraphics{patterns/07_jcc/simple/olly_unsigned3.png}
\caption{\olly: \TT{f\_unsigned()}: troisième saut conditionnel}
\label{fig:jcc_olly_unsigned_3}
\end{figure}

Dans le manuel d'Intel (\myref{x86_manuals}) nous pouvons voir que \JNB est déclenché
si CF=0 (Carry Flag - flag de retenue).
Ce qui n'est pas vrai dans notre cas, donc le troisième \printf sera exécuté.

\clearpage
Maintenant, regardons la fonction \TT{f\_signed()}, qui fonctionne avec des entiers
non signés.
Les flags sont mis de la même manière: C=1, P=1, A=1, Z=0, S=1, T=0, D=0, O=0.
Le premier saut conditionnel \JLE est effectué:

\begin{figure}[H]
\centering
\myincludegraphics{patterns/07_jcc/simple/olly_signed1.png}
\caption{\olly: \TT{f\_signed()}: premier saut conditionnel}
\label{fig:jcc_olly_signed_1}
\end{figure}

Dans les manuels d'Intel (\myref{x86_manuals}) nous trouvons que cette instruction
est déclenchée si ZF=1 ou SF$\neq$OF.
SF$\neq$OF dans notre cas, donc le saut est effectué.

\clearpage
Le second saut conditionnel, \JNZ, est effectué: si ZF=0 (Zero Flag):

\begin{figure}[H]
\centering
\myincludegraphics{patterns/07_jcc/simple/olly_signed2.png}
\caption{\olly: \TT{f\_signed()}: second saut conditionnel}
\label{fig:jcc_olly_signed_2}
\end{figure}

\clearpage
Le troisième saut conditionnel, \JGE, ne sera pas effectué car il ne l'est que
si SF=OF, et ce n'est pas vrai dans notre cas:

\begin{figure}[H]
\centering
\myincludegraphics{patterns/07_jcc/simple/olly_signed3.png}
\caption{\olly: \TT{f\_signed()}: troisième saut conditionnel}
\label{fig:jcc_olly_signed_3}
\end{figure}
