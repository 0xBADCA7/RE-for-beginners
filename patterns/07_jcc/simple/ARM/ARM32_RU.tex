\myparagraph{32-битный ARM}
\label{subsec:jcc_ARM}

\mysubparagraph{\OptimizingKeilVI (\ARMMode)}

\lstinputlisting[caption=\OptimizingKeilVI (\ARMMode),style=customasmARM]{patterns/07_jcc/simple/ARM/ARM_O3_signed.asm}

\myindex{ARM!Condition codes}
% FIXME \ref -> which instructions?
Многие инструкции в режиме ARM могут быть исполнены только при некоторых выставленных флагах.

Это нередко используется для сравнения чисел.

\myindex{ARM!\Instructions!ADD}
\myindex{ARM!\Instructions!ADDAL}
К примеру, инструкция \ADD на самом деле называется \TT{ADDAL} внутри, \TT{AL} означает \IT{Always}, то есть, исполнять всегда.
Предикаты кодируются в 4-х старших битах инструкции 32-битных ARM-инструкций (\IT{condition field}).
\myindex{ARM!\Instructions!B}
Инструкция безусловного перехода \TT{B} на самом деле условная и кодируется так же, 
как и прочие инструкции условных переходов, но имеет \TT{AL} в \IT{condition field}, 
то есть исполняется всегда (\IT{execute ALways}), игнорируя флаги.

\myindex{ARM!\Instructions!ADR}
\myindex{ARM!\Instructions!ADRcc}
\myindex{ARM!\Instructions!CMP}
Инструкция \TT{ADRGT} работает так же, как и \TT{ADR}, но исполняется только в случае,
если предыдущая инструкция \CMP,
сравнивая два числа, обнаруживает, что одно из них больше второго (\IT{Greater Than}).

\myindex{ARM!\Instructions!BL}
\myindex{ARM!\Instructions!BLcc}
Следующая инструкция \TT{BLGT} ведет себя так же, как и \TT{BL} и сработает, только если 
результат сравнения ``больше чем'' (\IT{Greater Than}).
\TT{ADRGT} записывает в \Reg{0} указатель на строку \TT{a>b\textbackslash{}n}, а \TT{BLGT} вызывает \printf.
Следовательно, эти инструкции с суффиксом \TT{-GT} исполнятся только в том случае, если значение
в \Reg{0} (там $a$) было больше, чем значение в \Reg{4} (там $b$).

\myindex{ARM!\Instructions!ADRcc}
\myindex{ARM!\Instructions!BLcc}
Далее мы увидим инструкции \TT{ADREQ} и \TT{BLEQ}.
Они работают так же, как и \TT{ADR} и \TT{BL}, но исполнятся только если значения при последнем сравнении были равны.
Перед ними расположен ещё один \CMP, потому что вызов \printf мог испортить состояние флагов.

\myindex{ARM!\Instructions!LDMccFD}
\myindex{ARM!\Instructions!LDMFD}
Далее мы увидим \TT{LDMGEFD}. Эта инструкция работает так же, как и \TT{LDMFD}\footnote{\ac{LDMFD}}, 
но сработает только если в результате сравнения одно из значений было больше или равно второму (\IT{Greater or Equal}).
Смысл инструкции \TT{LDMGEFD SP!, \{R4-R6,PC\}} 
в том, что это как бы эпилог функции, но он сработает только если $a>=b$, только тогда работа 
функции закончится.

\myindex{Function epilogue}
Но если это не так, то есть $a<b$, то исполнение дойдет до следующей инструкции 
\TT{LDMFD SP!, \{R4-R6,LR\}}. Это ещё один эпилог функции. Эта инструкция восстанавливает состояние регистров
\TT{R4-R6}, но и \ac{LR} вместо \ac{PC}, таким образом, пока что, не делая возврата из функции.

Последние две инструкции вызывают \printf 
со строкой <<a<b\textbackslash{}n>> в качестве единственного аргумента.
Безусловный переход на \printf вместо возврата из функции мы уже рассматривали в секции
 <<\PrintfSeveralArgumentsSectionName>>~(\myref{ARM_B_to_printf}).

\myindex{ARM!\Instructions!ADRcc}
\myindex{ARM!\Instructions!BLcc}
\myindex{ARM!\Instructions!LDMccFD}
Функция \TT{f\_unsigned} точно такая же, но там используются инструкции \TT{ADRHI}, \TT{BLHI}, и \TT{LDMCSFD}. Эти предикаты
(\IT{HI = Unsigned higher, CS = Carry Set (greater than or equal)})
аналогичны рассмотренным, но служат для работы с беззнаковыми значениями.

В функции \main ничего нового для нас нет:

\lstinputlisting[caption=\main,style=customasmARM]{patterns/07_jcc/simple/ARM/ARM_O3_main.asm}

Так, в режиме ARM можно обойтись без условных переходов.

\myindex{Конвейер RISC}
Почему это хорошо? Читайте здесь: \myref{branch_predictors}.

\myindex{x86!\Instructions!CMOVcc}
В x86 нет аналогичной возможности, если не считать инструкцию \TT{CMOVcc}, это то же что и \MOV, 
но она срабатывает только при определенных выставленных флагах, обычно выставленных при помощи \CMP во время сравнения.

\mysubparagraph{\OptimizingKeilVI (\ThumbMode)}

\lstinputlisting[caption=\OptimizingKeilVI (\ThumbMode),style=customasmARM]{patterns/07_jcc/simple/ARM/ARM_thumb_signed.asm}

\myindex{ARM!\Instructions!BLE}
\myindex{ARM!\Instructions!BNE}
\myindex{ARM!\Instructions!BGE}
\myindex{ARM!\Instructions!BLS}
\myindex{ARM!\Instructions!BCS}
\myindex{ARM!\Instructions!B}
\myindex{ARM!\ThumbMode}
В режиме Thumb только инструкции \TT{B} могут быть дополнены условием исполнения (\IT{condition code}), 
так что код для режима Thumb выглядит привычнее.

\TT{BLE} это обычный переход с условием \IT{Less than or Equal}, 
\TT{BNE} --- \IT{Not Equal}, 
\TT{BGE} --- \IT{Greater than or Equal}.

Функция \TT{f\_unsigned} точно такая же, но для работы с беззнаковыми величинами 
там используются инструкции \TT{BLS} 
(\IT{Unsigned lower or same}) и \TT{BCS} (\IT{Carry Set (Greater than or equal)}).
