\myparagraph{ARM 32-bit}
\label{subsec:jcc_ARM}

\mysubparagraph{\OptimizingKeilVI (\ARMMode)}

\lstinputlisting[caption=\OptimizingKeilVI (\ARMMode),style=customasmARM]{patterns/07_jcc/simple/ARM/ARM_O3_signed.asm}

\myindex{ARM!Condition codes}
% FIXME \ref -> which instructions?

Beaucoup d'instructions en mode ARM ne peuvent être exécutées que lorsque certains
flags sont mis.
E.g, ceci est souvent utilisé lorsque l'on compare les nombres

\myindex{ARM!\Instructions!ADD}
\myindex{ARM!\Instructions!ADDAL}

Par exemple, l'instruction \ADD est en fait appelée \TT{ADDAL} en interne, où \TT{AL}
signifie \IT{Always}, i.e., toujours exécuter.
Les prédicats sont encodés dans les 4 bits du haut des instructions ARM 32-bit. (\IT{condition field}).
\myindex{ARM!\Instructions!B}
L'instruction de saut inconditionnel \TT{B} est en fait conditionnelle et encodée
comme toutes les autres instructions de saut conditionnel, mais a \TT{AL} dans le
\IT{champ de condition}, et \IT{s'exécute toujours} (ALways), ignorants les flags.

\myindex{ARM!\Instructions!ADR}
\myindex{ARM!\Instructions!ADRcc}
\myindex{ARM!\Instructions!CMP}

L'instruction \TT{ADRGT} fonctionne comme \TT{ADR} mais ne s'exécute que dans le
cas où l'instruction \CMP précédente a trouvé un des nombres plus grand que l'autre,
en comparant les deux (\IT{Greater Than}).

\myindex{ARM!\Instructions!BL}
\myindex{ARM!\Instructions!BLcc}

L'instruction \TT{BLGT} se comporte exactement comme \TT{BL} et n'est effectuée
que si le résultat de la comparaison était \IT{Greater Than} (plus grand).
\TT{ADRGT} écrit un pointeur sur la chaîne \TT{a>b\textbackslash{}n} dans \Reg{0}
et \TT{BLGT} appelle \printf.
Donc, les instructions suffixées par \TT{-GT} ne sont exécutées que si la valeur
dans \Reg{0} (qui est $a$) est plus grande que la valeur dans \Reg{4} (qui est $b$).

\myindex{ARM!\Instructions!ADRcc}
\myindex{ARM!\Instructions!BLcc}

En avançant, nous voyons les instructions \TT{ADREQ} et \TT{BLEQ}.
Elles se comportent comme \TT{ADR} et \TT{BL}, mais ne sont exécutées que si les
opérandes étaient égaux lors de la dernière comparaison.
Un autre \CMP se trouve avant elles (car l'exécution de \printf pourrait avoir
modifiée les flags).

\myindex{ARM!\Instructions!LDMccFD}
\myindex{ARM!\Instructions!LDMFD}

Ensuite nous voyons \TT{LDMGEFD}, cette instruction fonctionne comme \TT{LDMFD}\footnote{\ac{LDMFD}},
mais n'est exécutée que si l'une des valeurs est supérieure ou égale à l'autre
(\IT{Greater or Equal}).
L'instruction \TT{LDMGEFD SP!, \{R4-R6,PC\}} se comporte comme une fonction épilogue,
mais elle ne sera exécutée que si $a>=b$, et seulement lorsque l'exécution de la
fonction se terminera.

\myindex{Function epilogue}

Mais si cette condition n'est pas satisfaite, i.e., $a<b$, alors le flux d'exécution
continue à l'instruction suivante, \TT{\q{LDMFD SP!, \{R4-R6,LR\}}}, qui est aussi
un épilogue de la fonction. Cette instruction ne restaure pas seulement l'état des
registres \TT{R4-R6}, mais aussi \ac{LR} au lieu de \ac{PC}, donc il ne retourne
pas de la fonction.
Les deux dernières instructions appellent \printf avec la chaîne <<a<b\textbackslash{}n>>
comme unique argument.
Nous avons déjà examiné un saut inconditionnel à la fonction \printf au lieu
d'un appel avec retour dans <<\PrintfSeveralArgumentsSectionName>> section~(\myref{ARM_B_to_printf}).

\myindex{ARM!\Instructions!ADRcc}
\myindex{ARM!\Instructions!BLcc}
\myindex{ARM!\Instructions!LDMccFD}
\TT{f\_unsigned} est très similaire, à part les instructions \TT{ADRHI}, \TT{BLHI},
et \TT{LDMCSFD} utilisées ici, ces prédicats (\IT{HI = Unsigned higher, CS = Carry
Set (greater than or equal)}) sont analogues à ceux examinés avant, mais pour des
valeurs non signées.

Il n'y a pas grand chose de nouveau pour nous dans la fonction \main:

\lstinputlisting[caption=\main,style=customasmARM]{patterns/07_jcc/simple/ARM/ARM_O3_main.asm}

C'est ainsi que vous pouvez vous débarrasser des sauts conditionnels en mode ARM.

\myindex{RISC pipeline}
Pourquoi est-ce que c'est si utile? Lire ici: \myref{branch_predictors}.

\myindex{x86!\Instructions!CMOVcc}

Il n'y a pas de telle caractéristique en x86, exceptée l'instruction \TT{CMOVcc},
qui est comme un \MOV, mais effectuée seulement lorsque certains flags sont mis,
en général mis par \CMP.

\mysubparagraph{\OptimizingKeilVI (\ThumbMode)}

\lstinputlisting[caption=\OptimizingKeilVI (\ThumbMode),style=customasmARM]{patterns/07_jcc/simple/ARM/ARM_thumb_signed.asm}

\myindex{ARM!\Instructions!BLE}
\myindex{ARM!\Instructions!BNE}
\myindex{ARM!\Instructions!BGE}
\myindex{ARM!\Instructions!BLS}
\myindex{ARM!\Instructions!BCS}
\myindex{ARM!\Instructions!B}
\myindex{ARM!\ThumbMode}

En mode Thumb, seules les instructions \TT{B} peuvent être complétées par un
\IT{condition codes}, (code de condition) donc le code Thumb paraît plus ordinaire.

\TT{BLE} est un saut conditionnel normal \IT{Less than or Equal} (inférieur ou égal),
\TT{BNE}---\IT{Not Equal} (non égal),
\TT{BGE}---\IT{Greater than or Equal} (supérieur ou égal).

\TT{f\_unsigned} est similaire, seules d'autres instructions sont utilisées
pour travailler avec des valeurs non-signées: \TT{BLS}
(\IT{Unsigned lower or same} non signée, inférieur ou égal) et \TT{BCS} (\IT{Carry
Set (Greater than or equal)} supérieur ou égal).
