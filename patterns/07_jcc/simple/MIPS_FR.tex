\subsubsection{MIPS}

Une des caractéristiques distinctives de MIPS est l'absence de flag. 
Apparemment, cela a été fait pour simplifier l'analyse des dépendances de données.

\myindex{x86!\Instructions!SETcc}
\myindex{MIPS!\Instructions!SLT}
\myindex{MIPS!\Instructions!SLTU}

Il y a des instructions similaires à \INS{SETcc} en x86: \INS{SLT} (\q{Set on Less Than}:
mettre si plus petit que, version signée) et \INS{SLTU} (version non signée).
Ces instructions mettent le registre de destination à 1 si la condition est vraie
ou à 0 autrement.

\myindex{MIPS!\Instructions!BEQ}
\myindex{MIPS!\Instructions!BNE}

Le registre de destination est ensuite testé avec \INS{BEQ} (\q{Branch on Equal}
branchement si égal) ou \INS{BNE} (\q{Branch on Not Equal} branchement si non égal)
et un saut peut survenir.
Donc, cette paire d'instructions doit être utilisée en MIPS pour comparer et effectuer
un branchement.
Essayons avec la version signée de notre fonction:

\lstinputlisting[caption=GCC 4.4.5 \NonOptimizing (IDA),style=customasmMIPS]{patterns/07_jcc/simple/O0_MIPS_signed_IDA_FR.lst}

\INS{SLT REG0, REG0, REG1} est réduit par IDA à sa forme plus courte:\\
\INS{SLT REG0, REG1}. 
\myindex{MIPS!\Pseudoinstructions!BEQZ}

Nous voyons également ici la pseudo instruction \INS{BEQZ} (\q{Branch if Equal to Zero}
branchement si égal à zéro),\\
qui est en fait \INS{BEQ REG, \$ZERO, LABEL}.

\myindex{MIPS!\Instructions!SLTU}

La version non signée est la même, mais \INS{SLTU} (version non signée, d'où
le \q{U} de unsigned) est utilisée au lieu de \INS{SLT}:

\lstinputlisting[caption=GCC 4.4.5 \NonOptimizing (IDA),style=customasmMIPS]{patterns/07_jcc/simple/O0_MIPS_unsigned_IDA.lst}

