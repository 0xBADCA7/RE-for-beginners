\clearpage
\myparagraph{x86 + MSVC + \olly}
\myindex{\olly}
\myindex{x86!\Registers!\Flags}
Wir sehen wie die Flags gesetzt werden, indem wir das Beispiel in \olly laufen lassen.
Beginnen wir mit \TT{f\_unsigned()}; diese Funtion arbeitet mit vorzeichenlosen Zahlen.

\CMP wird hier dreimal ausgeführt, aber das die Argumente jeweils identisch sind, sind die Flags jedes Mal die gleichen.

Ergebnis des ersten Vergleichs:

\begin{figure}[H]
\centering
\myincludegraphics{patterns/07_jcc/simple/olly_unsigned1.png}
\caption{\olly: \TT{f\_unsigned()}: erster bedingter Sprung}
\label{fig:jcc_olly_unsigned_1}
\end{figure}

Die Flags sind also: C=1, P=1, A=1, Z=0, S=1, T=0, D=0, O=0.
Ihre Namen werden in \olly mit einem Buchstaben abgekürzt.

\olly zeigt an, dass der (\JBE) Sprung jetzt ausgeführt werden wird.
Und tatsächlich, wenn wir in das Intel-Handbuch (\myref{x86_manuals}) schauen, finden wir, dass \JBE ausgeführt wird,
falls CF=1 oder ZF=1.
Da diese Bedingung hier wahr ist, wird der Sprung ausgeführt.


\clearpage
Der nächste bedingte Sprung:

\begin{figure}[H]
\centering
\myincludegraphics{patterns/07_jcc/simple/olly_unsigned2.png}
\caption{\olly: \TT{f\_unsigned()}: zweiter bedingter Sprung}
\label{fig:jcc_olly_unsigned_2}
\end{figure}

\olly zeigt an, dass \JNZ jetzt ausgeführt werden wird.
Tatsächlich wird \JNZ ausgeführt, falls ZF=0 (Zero Flag).

\clearpage
Der dritte bedingte Sprung (\JNB):

\begin{figure}[H]
\centering
\myincludegraphics{patterns/07_jcc/simple/olly_unsigned3.png}
\caption{\olly: \TT{f\_unsigned()}: dritter bedingter Sprung}
\label{fig:jcc_olly_unsigned_3}
\end{figure}
Im Intel-Handbuch (\myref{x86_manuals}) können wir nachlesen, dass \JNB ausgeführt wird, falls CF=0 (Carry Flag).
Das ist in unserem Beispiel nicht der Falls, sodass das dritte \printf ausgeführt wird.

\clearpage
Schauen wir uns nun in \olly die \TT{f\_signed()} Funtion an, die mit vorzeichenbehafteten Werte arbeitet.
Die Flags werden auf die gleiche Weise gesetzt: C=1, P=1, A=1, Z=0, S=1, T=0, D=0, O=0.
Der erste bedingte Sprung \JLE wird danach ausgeführt:

\begin{figure}[H]
\centering
\myincludegraphics{patterns/07_jcc/simple/olly_signed1.png}
\caption{\olly: \TT{f\_signed()}: erster bedingter Sprung}
\label{fig:jcc_olly_signed_1}
\end{figure}
Im Intel-Handbuch (\myref{x86_manuals}) können wir nachlesen, dass dieser Befehl ausgeführt wird, falls ZF=1 oder
SF$\neq$OF. In userem Fall gilt SF$\neq$OF, sodass der Sprung ausgeführt wird.


\clearpage
Der zweite \JNZ bedingte Sprung wird ausgeführt, falls ZF=0 (Zero Flag):

\begin{figure}[H]
\centering
\myincludegraphics{patterns/07_jcc/simple/olly_signed2.png}
\caption{\olly: \TT{f\_signed()}: zweiter bedingter Sprung}
\label{fig:jcc_olly_signed_2}
\end{figure}

\clearpage
Der dritter bedingte Sprung \JBE wird nicht ausgeführt, da dies nur geschieht, falls SF=OF, und dies in unserem Beispiel
nicht der Fall ist:

\begin{figure}[H]
\centering
\myincludegraphics{patterns/07_jcc/simple/olly_signed3.png}
\caption{\olly: \TT{f\_signed()}: dritter bedingter Sprung}
\label{fig:jcc_olly_signed_3}
\end{figure}
