\subsection{Calculating absolute value}
\label{sec:abs}

A simple function:

\lstinputlisting[style=customc]{abs.c}

\subsubsection{\Optimizing MSVC}

This is how the code is usually generated:

\lstinputlisting[caption=\Optimizing MSVC 2012 x64,style=customasmx86]{patterns/07_jcc/abs/abs_MSVC2012_Ox_x64_EN.asm}

GCC 4.9 does mostly the same.

\subsubsection{\OptimizingKeilVI: \ThumbMode}

\lstinputlisting[caption=\OptimizingKeilVI: \ThumbMode,style=customasmARM]{patterns/07_jcc/abs/abs_Keil_thumb_O3_EN.s}

\myindex{ARM!\Instructions!RSB}

ARM lacks a negate instruction, so the Keil compiler uses the \q{Reverse Subtract} instruction, which just subtracts with reversed operands.

\subsubsection{\OptimizingKeilVI: \ARMMode}

It is possible to add condition codes to some instructions in ARM mode, so that is what the Keil compiler does:

\lstinputlisting[caption=\OptimizingKeilVI: \ARMMode,style=customasmARM]{patterns/07_jcc/abs/abs_Keil_ARM_O3_EN.s}

Now there are no conditional jumps and this is good: \myref{branch_predictors}.

\subsubsection{\NonOptimizing GCC 4.9 (ARM64)}

\myindex{ARM!\Instructions!XOR}

ARM64 has instruction \INS{NEG} for negating:

\lstinputlisting[caption=\Optimizing GCC 4.9 (ARM64),style=customasmARM]{patterns/07_jcc/abs/abs_GCC49_ARM64_O0_EN.s}

\subsubsection{MIPS}

\lstinputlisting[caption=\Optimizing GCC 4.4.5 (IDA),style=customasmMIPS]{patterns/07_jcc/abs/MIPS_O3_IDA_EN.lst}

\myindex{MIPS!\Instructions!BLTZ}
Here we see a new instruction: \INS{BLTZ} (\q{Branch if Less Than Zero}).
\myindex{MIPS!\Instructions!SUBU}
\myindex{MIPS!\Pseudoinstructions!NEGU}

There is also the \INS{NEGU} pseudo instruction, which just does subtraction from zero.
The \q{U} suffix in both \INS{SUBU} and \INS{NEGU} implies that no exception to be raised in case of integer overflow.

\subsubsection{Branchless version?}

You could have also a branchless version of this code. This we will review later: \myref{chap:branchless_abs}.
