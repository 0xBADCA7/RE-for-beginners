\subsection{Вычисление абсолютной величины}
\label{sec:abs}

Это простая функция:

\lstinputlisting[style=customc]{abs.c}

\subsubsection{\Optimizing MSVC}

Обычный способ генерации кода:

\lstinputlisting[caption=\Optimizing MSVC 2012 x64,style=customasm]{patterns/07_jcc/abs/abs_MSVC2012_Ox_x64_RU.asm}

GCC 4.9 делает почти то же самое.

\subsubsection{\OptimizingKeilVI: \ThumbMode}

\lstinputlisting[caption=\OptimizingKeilVI: \ThumbMode,style=customasm]{patterns/07_jcc/abs/abs_Keil_thumb_O3_RU.s}

\myindex{ARM!\Instructions!RSB}
В ARM нет инструкции для изменения знака, так что компилятор Keil использует инструкцию \q{Reverse Subtract},
которая просто вычитает, но с операндами, переставленными наоборот.

\subsubsection{\OptimizingKeilVI: \ARMMode}

В режиме ARM можно добавлять коды условий к некоторым инструкций, что компилятор Keil и сделал:

\lstinputlisting[caption=\OptimizingKeilVI: \ARMMode,style=customasm]{patterns/07_jcc/abs/abs_Keil_ARM_O3_RU.s}

Теперь здесь нет условных переходов и это хорошо:
 
\myref{branch_predictors}.

\subsubsection{\NonOptimizing GCC 4.9 (ARM64)}

\myindex{ARM!\Instructions!XOR}
В ARM64 есть инструкция \INS{NEG} для смены знака:

\lstinputlisting[caption=\Optimizing GCC 4.9 (ARM64),style=customasm]{patterns/07_jcc/abs/abs_GCC49_ARM64_O0_RU.s}

\subsubsection{MIPS}

\lstinputlisting[caption=\Optimizing GCC 4.4.5 (IDA),style=customasm]{patterns/07_jcc/abs/MIPS_O3_IDA_RU.lst}

\myindex{MIPS!\Instructions!BLTZ}
Видим здесь новую инструкцию: \INS{BLTZ} (\q{Branch if Less Than Zero}).
\myindex{MIPS!\Instructions!SUBU}
\myindex{MIPS!\Pseudoinstructions!NEGU}
Тут есть также псевдоинструкция \INS{NEGU}, которая на самом деле вычитает из нуля.
Суффикс \q{U} в обоих инструкциях \INS{SUBU} и \INS{NEGU} означает, что при целочисленном переполнении исключение не
сработает.

\subsubsection{Версия без переходов?}

Возможна также версия и без переходов, мы рассмотрим её позже: \myref{chap:branchless_abs}.
