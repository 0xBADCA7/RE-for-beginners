\subsection{Ternary conditional operator}
\label{chap:cond}

The ternary conditional operator in \CCpp has the following syntax:

\begin{lstlisting}
expression ? expression : expression
\end{lstlisting}

Here is an example:

\lstinputlisting[style=customc]{patterns/07_jcc/cond_operator/cond.c}

\subsubsection{x86}

Old and non-optimizing compilers generate assembly code just as if an \TT{if/else} statement was used:

\lstinputlisting[caption=\NonOptimizing MSVC 2008,style=customasmx86]{patterns/07_jcc/cond_operator/MSVC2008_EN.asm}

\lstinputlisting[caption=\Optimizing MSVC 2008,style=customasmx86]{patterns/07_jcc/cond_operator/MSVC2008_Ox_EN.asm}

Newer compilers are more concise:

\lstinputlisting[caption=\Optimizing MSVC 2012 x64,style=customasmx86]{patterns/07_jcc/cond_operator/MSVC2012_Ox_x64_EN.asm}

\myindex{x86!\Instructions!CMOVcc}
\Optimizing GCC 4.8 for x86 also uses the \TT{CMOVcc} instruction, while the non-optimizing GCC 4.8 uses conditional jumps.

\subsubsection{ARM}

\myindex{x86!\Instructions!ADRcc}
\Optimizing Keil for ARM mode also uses the conditional instructions \TT{ADRcc}:

\lstinputlisting[label=cond_Keil_ARM_O3,caption=\OptimizingKeilVI (\ARMMode),style=customasmARM]{patterns/07_jcc/cond_operator/Keil_ARM_O3_EN.s}

Without manual intervention, the two instructions \TT{ADREQ} and \TT{ADRNE} cannot be executed in the same run.

\Optimizing Keil for Thumb mode needs to use conditional jump instructions, since there are no load instructions
that support conditional flags:

\lstinputlisting[caption=\OptimizingKeilVI (\ThumbMode),style=customasmARM]{patterns/07_jcc/cond_operator/Keil_thumb_O3_EN.s}

\subsubsection{ARM64}

\Optimizing GCC (Linaro) 4.9 for ARM64 also uses conditional jumps:

\lstinputlisting[label=cond_ARM64,caption=\Optimizing GCC (Linaro) 4.9,style=customasmARM]{patterns/07_jcc/cond_operator/ARM64_GCC_O3_EN.s}

That is because ARM64 does not have a simple load instruction with conditional flags,
like \TT{ADRcc} in 32-bit ARM mode or \INS{CMOVcc} in x86.

\myindex{ARM!\Instructions!CSEL}
It has, however, \q{Conditional SELect} instruction (\TT{CSEL})\InSqBrackets{\ARMSixFourRef p390, C5.5},
but GCC 4.9 does not seem to be smart enough to use it in such piece of code.

\subsubsection{MIPS}

Unfortunately, GCC 4.4.5 for MIPS is not very smart, either:

\lstinputlisting[caption=\Optimizing GCC 4.4.5 (\assemblyOutput),style=customasmMIPS]{patterns/07_jcc/cond_operator/MIPS_O3_EN.s}

\subsubsection{Let's rewrite it in an \TT{if/else} way}

\lstinputlisting[style=customc]{patterns/07_jcc/cond_operator/cond2.c}

\myindex{x86!\Instructions!CMOVcc}

Interestingly, optimizing GCC 4.8 for x86 was also able to use \TT{CMOVcc} in this case:

\lstinputlisting[caption=\Optimizing GCC 4.8,style=customasmx86]{patterns/07_jcc/cond_operator/cond2_GCC_O3_EN.s}

\Optimizing Keil in ARM mode generates code identical to \lstref{cond_Keil_ARM_O3}.

But the optimizing MSVC 2012 is not that good (yet).

\subsubsection{\Conclusion{}}

Why optimizing compilers try to get rid of conditional jumps? Read here about it: \myref{branch_predictors}.
