\section{空函数}
\label{empty_func}

最简单的函数就是什么都不做的函数:

\lstinputlisting[caption=\EN{\CCpp代码},style=customc]{patterns/00_empty/1.c}

我们来编译一下。

\subsection{x86}

以下的内容是在x86平台上,MSVC和GCC编译器所产生的优化过的输出:

\lstinputlisting[caption=\Optimizing GCC/MSVC (\assemblyOutput),style=customasmx86]{patterns/00_empty/1.s}

\myindex{x86!\Instructions!RET}
只有一条指令\RET, 该指令会将执行权返回给\gls{调用者}。

\subsection{ARM}

\lstinputlisting[caption=\OptimizingKeilVI (\ARMMode) \assemblyOutput,style=customasmARM]{patterns/00_empty/1_Keil_ARM_O3.s}

在ARM \ac{ISA}上,返回值并不存储在栈上,而是存放在连接寄存器(Link Register)中,所以\INS{BX LR}指令会跳转到该地址处——从而有效地将执行权返回给\gls{调用者}。

\subsection{MIPS}

在MIPS中,对于寄存器的命名,存在着两种不同的命名约定: 通过编号命名(从\$0到\$31)或者通过助记符命名(\$V0,\$A0等等)。

下面的GCC汇编输出通过编号来命名寄存器:

\lstinputlisting[caption=\Optimizing GCC 4.4.5 (\assemblyOutput),style=customasmMIPS]{patterns/00_empty/MIPS.s}

\dots 然而\IDA通过助记符来命名:

\lstinputlisting[caption=\Optimizing GCC 4.4.5 (IDA),style=customasmMIPS]{patterns/00_empty/MIPS_IDA.lst}

\myindex{MIPS!\Instructions!J}
第一条指令是跳转指令(J或JR),这将执行流导向\gls{调用者},跳转到存放在\$31(或\$RA)寄存器中的地址。

这个寄存器和ARM架构中的\ac{LR}寄存器类似。

第二条指令是\ac{NOP},这条指令什么都不会做。
我们目前可以忽略它。

\subsubsection{关于MIPS指令和寄存器名称的讨论}

在MIPS中寄存器名称和指令通常使用小写字母。
然而在本书中,为了一致性考虑,我们会坚持使用大写字母,以和本书中其他\ac{ISA}的惯例相符。

\subsection{实践中的空函数}

尽管空函数通常是毫无用处的,但是它们在底层代码中是常常出现的。

首先,用于调试的函数是很常见的,像下面这个函数:

\lstinputlisting[caption=\EN{\CCpp Code},style=customc]{patterns/00_empty/dbg_print.c}

在非DEBUG构建中 (例如,RELEASE构建), \TT{\_DEBUG}未被定义,
所以尽管在执行中\TT{dbg\_print()}函数仍会被调用,但函数体为空。

另外一种常见的软件保护方式是做出多种构建,一种提供给正版用户,一种作为试用版本。
在试用版本的构建中,会缺少一些重要的函数,像下面这样:

\lstinputlisting[caption=\EN{\CCpp代码},style=customc]{patterns/00_empty/demo.c}

\TT{save\_file()} 函数会在用户点击\TT{文件->保存}菜单时被调用。
试用版本中,会禁用该菜单项,但即使破解者强行启用了这一项,也只有空函数会被调用。

IDA会将这样的空函数以\TT{nullsub\_00},\TT{nullsub\_01}这样的名字标记。

