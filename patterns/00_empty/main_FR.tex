\section{Fonction vide}
\label{empty_func}

La fonction la plus simple possible est sans doute celle qui ne fait rien:

\lstinputlisting[caption=\EN{\CCpp Code}]{patterns/00_empty/1.c}

Compilons le!

\subsection{x86}

Voici ce que les compilateurs optimisés GCC et MSVC produisent sur une plateforme x86:

\lstinputlisting[caption=\Optimizing GCC/MSVC (\assemblyOutput)]{patterns/00_empty/1.s}

\myindex{x86!\Instructions!RET}
Il y a juste une instruction: \RET, qui retourne l'exécution vers le \gls{caller}. % TODO : translate caller ?

\subsection{ARM}

\lstinputlisting[caption=\OptimizingKeilVI (\ARMMode) ASM Output]{patterns/00_empty/1_Keil_ARM_O3.s}

L'adresse de retour n'est pas enregistrée sur la pile locale dans le \ac{ISA} ARM, mais plutôt dans le "link register", % TODO : translate ? 
donc l'instruction \INS{BX LR} force l'exécution à sauter à cette adresse---retournant effectivement l'exécution vers le \gls{caller}.

\subsection{MIPS}

Il y a deux conventions de nommage dans le monde de MIPS pour nommer les registres:
par nombre (de \$0 à \$31) ou pas un pseudo nom (\$V0, \$A0, etc.).

La sortie de l'assemblage GCC ci-dessous liste les registres par nombre:

\lstinputlisting[caption=\Optimizing GCC 4.4.5 (\assemblyOutput)]{patterns/00_empty/MIPS.s}

\dots tandis que \IDA le fait---avec les pseudo noms:

\lstinputlisting[caption=\Optimizing GCC 4.4.5 (IDA),style=customasm]{patterns/00_empty/MIPS_IDA.lst}

\myindex{MIPS!\Instructions!J}
La première instruction est l'instruction de saut(J ou JR) qui retourne le flux d'exécution vers le \gls{caller},
sautant vers l'adresse dans le registre \$31 (ou \$RA).

Ce registre est similaire à \ac{LR} en ARM.

La seconde instruction est \ac{NOP}, qui ne fait rien.
Nous pouvons l'ignorer pour l'instant.

\subsubsection{Un note à propos des Instructions MIPS/noms de registre}

Les registres et les noms des instructions dans le monde de MIPS sont traditionnellement écrits en minuscules.
Cependant, dans un souci d'homogénéité, nous allons continuer d'utiliser les lettres majuscules,
étant donné que c'est la convention suivie par tous les autres \ac{ISA}s présentés dans ce livre.

\subsection{Fonctions vides en pratique}

Malgré le fait que les fonctions vides soient inutiles, elles sont assez présentes dans le code bas niveau.

Tout d'abord, les fonctions de debogage sont assez populaires, comme celle-ci:

\lstinputlisting[caption=\EN{\CCpp Code}]{patterns/00_empty/dbg_print.c}

Dans une compilation en non-debug (e.g., ``release''), \TT{\_DEBUG} n'est pas défini,
donc la fonction \TT{dbg\_print()}, bien qu'elle soit appelée pendant l'exécution,
sera vide.

Un autre moyen de protection logicielle est de faire plusieurs compilations: une pour les clients, une pour démonstration.
La compilation de démonstration peut manquer de certaines fonctions importantes, comme ici:

\lstinputlisting[caption=\EN{\CCpp Code}]{patterns/00_empty/demo.c}

La fonction \TT{save\_file()} peut être appelée lorsque l'utilisateur clique sur le menu \TT{Fichier->Enregistrer}.
La version de démo peut être livrée avec cet item du menu désactivé, mais même si un logiciel cracker pourra l'activer,
la fonction vide avec aucun code utile sera appelée.

IDA repère de telles fonctions avec des noms comme \TT{nullsub\_00}, \TT{nullsub\_01}, etc.

