\section{Pusta funkcja}
\label{empty_func}

Najprostszą istniejącą funkcją jest funkcja, która nic nie robi:

\lstinputlisting[caption= Kod w \CCpp,style=customc]{patterns/00_empty/1.c}

Kompilujemy!

\subsection{x86}

Dla x86 i MSVC i GCC generuję się ten sam kod:

\lstinputlisting[caption=\Optimizing GCC/MSVC (\assemblyOutput),style=customasmx86]{patterns/00_empty/1.s}

\myindex{x86!\Instructions!RET}
Tu jest tylko jedna instrukcja \RET, która zwraca zarządzanie funkcji wywołującej \glslink{caller}{funkcję wywołującą}.

\subsection{ARM}

\lstinputlisting[caption=\OptimizingKeilVI (\ARMMode) ASM Output,style=customasmARM]{patterns/00_empty/1_Keil_ARM_O3.s}

Adres powrotu (\ac{RA}) w ARM zapisuję się nie w steku lokalnym, a w rejestrze \ac{LR}.
Dlatego instrukcja \INS{BX LR} przechodzi pod ten adres, czyli robi to samo - zwraca zarządzanie funkcji wywołującej.

\subsection{MIPS}

Są dwa sposoby deklaracji nazwy rejestru w MIPS. Według numeracji (od \$0 do \$31) lub nadając pseudo-nazwę (\$V0, \$A0, itd.).

Output assemblera w GCC pokazuję rejestry według numerów:

\lstinputlisting[caption=\Optimizing GCC 4.4.5 (\assemblyOutput),style=customasmMIPS]{patterns/00_empty/MIPS.s}

\dots a \IDA --- nadając pseudonazwę:

\lstinputlisting[caption=\Optimizing GCC 4.4.5 (IDA),style=customasmMIPS]{patterns/00_empty/MIPS_IDA.lst}

\myindex{MIPS!\Instructions!J}

Pierwsza instrukcja jest instrukcją skoku (J lub JR),
która zwraca zarządzanie \glslink{caller}{funkcję wywołującą}, skacząc pod adres w rejestrze \$31 (lub \$RA).

Jest to analog \ac{LR} w ARM.

Druga instrukcja to \ac{NOP}, która nic nie robi.
Na razie możemy nie zwracać na nią uwagi.

\subsubsection{Jeszcze małe co nieco o konwecji nazewnictwa w MIPS}
Nazwy rejestrów i instrukcji w MIPS tradycyjnie są zapisywane małymi literami.
Lecz my będziemy je zapisywać dużymi literami dlatego że nazwy instrukcji i rejestrów innych 
\ac{ISA} w tej książce są zapisywane dużymi.

\subsection{Puste funkcje na praktyce}

Mimo że puste funkcje są bezużyteczne, są one dość często spotykane w niskopoziomowym kodzie.

Po-pierwsze, popularne są funkcje wyprowadzające informację do logów, naprzykład:

\lstinputlisting[caption=Kod w \CCpp,style=customc]{patterns/00_empty/dbg_print.c}

Po drugie, popularnym sposobem na ochronę oprogramowania jest kompilacja kilku wersji: pierwsza dla legalnych konsumentów, druga - demonstracyjna. Demonstracyjna wersja może nie zawierać jakiejś ważnej funkcjonalności, naprzykład:
\lstinputlisting[caption=Kod w \CCpp,style=customc]{patterns/00_empty/demo.c}

Funkcja \TT{save\_file()} może być wywołana, kiedy użytkownik klika na menu \TT{File->Save}.
Demo-wersja może zawierać wyłączony przycisk menu, ale nawet jeśli kraker ją włączy, to będze wywoływany pusta funkcja w której nie ma użytecznego kodu.
IDA oznacza takie funkcje, naprzykład, w ten sposób \TT{nullsub\_00}, \TT{nullsub\_01}, itd.


