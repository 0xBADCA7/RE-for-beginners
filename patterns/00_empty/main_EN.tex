\section{Empty function}
\label{empty_func}

The simplest possible function is arguably one that does nothing:

\lstinputlisting[caption=\EN{\CCpp Code}]{patterns/00_empty/1.c}

Lets compile it!

\subsection{x86}

Here's what both the optimizing GCC and MSVC compilers produce on the x86 platform:

\lstinputlisting[caption=\Optimizing GCC/MSVC (\assemblyOutput)]{patterns/00_empty/1.s}

\myindex{x86!\Instructions!RET}
There is just one instruction: \RET, which returns execution to the \gls{caller}.

\subsection{ARM}

\lstinputlisting[caption=\OptimizingKeilVI (\ARMMode) ASM Output]{patterns/00_empty/1_Keil_ARM_O3.s}

The return address is not saved on the local stack in the ARM \ac{ISA}, but rather in the link register, 
so the \INS{BX LR} instruction causes execution to jump to that address---effectively returning execution
to the \gls{caller}.

\subsection{MIPS}

There are two naming conventions used in the world of MIPS when naming registers:
by number (from \$0 to \$31) or by pseudo name (\$V0, \$A0, etc.).

The GCC assembly output below lists registers by number:

\lstinputlisting[caption=\Optimizing GCC 4.4.5 (\assemblyOutput)]{patterns/00_empty/MIPS.s}

\dots while \IDA does it---by their pseudo names:

\lstinputlisting[caption=\Optimizing GCC 4.4.5 (IDA)]{patterns/00_empty/MIPS_IDA.lst}

\myindex{MIPS!\Instructions!J}
The first instruction is the jump instruction (J or JR) which returns the execution flow to the \gls{caller},
jumping to the address in the \$31 (or \$RA) register.

This is the register analogous to \ac{LR} in ARM.

The second instruction is \ac{NOP}, which does nothing.
We can ignore it so far.

\subsubsection{A note about MIPS instruction/register names}

Register and instruction names in the world of MIPS are traditionally written in lowercase.
However, for the sake of consistency, we'll stick to using uppercase letters,
as it is the convention followed by all other \ac{ISA}s featured this book.

\subsection{Empty functions in practice}

Despite the fact empty functions are useless, they are quite frequent in low-level code.

First of all, debugging functions are quite popular, like this one:

\lstinputlisting[caption=\EN{\CCpp Code}]{patterns/00_empty/dbg_print.c}

In non-debug build (e.g., ``release''), \TT{\_DEBUG} is not defined,
so \TT{dbg\_print()} function, despite still being called during execution,
will be empty.

Another popular way of software protection is make several builds: one for legal customers, and a demo build.
Demo build can lack some important functions, like this:

\lstinputlisting[caption=\EN{\CCpp Code}]{patterns/00_empty/demo.c}

\TT{save\_file()} function can be called when user click \TT{File->Save} menu.
Demo version may be delivered with this menu item disabled, but even if software cracker will enable it,
empty function with no useful code will be called.

IDA marks such functions with names like \TT{nullsub\_00}, \TT{nullsub\_01}, etc.

