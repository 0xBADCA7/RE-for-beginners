\newcommand{\comp}{\TT{comp()}\xspace}
\section{Pointer auf Funktionen}
\label{sec:pointerstofunctions}

\myindex{\CLanguageElements!\Pointers}
Ein Pointer auf eine Funktion ist genau wie alle anderen Pointer nur die Adresse, an der die Funktion im Codesegment
beginnt.

\myindex{Callbacks}
Sie werden oft für Callback-Funktionen verwendet\footnote{\href{http://go.yurichev.com/17071}{wikipedia}}.

Bekannte Beispiele hierfür sind:

\begin{itemize}
\item \qsort\footnote{\href{http://go.yurichev.com/17072}{wikipedia}},
{\TT{atexit()}}\footnote{\url{http://go.yurichev.com/17073}} aus der Standard C Bibliothek; 

\item *NIX OS Signale\footnote{\href{http://go.yurichev.com/17074}{wikipedia}};

\item Thread Start: \TT{CreateThread()} (win32), \TT{pthread\_create()} (POSIX);

\item viele win32 Funktionen wie \TT{EnumChildWindows()}\footnote{\href{http://go.yurichev.com/17075}{MSDN}}.

\item viele Stellen im Linux kernel, zum Beispiel werden die Treiber für das Dateisystem über Callback-Funktionen
aufgerufen:
\url{http://go.yurichev.com/17076}

\item Die GCC Plugin-Functions werden ebenfalls über Callbacks aufgerufen: 
\url{http://go.yurichev.com/17077}
\item Ein weiteres Beispiel für Pointer auf Funktionen ist eine Tabelle im \q{dwm} Linux Fenstermanager, der Shortcuts
definiert. Jeder Shortcur hat eine zugehörige Funktion aufzurufen, wenn eine spezielle Tastenkombination gedrückt wird:
\href{http://go.yurichev.com/17078}{GitHub}. Wie wir sehen können, ist eine solche Tabelle einfacher zu handhaben als
ein großer switch()-Ausdruck.
\end{itemize}

\myindex{\CStandardLibrary!qsort()}
Die Funktion \qsort ist eine Implementierung von Quicksort in der \CCpp Standardbibliothek.
Die Funktion ist in der Lage jeden Datentyp zu sortieren, solange dieser über eine Funktion zum Vergleich von zwei
Elementen verfügt und \qsort in der Lage ist, diese aufzurufen.

Die Vergleichsfunktion kann wie folgt definiert werden:

\begin{lstlisting}
int (*compare)(const void *, const void *)
\end{lstlisting}

Verwenden wir das folgende Beispiel:

\lstinputlisting[numbers=left,label=qsort_c_src,style=customc]{patterns/18_pointers_to_functions/17_1.c}

\subsection{MSVC}

Kompilieren wir es in MSVC 2010 (einige Teile wurden zur Verkürzung weggelassen) mit der Option \TT{\Ox}:

\lstinputlisting[caption=\Optimizing MSVC 2010: /GS- /MD,style=customasmx86]{patterns/18_pointers_to_functions/17_2_msvc_Ox.asm}
Bis hierhin nicht überraschend. 
Als viertes Argument wird die Adresse des Labels \TT{\_comp} übergeben, die eine Stelle ist, an der sich \comp befindet,
oder, mit anderen Worten: die Adresse des ersten Befehls dieser Funktion.

Wie kann \qsort diese Funktion aufrufen?

\myindex{Windows!MSVCR80.DLL}
Schauen wir uns diese Funktion, die sich in MSVCR80.DLL (ein MSVC DLL Modul mit Standard-C-Bibliotheksfunktionen)
befindet, näher an:

\lstinputlisting[caption=MSVCR80.DLL,style=customasmx86]{patterns/18_pointers_to_functions/17_3_MSVCR.lst}

\TT{comp}---ist das vierte Funktionsargument.
Hier wird der Control Flow an die Adresse in \TT{comp} Argument übergeben.
Davor werden zwei Argumente für \comp vorbereitet. Das Ergebnis wird nach der Ausführung überprüft.

Das ist ein Grund, warum es gefährlich ist, Pointer auf Funktionen zu verwenden.
Erstens: Wenn man \qsort mit einem falschen Funktionspointer aufruft, könnte \qsort den Control Flow an eine falsche
Stelle übergeben, der Prozess könnte abstürzen und dieser Bug wäre sehr schwer zu finden.

Zweitens: Die Typen der Callback-Funktion müssen ganz genau passend sein, denn der Aufruf der falschen Funktion mit
falschen Argumenten oder falschen Typen kann zu ernsthaften Problemen führen. Das Abstürzen des Prozesses ist hier kein
Problem~---das Problem ist die Ursache für den Absturz zu finden~---, da der Compiler während des Kompilierens die
potentiellen Probleme nicht entdeckt hat.

\clearpage
\subsubsection{MSVC + \olly}
\myindex{\olly}
Laden wir unser Beispiel in \olly und setzen einen Breakpoint auf \comp.
Wir sehen wie dir Werte beim ersten Aufruf von \comp verglichen werden:

\begin{figure}[H]
\centering
\myincludegraphics{patterns/18_pointers_to_functions/olly1.png}
\caption{\olly: erster Aufruf von \comp}
\label{fig:qsort_olly1}
\end{figure}
\olly zeigt die verglichenen Werte im Fenster unter dem Codefenster an.
Wir können auch erkennen, dass der \ac{SP} auf \ac{RA} zeigt, wo sich die Funktion \qsort (innerhalb von
\TT{MSVCR100.DLL}) befindet.

\clearpage
Durch Drücken von (F8) bis zum Befehl \TT{RETN} und einmaliges weiteres Drücken von F8 kehren wir zur Funktion
\qsort zurück:

\begin{figure}[H]
\centering
\myincludegraphics{patterns/18_pointers_to_functions/olly2.png}
\caption{\olly: der Code in \qsort direkt nach dem Aufruf von \comp}
\label{fig:qsort_olly2}
\end{figure}

Das war beispielhaft ein Aufruf der Vergleichsfunktion.

\clearpage
Hier ist noch ein Screenshot von dem Moment des zweiten Aufrufs von \comp---nun müssen die verglichenen Werte andere
sein:

\begin{figure}[H]
\centering
\myincludegraphics{patterns/18_pointers_to_functions/olly3.png}
\caption{\olly: zweiter Aufruf von \comp}
\label{fig:qsort_olly3}
\end{figure}


\subsubsection{MSVC + Tracer}
\myindex{tracer}
Schauen wir uns auch an, welche Paare verglichen werden.
Diese 10 Zahlen werden sortiert: 
1892, 45, 200, -98, 4087, 5, -12345, 1087, 88, -100000.

Wir erhalten die Adressen des ersten \CMP Befehls in \comp, d.i. \TT{0x0040100C} und wir haben hier einen Breakpoint
gesetzt:

\begin{lstlisting}
tracer.exe -l:17_1.exe bpx=17_1.exe!0x0040100C
\end{lstlisting}

Jetzt erhalten wir Informationen über die Register am Breakpoint:

\begin{lstlisting}
PID=4336|New process 17_1.exe
(0) 17_1.exe!0x40100c
EAX=0x00000764 EBX=0x0051f7c8 ECX=0x00000005 EDX=0x00000000
ESI=0x0051f7d8 EDI=0x0051f7b4 EBP=0x0051f794 ESP=0x0051f67c
EIP=0x0028100c
FLAGS=IF
(0) 17_1.exe!0x40100c
EAX=0x00000005 EBX=0x0051f7c8 ECX=0xfffe7960 EDX=0x00000000
ESI=0x0051f7d8 EDI=0x0051f7b4 EBP=0x0051f794 ESP=0x0051f67c
EIP=0x0028100c
FLAGS=PF ZF IF
(0) 17_1.exe!0x40100c
EAX=0x00000764 EBX=0x0051f7c8 ECX=0x00000005 EDX=0x00000000
ESI=0x0051f7d8 EDI=0x0051f7b4 EBP=0x0051f794 ESP=0x0051f67c
EIP=0x0028100c
FLAGS=CF PF ZF IF
...
\end{lstlisting}

Filtern wir \TT{EAX} und \TT{ECX} heraus, so erhalten wir:

\begin{lstlisting}
EAX=0x00000764 ECX=0x00000005
EAX=0x00000005 ECX=0xfffe7960
EAX=0x00000764 ECX=0x00000005
EAX=0x0000002d ECX=0x00000005
EAX=0x00000058 ECX=0x00000005
EAX=0x0000043f ECX=0x00000005
EAX=0xffffcfc7 ECX=0x00000005
EAX=0x000000c8 ECX=0x00000005
EAX=0xffffff9e ECX=0x00000005
EAX=0x00000ff7 ECX=0x00000005
EAX=0x00000ff7 ECX=0x00000005
EAX=0xffffff9e ECX=0x00000005
EAX=0xffffff9e ECX=0x00000005
EAX=0xffffcfc7 ECX=0xfffe7960
EAX=0x00000005 ECX=0xffffcfc7
EAX=0xffffff9e ECX=0x00000005
EAX=0xffffcfc7 ECX=0xfffe7960
EAX=0xffffff9e ECX=0xffffcfc7
EAX=0xffffcfc7 ECX=0xfffe7960
EAX=0x000000c8 ECX=0x00000ff7
EAX=0x0000002d ECX=0x00000ff7
EAX=0x0000043f ECX=0x00000ff7
EAX=0x00000058 ECX=0x00000ff7
EAX=0x00000764 ECX=0x00000ff7
EAX=0x000000c8 ECX=0x00000764
EAX=0x0000002d ECX=0x00000764
EAX=0x0000043f ECX=0x00000764
EAX=0x00000058 ECX=0x00000764
EAX=0x000000c8 ECX=0x00000058
EAX=0x0000002d ECX=0x000000c8
EAX=0x0000043f ECX=0x000000c8
EAX=0x000000c8 ECX=0x00000058
EAX=0x0000002d ECX=0x000000c8
EAX=0x0000002d ECX=0x00000058
\end{lstlisting}

Das entspricht 34 Paaren.
Der Quicksort-Algorithmus benötigt hier also 34 Vergleichsoperationen um die 10 Zahlen zu sortieren.

\clearpage
\subsubsection{MSVC + Tracer (Codebetrachtung)}
\myindex{tracer}
Wir können auch alle möglichen Registerwerte über die Ablaufverfolgung sammeln und in \IDA anzeigen.

Verfolgen wir also alle Befehle in \comp:

\begin{lstlisting}
tracer.exe -l:17_1.exe bpf=17_1.exe!0x00401000,trace:cc
\end{lstlisting}

Wir erhalten ein .idc-Skript und laden dieses in \IDA:

\begin{figure}[H]
\centering
\myincludegraphics{patterns/18_pointers_to_functions/tracer_cc.png}
\caption{Tracer und IDA. Werte teilweise rechts abgeschnitten}  
\label{fig:qsort_tracer_cc}
\end{figure}
\IDA hat der Funktion einen Namen gegeben (PtFuncCompare)---denn \IDA erkennt, dass der Pointer auf diese Funktion an
\qsort übergeben wird.

Wir sehen, dass die Pointer $a$ und $b$ auf diverse Stellen im Array zeigen, aber die Schrittweite ist stets 4, da im
Array 32-Bit-Werte gespeichert sind.

Wir sehen auch, dass die Befehle an den Stellen \TT{0x401010} und \TT{0x401012} nie ausgeführt wurden (deshalb wurden
sie weiß gelassen). Da liegt daran, dass \comp nie 0 zurückgegeben hat, da sich im Array keine zwei gleichen Elemente
befinden.

\subsection{GCC}

Kein großer Unterschied:

\begin{lstlisting}[caption=GCC,style=customasmx86]
                lea     eax, [esp+40h+var_28]
                mov     [esp+40h+var_40], eax
                mov     [esp+40h+var_28], 764h
                mov     [esp+40h+var_24], 2Dh
                mov     [esp+40h+var_20], 0C8h
                mov     [esp+40h+var_1C], 0FFFFFF9Eh
                mov     [esp+40h+var_18], 0FF7h
                mov     [esp+40h+var_14], 5
                mov     [esp+40h+var_10], 0FFFFCFC7h
                mov     [esp+40h+var_C], 43Fh
                mov     [esp+40h+var_8], 58h
                mov     [esp+40h+var_4], 0FFFE7960h
                mov     [esp+40h+var_34], offset comp
                mov     [esp+40h+var_38], 4
                mov     [esp+40h+var_3C], 0Ah
                call    _qsort
\end{lstlisting}

Die Funktion \comp:

\begin{lstlisting}[style=customasmx86]
                public comp
comp            proc near

arg_0           = dword ptr  8
arg_4           = dword ptr  0Ch

                push    ebp
                mov     ebp, esp
                mov     eax, [ebp+arg_4]
                mov     ecx, [ebp+arg_0]
                mov     edx, [eax]
                xor     eax, eax
                cmp     [ecx], edx
                jnz     short loc_8048458
                pop     ebp
                retn
loc_8048458:
                setnl   al
                movzx   eax, al
                lea     eax, [eax+eax-1]
                pop     ebp
                retn
comp            endp
\end{lstlisting}

\myindex{Linux!libc.so.6}
Die Implementierung von \qsort befindet sich in \TT{libc.so.6} und ist eigentlich nur ein Wrapper für \TT{qsort\_r()}.

Sie ruft \TT{quicksort()} auf, wo unsere selbst definierte Funktion über einen übergebenen Pointer aufgerufen wird:

\begin{lstlisting}[caption=(file libc.so.6{,} glibc version---2.10.1),style=customasmx86]
...
.text:0002DDF6                 mov     edx, [ebp+arg_10]
.text:0002DDF9                 mov     [esp+4], esi
.text:0002DDFD                 mov     [esp], edi
.text:0002DE00                 mov     [esp+8], edx
.text:0002DE04                 call    [ebp+arg_C]
...
\end{lstlisting}

\subsubsection{GCC + GDB (mit Quellcode)}
\myindex{GDB}
Natürlich kennen wir den C-Quellcode unseres Beispiels (\myref{qsort_c_src}), sodass wir einen Breakpoint ($b$) auf die
Zeile (11--die Zeile, in der der erste Vergleich auftaucht), setzen können.
Wir müssen das Beispiel mit hinzugefügten Debugging-Informationen kompilieren (\TT{-g}), sodass die Tabelle mit den
Adressen und zugehörigen Zeilennummern verfügbar ist.

Wir können mit den Variablennamen (\TT{p}) die Werte auch ausgeben:
Die Debugging-Informationen verraten uns auch, welche Register und/oder Elemente auf dem lokalen Stack welche Variablen
enthalten.

\myindex{Glibc}
Wir sehen auch den Stack (\TT{bt}) und können folgern, dass es eine Zwischenfunktion \TT{msort\_with\_tmp()} in Glibc
geben muss.

\lstinputlisting[caption=GDB session]{patterns/18_pointers_to_functions/GDB_source.txt}

\subsubsection{GCC + GDB (ohne Quellcode)}
\myindex{GDB}
Oft gibt es aber keinen Quellcode, sodass wir die \comp Funktion (\TT{disas}) disassemblieren und den ersten \CMP Befehl
finden müssen und dann an dieser Adresse einen Breakpoint ($b$) setzen.

An jedem Breakpoint werden wir alle Registerinhalte (\TT{info registers}) in den Dump schreiben.
Die Informationen zum Stack (\TT{bt}) sind ebenfalls verfügbar, aber nur teilweise vollständig: es gibt keine
Informationen zu den Zeilennummern in \comp.

\lstinputlisting[caption=GDB session]{patterns/18_pointers_to_functions/GDB_no_source.txt}

\subsection{Gefahr von Pointern auf Funktionen}
Wie wir sehen können, erwartet die Funktion \qsort einen Pointer auf eine Funktion, die zwei \IT{void*} Argumente nimmt
und einen Integer zurückgibt.
Wenn man viele Vergleichsfunktionen im Code finden (eine vergleicht Strings, ein andere Integer, etc.) kommt man hier
leicht durcheinander.
Man könnte versuchen, ein Array aus Strings mit der Vergleichsfunktion für Integer zu sortieren und der Compiler würde
bei diesem Bug nicht einmal eine Warnung ausgeben.
