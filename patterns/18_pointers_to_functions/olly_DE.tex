\clearpage
\subsubsection{MSVC + \olly}
\myindex{\olly}
Laden wir unser Beispiel in \olly und setzen einen Breakpoint auf \comp.
Wir sehen wie dir Werte beim ersten Aufruf von \comp verglichen werden:

\begin{figure}[H]
\centering
\myincludegraphics{patterns/18_pointers_to_functions/olly1.png}
\caption{\olly: erster Aufruf von \comp}
\label{fig:qsort_olly1}
\end{figure}
\olly zeigt die verglichenen Werte im Fenster unter dem Codefenster an.
Wir können auch erkennen, dass der \ac{SP} auf \ac{RA} zeigt, wo sich die Funktion \qsort (innerhalb von
\TT{MSVCR100.DLL}) befindet.

\clearpage
Durch Drücken von (F8) bis zum Befehl \TT{RETN} und einmaliges weiteres Drücken von F8 kehren wir zur Funktion
\qsort zurück:

\begin{figure}[H]
\centering
\myincludegraphics{patterns/18_pointers_to_functions/olly2.png}
\caption{\olly: der Code in \qsort direkt nach dem Aufruf von \comp}
\label{fig:qsort_olly2}
\end{figure}

Das war beispielhaft ein Aufruf der Vergleichsfunktion.

\clearpage
Hier ist noch ein Screenshot von dem Moment des zweiten Aufrufs von \comp---nun müssen die verglichenen Werte andere
sein:

\begin{figure}[H]
\centering
\myincludegraphics{patterns/18_pointers_to_functions/olly3.png}
\caption{\olly: zweiter Aufruf von \comp}
\label{fig:qsort_olly3}
\end{figure}
