\section{Метод}

Когда автор этой книги учил Си, а затем \Cpp, он просто писал небольшие фрагменты кода, компилировал и смотрел, что 
получилось на ассемблере. Так было намного проще понять%
\footnote{Честно говоря, он и до сих пор так делает, когда не понимает, как работает некий код.}.
Он делал это такое количество раз, что связь между кодом на \CCpp и тем, что генерирует компилятор, вбилась в его подсознание достаточно глубоко.
После этого не трудно, глядя на код на ассемблере, сразу в общих чертах понимать, что там было написано на Си. 
Возможно это поможет кому-то ещё.

%Здесь много примеров и для x86/x64 и для ARM.
%Те, кто уже хорошо знаком с одной из архитектур, могут легко пролистывать страницы.

Иногда здесь используются достаточно древние компиляторы, чтобы получить самый короткий (или простой) фрагмент кода.

Кстати, есть очень неплохой вебсайт где можно делать всё то же самое, с разными компиляторами, вместо того чтобы инсталлировать
их у себя.
Вы можете использовать и его: \url{http://gcc.beta.godbolt.org/}.

\section*{\Exercises}

Когда автор этой книги учил ассемблер, он также часто компилировал короткие функции на Си и затем постепенно 
переписывал их на ассемблер, с целью получить как можно более короткий код.
Наверное, этим не стоит заниматься в наше время на практике (потому что конкурировать с современными
компиляторами в плане эффективности очень трудно), но это очень хороший способ разобраться в ассемблере
лучше.
Так что вы можете взять любой фрагмент кода на ассемблере в этой книге и постараться сделать его короче.
Но не забывайте о тестировании своих результатов.

\section*{Уровни оптимизации и отладочная информация}

Исходный код можно компилировать различными компиляторами с различными уровнями оптимизации.
В типичном компиляторе этих уровней около трёх, где нулевой уровень~--- отключить оптимизацию.
Различают также уровни оптимизации кода по размеру и по скорости.
Неоптимизирующий компилятор работает быстрее, генерирует более понятный (хотя и более объемный) код.
Оптимизирующий компилятор работает медленнее и старается сгенерировать более быстрый (хотя и не обязательно краткий) код.
Наряду с уровнями оптимизации компилятор может включать в конечный файл отладочную информацию,
производя таким образом код, который легче отлаживать.
Одна очень важная черта отладочного кода в том, что он может содержать
связи между каждой строкой в исходном коде и адресом в машинном коде.
Оптимизирующие компиляторы обычно генерируют код, где целые строки из исходного кода
могут быть оптимизированы и не присутствовать в итоговом машинном коде.
Практикующий reverse engineer обычно сталкивается с обеими версиями, потому что некоторые разработчики
включают оптимизацию, некоторые другие --- нет. Вот почему мы постараемся поработать с примерами для обеих версий.
