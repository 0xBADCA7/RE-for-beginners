\subsubsection{ARM: 3 arguments}

Le schéma ARM traditionnel pour passer des arguments (convention d'appel) se
comporte de cette façon:
les 4 premiers arguments sont passés par les registres \Reg{0}-\Reg{3}; les autres
par la pile.
Cela ressemble au schéma de passage des arguments dans
fastcall~(\myref{fastcall}) ou win64~(\myref{sec:callingconventions_win64}).

\myparagraph{ARM 32-bit}

\mysubparagraph{\NonOptimizingKeilVI (\ARMMode)}

\begin{lstlisting}[caption=\NonOptimizingKeilVI (\ARMMode),style=customasmARM]
.text:00000000 main
.text:00000000 10 40 2D E9   STMFD   SP!, {R4,LR}
.text:00000004 03 30 A0 E3   MOV     R3, #3
.text:00000008 02 20 A0 E3   MOV     R2, #2
.text:0000000C 01 10 A0 E3   MOV     R1, #1
.text:00000010 08 00 8F E2   ADR     R0, aADBDCD     ; "a=%d; b=%d; c=%d"
.text:00000014 06 00 00 EB   BL      __2printf
.text:00000018 00 00 A0 E3   MOV     R0, #0          ; retourner 0
.text:0000001C 10 80 BD E8   LDMFD   SP!, {R4,PC}
\end{lstlisting}

Donc, les 4 premiers arguments sont passés par les registres \Reg{0}-\Reg{3} dans
cet ordre:
un pointeur sur la chaîne de format de \printf dans \Reg{0}, puis 1 dans \Reg{1},
2 dans \Reg{2} et 3 dans \Reg{3}.
L'instruction en \GTT{0x18} écrit 0 dans \Reg{0}---c'est la déclaration C
de \IT{return 0}.

\OptimizingKeilVI génère le même code.

\mysubparagraph{\OptimizingKeilVI (\ThumbMode)}

\begin{lstlisting}[caption=\OptimizingKeilVI (\ThumbMode),style=customasmARM]
.text:00000000 main
.text:00000000 10 B5        PUSH    {R4,LR}
.text:00000002 03 23        MOVS    R3, #3
.text:00000004 02 22        MOVS    R2, #2
.text:00000006 01 21        MOVS    R1, #1
.text:00000008 02 A0        ADR     R0, aADBDCD     ; "a=%d; b=%d; c=%d"
.text:0000000A 00 F0 0D F8  BL      __2printf
.text:0000000E 00 20        MOVS    R0, #0
.text:00000010 10 BD        POP     {R4,PC}
\end{lstlisting}

Il n'y a pas de différence significative avec le code non optimisé pour le mode ARM.

\mysubparagraph{\OptimizingKeilVI (\ARMMode) + supprimons le retour}
\label{ARM_B_to_printf}

Retravaillons légèrement l'exemple en supprimant \IT{return 0}:

\begin{lstlisting}[style=customc]
#include <stdio.h>

void main()
{
	printf("a=%d; b=%d; c=%d", 1, 2, 3);
};
\end{lstlisting}

Le résultat est quelque peu inhabituel:

\begin{lstlisting}[caption=\OptimizingKeilVI (\ARMMode),style=customasmARM]
.text:00000014 main
.text:00000014 03 30 A0 E3   MOV     R3, #3
.text:00000018 02 20 A0 E3   MOV     R2, #2
.text:0000001C 01 10 A0 E3   MOV     R1, #1
.text:00000020 1E 0E 8F E2   ADR     R0, aADBDCD     ; "a=%d; b=%d; c=%d\n"
.text:00000024 CB 18 00 EA   B       __2printf
\end{lstlisting}

\myindex{ARM!\Registers!Link Register}
\myindex{ARM!\Instructions!B}
\myindex{Épilogue de la fonction}
C'est la version optimisée (\Othree) pour le mode ARM et cette fois nous voyons
\INS{B} comme dernière instruction au lieu du \INS{BL} habituel.
Une autre différence entre cette version optimisée et la précédente (compilée
sans optimisation) est l'absence de fonctions prologue et épilogue (les instructions
qui préservent les valeurs des registres \Reg{0} et \ac{LR}).
\myindex{x86!\Instructions!JMP}
L'instruction \INS{B} saute simplement à une autre adresse, sans manipuler le registre
\ac{LR}, de façon similaire au \JMP en x86.
Pourquoi est-ce que fonctionne? Parce ce code est en fait bien équivalent au précédent.
Il y a deux raisons principales: 1) Ni la pile ni \ac{SP} (\glslink{stack pointer}{pointeur de pile})
ne sont modifiés;
2) l'appel à \printf est la dernière instruction, donc il ne se passe rien après.
A la fin, la fonction \printf rend simplement le contrôle à l'adresse stockée
dans \ac{LR}.
Puisqu'\ac{LR} contient actuellement l'adresse du point depuis lequel notre fonction
a été appelée alors le contrôle après \printf sera redonné à ce point.
Par conséquent, nous n'avons pas besoin de sauver \ac{LR} car il ne nous est pas
nécessaire de le modifier.
Et il ne nous est non plus pas nécessaire de modifier \ac{LR} car il n'y a pas d'autre
appel de fonction excepté \printf. Par ailleurs, après cet appel nous ne faisons
rien d'autre!
C'est la raison pour laquelle une telle optimisation est possible.

Cette optimisation est souvent utilisée dans les fonctions où la dernière déclaration
est un appel à une autre fonction.
Un exemple similaire est présenté ici:
\myref{jump_to_last_printf}.

\myparagraph{ARM64}

\mysubparagraph{GCC (Linaro) 4.9 \NonOptimizing}

\lstinputlisting[caption=GCC (Linaro) 4.9 \NonOptimizing,style=customasmARM]{patterns/03_printf/ARM/ARM3_O0_FR.lst}

\myindex{ARM!\Instructions!STP}

La première instruction \INS{STP} (\IT{Store Pair}) sauve \ac{FP} (X29) et \ac{LR} (X30) sur la pile.

La seconde instruction, \INS{ADD X29, SP, 0} crée la pile.
Elle écrit simplement la valeur de \ac{SP} dans X29.

\myindex{ARM!\Instructions!ADRP/ADD pair}
Ensuite nous voyons la paire d'instructions habituelle \INS{ADRP}/\ADD, qui crée
le pointeur sur la chaîne.
\myindex{ARM64!lo12}
\IT{lo12} signifie les 12 bits de poids faible, i.e., le linker va écrire les 12 bits
de poids faible de l'adresse LC1 dans l'opcode de l'instruction \ADD.
\GTT{\%d} dans la chaîne de format de \printf est un \Tint 32-bit, les 1, 2 et
3 sont chargés dans les parties 32-bit des registres.

GCC (Linaro) 4.9 \Optimizing génère le même code.

