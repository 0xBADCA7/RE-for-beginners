\subsubsection{x64: 8 argumentos}

\myindex{x86-64}
\label{example_printf8_x64}
Para ver como outros argumentos são passados pela pilha,
vamos mudar nosso exemplo novamente aumentando o numero de argumentos para 9 (\printf + 8 variáveis \Tint):

\lstinputlisting[style=customc]{patterns/03_printf/2.c}

\myparagraph{MSVC}

Como mencionado anteriormente, os primeiros 4 argumentos tem de ser passados pelos registradores \RCX, \RDX, \Reg{8}, \Reg{9} no Win64, enquanto o resto pela pilha.
Isso é exatamente o que veremos aqui.
Entretanto, a instrução \MOV, ao invés de \PUSH, é usada para preparar a pilha, portanto os valores são armazenados de uma maneira direta.

% TODO translate
\lstinputlisting[caption=MSVC 2012 x64,style=customasmx86]{patterns/03_printf/x86/2_MSVC_x64_EN.asm}

Um leitor observativo pode se indagar por que são alocados 8 bytes para valores int quando 4 já é suficiente?
Sim, mas lembre-se: 8 bytes são alocados para qualquer tipo de informação menor do que 64 bits.
Isso é estabelecido com o objetivo de ser conveniente: é mais facil calcular o endereço de um argumento arbitrário.
Além do mais, eles são alocados em endereços de memórias alinhados. Da mesma maneira no 32-bits: 4 bytes são reservados para todos os tipos de informação.

% also for local variables?

\PTBRph{}

