\subsubsection{x86: 3 аргумента}

\myparagraph{MSVC}

Компилируем при помощи MSVC 2010 Express, и в итоге получим:

\begin{lstlisting}[style=customasm]
$SG3830	DB	'a=%d; b=%d; c=%d', 00H

...

	push	3
	push	2
	push	1
	push	OFFSET $SG3830
	call	_printf
	add	esp, 16
\end{lstlisting}

Всё почти то же, за исключением того, что теперь видно, что аргументы для \printf заталкиваются в стек в обратном порядке: самый первый аргумент заталкивается последним.

Кстати, вспомним, что переменные типа \Tint в 32-битной системе, как известно, имеют ширину 32 бита, это 4 байта.

Итак, у нас всего 4 аргумента. $4*4 = 16$~--- именно 16 байт занимают в стеке указатель на строку плюс ещё 3 числа типа \Tint.

\myindex{x86!\Instructions!ADD}
\myindex{x86!\Registers!ESP}
\myindex{cdecl}
Когда при помощи инструкции \INS{ADD ESP, X} корректируется \glslink{stack pointer}{указатель стека} \ESP 
после вызова какой-либо функции, зачастую можно сделать вывод о том, сколько аргументов 
у вызываемой функции было, разделив X на 4.

Конечно, это относится только к cdecl-методу передачи аргументов через стек, и только для 32-битной среды.

См. также в соответствующем разделе о способах передачи аргументов через стек ~(\myref{sec:callingconventions}).

Иногда бывает так, что подряд идут несколько вызовов разных функций, но стек корректируется только один раз, после последнего вызова:

\begin{lstlisting}[style=customasm]
push a1
push a2
call ...
...
push a1
call ...
...
push a1
push a2
push a3
call ...
add esp, 24
\end{lstlisting}

Вот пример из реальной жизни:

\lstinputlisting[caption=x86,style=customasm]{patterns/03_printf/x86/add_example_RU.lst}

\clearpage
\myparagraph{MSVC и \olly}
\myindex{\olly}

Попробуем этот же пример в \olly.
Это один из наиболее популярных win32-отладчиков пользовательского режима.
Мы можем компилировать наш пример в MSVC 2012 
с опцией \GTT{/MD} что означает линковать с библиотекой \GTT{MSVCR*.DLL},
чтобы импортируемые функции были хорошо видны в отладчике.

Затем загружаем исполняемый файл в \olly.
Самая первая точка останова в \GTT{ntdll.dll}, нажмите F9 (запустить).
Вторая точка останова в \ac{CRT}-коде.
Теперь мы должны найти функцию \main.

Найдите этот код, прокрутив окно кода до самого верха (MSVC располагает функцию \main в самом начале секции кода): 

\begin{figure}[H]
\centering
\myincludegraphics{patterns/03_printf/x86/olly3_1.png}
\caption{\olly: самое начало функции \main}
\label{fig:printf3_olly_1}
\end{figure}

Кликните на инструкции \INS{PUSH EBP}, нажмите F2 (установка точки останова) и нажмите F9 (запустить).
Нам нужно произвести все эти манипуляции, чтобы пропустить \ac{CRT}-код, потому что нам он пока
не интересен.

\clearpage
Нажмите F8 (\stepover) 6 раз, т.е. пропустить 6 инструкций:

\begin{figure}[H]
\centering
\myincludegraphics{patterns/03_printf/x86/olly3_2.png}
\caption{\olly: перед исполнением \printf}
\label{fig:printf3_olly_2}
\end{figure}

Теперь \ac{PC} указывает на инструкцию \INS{CALL printf}.
\olly, как и другие отладчики, подсвечивает регистры со значениями, которые изменились.
Поэтому каждый раз когда мы нажимаем F8, \EIP изменяется и его значение подсвечивается красным.
\ESP также меняется, потому что значения заталкиваются в стек.\\
\\
Где находятся эти значения в стеке?
Посмотрите на правое нижнее окно в отладчике:

\begin{figure}[H]
\centering
\includegraphics[width=0.5\textwidth]{patterns/03_printf/x86/olly3_stack.png}

\caption{\olly: стек с сохраненными значениями (красная рамка добавлена в графическом редакторе)}
\end{figure}

Здесь видно 3 столбца: адрес в стеке, значение в стеке и ещё дополнительный комментарий
от \olly. 
\olly понимает \printf{}-строки, так что он показывает здесь и строку и 3 значения \IT{привязанных} к ней.

Можно кликнуть правой кнопкой мыши на строке формата, кликнуть на \q{Follow in dump}
и строка формата появится в окне слева внизу, где всегда виден какой-либо участок памяти.
Эти значения в памяти можно редактировать.
Можно изменить саму строку формата, и тогда результат работы нашего примера будет другой.
В данном случае пользы от этого немного, но для упражнения это полезно,
чтобы начать чувствовать как тут всё работает.

\clearpage
Нажмите F8 (\stepover).

В консоли мы видим вывод:

\lstinputlisting{patterns/03_printf/x86/console.txt}

Посмотрим как изменились регистры и состояние стека: 

\begin{figure}[H]
\centering
\myincludegraphics{patterns/03_printf/x86/olly3_3.png}
\caption{\olly после исполнения \printf}
\label{fig:printf3_olly_3}
\end{figure}

Регистр \EAX теперь содержит \GTT{0xD} (13).
Всё верно: \printf возвращает количество выведенных символов.
Значение \EIP изменилось. Действительно, теперь здесь адрес инструкции после \INS{CALL printf}.
Значения регистров \ECX и \EDX также изменились.
Очевидно, внутренности функции \printf используют их для каких-то своих нужд.

Очень важно то, что значение \ESP не изменилось. И аргументы-значения в стеке также!
Мы ясно видим здесь и строку формата и соответствующие ей 3 значения, они всё ещё здесь.
Действительно, по соглашению вызовов \IT{cdecl}, вызываемая функция не возвращает \ESP назад.
Это должна делать вызывающая функция (\gls{caller}).

\clearpage
Нажмите F8 снова, чтобы исполнилась инструкция \INS{ADD ESP, 10}:

\begin{figure}[H]
\centering
\myincludegraphics{patterns/03_printf/x86/olly3_4.png}
\caption{\olly: после исполнения инструкции \INS{ADD ESP, 10}}
\label{fig:printf3_olly_4}
\end{figure}

\ESP изменился, но значения всё ещё в стеке!
Конечно, никому не нужно заполнять эти значения нулями или что-то в этом роде.
Всё что выше указателя стека (\ac{SP}) 
это \IT{шум} или \IT{\garbage{}} и не имеет особой ценности.
Было бы очень затратно по времени очищать ненужные элементы стека, к тому же, никому это и не нужно.

\myparagraph{GCC}

Скомпилируем то же самое в Linux при помощи GCC 4.4.1 и посмотрим на результат в \IDA:

\begin{lstlisting}[style=customasm]
main      proc near

var_10    = dword ptr -10h
var_C     = dword ptr -0Ch
var_8     = dword ptr -8
var_4     = dword ptr -4

          push    ebp
          mov     ebp, esp
          and     esp, 0FFFFFFF0h
          sub     esp, 10h
          mov     eax, offset aADBDCD ; "a=%d; b=%d; c=%d"
          mov     [esp+10h+var_4], 3
          mov     [esp+10h+var_8], 2
          mov     [esp+10h+var_C], 1
          mov     [esp+10h+var_10], eax
          call    _printf
          mov     eax, 0
          leave
          retn
main      endp
\end{lstlisting}

Можно сказать, что этот короткий код, созданный GCC, отличается от кода MSVC только способом помещения 
значений в стек.
Здесь GCC снова работает со стеком напрямую без \PUSH/\POP.

\myparagraph{GCC и GDB}
\myindex{GDB}

Попробуем также этот пример и в \ac{GDB} в Linux.

\GTT{-g} означает генерировать отладочную информацию в выходном исполняемом файле.

\begin{lstlisting}
$ gcc 1.c -g -o 1
\end{lstlisting}

\begin{lstlisting}
$ gdb 1
GNU gdb (GDB) 7.6.1-ubuntu
...
Reading symbols from /home/dennis/polygon/1...done.
\end{lstlisting}

\begin{lstlisting}[caption=установим точку останова на \printf]
(gdb) b printf
Breakpoint 1 at 0x80482f0
\end{lstlisting}

Запукаем.
У нас нет исходного кода функции, поэтому \ac{GDB} не может его показать.

\begin{lstlisting}
(gdb) run
Starting program: /home/dennis/polygon/1 

Breakpoint 1, __printf (format=0x80484f0 "a=%d; b=%d; c=%d") at printf.c:29
29	printf.c: No such file or directory.
\end{lstlisting}

Выдать 10 элементов стека. Левый столбец~--- это адрес в стеке.

\begin{lstlisting}
(gdb) x/10w $esp
0xbffff11c:	0x0804844a	0x080484f0	0x00000001	0x00000002
0xbffff12c:	0x00000003	0x08048460	0x00000000	0x00000000
0xbffff13c:	0xb7e29905	0x00000001
\end{lstlisting}

Самый первый элемент это \ac{RA} (\GTT{0x0804844a}).
Мы можем удостовериться в этом, дизассемблируя память по этому адресу:

\begin{lstlisting}[label=NOP_as_XCHG_example,style=customasm]
(gdb) x/5i 0x0804844a
   0x804844a <main+45>:	mov    $0x0,%eax
   0x804844f <main+50>:	leave  
   0x8048450 <main+51>:	ret    
   0x8048451:	xchg   %ax,%ax
   0x8048453:	xchg   %ax,%ax
\end{lstlisting}

Две инструкции \INS{XCHG} это холостые инструкции, аналогичные \ac{NOP}.

Второй элемент (\GTT{0x080484f0}) это адрес строки формата:

\begin{lstlisting}
(gdb) x/s 0x080484f0
0x80484f0:	"a=%d; b=%d; c=%d"
\end{lstlisting}

Остальные 3 элемента (1, 2, 3) это аргументы функции \printf.
Остальные элементы это может быть и мусор в стеке, но могут быть и значения
от других функций, их локальные переменные, итд.
Пока что мы можем игнорировать их.

Исполняем \q{finish}. 
Это значит исполнять все инструкции до самого конца функции. 
В данном случае это означает исполнять до завершения \printf.

\begin{lstlisting}
(gdb) finish
Run till exit from #0  __printf (format=0x80484f0 "a=%d; b=%d; c=%d") at printf.c:29
main () at 1.c:6
6		return 0;
Value returned is $2 = 13
\end{lstlisting}

\ac{GDB} показывает, что вернула \printf в \EAX (13).
Это, так же как и в примере с \olly, количество напечатанных символов.

А ещё мы видим \q{return 0;} и что это выражение находится в файле \GTT{1.c} в строке 6.
Действительно, файл \GTT{1.c} лежит в текущем директории и \ac{GDB} находит там эту строку.
Как \ac{GDB} знает, какая строка Си-кода сейчас исполняется?
Компилятор, генерируя отладочную информацию, также сохраняет информацию о соответствии строк в исходном коде и адресов инструкций.
GDB это всё-таки отладчик уровня исходных текстов.

Посмотрим регистры.
13 в \EAX:

\begin{lstlisting}
(gdb) info registers
eax            0xd	13
ecx            0x0	0
edx            0x0	0
ebx            0xb7fc0000	-1208221696
esp            0xbffff120	0xbffff120
ebp            0xbffff138	0xbffff138
esi            0x0	0
edi            0x0	0
eip            0x804844a	0x804844a <main+45>
...
\end{lstlisting}

Попробуем дизассемблировать текущие инструкции.
Стрелка указывает на инструкцию, которая будет исполнена следующей.

\begin{lstlisting}[style=customasm]
(gdb) disas
Dump of assembler code for function main:
   0x0804841d <+0>:	push   %ebp
   0x0804841e <+1>:	mov    %esp,%ebp
   0x08048420 <+3>:	and    $0xfffffff0,%esp
   0x08048423 <+6>:	sub    $0x10,%esp
   0x08048426 <+9>:	movl   $0x3,0xc(%esp)
   0x0804842e <+17>:	movl   $0x2,0x8(%esp)
   0x08048436 <+25>:	movl   $0x1,0x4(%esp)
   0x0804843e <+33>:	movl   $0x80484f0,(%esp)
   0x08048445 <+40>:	call   0x80482f0 <printf@plt>
=> 0x0804844a <+45>:	mov    $0x0,%eax
   0x0804844f <+50>:	leave  
   0x08048450 <+51>:	ret    
End of assembler dump.
\end{lstlisting}

По умолчанию \ac{GDB} показывает дизассемблированный листинг в формате AT\&T.
Но можно также переключиться в формат Intel:

\begin{lstlisting}[style=customasm]
(gdb) set disassembly-flavor intel
(gdb) disas
Dump of assembler code for function main:
   0x0804841d <+0>:	push   ebp
   0x0804841e <+1>:	mov    ebp,esp
   0x08048420 <+3>:	and    esp,0xfffffff0
   0x08048423 <+6>:	sub    esp,0x10
   0x08048426 <+9>:	mov    DWORD PTR [esp+0xc],0x3
   0x0804842e <+17>:	mov    DWORD PTR [esp+0x8],0x2
   0x08048436 <+25>:	mov    DWORD PTR [esp+0x4],0x1
   0x0804843e <+33>:	mov    DWORD PTR [esp],0x80484f0
   0x08048445 <+40>:	call   0x80482f0 <printf@plt>
=> 0x0804844a <+45>:	mov    eax,0x0
   0x0804844f <+50>:	leave  
   0x08048450 <+51>:	ret    
End of assembler dump.
\end{lstlisting}

Исполняем следующую инструкцию.
\ac{GDB} покажет закрывающуюся скобку, означая, что это конец блока в функции.

\begin{lstlisting}
(gdb) step
7	};
\end{lstlisting}

Посмотрим регистры после исполнения инструкции \INS{MOV EAX, 0}.
\EAX здесь уже действительно ноль.

\begin{lstlisting}
(gdb) info registers
eax            0x0	0
ecx            0x0	0
edx            0x0	0
ebx            0xb7fc0000	-1208221696
esp            0xbffff120	0xbffff120
ebp            0xbffff138	0xbffff138
esi            0x0	0
edi            0x0	0
eip            0x804844f	0x804844f <main+50>
...
\end{lstlisting}
