\section{\PrintfSeveralArgumentsSectionName}

Now let's extend the \IT{\HelloWorldSectionName}~(\myref{sec:helloworld}) example, replacing \printf in
the \main function body with this:

\lstinputlisting[label=hw_c,style=customc]{patterns/03_printf/1.c}

% sections
\EN{\section{Network address calculation example}

As we know, a TCP/IP address (IPv4) consists of four numbers in the $0 \ldots 255$ range, i.e., four bytes.

Four bytes can be fit in a 32-bit variable easily, so an IPv4 host address, network mask or network address
can all be 32-bit integers.

From the user's point of view, the network mask is defined as four numbers and is formatted like 255.255.255.0 or so,
but network engineers (sysadmins) use a more compact notation (\ac{CIDR}), like \q{/8}, \q{/16}, etc.

This notation just defines the number of bits the mask has, starting at the \ac{MSB}.

\small
\begin{center}
\begin{tabular}{ | l | l | l | l | l | l | }
\hline
\HeaderColor Mask & 
\HeaderColor Hosts & 
\HeaderColor Usable &
\HeaderColor Netmask &
\HeaderColor Hex mask &
\HeaderColor \\
\hline
/30  & 4        & 2        & 255.255.255.252  & 0xfffffffc  & \\
\hline
/29  & 8        & 6        & 255.255.255.248  & 0xfffffff8  & \\
\hline
/28  & 16       & 14       & 255.255.255.240  & 0xfffffff0  & \\
\hline
/27  & 32       & 30       & 255.255.255.224  & 0xffffffe0  & \\
\hline
/26  & 64       & 62       & 255.255.255.192  & 0xffffffc0  & \\
\hline
/24  & 256      & 254      & 255.255.255.0    & 0xffffff00  & class C network \\
\hline
/23  & 512      & 510      & 255.255.254.0    & 0xfffffe00  & \\
\hline
/22  & 1024     & 1022     & 255.255.252.0    & 0xfffffc00  & \\
\hline
/21  & 2048     & 2046     & 255.255.248.0    & 0xfffff800  & \\
\hline
/20  & 4096     & 4094     & 255.255.240.0    & 0xfffff000  & \\
\hline
/19  & 8192     & 8190     & 255.255.224.0    & 0xffffe000  & \\
\hline
/18  & 16384    & 16382    & 255.255.192.0    & 0xffffc000  & \\
\hline
/17  & 32768    & 32766    & 255.255.128.0    & 0xffff8000  & \\
\hline
/16  & 65536    & 65534    & 255.255.0.0      & 0xffff0000  & class B network \\
\hline
/8   & 16777216 & 16777214 & 255.0.0.0        & 0xff000000  & class A network \\
\hline
\end{tabular}
\end{center}
\normalsize

Here is a small example, which calculates the network address by applying the network mask to the host address.

\lstinputlisting{\CURPATH/netmask.c}

\subsection{calc\_network\_address()}

\TT{calc\_network\_address()} function is simplest one: 
it just ANDs the host address with the network mask, resulting in the network address.

\lstinputlisting[caption=\Optimizing MSVC 2012 /Ob0,numbers=left]{\CURPATH/calc_network_address_MSVC_2012_Ox.asm}

At line 22 we see the most important \AND---here the network address is calculated.

\subsection{form\_IP()}

The \TT{form\_IP()} function just puts all 4 bytes into a 32-bit value.

Here is how it is usually done:

\begin{itemize}
\item Allocate a variable for the return value.  Set it to 0.

\item Take the fourth (lowest) byte, apply OR operation to this byte and return the value.
The return value contain the 4th byte now.

\item Take the third byte, shift it left by 8 bits.
You'll get a value like \TT{0x0000bb00} where \TT{bb} is your third byte.
Apply the OR operation to the resulting value and it.
The return value has contained \TT{0x000000aa} so far, so ORing the values will produce a value 
like \TT{0x0000bbaa}.

\item Take the second byte, shift it left by 16 bits.
You'll get a value like \TT{0x00cc0000}, where \TT{cc} is your second byte.
Apply the OR operation to the resulting value and return it.
The return value has contained \TT{0x0000bbaa} so far, so ORing the values will produce
a value like \TT{0x00ccbbaa}.

\item Take the first byte, shift it left by 24 bits.
You'll get a value like \TT{0xdd000000}, where \TT{dd} is your first byte.
Apply the OR operation to the resulting value and return it.
The return value contain \TT{0x00ccbbaa} so far, so ORing the values will produce
a value like \TT{0xddccbbaa}.

\end{itemize}

And this is how it's done by non-optimizing MSVC 2012:

\lstinputlisting[caption=\NonOptimizing MSVC 2012]{\CURPATH/form_IP_MSVC_2012_EN.asm}

Well, the order is different, but, of course, the order of the operations doesn't matter.

\Optimizing MSVC 2012 does essentially the same, but in a different way:

\lstinputlisting[caption=\Optimizing MSVC 2012 /Ob0]{\CURPATH/form_IP_MSVC_2012_Ox_EN.asm}

We could say that each byte is written to the lowest 8 bits of the return value, 
and then the return value is shifted left by one byte at each step.

Repeat 4 times for each input byte.

\par
That's it! Unfortunately, there are probably no other ways to do it.

There are no popular \ac{CPU}s or \ac{ISA}s which has instruction for composing a value from
bits or bytes.

It's all usually done by bit shifting and ORing.

\subsection{print\_as\_IP()}

\TT{print\_as\_IP()} does the inverse: splitting a 32-bit value into 4 bytes.

Slicing works somewhat simpler: just shift input value by 24, 16, 8 or 0 bits, take the 
bits from zeroth to seventh (lowest byte), and that's it:

\lstinputlisting[caption=\NonOptimizing MSVC 2012]{\CURPATH/print_as_IP_MSVC_2012_EN.asm}

\Optimizing MSVC 2012 does almost the same, but without unnecessary reloading of the input value:

\lstinputlisting[caption=\Optimizing MSVC 2012 /Ob0]{\CURPATH/print_as_IP_MSVC_2012_Ox.asm}

\subsection{form\_netmask() and set\_bit()}

\TT{form\_netmask()} makes a network mask value from \ac{CIDR} notation.
Of course, it would be much effective to use here some kind of a precalculated table, but we consider it in this
way intentionally, to demonstrate bit shifts.

We will also write a separate function \TT{set\_bit()}. 
It's a not very good idea to create a function
for such primitive operation, but it would be easy to understand how it all works.

\lstinputlisting[caption=\Optimizing MSVC 2012 /Ob0]{\CURPATH/form_netmask_MSVC_2012_Ox.asm}

\TT{set\_bit()} is primitive: it just shift left 1 to number of bits we need and then 
ORs it with the \q{input} value.
\TT{form\_netmask()} has a loop: it will set as many bits (starting from the \ac{MSB}) as 
passed in the \TT{netmask\_bits} argument

\subsection{Summary}

That's it!
We run it and getting:

\begin{lstlisting}
netmask=255.255.255.0
network address=10.1.2.0
netmask=255.0.0.0
network address=10.0.0.0
netmask=255.255.255.128
network address=10.1.2.0
netmask=255.255.255.192
network address=10.1.2.64
\end{lstlisting}
}
\RU{\section{\Stack}
\label{sec:stack}
\myindex{\Stack}

Стек в компьютерных науках~--- это одна из наиболее фундаментальных структур данных
\footnote{\href{http://go.yurichev.com/17119}{wikipedia.org/wiki/Call\_stack}}.
\ac{AKA} \ac{LIFO}.

Технически это просто блок памяти в памяти процесса + регистр \ESP в x86 или \RSP в x64, либо \ac{SP} в ARM, который указывает где-то в пределах этого блока.

\myindex{ARM!\Instructions!PUSH}
\myindex{ARM!\Instructions!POP}
\myindex{x86!\Instructions!PUSH}
\myindex{x86!\Instructions!POP}
Часто используемые инструкции для работы со стеком~--- это \PUSH и \POP (в x86 и Thumb-режиме ARM). 
\PUSH уменьшает \ESP/\RSP/\ac{SP} на 4 в 32-битном режиме (или на 8 в 64-битном),
затем записывает по адресу, на который указывает \ESP/\RSP/\ac{SP}, содержимое своего единственного операнда.

\POP это обратная операция~--- сначала достает из \glslink{stack pointer}{указателя стека} значение и помещает его в операнд 
(который очень часто является регистром) и затем увеличивает указатель стека на 4 (или 8).

В самом начале \glslink{stack pointer}{регистр-указатель} указывает на конец стека.
Конец стека находится в начале блока памяти, выделенного под стек. Это странно, но это так.
\PUSH уменьшает \glslink{stack pointer}{регистр-указатель}, а \POP~--- увеличивает.

В процессоре ARM, тем не менее, есть поддержка стеков, растущих как в сторону уменьшения, так и в сторону увеличения.

\myindex{ARM!\Instructions!STMFD}
\myindex{ARM!\Instructions!LDMFD}
\myindex{ARM!\Instructions!STMED}
\myindex{ARM!\Instructions!LDMED}
\myindex{ARM!\Instructions!STMFA}
\myindex{ARM!\Instructions!LDMFA}
\myindex{ARM!\Instructions!STMEA}
\myindex{ARM!\Instructions!LDMEA}

Например, инструкции \ac{STMFD}/\ac{LDMFD}, \ac{STMED}/\ac{LDMED} предназначены для descending-стека (растет назад, начиная с высоких адресов в сторону низких).\\
Инструкции \ac{STMFA}/\ac{LDMFA}, \ac{STMEA}/\ac{LDMEA} предназначены для ascending-стека (растет вперед, начиная с низких адресов в сторону высоких).

% It might be worth mentioning that STMED and STMEA write first,
% and then move the pointer,
% and that LDMED and LDMEA move the pointer first, and then read.
% In other words, ARM not only lets the stack grow in a non-standard direction,
% but also in a non-standard order.
% Maybe this can be in the glossary, which would explain why E stands for "empty".

\subsection{Почему стек растет в обратную сторону?}
\label{stack_grow_backwards}

Интуитивно мы можем подумать, что, как и любая другая структура данных, стек мог бы расти вперед, т.е. в сторону увеличения адресов.

Причина, почему стек растет назад, видимо, историческая.
Когда компьютеры были большие и занимали целую комнату, было очень легко разделить сегмент на две части: для \glslink{heap}{кучи} и для стека.
Заранее было неизвестно, насколько большой может быть \glslink{heap}{куча} или стек, так что это решение было самым простым.

\input{patterns/02_stack/stack_and_heap}

В \RitchieThompsonUNIX можно прочитать:

\begin{framed}
\begin{quotation}
The user-core part of an image is divided into three logical segments. The program text segment begins at location 0 in the virtual address space. During execution, this segment is write-protected and a single copy of it is shared among all processes executing the same program. At the first 8K byte boundary above the program text segment in the virtual address space begins a nonshared, writable data segment, the size of which may be extended by a system call. Starting at the highest address in the virtual address space is a stack segment, which automatically grows downward as the hardware's stack pointer fluctuates.
\end{quotation}
\end{framed}

Это немного напоминает как некоторые студенты
пишут два конспекта в одной тетрадке:
первый конспект начинается обычным образом, второй пишется с конца, перевернув тетрадку.
Конспекты могут встретиться где-то посредине, в случае недостатка свободного места.

% I think if we want to expand on this analogy,
% one might remember that the line number increases as as you go down a page.
% So when you decrease the address when pushing to the stack, visually,
% the stack does grow upwards.
% Of course, the problem is that in most human languages,
% just as with computers,
% we write downwards, so this direction is what makes buffer overflows so messy.

\subsection{Для чего используется стек?}

% subsections
\EN{\input{patterns/02_stack/01_saving_ret_addr_EN}}
\RU{\input{patterns/02_stack/01_saving_ret_addr_RU}}
\DE{\input{patterns/02_stack/01_saving_ret_addr_DE}}
\FR{\input{patterns/02_stack/01_saving_ret_addr_FR}}
\PTBR{\input{patterns/02_stack/01_saving_ret_addr_PTBR}}
\ITA{\input{patterns/02_stack/01_saving_ret_addr_ITA}}
\PL{\input{patterns/02_stack/01_saving_ret_addr_PL}}
\JPN{\input{patterns/02_stack/01_saving_ret_addr_JPN}}

\EN{\input{patterns/02_stack/02_args_passing_EN}}
\RU{\input{patterns/02_stack/02_args_passing_RU}}
\PTBR{\input{patterns/02_stack/02_args_passing_PTBR}}
\DE{\input{patterns/02_stack/02_args_passing_DE}}
\ITA{\input{patterns/02_stack/02_args_passing_ITA}}
\FR{\input{patterns/02_stack/02_args_passing_FR}}
\JPN{\input{patterns/02_stack/02_args_passing_JPN}}


\EN{\input{patterns/02_stack/03_local_vars_EN}}
\RU{\input{patterns/02_stack/03_local_vars_RU}}
\PTBR{\input{patterns/02_stack/03_local_vars_PTBR}}
\EN{\input{patterns/02_stack/04_alloca/main_EN}}
\FR{\input{patterns/02_stack/04_alloca/main_FR}}
\RU{\input{patterns/02_stack/04_alloca/main_RU}}
\PTBR{\input{patterns/02_stack/04_alloca/main_PTBR}}
\ITA{\input{patterns/02_stack/04_alloca/main_ITA}}
\DE{\input{patterns/02_stack/04_alloca/main_DE}}
\PL{\input{patterns/02_stack/04_alloca/main_PL}}
\JPN{\input{patterns/02_stack/04_alloca/main_JPN}}

\subsubsection{(Windows) SEH}
\myindex{Windows!Structured Exception Handling}

\ifdefined\RUSSIAN
В стеке хранятся записи \ac{SEH} для функции (если они присутствуют).
Читайте больше о нем здесь: (\myref{sec:SEH}).
\fi % RUSSIAN

\ifdefined\ENGLISH
\ac{SEH} records are also stored on the stack (if they are present).
Read more about it: (\myref{sec:SEH}).
\fi % ENGLISH

\ifdefined\BRAZILIAN
\ac{SEH} também são guardados na pilha (se estiverem presentes).
\PTBRph{}: (\myref{sec:SEH}).
\fi % BRAZILIAN

\ifdefined\ITALIAN
I record \ac{SEH}, se presenti, sono anch'essi memorizzati nello stack.
Maggiori informazioni qui: (\myref{sec:SEH}).
\fi % ITALIAN

\ifdefined\FRENCH
Les enregistrements \ac{SEH} sont aussi stockés dans la pile (s'ils sont présents).
Lire à ce propos: (\myref{sec:SEH}).
\fi % FRENCH


\ifdefined\POLISH
Na stosie są przechowywane wpisy \ac{SEH} dla funkcji (jeśli są one obecne).
Więcej o tym tutaj: (\myref{sec:SEH}).
\fi % POLISH

\ifdefined\JAPANESE
\ac{SEH}レコードはスタックにも格納されます(存在する場合)。
それについてもっと読む:(\myref{sec:SEH})
\fi % JAPANESE

\ifdefined\ENGLISH
\subsubsection{Buffer overflow protection}

More about it here~(\myref{subsec:bufferoverflow}).
\fi

\ifdefined\RUSSIAN
\subsubsection{Защита от переполнений буфера}

Здесь больше об этом~(\myref{subsec:bufferoverflow}).
\fi

\ifdefined\BRAZILIAN
\subsubsection{Proteção contra estouro de buffer}

Mais sobre aqui~(\myref{subsec:bufferoverflow}).
\fi

\ifdefined\ITALIAN
\subsubsection{Protezione da buffer overflow}

Maggiori informazioni qui~(\myref{subsec:bufferoverflow}).
\fi

\ifdefined\FRENCH
\subsubsection{Protection contre les débordements de tampon}

Lire à ce propos~(\myref{subsec:bufferoverflow}).
\fi


\ifdefined\POLISH
\subsubsection{Metody zabiezpieczenia przed przepełnieniem stosu}

Więcej o tym tutaj~(\myref{subsec:bufferoverflow}).
\fi

\ifdefined\JAPANESE
\subsubsection{バッファオーバーフロー保護}

詳細はこちら~(\myref{subsec:bufferoverflow})
\fi

\subsubsection{Автоматическое освобождение данных в стеке}

Возможно, причина хранения локальных переменных и SEH-записей в стеке в том, что после выхода из функции, всё эти данные освобождаются автоматически,
используя только одну инструкцию корректирования указателя стека (часто это \ADD).
Аргументы функций, можно сказать, тоже освобождаются автоматически в конце функции.
А всё что хранится в куче (\IT{heap}) нужно освобождать явно.

% sections
\EN{\input{patterns/02_stack/07_layout_EN}}
\RU{\input{patterns/02_stack/07_layout_RU}}
\PTBR{\input{patterns/02_stack/07_layout_PTBR}}
\EN{\input{patterns/02_stack/08_noise/main_EN}}
\RU{\input{patterns/02_stack/08_noise/main_RU}}
\ITA{\input{patterns/02_stack/08_noise/main_ITA}}
\DE{\input{patterns/02_stack/08_noise/main_DE}}

\subsection{\Exercise}

\begin{itemize}
	\item \url{http://challenges.re/27}
\end{itemize}



}
\PTBR{\subsubsection{x86: a função alloca()}
\label{alloca}
\myindex{\CStandardLibrary!alloca()}

\newcommand{\AllocaSrcPath}{C:\textbackslash{}Program Files (x86)\textbackslash{}Microsoft Visual Studio 10.0\textbackslash{}VC\textbackslash{}crt\textbackslash{}src\textbackslash{}intel}

A função \TT{alloca()}
\footnote{No MSVC, a implementação da função pode ser encontrada nos arquivos \TT{alloca16.asm} e \TT{chkstk.asm} em \\
\TT{\AllocaSrcPath{}}}
funciona da mesma maneira que \TT{malloc()}, mas aloca memória diretamente na pilha.
O bloco de memória alocado não precisa ser limpo através da chamada da função free(),
desde que o rodapé da função (\myref{sec:prologepilog}) retorna \ESP de volta para seu estado inicial e a memória alocada é simplesmente desassociada.
Sobre como a função \TT{alloca()} é implementada, em termos simples, essa função só desloca \ESP para baixo 
(em direção ao fundo da pilha) pelo número de bytes que você precisa e define o ESP como um ponteiro para o bloco alocado.

\RU{Попробуем:}\EN{Let's try:}\PTBR{Vamos tentar:}

\lstinputlisting[style=customc]{patterns/02_stack/04_alloca/2_1.c}

A função \TT{\_snprintf()} funciona exatamente como \printf, mas ao invés de jogar o resultado em stdout
(terminal ou console, por exemplo), ela escreve no buffer buf.
A função \puts copia o conteúdo para um buf do stdout.
Lógico, essas duas chamadas de funções podem ser substituídas por um \printf, mas nós temos que ilustrar o uso pequeno do buffer.

\myparagraph{MSVC}

Vamos compilar (MSVC 2010):

\lstinputlisting[caption=MSVC 2010,style=customasmx86]{patterns/02_stack/04_alloca/2_2_msvc.asm}

\myindex{Compiler intrinsic}
O único argumento da função alloca() é passado via EAX (ao invés de ser empurrado na pilha)
\footnote{Isso é devido ao fato de que alloca() é mais nativa do compilador do que uma função normal (\myref{sec:compiler_intrinsic}).
Um dos motivos que se faz necessário o separamento da função ao invés de um pouco de linhas de código no código,
é porque a implementação da alloca() no MSVC também tem código que é lido da memória que acabou de ser alocada,
para deixar o sistema operacional mapear a memória física para essa região da memória virtual.}.

Depois da chamada de \TT{alloca()}, \ESP aponta para o bloco de 600 bytes que nós podemos usar como memória para o array.

\myparagraph{GCC + \IntelSyntax}

\PTBRph{}

}
\ITA{\subsection{MIPS}

\subsubsection{3 argomenti}

\myparagraph{\Optimizing GCC 4.4.5}

La differenza principale con l'esempio \q{\HelloWorldSectionName} e' che in questo caso \printf e' chiamata 
al posto di \puts, e 3 argomenti aggiuntivi sono passati attraverso i registri \$5\dots \$7 (o \$A0\dots \$A2).
Questo e' il motivo per cui questi registri hanno il prefisso A-, che implica il loro uso per il passaggio di argomenti di funzioni.

% TODO translate to Italian:
\lstinputlisting[caption=\Optimizing GCC 4.4.5 (\assemblyOutput),style=customasm]{patterns/03_printf/MIPS/printf3.O3_EN.s}

% TODO translate to Italian:
\lstinputlisting[caption=\Optimizing GCC 4.4.5 (IDA)]{patterns/03_printf/MIPS/printf3.O3.IDA_EN.lst}

\IDA ha fuso le coppie di istruzioni \INS{LUI} e \INS{ADDIU} in una unica pseudoistruzione \INS{LA}.
Questo e' il motivo per cui non c'e' nessuna istruzione all'indirizzo 0x1C: perche' \INS{LA} \IT{occupa} 8 byte.%

\myparagraph{\NonOptimizing GCC 4.4.5}

\NonOptimizing GCC e' piu' verboso:

% TODO translate to Italian:
\lstinputlisting[caption=\NonOptimizing GCC 4.4.5 (\assemblyOutput),style=customasm]{patterns/03_printf/MIPS/printf3.O0_EN.s}

% TODO translate to Italian:
\lstinputlisting[caption=\NonOptimizing GCC 4.4.5 (IDA)]{patterns/03_printf/MIPS/printf3.O0.IDA_EN.lst}

\subsubsection{8 argomenti}

Usiamo nuovamente l'esempio con 9 argomenti dalla sezione prcedente: \myref{example_printf8_x64}.

\lstinputlisting[style=customc]{patterns/03_printf/2.c}

\myparagraph{\Optimizing GCC 4.4.5}

Solo i primi 4 argomenti sono passati nei registri \$A0 \dots \$A3, gli altri sono passati tramite lo stack.
\myindex{MIPS!O32}

Questa e' la calling convention O32 (che e' la piu' comune nel mondo MIPS).
Altre calling conventions (come N32) possono usare i registri per scopi diversi.

\myindex{MIPS!\Instructions!SW}

\INS{SW} e' l'abbreviazione di \q{Store Word} (da un registro alla memoria).
MIPS manca di istruzioni per memorizzare un valore in memoria, e' quindi necessario usare una coppia di istruzioni (LI/SW).

% TODO translate to Italian:
\lstinputlisting[caption=\Optimizing GCC 4.4.5 (\assemblyOutput),style=customasm]{patterns/03_printf/MIPS/printf8.O3_EN.s}

% TODO translate to Italian:
\lstinputlisting[caption=\Optimizing GCC 4.4.5 (IDA)]{patterns/03_printf/MIPS/printf8.O3.IDA_EN.lst}

\myparagraph{\NonOptimizing GCC 4.4.5}

\NonOptimizing GCC e' piu' verboso:

% TODO translate to Italian:
\lstinputlisting[caption=\NonOptimizing GCC 4.4.5 (\assemblyOutput),style=customasm]{patterns/03_printf/MIPS/printf8.O0_EN.s}

% TODO translate to Italian:
\lstinputlisting[caption=\NonOptimizing GCC 4.4.5 (IDA)]{patterns/03_printf/MIPS/printf8.O0.IDA_EN.lst}
}
\DE{\section{\Stack}
\label{sec:stack}
\myindex{\Stack}

Der Stack ist eine der fundamentalen Datenstrukturen in der Informatik.
\footnote{\href{http://go.yurichev.com/17119}{wikipedia.org/wiki/Call\_Stack}}.
\ac{AKA} \ac{LIFO}.

Technisch betrachtet ist es ein Stapelspeicher innerhalb des Prozessspeichers der zusammen mit den \ESP (x86), \RSP (x64) oder dem \ac{SP} (ARM) Register als ein Zeiger in diesem Speicherblock fungiert.

\myindex{ARM!\Instructions!PUSH}
\myindex{ARM!\Instructions!POP}
\myindex{x86!\Instructions!PUSH}
\myindex{x86!\Instructions!POP}

Die häufigsten Stack-Zugriffsinstruktionen sind die \PUSH- und \POP-Instruktionen (in beidem x86 und ARM Thumb-Modus). \PUSH subtrahiert vom \ESP/\RSP/\ac{SP} 4 Byte im 32-Bit Modus (oder 8 im 64-Bit Modus) und schreibt dann den Inhalt des Zeigers an die Adresse auf die von \ESP/\RSP/\ac{SP} gezeigt wird.

\POP ist die umgekehrte Operation: Die Daten des Zeigers für die Speicherregion auf die von \ac{SP}
gezeigt wird werden ausgelesen und die Inhalte in den Instruktionsoperanden geschreiben (oft ist das ein Register). Dann werden 4 (beziehungsweise 8) Byte zum \gls{stack pointer} addiert.

Nach der Stackallokation, zeigt der \gls{stack pointer} auf den Boden des Stacks.
\PUSH verringert den \gls{stack pointer} und \POP erhöht ihn.
Der Boden des Stacks ist eigentlich der Anfang der Speicherregion die für den Stack reserviert wurde.
Das wirkt zunächst seltsam, aber so funktioniert es.

ARM unterstützt beides, aufsteigende und absteigende Stacks.

\myindex{ARM!\Instructions!STMFD}
\myindex{ARM!\Instructions!LDMFD}
\myindex{ARM!\Instructions!STMED}
\myindex{ARM!\Instructions!LDMED}
\myindex{ARM!\Instructions!STMFA}
\myindex{ARM!\Instructions!LDMFA}
\myindex{ARM!\Instructions!STMEA}
\myindex{ARM!\Instructions!LDMEA}

Zum Beispiel die \ac{STMFD}/\ac{LDMFD} und \ac{STMED}/\ac{LDMED} Instruktionen sind alle dafür gedacht mit einem absteigendem Stack zu arbeiten ( wächst nach unten, fängt mit hohen Adressen an und entwickelt sich zu niedrigeren Adressen). Die \ac{STMFA}/\ac{LDMFA} und \ac{STMEA}/\ac{LDMEA} Instruktionen sind dazu gedacht mit einem aufsteigendem Stack zu arbeiten (wächst nach oben und fängt mit niedrigeren Adressen an und wächst nach oben).

% It might be worth mentioning that STMED and STMEA write first,
% and then move the pointer, and that LDMED and LDMEA move the pointer first, and then read.
% In other words, ARM not only lets the stack grow in a non-standard direction,
% but also in a non-standard order.
% Maybe this can be in the glossary, which would explain why E stands for "empty".

\subsection{Warum wächst der Stack nach unten?}
\label{stack_grow_backwards}

Intuitiv, würden man annehmen das der Stack nach oben wächst z.B Richtung höherer Adressen, so wie bei jeder anderen Datenstruktur.

Der Grund das der Stack rückwärts wächst ist wohl historisch bedingt. Als Computer so groß waren das sie einen ganzen Raum beansprucht haben war es einfach Speicher in zwei Sektionen zu unterteilen, einen Teil für den \gls{heap} und einen Teil für den Stack. Sicher war zu dieser Zeit nicht bekannt wie groß der \gls{heap} und der Stack wachsen würden, während der Programm Laufzeit, also war die Lösung die einfachste mögliche.

\input{patterns/02_stack/stack_and_heap}

In \RitchieThompsonUNIX können wir folgendes lesen:

\begin{framed}
\begin{quotation}
Der user-core eines Programm Images wird in drei logische Segmente unterteilt. Das Programm-Text Segment beginnt bei 0 im virtuellen Adress Speicher. Während der Ausführung wird das Segment als schreibgeschützt markiert und eine einzelne Kopie des Segments wird unter allen Prozessen geteilt die das Programm ausführen. An der ersten 8K grenze über dem Programm Text Segment im Virtuellen Speicher, fängt der ``nonshared'' Bereich an, der nach Bedarf von Syscalls erweitert werden kann. Beginnend bei der höchsten Adresse im Virtuellen Speicher ist das Stack Segment, das Automatisch nach unten wächst während der Hardware Stackpointer sich ändert.
\end{quotation}
\end{framed}

Das erinnert daran wie manche Schüler Notizen zu  zwei Vorträgen in einem Notebook dokumentieren:
Notizen für den ersten Vortrag werden normal notiert, und Notizen zur zum zweiten Vortrag werden 
ans Ende des Notizbuches geschrieben, indem man das Notizbuch umdreht. Die Notizen treffen sich irgendwann
im Notizbuch aufgrund des fehlenden Freien Platzes.

% I think if we want to expand on this analogy,
% one might remember that the line number increases as as you go down a page.
% So when you decrease the address when pushing to the stack, visually,
% the stack does grow upwards.
% Of course, the problem is that in most human languages,
% just as with computers,
% we write downwards, so this direction is what makes buffer overflows so messy.

\subsection{Für was wird der Stack benutzt?}

% subsections
\EN{\input{patterns/02_stack/01_saving_ret_addr_EN}}
\RU{\input{patterns/02_stack/01_saving_ret_addr_RU}}
\DE{\input{patterns/02_stack/01_saving_ret_addr_DE}}
\FR{\input{patterns/02_stack/01_saving_ret_addr_FR}}
\PTBR{\input{patterns/02_stack/01_saving_ret_addr_PTBR}}
\ITA{\input{patterns/02_stack/01_saving_ret_addr_ITA}}
\PL{\input{patterns/02_stack/01_saving_ret_addr_PL}}
\JPN{\input{patterns/02_stack/01_saving_ret_addr_JPN}}

\EN{\input{patterns/02_stack/02_args_passing_EN}}
\RU{\input{patterns/02_stack/02_args_passing_RU}}
\PTBR{\input{patterns/02_stack/02_args_passing_PTBR}}
\DE{\input{patterns/02_stack/02_args_passing_DE}}
\ITA{\input{patterns/02_stack/02_args_passing_ITA}}
\FR{\input{patterns/02_stack/02_args_passing_FR}}
\JPN{\input{patterns/02_stack/02_args_passing_JPN}}


\EN{\input{patterns/02_stack/03_local_vars_EN}}
\RU{\input{patterns/02_stack/03_local_vars_RU}}
\DE{\subsubsection{Local variable storage}

Eine Funktion kann platz für lokale Variablen allokieren in dem sie einfach den \gls{stack pointer}
verkleinert in richtung der niedrigsten Adresse des Stacks verschiebt. 

% I think here, "stack bottom" means the lowest address in the stack space,
% but the reader might also think it means towards the top of the stack space,
% like in a pop, so you might change "towards the stack bottom" to
% "towards the lowest address of the stack", or just take it out,
% since "decreasing" also suggests that.

Dieser Weg ist ziemlich schnell, egal wie viele Variablen deffiniert werden.
Es ist aber keine Anforderung lokale Variablen auf dem Stack zu speichern.
Man kann lokale Variablen speicher wo immer man will, aber traditionell speichert
man sie auf dem Stack.
}
\PTBR{\input{patterns/02_stack/03_local_vars_PTBR}}
\EN{\input{patterns/02_stack/04_alloca/main_EN}}
\FR{\input{patterns/02_stack/04_alloca/main_FR}}
\RU{\input{patterns/02_stack/04_alloca/main_RU}}
\PTBR{\input{patterns/02_stack/04_alloca/main_PTBR}}
\ITA{\input{patterns/02_stack/04_alloca/main_ITA}}
\DE{\input{patterns/02_stack/04_alloca/main_DE}}
\PL{\input{patterns/02_stack/04_alloca/main_PL}}
\JPN{\input{patterns/02_stack/04_alloca/main_JPN}}

\subsubsection{(Windows) SEH}
\myindex{Windows!Structured Exception Handling}

\ifdefined\RUSSIAN
В стеке хранятся записи \ac{SEH} для функции (если они присутствуют).
Читайте больше о нем здесь: (\myref{sec:SEH}).
\fi % RUSSIAN

\ifdefined\ENGLISH
\ac{SEH} records are also stored on the stack (if they are present).
Read more about it: (\myref{sec:SEH}).
\fi % ENGLISH

\ifdefined\BRAZILIAN
\ac{SEH} também são guardados na pilha (se estiverem presentes).
\PTBRph{}: (\myref{sec:SEH}).
\fi % BRAZILIAN

\ifdefined\ITALIAN
I record \ac{SEH}, se presenti, sono anch'essi memorizzati nello stack.
Maggiori informazioni qui: (\myref{sec:SEH}).
\fi % ITALIAN

\ifdefined\FRENCH
Les enregistrements \ac{SEH} sont aussi stockés dans la pile (s'ils sont présents).
Lire à ce propos: (\myref{sec:SEH}).
\fi % FRENCH


\ifdefined\POLISH
Na stosie są przechowywane wpisy \ac{SEH} dla funkcji (jeśli są one obecne).
Więcej o tym tutaj: (\myref{sec:SEH}).
\fi % POLISH

\ifdefined\JAPANESE
\ac{SEH}レコードはスタックにも格納されます(存在する場合)。
それについてもっと読む:(\myref{sec:SEH})
\fi % JAPANESE

\ifdefined\ENGLISH
\subsubsection{Buffer overflow protection}

More about it here~(\myref{subsec:bufferoverflow}).
\fi

\ifdefined\RUSSIAN
\subsubsection{Защита от переполнений буфера}

Здесь больше об этом~(\myref{subsec:bufferoverflow}).
\fi

\ifdefined\BRAZILIAN
\subsubsection{Proteção contra estouro de buffer}

Mais sobre aqui~(\myref{subsec:bufferoverflow}).
\fi

\ifdefined\ITALIAN
\subsubsection{Protezione da buffer overflow}

Maggiori informazioni qui~(\myref{subsec:bufferoverflow}).
\fi

\ifdefined\FRENCH
\subsubsection{Protection contre les débordements de tampon}

Lire à ce propos~(\myref{subsec:bufferoverflow}).
\fi


\ifdefined\POLISH
\subsubsection{Metody zabiezpieczenia przed przepełnieniem stosu}

Więcej o tym tutaj~(\myref{subsec:bufferoverflow}).
\fi

\ifdefined\JAPANESE
\subsubsection{バッファオーバーフロー保護}

詳細はこちら~(\myref{subsec:bufferoverflow})
\fi

\subsubsection{Automatisches deallokieren der Daten auf dem Stack}

Vielleicht ist der Grund warum man lokale Variablen und SEH Einträge auf dem Stack speichert, weil sie beim 
verlassen der Funktion automatisch aufgeräumt werden. Man braucht dabei nur eine Instruktion um die Position
des Stackpointers zu korrigieren (oftmals ist es die \ADD Instruktion). Funktions Argumente, könnte man sagen 
werden auch am Ende der Funktion deallokiert. Im Kontrast dazu, alles was auf dem \IT{heap} gespeichert wird muss
explizit deallokiert werden. 

% sections
\EN{\input{patterns/02_stack/07_layout_EN}}
\RU{\input{patterns/02_stack/07_layout_RU}}
\DE{\subsection{Ein typisches Stack Layout}

Ein typisches Stacklayout auf einer 32-Bit Umgebung sieht am Anfang 
der ausführung einer Funktion, noch bevor der ausführung der ersten 
Instruktion wie folgt aus:

\input{patterns/02_stack/stack_layout}


% I think this only applies to RISC architectures
% that don't have a POP instruction that only lets you read one value
% (ie. ARM and MIPS).
% In x86, the return address is saved before entering the function,
% and the function does not have the chance to save the frame pointer.
% Also, you should mention that this is how the stack looks like
% right after the function prologue,
% which some readers might think is the first instruction,
% but is needed to save the frame pointer.
}
\PTBR{\input{patterns/02_stack/07_layout_PTBR}}
\EN{\input{patterns/02_stack/08_noise/main_EN}}
\RU{\input{patterns/02_stack/08_noise/main_RU}}
\ITA{\input{patterns/02_stack/08_noise/main_ITA}}
\DE{\input{patterns/02_stack/08_noise/main_DE}}

\subsection{\Exercise}

\begin{itemize}
	\item \url{http://challenges.re/27}
\end{itemize}


}


\subsection{ARM}

\EN{\subsubsection{ARM: 3 arguments}

ARM's traditional scheme for passing arguments (calling convention) behaves as follows:
the first 4 arguments are passed through the \Reg{0}-\Reg{3} registers; the remaining arguments via the stack.
This resembles the arguments passing scheme in 
fastcall~(\myref{fastcall}) or win64~(\myref{sec:callingconventions_win64}).

\myparagraph{32-bit ARM}

\mysubparagraph{\NonOptimizingKeilVI (\ARMMode)}

\begin{lstlisting}[caption=\NonOptimizingKeilVI (\ARMMode),style=customasm]
.text:00000000 main
.text:00000000 10 40 2D E9   STMFD   SP!, {R4,LR}
.text:00000004 03 30 A0 E3   MOV     R3, #3
.text:00000008 02 20 A0 E3   MOV     R2, #2
.text:0000000C 01 10 A0 E3   MOV     R1, #1
.text:00000010 08 00 8F E2   ADR     R0, aADBDCD     ; "a=%d; b=%d; c=%d"
.text:00000014 06 00 00 EB   BL      __2printf
.text:00000018 00 00 A0 E3   MOV     R0, #0          ; return 0
.text:0000001C 10 80 BD E8   LDMFD   SP!, {R4,PC}
\end{lstlisting}

So, the first 4 arguments are passed via the \Reg{0}-\Reg{3} registers in this order:
a pointer to the \printf format string in 
\Reg{0}, then 1 in \Reg{1}, 2 in \Reg{2} and 3 in \Reg{3}.
The instruction at \GTT{0x18} writes 0 to \Reg{0}---this is \IT{return 0} C-statement.
There is nothing unusual so far.

\OptimizingKeilVI generates the same code.

\mysubparagraph{\OptimizingKeilVI (\ThumbMode)}

\begin{lstlisting}[caption=\OptimizingKeilVI (\ThumbMode),style=customasm]
.text:00000000 main
.text:00000000 10 B5        PUSH    {R4,LR}
.text:00000002 03 23        MOVS    R3, #3
.text:00000004 02 22        MOVS    R2, #2
.text:00000006 01 21        MOVS    R1, #1
.text:00000008 02 A0        ADR     R0, aADBDCD     ; "a=%d; b=%d; c=%d"
.text:0000000A 00 F0 0D F8  BL      __2printf
.text:0000000E 00 20        MOVS    R0, #0
.text:00000010 10 BD        POP     {R4,PC}
\end{lstlisting}

There is no significant difference from the non-optimized code for ARM mode.

\mysubparagraph{\OptimizingKeilVI (\ARMMode) + let's remove return}
\label{ARM_B_to_printf}

Let's rework example slightly by removing \IT{return 0}:

\begin{lstlisting}[style=customc]
#include <stdio.h>

void main()
{
	printf("a=%d; b=%d; c=%d", 1, 2, 3);
};
\end{lstlisting}

The result is somewhat unusual:

\begin{lstlisting}[caption=\OptimizingKeilVI (\ARMMode),style=customasm]
.text:00000014 main
.text:00000014 03 30 A0 E3   MOV     R3, #3
.text:00000018 02 20 A0 E3   MOV     R2, #2
.text:0000001C 01 10 A0 E3   MOV     R1, #1
.text:00000020 1E 0E 8F E2   ADR     R0, aADBDCD     ; "a=%d; b=%d; c=%d\n"
.text:00000024 CB 18 00 EA   B       __2printf
\end{lstlisting}

\myindex{ARM!\Registers!Link Register}
\myindex{ARM!\Instructions!B}
\myindex{Function epilogue}
This is the optimized (\Othree) version for ARM mode and this time we see \INS{B} as the last instruction instead of the familiar \INS{BL}.
Another difference between this optimized version and the previous one (compiled without optimization)
is the lack of function prologue and epilogue (instructions preserving the \Reg{0} and \ac{LR} registers values).
\myindex{x86!\Instructions!JMP}
The \INS{B} instruction just jumps to another address, without any manipulation of the \ac{LR} register,
similar to \JMP in x86.
Why does it work? Because this code is, in fact, effectively equivalent to the previous.
There are two main reasons: 1) neither the stack nor \ac{SP} (the \gls{stack pointer}) is modified;
2) the call to \printf is the last instruction, so there is nothing going on afterwards.
On completion, the \printf function simply returns the control to the address 
stored in \ac{LR}.
Since the \ac{LR} currently stores the address of the point from where our function
has been called then the control from \printf will be returned to that point.
Therefore we do not have to save \ac{LR} because we do not have necessity to modify \ac{LR}.
And we do not have necessity to modify \ac{LR} because there are no other function calls except \printf. Furthermore,
after this call we do not to do anything else!
That is the reason such optimization is possible.

This optimization is often used in functions where the last statement is a call to another function.
A similar example is presented here:
\myref{jump_to_last_printf}.

\myparagraph{ARM64}

\mysubparagraph{\NonOptimizing GCC (Linaro) 4.9}

\lstinputlisting[caption=\NonOptimizing GCC (Linaro) 4.9,style=customasm]{patterns/03_printf/ARM/ARM3_O0_EN.lst}

\myindex{ARM!\Instructions!STP}

The first instruction \INS{STP} (\IT{Store Pair}) saves \ac{FP} (X29) and \ac{LR} (X30) in the stack.
The second \INS{ADD X29, SP, 0} instruction forms the stack frame.
It is just writing the value of \ac{SP} into X29.

\myindex{ARM!\Instructions!ADRP/ADD pair}
Next, we see the familiar \INS{ADRP}/\ADD instruction pair, which forms a pointer to the string.
\myindex{ARM64!lo12}
\IT{lo12} meaning low 12 bits, i.e., linker will write low 12 bits of LC1 address into the opcode of \ADD instruction.
\GTT{\%d} in \printf string format is a 32-bit \Tint, so the 1, 2 and 3 are loaded into 32-bit register parts.

\Optimizing GCC (Linaro) 4.9 generates the same code.

}
\RU{\subsubsection{ARM: 3 аргумента}

В ARM традиционно принята такая схема передачи аргументов в функцию: 
4 первых аргумента через регистры \Reg{0}-\Reg{3}; а остальные~--- через стек.
Это немного похоже на то, как аргументы передаются в 
fastcall~(\myref{fastcall}) или win64~(\myref{sec:callingconventions_win64}).

\myparagraph{32-битный ARM}

\mysubparagraph{\NonOptimizingKeilVI (\ARMMode)}

\begin{lstlisting}[caption=\NonOptimizingKeilVI (\ARMMode),style=customasm]
.text:00000000 main
.text:00000000 10 40 2D E9   STMFD   SP!, {R4,LR}
.text:00000004 03 30 A0 E3   MOV     R3, #3
.text:00000008 02 20 A0 E3   MOV     R2, #2
.text:0000000C 01 10 A0 E3   MOV     R1, #1
.text:00000010 08 00 8F E2   ADR     R0, aADBDCD     ; "a=%d; b=%d; c=%d"
.text:00000014 06 00 00 EB   BL      __2printf
.text:00000018 00 00 A0 E3   MOV     R0, #0          ; return 0
.text:0000001C 10 80 BD E8   LDMFD   SP!, {R4,PC}
\end{lstlisting}

Итак, первые 4 аргумента передаются через регистры \Reg{0}-\Reg{3}, по порядку: 
указатель на формат-строку для \printf
в \Reg{0}, затем 1 в \Reg{1}, 2 в \Reg{2} и 3 в \Reg{3}.

Инструкция на \GTT{0x18} записывает 0 в \Reg{0} --- это выражение в Си \IT{return 0}.
Пока что здесь нет ничего необычного.
\OptimizingKeilVI генерирует точно такой же код.

\mysubparagraph{\OptimizingKeilVI (\ThumbMode)}

\begin{lstlisting}[caption=\OptimizingKeilVI (\ThumbMode),style=customasm]
.text:00000000 main
.text:00000000 10 B5        PUSH    {R4,LR}
.text:00000002 03 23        MOVS    R3, #3
.text:00000004 02 22        MOVS    R2, #2
.text:00000006 01 21        MOVS    R1, #1
.text:00000008 02 A0        ADR     R0, aADBDCD     ; "a=%d; b=%d; c=%d"
.text:0000000A 00 F0 0D F8  BL      __2printf
.text:0000000E 00 20        MOVS    R0, #0
.text:00000010 10 BD        POP     {R4,PC}
\end{lstlisting}

Здесь нет особых отличий от неоптимизированного варианта для режима ARM.
\mysubparagraph{\OptimizingKeilVI (\ARMMode) + убираем return}
\label{ARM_B_to_printf}

Немного переделаем пример, убрав \IT{return 0}:

\begin{lstlisting}[style=customc]
#include <stdio.h>

void main()
{
	printf("a=%d; b=%d; c=%d", 1, 2, 3);
};
\end{lstlisting}

Результат получится необычным:

\begin{lstlisting}[caption=\OptimizingKeilVI (\ARMMode),style=customasm]
.text:00000014 main
.text:00000014 03 30 A0 E3   MOV     R3, #3
.text:00000018 02 20 A0 E3   MOV     R2, #2
.text:0000001C 01 10 A0 E3   MOV     R1, #1
.text:00000020 1E 0E 8F E2   ADR     R0, aADBDCD     ; "a=%d; b=%d; c=%d\n"
.text:00000024 CB 18 00 EA   B       __2printf
\end{lstlisting}

\myindex{ARM!\Registers!Link Register}
\myindex{ARM!\Instructions!B}
\myindex{Function epilogue}
Это оптимизированная версия (\Othree) для режима ARM, и здесь мы видим последнюю инструкцию 
\INS{B} вместо привычной нам \INS{BL}.
Отличия между этой оптимизированной версией и предыдущей, скомпилированной без оптимизации, 
ещё и в том, что здесь нет пролога и эпилога функции (инструкций, сохраняющих состояние регистров \Reg{0} и \ac{LR}).
\myindex{x86!\Instructions!JMP}
Инструкция \INS{B} просто переходит на другой адрес, без манипуляций с регистром \ac{LR}, то есть это аналог \JMP в x86.
Почему это работает нормально? Потому что этот код эквивалентен предыдущему.

Основных причин две: 1) стек не модифицируется, как и \glslink{stack pointer}{указатель стека} \ac{SP}; 2) вызов функции \printf последний, после него ничего не происходит.
Функция \printf, отработав, просто возвращает управление по адресу, записанному в \ac{LR}.

Но в \ac{LR} находится адрес места, откуда была вызвана наша функция!
А следовательно, управление из \printf вернется сразу туда.

Значит нет нужды сохранять \ac{LR}, потому что нет нужны модифицировать \ac{LR}.
А нет нужды модифицировать \ac{LR}, потому что нет иных вызовов функций, кроме \printf, к тому же, после этого вызова не нужно ничего здесь больше делать!
Поэтому такая оптимизация возможна.

Эта оптимизация часто используется в функциях, где последнее выражение~--- это вызов другой функции.

Ещё один похожий пример описан здесь:
\myref{jump_to_last_printf}.

\myparagraph{ARM64}

\mysubparagraph{\NonOptimizing GCC (Linaro) 4.9}

\lstinputlisting[caption=\NonOptimizing GCC (Linaro) 4.9,style=customasm]{patterns/03_printf/ARM/ARM3_O0_RU.lst}

\myindex{ARM!\Instructions!STP}
Итак, первая инструкция \INS{STP} (\IT{Store Pair}) сохраняет \ac{FP} (X29) и \ac{LR} (X30) в стеке.
Вторая инструкция \INS{ADD X29, SP, 0} формирует стековый фрейм.
Это просто запись значения \ac{SP} в X29.

\myindex{ARM!\Instructions!ADRP/ADD pair}
Далее уже знакомая пара инструкций \INS{ADRP}/\ADD формирует указатель на строку.

\myindex{ARM64!lo12}
\IT{lo12} означает младшие 12 бит, т.е., линкер запишет младшие 12 бит адреса метки LC1 в опкод инструкции \ADD.
\GTT{\%d} в формате \printf это 32-битный \Tint, так что 1, 2 и 3 заносятся в 32-битные части регистров.
\Optimizing GCC (Linaro) 4.9 генерирует почти такой же код.

}
\FR{\subsubsection{ARM: 3 arguments}

Le schéma ARM traditionnel pour passer des arguments (convention d'appel) se
comporte de cette façon:
les 4 premiers arguments sont passés par les registres \Reg{0}-\Reg{3}; les autres
par la pile.
Cela ressemble au schéma de passage des arguments dans
fastcall~(\myref{fastcall}) ou win64~(\myref{sec:callingconventions_win64}).

\myparagraph{ARM 32-bit}

\mysubparagraph{\NonOptimizingKeilVI (\ARMMode)}

\begin{lstlisting}[caption=\NonOptimizingKeilVI (\ARMMode),style=customasmARM]
.text:00000000 main
.text:00000000 10 40 2D E9   STMFD   SP!, {R4,LR}
.text:00000004 03 30 A0 E3   MOV     R3, #3
.text:00000008 02 20 A0 E3   MOV     R2, #2
.text:0000000C 01 10 A0 E3   MOV     R1, #1
.text:00000010 08 00 8F E2   ADR     R0, aADBDCD     ; "a=%d; b=%d; c=%d"
.text:00000014 06 00 00 EB   BL      __2printf
.text:00000018 00 00 A0 E3   MOV     R0, #0          ; renvoyer 0
.text:0000001C 10 80 BD E8   LDMFD   SP!, {R4,PC}
\end{lstlisting}

Donc, les 4 premiers arguments sont passés par les registres \Reg{0}-\Reg{3} dans
cet ordre:
un pointeur sur la chaîne de format de \printf dans \Reg{0}, puis 1 dans \Reg{1},
2 dans \Reg{2} et 3 dans \Reg{3}.
L'instruction en \GTT{0x18} écrit 0 dans \Reg{0}---c'est la déclaration C
de \IT{return 0}.

\OptimizingKeilVI génère le même code.

\mysubparagraph{\OptimizingKeilVI (\ThumbMode)}

\begin{lstlisting}[caption=\OptimizingKeilVI (\ThumbMode),style=customasmARM]
.text:00000000 main
.text:00000000 10 B5        PUSH    {R4,LR}
.text:00000002 03 23        MOVS    R3, #3
.text:00000004 02 22        MOVS    R2, #2
.text:00000006 01 21        MOVS    R1, #1
.text:00000008 02 A0        ADR     R0, aADBDCD     ; "a=%d; b=%d; c=%d"
.text:0000000A 00 F0 0D F8  BL      __2printf
.text:0000000E 00 20        MOVS    R0, #0
.text:00000010 10 BD        POP     {R4,PC}
\end{lstlisting}

Il n'y a pas de différence significative avec le code non optimisé pour le mode ARM.

\mysubparagraph{\OptimizingKeilVI (\ARMMode) + supprimons le retour}
\label{ARM_B_to_printf}

Retravaillons légèrement l'exemple en supprimant \IT{return 0}:

\begin{lstlisting}[style=customc]
#include <stdio.h>

void main()
{
	printf("a=%d; b=%d; c=%d", 1, 2, 3);
};
\end{lstlisting}

Le résultat est quelque peu inhabituel:

\begin{lstlisting}[caption=\OptimizingKeilVI (\ARMMode),style=customasmARM]
.text:00000014 main
.text:00000014 03 30 A0 E3   MOV     R3, #3
.text:00000018 02 20 A0 E3   MOV     R2, #2
.text:0000001C 01 10 A0 E3   MOV     R1, #1
.text:00000020 1E 0E 8F E2   ADR     R0, aADBDCD     ; "a=%d; b=%d; c=%d\n"
.text:00000024 CB 18 00 EA   B       __2printf
\end{lstlisting}

\myindex{ARM!\Registers!Link Register}
\myindex{ARM!\Instructions!B}
\myindex{Épilogue de la fonction}
C'est la version optimisée (\Othree) pour le mode ARM et cette fois nous voyons
\INS{B} comme dernière instruction au lieu du \INS{BL} habituel.
Une autre différence entre cette version optimisée et la précédente (compilée
sans optimisation) est l'absence de fonctions prologue et épilogue (les instructions
qui préservent les valeurs des registres \Reg{0} et \ac{LR}).
\myindex{x86!\Instructions!JMP}
L'instruction \INS{B} saute simplement à une autre adresse, sans manipuler le registre
\ac{LR}, de façon similaire au \JMP en x86.
Pourquoi est-ce que fonctionne? Parce ce code est en fait bien équivalent au précédent.
Il y a deux raisons principales: 1) Ni la pile ni \ac{SP} (\glslink{stack pointer}{pointeur de pile})
ne sont modifiés;
2) l'appel à \printf est la dernière instruction, donc il ne se passe rien après.
A la fin, la fonction \printf rend simplement le contrôle à l'adresse stockée
dans \ac{LR}.
Puisque \ac{LR} contient actuellement l'adresse du point depuis lequel notre fonction
a été appelée alors le contrôle après \printf sera redonné à ce point.
Par conséquent, nous n'avons pas besoin de sauver \ac{LR} car il ne nous est pas
nécessaire de le modifier.
Et il ne nous est non plus pas nécessaire de modifier \ac{LR} car il n'y a pas d'autre
appel de fonction excepté \printf. Par ailleurs, après cet appel nous ne faisons
rien d'autre!
C'est la raison pour laquelle une telle optimisation est possible.

Cette optimisation est souvent utilisée dans les fonctions où la dernière déclaration
est un appel à une autre fonction.
Un exemple similaire est présenté ici:
\myref{jump_to_last_printf}.

\myparagraph{ARM64}

\mysubparagraph{GCC (Linaro) 4.9 \NonOptimizing}

\lstinputlisting[caption=GCC (Linaro) 4.9 \NonOptimizing,style=customasmARM]{patterns/03_printf/ARM/ARM3_O0_FR.lst}

\myindex{ARM!\Instructions!STP}

La première instruction \INS{STP} (\IT{Store Pair}) sauve \ac{FP} (X29) et \ac{LR} (X30) sur la pile.

La seconde instruction, \INS{ADD X29, SP, 0} crée la pile.
Elle écrit simplement la valeur de \ac{SP} dans X29.

\myindex{ARM!\Instructions!ADRP/ADD pair}
Ensuite nous voyons la paire d'instructions habituelle \INS{ADRP}/\ADD, qui crée
le pointeur sur la chaîne.
\myindex{ARM64!lo12}
\IT{lo12} signifie les 12 bits de poids faible, i.e., le linker va écrire les 12 bits
de poids faible de l'adresse LC1 dans l'opcode de l'instruction \ADD.
\GTT{\%d} dans la chaîne de format de \printf est un \Tint 32-bit, les 1, 2 et
3 sont chargés dans les parties 32-bit des registres.

GCC (Linaro) 4.9 \Optimizing génère le même code.

}
\ITA{\subsubsection{ARM: 3 arguments}

Lo schema tradizionale per il passaggio di argomenti (calling convention) di ARM si comporta in questo modo:
i primi 4 argomenti vengono passati attraverso i registri \Reg{0}-\Reg{3} , i restanti attraverso lo stack.
Cio' ricorda molto il metodo per il passaggio di argomenti in 
fastcall~(\myref{fastcall}) o win64~(\myref{sec:callingconventions_win64}).

\myparagraph{32-bit ARM}

\mysubparagraph{\NonOptimizingKeilVI (\ARMMode)}

\begin{lstlisting}[caption=\NonOptimizingKeilVI (\ARMMode),style=customasmARM]
.text:00000000 main
.text:00000000 10 40 2D E9   STMFD   SP!, {R4,LR}
.text:00000004 03 30 A0 E3   MOV     R3, #3
.text:00000008 02 20 A0 E3   MOV     R2, #2
.text:0000000C 01 10 A0 E3   MOV     R1, #1
.text:00000010 08 00 8F E2   ADR     R0, aADBDCD     ; "a=%d; b=%d; c=%d"
.text:00000014 06 00 00 EB   BL      __2printf
.text:00000018 00 00 A0 E3   MOV     R0, #0          ; return 0
.text:0000001C 10 80 BD E8   LDMFD   SP!, {R4,PC}
\end{lstlisting}

I primi 4 argomenti sono quindi passati attraverso i registri \Reg{0}-\Reg{3} nel seguente ordine:
un puntatore alla format string di \printf in 
\Reg{0}, 1 in \Reg{1}, 2 in \Reg{2} e 3 in \Reg{3}.
L'istruzione a \GTT{0x18} scrive 0 in \Reg{0}---questo equivale allo statement C \IT{return 0}.
Niente di insolito fino a qui.

\OptimizingKeilVI genera lo stesso codice.

\mysubparagraph{\OptimizingKeilVI (\ThumbMode)}

\begin{lstlisting}[caption=\OptimizingKeilVI (\ThumbMode),style=customasmARM]
.text:00000000 main
.text:00000000 10 B5        PUSH    {R4,LR}
.text:00000002 03 23        MOVS    R3, #3
.text:00000004 02 22        MOVS    R2, #2
.text:00000006 01 21        MOVS    R1, #1
.text:00000008 02 A0        ADR     R0, aADBDCD     ; "a=%d; b=%d; c=%d"
.text:0000000A 00 F0 0D F8  BL      __2printf
.text:0000000E 00 20        MOVS    R0, #0
.text:00000010 10 BD        POP     {R4,PC}
\end{lstlisting}

Non c'e' nessuna differenza significativa nel codice non ottimizzato per modo ARM.

\mysubparagraph{\OptimizingKeilVI (\ARMMode) + rimozione di return}
\label{ARM_B_to_printf}

Modifichiamo leggermente l'esempio rimuovendo \IT{return 0}:

\begin{lstlisting}[style=customc]
#include <stdio.h>

void main()
{
	printf("a=%d; b=%d; c=%d", 1, 2, 3);
};
\end{lstlisting}

Il risultato e' alquanto insolito:

\begin{lstlisting}[caption=\OptimizingKeilVI (\ARMMode),style=customasmARM]
.text:00000014 main
.text:00000014 03 30 A0 E3   MOV     R3, #3
.text:00000018 02 20 A0 E3   MOV     R2, #2
.text:0000001C 01 10 A0 E3   MOV     R1, #1
.text:00000020 1E 0E 8F E2   ADR     R0, aADBDCD     ; "a=%d; b=%d; c=%d\n"
.text:00000024 CB 18 00 EA   B       __2printf
\end{lstlisting}

\myindex{ARM!\Registers!Link Register}
\myindex{ARM!\Instructions!B}
\myindex{Function epilogue}
Questa e' la versione ottimizzata (\Othree) per ARM mode e stavolta notiamo \INS{B} come ultima istruzione, al posto della familiare \INS{BL}.
Un'altra differenza tra questa versione ottimizzata e la precedente (compilata senza ottimizzazione)
e' la mancanza di prologo ed epilogo della funzione (le istruzioni che preservano i valori dei registri \Reg{0} e \ac{LR}).
\myindex{x86!\Instructions!JMP}
L'istruzione \INS{B} salta semplicemente ad un altro indirizzo, senza alcuna manipolazione del registro \ac{LR}, in modo simile
a \JMP in x86.
Perche' funziona? Perche' questo codice e' infatti equivalente al precedente.
Principalmente per due motivi: 1) ne' lo stack ne' \ac{SP} (lo \gls{stack pointer}) vengono modificati;
2) la chiamata a \printf e' l'ultima istruzionie, quindi non succede niente dopo di essa.
Al completamento, \printf restituisce semplicemente il controllo all'indirizzo memorizzato in \ac{LR}.
Poiche' \ac{LR} attualmente contiene l'indirizzo del punto da cui la nostra funzione era stata chiamata, 
il controllo verra' restituito da \printf a quello stesso punto.
Pertanto non c'e' alcun bisogno di salvare \ac{LR} in quanto non abbiamo necessita' di modificare \ac{LR}.
E non vogliamo affatto modificare \ac{LR} poiche' non ci sono altre chiamate a funzione ad eccezzione di \printf. Inoltre, dopo
questa chiamata non abbiamo nient'altro da fare!
Queste le ragione per cui una simile ottimizzazione e' possibile.

Questa ottimizzazione e' spesso usata in funzioni in cui l'ultimo statement e' una chiamata ad un'altra funzione.
Un esempio simile e' fornito di seguito:
\myref{jump_to_last_printf}.

\myparagraph{ARM64}

\mysubparagraph{\NonOptimizing GCC (Linaro) 4.9}

% TODO translate to Italian:
\lstinputlisting[caption=\NonOptimizing GCC (Linaro) 4.9,style=customasmARM]{patterns/03_printf/ARM/ARM3_O0_EN.lst}

\myindex{ARM!\Instructions!STP}

La prima istruzione \INS{STP} (\IT{Store Pair}) salva \ac{FP} (X29) e \ac{LR} (X30) nello stack.
La seconda istruzione \INS{ADD X29, SP, 0} forma lo stack frame.
Scrive semplicemente il valore di \ac{SP} in X29.

\myindex{ARM!\Instructions!ADRP/ADD pair}
Successivamente vediamo la familiare coppia di istruzioni \INS{ADRP}/\ADD , che forma un puntatore alla stringa.
\myindex{ARM64!lo12}
\IT{lo12} indica 12 bit bassi (low 12 bits), ovvero, il linker scrivera' i 12 bit bassi dell'indirizzo di LC1 nell'opcode dell'istruzione \ADD .
\GTT{\%d} nella format string di \printf e' un \Tint a 32-bit, quindi u valori 1, 2 e 3 sono caricati nelle parti a 32-bit dei registri.

\Optimizing GCC (Linaro) 4.9 genera lo stesso codice.
}
\JPN{\subsubsection{ARM: 3つの引数}

引数を渡すためのARMの伝統的なスキーム(呼び出し規約)は、次のように動作します。
最初の4つの引数は\Reg{0}-\Reg{3}レジスタに渡されます。 残りの引数はスタックを介して。 
これはfastcall~(\myref{fastcall})またはwin64~(\myref{sec:callingconventions_win64})の引数渡しスキームに似ています。

\myparagraph{32ビットARM}

\mysubparagraph{\NonOptimizingKeilVI (\ARMMode)}

\begin{lstlisting}[caption=\NonOptimizingKeilVI (\ARMMode),style=customasmARM]
.text:00000000 main
.text:00000000 10 40 2D E9   STMFD   SP!, {R4,LR}
.text:00000004 03 30 A0 E3   MOV     R3, #3
.text:00000008 02 20 A0 E3   MOV     R2, #2
.text:0000000C 01 10 A0 E3   MOV     R1, #1
.text:00000010 08 00 8F E2   ADR     R0, aADBDCD     ; "a=%d; b=%d; c=%d"
.text:00000014 06 00 00 EB   BL      __2printf
.text:00000018 00 00 A0 E3   MOV     R0, #0          ; return 0
.text:0000001C 10 80 BD E8   LDMFD   SP!, {R4,PC}
\end{lstlisting}

したがって、最初の4つの引数は、\Reg{0}-\Reg{3}レジスタをこの順序で渡します。
\Reg{0}の \printf 形式文字列へのポインタ、\Reg{1}の1、\Reg{2}の2、\Reg{3}の3の順です。 
\GTT{0x18}の命令は\Reg{0}に0を書き込みます。 これは\IT{return 0} となるCの命令文です。 
珍しいことは何もありません。

\OptimizingKeilVI は同じコードを生成します。

\mysubparagraph{\OptimizingKeilVI (\ThumbMode)}

\begin{lstlisting}[caption=\OptimizingKeilVI (\ThumbMode),style=customasmARM]
.text:00000000 main
.text:00000000 10 B5        PUSH    {R4,LR}
.text:00000002 03 23        MOVS    R3, #3
.text:00000004 02 22        MOVS    R2, #2
.text:00000006 01 21        MOVS    R1, #1
.text:00000008 02 A0        ADR     R0, aADBDCD     ; "a=%d; b=%d; c=%d"
.text:0000000A 00 F0 0D F8  BL      __2printf
.text:0000000E 00 20        MOVS    R0, #0
.text:00000010 10 BD        POP     {R4,PC}
\end{lstlisting}

ARMモードの最適化されていないコードとの大きな違いはありません。

\mysubparagraph{\OptimizingKeilVI (\ARMMode) + returnを削除してみる}
\label{ARM_B_to_printf}

\IT{return 0}を取り除いて例を少し修正しましょう:

\begin{lstlisting}[style=customc]
#include <stdio.h>

void main()
{
	printf("a=%d; b=%d; c=%d", 1, 2, 3);
};
\end{lstlisting}

結果はちょっと珍しくなりました。

\begin{lstlisting}[caption=\OptimizingKeilVI (\ARMMode),style=customasmARM]
.text:00000014 main
.text:00000014 03 30 A0 E3   MOV     R3, #3
.text:00000018 02 20 A0 E3   MOV     R2, #2
.text:0000001C 01 10 A0 E3   MOV     R1, #1
.text:00000020 1E 0E 8F E2   ADR     R0, aADBDCD     ; "a=%d; b=%d; c=%d\n"
.text:00000024 CB 18 00 EA   B       __2printf
\end{lstlisting}

\myindex{ARM!\Registers!Link Register}
\myindex{ARM!\Instructions!B}
\myindex{Function epilogue}
これはARMモード用に最適化された(\Othree)バージョンであり、今回は\INS{B}を使い慣れた\INS{BL}ではなく最後の命令と見なします。
この最適化されたバージョンと前のバージョン(最適化なしでコンパイルされたもの)との別の違いは、
関数のプロローグとエピローグ(\Reg{0} と \ac{LR}レジスタの値を保持する命令)の欠如です。 
\INS{B}命令は、x86の \JMP と同様に、\ac{LR}レジスタの操作なしで別のアドレスにジャンプするだけです。
それはなぜ機能するのでしょうか?実際、このコードは以前のコードと事実上同等です。
主な理由は2つあります。1)スタックも\ac{SP}(\gls{stack pointer})も変更されていません。 
2) \printf の呼び出しが最後の命令なので、その後は何も起こりません。
完了すると、 \printf 関数は\ac{LR}に格納されているアドレスにコントロールを返します。 
\ac{LR}は現在、関数が呼び出されたポイントのアドレスを格納しているので、 \printf からの制御はそのポイントに返されます。
したがって、\ac{LR}を変更する必要がないため、\ac{LR}を節約する必要はありません。 
\printf 以外の関数呼び出しがないため、\ac{LR}を変更する必要はありません。
さらに、この呼び出しの後、私たちは何もしません!これがそのような最適化が可能な理由です。

この最適化は、最後のステートメントが別の関数の呼び出しである関数でよく使用されます。
同様の例をここに示します:\myref{jump_to_last_printf}

\myparagraph{ARM64}

\mysubparagraph{\NonOptimizing GCC (Linaro) 4.9}

\lstinputlisting[caption=\NonOptimizing GCC (Linaro) 4.9,style=customasmARM]{patterns/03_printf/ARM/ARM3_O0_JPN.lst}

\myindex{ARM!\Instructions!STP}

第1の命令\INS{STP} (\IT{ストアペア})は、\ac{FP}(X29)および\ac{LR}(X30)をスタックに保存します。 
2番目の\INS{ADD X29, SP, 0}命令がスタックフレームを形成します。 
\ac{SP}の値をX29に書き込むだけです。

\myindex{ARM!\Instructions!ADRP/ADD pair}
次に、使い慣れた\INS{ADRP}/\ADD 命令ペアを参照します。これは、文字列へのポインタを形成します。 
\myindex{ARM64!lo12}
すなわち、\IT{lo12}は、LC1アドレスの下位12ビットを \ADD 命令のオペコードに書き込みます。
\printf の文字列書式の\GTT{\%d}は32ビット \Tint なので、1,2および3は32ビットのレジスタ部分にロードされます。

\Optimizing GCC (Linaro) 4.9 は同じコードを生成します。
}

\EN{\subsubsection{ARM: 8 arguments}

Let's use again the example with 9 arguments from the previous section: \myref{example_printf8_x64}.

\lstinputlisting[style=customc]{patterns/03_printf/2.c}

\myparagraph{\OptimizingKeilVI: \ARMMode}

\begin{lstlisting}
.text:00000028             main
.text:00000028
.text:00000028             var_18 = -0x18
.text:00000028             var_14 = -0x14
.text:00000028             var_4  = -4
.text:00000028
.text:00000028 04 E0 2D E5  STR    LR, [SP,#var_4]!
.text:0000002C 14 D0 4D E2  SUB    SP, SP, #0x14
.text:00000030 08 30 A0 E3  MOV    R3, #8
.text:00000034 07 20 A0 E3  MOV    R2, #7
.text:00000038 06 10 A0 E3  MOV    R1, #6
.text:0000003C 05 00 A0 E3  MOV    R0, #5
.text:00000040 04 C0 8D E2  ADD    R12, SP, #0x18+var_14
.text:00000044 0F 00 8C E8  STMIA  R12, {R0-R3}
.text:00000048 04 00 A0 E3  MOV    R0, #4
.text:0000004C 00 00 8D E5  STR    R0, [SP,#0x18+var_18]
.text:00000050 03 30 A0 E3  MOV    R3, #3
.text:00000054 02 20 A0 E3  MOV    R2, #2
.text:00000058 01 10 A0 E3  MOV    R1, #1
.text:0000005C 6E 0F 8F E2  ADR    R0, aADBDCDDDEDFDGD ; "a=%d; b=%d; c=%d; d=%d; e=%d; f=%d; g=%"...
.text:00000060 BC 18 00 EB  BL     __2printf
.text:00000064 14 D0 8D E2  ADD    SP, SP, #0x14
.text:00000068 04 F0 9D E4  LDR    PC, [SP+4+var_4],#4
\end{lstlisting}

This code can be divided into several parts:

\myindex{Function prologue}
\begin{itemize}
\item Function prologue:

\myindex{ARM!\Instructions!STR}
The very first \INS{STR LR, [SP,\#var\_4]!} instruction saves \ac{LR} on the stack, because we are going to use this register for the \printf call.
Exclamation mark at the end indicates \IT{pre-index}.

This implies that \ac{SP} is to be decreased by 4 first, and then \ac{LR} will be saved at the address stored in \ac{SP}.
This is similar to \PUSH in x86.
Read more about it at: \myref{ARM_postindex_vs_preindex}.

\myindex{ARM!\Instructions!SUB}
The second \INS{SUB SP, SP, \#0x14} instruction decreases \ac{SP} (the \gls{stack pointer}) in order to allocate \GTT{0x14} (20) bytes on the stack.
Indeed, we have to pass 5 32-bit values via the stack to the \printf function, and each one occupies 4 bytes, which is exactly $5*4=20$.
The other 4 32-bit values are to be passed through registers.

\item Passing 5, 6, 7 and 8 via the stack: they are stored in the \Reg{0}, \Reg{1}, \Reg{2} and \Reg{3} registers respectively.\\
Then, the \INS{ADD R12, SP, \#0x18+var\_14} instruction writes the stack address where these 4 variables are to be stored, into the \Reg{12} register.
\myindex{IDA!var\_?}
\IT{var\_14} is an assembly macro, equal to -0x14, created by \IDA to conveniently display the code accessing the stack.
The \IT{var\_?} macros generated by \IDA reflect local variables in the stack.

So, \GTT{SP+4} is to be stored into the \Reg{12} register. \\
\myindex{ARM!\Instructions!STMIA}
The next \INS{STMIA R12, {R0-R3}} instruction writes registers \Reg{0}-\Reg{3} contents to the memory pointed by \Reg{12}.
\INS{STMIA} abbreviates \IT{Store Multiple Increment After}. 
\IT{\q{Increment After}} implies that \Reg{12} is to be increased by 4 after each register value is written.

\item Passing 4 via the stack: 4 is stored in \Reg{0} and then this value, with the help of the \\
\INS{STR R0, [SP,\#0x18+var\_18]} instruction is saved on the stack.
\IT{var\_18} is -0x18, so the offset is to be 0, thus the value from the \Reg{0} register (4) is to be written to the address written in \ac{SP}.

\item Passing 1, 2 and 3 via registers:
The values of the first 3 numbers (a, b, c) (1, 2, 3 respectively) are passed through the 
\Reg{1}, \Reg{2} and \Reg{3}
registers right before the \printf call, and the other
5 values are passed via the stack:

\item \printf call.

\myindex{Function epilogue}
\item Function epilogue:

The \INS{ADD SP, SP, \#0x14} instruction restores the \ac{SP} pointer back to its former value, thus cleaning the stack.
Of course, what has been stored on the stack will stay there, but it will all be rewritten during the execution of subsequent functions.

\myindex{ARM!\Instructions!LDR}
The \INS{LDR PC, [SP+4+var\_4],\#4} instruction loads the saved \ac{LR} value from the stack into the \ac{PC} register, thus causing the function to exit.
There is no exclamation mark---indeed, \ac{PC} is loaded first from the address stored in \ac{SP} ($4+var\_4=4+(-4)=0$, so this instruction is analogous to \INS{LDR PC, [SP],\#4}), and then \ac{SP} is increased by 4.
This is referred as \IT{post-index}\footnote{Read more about it: \myref{ARM_postindex_vs_preindex}.}.
Why does \IDA display the instruction like that?
Because it wants to illustrate the stack layout and the fact that \GTT{var\_4} is allocated for saving the \ac{LR} value in the local stack.
This instruction is somewhat similar to \INS{POP PC} in x86\footnote{It is impossible to set \GTT{IP/EIP/RIP} value using \POP in x86, but anyway, you got the analogy right.}.

\end{itemize}

\myparagraph{\OptimizingKeilVI: \ThumbMode}

\begin{lstlisting}
.text:0000001C             printf_main2
.text:0000001C
.text:0000001C             var_18 = -0x18
.text:0000001C             var_14 = -0x14
.text:0000001C             var_8  = -8
.text:0000001C
.text:0000001C 00 B5        PUSH    {LR}
.text:0000001E 08 23        MOVS    R3, #8
.text:00000020 85 B0        SUB     SP, SP, #0x14
.text:00000022 04 93        STR     R3, [SP,#0x18+var_8]
.text:00000024 07 22        MOVS    R2, #7
.text:00000026 06 21        MOVS    R1, #6
.text:00000028 05 20        MOVS    R0, #5
.text:0000002A 01 AB        ADD     R3, SP, #0x18+var_14
.text:0000002C 07 C3        STMIA   R3!, {R0-R2}
.text:0000002E 04 20        MOVS    R0, #4
.text:00000030 00 90        STR     R0, [SP,#0x18+var_18]
.text:00000032 03 23        MOVS    R3, #3
.text:00000034 02 22        MOVS    R2, #2
.text:00000036 01 21        MOVS    R1, #1
.text:00000038 A0 A0        ADR     R0, aADBDCDDDEDFDGD ; "a=%d; b=%d; c=%d; d=%d; e=%d; f=%d; g=%"...
.text:0000003A 06 F0 D9 F8  BL      __2printf
.text:0000003E
.text:0000003E             loc_3E   ; CODE XREF: example13_f+16
.text:0000003E 05 B0        ADD     SP, SP, #0x14
.text:00000040 00 BD        POP     {PC}
\end{lstlisting}

The output is almost like in the previous example. However, this is Thumb code and the values are packed into stack differently: 
8 goes first, then 5, 6, 7, and 4 goes third.

\myparagraph{\OptimizingXcodeIV: \ARMMode}

\begin{lstlisting}
__text:0000290C             _printf_main2
__text:0000290C
__text:0000290C             var_1C = -0x1C
__text:0000290C             var_C  = -0xC
__text:0000290C
__text:0000290C 80 40 2D E9   STMFD  SP!, {R7,LR}
__text:00002910 0D 70 A0 E1   MOV    R7, SP
__text:00002914 14 D0 4D E2   SUB    SP, SP, #0x14
__text:00002918 70 05 01 E3   MOV    R0, #0x1570
__text:0000291C 07 C0 A0 E3   MOV    R12, #7
__text:00002920 00 00 40 E3   MOVT   R0, #0
__text:00002924 04 20 A0 E3   MOV    R2, #4
__text:00002928 00 00 8F E0   ADD    R0, PC, R0
__text:0000292C 06 30 A0 E3   MOV    R3, #6
__text:00002930 05 10 A0 E3   MOV    R1, #5
__text:00002934 00 20 8D E5   STR    R2, [SP,#0x1C+var_1C]
__text:00002938 0A 10 8D E9   STMFA  SP, {R1,R3,R12}
__text:0000293C 08 90 A0 E3   MOV    R9, #8
__text:00002940 01 10 A0 E3   MOV    R1, #1
__text:00002944 02 20 A0 E3   MOV    R2, #2
__text:00002948 03 30 A0 E3   MOV    R3, #3
__text:0000294C 10 90 8D E5   STR    R9, [SP,#0x1C+var_C]
__text:00002950 A4 05 00 EB   BL     _printf
__text:00002954 07 D0 A0 E1   MOV    SP, R7
__text:00002958 80 80 BD E8   LDMFD  SP!, {R7,PC}
\end{lstlisting}

\myindex{ARM!\Instructions!STMFA}
\myindex{ARM!\Instructions!STMIB}
Almost the same as what we have already seen, with the
exception of \INS{STMFA} (Store Multiple Full Ascending) instruction, which is a synonym of \INS{STMIB} (Store Multiple Increment Before) instruction. 
This instruction increases the value in the \ac{SP} register and only then writes the next register value into the memory, rather than performing those two actions in the opposite order.

Another thing that catches the eye is that the instructions are arranged seemingly random.
For example, the value in the \Reg{0} register is manipulated in three
places, at addresses \GTT{0x2918}, \GTT{0x2920} and \GTT{0x2928}, when it would be possible to do it in one point.

However, the optimizing compiler may have its own reasons on how to order the instructions so to achieve higher efficiency during the execution.

Usually, the processor attempts to simultaneously execute instructions located side-by-side.\\
For example, instructions like \INS{MOVT R0, \#0} and
\INS{ADD R0, PC, R0} cannot be executed simultaneously since they both modify the \Reg{0} register. 
On the other hand, \INS{MOVT R0, \#0} and \INS{MOV R2, \#4} 
instructions can be executed
simultaneously since the effects of their execution are not conflicting with each other.
Presumably, the compiler tries to generate code in such a manner (wherever it is possible).
 
\myparagraph{\OptimizingXcodeIV: \ThumbTwoMode}

\begin{lstlisting}
__text:00002BA0               _printf_main2
__text:00002BA0
__text:00002BA0               var_1C = -0x1C
__text:00002BA0               var_18 = -0x18
__text:00002BA0               var_C  = -0xC
__text:00002BA0
__text:00002BA0 80 B5          PUSH     {R7,LR}
__text:00002BA2 6F 46          MOV      R7, SP
__text:00002BA4 85 B0          SUB      SP, SP, #0x14
__text:00002BA6 41 F2 D8 20    MOVW     R0, #0x12D8
__text:00002BAA 4F F0 07 0C    MOV.W    R12, #7
__text:00002BAE C0 F2 00 00    MOVT.W   R0, #0
__text:00002BB2 04 22          MOVS     R2, #4
__text:00002BB4 78 44          ADD      R0, PC  ; char *
__text:00002BB6 06 23          MOVS     R3, #6
__text:00002BB8 05 21          MOVS     R1, #5
__text:00002BBA 0D F1 04 0E    ADD.W    LR, SP, #0x1C+var_18
__text:00002BBE 00 92          STR      R2, [SP,#0x1C+var_1C]
__text:00002BC0 4F F0 08 09    MOV.W    R9, #8
__text:00002BC4 8E E8 0A 10    STMIA.W  LR, {R1,R3,R12}
__text:00002BC8 01 21          MOVS     R1, #1
__text:00002BCA 02 22          MOVS     R2, #2
__text:00002BCC 03 23          MOVS     R3, #3
__text:00002BCE CD F8 10 90    STR.W    R9, [SP,#0x1C+var_C]
__text:00002BD2 01 F0 0A EA    BLX      _printf
__text:00002BD6 05 B0          ADD      SP, SP, #0x14
__text:00002BD8 80 BD          POP      {R7,PC}
\end{lstlisting}

The output is almost the same as in the previous example, with the exception that Thumb-instructions are used instead.
% FIXME: also STMIA is used instead of STMIB,
% which is why it uses LR, which is 4 bytes ahead of SP

\myparagraph{ARM64}

\mysubparagraph{\NonOptimizing GCC (Linaro) 4.9}

\lstinputlisting[caption=\NonOptimizing GCC (Linaro) 4.9]{patterns/03_printf/ARM/ARM8_O0_EN.lst}

The first 8 arguments are passed in X- or W-registers: \ARMPCS.
A string pointer requires a 64-bit register, so it's passed in \RegX{0}.
All other values have a \Tint 32-bit type, so they are stored in the 32-bit part of the registers (W-).
The 9th argument (8) is passed via the stack.
Indeed: it's not possible to pass large number of arguments through registers, because the number of registers is limited.

\Optimizing GCC (Linaro) 4.9 generates the same code.
}
\RU{\subsubsection{ARM: 8 аргументов}

Снова воспользуемся примером с 9-ю аргументами из предыдущей секции: \myref{example_printf8_x64}.

\lstinputlisting[style=customc]{patterns/03_printf/2.c}

\myparagraph{\OptimizingKeilVI: \ARMMode}

\begin{lstlisting}[style=customasmARM]
.text:00000028             main
.text:00000028
.text:00000028             var_18 = -0x18
.text:00000028             var_14 = -0x14
.text:00000028             var_4  = -4
.text:00000028
.text:00000028 04 E0 2D E5  STR    LR, [SP,#var_4]!
.text:0000002C 14 D0 4D E2  SUB    SP, SP, #0x14
.text:00000030 08 30 A0 E3  MOV    R3, #8
.text:00000034 07 20 A0 E3  MOV    R2, #7
.text:00000038 06 10 A0 E3  MOV    R1, #6
.text:0000003C 05 00 A0 E3  MOV    R0, #5
.text:00000040 04 C0 8D E2  ADD    R12, SP, #0x18+var_14
.text:00000044 0F 00 8C E8  STMIA  R12, {R0-R3}
.text:00000048 04 00 A0 E3  MOV    R0, #4
.text:0000004C 00 00 8D E5  STR    R0, [SP,#0x18+var_18]
.text:00000050 03 30 A0 E3  MOV    R3, #3
.text:00000054 02 20 A0 E3  MOV    R2, #2
.text:00000058 01 10 A0 E3  MOV    R1, #1
.text:0000005C 6E 0F 8F E2  ADR    R0, aADBDCDDDEDFDGD ; "a=%d; b=%d; c=%d; d=%d; e=%d; f=%d; g=%"...
.text:00000060 BC 18 00 EB  BL     __2printf
.text:00000064 14 D0 8D E2  ADD    SP, SP, #0x14
.text:00000068 04 F0 9D E4  LDR    PC, [SP+4+var_4],#4
\end{lstlisting}

Этот код можно условно разделить на несколько частей:

\myindex{Function prologue}
\begin{itemize}
\item Пролог функции:

\myindex{ARM!\Instructions!STR}
Самая первая инструкция \INS{STR LR, [SP,\#var\_4]!} 
сохраняет в стеке \ac{LR}, ведь нам придется использовать этот регистр для вызова \printf.
Восклицательный знак в конце означает \IT{pre-index}.
Это значит, что в начале \ac{SP} должно быть уменьшено на 4, затем по адресу в \ac{SP} должно быть записано значение \ac{LR}.

Это аналог знакомой в x86 инструкции \PUSH. Читайте больше об этом: \myref{ARM_postindex_vs_preindex}.

\myindex{ARM!\Instructions!SUB}
Вторая инструкция \INS{SUB SP, SP, \#0x14} уменьшает \glslink{stack pointer}{указатель стека} \ac{SP}, но, на самом деле, эта процедура нужна для выделения в локальном стеке места размером \GTT{0x14} (20) байт.
Действительно, нам нужно передать 5 32-битных значений через стек в \printf. Каждое значение занимает 4 байта, все вместе~--- $5*4=20$.
Остальные 4 32-битных значения будут переданы через регистры.

\item Передача 5, 6, 7 и 8 через стек:
они записываются в регистры \Reg{0}, \Reg{1}, \Reg{2} и \Reg{3} соответственно.\\
Затем инструкция \INS{ADD R12, SP, \#0x18+var\_14} 
записывает в регистр \Reg{12} адрес места в стеке, куда будут помещены эти 4 значения.
\myindex{IDA!var\_?}
\IT{var\_14}~--- это макрос ассемблера, равный -0x14.
Такие макросы создает \IDA, чтобы удобнее было показывать, как код обращается к стеку.

Макросы \IT{var\_?}, создаваемые \IDA, отражают локальные переменные в стеке. Так что в \Reg{12} будет записано \GTT{SP+4}.

\myindex{ARM!\Instructions!STMIA}
Следующая инструкция \INS{STMIA R12, {R0-R3}} записывает содержимое регистров \Reg{0}-\Reg{3} по адресу в памяти, на который указывает \Reg{12}.

Инструкция \INS{STMIA} означает \IT{Store Multiple Increment After}.\\
\IT{Increment After} означает, что \Reg{12} будет увеличиваться на 4 после записи каждого значения регистра.

\item Передача 4 через стек:
4 записывается в \Reg{0}, затем инструкция \INS{STR R0, [SP,\#0x18+var\_18]} записывает его в стек.
\IT{var\_18} равен -0x18, смещение будет 0, 
так что значение из регистра \Reg{0} (4) запишется туда, куда указывает \ac{SP}.

\item Передача 1, 2 и 3 через регистры:

Значения для первых трех чисел (a, b, c) (1, 2, 3 соответственно) передаются в регистрах 
\Reg{1}, \Reg{2} и \Reg{3} перед самим вызовом \printf, а остальные 5 значений передаются через стек, и вот как:

\item Вызов \printf.

\myindex{Function epilogue}
\item Эпилог функции:

Инструкция \INS{ADD SP, SP, \#0x14} возвращает \ac{SP} на прежнее место, 
аннулируя таким образом всё, что было записано в стек.
Конечно, то что было записано в стек, там пока и останется, но всё это будет многократно 
перезаписано во время исполнения последующих функций.

\myindex{ARM!\Instructions!LDR}
Инструкция \INS{LDR PC, [SP+4+var\_4],\#4} загружает в \ac{PC} 
сохраненное значение \ac{LR} из стека, обеспечивая таким образом выход из функции.

Здесь нет восклицательного знака~--- действительно, сначала \ac{PC} загружается из места, куда указывает \ac{SP}
($4+var\_4=4+(-4)=0$, так что эта инструкция аналогична \INS{LDR PC, [SP],\#4}), затем \ac{SP} увеличивается 
на 4.
Это называется \IT{post-index}\footnote{Читайте больше об этом: \myref{ARM_postindex_vs_preindex}.}.
Почему \IDA показывает инструкцию именно так?
Потому что она хочет показать разметку стека и тот факт, что \GTT{var\_4} выделена в локальном стеке именно для сохраненного
значения \ac{LR}.
Эта инструкция в каком-то смысле аналогична \INS{POP PC} в x86
\footnote{В x86 невозможно установить значение \GTT{IP/EIP/RIP} используя \POP, но будем надеяться, вы поняли аналогию.}.

\end{itemize}

\myparagraph{\OptimizingKeilVI: \ThumbMode}

\begin{lstlisting}[style=customasmARM]
.text:0000001C             printf_main2
.text:0000001C
.text:0000001C             var_18 = -0x18
.text:0000001C             var_14 = -0x14
.text:0000001C             var_8  = -8
.text:0000001C
.text:0000001C 00 B5        PUSH    {LR}
.text:0000001E 08 23        MOVS    R3, #8
.text:00000020 85 B0        SUB     SP, SP, #0x14
.text:00000022 04 93        STR     R3, [SP,#0x18+var_8]
.text:00000024 07 22        MOVS    R2, #7
.text:00000026 06 21        MOVS    R1, #6
.text:00000028 05 20        MOVS    R0, #5
.text:0000002A 01 AB        ADD     R3, SP, #0x18+var_14
.text:0000002C 07 C3        STMIA   R3!, {R0-R2}
.text:0000002E 04 20        MOVS    R0, #4
.text:00000030 00 90        STR     R0, [SP,#0x18+var_18]
.text:00000032 03 23        MOVS    R3, #3
.text:00000034 02 22        MOVS    R2, #2
.text:00000036 01 21        MOVS    R1, #1
.text:00000038 A0 A0        ADR     R0, aADBDCDDDEDFDGD ; "a=%d; b=%d; c=%d; d=%d; e=%d; f=%d; g=%"...
.text:0000003A 06 F0 D9 F8  BL      __2printf
.text:0000003E
.text:0000003E             loc_3E   ; CODE XREF: example13_f+16
.text:0000003E 05 B0        ADD     SP, SP, #0x14
.text:00000040 00 BD        POP     {PC}
\end{lstlisting}

Это почти то же самое что и в предыдущем примере, только код для Thumb и значения помещаются в 
стек немного иначе: сначала 8 за первый раз, затем 5, 6, 7 за второй раз и 4 за третий раз.

\myparagraph{\OptimizingXcodeIV: \ARMMode}

\begin{lstlisting}[style=customasmARM]
__text:0000290C             _printf_main2
__text:0000290C
__text:0000290C             var_1C = -0x1C
__text:0000290C             var_C  = -0xC
__text:0000290C
__text:0000290C 80 40 2D E9   STMFD  SP!, {R7,LR}
__text:00002910 0D 70 A0 E1   MOV    R7, SP
__text:00002914 14 D0 4D E2   SUB    SP, SP, #0x14
__text:00002918 70 05 01 E3   MOV    R0, #0x1570
__text:0000291C 07 C0 A0 E3   MOV    R12, #7
__text:00002920 00 00 40 E3   MOVT   R0, #0
__text:00002924 04 20 A0 E3   MOV    R2, #4
__text:00002928 00 00 8F E0   ADD    R0, PC, R0
__text:0000292C 06 30 A0 E3   MOV    R3, #6
__text:00002930 05 10 A0 E3   MOV    R1, #5
__text:00002934 00 20 8D E5   STR    R2, [SP,#0x1C+var_1C]
__text:00002938 0A 10 8D E9   STMFA  SP, {R1,R3,R12}
__text:0000293C 08 90 A0 E3   MOV    R9, #8
__text:00002940 01 10 A0 E3   MOV    R1, #1
__text:00002944 02 20 A0 E3   MOV    R2, #2
__text:00002948 03 30 A0 E3   MOV    R3, #3
__text:0000294C 10 90 8D E5   STR    R9, [SP,#0x1C+var_C]
__text:00002950 A4 05 00 EB   BL     _printf
__text:00002954 07 D0 A0 E1   MOV    SP, R7
__text:00002958 80 80 BD E8   LDMFD  SP!, {R7,PC}
\end{lstlisting}

\myindex{ARM!\Instructions!STMFA}
\myindex{ARM!\Instructions!STMIB}
Почти то же самое, что мы уже видели, за исключением того, что \INS{STMFA} (Store Multiple Full Ascending)~--- 
это синоним инструкции \INS{STMIB} (Store Multiple Increment Before). 
Эта инструкция увеличивает \ac{SP} и только затем записывает в память значение очередного регистра, но не наоборот.

Далее бросается в глаза то, что инструкции как будто бы расположены случайно.
Например, значение в регистре \Reg{0} подготавливается в трех местах, по адресам \GTT{0x2918}, \GTT{0x2920} и \GTT{0x2928}, 
когда это можно было бы сделать в одном месте.
Однако, у оптимизирующего компилятора могут быть свои доводы о том, как лучше составлять инструкции 
друг с другом для лучшей эффективности исполнения.
Процессор обычно пытается исполнять одновременно идущие друг за другом инструкции.
К примеру, инструкции \INS{MOVT R0, \#0} и \INS{ADD R0, PC, R0} не могут быть исполнены одновременно,
потому что обе инструкции модифицируют регистр \Reg{0}. 
А вот инструкции \INS{MOVT R0, \#0} и \INS{MOV R2, \#4} легко можно исполнить одновременно, 
потому что эффекты от их исполнения никак не конфликтуют друг с другом.
Вероятно, компилятор старается генерировать код именно таким образом там, где это возможно.
 
\myparagraph{\OptimizingXcodeIV: \ThumbTwoMode}

\begin{lstlisting}[style=customasmARM]
__text:00002BA0               _printf_main2
__text:00002BA0
__text:00002BA0               var_1C = -0x1C
__text:00002BA0               var_18 = -0x18
__text:00002BA0               var_C  = -0xC
__text:00002BA0
__text:00002BA0 80 B5          PUSH     {R7,LR}
__text:00002BA2 6F 46          MOV      R7, SP
__text:00002BA4 85 B0          SUB      SP, SP, #0x14
__text:00002BA6 41 F2 D8 20    MOVW     R0, #0x12D8
__text:00002BAA 4F F0 07 0C    MOV.W    R12, #7
__text:00002BAE C0 F2 00 00    MOVT.W   R0, #0
__text:00002BB2 04 22          MOVS     R2, #4
__text:00002BB4 78 44          ADD      R0, PC  ; char *
__text:00002BB6 06 23          MOVS     R3, #6
__text:00002BB8 05 21          MOVS     R1, #5
__text:00002BBA 0D F1 04 0E    ADD.W    LR, SP, #0x1C+var_18
__text:00002BBE 00 92          STR      R2, [SP,#0x1C+var_1C]
__text:00002BC0 4F F0 08 09    MOV.W    R9, #8
__text:00002BC4 8E E8 0A 10    STMIA.W  LR, {R1,R3,R12}
__text:00002BC8 01 21          MOVS     R1, #1
__text:00002BCA 02 22          MOVS     R2, #2
__text:00002BCC 03 23          MOVS     R3, #3
__text:00002BCE CD F8 10 90    STR.W    R9, [SP,#0x1C+var_C]
__text:00002BD2 01 F0 0A EA    BLX      _printf
__text:00002BD6 05 B0          ADD      SP, SP, #0x14
__text:00002BD8 80 BD          POP      {R7,PC}
\end{lstlisting}

Почти то же самое, что и в предыдущем примере,
лишь за тем исключением, что здесь используются Thumb-инструкции.

% FIXME: also STMIA is used instead of STMIB,
% which is why it uses LR, which is 4 bytes ahead of SP

\myparagraph{ARM64}

\mysubparagraph{\NonOptimizing GCC (Linaro) 4.9}

\lstinputlisting[caption=\NonOptimizing GCC (Linaro) 4.9,style=customasmARM]{patterns/03_printf/ARM/ARM8_O0_RU.lst}

Первые 8 аргументов передаются в X- или W-регистрах: \ARMPCS.
Указатель на строку требует 64-битного регистра, так что он передается в \RegX{0}.
Все остальные значения имеют 32-битный тип \Tint, так что они записываются в 32-битные части регистров (W-).
Девятый аргумент (8) передается через стек.
Действительно, невозможно передать большое количество аргументов в регистрах, потому что количество регистров ограничено.

\Optimizing GCC (Linaro) 4.9 генерирует почти такой же код.
}
\FR{\subsubsection{ARM: 8 arguments}

Utilisons de nouveau l'exemple avec 9 arguments de la section précédente: \myref{example_printf8_x64}.

\lstinputlisting[style=customc]{patterns/03_printf/2.c}

\myparagraph{\OptimizingKeilVI: \ARMMode}

\begin{lstlisting}[style=customasmARM]
.text:00000028             main
.text:00000028
.text:00000028             var_18 = -0x18
.text:00000028             var_14 = -0x14
.text:00000028             var_4  = -4
.text:00000028
.text:00000028 04 E0 2D E5  STR    LR, [SP,#var_4]!
.text:0000002C 14 D0 4D E2  SUB    SP, SP, #0x14
.text:00000030 08 30 A0 E3  MOV    R3, #8
.text:00000034 07 20 A0 E3  MOV    R2, #7
.text:00000038 06 10 A0 E3  MOV    R1, #6
.text:0000003C 05 00 A0 E3  MOV    R0, #5
.text:00000040 04 C0 8D E2  ADD    R12, SP, #0x18+var_14
.text:00000044 0F 00 8C E8  STMIA  R12, {R0-R3}
.text:00000048 04 00 A0 E3  MOV    R0, #4
.text:0000004C 00 00 8D E5  STR    R0, [SP,#0x18+var_18]
.text:00000050 03 30 A0 E3  MOV    R3, #3
.text:00000054 02 20 A0 E3  MOV    R2, #2
.text:00000058 01 10 A0 E3  MOV    R1, #1
.text:0000005C 6E 0F 8F E2  ADR    R0, aADBDCDDDEDFDGD ; "a=%d; b=%d; c=%d; d=%d; e=%d; f=%d; g=%"...
.text:00000060 BC 18 00 EB  BL     __2printf
.text:00000064 14 D0 8D E2  ADD    SP, SP, #0x14
.text:00000068 04 F0 9D E4  LDR    PC, [SP+4+var_4],#4
\end{lstlisting}

Ce code peut être divisé en plusieurs parties:

\myindex{Prologue de la fonction}
\begin{itemize}
\item Prologue de la fonction:

\myindex{ARM!\Instructions!STR}
La toute première instruction \INS{STR LR, [SP,\#var\_4]!} sauve \ac{LR} sur la pile,
car nous allons utiliser ce registre pour l'appel à \printf.
Le point d'exclamation à la fin indique un \IT{pré-index}.

Cela signifie que \ac{SP} est d'abord décrémenté de 4, et qu'ensuite \ac{LR}
va être sauvé à l'adresse stockée dans \ac{SP}.
C'est similaire à \PUSH en x86.
Lire aussi à ce propos: \myref{ARM_postindex_vs_preindex}.

\myindex{ARM!\Instructions!SUB}
La seconde instruction \INS{SUB SP, SP, \#0x14} décrémente \ac{SP} (le \glslink{stack pointer}{pointeur de pile})
afin d'allouer \GTT{0x14} (20) octets sur la pile.
En effet, nous devons passer 5 valeurs de 32-bit par la pile à la fonction \printf,
et chacune occupe 4 octets, ce qui fait exactement $5*4=20$.
Les 4 autres valeurs de 32-bit sont passées par les registres.

\item Passer 5, 6, 7 et 8 par la pile: ils sont stockés dans les registres \Reg{0},
\Reg{1}, \Reg{2} et \Reg{3} respectivement.\\
Ensuite, l'instruction \INS{ADD R12, SP, \#0x18+var\_14} écrit l'adresse de la pile
où ces 4 variables doivent être stockées dans le registre \Reg{12}.
\myindex{IDA!var\_?}
\IT{var\_14} est une macro d'assemblage, égal à -0x14, créée par \IDA pour afficher
commodément le code accédant à la pile.
Les macros \IT{var\_?} générée par \IDA reflètent les variables locales dans la pile.

Donc, \GTT{SP+4} doit être stocké dans le registre \Reg{12}. \\
\myindex{ARM!\Instructions!STMIA}
L'instruction suivante \INS{STMIA R12, {R0-R3}} écrit le contenu des registres
\Reg{0}-\Reg{3} dans la mémoire pointée par \Reg{12}.
\INS{STMIA} est l'abréviation de \IT{Store Multiple Increment After} (stocker plusieur incrémenter après).
\IT{\q{Increment After}} signifie que \Reg{12} doit être incrémenté de 4 après
l'écriture de chaque valeur d'un registre.

\item Passer 4 par la pile: 4 est stocké dans \Reg{0} et ensuite, cette valeur, avec l'aide de \\
l'instruction \INS{STR R0, [SP,\#0x18+var\_18]} est sauvée dans la pile.
\IT{var\_18} est -0x18, donc l'offset est 0, donc la valeur du registre \Reg{0}
(4) est écrite à l'adresse écrite dans \ac{SP}.

\item Passer 1, 2 et 3 par des registres:
Les valeurs des 3 premiers nombres (a,b,c) (respectivement 1, 2, 3) sont passées
par les registres \Reg{1}, \Reg{2} et \Reg{3} juste avant l'appel de \printf, et
les 5 autres valeurs sont passées par la pile:

\item appel de \printf

\myindex{Épilogue de fonction}
\item Épilogue de fonction:

L'instruction \INS{ADD SP, SP, \#0x14} restaure le pointeur \ac{SP} à sa valeur
précédente, nettoyant ainsi la pile.
Bien sûr, ce qui a été stocké sur la pile y reste, mais sera ré-écrit lors de
l'exécution ultérieure de fonctions.

\myindex{ARM!\Instructions!LDR}
L'instruction \INS{LDR PC, [SP+4+var\_4],\#4} charge la valeur sauvée de \ac{LR}
depuis la pile dans le registre \ac{PC}, provoquant ainsi la sortie de la fonction.
Il n'y a pas de point d'xclamation---effectivement, \ac{PC} est d'abord chargé
depuis l'adresse stockées dans \ac{SP} ($4+var\_4=4+(-4)=0$, donc cette instruction
est analogue à INS{LDR PC, [SP],\#4}), et ensuite \ac{SP} est incrémenté de 4.
Il s'agit de \IT{post-index}\footnote{Lire à ce propos: \myref{ARM_postindex_vs_preindex}.}.
Pourquoi est-ce qu'\IDA affiche l'instruction comme ça?
Parqu'il veut illustrer la disposition de la pile et le fait que \GTT{var\_4} est
alloué pour sauver la valeur de \ac{LR} dans la pile locale.
Cette instruction est quelque peu similaire à \INS{POP PC} en x86\footnote{Il est
impossible de définir la valeur de \GTT{IP/EIP/RIP} en utilisant \POP en x86, mais
de toutes façons, vous avez le droit de faire l'analogie.}.

\end{itemize}

\myparagraph{\OptimizingKeilVI: \ThumbMode}

\begin{lstlisting}[style=customasmARM]
.text:0000001C             printf_main2
.text:0000001C
.text:0000001C             var_18 = -0x18
.text:0000001C             var_14 = -0x14
.text:0000001C             var_8  = -8
.text:0000001C
.text:0000001C 00 B5        PUSH    {LR}
.text:0000001E 08 23        MOVS    R3, #8
.text:00000020 85 B0        SUB     SP, SP, #0x14
.text:00000022 04 93        STR     R3, [SP,#0x18+var_8]
.text:00000024 07 22        MOVS    R2, #7
.text:00000026 06 21        MOVS    R1, #6
.text:00000028 05 20        MOVS    R0, #5
.text:0000002A 01 AB        ADD     R3, SP, #0x18+var_14
.text:0000002C 07 C3        STMIA   R3!, {R0-R2}
.text:0000002E 04 20        MOVS    R0, #4
.text:00000030 00 90        STR     R0, [SP,#0x18+var_18]
.text:00000032 03 23        MOVS    R3, #3
.text:00000034 02 22        MOVS    R2, #2
.text:00000036 01 21        MOVS    R1, #1
.text:00000038 A0 A0        ADR     R0, aADBDCDDDEDFDGD ; "a=%d; b=%d; c=%d; d=%d; e=%d; f=%d; g=%"...
.text:0000003A 06 F0 D9 F8  BL      __2printf
.text:0000003E
.text:0000003E             loc_3E   ; CODE XREF: example13_f+16
.text:0000003E 05 B0        ADD     SP, SP, #0x14
.text:00000040 00 BD        POP     {PC}
\end{lstlisting}

La sortie est presque comme dans les exemples précédents. Toutefois, c'est du code
Thumb et les valeurs sont arrangées différemment dans la pile:
8 vient en premier, puis 5, 6, 7 et 4 vient en troisième premier, puis 5, 6, 7 et 4 vient en troisième..

\myparagraph{\OptimizingXcodeIV: \ARMMode}

\begin{lstlisting}[style=customasmARM]
__text:0000290C             _printf_main2
__text:0000290C
__text:0000290C             var_1C = -0x1C
__text:0000290C             var_C  = -0xC
__text:0000290C
__text:0000290C 80 40 2D E9   STMFD  SP!, {R7,LR}
__text:00002910 0D 70 A0 E1   MOV    R7, SP
__text:00002914 14 D0 4D E2   SUB    SP, SP, #0x14
__text:00002918 70 05 01 E3   MOV    R0, #0x1570
__text:0000291C 07 C0 A0 E3   MOV    R12, #7
__text:00002920 00 00 40 E3   MOVT   R0, #0
__text:00002924 04 20 A0 E3   MOV    R2, #4
__text:00002928 00 00 8F E0   ADD    R0, PC, R0
__text:0000292C 06 30 A0 E3   MOV    R3, #6
__text:00002930 05 10 A0 E3   MOV    R1, #5
__text:00002934 00 20 8D E5   STR    R2, [SP,#0x1C+var_1C]
__text:00002938 0A 10 8D E9   STMFA  SP, {R1,R3,R12}
__text:0000293C 08 90 A0 E3   MOV    R9, #8
__text:00002940 01 10 A0 E3   MOV    R1, #1
__text:00002944 02 20 A0 E3   MOV    R2, #2
__text:00002948 03 30 A0 E3   MOV    R3, #3
__text:0000294C 10 90 8D E5   STR    R9, [SP,#0x1C+var_C]
__text:00002950 A4 05 00 EB   BL     _printf
__text:00002954 07 D0 A0 E1   MOV    SP, R7
__text:00002958 80 80 BD E8   LDMFD  SP!, {R7,PC}
\end{lstlisting}

\myindex{ARM!\Instructions!STMFA}
\myindex{ARM!\Instructions!STMIB}
Presque la même chose que ce que nous avons déjà vu, avec l'exception de l'instruction
\INS{STMFA} (Store Multiple Full Ascending), qui est un synonyme de l'instruction
\INS{STMIB} (Store Multiple Increment Before).
Cette instruction incrémente la valeur du registre \ac{SP} et écrit seulement
après la valeur registre suivant dans la mémoire, plutôt que d'effectuer ces
deux actions dans l'ordre inverse.

Une autre chose qui accroche le regard est que les instructions semblent être arrangées
de manière aléatoire.
Par exemple, la valeur dans le registre \Reg{0} est manipulée en trois endroits,
aux adresses \GTT{0x2918}, \GTT{0x2920} et \GTT{0x2928}, alors qu'il serait possible
de le faire en un seul endroit.

Toutefois, le compilateur qui optimise doit avoir ses propres raisons d'ordonner
les instructions pour avoir une plus grande éfficacité à l'exécution.

D'habitude, le processeur essaye d'exécuter simultanément des instructions situées côte à côte.\\
Par exemple, des instructions comme \INS{MOVT R0, \#0} et \INS{ADD R0, PC, R0} ne
peuvent pas être exécutées simultanément puisqu'elles modifient toutes deux le
registre \Reg{0}.
D'un autre côté, les instructions \INS{MOVT R0, \#0} et \INS{MOV R2, \#4} peuvent
être exécutées simlutanément puisque leurs effets n'interfèrent pas l'un avec l'autre
lors de leurs exécution.
Probablement que le compilateur essaye de générer du code arrangé de cette façon
(lorsque c'est possible).

\myparagraph{\OptimizingXcodeIV: \ThumbTwoMode}

\begin{lstlisting}[style=customasmARM]
__text:00002BA0               _printf_main2
__text:00002BA0
__text:00002BA0               var_1C = -0x1C
__text:00002BA0               var_18 = -0x18
__text:00002BA0               var_C  = -0xC
__text:00002BA0
__text:00002BA0 80 B5          PUSH     {R7,LR}
__text:00002BA2 6F 46          MOV      R7, SP
__text:00002BA4 85 B0          SUB      SP, SP, #0x14
__text:00002BA6 41 F2 D8 20    MOVW     R0, #0x12D8
__text:00002BAA 4F F0 07 0C    MOV.W    R12, #7
__text:00002BAE C0 F2 00 00    MOVT.W   R0, #0
__text:00002BB2 04 22          MOVS     R2, #4
__text:00002BB4 78 44          ADD      R0, PC  ; char *
__text:00002BB6 06 23          MOVS     R3, #6
__text:00002BB8 05 21          MOVS     R1, #5
__text:00002BBA 0D F1 04 0E    ADD.W    LR, SP, #0x1C+var_18
__text:00002BBE 00 92          STR      R2, [SP,#0x1C+var_1C]
__text:00002BC0 4F F0 08 09    MOV.W    R9, #8
__text:00002BC4 8E E8 0A 10    STMIA.W  LR, {R1,R3,R12}
__text:00002BC8 01 21          MOVS     R1, #1
__text:00002BCA 02 22          MOVS     R2, #2
__text:00002BCC 03 23          MOVS     R3, #3
__text:00002BCE CD F8 10 90    STR.W    R9, [SP,#0x1C+var_C]
__text:00002BD2 01 F0 0A EA    BLX      _printf
__text:00002BD6 05 B0          ADD      SP, SP, #0x14
__text:00002BD8 80 BD          POP      {R7,PC}
\end{lstlisting}

La sortie est presque la même que dans l'exemple précédent, avec l'exception que
des instructions Thumb sont utilisées à la place.
% FIXME: also STMIA is used instead of STMIB,
% which is why it uses LR, which is 4 bytes ahead of SP

\myparagraph{ARM64}

\mysubparagraph{GCC (Linaro) 4.9 \NonOptimizing}

\lstinputlisting[caption=GCC (Linaro) 4.9 \NonOptimizing,style=customasmARM]{patterns/03_printf/ARM/ARM8_O0_EN.lst}

Les 8 premiers arguments sont passés dans des registres X- ou W-: \ARMPCS.
Un pointeur de chaîne nécessite un registre 64-bit, donc il est passé dans \RegX{0}.
Toutes les autres valeurs ont un type \Tint 32-bit, donc elles sont stockées dans
la partie 32-bit des registres (W-).
Le 9ème argument (8) est passé par la pile.
En effet: il m'est pas possible de passer un grand nombre d'arguments par les registres,
car le nombre de registres est limité.

GCC (Linaro) 4.9 \Optimizing génère le même code.
}
\ITA{\subsubsection{ARM: 8 arguments}

Usiamo nuovamente l'esempio con 9 argomenti della sezione precedente: \myref{example_printf8_x64}.

\lstinputlisting[style=customc]{patterns/03_printf/2.c}

\myparagraph{\OptimizingKeilVI: \ARMMode}

\begin{lstlisting}
.text:00000028             main
.text:00000028
.text:00000028             var_18 = -0x18
.text:00000028             var_14 = -0x14
.text:00000028             var_4  = -4
.text:00000028
.text:00000028 04 E0 2D E5  STR    LR, [SP,#var_4]!
.text:0000002C 14 D0 4D E2  SUB    SP, SP, #0x14
.text:00000030 08 30 A0 E3  MOV    R3, #8
.text:00000034 07 20 A0 E3  MOV    R2, #7
.text:00000038 06 10 A0 E3  MOV    R1, #6
.text:0000003C 05 00 A0 E3  MOV    R0, #5
.text:00000040 04 C0 8D E2  ADD    R12, SP, #0x18+var_14
.text:00000044 0F 00 8C E8  STMIA  R12, {R0-R3}
.text:00000048 04 00 A0 E3  MOV    R0, #4
.text:0000004C 00 00 8D E5  STR    R0, [SP,#0x18+var_18]
.text:00000050 03 30 A0 E3  MOV    R3, #3
.text:00000054 02 20 A0 E3  MOV    R2, #2
.text:00000058 01 10 A0 E3  MOV    R1, #1
.text:0000005C 6E 0F 8F E2  ADR    R0, aADBDCDDDEDFDGD ; "a=%d; b=%d; c=%d; d=%d; e=%d; f=%d; g=%"...
.text:00000060 BC 18 00 EB  BL     __2printf
.text:00000064 14 D0 8D E2  ADD    SP, SP, #0x14
.text:00000068 04 F0 9D E4  LDR    PC, [SP+4+var_4],#4
\end{lstlisting}

Il codice puo' essre diviso in piu' parti

\myindex{Function prologue}
\begin{itemize}
\item Preambolo della funzione:

\myindex{ARM!\Instructions!STR}
La prima istruzione \INS{STR LR, [SP,\#var\_4]!} salva \ac{LR} sullo stack, poiche' questo registro sara' usato per la chiamata a \printf.
Il punto esclamativo all fine indica il \IT{pre-index}.

Questo implica che \ac{SP} deve essere prima decrementato di 4, e successivamente \ac{LR} sara' salvato all'indirizzo memorizzato in \ac{SP}.
Tutto cio' e' simile a \PUSH in x86.
Maggiori informazioni qui: \myref{ARM_postindex_vs_preindex}.

\myindex{ARM!\Instructions!SUB}
La seconda istruzione \INS{SUB SP, SP, \#0x14} decrementa \ac{SP} (lo \gls{stack pointer}) per allocare \GTT{0x14} (20) byte sullo stack.
Infatti dobbiamo passare 5 valori a 32-bit tramite lo stack per la funzione \printf, e ciascuno di essi occupa 4 byte, che e' esattamente $5*4=20$.
Gli altri 4 valori a 32-bit saranno passati tramite registri.

\item Passaggio di 5, 6, 7 e 8 tramite lo stack: sono memorizzati nei registri \Reg{0}, \Reg{1}, \Reg{2} e \Reg{3}, rispettivamente.\\
Successivamente l'istruzione \INS{ADD R12, SP, \#0x18+var\_14} scrive l'indirizzo dello stack, dove queste 4 variabili saranno memorizzate,
nel registri \Reg{12}.
\myindex{IDA!var\_?}
\IT{var\_14} e' una macro assembly, uguale a -0x14, creata da \IDA per visualizzare in maniera conveniente il codice che accede allo stack.
Le macro \IT{var\_?} generate da \IDA riflettono le variabili locali nello stack.

Quindi, \GTT{SP+4} sara' memorizzato nel registro \Reg{12}.
\myindex{ARM!\Instructions!STMIA}
L'istruzione successiva \INS{STMIA R12, {R0-R3}} scrive il contenuto dei registri \Reg{0}-\Reg{3} alla memoria puntata da \Reg{12}.
\INS{STMIA} e' abbreviazione per \IT{Store Multiple Increment After}. 
\IT{\q{Increment After}} (incrementa dopo) implica che \Reg{12} deve essere incrementato di 4 dopo ciascuna scrittura di un valore nei registri.

\item Passaggio di 4 tramite lo stack: 4 e' memorizzato in \Reg{0} e questo valore, con l'aiuto dell'istruzione \INS{STR R0, [SP,\#0x18+var\_18]}, viene salvato sullo stack.
\IT{var\_18} e' -0x18, quindi l'offset deve essere 0, da cui il valore dal registro \Reg{0} (4) sara' scritto all'indirizzo memorizzato in \ac{SP}.

\item Passaggio di 1, 2 e 3 tramite registri:
I valori dei primi 3 numeri (a, b, c) (1, 2, 3 rispettivamente) sono passatti attraverso i registri
\Reg{1}, \Reg{2} e \Reg{3}
poco prima della chiamata a \printf call, e gli altri 5 valori sono passati tramite lo stack:

\item chiamata a \printf.

\myindex{Function epilogue}
\item epilogo della funzione:


L'istruzione \INS{ADD SP, SP, \#0x14} ripristina il puntatore \ac{SP} al suo valore precedente, pulendo quindi lo stack.
Ovviamente quello che era stato memorizzato nello stack rimarra li', e sara' probabilmente riscritto interamente durante l'e
secuzione delle funzioni seguenti.

\myindex{ARM!\Instructions!LDR}
L'istruzione \INS{LDR PC, [SP+4+var\_4],\#4} carica il valore di \ac{LR} salvato dallo stack nel nel registro \ac{PC}, causando quindi
l'uscita dalla funzione.
Non c'e' il punto esclamativo ---infatti \ac{PC} e' caricato prima dall'indirizzo memorizzato in \ac{SP} ($4+var\_4=4+(-4)=0$, questa istruzione e' quindi analoga a \INS{LDR PC, [SP],\#4}), e successivamente \ac{SP} e' incrementato di 4.
Questo e' detto \IT{post-index}\footnote{Maggiori dettagli: \myref{ARM_postindex_vs_preindex}.}.
Perche' \IDA mostra l'istruzione in quel modo?
Perche' vuole illustrare il layout dello stack ed il fatto che \GTT{var\_4} e' allocata per salvare il valore di \ac{LR} nello stack locale.
Questa istruzione e' piu' o meno simile a \INS{POP PC} in x86\footnote{E' impossibile settare il valore di \GTT{IP/EIP/RIP} usando \POP in x86, ma in ogni caso hai capito l'analogia.}.

\end{itemize}

\myparagraph{\OptimizingKeilVI: \ThumbMode}

\begin{lstlisting}
.text:0000001C             printf_main2
.text:0000001C
.text:0000001C             var_18 = -0x18
.text:0000001C             var_14 = -0x14
.text:0000001C             var_8  = -8
.text:0000001C
.text:0000001C 00 B5        PUSH    {LR}
.text:0000001E 08 23        MOVS    R3, #8
.text:00000020 85 B0        SUB     SP, SP, #0x14
.text:00000022 04 93        STR     R3, [SP,#0x18+var_8]
.text:00000024 07 22        MOVS    R2, #7
.text:00000026 06 21        MOVS    R1, #6
.text:00000028 05 20        MOVS    R0, #5
.text:0000002A 01 AB        ADD     R3, SP, #0x18+var_14
.text:0000002C 07 C3        STMIA   R3!, {R0-R2}
.text:0000002E 04 20        MOVS    R0, #4
.text:00000030 00 90        STR     R0, [SP,#0x18+var_18]
.text:00000032 03 23        MOVS    R3, #3
.text:00000034 02 22        MOVS    R2, #2
.text:00000036 01 21        MOVS    R1, #1
.text:00000038 A0 A0        ADR     R0, aADBDCDDDEDFDGD ; "a=%d; b=%d; c=%d; d=%d; e=%d; f=%d; g=%"...
.text:0000003A 06 F0 D9 F8  BL      __2printf
.text:0000003E
.text:0000003E             loc_3E   ; CODE XREF: example13_f+16
.text:0000003E 05 B0        ADD     SP, SP, #0x14
.text:00000040 00 BD        POP     {PC}
\end{lstlisting}

L'output e' quasi identico al precedente esempio. Tuttavia questo e' codice Thumb e i valori sono disposti nello stack in modo differente:
8 per primo, quindi 5, 6, 7, e infine 4.

\myparagraph{\OptimizingXcodeIV: \ARMMode}

\begin{lstlisting}
__text:0000290C             _printf_main2
__text:0000290C
__text:0000290C             var_1C = -0x1C
__text:0000290C             var_C  = -0xC
__text:0000290C
__text:0000290C 80 40 2D E9   STMFD  SP!, {R7,LR}
__text:00002910 0D 70 A0 E1   MOV    R7, SP
__text:00002914 14 D0 4D E2   SUB    SP, SP, #0x14
__text:00002918 70 05 01 E3   MOV    R0, #0x1570
__text:0000291C 07 C0 A0 E3   MOV    R12, #7
__text:00002920 00 00 40 E3   MOVT   R0, #0
__text:00002924 04 20 A0 E3   MOV    R2, #4
__text:00002928 00 00 8F E0   ADD    R0, PC, R0
__text:0000292C 06 30 A0 E3   MOV    R3, #6
__text:00002930 05 10 A0 E3   MOV    R1, #5
__text:00002934 00 20 8D E5   STR    R2, [SP,#0x1C+var_1C]
__text:00002938 0A 10 8D E9   STMFA  SP, {R1,R3,R12}
__text:0000293C 08 90 A0 E3   MOV    R9, #8
__text:00002940 01 10 A0 E3   MOV    R1, #1
__text:00002944 02 20 A0 E3   MOV    R2, #2
__text:00002948 03 30 A0 E3   MOV    R3, #3
__text:0000294C 10 90 8D E5   STR    R9, [SP,#0x1C+var_C]
__text:00002950 A4 05 00 EB   BL     _printf
__text:00002954 07 D0 A0 E1   MOV    SP, R7
__text:00002958 80 80 BD E8   LDMFD  SP!, {R7,PC}
\end{lstlisting}

\myindex{ARM!\Instructions!STMFA}
\myindex{ARM!\Instructions!STMIB}
Quasi lo stesso codice visto prima, ad eccezione dell'istruzione \INS{STMFA} (Store Multiple Full Ascending),
che e' sinonimo di \INS{STMIB} (Store Multiple Increment Before). 
Questa istruzione incrementa il valore nel registro \ac{SP} e solo successivamente scrive il prossimo valore del registro in memoria, invece che operare le due azioni in ordine inverso.

Un'altra cosa che salta all'occhio e' che le istruzioni sono disponste in maniera apparentemente casuale.
Ad esempio, il valore nel registro \Reg{0} e' manipolato in tre posti diversi
agli indirizzi \GTT{0x2918}, \GTT{0x2920} e \GTT{0x2928}, quando invece sarebbe stato possibile farlo in un punto solo.

Ad ogni modo, il compilatore ottimizzante avra' avuto le sue ragioni per ordinare le istruzioni in questa maniera ed ottenere una maggiore efficacia durante l'esecuzione del codice.

Solitamente il processero prova ad eseguire simultaneamentele istruzioni vicine.
Ad esempio, istruzioni come \INS{MOVT R0, \#0} e
\INS{ADD R0, PC, R0} non possono essere eseguite simultaneamente poiche' entrambe modificano il registro \Reg{0}. 
D'altra parte, \INS{MOVT R0, \#0} e \INS{MOV R2, \#4} 
possono invece essere eseguite simultaneamente poiche' l'effetto della loro esecuzione non genera conflitti tra loro.
Presumibilmente, il compilatore prova a generare codice in questo modo (quando possibile).
 
\myparagraph{\OptimizingXcodeIV: \ThumbTwoMode}

\begin{lstlisting}
__text:00002BA0               _printf_main2
__text:00002BA0
__text:00002BA0               var_1C = -0x1C
__text:00002BA0               var_18 = -0x18
__text:00002BA0               var_C  = -0xC
__text:00002BA0
__text:00002BA0 80 B5          PUSH     {R7,LR}
__text:00002BA2 6F 46          MOV      R7, SP
__text:00002BA4 85 B0          SUB      SP, SP, #0x14
__text:00002BA6 41 F2 D8 20    MOVW     R0, #0x12D8
__text:00002BAA 4F F0 07 0C    MOV.W    R12, #7
__text:00002BAE C0 F2 00 00    MOVT.W   R0, #0
__text:00002BB2 04 22          MOVS     R2, #4
__text:00002BB4 78 44          ADD      R0, PC  ; char *
__text:00002BB6 06 23          MOVS     R3, #6
__text:00002BB8 05 21          MOVS     R1, #5
__text:00002BBA 0D F1 04 0E    ADD.W    LR, SP, #0x1C+var_18
__text:00002BBE 00 92          STR      R2, [SP,#0x1C+var_1C]
__text:00002BC0 4F F0 08 09    MOV.W    R9, #8
__text:00002BC4 8E E8 0A 10    STMIA.W  LR, {R1,R3,R12}
__text:00002BC8 01 21          MOVS     R1, #1
__text:00002BCA 02 22          MOVS     R2, #2
__text:00002BCC 03 23          MOVS     R3, #3
__text:00002BCE CD F8 10 90    STR.W    R9, [SP,#0x1C+var_C]
__text:00002BD2 01 F0 0A EA    BLX      _printf
__text:00002BD6 05 B0          ADD      SP, SP, #0x14
__text:00002BD8 80 BD          POP      {R7,PC}
\end{lstlisting}

L'output e' quasi lo stesso dell'esempio precedente, ad eccezione dell'uso di istruzioni Thumb. 
% FIXME: also STMIA is used instead of STMIB,
% which is why it uses LR, which is 4 bytes ahead of SP

\myparagraph{ARM64}

\mysubparagraph{\NonOptimizing GCC (Linaro) 4.9}

% TODO translate to Italian:
\lstinputlisting[caption=\NonOptimizing GCC (Linaro) 4.9]{patterns/03_printf/ARM/ARM8_O0_EN.lst}

I primi 8 argomenti sono passati nei registri X- o W-: \ARMPCS.
Un puntatore ad una string richiede un registro a 64-bit, quindi e' passato in \RegX{0}.
Tutti gli altri valori hanno timpo \Tint a 32-bit, quindi sono memorizzati nella parte a 32-bit dei registri (W-).
Il nono argomento (8) e' passato tramite lo stack.
Infatti non e' possibile passare un grande numero di argomenti tramite registri, in quanto il loro numero e' limitato.

\Optimizing GCC (Linaro) 4.9 genera lo stesso codice.
}
\JPN{\subsubsection{ARM: 8つの引数}

前のセクションの9つの引数を使って例を再利用してみましょう:\myref{example_printf8_x64}

\lstinputlisting[style=customc]{patterns/03_printf/2.c}

\myparagraph{\OptimizingKeilVI: \ARMMode}

\begin{lstlisting}[style=customasmARM]
.text:00000028             main
.text:00000028
.text:00000028             var_18 = -0x18
.text:00000028             var_14 = -0x14
.text:00000028             var_4  = -4
.text:00000028
.text:00000028 04 E0 2D E5  STR    LR, [SP,#var_4]!
.text:0000002C 14 D0 4D E2  SUB    SP, SP, #0x14
.text:00000030 08 30 A0 E3  MOV    R3, #8
.text:00000034 07 20 A0 E3  MOV    R2, #7
.text:00000038 06 10 A0 E3  MOV    R1, #6
.text:0000003C 05 00 A0 E3  MOV    R0, #5
.text:00000040 04 C0 8D E2  ADD    R12, SP, #0x18+var_14
.text:00000044 0F 00 8C E8  STMIA  R12, {R0-R3}
.text:00000048 04 00 A0 E3  MOV    R0, #4
.text:0000004C 00 00 8D E5  STR    R0, [SP,#0x18+var_18]
.text:00000050 03 30 A0 E3  MOV    R3, #3
.text:00000054 02 20 A0 E3  MOV    R2, #2
.text:00000058 01 10 A0 E3  MOV    R1, #1
.text:0000005C 6E 0F 8F E2  ADR    R0, aADBDCDDDEDFDGD ; "a=%d; b=%d; c=%d; d=%d; e=%d; f=%d; g=%"...
.text:00000060 BC 18 00 EB  BL     __2printf
.text:00000064 14 D0 8D E2  ADD    SP, SP, #0x14
.text:00000068 04 F0 9D E4  LDR    PC, [SP+4+var_4],#4
\end{lstlisting}

このコードはいくつかの部分に分けることができます:

\myindex{Function prologue}
\begin{itemize}
\item 関数プロローグ:

\myindex{ARM!\Instructions!STR}
最初の\INS{STR LR, [SP,\#var\_4]!}命令は、このレジスタを \printf 呼び出しに使用する予定であるため、\ac{LR}をスタックに保存します。 
最後の感嘆符は、事前索引を示します。

これは、まず\ac{SP}を4減少させた後、\ac{SP}に格納されたアドレスに\ac{LR}を保存することを意味します。
これはx86のPUSHに似ています。 
もっと読む:\myref{ARM_postindex_vs_preindex}

\myindex{ARM!\Instructions!SUB}
第2の\INS{SUB SP, SP, \#0x14}命令は、スタック上に0x14(20)バイトを割り当てるために\ac{SP}(\gls{stack pointer})を減少させる。 
実際には、スタックを介して5つの32ビット値を \printf 関数に渡さなければならず、それぞれが4バイトを占めます。これは正確に$5*4=20$です。
他の4つの32ビット値は、レジスタに通される。

\item スタックを介して5,6,7および8を渡す:それらはそれぞれ\Reg{0}, \Reg{1}, \Reg{2} と \Reg{3}レジスタに格納される。\\
次に、\INS{ADD R12, SP, \#0x18+var\_14}命令は、これら4つの変数が格納されるスタックアドレスを\Reg{12}レジスタに書き込みます。 
\myindex{IDA!var\_?}
\IT{var\_14}は、スタックにアクセスするコードを便利に表示するために \IDA によって作成された-0x14に等しいアセンブリマクロです。 
\IDA によって生成される\IT{var\_?}マクロは、スタック内のローカル変数を反映します。

したがって、\GTT{SP+4}は\Reg{12}レジスタに格納されます。\\
\myindex{ARM!\Instructions!STMIA}
次の\INS{STMIA R12, {R0-R3}}命令は、レジスタ\Reg{0}-\Reg{3}の内容を\Reg{12}が指すメモリに書き込みます。 
\INS{STMIA}は、\IT{後に複数のインクリメント}を格納します。 
\IT{\q{Increment After}}は、各レジスタ値が書き込まれた後に\Reg{12}が4ずつ増加することを意味する。

\item スタックを介して4を渡す:4が\Reg{0}に格納され、\INS{STR R0, [SP,\#0x18+var\_18]}命令の助けを借りてこの値がスタックに保存されます。
\IT{var\_18}は-0x18なので、オフセットは0になります。したがって、\Reg{0}レジスタ(4)の値は\ac{SP}に書き込まれたアドレスに書き込まれます。

\item レジスタ経由で1,2,3を渡す
最初の3つの数値(a、b、c)(それぞれ1,2,3)の値は、printf()呼び出しの直前に\Reg{1}、 \Reg{2}、 \Reg{3}レジスタに渡されます。
他の5つの値はスタックを介して渡されます。

\item \printf 呼び出し。

\myindex{Function epilogue}
\item 関数エピローグ:

\INS{ADD SP, SP, \#0x14}命令は\ac{SP}ポインタを元の値に戻して、スタックに格納されているものをすべて取り消します。
もちろん、スタックに格納されているものはそこにとどまりますが、後続の関数の実行中にすべて書き換えられます。

\myindex{ARM!\Instructions!LDR}
\INS{LDR PC, [SP+4+var\_4],\#4}命令は、保存された\ac{LR}値をスタックから\ac{PC}レジスタにロードして、機能を終了させます。
感嘆符はありません。次に、\ac{SP}は($4+var\_4=4+(-4)=0$に格納されているアドレスから最初にロードされるため、この命令は\INS{LDR PC, [SP],\#4}に似ています)、 \ac{SP}は4だけ増加します。これは\IT{ポストインデックス}\footnote{もっと読む: \myref{ARM_postindex_vs_preindex}}と呼ばれます。
なぜ \IDA はそのような指示を表示するのですか?
なぜなら、スタックレイアウトと、\GTT{var\_4}がローカルスタックの\ac{LR}値を保存するために割り当てられているという事実を説明したいからです。
この命令は、x86の\INS{POP PC}と多少似ています。
\footnote{x86では \POP を使って\GTT{IP/EIP/RIP}の値を設定することは不可能ですが、アナロジーとしてはよいでしょう}

\end{itemize}

\myparagraph{\OptimizingKeilVI: \ThumbMode}

\begin{lstlisting}[style=customasmARM]
.text:0000001C             printf_main2
.text:0000001C
.text:0000001C             var_18 = -0x18
.text:0000001C             var_14 = -0x14
.text:0000001C             var_8  = -8
.text:0000001C
.text:0000001C 00 B5        PUSH    {LR}
.text:0000001E 08 23        MOVS    R3, #8
.text:00000020 85 B0        SUB     SP, SP, #0x14
.text:00000022 04 93        STR     R3, [SP,#0x18+var_8]
.text:00000024 07 22        MOVS    R2, #7
.text:00000026 06 21        MOVS    R1, #6
.text:00000028 05 20        MOVS    R0, #5
.text:0000002A 01 AB        ADD     R3, SP, #0x18+var_14
.text:0000002C 07 C3        STMIA   R3!, {R0-R2}
.text:0000002E 04 20        MOVS    R0, #4
.text:00000030 00 90        STR     R0, [SP,#0x18+var_18]
.text:00000032 03 23        MOVS    R3, #3
.text:00000034 02 22        MOVS    R2, #2
.text:00000036 01 21        MOVS    R1, #1
.text:00000038 A0 A0        ADR     R0, aADBDCDDDEDFDGD ; "a=%d; b=%d; c=%d; d=%d; e=%d; f=%d; g=%"...
.text:0000003A 06 F0 D9 F8  BL      __2printf
.text:0000003E
.text:0000003E             loc_3E   ; CODE XREF: example13_f+16
.text:0000003E 05 B0        ADD     SP, SP, #0x14
.text:00000040 00 BD        POP     {PC}
\end{lstlisting}

出力は前の例とほぼ同じです。 ただし、これはThumbコードであり、値はスタックに別々にパックされます。
8が先に進み、次に5,6,7、および4が3番目に進みます。

\myparagraph{\OptimizingXcodeIV: \ARMMode}

\begin{lstlisting}[style=customasmARM]
__text:0000290C             _printf_main2
__text:0000290C
__text:0000290C             var_1C = -0x1C
__text:0000290C             var_C  = -0xC
__text:0000290C
__text:0000290C 80 40 2D E9   STMFD  SP!, {R7,LR}
__text:00002910 0D 70 A0 E1   MOV    R7, SP
__text:00002914 14 D0 4D E2   SUB    SP, SP, #0x14
__text:00002918 70 05 01 E3   MOV    R0, #0x1570
__text:0000291C 07 C0 A0 E3   MOV    R12, #7
__text:00002920 00 00 40 E3   MOVT   R0, #0
__text:00002924 04 20 A0 E3   MOV    R2, #4
__text:00002928 00 00 8F E0   ADD    R0, PC, R0
__text:0000292C 06 30 A0 E3   MOV    R3, #6
__text:00002930 05 10 A0 E3   MOV    R1, #5
__text:00002934 00 20 8D E5   STR    R2, [SP,#0x1C+var_1C]
__text:00002938 0A 10 8D E9   STMFA  SP, {R1,R3,R12}
__text:0000293C 08 90 A0 E3   MOV    R9, #8
__text:00002940 01 10 A0 E3   MOV    R1, #1
__text:00002944 02 20 A0 E3   MOV    R2, #2
__text:00002948 03 30 A0 E3   MOV    R3, #3
__text:0000294C 10 90 8D E5   STR    R9, [SP,#0x1C+var_C]
__text:00002950 A4 05 00 EB   BL     _printf
__text:00002954 07 D0 A0 E1   MOV    SP, R7
__text:00002958 80 80 BD E8   LDMFD  SP!, {R7,PC}
\end{lstlisting}

\myindex{ARM!\Instructions!STMFA}
\myindex{ARM!\Instructions!STMIB}
\INS{STMFA}(Store Multiple Increment Before)命令の同義語である\INS{STMIB}(Store Multiple Full Ascending)命令を除き、既に見てきたものとほぼ同じです。
この命令は、\ac{SP}レジスタの値を増加させ、逆の順序でこれら2つの動作を実行するのではなく、次のレジスタ値をメモリに書き込むだけです。

目を引くもう一つのことは、命令が一見無作為に配置されていることです。
例えば、R0レジスタの値は、アドレス\GTT{0x2918}, \GTT{0x2920} and \GTT{0x2928}の3箇所で操作できます。

しかしながら、最適化コンパイラは、実行中により高い効率を達成するために、命令をどのように順序付けするかに関する独自の理由を有することができる。

通常、プロセッサは、並んで配置された命令を同時に実行しようと試みます。
たとえば、\INS{MOVT R0, \#0}、\INS{ADD R0, PC, R0}などの命令は、
両方とも\Reg{0}レジスタを変更するため、同時に実行することはできません。
一方、\INS{MOVT R0, \#0}、\INS{MOV R2, \#4}命令は、
実行の影響が互いに矛盾しないため、同時に実行することができます。
おそらく、コンパイラはそのような方法でコードを生成しようと試みます(どこでも可能です)。

\myparagraph{\OptimizingXcodeIV: \ThumbTwoMode}

\begin{lstlisting}[style=customasmARM]
__text:00002BA0               _printf_main2
__text:00002BA0
__text:00002BA0               var_1C = -0x1C
__text:00002BA0               var_18 = -0x18
__text:00002BA0               var_C  = -0xC
__text:00002BA0
__text:00002BA0 80 B5          PUSH     {R7,LR}
__text:00002BA2 6F 46          MOV      R7, SP
__text:00002BA4 85 B0          SUB      SP, SP, #0x14
__text:00002BA6 41 F2 D8 20    MOVW     R0, #0x12D8
__text:00002BAA 4F F0 07 0C    MOV.W    R12, #7
__text:00002BAE C0 F2 00 00    MOVT.W   R0, #0
__text:00002BB2 04 22          MOVS     R2, #4
__text:00002BB4 78 44          ADD      R0, PC  ; char *
__text:00002BB6 06 23          MOVS     R3, #6
__text:00002BB8 05 21          MOVS     R1, #5
__text:00002BBA 0D F1 04 0E    ADD.W    LR, SP, #0x1C+var_18
__text:00002BBE 00 92          STR      R2, [SP,#0x1C+var_1C]
__text:00002BC0 4F F0 08 09    MOV.W    R9, #8
__text:00002BC4 8E E8 0A 10    STMIA.W  LR, {R1,R3,R12}
__text:00002BC8 01 21          MOVS     R1, #1
__text:00002BCA 02 22          MOVS     R2, #2
__text:00002BCC 03 23          MOVS     R3, #3
__text:00002BCE CD F8 10 90    STR.W    R9, [SP,#0x1C+var_C]
__text:00002BD2 01 F0 0A EA    BLX      _printf
__text:00002BD6 05 B0          ADD      SP, SP, #0x14
__text:00002BD8 80 BD          POP      {R7,PC}
\end{lstlisting}

Thumb命令が代わりに使用される点を除いて、出力は前の例とほぼ同じです。
% FIXME: also STMIA is used instead of STMIB,
% which is why it uses LR, which is 4 bytes ahead of SP

\myparagraph{ARM64}

\mysubparagraph{\NonOptimizing GCC (Linaro) 4.9}

\lstinputlisting[caption=\NonOptimizing GCC (Linaro) 4.9,style=customasmARM]{patterns/03_printf/ARM/ARM8_O0_JPN.lst}

最初の8つの引数は、XレジスタまたはWレジスタに渡されます。 \ARMPCS
文字列ポインタは64ビットのレジスタを必要とするため、\RegX{0}で渡されます。 
それ以外の値はすべてint型32ビット型なので、レジスタ(W-)の32ビット部分に格納されます。 
第9引数(8)はスタックを介して渡されます。 
実際には、レジスタの数が限られているため、多数の引数をレジスタに渡すことはできません。

\Optimizing GCC (Linaro) 4.9 は同じコードを生成します。
}

\EN{\subsection{MIPS}

\subsubsection{3 arguments}

\myparagraph{\Optimizing GCC 4.4.5}

The main difference with the \q{\HelloWorldSectionName} example is that in this case \printf is called
instead of \puts and 3 more arguments are passed through the registers \$5\dots \$7 (or \$A0\dots \$A2).
That is why these registers are prefixed with A-, which implies they are used for function arguments passing.

\lstinputlisting[caption=\Optimizing GCC 4.4.5 (\assemblyOutput),style=customasm]{patterns/03_printf/MIPS/printf3.O3_EN.s}

\lstinputlisting[caption=\Optimizing GCC 4.4.5 (IDA),style=customasm]{patterns/03_printf/MIPS/printf3.O3.IDA_EN.lst}

\IDA has coalesced pair of \INS{LUI} and \INS{ADDIU} instructions into one \INS{LA} pseudo instruction.
That's why there are no instruction at address 0x1C: because \INS{LA} \IT{occupies} 8 bytes.

\myparagraph{\NonOptimizing GCC 4.4.5}

\NonOptimizing GCC is more verbose:

\lstinputlisting[caption=\NonOptimizing GCC 4.4.5 (\assemblyOutput),style=customasm]{patterns/03_printf/MIPS/printf3.O0_EN.s}

\lstinputlisting[caption=\NonOptimizing GCC 4.4.5 (IDA),style=customasm]{patterns/03_printf/MIPS/printf3.O0.IDA_EN.lst}

\subsubsection{8 arguments}

Let's use again the example with 9 arguments from the previous section: \myref{example_printf8_x64}.

\lstinputlisting[style=customc]{patterns/03_printf/2.c}

\myparagraph{\Optimizing GCC 4.4.5}

Only the first 4 arguments are passed in the \$A0 \dots \$A3 registers, the rest are passed via the stack.
\myindex{MIPS!O32}

This is the O32 calling convention (which is the most common one in the MIPS world).
Other calling conventions (like N32) may use the registers for different purposes.

\myindex{MIPS!\Instructions!SW}

\INS{SW} abbreviates \q{Store Word} (from register to memory).
MIPS lacks instructions for storing a value into memory, so an instruction pair has to be used instead (\INS{LI}/\INS{SW}).

\lstinputlisting[caption=\Optimizing GCC 4.4.5 (\assemblyOutput),style=customasm]{patterns/03_printf/MIPS/printf8.O3_EN.s}

\lstinputlisting[caption=\Optimizing GCC 4.4.5 (IDA),style=customasm]{patterns/03_printf/MIPS/printf8.O3.IDA_EN.lst}

\myparagraph{\NonOptimizing GCC 4.4.5}

\NonOptimizing GCC is more verbose:

\lstinputlisting[caption=\NonOptimizing GCC 4.4.5 (\assemblyOutput),style=customasm]{patterns/03_printf/MIPS/printf8.O0_EN.s}

\lstinputlisting[caption=\NonOptimizing GCC 4.4.5 (IDA),style=customasm]{patterns/03_printf/MIPS/printf8.O0.IDA_EN.lst}

}
\RU{\subsection{MIPS}

\subsubsection{3 аргумента}

\myparagraph{\Optimizing GCC 4.4.5}

Главное отличие от примера \q{\HelloWorldSectionName} в том, что здесь на самом деле
вызывается \printf вместо \puts и ещё три аргумента передаются в регистрах  \$5\dots \$7 (или \$A0\dots \$A2).
Вот почему эти регистры имеют префикс A-. Это значит, что они используются для передачи аргументов.

\lstinputlisting[caption=\Optimizing GCC 4.4.5 (\assemblyOutput),style=customasm]{patterns/03_printf/MIPS/printf3.O3_RU.s}

\lstinputlisting[caption=\Optimizing GCC 4.4.5 (IDA),style=customasm]{patterns/03_printf/MIPS/printf3.O3.IDA_RU.lst}

\IDA объединила пару инструкций \INS{LUI} и \INS{ADDIU} в одну псевдоинструкцию \INS{LA}.
Вот почему здесь нет инструкции по адресу 0x1C: потому что \INS{LA} \IT{занимает} 8 байт.

\myparagraph{\NonOptimizing GCC 4.4.5}

\NonOptimizing GCC более многословен:

\lstinputlisting[caption=\NonOptimizing GCC 4.4.5 (\assemblyOutput),style=customasm]{patterns/03_printf/MIPS/printf3.O0_RU.s}

\lstinputlisting[caption=\NonOptimizing GCC 4.4.5 (IDA),style=customasm]{patterns/03_printf/MIPS/printf3.O0.IDA_RU.lst}

\subsubsection{8 аргументов}

Снова воспользуемся примером с 9-ю аргументами из предыдущей секции: \myref{example_printf8_x64}.

\lstinputlisting[style=customc]{patterns/03_printf/2.c}

\myparagraph{\Optimizing GCC 4.4.5}

Только 4 первых аргумента передаются в регистрах \$A0 \dots \$A3, так что остальные передаются через стек.

\myindex{MIPS!O32}
Это соглашение о вызовах O32 (самое популярное в мире MIPS).
Другие соглашения о вызовах (например N32) могут наделять регистры другими функциями.

\myindex{MIPS!\Instructions!SW}
\INS{SW} означает \q{Store Word} (записать слово из регистра в память).
В MIPS нет инструкции для записи значения в память, так что для этого используется пара инструкций (\INS{LI}/\INS{SW}).

\lstinputlisting[caption=\Optimizing GCC 4.4.5 (\assemblyOutput),style=customasm]{patterns/03_printf/MIPS/printf8.O3_RU.s}

\lstinputlisting[caption=\Optimizing GCC 4.4.5 (IDA),style=customasm]{patterns/03_printf/MIPS/printf8.O3.IDA_RU.lst}

\myparagraph{\NonOptimizing GCC 4.4.5}

\NonOptimizing GCC более многословен:

\lstinputlisting[caption=\NonOptimizing GCC 4.4.5 (\assemblyOutput),style=customasm]{patterns/03_printf/MIPS/printf8.O0_RU.s}

\lstinputlisting[caption=\NonOptimizing GCC 4.4.5 (IDA),style=customasm]{patterns/03_printf/MIPS/printf8.O0.IDA_RU.lst}

}
\ITA{\subsection{MIPS}

\subsubsection{3 argomenti}

\myparagraph{\Optimizing GCC 4.4.5}

La differenza principale con l'esempio \q{\HelloWorldSectionName} e' che in questo caso \printf e' chiamata 
al posto di \puts, e 3 argomenti aggiuntivi sono passati attraverso i registri \$5\dots \$7 (o \$A0\dots \$A2).
Questo e' il motivo per cui questi registri hanno il prefisso A-, che implica il loro uso per il passaggio di argomenti di funzioni.

% TODO translate to Italian:
\lstinputlisting[caption=\Optimizing GCC 4.4.5 (\assemblyOutput),style=customasm]{patterns/03_printf/MIPS/printf3.O3_EN.s}

% TODO translate to Italian:
\lstinputlisting[caption=\Optimizing GCC 4.4.5 (IDA)]{patterns/03_printf/MIPS/printf3.O3.IDA_EN.lst}

\IDA ha fuso le coppie di istruzioni \INS{LUI} e \INS{ADDIU} in una unica pseudoistruzione \INS{LA}.
Questo e' il motivo per cui non c'e' nessuna istruzione all'indirizzo 0x1C: perche' \INS{LA} \IT{occupa} 8 byte.%

\myparagraph{\NonOptimizing GCC 4.4.5}

\NonOptimizing GCC e' piu' verboso:

% TODO translate to Italian:
\lstinputlisting[caption=\NonOptimizing GCC 4.4.5 (\assemblyOutput),style=customasm]{patterns/03_printf/MIPS/printf3.O0_EN.s}

% TODO translate to Italian:
\lstinputlisting[caption=\NonOptimizing GCC 4.4.5 (IDA)]{patterns/03_printf/MIPS/printf3.O0.IDA_EN.lst}

\subsubsection{8 argomenti}

Usiamo nuovamente l'esempio con 9 argomenti dalla sezione prcedente: \myref{example_printf8_x64}.

\lstinputlisting[style=customc]{patterns/03_printf/2.c}

\myparagraph{\Optimizing GCC 4.4.5}

Solo i primi 4 argomenti sono passati nei registri \$A0 \dots \$A3, gli altri sono passati tramite lo stack.
\myindex{MIPS!O32}

Questa e' la calling convention O32 (che e' la piu' comune nel mondo MIPS).
Altre calling conventions (come N32) possono usare i registri per scopi diversi.

\myindex{MIPS!\Instructions!SW}

\INS{SW} e' l'abbreviazione di \q{Store Word} (da un registro alla memoria).
MIPS manca di istruzioni per memorizzare un valore in memoria, e' quindi necessario usare una coppia di istruzioni (LI/SW).

% TODO translate to Italian:
\lstinputlisting[caption=\Optimizing GCC 4.4.5 (\assemblyOutput),style=customasm]{patterns/03_printf/MIPS/printf8.O3_EN.s}

% TODO translate to Italian:
\lstinputlisting[caption=\Optimizing GCC 4.4.5 (IDA)]{patterns/03_printf/MIPS/printf8.O3.IDA_EN.lst}

\myparagraph{\NonOptimizing GCC 4.4.5}

\NonOptimizing GCC e' piu' verboso:

% TODO translate to Italian:
\lstinputlisting[caption=\NonOptimizing GCC 4.4.5 (\assemblyOutput),style=customasm]{patterns/03_printf/MIPS/printf8.O0_EN.s}

% TODO translate to Italian:
\lstinputlisting[caption=\NonOptimizing GCC 4.4.5 (IDA)]{patterns/03_printf/MIPS/printf8.O0.IDA_EN.lst}
}
\FR{\subsection{MIPS}

\subsubsection{3 arguments}

\myparagraph{GCC 4.4.5 \Optimizing}

La différence principale avec l'exemple \q{\HelloWorldSectionName} est que dans ce cas, \printf est appelée
à la place de \puts et 3 arguments de plus sont passés à travers les registres \$5\dots \$7 (ou \$A0\dots \$A2).
C'est pourquoi ces registres sont préfixés avec A-, ceci sous-entend qu'ils
sont utilisés pour le passage des arguments aux fonctions.

\lstinputlisting[caption=GCC 4.4.5 \Optimizing (\assemblyOutput),style=customasmMIPS]{patterns/03_printf/MIPS/printf3.O3_FR.s}

\lstinputlisting[caption=GCC 4.4.5 \Optimizing (IDA),style=customasmMIPS]{patterns/03_printf/MIPS/printf3.O3.IDA_FR.lst}

\IDA a agrégé la paire d'instructions \INS{LUI} et \INS{ADDIU} en une pseudo instruction \INS{LA}.
C'est pourquoi il n'y a pas d'instruction à l'adresse 0x1C: car \INS{LA} \IT{occupe} 8 octets.

\myparagraph{GCC 4.4.5 \NonOptimizing}

GCC \NonOptimizing est plus verbeux:

\lstinputlisting[caption=GCC 4.4.5 \NonOptimizing (\assemblyOutput),style=customasmMIPS]{patterns/03_printf/MIPS/printf3.O0_FR.s}

\lstinputlisting[caption=GCC 4.4.5 \NonOptimizing (IDA),style=customasmMIPS]{patterns/03_printf/MIPS/printf3.O0.IDA_FR.lst}

\subsubsection{8 arguments}

Utilisons encore l'exemple de la section précédente avec 9 arguments: \myref{example_printf8_x64}.

\lstinputlisting[style=customc]{patterns/03_printf/2.c}

\myparagraph{GCC 4.4.5 \Optimizing}

Seul les 4 premiers arguments sont passés dans les registres \$A0 \dots \$A3,
les autres sont passés par la pile.
\myindex{MIPS!O32}

C'est la convention d'appel O32 (qui est la plus commune dans le monde MIPS).
D'autres conventions d'appel (comme N32) peuvent utiliser les registres à d'autres fins.

\myindex{MIPS!\Instructions!SW}

\INS{SW} est l'abbréviation de \q{Store Word} (depuis un registre vers la mémoire).
En MIPS, il manque une instructions pour stocker une valeur dans la mémoire, donc
une paire d'instruction doit être utilisée à la place (\INS{LI}/\INS{SW}).

\lstinputlisting[caption=GCC 4.4.5 \Optimizing (\assemblyOutput),style=customasmMIPS]{patterns/03_printf/MIPS/printf8.O3_FR.s}

\lstinputlisting[caption=GCC 4.4.5 \Optimizing (IDA),style=customasmMIPS]{patterns/03_printf/MIPS/printf8.O3.IDA_FR.lst}

\myparagraph{GCC 4.4.5 \NonOptimizing}

GCC \NonOptimizing est plus verbeux:

\lstinputlisting[caption=\NonOptimizing GCC 4.4.5 (\assemblyOutput),style=customasmMIPS]{patterns/03_printf/MIPS/printf8.O0_FR.s}

\lstinputlisting[caption=\NonOptimizing GCC 4.4.5 (IDA),style=customasmMIPS]{patterns/03_printf/MIPS/printf8.O0.IDA_FR.lst}

}


\subsection{\Conclusion{}}

Here is a rough skeleton of the function call:

\begin{lstlisting}[caption=x86,style=customasmx86]
...
PUSH 3rd argument
PUSH 2nd argument
PUSH 1st argument
CALL function
; modify stack pointer (if needed)
\end{lstlisting}

\begin{lstlisting}[caption=x64 (MSVC),style=customasmx86]
MOV RCX, 1st argument
MOV RDX, 2nd argument
MOV R8, 3rd argument
MOV R9, 4th argument
...
PUSH 5th, 6th argument, etc. (if needed)
CALL function
; modify stack pointer (if needed)
\end{lstlisting}

\begin{lstlisting}[caption=x64 (GCC),style=customasmx86]
MOV RDI, 1st argument
MOV RSI, 2nd argument
MOV RDX, 3rd argument
MOV RCX, 4th argument
MOV R8, 5th argument
MOV R9, 6th argument
...
PUSH 7th, 8th argument, etc. (if needed)
CALL function
; modify stack pointer (if needed)
\end{lstlisting}

\begin{lstlisting}[caption=ARM,style=customasmARM]
MOV R0, 1st argument
MOV R1, 2nd argument
MOV R2, 3rd argument
MOV R3, 4th argument
; pass 5th, 6th argument, etc., in stack (if needed)
BL function
; modify stack pointer (if needed)
\end{lstlisting}

\begin{lstlisting}[caption=ARM64,style=customasmARM]
MOV X0, 1st argument
MOV X1, 2nd argument
MOV X2, 3rd argument
MOV X3, 4th argument
MOV X4, 5th argument
MOV X5, 6th argument
MOV X6, 7th argument
MOV X7, 8th argument
; pass 9th, 10th argument, etc., in stack (if needed)
BL function
; modify stack pointer (if needed)
\end{lstlisting}

\myindex{MIPS!O32}
\begin{lstlisting}[caption=MIPS (O32 calling convention),style=customasmMIPS]
LI $4, 1st argument ; AKA $A0
LI $5, 2nd argument ; AKA $A1
LI $6, 3rd argument ; AKA $A2
LI $7, 4th argument ; AKA $A3
; pass 5th, 6th argument, etc., in stack (if needed)
LW temp_reg, address of function
JALR temp_reg
\end{lstlisting}

\subsection{By the way}

\myindex{fastcall}
By the way, this difference between the arguments passing in x86, x64, 
fastcall, ARM and MIPS is a good illustration of the fact that the CPU is oblivious to how the arguments are passed to functions. 
It is also possible to create a hypothetical compiler able to pass arguments 
via a special structure without using stack at all.

\myindex{MIPS!O32}
MIPS \$A0 \dots \$A3 registers are labeled this way only for convenience (that is in the O32 calling convention).
Programmers may use any other register (well, maybe except \$ZERO) 
to pass data or use any other calling convention.

The \ac{CPU} is not aware of calling conventions whatsoever.

We may also recall how new coming assembly language programmers passing arguments into
other functions: usually via registers, without any explicit order, or even via global variables.
Of course, it works fine.

