\section[Linear congruential generator]{Linearer Kongruenzgenerator als Pseudozufallszahlengenerator}
\myindex{\CStandardLibrary!rand()}
\label{LCG_simple}
Ein linearer Kongruenzgenerator ist die wohl einfachste Form der Zufallszahlenerzeugung.

Er wird heutzutage nicht mehr so häufig eingesetzt\footnote{Der Mersenne-Twister ist besser}, kann aber, weil er so
einfach ist (nur eine Multiplikation, eine Addition und eine AND-Operation), dennoch gut als Beispiel dienen.

\lstinputlisting[style=customc]{patterns/145_LCG/rand_DE.c}
Es gibt hier zwei Funktionen: die erste wird verwendet und den internen Zustand zu initialisieren und die zweite wird
zum Erzeugen der Pseudozufallszahlen aufgerufen.

Wir sehen, dass im Algorithmus zwei Konstanten verwendet werden.
Sie stamen aus [William H. Press and Saul A. Teukolsky and William T. Vetterling and Brian P. Flannery, \IT{Numerical Recipes}, (2007)].

Definieren wir sie über den \TT{\#define} \CCpp Ausdruck. Es handelt sich um ein Makro.

Der Unterschied zwischen einem \CCpp Makro und einer Konstanten ist, dass alle Makros durch den \CCpp Präprozessor durch
mit ihrem Wert ersetzt werden und dadurch im Gegensatz zu Variablen keinen Speicherplatz verbrauchen.

Eine Konstante ist im Gegensatz dazu eine nur lesbare Variable.

Es ist möglich einen Pointer (oder eine Adresse) einer Konstanten zu verwenden; das ist mit einem Makro nicht möglich.

Die letzte AND-Operation wird benötigt, da \TT{my\_rand()} gemäß C-Standard einen Wert zwischen 0 und 32767 zurückgeben
muss.

Wenn man 32-Bit-Pseudozufallszahlen benötigt, kann die AND-Operation einfach weggelassen werden.

\subsection{x86}

\lstinputlisting[caption=\Optimizing MSVC 2013,style=customasmx86]{patterns/145_LCG/rand_MSVC_2013_x86_Ox.asm}
Hier sehen wir, dass beide Konstanten in den Code eingebettet wurden. Es wurde kein Speicher für sie reserviert.

Die Funktion \TT{my\_srand()} kopiert ihren Eingabewert in die interne Variable \TT{rand\_state}.

\TT{my\_rand()} nimmt diese, berechnet den nächsten \TT{rand\_state}, schneidet ihn ab und belässt ihn im \EAX Register.

Die nicht optimierte Version ist umfangreicher:

\lstinputlisting[caption=\NonOptimizing MSVC 2013,style=customasmx86]{patterns/145_LCG/rand_MSVC_2013_x86.asm}

\subsection{x64}
Die x64 Version ist größtenteils identisch und verwendet 32-Bit-Register anstelle der 64-Bit-Register (wir arbeiten
hier mit \Tint Werten).

Die Funktion \TT{my\_srand()} nimmt seinen Eingabewert aus dem Register \ECX und nicht vom Stack:

\lstinputlisting[caption=\Optimizing MSVC 2013 x64,style=customasmx86]{patterns/145_LCG/rand_MSVC_2013_x64_Ox_DE.asm}

GCC erzeugt fast den gleichen Code.

\subsection{32-bit ARM}

\lstinputlisting[caption=\OptimizingKeilVI (\ARMMode),style=customasmARM]{patterns/145_LCG/rand.s_Keil_ARM_O3_DE.s}
Es ist nicht möglich 32-Bit-Konstanten in ARM Befehle einzubetten, sodass Keil diese extern speichern und dann
zusätzlich laden muss. Eine interessante Sache ist, dass es ebenfalls nicth möglich ist, die Konstante 0x7FFF
einzubetten.
Was Keil dann tut, ist \IT{rand\_state} um 17 Bits nach links zu verschieben und dann um 17 Bits nach rechts zu
Verchieben.
Die entspricht Ausdruck $(rand\_state \ll 17) \gg 17$ in \CCpp.
Es scheint eine nutzlose Operation zu sein, löscht aber die oberen 17 Bits und lässt die 15 niederen Bits intakt und das
ist genau was wir wollen.\\\\

\Optimizing Keil für Thumb mode erzeugt fast den gleichen Code.

\subsection{MIPS}

\lstinputlisting[caption=\Optimizing GCC 4.4.5 (IDA),style=customasmMIPS]{patterns/145_LCG/MIPS_O3_IDA_DE.lst}
Hier sehen wir nur eine Konstante (0x3C6EF35F oder 1013904223).
Wo befindet sich die andere (1664525)?

Es scheint, dass die Multiplikation mit 1664525 nur durch Verschieben und Addieren durchgeführt wird.
Überprüfen wir diese Vermutung:

\lstinputlisting[style=customc]{patterns/145_LCG/test.c}

\lstinputlisting[caption=\Optimizing GCC 4.4.5 (IDA),style=customasmMIPS]{patterns/145_LCG/test_O3_MIPS.lst}

Tatsächlich!

\subsubsection{MIPS Relocation}
Wir werden uns auch anschauen wie solche Operationen wie das Laden und Werten aus dem Speicher und das Speichern
tatsächlich funktionieren.

Die Listings wurden mit IDA erzeugt, was einige Details versteckt.

Wir lassen objdump zweimal laufen: um das disassemblierte Listing und die Relocation List zu erhalten:

\lstinputlisting[caption=\Optimizing GCC 4.4.5 (objdump)]{patterns/145_LCG/MIPS_O3_objdump.txt}
Betrachten wir die beiden Relocations für die Funktion \TT{my\_srand()}.

Die erste für die Adresse 0 hat den Typ \TT{R\_MIPS\_HI16}.
Die zweite für die Adresse 8 hat den Typ \TT{R\_MIPS\_LO16}.

Das bedeutet, dass die Adresse zu Beginn des .bss Segments zu den Befehlen an der Adresse 0 (höherer Teil der Adresse)
bzw. 8 (niederer Teil der Adresse) geschrieben wird.

Die Variable \TT{rand\_state} befindet sich ganz am Anfang des .bss Segments.

Wir finden hier Nullen in den Operanden der Befehle \LUI und \SW, da sich hier noch nichts befindet---der Compiler weiß
noch nicht was hier hingeschrieben werden soll.

Der Linker wird dieses Problem beheben und der höhere Teil der Adresse wird in den Operanden von \LUI geschriebne und
der niedere Teil der Adresse in den Operanden von \SW.

\SW addiert den niederen Teil der Adresse und den Inhalt des Registers \$V0 (hier befindet sich der höhere Teil).

Das gleiche geschieht mit der Funktion \TT{my\_rand()}: Die R\_MIPS\_HI16 Relocation teilt dem Linker mit, dass der
höhere Teil der Adresse des .bss Segments in den Befehl \LUI geschrieben wird.

Der höhere Teil der Adresse der Variablen rand\_state befindet sich im Register \$V1.

Der Befehl \LW an der Adresse 0x10 addiert den höheren und niederen Teil und lädt den Wert der Variablen rand\_state
nach \$V0.

Der Befehl \SW an der Adresse 0x54 summiert erneut auf und speichert dann den neuen Wert in der globale Variable
rand\_state.

\IDA arbeitet die Relocations beim Laden ab, sodass diese Details verborgen bleiben, aber wir sollten wissen, dass es
sie gibt.
% TODO add example of compiled binary, GDB example, etc...


\subsection{Threadsichere Version des Beispiels}
Eine threadsichere Version des Beispiels wird später hier gezeigt:\myref{LCG_TLS}. 
