\subsection{MIPS}

\lstinputlisting[caption=\Optimizing GCC 4.4.5 (IDA),style=customasmMIPS]{patterns/145_LCG/MIPS_O3_IDA_DE.lst}
Hier sehen wir nur eine Konstante (0x3C6EF35F oder 1013904223).
Wo befindet sich die andere (1664525)?

Es scheint, dass die Multiplikation mit 1664525 nur durch Verschieben und Addieren durchgeführt wird.
Überprüfen wir diese Vermutung:

\lstinputlisting[style=customc]{patterns/145_LCG/test.c}

\lstinputlisting[caption=\Optimizing GCC 4.4.5 (IDA),style=customasmMIPS]{patterns/145_LCG/test_O3_MIPS.lst}

Tatsächlich!

\subsubsection{MIPS Relocation}
Wir werden uns auch anschauen wie solche Operationen wie das Laden und Werten aus dem Speicher und das Speichern
tatsächlich funktionieren.

Die Listings wurden mit IDA erzeugt, was einige Details versteckt.

Wir lassen objdump zweimal laufen: um das disassemblierte Listing und die Relocation List zu erhalten:

\lstinputlisting[caption=\Optimizing GCC 4.4.5 (objdump)]{patterns/145_LCG/MIPS_O3_objdump.txt}
Betrachten wir die beiden Relocations für die Funktion \TT{my\_srand()}.

Die erste für die Adresse 0 hat den Typ \TT{R\_MIPS\_HI16}.
Die zweite für die Adresse 8 hat den Typ \TT{R\_MIPS\_LO16}.

Das bedeutet, dass die Adresse zu Beginn des .bss Segments zu den Befehlen an der Adresse 0 (höherer Teil der Adresse)
bzw. 8 (niederer Teil der Adresse) geschrieben wird.

Die Variable \TT{rand\_state} befindet sich ganz am Anfang des .bss Segments.

Wir finden hier Nullen in den Operanden der Befehle \LUI und \SW, da sich hier noch nichts befindet---der Compiler weiß
noch nicht was hier hingeschrieben werden soll.

Der Linker wird dieses Problem beheben und der höhere Teil der Adresse wird in den Operanden von \LUI geschriebne und
der niedere Teil der Adresse in den Operanden von \SW.

\SW addiert den niederen Teil der Adresse und den Inhalt des Registers \$V0 (hier befindet sich der höhere Teil).

Das gleiche geschieht mit der Funktion \TT{my\_rand()}: Die R\_MIPS\_HI16 Relocation teilt dem Linker mit, dass der
höhere Teil der Adresse des .bss Segments in den Befehl \LUI geschrieben wird.

Der höhere Teil der Adresse der Variablen rand\_state befindet sich im Register \$V1.

Der Befehl \LW an der Adresse 0x10 addiert den höheren und niederen Teil und lädt den Wert der Variablen rand\_state
nach \$V0.

Der Befehl \SW an der Adresse 0x54 summiert erneut auf und speichert dann den neuen Wert in der globale Variable
rand\_state.

\IDA arbeitet die Relocations beim Laden ab, sodass diese Details verborgen bleiben, aber wir sollten wissen, dass es
sie gibt.
% TODO add example of compiled binary, GDB example, etc...
