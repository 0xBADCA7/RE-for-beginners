\subsection{Molti casi}

Se uno statement \TT{switch()} contiene molti casi, per il compilatore non è molto conveniente enettere codice troppo lungo con un sacco di 
istruzioni \JE/\JNE.

\lstinputlisting[label=switch_lot_c,style=customc]{patterns/08_switch/2_lot/lot.c}

\subsubsection{x86}

\myparagraph{\NonOptimizing MSVC}

We get (MSVC 2010):

\lstinputlisting[caption=MSVC 2010,style=customasmx86]{patterns/08_switch/2_lot/lot_msvc_EN.asm}

\myindex{jumptable}

What we see here is a set of \printf calls with various arguments. 
All they have not only addresses in the memory of the process, but also internal symbolic labels assigned 
by the compiler. 
All these labels are also mentioned in the \TT{\$LN11@f} internal table.

At the function start, if $a$ is greater than 4, control flow is passed to label 
\TT{\$LN1@f}, where \printf with argument \TT{'something unknown'} is called.

But if the value of $a$ is less or equals to 4, then it gets multiplied by 4 and added with the \TT{\$LN11@f} 
table address. That is how an address inside the table is constructed, pointing exactly to the 
element we need. For example, let's say $a$ is equal to 2. $2*4 = 8$ (all table elements 
are addresses in a 32-bit process and that is why all elements are 4 bytes wide). 
The address of the \TT{\$LN11@f} table + 8 is the table element where the \TT{\$LN4@f} label is stored.
\JMP fetches the \TT{\$LN4@f} address from the table and jumps to it.

This table is sometimes called \IT{jumptable} or \IT{branch table}\footnote{The whole method was once called 
\IT{computed GOTO} in early versions of Fortran:
\href{http://go.yurichev.com/17122}{wikipedia}.
Not quite relevant these days, but what a term!}.

Then the corresponding \printf is called with argument \TT{'two'}.\\
Literally, the \TT{jmp DWORD PTR \$LN11@f[ecx*4]} instruction implies
\IT{jump to the DWORD that is stored at address} \TT{\$LN11@f + ecx * 4}.

\TT{npad} (\myref{sec:npad}) is an assembly language macro that align the next label so that it will be stored at an address aligned on a 4 byte
(or 16 byte) boundary.
This is very suitable for the processor since it is able to fetch 32-bit values from memory through the memory bus,
cache memory, etc., in a more effective way if it is aligned.

\clearpage
\mysubparagraph{\olly}
\myindex{\olly}

Let's try this example in \olly.
The input value of the function (2) is loaded into \EAX: 

\begin{figure}[H]
\centering
\myincludegraphics{patterns/08_switch/2_lot/olly1.png}
\caption{\olly: function's input value is loaded in \EAX}
\label{fig:switch_lot_olly1}
\end{figure}

\clearpage
The input value is checked, is it bigger than 4? 
If not, the \q{default} jump is not taken:
\begin{figure}[H]
\centering
\myincludegraphics{patterns/08_switch/2_lot/olly2.png}
\caption{\olly: 2 is no bigger than 4: no jump is taken}
\label{fig:switch_lot_olly2}
\end{figure}

\clearpage
Here we see a jumptable:

\begin{figure}[H]
\centering
\myincludegraphics{patterns/08_switch/2_lot/olly3.png}
\caption{\olly: calculating destination address using jumptable}
\label{fig:switch_lot_olly3}
\end{figure}

Here we've clicked \q{Follow in Dump} $\rightarrow$ \q{Address constant}, so now we see the \IT{jumptable} in the data window.
These are 5 32-bit values\footnote{They are underlined by \olly because
these are also FIXUPs: \myref{subsec:relocs}, we are going to come back to them later}.
\ECX is now 2, so the third element (can be indexed as 2\footnote{About indexing, see also: \ref{arrays_at_one}}) of the table is to be used.
It's also possible to click \q{Follow in Dump} $\rightarrow$ 
\q{Memory address} and \olly will show the element addressed by the \JMP instruction. 
That's \TT{0x010B103A}.

\clearpage
After the jump we are at \TT{0x010B103A}: the code printing \q{two} will now be executed:

\begin{figure}[H]
\centering
\myincludegraphics{patterns/08_switch/2_lot/olly4.png}
\caption{\olly: now we at the \IT{case:} label}
\label{fig:switch_lot_olly4}
\end{figure}


\myparagraph{\NonOptimizing GCC}
\label{switch_lot_GCC}

Let's see what GCC 4.4.1 generates:

\lstinputlisting[caption=GCC 4.4.1,style=customasmx86]{patterns/08_switch/2_lot/lot_gcc.asm}

\myindex{x86!\Registers!JMP}

It is almost the same, with a little nuance: argument \TT{arg\_0} is multiplied by 4 by
shifting it to left by 2 bits (it is almost the same as multiplication by 4)~(\myref{SHR}).
Then the address of the label is taken from the \TT{off\_804855C} array, stored in 
\EAX, and then \TT{JMP EAX} does the actual jump.


\subsubsection{ARM: \OptimizingKeilVI (\ARMMode)}
\label{sec:SwitchARMLot}

\lstinputlisting[caption=\OptimizingKeilVI (\ARMMode),style=customasmARM]{patterns/08_switch/2_lot/lot_ARM_ARM_O3.asm}

This code makes use of the ARM mode feature in which all instructions have a fixed size of 4 bytes.

Let's keep in mind that the maximum value for $a$ is 4 and any greater value will cause
\IT{<<something unknown\textbackslash{}n>>} string to be printed.

\myindex{ARM!\Instructions!CMP}
\myindex{ARM!\Instructions!ADDCC}
The first \TT{CMP R0, \#5} instruction compares the input value of $a$ with 5.

\footnote{ADD---addition}
The next \TT{ADDCC PC, PC, R0,LSL\#2} instruction is being executed only if $R0 < 5$ (\IT{CC=Carry clear / Less than}). 
Consequently, if \TT{ADDCC} does not trigger (it is a $R0 \geq 5$ case), a jump to \IT{default\_case} label will occur.

But if $R0 < 5$ and \TT{ADDCC} triggers, the following is to be happen:

The value in \Reg{0} is multiplied by 4.
In fact, \TT{LSL\#2} at the instruction's suffix stands for \q{shift left by 2 bits}.
But as we will see later~(\myref{division_by_shifting}) in section \q{\ShiftsSectionName}, 
shift left by 2 bits is equivalent to multiplying by 4.

Then we add $R0*4$ to the current value in \ac{PC}, 
thus jumping to one of the \TT{B} (\IT{Branch}) instructions located below.

At the moment of the execution of \TT{ADDCC}, the value in \ac{PC} is 8 bytes ahead (\TT{0x180})
than the address at which the \TT{ADDCC} instruction is located (\TT{0x178}), 
or, in other words, 2 instructions ahead.

\myindex{ARM!Pipeline}

This is how the pipeline in ARM processors works: when \TT{ADDCC} is executed,
the processor at the moment
is beginning to process the instruction after the next one,
so that is why \ac{PC} points there. This has to be memorized.

If $a=0$, then is to be added to the value in \ac{PC},
and the actual value of the \ac{PC} will be written into \ac{PC} (which is 8 bytes ahead)
and a jump to the label \IT{loc\_180} will happen,
which is 8 bytes ahead of the point where the \TT{ADDCC} instruction is.

If $a=1$, then $PC+8+a*4 = PC+8+1*4 = PC+12 = 0x184$ will be written to \ac{PC},
which is the address of the \IT{loc\_184} label.

With every 1 added to $a$, the resulting \ac{PC} is increased by 4.

4 is the instruction length in ARM mode and also, the length of each \TT{B} instruction,
of which there are 5 in row.

Each of these five \TT{B} instructions passes control further, to what was programmed in the \IT{switch()}.

Pointer loading of the corresponding string occurs there, etc.

\subsubsection{ARM: \OptimizingKeilVI (\ThumbMode)}

\lstinputlisting[caption=\OptimizingKeilVI (\ThumbMode),style=customasmARM]{patterns/08_switch/2_lot/lot_ARM_thumb_O3.asm}

\myindex{ARM!\ThumbMode}
\myindex{ARM!\ThumbTwoMode}

One cannot be sure that all instructions in Thumb and Thumb-2 modes has the same size.
It can even be said that in these modes the instructions have variable lengths, just like in x86.

\myindex{jumptable}

So there is a special table added that contains information about how much cases are there (not including 
default-case), and an offset for each with a label to which control must be passed in 
the corresponding case.

\myindex{ARM!Mode switching}
\myindex{ARM!\Instructions!BX}

A special function is present here in order to deal with the table and pass control, \\
named \IT{\_\_ARM\_common\_switch8\_thumb}. 
It starts with \TT{BX PC}, whose function is to switch the processor to ARM-mode.
Then you see the function for table processing. 

It is too advanced to describe it here now, so let's omit it.
% TODO explain it...

\myindex{ARM!\Registers!Link Register}

It is interesting to note that the function uses the \ac{LR} register as a pointer to the table.

Indeed, after calling of this function, \ac{LR} contains the address after\\
\TT{BL \_\_ARM\_common\_switch8\_thumb} instruction, where the table starts.

It is also worth noting that the code is generated as a separate function in order to reuse it, 
so the compiler doesn't generate the same code for every switch() statement.

\IDA successfully perceived it as a service function and a table, and added comments to the labels like\\
\TT{jumptable 000000FA case 0}.


\subsubsection{MIPS}

\lstinputlisting[caption=\Optimizing GCC 4.4.5 (IDA),style=customasm]{patterns/08_switch/2_lot/MIPS_O3_IDA_EN.lst}

\myindex{MIPS!\Instructions!SLTIU}

The new instruction for us is \INS{SLTIU} (\q{Set on Less Than Immediate Unsigned}).
\myindex{MIPS!\Instructions!SLTU}

This is the same as \INS{SLTU} (\q{Set on Less Than Unsigned}), but \q{I} stands for \q{immediate}, 
i.e., a number has to be specified in the instruction itself.

\myindex{MIPS!\Instructions!BNEZ}
\INS{BNEZ} is \q{Branch if Not Equal to Zero}.

Code is very close to the other \ac{ISA}s.
\myindex{MIPS!\Instructions!SLL}
\INS{SLL} (\q{Shift Word Left Logical}) does multiplication by 4.

MIPS is a 32-bit CPU after all, so all addresses in the \IT{jumptable} are 32-bit ones.



\subsubsection{\Conclusion{}}

Stuttura approssimativa di \IT{switch()}:

% TODO: ARM, MIPS skeleton
\lstinputlisting[caption=x86,style=customasmx86]{patterns/08_switch/2_lot/skel1_EN.lst}

Il salto agli indirizzi nella tabella di jump può anche essere implementato usando questa istruzione: \\
\TT{JMP jump\_table[REG*4]}.
oppure \TT{JMP jump\_table[REG*8]} in x64.

Una \IT{jumptable} è semplicemente un array di puntatori, come quello descritto più avanti: \myref{array_of_pointers_to_strings}.
