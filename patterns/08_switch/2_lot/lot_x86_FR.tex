\subsubsection{x86}

\myparagraph{MSVC \NonOptimizing}

Nous obtenons (MSVC 2010):

\lstinputlisting[caption=MSVC 2010,style=customasmx86]{patterns/08_switch/2_lot/lot_msvc_FR.asm}

\myindex{jumptable}

Ce que nous voyons ici est un ensemble d'appels à \printf avec des arguments variés.
Ils ont tous, non seulement des adresses dans la mémoire du processus, mais aussi
des labels symboliques internes assignés par le compilateur.
Tous ces labels ont aussi mentionnés dans la table interne \TT{\$LN11@f}.

Au début de la fonctions, si $a$ est supérieur à 4, l'exécution est passée au
labal \TT{\$LN1@f}, où \printf est appelé avec l'argument \TT{'something unknown'}.

Mais si la valeur de $a$ est inférieure ou égale à 4, elle est alors multipliée
par 4 et ajoutée à l'adresse de la table \TT{\$LN11@f}. C'est ainsi qu'une adresse
à l'intérieur de la table est construite, pointant exactement sur l'élément dont
nous avons besoin. Par exemple, supposons que $a$ soit égale à 2. $2*4 = 8$ (tous
les éléments de la table sont adressés dans un processus 32-bit, c'est pourquoi les
éléments ont une taille de 4 octets).
L'adresse de la table \TT{\$LN11@f} + 8 est celle de l'élément de la table où
le label \TT{\$LN4@f} est stocké.
\JMP prend l'adresse de \TT{\$LN4@f} dans la table et y saute.

Cette table est quelquefois appelée \IT{jumptable} (table de saut) ou \IT{branch table}
(table de branchement)\footnote{L'ensemble de la méthode était appelé \IT{computed
GOTO} (GOTO calculés) dans les premières versions de ForTran:
\href{http://go.yurichev.com/17122}{Wikipédia}.
Pas très pertinent de nos jours, mais quel terme!}.

Le \printf correspondant est appelé avec l'argument \TT{'two'}.\\
Littéralement, l'instruction \TT{jmp DWORD PTR \$LN11@f[ecx*4]} signifie
\IT{sauter au DWORD qui est stocké à l'adresse} \TT{\$LN11@f + ecx * 4}.

\TT{npad} (\myref{sec:npad}) est une macro du langage d'assemblage qui aligne le
label suivant de telle sorte qu'il soit stocké à une adresse alignée sur une limite
de 4 octets (ou 16 octets).
C'est très adapté pour le processeur puisqu'il est capable d'aller chercher des
valeurs 32-bit dans la mémoire à travers le bus mémoire, la mémoire cache, etc.,
de façons beaucoup plus efficace si c'est aligné.

\clearpage
\mysubparagraph{\olly}
\myindex{\olly}

Essayons cet exemple dans \olly.
La valeur d'entrée de la fonction (2) est chargée dans \EAX:

\begin{figure}[H]
\centering
\myincludegraphics{patterns/08_switch/2_lot/olly1.png}
\caption{\olly: la valeur d'entrée de la fonction est chargée dans \EAX}
\label{fig:switch_lot_olly1}
\end{figure}

\clearpage
La valeur entrée est testée, est-elle plus grande que 4?
Si non, le saut par \q{défaut} n'est pas pris:
\begin{figure}[H]
\centering
\myincludegraphics{patterns/08_switch/2_lot/olly2.png}
\caption{\olly: 2 n'est pas plus grand que 4: le saut n'est pas pris}
\label{fig:switch_lot_olly2}
\end{figure}

\clearpage
Ici, nous voyons une table des sauts:

\begin{figure}[H]
\centering
\myincludegraphics{patterns/08_switch/2_lot/olly3.png}
\caption{\olly: calcul de l'adresse de destination en utilisant la table des sauts}
\label{fig:switch_lot_olly3}
\end{figure}

Ici, nous avons cliqué \q{Follow in Dump} $\rightarrow$ \q{Address constant}, donc
nous voyons maintenant la \IT{jumptable} dans la fenêtre des données.
Il y a 5 valeurs 32-bit\footnote{Elles sont soulignées par \olly car ce sont aussi
des FIXUPs: \myref{subsec:relocs}, nous y reviendrons plus tard}.
\ECX contient maintenant 2, donc le troisième élément (peut être indexé par 2\footnote{Á
propos des index de tableaux, lire aussi: \ref{arrays_at_one}}) de la table va être
utilisé.
Il est également possible de cliquer sur \q{Follow in Dump} $\rightarrow$ \q{Memory
address} et \olly va montrer l'élément adressé par l'instruction \JMP.
Il s'agit de \TT{0x010B103A}.

\clearpage
Après le saut, nous sommes en \TT{0x010B103A}: le code qui affiche \q{two} va être
exécuté:

\begin{figure}[H]
\centering
\myincludegraphics{patterns/08_switch/2_lot/olly4.png}
\caption{\olly: maintenant nous sommes au \IT{cas:} label}
\label{fig:switch_lot_olly4}
\end{figure}


\myparagraph{GCC \NonOptimizing}
\label{switch_lot_GCC}

Voyons ce que GCC 4.4.1 génère:

\lstinputlisting[caption=GCC 4.4.1,style=customasmx86]{patterns/08_switch/2_lot/lot_gcc.asm}

\myindex{x86!\Registers!JMP}

C'est presque la même chose, avec une petite nuance: l'argument \TT{arg\_0} est multiplié
par 4 en décalant de 2 bits vers la gauche (c'est presque comme multiplier par 4)~(\myref{SHR}).
Ensuite l'adresse du label est prise depuis le tableau \TT{off\_804855C}, stockée
dans \EAX, et ensuite \TT{JMP EAX} effectue le saut réel.

