\subsection{Fall-through}

Another very popular usage of \TT{switch()} is the fall-through.
Here is a small example:

\lstinputlisting[numbers=left,style=customc]{patterns/08_switch/4_fallthrough/fallthrough.c}

If $type=1$ (R), $read$ is to be set to 1, if $type=2$ (W), $write$ is to be set to 1.
In case of $type=3$ (RW), both $read$ and $write$ is to be set to 1.

The code at line 14 is executed in two cases: if $type=RW$ or if $type=W$.
There is no \q{break} for \q{case RW}x and that's OK.

\subsubsection{MSVC x86}

\lstinputlisting[caption=MSVC 2012,style=customasm]{patterns/08_switch/4_fallthrough/fallthrough_MSVC.asm}

The code mostly resembles what is in the source.
There are no jumps between labels \TT{\$LN4@f} and \TT{\$LN3@f}: so when code flow is at 
\TT{\$LN4@f}, $read$ is first set to 1, then $write$.
This is why it's called fall-through: code flow falls through one piece of code
(setting $read$) to another (setting $write$).

If $type=W$, we land at \TT{\$LN3@f}, so no code setting $read$ to 1 is executed.

\subsubsection{ARM64}

\lstinputlisting[caption=GCC (Linaro) 4.9,style=customasm]{patterns/08_switch/4_fallthrough/fallthrough_ARM64_EN.s}

Merely the same thing.
There are no jumps between labels \TT{.L4} and \TT{.L3}.

