\subsection{Fall-through}

Ещё одно популярное использование оператора \TT{switch()} это т.н. \q{fallthrough} (\q{провал}).
Вот простой пример\footnote{Взято отсюда: \url{https://github.com/azonalon/prgraas/blob/master/prog1lib/lecture_examples/is_whitespace.c}}:

\lstinputlisting[numbers=left,style=customc]{patterns/08_switch/4_fallthrough/fallthrough1.c}

Немного сложнее, из ядра Linux\footnote{\url{https://github.com/torvalds/linux/blob/master/drivers/media/dvb-frontends/lgdt3306a.c}}:

\lstinputlisting[numbers=left,style=customc]{patterns/08_switch/4_fallthrough/fallthrough2.c}

\lstinputlisting[caption=Оптимизирующий GCC 5.4.0 x86,numbers=left,style=customc]{patterns/08_switch/4_fallthrough/fallthrough2.s}

На метку \TT{.L5} управление может перейти если на входе ф-ции число 3250.
Но на эту метку можно попасть и с другой стороны:
мы видим что между вызовом \printf и меткой \TT{.L5} нет никаких пероходов.

Теперь мы можем понять, почему иногда \IT{switch()} является источником ошибок: если забыть дописать \IT{break},
это прекратит выражение \IT{switch()} в \IT{fallthrough}, и вместо одного блока для каждого условия,
будет исполняться сразу несколько.

