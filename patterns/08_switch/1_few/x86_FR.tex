\subsubsection{x86}

\myparagraph{MSVC \NonOptimizing}

Résultat (MSVC 2010):

\lstinputlisting[caption=MSVC 2010,style=customasmx86]{patterns/08_switch/1_few/few_msvc.asm}

Notre fonction avec quelques cas dans switch() est en fait analogue à cette construction:

\lstinputlisting[label=switch_few_ifelse,style=customc]{patterns/08_switch/1_few/few_analogue.c}

\myindex{\CLanguageElements!switch}
\myindex{\CLanguageElements!if}

Si nous utilisons switch() avec quelques cas, il est impossible de savoir si il y
avait un vrai switch() dans le code source, ou un ensemble de directives if().
\myindex{\SyntacticSugar}

Ceci implique que switch() est comme un sucre syntaxique pour un grand nombre de if()
imbriqués.

Il n'y a rien de particulièrement nouveau pour nous dans le code généré, à l'exception
que le compilateur déplace la variable d'entrée $a$ dans une variable locale temporaire
\TT{tv64} \footnote{Les variables locales sur la pile sont préfixées avec \TT{tv}---c'est
ainsi que MSVC appelle les variables internes dont il a besoin}.

Si nous compilons ceci avec GCC 4.4.1, nous obtenons presque le même résultat,
même avec le plus haut niveau d'optimisation (\Othree option).

\myparagraph{MSVC \Optimizing}

% TODO separate various kinds of \TT
% idea: enclose command lines in a specific environment, like \cmdline{} 
% assembly instructions in \asm{} (now both \TT and \q{} are used),
% variables in,  like \var{}
% messages (string constants) in something else, like \strconst
% to separate them all. Now they all use \TT, which is not best
% \INS{} for all instructions including operands? --DY
Maintenant compilons avec l'optimisation dans MSVC (\Ox): \TT{cl 1.c /Fa1.asm /Ox}

\label{JMP_instead_of_RET}
\lstinputlisting[caption=MSVC,style=customasmx86]{patterns/08_switch/1_few/few_msvc_Ox.asm}

Ici, nous voyons quelques hacks moches.

\myindex{x86!\Instructions!JZ}
\myindex{x86!\Instructions!JE}
\myindex{x86!\Instructions!SUB}

Premièrement: la valeurs de $a$ est mise dans \EAX et 0 en est soustrait. Ca semble
absurde, mais cela est fait pour vérifier si la valeur dans \EAX est 0. Si oui, le
flag \ZF est mis (e.g. soustraire de 0 est 0) et le premier saut conditionnel \JE
(\IT{Jump if Equal} saut si égal ou synonyme \JZ~---\IT{Jump if Zero} saut si zéro)
va être effectué et le déroulement du programme passera au label \TT{\$LN4@f}, oú
le message \TT{'zero'} est affiché.
Si le premier saut n'est pas effectué, 1 est soustrait de la valeur d'entrée et si
à une étape le résultat est 0, le saut correspondant sera effectué.

Et si aucun saut n'est exécuté, l'exécution passera au \printf avec comme argument
la chaîne \TT{'something unknown'}.

\label{jump_to_last_printf}
\myindex{\Stack}

Deuxièmement: nous voyons quelque chose d'inhabituel pour nous: un pointeur sur une
chaîne est mis dans la variable $a$ et ensuite \printf est appelé non pas par \CALL,
mais par \JMP. Il y a une explication simple à cela: l'\glslink{caller}{appelant}
pousse une valeur sur la pile et appelle notre fonction via \CALL.
\CALL lui même pousse l'adresse de retour (\ac{RA}) sur la pile et fait un saut
inconditionnel à l'adresse de notre fonction.
Notre fonction, à tout moment de son exécution (car elle ne contient pas d'instruction
qui modifie le pointeur de pile) à le schéma suivant pour sa pile:

\begin{itemize}
\item\ESP---pointe sur \ac{RA}
\item\TT{ESP+4}---pointe sur la variable $a$
\end{itemize}

D'un autre côté, lorsque nous appelons \printf ici nous avons exactement la même
disposition de pile, excepté pour le premier argument de \printf, qui doit pointer
sur la chaîne. Et c'est ce que fait notre code.

Il remplace le premier argument de la fonction par l'adresse de la chaîne et saute
à \printf, comme si nous n'avions pas appelé notre fonction \ttf, mais directement
\printf.
\printf affiche la chaîne sur la \glslink{stdout}{sortie standard} et ensuite exécute
l'instruction \RET qui POPs \ac{RA} de la pile et l'exécution est renvoyée non pas
à \ttf mais plutôt à l'\glslink{caller}{appelant} de \ttf, ignorant la fin de la
fonction \ttf.

\myindex{\CStandardLibrary!longjmp()}
\newcommand{\URLSJ}{\href{http://go.yurichev.com/17121}{wikipedia}}

% TODO \myref{}
Tout ceci est possible car \printf est appeléi, dans tous les cas, tout à la fin
de la fonction \ttf.
Dans un certain sens, c'est similaire à la fonction \TT{longjmp()}\footnote{\URLSJ}.
Et bien sûr, c'est fait dans un but de vitesse d'exécution.

Un cas similaire avec le compilateur ARM est décrit dans la section \q{\PrintfSeveralArgumentsSectionName},
ici~(\myref{ARM_B_to_printf}).

\clearpage
\mysubparagraph{\olly}

Comme cet exemple est compliqué, traçons-le dans \olly.

\olly peut détecter des constructions avec switch(), et ajoute des commentaires utiles.
\EAX contient 2 au début, c'est la valeur du paramètre de la fonction:

\begin{figure}[H]
\centering
\myincludegraphics{patterns/08_switch/1_few/olly1.png}
\caption{\olly: \EAX 
contient maintenant le premier (et unique) argument de la fonction}
\label{fig:switch_few_olly1}
\end{figure}

\clearpage
0 est soustrait de 2 dans \EAX.
Bien sûr, \EAX contient toujours 2.
Mais le flag \ZF est maintenant à 0, indiquant que le résultat est différent de
zéro:

\begin{figure}[H]
\centering
\myincludegraphics{patterns/08_switch/1_few/olly2.png}
\caption{\olly: \SUB exécuté}
\label{fig:switch_few_olly2}
\end{figure}

\clearpage
\DEC est exécuté et \EAX contient maintenant 1.
Mais 1 est différent de zéro, donc le flag \ZF est toujours à 0:

\begin{figure}[H]
\centering
\myincludegraphics{patterns/08_switch/1_few/olly3.png}
\caption{\olly: premier \DEC exécuté}
\label{fig:switch_few_olly3}
\end{figure}

\clearpage
Le \DEC suivant est exécuté.
\EAX contient maintenant 0 et le flag \ZF est mis, car le résultat devient zéro:

\begin{figure}[H]
\centering
\myincludegraphics{patterns/08_switch/1_few/olly4.png}
\caption{\olly: second \DEC exécuté}
\label{fig:switch_few_olly4}
\end{figure}

\olly montre que le saut va être effectué (\IT{Jump is taken}).

\clearpage
Un pointeur sur la chaîne \q{two} est maintenant écrit sur la pile:

\begin{figure}[H]
\centering
\myincludegraphics{patterns/08_switch/1_few/olly5.png}
\caption{\olly: 
pointeur sur la chaîne qui va être écrite à la place du premier argument}
\label{fig:switch_few_olly5}
\end{figure}

% TODO: homogenize numbers
% now they are inconsistent: sometimes plain text, sometimes in math mode
% some kind of \expr{} both for numbers and expressions? --DY
Veuillez noter: l'argument de la fonction courante est 2, et 2 est maintenant
sur la pile, à l'adresse \TT{0x001EF850}.

\clearpage
\MOV écrit le pointeur sur la chaîne à l'adresse \TT{0x001EF850} (voir la fenêtre de la pile).
Puis, le saut est effectué.
Ceci est la première instruction de la fonction \printf dans MSVCR100.DLL (Cet exemple
a été compilé avec le switch /MD):

\begin{figure}[H]
\centering
\myincludegraphics{patterns/08_switch/1_few/olly6.png}
\caption{\olly: première instruction de \printf dans MSVCR100.DLL}
\label{fig:switch_few_olly6}
\end{figure}

Maintenant \printf traite la chaîne à l'adresse \TT{0x00FF3010} comme c'est son
seul argument et l'affiche.

\clearpage
Ceci est la dernière instruction de \printf:

\begin{figure}[H]
\centering
\myincludegraphics{patterns/08_switch/1_few/olly7.png}
\caption{\olly: dernière instruction de \printf dans MSVCR100.DLL}
\label{fig:switch_few_olly7}
\end{figure}

La chaîne \q{two} vient juste d'être affichée dans la fenêtre console.

\clearpage
Maintenant, appuyez sur F7 ou F8 (\stepover) et le retour ne se fait pas sur \ttf,
mais sur \main:

\begin{figure}[H]
\centering
\myincludegraphics{patterns/08_switch/1_few/olly8.png}
\caption{\olly: retourne à \main}
\label{fig:switch_few_olly8}
\end{figure}

Oui, le saut a été direct, depuis les entrailles de \printf vers \main.
Car \ac{RA} dans la pile pointe non pas quelque part dans \ttf, mais en fait sur
\main.
Et \CALL \TT{0x00FF1000} a été l'instruction qui a appelé \ttf.


