\clearpage
\mysubparagraph{\olly}
Da dieses Beispiel komliziert ist, untersuchen wir es mit \olly.

\olly kann solche switch() Konstruktionen erkennen und einige nützliche Kommentare hinzufügen.
\EAX ist zu Beginn auf 2 gesetzt, das entspricht dem Eingabewert der Funktion:


\begin{figure}[H]
\centering
\myincludegraphics{patterns/08_switch/1_few/olly1.png}
\caption{\olly: \EAX 
enthält jetzt das erste (und einzige) Functionsargument}
\label{fig:switch_few_olly1}
\end{figure}

\clearpage
0 wird in vor der 2 in \EAX abgezogen. Natürlich enthält \EAX danach immernoch den Wert 2, aber das \ZF Flag ist jetzt
0, da das Ergebnis der letzten Berechnung nicht 0 ergeben hat.
\begin{figure}[H]
\centering
\myincludegraphics{patterns/08_switch/1_few/olly2.png}
\caption{\olly: \SUB wurde ausgeführt}
\label{fig:switch_few_olly2}
\end{figure}

\clearpage
\DEC wird ausgeführt und \EAX enthält nun 1.
Da 1 ein von null verschiedener Wert ist, ist das \ZF Flag immer noch 0:

\begin{figure}[H]
\centering
\myincludegraphics{patterns/08_switch/1_few/olly3.png}
\caption{\olly: erstes \DEC wurde ausgeführt}
\label{fig:switch_few_olly3}
\end{figure}

\clearpage
Das nächste \DEC wird ausgeführt.
\EAX enthält jetzt 0 und das \ZF Flag wird gesetzt, da eine Berechnung 0 ergeben hat.

\begin{figure}[H]
\centering
\myincludegraphics{patterns/08_switch/1_few/olly4.png}
\caption{\olly: zweites \DEC wurde ausgeführt}
\label{fig:switch_few_olly4}
\end{figure}
\olly zeigt an, dass dieser Sprung jetzt getätigt wird.

\clearpage
Ein Pointer auf den String \q{two} wird jetzt auf den Stack geschrieben:

\begin{figure}[H]
\centering
\myincludegraphics{patterns/08_switch/1_few/olly5.png}
\caption{\olly: 
Pointer auf den String wird an die Stelle des ersten Arguments geschrieben}
\label{fig:switch_few_olly5}
\end{figure}

% TODO: homogenize numbers
% now they are inconsistent: sometimes plain text, sometimes in math mode
% some kind of \expr{} both for numbers and expressions? --DY
Man beachte: das aktuelle Argument der Funktion ist 2 und diese 2 befindet sich im Stack nun an der Adresse
\TT{0x001EF850}.

\clearpage
\MOV schreibt den Pointer auf den String an die Adresse \TT{0x001EF850} (siehe Stackfenster).
Dann wird der Sprung ausgeführt. Dies ist der erste Befehl der Funktion \printf in MSVCR100.DLL. (Dieses Beispiel wurde
mit der Option /MD kompiliert.)
 

\begin{figure}[H]
\centering
\myincludegraphics{patterns/08_switch/1_few/olly6.png}
\caption{\olly: erster Befehl von \printf in MSVCR100.DLL}
\label{fig:switch_few_olly6}
\end{figure}

Die Funktion \printf behandelt den String an der Adresse \TT{0x00FF3010} als (einziges) Argument und gibt ihn aus.

\clearpage
Dies ist der letzte Befehl von \printf:

\begin{figure}[H]
\centering
\myincludegraphics{patterns/08_switch/1_few/olly7.png}
\caption{\olly: letzter Befehl von \printf in MSVCR100.DLL}
\label{fig:switch_few_olly7}
\end{figure}

Der String \q{two} wurde gerade auf der Konsole ausgegeben.

\clearpage
Wir drücken nun F7 oder F8 (\stepover) und kehren\dots nicht zu \ttf , sondern zu \main zurück:

\begin{figure}[H]
\centering
\myincludegraphics{patterns/08_switch/1_few/olly8.png}
\caption{\olly: zurück zu \main}
\label{fig:switch_few_olly8}
\end{figure}
Dieser Sprung wird direkt von \printf zu \main durchgeführt, da \ac{RA} im Stack nicht auf eine Stelle in \ttf, sondern
auf \main zeigt.
Der Befehl \CALL \TT{0x00FF1000} war der eigentliche Befehl, der \ttf aufgerufen hat.

