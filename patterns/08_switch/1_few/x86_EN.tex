\subsubsection{x86}

\myparagraph{\NonOptimizing MSVC}

Result (MSVC 2010):

\lstinputlisting[caption=MSVC 2010,style=customasmx86]{patterns/08_switch/1_few/few_msvc.asm}

Our function with a few cases in switch() is in fact analogous to this construction:

\lstinputlisting[label=switch_few_ifelse,style=customc]{patterns/08_switch/1_few/few_analogue.c}

\myindex{\CLanguageElements!switch}
\myindex{\CLanguageElements!if}

If we work with switch() with a few cases it is impossible to be sure if it was
a real switch() in the source code, or just a pack of if() statements.
\myindex{\SyntacticSugar}

This implies that switch() is like syntactic sugar for a large number of nested if()s.

There is nothing especially new to us in the generated code,
with the exception of the compiler moving input variable $a$ to a temporary local variable \TT{tv64}
\footnote{Local variables in stack are prefixed with \TT{tv}---that's how MSVC names internal variables for its needs}.

If we compile this in GCC 4.4.1, we'll get almost the same result, even with maximal optimization
turned on (\Othree option).

\myparagraph{\Optimizing MSVC}

% TODO separate various kinds of \TT
% idea: enclose command lines in a specific environment, like \cmdline{} 
% assembly instructions in \asm{} (now both \TT and \q{} are used),
% variables in,  like \var{}
% messages (string constants) in something else, like \strconst
% to separate them all. Now they all use \TT, which is not best
% \INS{} for all instructions including operands? --DY
Now let's turn on optimization in MSVC (\Ox): \TT{cl 1.c /Fa1.asm /Ox}

\label{JMP_instead_of_RET}
\lstinputlisting[caption=MSVC,style=customasmx86]{patterns/08_switch/1_few/few_msvc_Ox.asm}

Here we can see some dirty hacks.

\myindex{x86!\Instructions!JZ}
\myindex{x86!\Instructions!JE}
\myindex{x86!\Instructions!SUB}

First: the value of $a$ is placed in \EAX and 0 is subtracted from it. Sounds absurd, but it is done to check if 
the value in \EAX is 0. If yes, the \ZF flag is to be set (e.g. subtracting from 0 is 0) 
and the first conditional jump \JE (\IT{Jump if Equal} or synonym \JZ~---\IT{Jump if Zero}) is to be triggered 
and control flow is to be passed to the \TT{\$LN4@f} label, where the \TT{'zero'} message is being printed. 
If the first jump doesn't get triggered, 1 is subtracted from the input value and if at some stage the result is 0, 
the corresponding jump is to be triggered.

And if no jump gets triggered at all, the control flow passes to \printf with string argument \\
\TT{'something unknown'}.

\label{jump_to_last_printf}
\myindex{\Stack}

Second: we see something unusual for us: a string pointer is placed into the $a$ variable, and 
then \printf is called not via \CALL, but via \JMP. There is a simple explanation for that: 
the \gls{caller} pushes a value to the stack and calls our function via \CALL. 
\CALL itself pushes the return address (\ac{RA}) to the stack and does an unconditional jump to our function address. 
Our function at any point of execution (since it do not contain any instruction that moves the stack 
pointer) has the following stack layout:

\begin{itemize}
\item\ESP---points to \ac{RA}
\item\TT{ESP+4}---points to the $a$ variable 
\end{itemize}

On the other side, when we have to call \printf here we need exactly the same stack 
layout, except for the first \printf argument, which needs to point to the string. 
And that is what our code does.

It replaces the function's first argument with the address of the string and 
jumps to \printf, as if we didn't call our function \ttf, but directly \printf.
\printf prints a string to \gls{stdout} and then executes the \RET instruction, which POPs 
\ac{RA} from the stack and control flow is returned not to \ttf but rather to \ttf's \gls{caller}, 
bypassing the end of the \ttf function.

\myindex{\CStandardLibrary!longjmp()}
\newcommand{\URLSJ}{\href{http://go.yurichev.com/17121}{wikipedia}}

All this is possible because \printf is called right at the end of the \ttf function in all cases. 
In some way, it is similar to the \TT{longjmp()}\footnote{\URLSJ} function.
And of course, it is all done for the sake of speed.

A similar case with the ARM compiler is described in \q{\PrintfSeveralArgumentsSectionName}
section, here~(\myref{ARM_B_to_printf}).

\clearpage
\mysubparagraph{\olly}
\myindex{\olly}

Let's try this example in \olly.
The input value of the function (2) is loaded into \EAX: 

\begin{figure}[H]
\centering
\myincludegraphics{patterns/08_switch/2_lot/olly1.png}
\caption{\olly: function's input value is loaded in \EAX}
\label{fig:switch_lot_olly1}
\end{figure}

\clearpage
The input value is checked, is it bigger than 4? 
If not, the \q{default} jump is not taken:
\begin{figure}[H]
\centering
\myincludegraphics{patterns/08_switch/2_lot/olly2.png}
\caption{\olly: 2 is no bigger than 4: no jump is taken}
\label{fig:switch_lot_olly2}
\end{figure}

\clearpage
Here we see a jumptable:

\begin{figure}[H]
\centering
\myincludegraphics{patterns/08_switch/2_lot/olly3.png}
\caption{\olly: calculating destination address using jumptable}
\label{fig:switch_lot_olly3}
\end{figure}

Here we've clicked \q{Follow in Dump} $\rightarrow$ \q{Address constant}, so now we see the \IT{jumptable} in the data window.
These are 5 32-bit values\footnote{They are underlined by \olly because
these are also FIXUPs: \myref{subsec:relocs}, we are going to come back to them later}.
\ECX is now 2, so the third element (can be indexed as 2\footnote{About indexing, see also: \ref{arrays_at_one}}) of the table is to be used.
It's also possible to click \q{Follow in Dump} $\rightarrow$ 
\q{Memory address} and \olly will show the element addressed by the \JMP instruction. 
That's \TT{0x010B103A}.

\clearpage
After the jump we are at \TT{0x010B103A}: the code printing \q{two} will now be executed:

\begin{figure}[H]
\centering
\myincludegraphics{patterns/08_switch/2_lot/olly4.png}
\caption{\olly: now we at the \IT{case:} label}
\label{fig:switch_lot_olly4}
\end{figure}


