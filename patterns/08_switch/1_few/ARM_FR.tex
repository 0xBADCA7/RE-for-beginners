\subsubsection{ARM: \OptimizingKeilVI (\ARMMode)}
\myindex{\CLanguageElements!switch}

\lstinputlisting[style=customasmARM]{patterns/08_switch/1_few/few_ARM_ARM_O3.asm}

Á nouveau, en investiguant ce code, nous ne pouvons pas dire si il y avait un switch()
dans le code source d'origine ou juste un ensemble de déclarations if().

\myindex{ARM!\Instructions!ADRcc}

En tout cas, nous voyons ici des instructions conditionnelles (comme \ADREQ (\IT{Equal}))
qui ne sont exécutées que si $R0=0$, et qui chargent ensuite l'adresse de la chaîne
\IT{<<zero\textbackslash{}n>>} dans \Reg{0}.
\myindex{ARM!\Instructions!BEQ}
L'instruction suivante \ac{BEQ} redirige le flux d'exécution en \TT{loc\_170}, si $R0=0$.

Le lecteur attentif peut se demander si \ac{BEQ} s'exécute correctement puisque \ADREQ
a déjà mis une autre valeur dans le registre \Reg{0}.

Oui, elle s'exécutera correctement, car \ac{BEQ} vérifie les flags mis par l'instruction
\CMP et \ADREQ ne modifie aucun flag.

Les instructions restantes nous sont déjà familières.
Il y a seulement un appel à \printf, à la fin, et nous avons déjà examiné cette
astuce ici~(\myref{ARM_B_to_printf}).
A la fin, il y a trois chemins vers \printf{}.

\myindex{ARM!\Instructions!ADRcc}
\myindex{ARM!\Instructions!CMP}
La dernière instruction, \TT{CMP R0, \#2}, est nécessaire pour vérifier si $a=2$.

Si ce n'est pas vrai, alors \ADRNE charge un pointeur sur la chaîne \IT{<<something unknown \textbackslash{}n>>}
dans \Reg{0}, puisque $a$ a déjà été comparée pour savoir s'elle est égale
à 0 ou 1, et nous sommes sûrs que la variable $a$ n'est pas égale à l'un de
ces nombres, à ce point.
Et si $R0=2$, un pointeur sur la chaîne \IT{<<two\textbackslash{}n>>} sera chargé
par \ADREQ dans \Reg{0}.

\subsubsection{ARM: \OptimizingKeilVI (\ThumbMode)}

\lstinputlisting[style=customasmARM]{patterns/08_switch/1_few/few_ARM_thumb_O3.asm}

% FIXME а каким можно? к каким нельзя? \myref{} ->

Comme il y déjà été dit, il n'est pas possible d'ajouter un prédicat conditionnel
à la plupart des instructions en mode Thumb, donc ce dernier est quelque peu similaire
au code \ac{CISC}-style x86, facilement compréhensible.

\subsubsection{ARM64: GCC (Linaro) 4.9 \NonOptimizing}

\lstinputlisting[style=customasmARM]{patterns/08_switch/1_few/ARM64_GCC_O0_FR.lst}

Le type de la valeur d'entrée est \Tint, par conséquent le registre \RegW{0} est
utilisé pour garder la veleur au lieu du registre complet \RegX{0}.

Les pointeurs de chaîne sont passés à \puts en utilisant la paire d'instructions
\INS{ADRP}/\INS{ADD} comme expliqué dans l'exemple \q{\HelloWorldSectionName}:~\myref{pointers_ADRP_and_ADD}.

\subsubsection{ARM64: GCC (Linaro) 4.9 \Optimizing}

\lstinputlisting[style=customasmARM]{patterns/08_switch/1_few/ARM64_GCC_O3_FR.lst}

Morceau de code mieux optimisé.
L'instruction \TT{CBZ} (\IT{Compare and Branch on Zero} comparer et sauter si zéro)
effectue un saut si \RegW{0} vaut zéro.
Il y a alors un saut direct à \puts au lieu de l'appeler, comme cela a été expliqué
avant:~\myref{JMP_instead_of_RET}.
