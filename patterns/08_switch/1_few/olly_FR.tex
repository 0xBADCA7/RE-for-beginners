\clearpage
\mysubparagraph{\olly}

Comme cet exemple est compliqué, traçons-le dans \olly.

\olly peut détecter des constructions avec switch(), et ajoute des commentaires utiles.
\EAX contient 2 au début, c'est la valeur du paramètre de la fonction:

\begin{figure}[H]
\centering
\myincludegraphics{patterns/08_switch/1_few/olly1.png}
\caption{\olly: \EAX 
contient maintenant le premier (et unique) argument de la fonction}
\label{fig:switch_few_olly1}
\end{figure}

\clearpage
0 est soustrait de 2 dans \EAX.
Bien sûr, \EAX contient toujours 2.
Mais le flag \ZF est maintenant à 0, indiquant que le résultat est différent de
zéro:

\begin{figure}[H]
\centering
\myincludegraphics{patterns/08_switch/1_few/olly2.png}
\caption{\olly: \SUB exécuté}
\label{fig:switch_few_olly2}
\end{figure}

\clearpage
\DEC est exécuté et \EAX contient maintenant 1.
Mais 1 est différent de zéro, donc le flag \ZF est toujours à 0:

\begin{figure}[H]
\centering
\myincludegraphics{patterns/08_switch/1_few/olly3.png}
\caption{\olly: premier \DEC exécuté}
\label{fig:switch_few_olly3}
\end{figure}

\clearpage
Le \DEC suivant est exécuté.
\EAX contient maintenant 0 et le flag \ZF est mis, car le résultat devient zéro:

\begin{figure}[H]
\centering
\myincludegraphics{patterns/08_switch/1_few/olly4.png}
\caption{\olly: second \DEC exécuté}
\label{fig:switch_few_olly4}
\end{figure}

\olly montre que le saut va être effectué (\IT{Jump is taken}).

\clearpage
Un pointeur sur la chaîne \q{two} est maintenant écrit sur la pile:

\begin{figure}[H]
\centering
\myincludegraphics{patterns/08_switch/1_few/olly5.png}
\caption{\olly: 
pointeur sur la chaîne qui va être écrite à la place du premier argument}
\label{fig:switch_few_olly5}
\end{figure}

% TODO: homogenize numbers
% now they are inconsistent: sometimes plain text, sometimes in math mode
% some kind of \expr{} both for numbers and expressions? --DY
Veuillez noter: l'argument de la fonction courante est 2, et 2 est maintenant
sur la pile, à l'adresse \TT{0x001EF850}.

\clearpage
\MOV écrit le pointeur sur la chaîne à l'adresse \TT{0x001EF850} (voir la fenêtre de la pile).
Puis, le saut est effectué.
Ceci est la première instruction de la fonction \printf dans MSVCR100.DLL (Cet exemple
a été compilé avec le switch /MD):

\begin{figure}[H]
\centering
\myincludegraphics{patterns/08_switch/1_few/olly6.png}
\caption{\olly: première instruction de \printf dans MSVCR100.DLL}
\label{fig:switch_few_olly6}
\end{figure}

Maintenant \printf traite la chaîne à l'adresse \TT{0x00FF3010} comme c'est son
seul argument et l'affiche.

\clearpage
Ceci est la dernière instruction de \printf:

\begin{figure}[H]
\centering
\myincludegraphics{patterns/08_switch/1_few/olly7.png}
\caption{\olly: dernière instruction de \printf dans MSVCR100.DLL}
\label{fig:switch_few_olly7}
\end{figure}

La chaîne \q{two} vient juste d'être affichée dans la fenêtre console.

\clearpage
Maintenant, appuyez sur F7 ou F8 (\stepover) et le retour ne se fait pas sur \ttf,
mais sur \main:

\begin{figure}[H]
\centering
\myincludegraphics{patterns/08_switch/1_few/olly8.png}
\caption{\olly: retourne à \main}
\label{fig:switch_few_olly8}
\end{figure}

Oui, le saut a été direct, depuis les entrailles de \printf vers \main.
Car \ac{RA} dans la pile pointe non pas quelque part dans \ttf, mais en fait sur
\main.
Et \CALL \TT{0x00FF1000} a été l'instruction qui a appelé \ttf.
