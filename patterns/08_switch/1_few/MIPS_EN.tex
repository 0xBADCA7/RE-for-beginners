\subsubsection{MIPS}

\lstinputlisting[caption=\Optimizing GCC 4.4.5 (IDA),style=customasm]{patterns/08_switch/1_few/MIPS_O3_IDA_EN.lst}

\myindex{MIPS!\Instructions!JR}

The function always ends with calling \puts, so here we see a jump to \puts (\INS{JR}: \q{Jump Register}) instead of \q{jump and link}.
We talked about this earlier: \myref{JMP_instead_of_RET}.

\myindex{MIPS!Load delay slot}
We also often see \INS{NOP} instructions after \INS{LW} ones.
This is \q{load delay slot}: another \IT{delay slot} in MIPS.
\myindex{MIPS!\Instructions!LW}

An instruction next to \INS{LW} may execute at the moment while \INS{LW} loads value from memory. 

However, the next instruction must not use the result of \INS{LW}.

Modern MIPS CPUs have a feature to wait if the next instruction uses result of \INS{LW}, so this is somewhat outdated,
but GCC still adds NOPs for older MIPS CPUs.
In general, it can be ignored.
