\subsubsection{x86}

\myparagraph{\NonOptimizing MSVC}

Risultato (MSVC 2010):

\lstinputlisting[caption=MSVC 2010,style=customasmx86]{patterns/08_switch/1_few/few_msvc.asm}

La nostra funzione con pochi casi nello switch() è praticamente analaco a questa costruzione:

\lstinputlisting[label=switch_few_ifelse,style=customc]{patterns/08_switch/1_few/few_analogue.c}

\myindex{\CLanguageElements!switch}
\myindex{\CLanguageElements!if}

Nel caso di switch() con un piccolo numero di casi, è impossibile essere sicuri che nel sorgente originale ci fosse
veramente la funzione switch() o soltanto una serie di costrutti if().


\myindex{\SyntacticSugar}

Ciò vuol dire che switch() si comporta come "zucchero sintattico", equivalente ad un alto numero di if() annidati.

Dal nostro punto di vista non c'è niente di particolarmente nuovo nel codice generato,
con l'eccezione dello spostamento (da parte del compilatore) della variabile in input $a$ in una variabile locale temporanea \TT{tv64}
\footnote{Le variabili locali nello stack hanno il prefisso \TT{tv}--- MSVC nomina così le varibili locali per i suoi scopi}.

Se compiliamo questo codice con 4.4.1 otteniamo pressoché lo stesso risultato, anche con il maggior livello di ottimizzazione (\Othree option).

\myparagraph{\Optimizing MSVC}

% TODO separate various kinds of \TT
% idea: enclose command lines in a specific environment, like \cmdline{} 
% assembly instructions in \asm{} (now both \TT and \q{} are used),
% variables in,  like \var{}
% messages (string constants) in something else, like \strconst
% to separate them all. Now they all use \TT, which is not best
% \INS{} for all instructions including operands? --DY
Now let's turn on optimization in MSVC (\Ox): \TT{cl 1.c /Fa1.asm /Ox}

\label{JMP_instead_of_RET}
\lstinputlisting[caption=MSVC,style=customasmx86]{patterns/08_switch/1_few/few_msvc_Ox.asm}

Qui possiamo vedere un po' di trucchetti.

\myindex{x86!\Instructions!JZ}
\myindex{x86!\Instructions!JE}
\myindex{x86!\Instructions!SUB}

Primo: il valore di $a$ è messo in \EAX, e gli viene sottratto 0. Sembra assurdo, ma è fatto per controllare se il valore 
in \EAX è 0. Se si, il flag \ZF viene settato (i.e. sottrarre 0 a 0 da 0)
e il primo jump condizionale \JE (\IT{Jump if Equal} o il sinonimo \JZ~---\IT{Jump if Zero}) è innescato e il controllo del flusso 
passato alla label \TT{\$LN4@f}, dove il messaggio \TT{'zero'} viene stampato. 
Se il primo jump non viene innescato, 1 viene sottratto dal valore in input e se ad un certo punto il risultato è 0, il jump corrispondente
viene innescato.

Se nessun jump viene innescato, il controllo del flusso passa a \printf con l'argomento \\
\TT{'something unknown'}.

\label{jump_to_last_printf}
\myindex{\Stack}

Secondo: notiamo qualcosa di inusuale per noi: un puntatore a stringa viene messo nella variabile $a$, e 
successivamente viene chiamata \printf non tramite \CALL, ma via \JMP. C'è una spiegazione semplice dietro ciò: 
il \gls{caller} mette un valore sullo stack e chiama la nostra funzione tramite \CALL. 
La stessa \CALL fa il push del return address (\ac{RA}) sullo stack e fa un salto non condizionale all'indirizzo della nostra funzione. 
La nostra funzione, in qualunque punto della sua esecuzione (poiché non contiene istruzioni che spostano lo stack pointer)
ha il seguente layout dello stack:

\begin{itemize}
\item\ESP---punta a \ac{RA}
\item\TT{ESP+4}---punta alla variabile $a$ 
\end{itemize}

Dall'altro lato, quando dobbiamo chiamare \printf qui abbiamo bisogno esattamente dello stesso layout dello stack,
eccetto per il primo argomento di \printf, che deve puntare alla stringa. 
E questo è ciò che fa il codice.

Sostituisce il primo argomento della funzione con l'indirizzo della stringa, e salta a \printf, come se non avessimo chiamato
la nostra funzione \ttf, ma direttamente \printf.
\printf stampa una stringa su \gls{stdout} e successivamente esegue l'istruzione \RET , che fa il POP del 
\ac{RA} dallo stack, e il controllo del flusso viene restituito non a \ttf ma al \gls{caller} di \ttf, 
bypassando la fine della funzione \ttf.

\myindex{\CStandardLibrary!longjmp()}
\newcommand{\URLSJ}{\href{http://go.yurichev.com/17121}{wikipedia}}

% TODO \myref{}
Tutto ciò è possibile perchè \printf è chiamata proprio alla fine della funzione \ttf in tutti i casi.
In qualche modo è simile alla funzione \TT{longjmp()}\footnote{\URLSJ} function.
E, ovviamente, tutto ciò viene fatto a favore della velocità di esecuzione.

Un simile caso con il compilatore ARM è descritto nella sezione \q{\PrintfSeveralArgumentsSectionName}, qui~(\myref{ARM_B_to_printf}).

\clearpage
\mysubparagraph{\olly}
\myindex{\olly}

Let's try this example in \olly.
The input value of the function (2) is loaded into \EAX: 

\begin{figure}[H]
\centering
\myincludegraphics{patterns/08_switch/2_lot/olly1.png}
\caption{\olly: function's input value is loaded in \EAX}
\label{fig:switch_lot_olly1}
\end{figure}

\clearpage
The input value is checked, is it bigger than 4? 
If not, the \q{default} jump is not taken:
\begin{figure}[H]
\centering
\myincludegraphics{patterns/08_switch/2_lot/olly2.png}
\caption{\olly: 2 is no bigger than 4: no jump is taken}
\label{fig:switch_lot_olly2}
\end{figure}

\clearpage
Here we see a jumptable:

\begin{figure}[H]
\centering
\myincludegraphics{patterns/08_switch/2_lot/olly3.png}
\caption{\olly: calculating destination address using jumptable}
\label{fig:switch_lot_olly3}
\end{figure}

Here we've clicked \q{Follow in Dump} $\rightarrow$ \q{Address constant}, so now we see the \IT{jumptable} in the data window.
These are 5 32-bit values\footnote{They are underlined by \olly because
these are also FIXUPs: \myref{subsec:relocs}, we are going to come back to them later}.
\ECX is now 2, so the third element (can be indexed as 2\footnote{About indexing, see also: \ref{arrays_at_one}}) of the table is to be used.
It's also possible to click \q{Follow in Dump} $\rightarrow$ 
\q{Memory address} and \olly will show the element addressed by the \JMP instruction. 
That's \TT{0x010B103A}.

\clearpage
After the jump we are at \TT{0x010B103A}: the code printing \q{two} will now be executed:

\begin{figure}[H]
\centering
\myincludegraphics{patterns/08_switch/2_lot/olly4.png}
\caption{\olly: now we at the \IT{case:} label}
\label{fig:switch_lot_olly4}
\end{figure}


