\subsubsection{x86}

\myparagraph{\NonOptimizing MSVC}

Ergebnis (MSVC 2010):

\lstinputlisting[caption=MSVC 2010,style=customasmx86]{patterns/08_switch/1_few/few_msvc.asm}
Unsere Funktionen mit ein paar Fällen in switch() ist analog zu dieser Konstruktion:

\lstinputlisting[label=switch_few_ifelse,style=customc]{patterns/08_switch/1_few/few_analogue.c}

\myindex{\CLanguageElements!switch}
\myindex{\CLanguageElements!if}
Wenn wir mit einem switch() mit einigen wenigen Fällen arbeiten, ist es unmöglich sicher zu sein, dass es sich
tatsächlich im Quellcode um ein switch() handelt und nicht um eine Reihe von if()-Anweisungen.

\myindex{\SyntacticSugar}
Das bedeutet, dass switch() ein syntaktischer Zucker für eine große Anzahl von verschachtelten if()-Anweisungen ist.

Hier ist nichts Neues für uns im erzeugten Code, mit der Ausnahme, dass der Compiler die Eingabevariable $a$ in einer
temporären Variable \TT{tv64} speichert\footnote{Lokale Variablen im Stack haben den Präfix \TT{tv}---so benennt MSVC
interne Variablen für seine Zwecke}.

Wenn wir diesen Code mit GCC 4.4.1 kompilieren, erhalten wir fast das gleiche Ergebnis, sogar unter Verwendung maximaler
Optimierung (Option \Othree).

\myparagraph{\Optimizing MSVC}

% TODO separate various kinds of \TT
% idea: enclose command lines in a specific environment, like \cmdline{} 
% assembly instructions in \asm{} (now both \TT and \q{} are used),
% variables in,  like \var{}
% messages (string constants) in something else, like \strconst
% to separate them all. Now they all use \TT, which is not best
% \INS{} for all instructions including operands? --DY
Aktivieren wir nun Optimierung in MSVC (\Ox): \TT{cl 1.c /Fa1.asm /Ox}

\label{JMP_instead_of_RET}
\lstinputlisting[caption=MSVC,style=customasmx86]{patterns/08_switch/1_few/few_msvc_Ox.asm}
Hier finden wir ein paar schmutzige Tricks.

\myindex{x86!\Instructions!JZ}
\myindex{x86!\Instructions!JE}
\myindex{x86!\Instructions!SUB}
Zunächst: der Wert von $a$ wird in \EAX abgelegt und 0 wird davon abgezogen. Klingt absurd, muss aber getan werden, um
zu prüfen, ob der Wert in \EAX null ist. Falls ja, wird das \ZF Flag gesetzt (denn 0 minus 0 ist 0) und der erste
bedingte Sprung \JE (\IT{Jump if Equal}) oder synonym \JZ~---\IT{Jump if Zero}) wird ausgeführt und der Control Flow
wird an das Label \TT{\$LN4@f} übergeben, an dem die Nachricht \TT{'zero'} ausgegeben wird.
Wenn der erste Sprung nicht ausgeführt wird, wird 1 vom Eingabewert abgezogen und wenn an irgendeinem Punkt der
Ausführung ein Ergebnis von 0 auftritt, wird der zugehörige Sprung ausgeführt. 

Falls aber keiner der Sprünge ausgeführt wird, wird der Control Flow mit dem String Argument \TT{'something unknown'}
an \printf übergeben.

\label{jump_to_last_printf}
\myindex{\Stack}
Zweitens: wir sehen etwas für uns Ungewohntes: ein Pointer auf einen String wird in der Variablen $a$ abgelegt und
anschließend wird \printf nicht über \CALL, sondern per \JMP aufgerufen. Es gibt hierfür eine einfache Erklärung:
Der \glslink{caller}{Aufrufer} legt einen Wert auf dem Stack ab und ruft unsere Funktion über \CALL auf.
\CALL selbst legt die Rücksprungadresse (\ac{RA}) auf dem Stack ab und macht einen unbedingten Sprung zur Adresse
unserer Funktion. Unsere Funktion hat an jedem Punkt der Ausführung (da sie keine Befehle enthält, die den Stackpointer
verändern) das folgende Stacklayout:

\begin{itemize}
\item\ESP---zeigt auf \ac{RA}
\item\TT{ESP+4}---zeigt auf die Variable $a$ 
\end{itemize}
Wenn wir andererseits \printf aufrufen benötigen wir hier exakt das gleiche Stacklayout mit dem Unterschied des ersten
Arguments von \printf, welches auf den auszugebenden String zeigt. Unser Code tut hier das Folgende:

Er ersetzt das erste Argument der Funktion mit der Adresse (d.i. dem Pointer) auf den String und springt zu \printf als
ob wir nicht unsere Funktion \ttf, sondern direkt \printf aufrufen würden. Die Funktion \printf gibt einen String auf
\gls{stdout} aus und führt dann den \RET Befehl aus, der die \ac{RA} vom Stack holt. Der Control Flow wird nicht an \ttf
übergeben, sondern an den Aufrufer von \ttf, womit effektiv die Funktion \ttf umgangen wird.

\myindex{\CStandardLibrary!longjmp()}
\newcommand{\URLSJ}{\href{http://go.yurichev.com/17121}{wikipedia}}

% TODO \myref{}
All dies ist möglich, da \printf in allen Fällen ganz am Ende der Funktion \ttf aufgerufen wird. In gewisser Weise
besteht Ähnlichkeit zur Funktion \TT{longjmp()}\footnote{\URLSJ}. Natürlich geschieht all dies um die
Ausführungsgeschwindigkeit zu erhöhen.

Ein ähnlicher Fall mit dem ARM Compiler wird im Abschnitt \q{\PrintfSeveralArgumentsSectionName}
beschrieben:~(\myref{ARM_B_to_printf}).


\clearpage
\mysubparagraph{\olly}
Da dieses Beispiel komliziert ist, untersuchen wir es mit \olly.

\olly kann solche switch() Konstruktionen erkennen und einige nützliche Kommentare hinzufügen.
\EAX ist zu Beginn auf 2 gesetzt, das entspricht dem Eingabewert der Funktion:


\begin{figure}[H]
\centering
\myincludegraphics{patterns/08_switch/1_few/olly1.png}
\caption{\olly: \EAX 
enthält jetzt das erste (und einzige) Functionsargument}
\label{fig:switch_few_olly1}
\end{figure}

\clearpage
0 wird in vor der 2 in \EAX abgezogen. Natürlich enthält \EAX danach immernoch den Wert 2, aber das \ZF Flag ist jetzt
0, da das Ergebnis der letzten Berechnung nicht 0 ergeben hat.
\begin{figure}[H]
\centering
\myincludegraphics{patterns/08_switch/1_few/olly2.png}
\caption{\olly: \SUB wurde ausgeführt}
\label{fig:switch_few_olly2}
\end{figure}

\clearpage
\DEC wird ausgeführt und \EAX enthält nun 1.
Da 1 ein von null verschiedener Wert ist, ist das \ZF Flag immer noch 0:

\begin{figure}[H]
\centering
\myincludegraphics{patterns/08_switch/1_few/olly3.png}
\caption{\olly: erstes \DEC wurde ausgeführt}
\label{fig:switch_few_olly3}
\end{figure}

\clearpage
Das nächste \DEC wird ausgeführt.
\EAX enthält jetzt 0 und das \ZF Flag wird gesetzt, da eine Berechnung 0 ergeben hat.

\begin{figure}[H]
\centering
\myincludegraphics{patterns/08_switch/1_few/olly4.png}
\caption{\olly: zweites \DEC wurde ausgeführt}
\label{fig:switch_few_olly4}
\end{figure}
\olly zeigt an, dass dieser Sprung jetzt getätigt wird.

\clearpage
Ein Pointer auf den String \q{two} wird jetzt auf den Stack geschrieben:

\begin{figure}[H]
\centering
\myincludegraphics{patterns/08_switch/1_few/olly5.png}
\caption{\olly: 
Pointer auf den String wird an die Stelle des ersten Arguments geschrieben}
\label{fig:switch_few_olly5}
\end{figure}

% TODO: homogenize numbers
% now they are inconsistent: sometimes plain text, sometimes in math mode
% some kind of \expr{} both for numbers and expressions? --DY
Man beachte: das aktuelle Argument der Funktion ist 2 und diese 2 befindet sich im Stack nun an der Adresse
\TT{0x001EF850}.

\clearpage
\MOV schreibt den Pointer auf den String an die Adresse \TT{0x001EF850} (siehe Stackfenster).
Dann wird der Sprung ausgeführt. Dies ist der erste Befehl der Funktion \printf in MSVCR100.DLL. (Dieses Beispiel wurde
mit der Option /MD kompiliert.)
 

\begin{figure}[H]
\centering
\myincludegraphics{patterns/08_switch/1_few/olly6.png}
\caption{\olly: erster Befehl von \printf in MSVCR100.DLL}
\label{fig:switch_few_olly6}
\end{figure}

Die Funktion \printf behandelt den String an der Adresse \TT{0x00FF3010} als (einziges) Argument und gibt ihn aus.

\clearpage
Dies ist der letzte Befehl von \printf:

\begin{figure}[H]
\centering
\myincludegraphics{patterns/08_switch/1_few/olly7.png}
\caption{\olly: letzter Befehl von \printf in MSVCR100.DLL}
\label{fig:switch_few_olly7}
\end{figure}

Der String \q{two} wurde gerade auf der Konsole ausgegeben.

\clearpage
Wir drücken nun F7 oder F8 (\stepover) und kehren\dots nicht zu \ttf , sondern zu \main zurück:

\begin{figure}[H]
\centering
\myincludegraphics{patterns/08_switch/1_few/olly8.png}
\caption{\olly: zurück zu \main}
\label{fig:switch_few_olly8}
\end{figure}
Dieser Sprung wird direkt von \printf zu \main durchgeführt, da \ac{RA} im Stack nicht auf eine Stelle in \ttf, sondern
auf \main zeigt.
Der Befehl \CALL \TT{0x00FF1000} war der eigentliche Befehl, der \ttf aufgerufen hat.



