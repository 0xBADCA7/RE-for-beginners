\section{64 бита}

\subsection{x86-64}
\myindex{x86-64}
\label{x86-64}

Это расширение x86-архитектуры до 64 бит.

С точки зрения начинающего reverse engineer-а, наиболее важные отличия от 32-битного x86 это:

\myindex{\CLanguageElements!\Pointers}
\begin{itemize}

\item
Почти все регистры (кроме FPU и SIMD) расширены до 64-бит и получили префикс R-. 
И еще 8 регистров добавлено. 
В итоге имеются эти \ac{GPR}-ы:
 \RAX, \RBX, \RCX, \RDX, 
\RBP, \RSP, \RSI, \RDI, \Reg{8}, \Reg{9}, \Reg{10}, 
\Reg{11}, \Reg{12}, \Reg{13}, \Reg{14}, \Reg{15}. 

К ним также можно обращаться так же, как и прежде. Например, для доступа к младшим 32 битам \RAX 
можно использовать \EAX:

\RegTableOne{RAX}{EAX}{AX}{AH}{AL}

У новых регистров \GTT{R8-R15} также имеются их \IT{младшие части}: \GTT{R8D-R15D} 
(младшие 32-битные части), 
\GTT{R8W-R15W} (младшие 16-битные части), \GTT{R8L-R15L} (младшие 8-битные части).

\RegTableFour{R8}{R8D}{R8W}{R8L}

Удвоено количество SIMD-регистров: с 8 до 16: \XMM{0}-\XMM{15}.

\item
В win64 передача всех параметров немного иная, это немного похоже на fastcall 
(\myref{fastcall}).
Первые 4 аргумента записываются в регистры \RCX, \RDX, \Reg{8}, \Reg{9}, а остальные ~--- в стек. 
Вызывающая функция также должна подготовить место из 32 байт чтобы вызываемая функция могла сохранить 
там первые 4 аргумента и использовать эти регистры по своему усмотрению. 
Короткие функции могут использовать аргументы прямо из регистров, но б\'{о}льшие функции могут сохранять 
их значения на будущее.

Соглашение System V AMD64 ABI (Linux, *BSD, \MacOSX)\SysVABI также напоминает
fastcall, использует 6 регистров 
\RDI, \RSI, \RDX, \RCX, \Reg{8}, \Reg{9} для первых шести аргументов.
Остальные передаются через стек.

См. также в соответствующем разделе о способах передачи аргументов через стек ~(\myref{sec:callingconventions}).

\item
\Tint в \CCpp остается 32-битным для совместимости.

\item
Все указатели теперь 64-битные.

\end{itemize}

\myindex{Register allocation}
Из-за того, что регистров общего пользования теперь вдвое больше, у компиляторов теперь больше 
свободного места для маневра, называемого \glslink{register allocator}{register allocation}.
Для нас это означает, что в итоговом коде будет меньше локальных переменных.

\myindex{DES}
Для примера, функция вычисляющая первый S-бокс алгоритма шифрования DES, 
она обрабатывает сразу 32/64/128/256 значений, в зависимости от типа \GTT{DES\_type} (uint32, uint64, SSE2 или AVX), 
методом bitslice DES (больше об этом методе читайте здесь~(\myref{bitslicedes})):

\lstinputlisting[style=customc]{patterns/20_x64/19_1.c}

Здесь много локальных переменных. Конечно, далеко не все они будут в локальном стеке. 
Компилируем обычным MSVC 2008 с опцией \GTT{/Ox}:

\lstinputlisting[caption=\Optimizing MSVC 2008,style=customasmx86]{patterns/20_x64/19_2_msvc_Ox.asm}

5 переменных компилятору пришлось разместить в локальном стеке.

Теперь попробуем то же самое только в 64-битной версии MSVC 2008:

\lstinputlisting[caption=\Optimizing MSVC 2008,style=customasmx86]{patterns/20_x64/19_3_msvc_x64.asm}

Компилятор ничего не выделил в локальном стеке, а \GTT{x36} это синоним для \GTT{a5}.

\iffalse
% FIXME1 невнятно
Кстати, видно, что функция сохраняет регистры \RCX, \RDX в отведенных для 
этого вызываемой функцией местах, 
а \Reg{8} и \Reg{9} не сохраняет, а начинает использовать их сразу.

\fi

Кстати, существуют процессоры с еще большим количеством \ac{GPR}, например, 
Itanium ~--- 128 регистров.

\subsection{ARM}

64-битные инструкции появились в ARMv8.

\subsection{Числа с плавающей запятой}

О том как происходит работа с числами с плавающей запятой в x86-64, читайте здесь: \myref{floating_SIMD}.

\subsection{Критика 64-битной архитектуры}

Некоторые люди иногда сетуют на то что указатели теперь 64-битные:
ведь теперь для хранения всех указателей нужно в 2 раза больше места 
в памяти, в т.ч. и в кэш-памяти, не смотря на то что x64-процессоры могут адресовать только 48 бит внешней \ac{RAM}.

Некоторые люди делают свои аллокаторы памяти.
\myindex{CryptoMiniSat}
Интересен случай с CryptoMiniSat\footnote{\url{https://github.com/msoos/cryptominisat/}}.
Эта программа довольно редко использует более 4GiB памяти, но она очень активно использует указатели.
Так что, на 32-битной платформе она требовала меньше памяти, чем на 64-битной.
Чтобы справиться с этой проблемой, автор создал свой аллокатор (в файлах \IT{clauseallocator.(h|cpp)}),
который позволяет иметь доступ к выделенной памяти используя 32-битные идентификаторы вместо 64-битных указателей.

