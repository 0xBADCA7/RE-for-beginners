\clearpage
\myparagraph{\olly et les champs alignés par défaut}
\myindex{\olly}

Examinons dans \olly notre exemple lorsque les champs sont alignés par défaut sur des frontières de 4 octets:

\begin{figure}[H]
\centering
\myincludegraphics{patterns/15_structs/4_packing/olly_packing_4.png}
\caption{\olly: Before \printf execution}
\label{fig:packing_olly_4}
\end{figure}

Nous voyons nos quatre champs dans la fénêtre de données.

Mais d'ou viennent ces octets aléatoires (0x30, 0x37, 0x01) situé à côté des premier (a) et troisième (c)
champs ?

Si nous revenons à notre listing \myref{src:struct_packing_4}, nous constatons que ces deux champs sont de
type \Tchar. Seul un octet est écrit pour chacun d'eux: 1 et 3 respectivement (lignes 6 et 8).

Les trois autres octets des deux mots de 32 bits ne sont pas modifiés en mémoire! Des débris aléatoires des 
précédentes opérations demeurent donc là.

\myindex{x86!\Instructions!MOVSX}

Ces débris n'influencent nullement le résultat de la fonction \printf parce que les valeurs qui lui sont 
passées sont préparés avec l'instruction \MOVSX qui opère sur des octets et non pas sur des mots: 
\lstref{src:struct_packing_4} (lignes 34 et 38).

L'instruction \MOVSX (extension de signe) est utilisée ici car le type \Tchar est par défaut une valeur 
signée pour MSVC et GCC. Si l'un des types \TT{unsigned char} ou \TT{uint8\_t} était utilisé ici, ce serait 
l'instruction \MOVZX que le compilateur aurait choisi.

\clearpage
\myparagraph{\olly et les champs alignés sur des frontières de 1 octet}
\myindex{\olly}

Les choses sont beaucoup plus simples ici. Les 4 champs occupent 10 octets et les valeurs sont stockées 
côte-à-côte.

\begin{figure}[H]
\centering
\myincludegraphics{patterns/15_structs/4_packing/olly_packing_1.png}
\caption{\olly: Avant appel de la fonction \printf}
\label{fig:packing_olly_1}
\end{figure}
