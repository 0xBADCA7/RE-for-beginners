\subsubsection{MIPS}
\label{MIPS_structure_big_endian}

\lstinputlisting[caption=\Optimizing GCC 4.4.5 (IDA),numbers=left,style=customasmMIPS]{patterns/15_structs/4_packing/MIPS_O3_IDA_FR.lst}

Les champs de la structure sont fournis dans les registres \$A0..\$A3 puis transformé dans les registres 
\$A1..\$A3 pour l'utilisation par \printf, tandis que le 4ème champ (provenant de \$A3) est passé sur la 
pile en utilisant l'instruction \INS{SW}.

Mais à quoi servent ces deux instructions SRA (\q{Shift Word Right Arithmetic}) lors de la préparation des 
champs \Tchar?

MIPS est une architecture grand-boutien (big-endian) par défaut \myref{sec:endianness}, de même que la 
distribution Debian Linux que nous utilisons.

En conséquence, lorsqu'un octet est stocké dans un emplacement 32bits d'une structure, ils occupent les 
bits 31..24 bits.

Quand une variable \Tchar doit être étendue en une valeur sur 32 bits, elle doit tout d'abord être décalée 
vers la droite de 24 bits.

\Tchar étant un type signé, un décalage arithmétique est utilisé ici, à la place d'un décalage logique.
