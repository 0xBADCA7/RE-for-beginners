\subsubsection{MIPS}
\label{MIPS_structure_big_endian}

\lstinputlisting[caption=\Optimizing GCC 4.4.5 (IDA),numbers=left,style=customasmMIPS]{patterns/15_structs/4_packing/MIPS_O3_IDA_FR.lst}

Les champs de la structure sont fournis dans les registres \$A0..\$A3 puis transform� dans les registres 
\$A1..\$A3 pour l'utilisation par \printf, tandis que le 4�me champ (provenant de \$A3) est pass� sur la 
pile en utilisant l'instruction \INS{SW}.

Mais � quoi servent ces deux instructions SRA (\q{Shift Word Right Arithmetic}) lors de la pr�paration des 
champs \Tchar?

MIPS est une architecture grand-boutien (big-endian) par d�faut \myref{sec:endianness}, de m�me que la 
distribution Debian Linux que nous utilisons.

En cons�quence, lorsqu'un octet est stock� dans un emplacement 32bits d'une structure, ils occupent les 
bits 31..24 bits.

Quand une variable \Tchar doit �tre �tendue en une valeur sur 32 bits, elle doit tout d'abord �tre d�cal�e 
vers la droite de 24 bits.

\Tchar �tant un type sign�, un d�calage arithm�tique est utilis� ici, � la place d'un d�calage logique.
