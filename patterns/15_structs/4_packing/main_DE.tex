\subsection{\StructurePackingSectionName}
\label{structure_packing}
Ein wichtiges Thema ist das Packen von Feldern in structsfootnote{Siehe auch: \URLWPDA}.

Betrachten wir ein einfaches Beispiel:

\lstinputlisting[style=customc]{patterns/15_structs/4_packing/packing.c}
Wie wir sehen haben wir zwei \Tchar Felder (jedes ist exakt ein Byte groß) und zwei weitere vom Typ \Tint (zu je 4
Byte).

% subsections:
\subsubsection{x86}

Das Beispiel kompiliert zu folgendem Code:

\lstinputlisting[caption=MSVC 2012 /GS-
/Ob0,label=src:struct_packing_4,numbers=left,style=customasmx86]{patterns/15_structs/4_packing/packing_DE.asm}
Wir übergeben das struct als Ganzes, aber im Code können wir sehen, dass das struct in ein temporäres struct kopiert
wird (ein Platz hierfür wird in Zeile 10 auf dem Stack reserviert), und dass dann alle 4 Felder einzeln (in den Zeilen
12\ldots 19) kopiert werden und anschließend ihr Pointer (Adresse) übergeben wird.

Das struct wird kopiert, da nicht bekannt ist, ob die Funktion \ttf{} das struct verändern wird oder nicht.
Wenn es verändert wird, muss das struct in \main auf dem vorherigen Stand bleiben.

Wir könnten \CCpp Pointer verwenden und der erzeugte Code wäre fast der gleiche nur ohne das Kopieren.

Wie wir sehen können, wird die Adresse von jedem Feld auf einer 4 Byte Grenze angeordnet. Das ist der Grund dafür, dass
jeder \Tchar hier 4 Byte belegt (wie ein \Tint). Es ist für die CPU einfacher auf Speicher an entsprechend angeordneten
Adressen zuzugreifen und Daten von dort im Cache zwischenzuspeichern.

Trotzdem ist dieses Vorgehen nicht besonders ökonomisch.

Komplieren wir mit der Option (\TT{\Zp1})(\IT{/Zp[n] pack structures on n-byte boundary}).

\lstinputlisting[caption=MSVC 2012 /GS-
/Zp1,label=src:struct_packing_1,numbers=left,style=customasmx86]{patterns/15_structs/4_packing/packing_msvc_Zp1_DE.asm}
Das struct benötigt nun lediglich 10 Byte und jeder \Tchar Wert genau 1 Byte. Welchen Vorteil bringt uns das? 
In erster Linie spart man Platz. Ein Nachteil hieran~---die CPU greift hier auf die Felder langsamer zu als es möglich
wäre.

\label{short_struct_copying_using_MOV}
Das struct wird auch in \main kopiert. Nicht Feld für Feld, sondern alle 10 Byte direkt durch Verwendung von drei Paaren
von \MOV Befehlen. Warum aber nicht 4 Paare?

Der Compiler hat entschieden, dass es besser ist, die 10 Byte mit 3 \MOV Befehlspaaren zu kopieren als zwei 32-Bit-Worte
und dann zwei Byte mit insgesamt 4 \MOV Paaren.

Solch eine Implementierung, die zum Kopieren \MOV anstelle eines Aufrufs von \TT{memcpy()} verwendet, ist sehr
gebräuchlich, da es schneller ist als ein Funktionsaufruf von \TT{memcpy()}---zumindest für kurze Blöcke:
\myref{copying_short_blocks}.
Man kann sich leicht überlegen, dass, wenn ein struct in vielen Quellcode und Objektdateien verwendet wird, alle diese
mit der gleichen Konvention bezüglich des Packens in structs kompiliert werden müssen.

\newcommand{\FNURLMSDNZP}{\footnote{\href{http://go.yurichev.com/17067}
{MSDN: Working with Packing Structures}}}
\newcommand{\FNURLGCCPC}{\footnote{\href{http://go.yurichev.com/17068}
{Structure-Packing Pragmas}}}
Neben der Option \TT{Zp} in MSVC, die die Anordnung der Felder von structs festlegt, gibt es auch die
Compileroption \TT{\#pragma pack}, die direkt im Quellcode definiert werden kann.
Sie ist sowohl in MSVC\FNURLMSDNZP als auch in GCC\FNURLGCCPC{} verfügbar.

Gehen wir zurück zum \TT{SYSTEMTIME} struct, das aus 16-Bit-Feldern besteht.
Wie kann unser Compiler wissen, dass diese in 1-Byte-Anordnung gepackt werden müssen?

Die Datei \TT{WinNT.h} enthält dies:

\begin{lstlisting}[caption=WinNT.h,style=customc]
#include "pshpack1.h"
\end{lstlisting}

Und dies:

\begin{lstlisting}[caption=WinNT.h,style=customc]
#include "pshpack4.h"                   // 4 byte packing is the default
\end{lstlisting}

Die Datei PshPack1.h sieht wie folgt aus:

\begin{lstlisting}[caption=PshPack1.h,style=customc]
#if ! (defined(lint) || defined(RC_INVOKED))
#if ( _MSC_VER >= 800 && !defined(_M_I86)) || defined(_PUSHPOP_SUPPORTED)
#pragma warning(disable:4103)
#if !(defined( MIDL_PASS )) || defined( __midl )
#pragma pack(push,1)
#else
#pragma pack(1)
#endif
#else
#pragma pack(1)
#endif
#endif /* ! (defined(lint) || defined(RC_INVOKED)) */
\end{lstlisting}
Dies sagt dem Compiler wie die structs, die hinter \TT{\#pragma pack} definiert werden, gepackt werden müssen.

\clearpage
\myparagraph{\olly + standardmäßig gepackte Felder}
\myindex{\olly}
Betrachten wir unser Beispiel (in dem die Felder standardmäßig auf 4 Byte angeordnet werden) in \olly:

\begin{figure}[H]
\centering
\myincludegraphics{patterns/15_structs/4_packing/olly_packing_4.png}
\caption{\olly: vor der Ausführung von \printf}
\label{fig:packing_olly_4}
\end{figure}
Wir sehen unsere 4 Felder im Datenfenster.

Wir fragen uns aber, woher die Zufallsbytes (0x30, 0x37, 0x01) stammen, die neben dem ersten ($a$) und dritten ($c$)
Feld liegen.

Betrachten wir unser Listing \myref{src:struct_packing_4}, erkennen wir, dass das erste und dritte Feld vom Typ \Tchar
ist, und daher nur ein Byte geschrieben wird, nämlich 1 bzw. 3 (Zeilen 6 und 8).

Die übrigen 3 Byte des 32-Bit-Wortes werden im Speicher nicht verändert!
Deshalb befinden sich hier zufällige Reste.

\myindex{x86!\Instructions!MOVSX}
Diese Reste beeinflussen den Output von \printf in keinster Weise, da die Werte für die Funktion mithilfe von \MOVSX
vorbereitet werden, der Bytes und nicht Worte als Argumente hat: 
\lstref{src:struct_packing_4} (Zeilen 34 und 38).
Der vorzeichenerweiternde Befehl \MOVSX wird hier übrigens verwendet, da \Tchar standardmäßig in MSVC und GCC
vorzeichenbehaftet ist.
Würde hier der Datentyp \TT{unsigned char} oder \TT{uint8\_t} verwendet, würde der Befehl \MOVZX stattdessen verwendet.

\clearpage
\myparagraph{\olly + Felder auf 1 Byte Grenzen angeordnet}
\myindex{\olly}
Hier sind die Dinge viel klarer ersichtlich: 4 Felder benötigen 16 Byte und die Werte werden nebeneinander gespeichert.

\begin{figure}[H]
\centering
\myincludegraphics{patterns/15_structs/4_packing/olly_packing_1.png}
\caption{\olly: Vor der Ausführung von \printf}
\label{fig:packing_olly_1}
\end{figure}


\subsubsection{ARM}

\myparagraph{\OptimizingKeilVI (\ThumbMode)}

\lstinputlisting[caption=\OptimizingKeilVI (\ThumbMode),style=customasmARM]{patterns/15_structs/4_packing/packing_Keil_thumb.asm}
Wir sehen, dass hier ein struct anstelle eines Pointers auf ein struct übergeben wird und da die ersten vier
Funktionsargumente in ARM über Register übergeben werden, werden die Felder des structs mittels \TT{R0-R3} übergeben.

\myindex{ARM!\Instructions!LDRB}
\myindex{x86!\Instructions!MOVSX}
\TT{LDRB} lädt ein Byte aus dem Speicher und erweitert es unter Berücksichtigung des Vorzeichens auf 32 Bit.
Dies entspricht \MOVSX in x86. 
Hier wird der Befehl verwendet, um die Felder $a$ und $a$ des structs zu laden.

\myindex{Funktionsepilog}
Eine weitere Sache, die ins Auge sticht, ist, dass anstelle eines Funktionsepilogs ein Sprung zum Epilog einer anderen
Funktion vorliegt.
Bei der anderen handelt es sich um eine komplett unterschiedliche Funktion, die in keiner Weise mit unserer
zusammenhängt, aber denselben Epilog hat (möglicherweise, da diese ebenfalls 5 lokale Variablen enthält ($5*4=0x14$)).

Außerdem befindet sie sich im Code in der Nähe (man vergleiche die Adressen).

Tatsächlich spielt es auch keine Rolle, welcher Epilog ausgeführt wird, solange dieser wie gewünscht funktioniert.

Offensichtlich hat Keil aus ökonomischen Gründen entschieden, einen Teil der anderen Funktion wiederzuverwenden.

Der Epilog benötigt 4 Bytes---während der Sprung nur 2 verbraucht.

\myparagraph{ARM + \OptimizingXcodeIV (\ThumbTwoMode)}

\lstinputlisting[caption=\OptimizingXcodeIV (\ThumbTwoMode),style=customasmARM]{patterns/15_structs/4_packing/packing_Xcode_thumb.asm}

\myindex{ARM!\Instructions!SXTB}
\myindex{x86!\Instructions!MOVSX}
\TT{SXTB} (\IT{Signed Extend Byte}) ist analog zu \MOVSX in x86.
Der ganze Rest ist identisch.


\subsubsection{MIPS}
\label{MIPS_structure_big_endian}

\lstinputlisting[caption=\Optimizing GCC 4.4.5 (IDA),numbers=left,style=customasmMIPS]{patterns/15_structs/4_packing/MIPS_O3_IDA.lst}
Die Felder des structs landen in den Registern \$A0..\$A3 und werden dann für \printf nach \$A1..\$A3 verschoben,
während das vierte Feld (aus \$A3) mit \INS{SW} über den lokalen Stack übergeben wird.

Wir fragen uns aber, warum es hier zwei SRA (\q{Shift Word Right Arithmetic}) Befehle für die beiden \Tchar Felder gibt.

MIPS ist eine Big-Endian-Architektur\myref{sec:endianness} und das Debian Linux, mit dem wir arbeiten, ist ebenfalls Big
Endian.

Wenn nun Byte Variablen in 32-Bit-Feldern im struct gespeichert werden, besetzen sie die hohen Bits 24..31.

Wenn ein Variable vom Typ \Tchar zu einem 32-Bit-Wert erweitert werden muss, muss sie um 24 Bits nach rechts verschoben
werden.

\Tchar ist ein vorzeichenbehafteter Typ, sodass hier eine arithmetische Verschiebung anstelle einer logischen erfolgen
muss.


\subsubsection{Eine Sache noch}
Ein struct als Funktionsargument zu übergeben (anstelle eines Pointers auf ein struct) ist das gleiche wie alle Felder
des structs einzeln zu übergeben.

Wenn die Felder im struct standardmäßig gepackt werden, kann die Funktion f() wie folgt neu geschrieben werden:

\begin{lstlisting}[style=customc]
void f(char a, int b, char c, int d)
{
    printf ("a=%d; b=%d; c=%d; d=%d\n", a, b, c, d);
};
\end{lstlisting}
Das führt schlussendlich zum gleichen Code.
