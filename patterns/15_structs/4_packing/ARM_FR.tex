\subsubsection{ARM}

\myparagraph{\OptimizingKeilVI (\ThumbMode)}

\lstinputlisting[caption=\OptimizingKeilVI (\ThumbMode),style=customasmARM]{patterns/15_structs/4_packing/packing_Keil_thumb.asm}

Rappelons-nous que c'est une structure qui est passée ici et non pas un pointeur vers une structure. Comme 
les 4 premiers arguments d'une fonction sont passés dans les registres sur les processeurs ARM, les champs 
de la structure sont passés dans les registres \TT{R0-R3}.

\myindex{ARM!\Instructions!LDRB}
\myindex{x86!\Instructions!MOVSX}
\TT{LDRB} charge un octet présent en mémoire et l'étend sur 32bits en prenant en compte son signe. Cette 
opération est similaire à celle effectuée par \MOVSX dans les architectures x86. Elle est utilisée ici pour 
charger les champs $a$ et $c$ de la structure.

\myindex{Epilogue de fonction}

Un autre détail que nous remarquons aisément est que la fonction ne s'achève pas sur un épilogue qui lui est 
propre. A la place, il y a un saut vers l'épilogue d'une autre fonction! Qui plus est celui d'une fonction 
très différente sans aucun lien avec la nôtre. Cependant elle possède exactement le même épilogue, 
probablement parce qu'elle accepte utilise elle aussi 5 variables locales ($5*4=0x14$).

De plus elle est située à une adresse proche.

En réalité, peut importe l'épilogue qui est utilisé du moment que le fonctionnement est celui attendu.

Il semble donc que le compilateur Keil décide de réutiliser à des fins d'économie un fragment d'une autre 
fonction. Notre épilogue aurait nécessité 4 octets. L'instruction de saut n'en utilise que 2.

\myparagraph{ARM + \OptimizingXcodeIV (\ThumbTwoMode)}

\lstinputlisting[caption=\OptimizingXcodeIV (\ThumbTwoMode),style=customasmARM]{patterns/15_structs/4_packing/packing_Xcode_thumb.asm}

\myindex{ARM!\Instructions!SXTB}
\myindex{x86!\Instructions!MOVSX}
\TT{SXTB} (\IT{Signed Extend Byte}) est similaire à \MOVSX pour les architectures x86.
Pour le reste---c'est identique.
