\subsubsection{ARM}

\myparagraph{\OptimizingKeilVI (\ThumbMode)}

\lstinputlisting[caption=\OptimizingKeilVI (\ThumbMode),style=customasmARM]{patterns/15_structs/4_packing/packing_Keil_thumb.asm}
Wir sehen, dass hier ein struct anstelle eines Pointers auf ein struct übergeben wird und da die ersten vier
Funktionsargumente in ARM über Register übergeben werden, werden die Felder des structs mittels \TT{R0-R3} übergeben.

\myindex{ARM!\Instructions!LDRB}
\myindex{x86!\Instructions!MOVSX}
\TT{LDRB} lädt ein Byte aus dem Speicher und erweitert es unter Berücksichtigung des Vorzeichens auf 32 Bit.
Dies entspricht \MOVSX in x86. 
Hier wird der Befehl verwendet, um die Felder $a$ und $a$ des structs zu laden.

\myindex{Funktionsepilog}
Eine weitere Sache, die ins Auge sticht, ist, dass anstelle eines Funktionsepilogs ein Sprung zum Epilog einer anderen
Funktion vorliegt.
Bei der anderen handelt es sich um eine komplett unterschiedliche Funktion, die in keiner Weise mit unserer
zusammenhängt, aber denselben Epilog hat (möglicherweise, da diese ebenfalls 5 lokale Variablen enthält ($5*4=0x14$)).

Außerdem befindet sie sich im Code in der Nähe (man vergleiche die Adressen).

Tatsächlich spielt es auch keine Rolle, welcher Epilog ausgeführt wird, solange dieser wie gewünscht funktioniert.

Offensichtlich hat Keil aus ökonomischen Gründen entschieden, einen Teil der anderen Funktion wiederzuverwenden.

Der Epilog benötigt 4 Bytes---während der Sprung nur 2 verbraucht.

\myparagraph{ARM + \OptimizingXcodeIV (\ThumbTwoMode)}

\lstinputlisting[caption=\OptimizingXcodeIV (\ThumbTwoMode),style=customasmARM]{patterns/15_structs/4_packing/packing_Xcode_thumb.asm}

\myindex{ARM!\Instructions!SXTB}
\myindex{x86!\Instructions!MOVSX}
\TT{SXTB} (\IT{Signed Extend Byte}) ist analog zu \MOVSX in x86.
Der ganze Rest ist identisch.

