\subsubsection{ARM}

\myparagraph{\OptimizingKeilVI (\ThumbMode)}

\lstinputlisting[caption=\OptimizingKeilVI (\ThumbMode),style=customasmARM]{patterns/15_structs/4_packing/packing_Keil_thumb.asm}

Как мы помним, здесь передается не указатель на структуру, а сама структура, а так как в ARM первые 4 аргумента
функции передаются через регистры, то поля структуры передаются через \TT{R0-R3}.

\myindex{ARM!\Instructions!LDRB}
\myindex{x86!\Instructions!MOVSX}
Инструкция \TT{LDRB} загружает один байт из памяти и расширяет до 32-бит учитывая знак.

Это то же что и инструкция \MOVSX в x86.
Она здесь применяется для загрузки полей $a$ и $c$ из структуры.

\myindex{Function epilogue}
Еще что бросается в глаза, так это то что вместо эпилога функции, переход на эпилог другой функции!

Действительно, то была совсем другая, не относящаяся к этой, функция, однако, она имела точно такой же эпилог 
(видимо, тоже хранила в стеке 5 локальных переменных ($5*4=0x14$)).
К тому же, она находится рядом (обратите внимание на адреса).

Действительно, нет никакой разницы, какой эпилог исполнять, если он работает так же, как нам нужно.

Keil решил использовать часть другой функции, вероятно, из-за экономии.

Эпилог занимает 4 байта, а переход ~--- только 2.

\myparagraph{ARM + \OptimizingXcodeIV (\ThumbTwoMode)}

\lstinputlisting[caption=\OptimizingXcodeIV (\ThumbTwoMode),style=customasmARM]{patterns/15_structs/4_packing/packing_Xcode_thumb.asm}

\myindex{ARM!\Instructions!SXTB}
\myindex{x86!\Instructions!MOVSX}
\TT{SXTB} (\IT{Signed Extend Byte}) это также аналог \MOVSX в x86.
Всё остальное ~--- так же.

