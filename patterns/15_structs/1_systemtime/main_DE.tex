\subsection{MSVC: SYSTEMTIME Beispiel}
\label{sec:SYSTEMTIME}

\newcommand{\FNSYSTEMTIME}{\footnote{\href{http://go.yurichev.com/17260}{MSDN: SYSTEMTIME structure}}}
Betrachten wir das SYSTEMTIME\FNSYSTEMTIME{} struct in win32, das die Systemzeit beschreibt.

Das struct ist folgendermaßen definiert:

\begin{lstlisting}[caption=WinBase.h,style=customc]
typedef struct _SYSTEMTIME {
  WORD wYear;
  WORD wMonth;
  WORD wDayOfWeek;
  WORD wDay;
  WORD wHour;
  WORD wMinute;
  WORD wSecond;
  WORD wMilliseconds;
} SYSTEMTIME, *PSYSTEMTIME;
\end{lstlisting}
Schreiben wir eine C-Funktion, um die aktuelle Zeit auszugeben:

\lstinputlisting[style=customc]{patterns/15_structs/1_systemtime/systemtime.c}

Wir erhalten das Folgende (MSVC 2010):

\lstinputlisting[caption=MSVC 2010 /GS-,style=customasmx86]{patterns/15_structs/1_systemtime/systemtime.asm}
Für dieses struct werden 16 Byte im lokalen Stack reserviert~---das entspricht genau \TT{sizeof(WORD)*8} (es gibt 8
WORD Variablen in diesem struct).

\newcommand{\FNMSDNGST}{\footnote{\href{http://go.yurichev.com/17261}{MSDN: GetSystemTime function}}}
Man beachte, dass dieses struct mit dem \TT{wYear} Feld beginnt.
Man kann sagen, dass ein Pointer auf das SYSTEMTIME struct an die Funktion \TT{GetSystemTime()}\FNSYSTEMTIME übergeben
wird, aber man könnte auch sagen, dass ein Pointer auf das Feld \TT{wYear} übergeben wird, denn dabei handelt es sich um
dasselbe!
\TT{GetSystemTime()} schreibt das aktuelle Jahr in den WORD Pointer, verschiebt um 2 Byte, schreibt den aktuellen Monat,
usw. usf. 

\clearpage
\subsubsection{\olly}
\myindex{\olly}
Kompilieren wir dieses Beispiel in MSVC 2010 mit \TT{/GS- /MD} und laden es in \olly.

Öffnen wir die Fenster für Daten und Stack an der Adresse, die als erstes Argument der Funktion \TT{GetSystemTime()}
übergeben wird und warten, bis das Programm an dieser Stelle ist. Wir sehen das folgende:

\begin{figure}[H]
\centering
\myincludegraphics{patterns/15_structs/1_systemtime/olly_systemtime1.png}
\caption{\olly: \TT{GetSystemTime()} wurde gerade ausgeführt}
\label{fig:struct_olly_1}
\end{figure}
Die Systemzeit, die diese Ausführung der Funktion auf meinem Computer liefert, ist 9. Dezember 2014, 22:29:52:

\lstinputlisting[caption=\printf output]{patterns/15_structs/1_systemtime/console.txt}
Wir sehen also diese 16 Byte im Datenfenster:
 
\begin{lstlisting}
DE 07 0C 00 02 00 09 00 16 00 1D 00 34 00 D4 03
\end{lstlisting}
Je zwei Byte repräsentieren ein Feld des structs. 
Da die \glslink{endianness}{Endianess} hier \IT{little Endian} ist, finden wir das niederwertige Byte zuerst und danach das
höherwertige.

Es werden also die folgenden Werte aktuell im Speicher gehalten:

\begin{center}
\begin{tabular}{ | l | l | l | }
\hline
\headercolor{} Hexadezimalzahl & 
\headercolor{} Dezimalzahl & 
\headercolor{} Feldname \\
\hline
0x07DE & 2014	& wYear \\
\hline
0x000C & 12	& wMonth \\
\hline
0x0002 & 2	& wDayOfWeek \\
\hline
0x0009 & 9	& wDay \\
\hline
0x0016 & 22	& wHour \\
\hline
0x001D & 29	& wMinute \\
\hline
0x0034 & 52	& wSecond \\
\hline	
0x03D4 & 980	& wMilliseconds \\
\hline
\end{tabular}
\end{center}
Wir finden die gleichen Werte im Stackfenster, aber sie werden als 32-Bit-Werte gruppiert.

Die Funktion \printf nimmt sich die Werte, die sie braucht und schreibt sie in die Konsole.

Manche Werte werden von \printf nicht ausgegeben (\TT{wDayOfWeek} und \TT{wMilliseconds}), aber sie sind im Speicher
jederzeit für uns verfügbar.


\subsubsection{Ein struct durch ein Array ersetzen}
Die Tatsache, dass die Felder eines structs Variablen sind, die nebeneinander angeordnet sind, kann leicht durch
folgendes Beispiel belegt werden. Wir erinnern uns an die Beschreibung des \TT{SYSTEMTIME} structs und schreiben unser
Beispiel wie folgt um: 

\lstinputlisting[style=customc]{patterns/15_structs/1_systemtime/systemtime2.c}
Der Compiler meckert ein wenig:

\begin{lstlisting}
systemtime2.c(7) : warning C4133: 'function' : incompatible types - from 'WORD [8]' to 'LPSYSTEMTIME'
\end{lstlisting}

Trotzdem erzeugt er den folgenden Code:

\lstinputlisting[caption=\NonOptimizing MSVC 2010,style=customasmx86]{patterns/15_structs/1_systemtime/systemtime2.asm}
Dieses Programm funktioniert genau wie das erste!

Sehr interessant ist, dass dieser Assemblercode nicht vom entsprechenden Code des ersten Beispiels unterschieden werden
kann.

Beim bloßen Ansehen des Codes kann man also nicht feststellen, ob ein struct oder ein Array deklariert wurde.

Trotzdem würde man letzteres nicht annehmen, da es sehr ungebräuchlich ist.

DIe Felder des structs können durch den Entwickler ausgetauscht oder verändert werden, etc.

Wir untersuchen dieses Beispiel nicht in \olly, da es mit dem Beispiel mit dem struct identisch ist.

