\subsection{MSVC: exemple SYSTEMTIME}
\label{sec:SYSTEMTIME}

\newcommand{\FNSYSTEMTIME}{\footnote{\href{http://go.yurichev.com/17260}{MSDN: SYSTEMTIME structure}}}

Considérons la structure win32 SYSTEMTIME\FNSYSTEMTIME{} qui décrit un instant dans le temps. Voici 
comment elle est définie:

\begin{lstlisting}[caption=WinBase.h,style=customc]
typedef struct _SYSTEMTIME {
  WORD wYear;
  WORD wMonth;
  WORD wDayOfWeek;
  WORD wDay;
  WORD wHour;
  WORD wMinute;
  WORD wSecond;
  WORD wMilliseconds;
} SYSTEMTIME, *PSYSTEMTIME;
\end{lstlisting}

Ecrivons une fonction C pour récupérer l'instant qu'il est:

\lstinputlisting[style=customc]{patterns/15_structs/1_systemtime/systemtime.c}

Le résultat de la compilation avec MSVC 2010 donne:

\lstinputlisting[caption=MSVC 2010 /GS-,style=customasmx86]{patterns/15_structs/1_systemtime/systemtime.asm}

16 octets sont réservés sur la pile pour cette structure, ce qui correspond exactement à 
\TT{sizeof(WORD)*8}. La structure comprend effectivement 8 variables d'un WORD chacun.

\newcommand{\FNMSDNGST}{\footnote{\href{http://go.yurichev.com/17261}{MSDN: GetSystemTime function}}}

Faites attention au fait que le premier membre de la structure est le champ \TT{wYear}.
On peut donc considérer que la fonction \TT{GetSystemTime()}\FNSYSTEMTIME reçoit comme argument 
un pointeur sur la structure SYSTEMTIME, ou bien qu'elle reçoit un pointeur sur le champ \TT{wYear}. 
Et en fait c'est exactement la même chose!
\TT{GetSystemTime()} écrit l'année courante dans à l'adresse du WORD qu'il a reçu, avance de 2 
octets, écrit le mois courant et ainsi de suite.

\clearpage
\subsubsection{\olly}
\myindex{\olly}

Compilons cet exemple avec MSVC 2010 et les options \TT{/GS- /MD}, puis exécutons le avec \olly.

Ouvrons la fenêtre des données et celle de la pile à l'adresse du premier argument fourni à la 
fonction \TT{GetSystemTime()}, puis attendons que cette fonction se termine. Nous constatons :

\begin{figure}[H]
\centering
\myincludegraphics{patterns/15_structs/1_systemtime/olly_systemtime1.png}
\caption{\olly: Juste après l'appel à \TT{GetSystemTime()}}
\label{fig:struct_olly_1}
\end{figure}

Sur mon ordinateur, le résultat de l'appel à la fonction est 9 décembre 2014, 22:29:52:

\lstinputlisting[caption=\printf output]{patterns/15_structs/1_systemtime/console.txt}

Nous observons donc ces 16 octets dans la fenêtre de données:
\begin{lstlisting}
DE 07 0C 00 02 00 09 00 16 00 1D 00 34 00 D4 03
\end{lstlisting}

Chaque paire d'octets représente l'un des champs de la structure. 
Puisque nous sommes en mode petit-boutien l'octet de poids faible est situé en premier, suivi de 
l'octet de poids fort.

Les valeurs effectivement présentes en mémoire sont donc les suivantes:

\begin{center}
\begin{tabular}{ | l | l | l | }
\hline
\headercolor{} nombre hexadécimal & 
\headercolor{} nombre décimal & 
\headercolor{} nom du champ \\
\hline
0x07DE & 2014	& wYear \\
\hline
0x000C & 12	& wMonth \\
\hline
0x0002 & 2	& wDayOfWeek \\
\hline
0x0009 & 9	& wDay \\
\hline
0x0016 & 22	& wHour \\
\hline
0x001D & 29	& wMinute \\
\hline
0x0034 & 52	& wSecond \\
\hline	
0x03D4 & 980	& wMilliseconds \\
\hline
\end{tabular}
\end{center}

Les mêmes valeurs apparaissent dans la fenêtre de la pile, mais elle y sont regroupées sous forme 
de valeurs 32 bits.

La fonction \printf utilise les valeurs qui lui sont nécessaires et les affiche à la console.

Bien que certaines valeurs telles que (\TT{wDayOfWeek} et \TT{wMilliseconds}) ne soient pas 
affichées par \printf, elles sont bien présentes en mémoire, prêtes à être utilisées.


\subsubsection{Remplacer la structure par un tableau}

Le fait que les champs d'une structure ne sont que des variables situées côte-à-côte peut être 
aisément démontré de la manière suivante.
Tout en conservant à l'esprit la description de la structure \TT{SYSTEMTIME}, il est possible de 
réécrire cet exemple simple de la manière suivante:

\lstinputlisting[style=customc]{patterns/15_structs/1_systemtime/systemtime2.c}

Le compilateur ronchonne certes un peu:

\begin{lstlisting}
systemtime2.c(7) : warning C4133: 'function' : incompatible types - from 'WORD [8]' to 'LPSYSTEMTIME'
\end{lstlisting}

Mais, il consent quand même à produire le code suivant:

\lstinputlisting[caption=\NonOptimizing MSVC 2010,style=customasmx86]{patterns/15_structs/1_systemtime/systemtime2.asm}

Qui fonctionne à l'identique du précédent!

Il est extrêmement intéressant de constater que le code assembleur produit est impossible à 
distinguer de celui produit par la compilation précédente.

Et ainsi celui qui observe ce code assembleur est incapable de dédicer avec certitude si une 
structure ou un tableau était déclaré dans le code source en C. 

Cela étant, aucun esprit sain ne s'amuserai à déclarer un tableau ici. Car il faut aussi compter 
avec la possibilité que la structure soit modifiée par les développeurs, que les champs soient 
triés dans un autre ordre ...

Nous n'étudierons pas cet exemple avec \olly, car les résultats seraient identiques à ceux que nous 
avons observé en utilisant la structure.

