\subsection{MSVC: SYSTEMTIME example}
\label{sec:SYSTEMTIME}

\newcommand{\FNSYSTEMTIME}{\footnote{\href{http://go.yurichev.com/17260}{MSDN: SYSTEMTIME structure}}}

Let's take the SYSTEMTIME\FNSYSTEMTIME{} win32 structure that describes time.

This is how it's defined:

\begin{lstlisting}[caption=WinBase.h,style=customc]
typedef struct _SYSTEMTIME {
  WORD wYear;
  WORD wMonth;
  WORD wDayOfWeek;
  WORD wDay;
  WORD wHour;
  WORD wMinute;
  WORD wSecond;
  WORD wMilliseconds;
} SYSTEMTIME, *PSYSTEMTIME;
\end{lstlisting}

Let's write a C function to get the current time:

\lstinputlisting[style=customc]{patterns/15_structs/1_systemtime/systemtime.c}

We get (MSVC 2010):

\lstinputlisting[caption=MSVC 2010 /GS-,style=customasm]{patterns/15_structs/1_systemtime/systemtime.asm}

16 bytes are allocated for this structure in the local stack~---that is exactly \TT{sizeof(WORD)*8}
(there are 8 WORD variables in the structure).

\newcommand{\FNMSDNGST}{\footnote{\href{http://go.yurichev.com/17261}{MSDN: GetSystemTime function}}}

Pay attention to the fact that the structure begins with the \TT{wYear} field.
It can be said that a pointer to the SYSTEMTIME structure is passed to the \TT{GetSystemTime()}\FNSYSTEMTIME,
but it is also can be said that a pointer to the \TT{wYear} field is passed, and that is the same!
\TT{GetSystemTime()} writes the current year to the WORD pointer pointing to, then shifts 2 bytes
ahead, writes current month, etc., etc.

\clearpage
\mysubparagraph{\olly}
\myindex{\olly}

Let's try this example in \olly.
The input value of the function (2) is loaded into \EAX: 

\begin{figure}[H]
\centering
\myincludegraphics{patterns/08_switch/2_lot/olly1.png}
\caption{\olly: function's input value is loaded in \EAX}
\label{fig:switch_lot_olly1}
\end{figure}

\clearpage
The input value is checked, is it bigger than 4? 
If not, the \q{default} jump is not taken:
\begin{figure}[H]
\centering
\myincludegraphics{patterns/08_switch/2_lot/olly2.png}
\caption{\olly: 2 is no bigger than 4: no jump is taken}
\label{fig:switch_lot_olly2}
\end{figure}

\clearpage
Here we see a jumptable:

\begin{figure}[H]
\centering
\myincludegraphics{patterns/08_switch/2_lot/olly3.png}
\caption{\olly: calculating destination address using jumptable}
\label{fig:switch_lot_olly3}
\end{figure}

Here we've clicked \q{Follow in Dump} $\rightarrow$ \q{Address constant}, so now we see the \IT{jumptable} in the data window.
These are 5 32-bit values\footnote{They are underlined by \olly because
these are also FIXUPs: \myref{subsec:relocs}, we are going to come back to them later}.
\ECX is now 2, so the third element (can be indexed as 2\footnote{About indexing, see also: \ref{arrays_at_one}}) of the table is to be used.
It's also possible to click \q{Follow in Dump} $\rightarrow$ 
\q{Memory address} and \olly will show the element addressed by the \JMP instruction. 
That's \TT{0x010B103A}.

\clearpage
After the jump we are at \TT{0x010B103A}: the code printing \q{two} will now be executed:

\begin{figure}[H]
\centering
\myincludegraphics{patterns/08_switch/2_lot/olly4.png}
\caption{\olly: now we at the \IT{case:} label}
\label{fig:switch_lot_olly4}
\end{figure}


\subsubsection{Replacing the structure with array}

The fact that the structure fields are just variables located side-by-side, can be easily demonstrated by doing the following.
Keeping in mind the \TT{SYSTEMTIME} structure description, it's possible to rewrite this simple example like this:

\lstinputlisting[style=customc]{patterns/15_structs/1_systemtime/systemtime2.c}

The compiler grumbles a bit:

\begin{lstlisting}
systemtime2.c(7) : warning C4133: 'function' : incompatible types - from 'WORD [8]' to 'LPSYSTEMTIME'
\end{lstlisting}

But nevertheless, it produces this code:

\lstinputlisting[caption=\NonOptimizing MSVC 2010,style=customasm]{patterns/15_structs/1_systemtime/systemtime2.asm}

And it works just as the same!

It is very interesting that the
result in assembly form cannot be distinguished from the result of the previous compilation.

So by looking at this code, one cannot say for sure if there was a structure declared, or an array. 

Nevertheless, no sane person would do it, as it is not convenient. 

Also the structure fields may be changed by developers, swapped, etc.

We will not study this example in \olly, because it will be just the same as in the case with the structure.

