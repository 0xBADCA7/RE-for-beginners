\clearpage
\subsubsection{\olly}
\myindex{\olly}

Compilons cet exemple avec MSVC 2010 et les options \TT{/GS- /MD}, puis exécutons le avec \olly.

Ouvrons la fenêtre des données et celle de la pile à l'adresse du premier argument fourni à la 
fonction \TT{GetSystemTime()}, puis attendons que cette fonction se termine. Nous constatons :

\begin{figure}[H]
\centering
\myincludegraphics{patterns/15_structs/1_systemtime/olly_systemtime1.png}
\caption{\olly: Juste après l'appel à \TT{GetSystemTime()}}
\label{fig:struct_olly_1}
\end{figure}

Sur mon ordinateur, le résultat de l'appel à la fonction est 9 décembre 2014, 22:29:52:

\lstinputlisting[caption=\printf output]{patterns/15_structs/1_systemtime/console.txt}

Nous observons donc ces 16 octets dans la fenêtre de données:
\begin{lstlisting}
DE 07 0C 00 02 00 09 00 16 00 1D 00 34 00 D4 03
\end{lstlisting}

Chaque paire d'octets représente l'un des champs de la structure. 
Puisque nous sommes en mode petit-boutien l'octet de poids faible est situé en premier, suivi de 
l'octet de poids fort.

Les valeurs effectivement présentes en mémoire sont donc les suivantes:

\begin{center}
\begin{tabular}{ | l | l | l | }
\hline
\headercolor{} nombre hexadécimal & 
\headercolor{} nombre décimal & 
\headercolor{} nom du champ \\
\hline
0x07DE & 2014	& wYear \\
\hline
0x000C & 12	& wMonth \\
\hline
0x0002 & 2	& wDayOfWeek \\
\hline
0x0009 & 9	& wDay \\
\hline
0x0016 & 22	& wHour \\
\hline
0x001D & 29	& wMinute \\
\hline
0x0034 & 52	& wSecond \\
\hline	
0x03D4 & 980	& wMilliseconds \\
\hline
\end{tabular}
\end{center}

Les mêmes valeurs apparaissent dans la fenêtre de la pile, mais elle y sont regroupées sous forme 
de valeurs 32 bits.

La fonction \printf utilise les valeurs qui lui sont nécessaires et les affiche à la console.

Bien que certaines valeurs telles que (\TT{wDayOfWeek} et \TT{wMilliseconds}) ne soient pas 
affichées par \printf, elles sont bien présentes en mémoire, prêtes à être utilisées.
