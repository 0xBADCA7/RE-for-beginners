\subsubsection{Structure as a set of values}

In order to illustrate that the structure is just variables laying side-by-side in one place, 
let's rework our example while looking at the \IT{tm} structure definition again: \lstref{struct_tm}.

\lstinputlisting{patterns/15_structs/3_tm_linux/as_array/GCC_tm2.c}

\myindex{\CStandardLibrary!localtime\_r()}
N.B. 
The pointer to the \TT{tm\_sec} field is passed into \TT{localtime\_r}, i.e., 
to the first element of the \q{structure}.

The compiler warns us:

\begin{lstlisting}[caption=GCC 4.7.3]
GCC_tm2.c: In function 'main':
GCC_tm2.c:11:5: warning: passing argument 2 of 'localtime_r' from incompatible pointer type [enabled by default]
In file included from GCC_tm2.c:2:0:
/usr/include/time.h:59:12: note: expected 'struct tm *' but argument is of type 'int *'
\end{lstlisting}

But nevertheless, it generates this:

\lstinputlisting[caption=GCC 4.7.3]{patterns/15_structs/3_tm_linux/as_array/GCC_tm2.asm}

This code is identical to what we saw previously and it is
not possible to say, was it a structure in original source code or just a pack of variables.

And this works. 
However, it is not recommended to do this in practice. 

Usually, non-optimizing compilers allocates variables in the local stack in the 
same order as they were declared in the function.

Nevertheless, there is no guarantee.

By the way, some other compiler may warn about the \TT{tm\_year}, \TT{tm\_mon}, \TT{tm\_mday},
\TT{tm\_hour}, \TT{tm\_min} variables, but not \TT{tm\_sec}
 are used without being initialized.

Indeed, the compiler is not aware that these are to be filled by\\
\TT{localtime\_r()} function.

We chose this example, since all structure fields are of type \Tint.%

This would not work if structure fields are 16-bit (\TT{WORD}), 
like in the case of the \TT{SYSTEMTIME} structure---\TT{GetSystemTime()} 
will fill them incorrectly 
(because the local variables are aligned on a 32-bit boundary).
Read more about it in next section: 
\q{\StructurePackingSectionName} (\myref{structure_packing}).

So, a structure is just a pack of variables laying in one place, side-by-side.
We could say that the structure is the instruction to the compiler, directing it to hold variables in one place.
By the way, in some very early C versions (before 1972), there were no structures at all \RitchieDevC.

There is no debugger example here: it is just the same as you already saw.

\subsubsection{Structure as an array of 32-bit words}

\lstinputlisting{patterns/15_structs/3_tm_linux/as_array/GCC_tm3.c}

We just \IT{cast} a pointer to structure to an array of \Tint{}'s.
And that works!
We run the example at 23:51:45 26-July-2014.

\begin{lstlisting}[label=GCC_tm3_output]
0x0000002D (45)
0x00000033 (51)
0x00000017 (23)
0x0000001A (26)
0x00000006 (6)
0x00000072 (114)
0x00000006 (6)
0x000000CE (206)
0x00000001 (1)
\end{lstlisting}

The variables here 
are in the same order as they are enumerated in the definition of the structure: \myref{struct_tm}.

Here is how it gets compiled:

\lstinputlisting[caption=\Optimizing GCC 4.8.1]{patterns/15_structs/3_tm_linux/as_array/GCC_tm3_EN.lst}

Indeed: the space in the local stack is first treated as a structure, and then it's treated as an array.

It's even possible to modify the fields of the structure through this pointer.

And again, it's dubiously hackish way to do things, not recommended for use in production code.

\mysubparagraph{\Exercise}

As an exercise, try to modify (increase by 1) the current month number, treating the structure as 
an array.

\subsubsection{Structure as an array of bytes}

We can go even further. Let's \IT{cast} the pointer to an array of bytes and dump it:

\lstinputlisting{patterns/15_structs/3_tm_linux/as_array/GCC_tm4.c}

\begin{lstlisting}
0x2D 0x00 0x00 0x00 
0x33 0x00 0x00 0x00 
0x17 0x00 0x00 0x00 
0x1A 0x00 0x00 0x00 
0x06 0x00 0x00 0x00 
0x72 0x00 0x00 0x00 
0x06 0x00 0x00 0x00 
0xCE 0x00 0x00 0x00 
0x01 0x00 0x00 0x00 
\end{lstlisting}

We also run this example at 23:51:45 26-July-2014
\footnote{The time and date are the same for demonstration purposes. Byte values are fixed up.}.
The values are just the same as in the previous dump 
(\myref{GCC_tm3_output}), and of course, the lowest byte goes first, because this is a little-endian architecture 
(\myref{sec:endianness}).

\lstinputlisting[caption=\Optimizing GCC 4.8.1]{patterns/15_structs/3_tm_linux/as_array/GCC_tm4_EN.lst}
