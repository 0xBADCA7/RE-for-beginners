\subsection{Structures imbriquées}

Maintenant qu'en est-il lorsqu'une structure est définie au sein d'une autre structure ?

\lstinputlisting[style=customc]{patterns/15_structs/5_nested/nested.c}

\dots dans ce cas, l'ensemble des champs de \TT{inner\_struct} doivent être situés entre les champs a,b et 
d,e de \TT{outer\_struct}.

Compilons (MSVC 2010):

\lstinputlisting[caption=\Optimizing MSVC 2010 /Ob0,style=customasmx86]{patterns/15_structs/5_nested/nested_msvc.asm}

Un point troublant est qu'en observant le code assembleur généré, nous n'avons aucun indice qui laisse penser 
qu'il existe une structure imbriquée!
Nous pouvonsdonc dire que les structures imbriquées sont fusionnées avec leur conteneur pour former une seule 
structure \IT{linear} ou \IT{one-dimensional}.

Bien entendu, si nous remplacons la déclaration \TT{struct inner\_struct c;} par \TT{struct inner\_struct *c;} 
(en introduisant donc un pointeur) la situation sera totalement différente.
% FIXME1: нарисовать вложенную структуру и развернутую

\clearpage
\subsubsection{\olly}
\myindex{\olly}

Chargeons notre exemple dans \olly et observons \TT{outer\_struct} en mémoire:

\begin{figure}[H]
\centering
\myincludegraphics{patterns/15_structs/5_nested/olly.png}
\caption{\olly: Avant appel de la fonction \printf}
\label{fig:nested_olly}
\end{figure}

Les valeurs sont organisées en mémoire de la manière suivante:
\begin{itemize}
\item \IT{(outer\_struct.a)} (octet) 1 + 3 octets de détritus;
\item \IT{(outer\_struct.b)} (mot de 32 bits) 2;
\item \IT{(inner\_struct.a)} (mot de 32 bits) 0x64 (100);
\item \IT{(inner\_struct.b)} (mot de 32 bits) 0x65 (101);
\item \IT{(outer\_struct.d)} (octet) 3 + 3 octets de détritus;
\item \IT{(outer\_struct.e)} (mot de 32 bits) 4.
\end{itemize}

