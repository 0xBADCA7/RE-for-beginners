\subsection{Выделяем место для структуры через malloc()}
\label{struct_malloc_example}

Однако, бывает и так, что проще хранить структуры не в стеке, а в \glslink{heap}{куче}:


\lstinputlisting[style=customc]{patterns/15_structs/2_using_malloc/systemtime_malloc.c}

Скомпилируем на этот раз с оптимизацией (\Ox) чтобы было проще увидеть то, что нам нужно.


\lstinputlisting[caption=\Optimizing MSVC,style=customasmx86]{patterns/15_structs/2_using_malloc/systemtime_malloc.asm}

\myindex{\CStandardLibrary!malloc()}
Итак, \TT{sizeof(SYSTEMTIME) = 16}, именно столько байт выделяется при помощи \TT{malloc()}. 
Она возвращает указатель на только что выделенный блок памяти в \EAX, который копируется в \ESI. 
Win32 функция \TT{GetSystemTime()} обязуется сохранить состояние \ESI, 
поэтому здесь оно нигде не сохраняется и продолжает использоваться после вызова \TT{GetSystemTime()}.


\myindex{x86!\Instructions!MOVZX}

Новая инструкция ~--- \MOVZX (\IT{Move with Zero eXtend}). 
Она нужна почти там же где и \MOVSX, 
только всегда очищает остальные биты в 0. Дело в том, что \printf требует 32-битный тип \Tint, 
а в структуре лежит WORD ~--- это 16-битный беззнаковый тип. Поэтому копируя значение из WORD в \Tint, 
нужно очистить биты от 16 до 31, иначе там будет просто случайный мусор, оставшийся от предыдущих действий 
с регистрами.


В этом примере можно также представить структуру как массив 8-и WORD-ов:


\lstinputlisting[style=customc]{patterns/15_structs/2_using_malloc/systemtime_malloc2.c}

Получим такое:

\lstinputlisting[caption=\Optimizing MSVC,style=customasmx86]{patterns/15_structs/2_using_malloc/systemtime_malloc2.asm}

И снова мы получаем идентичный код, неотличимый от предыдущего.

Но и снова нужно отметить, что в реальности так лучше не делать, 
если только вы не знаете точно, что вы делаете.


