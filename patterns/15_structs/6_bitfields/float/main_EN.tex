\subsubsection{\WorkingWithFloatAsWithStructSubSubSectionName}
\label{sec:floatasstruct}

As we already noted in the section about FPU~(\myref{sec:FPU}), 
both \Tfloat and \Tdouble types consist of a \IT{sign}, 
a \IT{significand} (or \IT{fraction}) and an \IT{exponent}.
But will we be able to work with these fields directly? Let's try this with \Tfloat.

\bigskip
% a hack used here! http://tex.stackexchange.com/questions/73524/bytefield-package
\begin{center}
\begin{bytefield}{32}
	\bitheader[endianness=big]{0,22,23,30,31} \\
	\bitbox{1}{S} & 
	\bitbox{8}{%
		\RU{экспонента}%
		\EN{exponent}%
		\ES{exponente}%
		\PTBRph{}%
		\DEph{}\PLph{}%
		\ITAph{}%
		\FR{exposant}%
		\JPN{指数}
	} & 
	\bitbox{23}{%
		\RU{мантисса}%
		\EN{mantissa or fraction}%
		\ES{mantisa o fracci\'on}%
		\PTBRph{}%
		\DEph{}\PLph{}%
		\ITAph{}%
		\FR{mantisse ou fraction}%
		\JPN{対数または比}
	}
\end{bytefield}
\end{center}

\begin{center}
( S\EMDASH{}%
	\RU{знак}%
	\EN{sign}%
	\ES{signo}%
	\PTBRph{}%
	\DEph{}\PLph{}%
	\ITAph{}%
	\FR{signe}%
	\JPN{記号}
)
\end{center}


\lstinputlisting[style=customc]{patterns/15_structs/6_bitfields/float/float_EN.c}

The \TT{float\_as\_struct} structure occupies the same amount of  memory as \Tfloat, i.e., 4 bytes or 32 bits.

Now we are setting the negative sign in the input value and also, by adding 2 to the exponent, 
we thereby multiply the whole number by \TT{$2^2$}, i.e., by 4.

Let's compile in MSVC 2008 without optimization turned on:

\lstinputlisting[caption=\NonOptimizing MSVC 2008,style=customasm]{patterns/15_structs/6_bitfields/float/float_msvc_EN.asm}

A bit redundant.
If it was compiled with \Ox flag there would be no \TT{memcpy()} call,
the \TT{f} variable is used directly.
But it is easier to understand by looking at the unoptimized version.

What would GCC 4.4.1 with \Othree do?

\lstinputlisting[caption=\Optimizing GCC 4.4.1,style=customasm]{patterns/15_structs/6_bitfields/float/float_gcc_O3_EN.asm}

The \ttf function is almost understandable. However, what is interesting is that GCC was able to calculate
the result of \TT{f(1.234)} during compilation despite all this hodge-podge with the structure fields
and prepared this argument to \printf{} as precalculated at compile time!
