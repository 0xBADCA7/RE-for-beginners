\subsubsection{CPUID Beispiel}
Die Sprache \CCpp erlaubt die Definition der exakten Anzahl von Bits für jedes Feld in einem struct.
Das ist sehr nützlich, wenn man Speicherplatz sparen muss.
Zum Beispiel genügt ein Bit für eine \Tbool Variable.
Natürlich ist dieses Vorgehen nicht angebracht, wenn die Geschwindigkeit wichtig ist.

% FIXME!
% another use of this is to parse binary protocols/packets, for example
% the definition of struct iphdr in include/linux/ip.h

\newcommand{\FNCPUID}{\footnote{\href{http://go.yurichev.com/17069}{wikipedia}}}

\myindex{x86!\Instructions!CPUID}
\label{cpuid}
Betrachten wir das Beispiel des \CPUID\FNCPUID Befehls.
Dieser Befehl liefert Informationen über die aktuelle CPU und ihre Eigenschaften.

Wenn \EAX vor der Ausführung des Befehls auf 1 gesetzt ist, liefert \CPUID diese Informationen gepackt in das \EAX
Register zurück:

\begin{center}
\begin{tabular}{ | l | l | }
\hline
3:0 (4 bits)& Schrittweite \\
7:4 (4 bits) & Modell \\
11:8 (4 bits) & Familie \\
13:12 (2 bits) & Prozessortype \\
19:16 (4 bits) & Erweitertes Modell \\
27:20 (8 bits) & Erweiterte Familie \\
\hline
\end{tabular}
\end{center}

\newcommand{\FNGCCAS}{\footnote{\href{http://go.yurichev.com/17070}
{Mehr zum internen GCC Assembler}}}
MSVC 2010 verfügt über ein \CPUID Makro, aber GCC 4.4.1 nicht.
Erstellen wir also für uns eine solche Funktion in GCC, indem wir den built-in Assembler\FNGCCAS verwenden.

\lstinputlisting[style=customc]{patterns/15_structs/6_bitfields/cpuid/CPUID.c}
Nachdem \CPUID die Register \EAX/\EBX/\ECX/\EDX befüllt hat, werden deren Inhalte in das Array \TT{b[]} geschrieben.
Danach haben wir einen Pointer auf das \TT{CPUID\_1\_EAX} struct und zeigen auf den Wert in \EAX aus dem Array \TT{b[]}.

Mit anderen Worten: wir behandeln einen 32-Bit \Tint wie ein struct.
Danach lesen wir spezifische Bits aus dem struct.

\myparagraph{MSVC}
Kompilieren wir das Beispiel in MSVC 2008 mit der Option \Ox:

\lstinputlisting[caption=\Optimizing MSVC 2008,style=customasmx86]{patterns/15_structs/6_bitfields/cpuid/CPUID_msvc_Ox.asm}

\myindex{x86!\Instructions!SHR}
Der Befehl \TT{SHR} verschiebt den Wert in \EAX um die Anzahl der Bits die überprungen werden müssen, d.h. wir
ignorieren einige Bits am rechten Rand.

\myindex{x86!\Instructions!AND}
Der Befehl \AND löscht die nicht benötigten Bits am linken Rand bzw. belässt nur die Bits in \EAX, die wir auch
benötigen.

\clearpage
\myparagraph{MSVC + \olly}
\myindex{\olly}

Laden wir unser Beispiel in \olly 
und schauen welche Werte sich nach der Ausführung von CPUID in den Register \EAX/\EBX/\ECX und \EDX befinden:

\begin{figure}[H]
\centering
\myincludegraphics{patterns/15_structs/6_bitfields/cpuid/olly.png}
\caption{\olly: Nach Ausführung von CPUID}
\label{fig:cpuid_olly_1}
\end{figure}

EAX enthält \TT{0x000206A7} (meine \ac{CPU} ist ein Intel Xeon E3-1220).\\
Dies entspricht $0b0000 0000 0000 0010 0000 0110 1010 0111$ in Binärdarstellung.

Die Bits werden wie folgt durch die Felder aufgeteilt:

\begin{center}
\begin{tabular}{ | l | l | l | }
\hline
\headercolor{} Feld &
\headercolor{} in binär &
\headercolor{} in dezimal \\
\hline
reserved2		& 0000 & 0 \\
\hline
extended\_family\_id	& 00000000 & 0 \\
\hline
extended\_model\_id	& 0010 & 2 \\
\hline
reserved1		& 00 & 0 \\
\hline
processor\_id		& 00 & 0 \\
\hline
family\_id		& 0110 & 6 \\
\hline
model			& 1010 & 10 \\
\hline
stepping		& 0111 & 7 \\
\hline
\end{tabular}
\end{center}

\lstinputlisting[caption=Console output]{patterns/15_structs/6_bitfields/cpuid/console.txt}


\myparagraph{GCC}
Versuchen wir es mit GCC 4.4.1 mit der Option \Othree

\lstinputlisting[caption=\Optimizing GCC 4.4.1,style=customasmx86]{patterns/15_structs/6_bitfields/cpuid/CPUID_gcc_O3.asm}
Fast das gleiche. Das einzig Bemerkenswerte ist, dass GCC die Berechnung von \TT{extended\_model\_id} und
\TT{extended\_family\_id} in einem Block kombiniert, anstatt sie vor jedem Aufruf von \printf getrennt zu berechnen.
