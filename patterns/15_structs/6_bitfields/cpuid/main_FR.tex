\subsubsection{Exemple CPUID}

Le langage \CCpp permet de d�finir pr�cis�ment le nombre de bits occup�s par chaque champ d'une structure. 
Ceci est tr�s utile lorsque l'on cherche � �conomise de la place. Par exemple, chaque bit permet de 
repr�senter une variable \Tbool. Bien entendu, c'est au d�triment de la vitesse d'ex�cution.
% FIXME!
% another use of this is to parse binary protocols/packets, for example
% the definition of struct iphdr in include/linux/ip.h

\newcommand{\FNCPUID}{\footnote{\href{http://go.yurichev.com/17069}{wikipedia}}}

\myindex{x86!\Instructions!CPUID}
\label{cpuid}

Prenons par exemple l'instruction \CPUID\FNCPUID. Elle retroune des informations au sujet de la CPU qui 
ex�cute le programme et de ses capacit�s.

Si le registre \EAX est positionn� � la valeur 1 avant d'invoquer cette instruction, \CPUID va retourn� 
les informations suivantes dans le registre \EAX:

\begin{center}
\begin{tabular}{ | l | l | }
\hline
3:0 (4 bits)& Stepping \\
7:4 (4 bits) & Mod�le \\
11:8 (4 bits) & Famille \\
13:12 (2 bits) & Type de processeur \\
19:16 (4 bits) & Sous-mod�le \\
27:20 (8 bits) & Sous-famille \\
\hline
\end{tabular}
\end{center}

\newcommand{\FNGCCAS}{\footnote{\href{http://go.yurichev.com/17070}
{Compl�ment sur le fonctionnement interne de l'assembleur GCC}}}

MSVC 2010 fourni une macro \CPUID, qui est absente de GCC 4.4.1. Tentons donc de r�diger nous m�me cette 
fonction pour une utilisation dans GCC gr�ce � l'assembleur\FNGCCAS int�gr� � ce compilateur.

\lstinputlisting[style=customc]{patterns/15_structs/6_bitfields/cpuid/CPUID.c}

Apr�s que l'instruction \CPUID ait rempli les registres \EAX/\EBX/\ECX/\EDX, ceux-ci doivent �tre recopi�s 
dans le tableau \TT{b[]}. Nous affectons dont le pointeur de structure \TT{CPUID\_1\_EAX} pour qu'il 
contienne l'adresse du tableau \TT{b[]}.

En d'autres termes, nous traitons une valeur \Tint comme une structure, puis nous lisons des bits sp�cifiques 
de la structure.

\myparagraph{MSVC}

Compilons notre ewemple avec MSVC 2008 en utilisant l'option \Ox:

\lstinputlisting[caption=\Optimizing MSVC 2008,style=customasmx86]{patterns/15_structs/6_bitfields/cpuid/CPUID_msvc_Ox.asm}

\myindex{x86!\Instructions!SHR}

L'instruction \TT{SHR} va d�caler la valeur du registre \EAX d'un certain nombre de bits qui vont �tre 
abandonn�es. Nous ignorons donc certains des bits de la partie droite.

\myindex{x86!\Instructions!AND}

L'instruction \AND "efface" les bits inutiles sur la gauche, ou en d'autres termes, ne laisse dans le 
registre \EAX que les bits qui nous int�ressent.

\clearpage
\myparagraph{MSVC + \olly}
\myindex{\olly}

Chargeons notre exemple dans \olly et voyons quelles valeurs sont présentes dans EAX/EBX/ECX/EDX après 
exécution de l'instruction CPUID: 

\begin{figure}[H]
\centering
\myincludegraphics{patterns/15_structs/6_bitfields/cpuid/olly.png}
\caption{\olly: Après exécution de CPUID}
\label{fig:cpuid_olly_1}
\end{figure}

La valeur de EAX est \TT{0x000206A7} (ma \ac{CPU} est un Intel Xeon E3-1220).\\
Cette valeur exprimée en binaire vaut $0b0000 0000 0000 0010 0000 0110 1010 0111$.

Voici la manière dont les bits sont répartis sur les différents champs:

\begin{center}
\begin{tabular}{ | l | l | l | }
\hline
\headercolor{} champ &
\headercolor{} format binaire &
\headercolor{} format décimal \\
\hline
reserved2		& 0000 & 0 \\
\hline
extended\_family\_id	& 00000000 & 0 \\
\hline
extended\_model\_id	& 0010 & 2 \\
\hline
reserved1		& 00 & 0 \\
\hline
processor\_id		& 00 & 0 \\
\hline
family\_id		& 0110 & 6 \\
\hline
model			& 1010 & 10 \\
\hline
stepping		& 0111 & 7 \\
\hline
\end{tabular}
\end{center}

\lstinputlisting[caption=Console output]{patterns/15_structs/6_bitfields/cpuid/console.txt}


\myparagraph{GCC}

Essayons maintenant une compilation avec GCC 4.4.1 en utilisant l'option \Othree.

\lstinputlisting[caption=\Optimizing GCC 4.4.1,style=customasmx86]{patterns/15_structs/6_bitfields/cpuid/CPUID_gcc_O3.asm}

Le r�sultat est quasiment identique.
Le seul �l�ment notable est que GCC combine en quelques sortes le calcul de \TT{extended\_model\_id} et 
\TT{extended\_family\_id} en un seul bloc au lieu de les calculer s�par�ment avant chaque appel � \printf.
