\clearpage
\myparagraph{MSVC + \olly}
\myindex{\olly}

Chargeons notre exemple dans \olly et voyons quelles valeurs sont présentes dans EAX/EBX/ECX/EDX après 
exécution de l'instruction CPUID: 

\begin{figure}[H]
\centering
\myincludegraphics{patterns/15_structs/6_bitfields/cpuid/olly.png}
\caption{\olly: Après exécution de CPUID}
\label{fig:cpuid_olly_1}
\end{figure}

La valeur de EAX est \TT{0x000206A7} (ma \ac{CPU} est un Intel Xeon E3-1220).\\
Cette valeur exprimée en binaire vaut $0b0000 0000 0000 0010 0000 0110 1010 0111$.

Voici la manière dont les bits sont répartis sur les différents champs:

\begin{center}
\begin{tabular}{ | l | l | l | }
\hline
\headercolor{} champ &
\headercolor{} format binaire &
\headercolor{} format décimal \\
\hline
reserved2		& 0000 & 0 \\
\hline
extended\_family\_id	& 00000000 & 0 \\
\hline
extended\_model\_id	& 0010 & 2 \\
\hline
reserved1		& 00 & 0 \\
\hline
processor\_id		& 00 & 0 \\
\hline
family\_id		& 0110 & 6 \\
\hline
model			& 1010 & 10 \\
\hline
stepping		& 0111 & 7 \\
\hline
\end{tabular}
\end{center}

\lstinputlisting[caption=Console output]{patterns/15_structs/6_bitfields/cpuid/console.txt}
