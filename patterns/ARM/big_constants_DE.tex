\subsection{Laden einer Konstante in ein Register}
\label{ARM_big_constants}

\subsubsection{32-Bit ARM}
\label{ARM_big_constants_loading}
Wie wir bereits wissen, haben alle Befehle im ARM mode eine Länge von 4 Byte bzw. 2 Byte im Thumb mode.

Wie können wir nun einen 32-Bit-Wert in ein Register laden, wenn es nicht möglich ist, ihn in einen Befehl
hineinzukodieren?

Versuchen wir es:

\begin{lstlisting}[style=customc]
unsigned int f()
{
	return 0x12345678;
};
\end{lstlisting}

\begin{lstlisting}[caption=GCC 4.6.3 -O3 \ARMMode,style=customasmARM]
f:
        ldr     r0, .L2
        bx      lr
.L2:
        .word   305419896 ; 0x12345678
\end{lstlisting}
Der Wert \TT{0x12345678} wird im Speicher geparkt und wenn nötig geladen.
Es ist aber möglich, den zusätzlichen Speicherzugriff loszuwerden.

\begin{lstlisting}[caption=GCC 4.6.3 -O3 -march{=}armv7-a (\ARMMode),style=customasmARM]
movw    r0, #22136      ; 0x5678
movt    r0, #4660       ; 0x1234
bx      lr
\end{lstlisting}
Wir sehen, dass der Wert stückweise in das Register geladen wird: zuerst der niedere Teil (mit \INS{MOVW}), dann der
höhere (mit \INS{MOVT}).

Das bedeutet, dass im ARM mode 2 Befehle nötig sind, um einen 32-Bit-Wert in ein Register zu laden.

Das stellt kein wirkliches Problem dar, denn es gibt in realen Code nicht allzu viele Konstanten (außer 0 und 1).

Hat dies auch zur Folge, dass die Version mit zwei Befehlen langsamer ist als die mit einem?

Eher nicht. Wahrscheinlicher ist, dass moderne ARM Prozessoren in der Lage sind, solche Sequenzen zu erkennen und
schnell auszuführen. 

Auch \IDA vermag solche Muster im Code zu erkennen und disassembliert die Funktion wie folgt:

\begin{lstlisting}[style=customasmARM]
MOV    R0, 0x12345678
BX     LR
\end{lstlisting}

\subsubsection{ARM64}

\begin{lstlisting}[style=customc]
uint64_t f()
{
	return 0x12345678ABCDEF01;
};
\end{lstlisting}

\begin{lstlisting}[caption=GCC 4.9.1 -O3,style=customasmARM]
mov	x0, 61185   ; 0xef01
movk	x0, 0xabcd, lsl 16
movk	x0, 0x5678, lsl 32
movk	x0, 0x1234, lsl 48
ret
\end{lstlisting}

\myindex{ARM!\Instructions!MOVK}
\TT{MOVK} steht für \q{MOV Keep}, d.h. der Befehl schreibt einen 16-Bit-Wert in das Register und lässt die übrigen Bits
unverändert.
\myindex{ARM!Optional operators!LSL}
Der Suffix \TT{LSL} verschiebt den Wert um 16, 32 und 48 Bits in jedem Schritt nach links. Das Verschieben wird vor dem
Laden durchgeführt.

Das bedeutet, dass 4 Befehle notwendig sind, um einen 64-Bit-Wert in ein Register zu laden.

\myparagraph{Fließkommazahl in einem Register speichern}
Es ist möglich mit nur einem Befehl eine Fließkommazahl in einem D-Register zu speichern.

Ein Beispiel:

\begin{lstlisting}[style=customc]
double a()
{
	return 1.5;
};
\end{lstlisting}

\begin{lstlisting}[caption=GCC 4.9.1 -O3 + objdump,style=customasmARM]
0000000000000000 <a>:
   0:   1e6f1000        fmov    d0, #1.500000000000000000e+000
   4:   d65f03c0        ret
\end{lstlisting}
Die Zahl $1.5$ war tatsächlich in dem 32-Bit-Befehl kodiert. Aber wie ist das möglich?
\myindex{ARM!\Instructions!FMOV}
In ARM64 stehen 8 Bits im \TT{FMOV} Befehl für das Kodieren von Fließkommazahlen zur Verfügung.
Der Algorithmus heißt \TT{VFPExpandImm()} in \ARMSixFourRefURL. 


\myindex{minifloat}
Dieses Vorgehen wird auch \IT{minifloat} genannt\footnote{\href{http://go.yurichev.com/17139}{wikipedia}}.

Wir können verschiedene Werte ausprobieren: der Compiler ist in der Lage $30.0$ und $31.0$ zu kodieren, aber nicht
$32.0$, da hier 8 Bytes für diese Zahl im IEEE 754 Format reserviert werden müssen.

\begin{lstlisting}[style=customc]
double a()
{
	return 32;
};
\end{lstlisting}

\begin{lstlisting}[caption=GCC 4.9.1 -O3,style=customasmARM]
a:
	ldr	d0, .LC0
	ret
.LC0:
	.word	0
	.word	1077936128
\end{lstlisting}
