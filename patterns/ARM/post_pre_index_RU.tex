\subsection{Режимы адресации}
\myindex{ARM!Режимы адресации}
\label{ARM_postindex_vs_preindex}
\myindex{\CLanguageElements!\PostIncrement}
\myindex{\CLanguageElements!\PostDecrement}
\myindex{\CLanguageElements!\PreIncrement}
\myindex{\CLanguageElements!\PreDecrement}

В ARM64 возможна такая инструкция:

\myindex{ARM!\Instructions!LDR}
\begin{lstlisting}[style=customasmARM]
ldr	x0, [x29,24]
\end{lstlisting}

И это означает прибавить 24 к значению в X29 и загрузить значение по этому адресу.
Обратите внимание что 24 внутри скобок.

А если снаружи скобок, то весь смысл меняется:

\begin{lstlisting}[style=customasmARM]
ldr	w4, [x1],28
\end{lstlisting}

Это означает, загрузить значение по адресу в X1, затем прибавить 28 к X1.

\myindex{PDP-11}
ARM позволяет прибавлять некоторую константу к адресу, с которого происходит загрузка, либо вычитать.

Причем, позволяет это делать до загрузки или после.

Такого режима адресации в x86 нет, но он есть в некоторых других процессорах, даже на PDP-11.

Существует байка, что режимы пре-инкремента, пост-инкремента, 
пре-декремента и пост-декремента адреса в PDP-11,
были \q{виновны} в появлении таких конструкций языка Си (который разрабатывался на PDP-11) как
*ptr++, *++ptr, *ptr-{}-, *-{}-ptr. 
Кстати, это является трудно запоминаемой особенностью в Си.

Дела обстоят так:

% FIXME: add ARM assembly...
\small
\begin{center}
\begin{tabular}{ | l | l | l | l | }
\hline
\headercolor{} термин в Си & 
\headercolor{} термин в ARM & 
\headercolor{} выражение Си & 
\headercolor{} как это работает \\
\hline
\PostIncrement & 
post-indexed addressing & 
\TT{*ptr++} & 
использовать значение \TT{*ptr}, \\
& & & затем инкремент \\
& & & указателя \TT{ptr} \\
\hline
\PostDecrement & 
post-indexed addressing & 
\TT{*ptr-{}-} & 
использовать значение \TT{*ptr}, \\
& & & затем \glslink{decrement}{декремент} \\
& & & указателя \TT{ptr} \\
\hline
\PreIncrement & 
pre-indexed addressing & 
\TT{*++ptr} & 
инкремент указателя \TT{ptr}, \\
& & & затем использовать \\
& & & значение \TT{*ptr} \\
\hline
\PreDecrement & 
pre-indexed addressing & 
\TT{*-{}-ptr} & 
\glslink{decrement}{декремент} указателя \TT{ptr}, \\
& & & затем использовать \\
& & & значение \TT{*ptr} \\
\hline
\end{tabular}
\end{center}
\normalsize

Pre-indexing маркируется как 
восклицательный знак в ассемблере ARM.
Для примера, смотрите строку 2 в \lstref{hw_ARM64_GCC}.

Деннис Ритчи (один из создателей ЯП Си) указывал, что, это, вероятно, придумал Кен Томпсон 
(еще один создатель Си),
потому что подобная возможность процессора имелась еще в PDP-7
\footnote{\url{http://yurichev.com/mirrors/C/c_dmr_postincrement.txt}}, \RitchieDevC{}.
Таким образом, компиляторы с ЯП Си на тот процессор, где это есть, могут использовать это.

Всё это очень удобно для работы с массивами.

