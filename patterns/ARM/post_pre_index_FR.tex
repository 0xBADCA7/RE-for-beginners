\subsection{Modes d'adressage}
\myindex{ARM!Addressing modes}
\label{ARM_postindex_vs_preindex}
\myindex{\CLanguageElements!\PostIncrement}
\myindex{\CLanguageElements!\PostDecrement}
\myindex{\CLanguageElements!\PreIncrement}
\myindex{\CLanguageElements!\PreDecrement}

Cette instruction est possible en ARM64:

\myindex{ARM!\Instructions!LDR}
\begin{lstlisting}[style=customasmARM]
ldr	x0, [x29,24]
\end{lstlisting}

Ceci signifie ajouter 24 à la valeur dans X29 et charger la valeur à cette adresse.

Notez s'il vous plait que 24 est à l'intérieur des parenthèses.
La signification est différente si le nombre est à l'extérieur des parenthèses:

\begin{lstlisting}[style=customasmARM]
ldr	w4, [x1],28
\end{lstlisting}

Ceci signifie charger la valeur à l'adresse dans X1, puis ajouter 28 à X1.

\myindex{PDP-11}

ARM permet d'ajouter ou de soustraire une constante à/de l'adresse utilisée pour charger.

Et il est possible de faire cela à la fois avant et après le chargement.

Il n'y a pas de tels modes d'adressage en x86, mais ils sont présents dans d'autres
processeurs, même sur le PDP-11.

Il y a une légende disant que les modes pré-incrémentation, post-incrémentation,
pré-décrémentation et post-décrémentation du PDP-11, sont \q{responsables} de l'apparition
du genre de constructions en langage C (qui a été développé sur PDP-11) comme
*ptr++, *++ptr, *ptr-{}-, *-{}-ptr. 

Á propos, ce sont des caractéristiques de C difficiles à mémoriser.
Voici comment ça se passe:

\small
% FIXME: add ARM assembly...
\begin{center}
\begin{tabular}{ | l | l | l | l | }
\hline
\headercolor{} terme C &
\headercolor{} terme ARM &
\headercolor{} déclaration C &
\headercolor{} ce que ça fait \\
\hline
\PostIncrement &
adressage post-indexé &
\TT{*ptr++} &
utiliser la valeur \TT{*ptr}, \\
& & & puis \glslink{increment}{incrémenter} \\
& & & le pointeur \TT{ptr} \\
\hline
\PostDecrement &
adressage post-indexé&
\TT{*ptr-{}-} &
utiliser la valeur \TT{*ptr}, \\
& & & puis \glslink{decrement}{décrémenter} \\
& & & le pointeur \TT{ptr} \\
\hline
\PreIncrement &
adressage pré-indexé &
\TT{*++ptr} &
\glslink{increment}{incrémenter} le pointeur \TT{ptr}, \\
& & & puis utiliser \\
& & & la valeur \TT{*ptr} \\
\hline
\PreDecrement &
adressage pré-indexé &
\TT{*-{}-ptr} &
\glslink{decrement}{décrémenter} le pointeur \TT{ptr}, \\
& & & puis utiliser \\
& & & la valeur \TT{*ptr} \\
\hline
\end{tabular}
\end{center}
\normalsize

La pré-indexation est marquée avec un point d'exclamation en langage d'assemblage
ARM.
Par exemple, voir ligne 2 dans \lstref{hw_ARM64_GCC}.

Dennis Ritchie (un des créateurs du langage C) a mentionné que cela a vraisemblablement
été inventé par Ken Thompson (un autre créateur du C) car cette possibilité était
présente sur le PDP-7
\footnote{\url{http://yurichev.com/mirrors/C/c_dmr_postincrement.txt}}, \RitchieDevC{}.

Ainsi, les compilateurs de langage C peuvent l'utiliser, si elle est présente sur
le processeur cible.

C'est très pratique pour le traitement de tableau.
