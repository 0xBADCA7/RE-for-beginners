\subsubsection{ARM}

\myparagraph{ARM: \OptimizingKeilVI (\ThumbMode)}

\lstinputlisting[caption=\OptimizingKeilVI (\ThumbMode)style=customasmARM]{patterns/04_scanf/3_checking_retval/ex3_ARM_Keil_thumb_O3.asm}

\myindex{ARM!\Instructions!CMP}
\myindex{ARM!\Instructions!BEQ}
Здесь для нас есть новые инструкции: \CMP и \ac{BEQ}.

\CMP аналогична той что в x86: она отнимает один аргумент от второго и сохраняет флаги.

% TODO: в мануале ARM $op1 + NOT(op2) + 1$ вместо вычитания

\myindex{ARM!\Registers!Z}
\myindex{x86!\Instructions!JZ}
\ac{BEQ} совершает переход по другому адресу, 
если операнды при сравнении были равны, 
либо если результат последнего вычисления был 0, либо если флаг Z равен 1.
То же что и \JZ в x86.

Всё остальное просто: исполнение разветвляется на две ветки, затем они сходятся там, 
где в \Reg{0} записывается 0 как возвращаемое из функции значение и происходит выход из функции.

\myparagraph{ARM64}

\lstinputlisting[caption=\NonOptimizing GCC 4.9.1 ARM64,numbers=left,style=customasmARM]{patterns/04_scanf/3_checking_retval/ARM64_GCC491_O0_RU.s}

\myindex{ARM!\Instructions!CMP}
\myindex{ARM!\Instructions!Bcc}

Исполнение здесь разветвляется, используя пару инструкций \INS{CMP}/\INS{BNE} (Branch if Not Equal: переход если не равно).

