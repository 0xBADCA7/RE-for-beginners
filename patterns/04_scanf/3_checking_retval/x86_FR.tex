\subsubsection{MSVC: x86}

Voici ce que nous obtenons dans la sortie assembleur (MSVC 2010):

\lstinputlisting[style=customasmx86]{patterns/04_scanf/3_checking_retval/ex3_MSVC_x86.asm}

\myindex{x86!\Registers!EAX}
La fonction \glslink{caller}{appelante} (\main) à besoin du résultat de la fonction
\glslink{callee}{appelée}, donc la fonction \glslink{callee}{appelée} le renvoie
dans la registre \EAX.

\myindex{x86!\Instructions!CMP}
Nous le vérifions avec l'aide de l'instruction \TT{CMP EAX, 1} (\IT{CoMPare}).
En d'autres mots, nous comparons la valeur dans le registre \EAX avec 1.

\myindex{x86!\Instructions!JNE}
Une instruction de saut conditonnelle \JNE suit l'instruction \CMP. \JNE signifie
\IT{Jump if Not Equal} (saut si non égal).

Donc, si la valeur dans le registre \EAX n'est pas égale à 1, le \ac{CPU} va poursuivre
l'exécution à l'adresse mentionnée dans l'opérande \JNE, dans notre cas \TT{\$LN2@main}.
Passez le contrôle à cette adresse résulte en l'exécution par le \ac{CPU} de
\printf avec l'argument \TT{What you entered? Huh?}.
Mais si tout est bon, le saut conditionnel n'est pas pris, et un autre appel à \printf
est exécuté, avec deux arguments:\\
\TT{'You entered \%d...'} et la valeur de \TT{x}.

\myindex{x86!\Instructions!XOR}
\myindex{\CLanguageElements!return}
Puisque dans ce cas le second \printf n'a pas été exécuté, il y a un \JMP qui le précède (saut inconditionnel).
Il passe le contrôle au point après le second \printf et juste avant l'instruction \TT{XOR EAX, EAX}, qui implémente \TT{return 0}.

% FIXME internal \ref{} to x86 flags instead of wikipedia
\myindex{x86!\Registers!\Flags}
Donc, on peut dire que comparer une valeur avec une autre est \IT{usuellement} implémenté
par la paire d'instructions \CMP/\Jcc, oú \IT{cc} est un \IT{code de condition}.
\CMP compare deux valeurs et met les flags\footnote{flags x86, voir aussi: \href{http://go.yurichev.com/17120}{wikipedia}.}
du processeur.
\Jcc vérifie ces flags et décide de passer le contrôle à l'adresse spécifiée ou non.

\myindex{x86!\Instructions!CMP}
\myindex{x86!\Instructions!SUB}
\myindex{x86!\Instructions!JNE}
\myindex{x86!\Registers!ZF}
\label{CMPandSUB} 
Cela peut sembler paradoxal, mais l'instruction \CMP est en fait un \SUB (soustraction).
Toutes les instructions arithmétiques mettent les flags du processeur, pas seulement \CMP.
Si nous comparons 1 et 1, $1-1$ donne 0 donc le flag \ZF va être mis (signifiant
que le dernier résultat est 0).
Dans aucune autre circonstance \ZF ne sera mis, à l'exception que les opérandes
ne soient égales.
\JNE vérifie seulement le flag \ZF et saute seulement si il n'est pas mis. \JNE
est un synonyme pour \JNZ (\IT{Jump if Not Zero} (saut si non zéro)).
L'assembleur génère le même opcode pour les instructions \JNE et \JNZ.
Donc, l'instruction \CMP peut être remplacée par une instruction \SUB et presque
tout ira bien, à la différence que \SUB altère la valeur de la première opérande.
\CMP est un \IT{SUB sans sauver le résultat, mais modifiant les flags}.

\subsubsection{MSVC: x86: IDA}

\myindex{IDA}
C'est le moment de lancer \IDA et d'essayer de faire quelque chose avec.
Á propos, pour les débutants, c'est une bonne idée d'utiliser l'option \TT{/MD}
de MSVC, qui signifie que toutes les fonctions standards ne vont pas être liées
avec le fichier exécutable, mais vont à la place être importées depuis le fichier
\TT{MSVCR*.DLL}.
Ainsi il est plus facile de voir quelles fonctions standards sont utilisées et oú.

En analysant du code dans \IDA, il est très utile de laisser des notes pour soi-même
(et les autres).
En la circonstance, analysons cet exemple, nous voyons que \TT{JNZ} sera déclenché
en cas d'erreur.
Donc il est possible de déplacer le curseur sur le label, de presser \q{n} et de
lui donner le nom \q{error}.
Créons un autre label---dans \q{exit}.
Voici mon résultat:

\lstinputlisting[style=customasmx86]{patterns/04_scanf/3_checking_retval/ex3.lst}

Maintenant, il est légèrement plus facile de comprendre le code.
Toutefois, ce n'est pas une bonne idée de commenter chaque instruction.

% FIXME draw button?
Vous pouvez aussi cacher(replier) des parties d'une fonction dans \IDA.
Pour faire cela, marquez le bloc, puis appueaz sur \q{--} sur le pavé numérique et
entrez le texte qui doit être affiché à la place.

Cachons deux blocs et donnons leurs un nom:

\lstinputlisting[style=customasmx86]{patterns/04_scanf/3_checking_retval/ex3_2.lst}

% FIXME draw button?
Pour étendre les parties de code précédemment cachées. utilisez \q{+} sur le
pavé numérique.

\clearpage
En appuyant sur \q{space}, nous voyons comment \IDA représente une fonction sous
forme de graphe:

\begin{figure}[H]
\centering
\myincludegraphics{patterns/04_scanf/3_checking_retval/IDA.png}
\caption{IDA en mode graphe}
\label{fig:ex3_IDA_1}
\end{figure}

Il y a deux flèches après chaque saut conditionnel: une verte et une rouge.
La flèche verte pointe vers le bloc qui sera exécuté si le saut est déclenché,
et la rouge sinon.

\clearpage
Il est possible de replier des noeuds dans ce mode et de leurs donner aussi un nom (\q{group nodes}).
Essayons avec 3 blocs:

\begin{figure}[H]
\centering
\myincludegraphics{patterns/04_scanf/3_checking_retval/IDA2.png}
\caption{IDA en mode graphe avec 3 noeuds repliés}
\label{fig:ex3_IDA_2}
\end{figure}

C'est très pratique.
%%It could be said that a very important part of the reverse engineers' job (and any other researcher as well) is to reduce the amount of information they deal with.
On peut dire qu'une part importante du travail des rétro-ingénieurs (et de tout
autre chercheur également) est de réduire la quantité d'information avec laquelle
travailler.

\clearpage
\subsubsection{MSVC: x86 + \olly}

Essayons de hacker notre programme dans \olly, pour le forcer à penser que \scanf
fonctionne toujours sans erreur.
Lorsque l'adresse d'une variable locale est passée à \scanf, la variable contient
initiallement toujours des restes de données aléatoires, dans ce cas \TT{0x6E494714}:

\begin{figure}[H]
\centering
\myincludegraphics{patterns/04_scanf/3_checking_retval/olly_1.png}
\caption{\olly: passer l'adressed de la variable à \scanf}
\label{fig:scanf_ex3_olly_1}
\end{figure}

\clearpage
Lorsque \scanf s'exécute dans la console, entrons quelque chose qui n'est pas du
tout un nombre, comme \q{asdasd}.
\scanf termine avec 0 dans \EAX, ce qui indique qu'une erreur s'est produite:

\begin{figure}[H]
\centering
\myincludegraphics{patterns/04_scanf/3_checking_retval/olly_2.png}
\caption{\olly: \scanf renvoyant une erreur}
\label{fig:scanf_ex3_olly_2}
\end{figure}

Nous pouvons vérifier la variable locale dans le pile et noter qu'elle n'a pas changé.
En effet, qu'aurait écrit \scanf ici?
Elle n'a simplement rien fait à part renvoyer zéro.

Essayons de \q{hacker} notre programme.
Clique-droit sur \EAX,
Parmis les options il y a \q{Set to 1} (mettre à 1).
C'est ce dont nous avons besoin.

Nous avons maintenant 1 dans \EAX, donc la vérification suivante va s'exécuter comme
souhaiter et \printf va afficher la valeur de la variable dans la pile.

Lorsque nous lançons le programme (F9) nous pouvons voir ceci dans la fenêtre
de la console:

\lstinputlisting[caption=fenêtre console]{patterns/04_scanf/3_checking_retval/console.txt}

En effet, 1850296084 est la représentation en décimal du nombre dans la pile (\TT{0x6E494714})!


\clearpage
\subsubsection{MSVC: x86 + Hiew}
\myindex{Hiew}

Cela peut également être utilisé comme un exemple simple de modification de fichier
exécutable.
Nous pouvons essayer de modifier l'exécutable de telle sorte que le programme va
toujours afficher notre entrée, quelle quelle soit.

En supposant que l'exécutable est compilé avec la bibliothèque externe \TT{MSVCR*.DLL}
(i.e., avec l'option \TT{/MD}) \footnote{c'est aussi appelé \q{dynamic linking}},
nous voyons la fonction \main au début de la section \TT{.text}.
Ouvrons l'exécutable dans Hiew et cherchons le début de la section \TT{.text} (Enter,
F8, F6, Enter, Enter).

Nous pouvons voir cela:

\begin{figure}[H]
\centering
\myincludegraphics{patterns/04_scanf/3_checking_retval/hiew_1.png}
\caption{Hiew: fonction \main}
\label{fig:scanf_ex3_hiew_1}
\end{figure}

Hiew trouve les chaîne \ac{ASCIIZ} et les affiche, comme il le fait avec le nom
des fonctions importées.

\clearpage
Déplacez le curseur à l'adresse \TT{.00401027} (oú se trouve l'instruction \TT{JNZ},
que l'on doit sauter), appuyez sur F3, et ensuite tapez \q{9090} (qui signifie deux
\ac{NOP}s):

\begin{figure}[H]
\centering
\myincludegraphics{patterns/04_scanf/3_checking_retval/hiew_2.png}
\caption{Hiew: remplacement de \TT{JNZ} par deux \ac{NOP}s}
\label{fig:scanf_ex3_hiew_2}
\end{figure}

Appuyez sur F9 (update). Maintenant, l'exécutable est sauvé sur le disque. Il va
se comporter comme nous le voulions.

Deux \ac{NOP}s ne constitue probablement pas l'approche la plus \ae{}sthétique.
Une autre façon de modifier cette instruction est d'écrire simplement 0 dans le
second octet de l'opcode ((\gls{jump offset}), donc ce \TT{JNZ} va toujours sauter
à l'instruction suivante.

Nous pouvons également faire le contraire: remplacer le premier octet avec \TT{EB}
sans modifier le second octet (\gls{jump offset}).
Nous obtiendrions un saut inconditionnel qui est toujours déclenché.
Dans ce cas le message d'erreur sera affiché à chaque fois, peu importe l'entrée.

