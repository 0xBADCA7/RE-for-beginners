\subsubsection{ARM}

\myparagraph{ARM: \OptimizingKeilVI (\ThumbMode)}

\lstinputlisting[caption=\OptimizingKeilVI (\ThumbMode),style=customasmARM]{patterns/04_scanf/3_checking_retval/ex3_ARM_Keil_thumb_O3.asm}

\myindex{ARM!\Instructions!CMP}
\myindex{ARM!\Instructions!BEQ}

Les nouvelles instructions sont \CMP et \ac{BEQ}.

\CMP est similaire à l'instruction x86 du même nom, elle soustrait l'un des arguments
à l'autre et met à jour les flags si nécessaire.

\myindex{ARM!\Registers!Z}
\myindex{x86!\Instructions!JZ}
\ac{BEQ} saute à une autre adresse si les opérandes étaient égales l'une à l'autre,
ou, si le résultat du dernier calcul était 0, ou si le flag Z est à 1.
Elle se comporte comme \JZ en x86.

Tout le reste est simple: le flux d'exécution se sépare en deux branches, puis les
branches convergent vers le point où 0 est écrit dans le registre \Reg{0} comme
valeur de retour de la fonction, et cette dernière se termine.

\myparagraph{ARM64}

\lstinputlisting[caption=GCC 4.9.1 ARM64 \NonOptimizing,numbers=left,style=customasmARM]{patterns/04_scanf/3_checking_retval/ARM64_GCC491_O0_FR.s}

\myindex{ARM!\Instructions!CMP}
\myindex{ARM!\Instructions!Bcc}
Dans ce cas, le flux de code se sépare avec l'utilisation de la paire d'instructions
\INS{CMP}/\INS{BNE} (Branch if Not Equal) (branchement si non égal).

