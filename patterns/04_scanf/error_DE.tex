\subsection{H�ufiger Fehler}
Ein h�ufiger (Tipp)-Fehler besteht darin, den Wert von \IT{x} anstatt eines Pointers auf \IT{x} zu �bergeben.

\lstinputlisting[style=customc]{patterns/04_scanf/error.c}

Was geschieht hier?
\IT{x} ist nicht uninitialisiert und enth�lt Zufallswerte vom lokalen Stack. Wenn \scanf aufgerufen wird, nimmt es den eingegebenen String vom Benutzer, wandelt ihn in eine Zahl um und versucht ihn nach \IT{x} zu schreiben, wobei \IT{x} wie eine Speicheradresse behandelt wird.
Aber hier liegen Zufallswerte vor, sodass \scanf versucht an eine zuf�llige Speicherstelle zu schreiben. 
H�chstwahrscheinlich wird der Prozess dadurch abst�rzen.

\myindex{0xCCCCCCCC}
\myindex{0x0BADF00D}
Bemerkenswert ist, dass manche \ac{CRT}-Bibliotheken im Debug Build gut erkennbare Muster in den gerade reservierten Speicher schreiben, wie z.B. 0xCCCCCCCC oder 0x0BADF00D usw.
In diesem Fall k�nnte \IT{x} den Wert 0xCCCCCCCC enthalten und \scanf w�rde versuchen in die Adresse 0xCCCCCCCC zu schreiben. 
Wenn man nun bemerkt, dass irgendein Code im Prozess in die Adresse 0xCCCCCCCC schreiben m�chte, wei� man, dass eine uninitialisierte Variable (oder ein Pointer) verwendet werden.
Dies ist besser als wenn der frisch reservierte Speicher einfach gel�scht w�rde.


