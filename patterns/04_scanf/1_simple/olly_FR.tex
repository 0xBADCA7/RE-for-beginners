\clearpage
\subsubsection{MSVC + \olly}
\myindex{\olly}

% TODO look in French olly for text translation, if exists?
Essayons cet exemple dans \olly.
Chargeons-le et appuyons sur F8 (\stepover) jusqu'à ce que nous atteignons notre
exécutable au lieu de \TT{ntdll.dll}.
Défiler vers le haut jusqu'à ce que \main apparaisse.

Cliquer sur la première instruction  (\TT{PUSH EBP}), appuyer sur F2 (\IT{set a
breakpoint}), puis F9 (\IT{Run}).
Le point d'arrêt sera déclenché lorsque \main commencera.

Continuons jusqu'au point où la variable $x$ est calculée:

\begin{figure}[H]
\centering
\myincludegraphics{patterns/04_scanf/1_simple/ex1_olly_1.png}
\caption{\olly: L'adresse de la variable locale est calculée}
\label{fig:scanf_ex1_olly_1}
\end{figure}

Cliquer droit sur \EAX dans la fenêtre des registres et choisir \q{Follow in stack}.

Cette adresse va apparaître dans la fenêtre de la pile.
La flèche rouge a été ajoutée, pointant la variable dans la pile locale.
A ce point, cet espace contient des restes de données (\TT{0x6E494714}).
Maintenant. avec l'aide de l'instruction \PUSH, l'adresse de cet élément de pile
va être stockée sur la même pile à la position suivante.
Appuyons sur F8 jusqu'à la fin de l'exécution de \scanf.
Pendant l'exécution de \scanf, entrons, par exemple, 123, dans la fenêtre de la console:

\lstinputlisting{patterns/04_scanf/1_simple/console.txt}

\clearpage
\scanf a déjà fini de s'exécuter:

\begin{figure}[H]
\centering
\myincludegraphics{patterns/04_scanf/1_simple/ex1_olly_3.png}
\caption{\olly: \scanf s'est exécutée}
\label{fig:scanf_ex1_olly_3}
\end{figure}

\scanf renvoie 1 dans \EAX, ce qui indique qu'elle a lu avec succès une valeur.
Si nous regardons de nouveau l'élément de la pile correspondant à la variable
locale, il contient maintenant \TT{0x7B} (123).

\clearpage

Plus tard, cette valeur est copiée de la pile vers le registre \ECX et passée à \printf:

\begin{figure}[H]
\centering
\myincludegraphics{patterns/04_scanf/1_simple/ex1_olly_4.png}
\caption{\olly: préparation de la valeur pour la passer à \printf}
\label{fig:scanf_ex1_olly_4}
\end{figure}
