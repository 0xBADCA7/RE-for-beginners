\subsubsection{x86}

\myparagraph{MSVC}

Что получаем на ассемблере, компилируя в MSVC 2010:

\lstinputlisting[style=customasm]{patterns/04_scanf/1_simple/ex1_MSVC_RU.asm}

Переменная \TT{x} является локальной.

По стандарту \CCpp она доступна только из этой же функции и нигде более. 
Так получилось, что локальные переменные располагаются в стеке. 
Может быть, можно было бы использовать и другие варианты, но в x86 это традиционно так.

\myindex{x86!\Instructions!PUSH}
Следующая после пролога инструкция \TT{PUSH ECX} не ставит своей целью сохранить 
значение регистра \ECX. 
(Заметьте отсутствие соответствующей инструкции \TT{POP ECX} в конце функции).

Она на самом деле выделяет в стеке 4 байта для хранения \TT{x} в будущем.

\label{stack_frame}
\myindex{\Stack!Стековый фрейм}
\myindex{x86!\Registers!EBP}
Доступ к \TT{x} будет осуществляться при помощи объявленного макроса \TT{\_x\$} (он равен -4) и регистра \EBP указывающего на текущий фрейм.

Во всё время исполнения функции \EBP указывает на текущий \glslink{stack frame}{фрейм} и через \TT{EBP+смещение}
можно получить доступ как к локальным переменным функции, так и аргументам функции.

\myindex{x86!\Registers!ESP}
Можно было бы использовать \ESP, но он во время исполнения функции часто меняется, а это не удобно. 
Так что можно сказать, что \EBP это \IT{замороженное состояние} \ESP на момент начала исполнения функции.

% FIXME1 это уже было в 02_stack?
Разметка типичного стекового \glslink{stack frame}{фрейма} в 32-битной среде:

\begin{center}
\begin{tabular}{ | l | l | }
\hline
\dots & \dots \\
\hline
EBP-8 & локальная переменная \#2, \MarkedInIDAAs{} \TT{var\_8} \\
\hline
EBP-4 & локальная переменная \#1, \MarkedInIDAAs{} \TT{var\_4} \\
\hline
EBP & сохраненное значение \EBP \\
\hline
EBP+4 & адрес возврата \\
\hline
EBP+8 & \argument \#1, \MarkedInIDAAs{} \TT{arg\_0} \\
\hline
EBP+0xC & \argument \#2, \MarkedInIDAAs{} \TT{arg\_4} \\
\hline
EBP+0x10 & \argument \#3, \MarkedInIDAAs{} \TT{arg\_8} \\
\hline
\dots & \dots \\
\hline
\end{tabular}
\end{center}

У функции \scanf в нашем примере два аргумента.

Первый~--- указатель на строку, содержащую \TT{\%d} и второй~--- адрес переменной \TT{x}.

\myindex{x86!\Instructions!LEA}
Вначале адрес \TT{x} помещается в регистр \EAX при помощи инструкции \TT{lea eax, DWORD PTR \_x\$[ebp]}.

Инструкция \LEA означает \IT{load effective address}, и часто используется для формирования адреса чего-либо ~(\myref{sec:LEA}).

Можно сказать, что в данном случае \LEA просто помещает в \EAX результат суммы значения в регистре \EBP и макроса \TT{\_x\$}.

Это тоже что и \INS{lea eax, [ebp-4]}.

Итак, от значения \EBP отнимается 4 и помещается в \EAX.
Далее значение \EAX заталкивается в стек и вызывается \scanf.

После этого вызывается \printf. Первый аргумент вызова строка:
\TT{You entered \%d...\textbackslash{}n}.

Второй аргумент: \INS{mov ecx, [ebp-4]}.
Эта инструкция помещает в \ECX не адрес переменной \TT{x}, а её значение.

Далее значение \ECX заталкивается в стек и вызывается \printf.

\EN{\clearpage
\mysubparagraph{\olly}
\myindex{\olly}

Let's try this example in \olly.
The input value of the function (2) is loaded into \EAX: 

\begin{figure}[H]
\centering
\myincludegraphics{patterns/08_switch/2_lot/olly1.png}
\caption{\olly: function's input value is loaded in \EAX}
\label{fig:switch_lot_olly1}
\end{figure}

\clearpage
The input value is checked, is it bigger than 4? 
If not, the \q{default} jump is not taken:
\begin{figure}[H]
\centering
\myincludegraphics{patterns/08_switch/2_lot/olly2.png}
\caption{\olly: 2 is no bigger than 4: no jump is taken}
\label{fig:switch_lot_olly2}
\end{figure}

\clearpage
Here we see a jumptable:

\begin{figure}[H]
\centering
\myincludegraphics{patterns/08_switch/2_lot/olly3.png}
\caption{\olly: calculating destination address using jumptable}
\label{fig:switch_lot_olly3}
\end{figure}

Here we've clicked \q{Follow in Dump} $\rightarrow$ \q{Address constant}, so now we see the \IT{jumptable} in the data window.
These are 5 32-bit values\footnote{They are underlined by \olly because
these are also FIXUPs: \myref{subsec:relocs}, we are going to come back to them later}.
\ECX is now 2, so the third element (can be indexed as 2\footnote{About indexing, see also: \ref{arrays_at_one}}) of the table is to be used.
It's also possible to click \q{Follow in Dump} $\rightarrow$ 
\q{Memory address} and \olly will show the element addressed by the \JMP instruction. 
That's \TT{0x010B103A}.

\clearpage
After the jump we are at \TT{0x010B103A}: the code printing \q{two} will now be executed:

\begin{figure}[H]
\centering
\myincludegraphics{patterns/08_switch/2_lot/olly4.png}
\caption{\olly: now we at the \IT{case:} label}
\label{fig:switch_lot_olly4}
\end{figure}
}
\RU{\clearpage
\mysubparagraph{\olly}
\myindex{\olly}

Попробуем этот пример в \olly.
Входное значение функции (2) загружается в \EAX: 

\begin{figure}[H]
\centering
\myincludegraphics{patterns/08_switch/2_lot/olly1.png}
\caption{\olly: входное значение функции загружено в \EAX}
\label{fig:switch_lot_olly1}
\end{figure}

\clearpage
Входное значение проверяется, не больше ли оно чем 4? 
Нет, переход по умолчанию (\q{default}) не будет исполнен:

\begin{figure}[H]
\centering
\myincludegraphics{patterns/08_switch/2_lot/olly2.png}
\caption{\olly: 2 не больше чем 4: переход не сработает}
\label{fig:switch_lot_olly2}
\end{figure}

\clearpage
Здесь мы видим jumptable:

\begin{figure}[H]
\centering
\myincludegraphics{patterns/08_switch/2_lot/olly3.png}
\caption{\olly: вычисляем адрес для перехода используя jumptable}
\label{fig:switch_lot_olly3}
\end{figure}

Кстати, щелкнем по \q{Follow in Dump} $\rightarrow$ \q{Address constant}, так что теперь \IT{jumptable} видна в окне данных.

Это 5 32-битных значений\footnote{Они подчеркнуты в \olly, потому что это также и FIXUP-ы: \myref{subsec:relocs}, мы вернемся к ним позже}.
\ECX сейчас содержит 2, так что третий элемент (либо второй, если считать с нулевого) таблицы будет использован.
Кстати, можно также щелкнуть \q{Follow in Dump} $\rightarrow$ \q{Memory address} и \olly покажет элемент, который сейчас адресуется в инструкции \JMP. 
Это \TT{0x010B103A}.

\clearpage
Переход сработал и мы теперь на \TT{0x010B103A}: сейчас будет исполнен код, выводящий строку \q{two}:

\begin{figure}[H]
\centering
\myincludegraphics{patterns/08_switch/2_lot/olly4.png}
\caption{\olly: теперь мы на соответствующей метке \IT{case:}}
\label{fig:switch_lot_olly4}
\end{figure}
}
\ITA{\clearpage
\mysubparagraph{\olly}
\myindex{\olly}

Esaminiamo questo esempio con \olly.
Il valore di input della funzione (2) viene caricato \EAX: 

\begin{figure}[H]
\centering
\myincludegraphics{patterns/08_switch/2_lot/olly1.png}
\caption{\olly: il valore di input è caricato in \EAX}
\label{fig:switch_lot_olly1}
\end{figure}

\clearpage
Il valore viene controllato, è maggiore di 4?
Se no, il \q{default} jump non viene innescato:
\begin{figure}[H]
\centering
\myincludegraphics{patterns/08_switch/2_lot/olly2.png}
\caption{\olly: 2 non è maggiore di 4: il salto non viene fatto}
\label{fig:switch_lot_olly2}
\end{figure}

\clearpage
Qui vediamo un jumptable:

\begin{figure}[H]
\centering
\myincludegraphics{patterns/08_switch/2_lot/olly3.png}
\caption{\olly: calcolo dell'indirizzo di destinazione mediante jumptable}
\label{fig:switch_lot_olly3}
\end{figure}

Qui abbiamo cliccato \q{Follow in Dump} $\rightarrow$ \q{Address constant}, così da vedere la \IT{jumptable} nella data window.
Sono 5 valori a 32-bit \footnote{Sono sottolineati da \olly poiché
sono anche FIXUPs: \myref{subsec:relocs}, torneremo su questo argomento più avanti}.
\ECX adesso è 2, quindi il terzo elemento (avente indice 2\footnote{Per l'indicizzazione, vedi anche: \ref{arrays_at_one}}) della tabella.
E' anche possibile cliccare su \q{Follow in Dump} $\rightarrow$ 
\q{Memory address} e \olly mostrerà l'elemento a cui punta l'istruzione \JMP. 
In questo caso è \TT{0x010B103A}.

\clearpage
Dopo il salto ci troviamo a \TT{0x010B103A}: il codice che stampa \q{two} sarà ora eseguito:

\begin{figure}[H]
\centering
\myincludegraphics{patterns/08_switch/2_lot/olly4.png}
\caption{\olly: ora ci troviamo alla label \IT{case:}}
\label{fig:switch_lot_olly4}
\end{figure}
}
\DE{\clearpage
\subsubsection{MSVC + \olly}
\myindex{\olly}
Schauen wir uns diese Beispiel in \olly an.
Wir laden es und dr�cken F8 (\stepover) bis wir unsere ausf�hrbare Datei anstelle von \TT{ntdll.dll} erreicht haben. Wir scrollen nach oben bis \main erscheint.

Wir klicken auf den ersten Befehl (\TT{PUSH EBO}), dr�cken F2 (\IT{set a breakpoint}), dann F9 (\IT{Run}). Der Breakpoint wird ausgel�st, wenn die Funktion \main beginnt.

Verfolgen wir den Ablauf bis zu der Stelle, an der die Adresse der Variablen $x$ berechnet wird:

\begin{figure}[H]
\centering
\myincludegraphics{patterns/04_scanf/1_simple/ex1_olly_1.png}
\caption{\olly: Die Adresse der lokalen Variable wird berechnet.}
\label{fig:scanf_ex1_olly_1}
\end{figure}

Wir machen einen Rechtsklick auf \EAX in Registerfenster und w�hlen \q{Follow in stack}. 

Diese Adresse wird im Stackfenster erscheinen. Der rote Pfeil wurde nachtr�glich hinzugef�gt; er zeigt auf die Variable im lokalen Stack. Im Moment enth�lt diese Speicherstelle Zufallswerte (\TT{0x6E494714}). Jetzt wird mithilfe des \PUSH Befehls die Adresse dieses Stackelements auf demselben Stack an der folgenden Position gespeichert. 
Verfolgen wir den Ablauf mit F8 bis die Ausf�hrung von \scanf abgeschlossen ist. W�hrend der Ausf�hrung von \scanf geben wir beispielsweise 123 in der Konsole ein:

\lstinputlisting{patterns/04_scanf/1_simple/console.txt}

\clearpage
\scanf ist bereits beendet:

\begin{figure}[H]
\centering
\myincludegraphics{patterns/04_scanf/1_simple/ex1_olly_3.png}
\caption{\olly: \scanf wurde ausgef�hrt}
\label{fig:scanf_ex1_olly_3}
\end{figure}

\scanf liefert 1 im \EAX Register zur�ck, was aussagt, dass die Funktion einen Wert erfolgreich eingelesen hat. Wenn wir wiederum auf das zugeh�rige Stackelement f�r die lokale Variable schauen, enth�lt diese nun den Wert \TT{0x7B} (dez. 123).

\clearpage
Im weiteren Verlauf wird dieser Wert vom Stack in das \ECX Register kopiert und an \printf �bergeben:

\begin{figure}[H]
\centering
\myincludegraphics{patterns/04_scanf/1_simple/ex1_olly_4.png}
\caption{\olly: Wert f�r �bergabe an \printf vorbereiten.}
\label{fig:scanf_ex1_olly_4}
\end{figure}

}
\FR{\clearpage
\subsubsection{MSVC + \olly}
\myindex{\olly}

% TODO look in French olly for text translation, if exists?
Essayons cet exemple dans \olly.
Chargeons-le et appuyons sur F8 (\stepover) jusqu'à ce que nous atteignons notre
exécutable au lieu de \TT{ntdll.dll}.
Défiler vers le haut jusqu'à ce que \main apparaisse.

Cliquer sur la première instruction  (\TT{PUSH EBP}), appuyer sur F2 (\IT{set a
breakpoint}), puis F9 (\IT{Run}).
Le point d'arrêt sera déclenché lorsque \main commencera.

Continuons jusqu'au point où la variable $x$ est calculée:

\begin{figure}[H]
\centering
\myincludegraphics{patterns/04_scanf/1_simple/ex1_olly_1.png}
\caption{\olly: L'adresse de la variable locale est calculée}
\label{fig:scanf_ex1_olly_1}
\end{figure}

Cliquer droit sur \EAX dans la fenêtre des registres et choisir \q{Follow in stack}.

Cette adresse va apparaître dans la fenêtre de la pile.
La flèche rouge a été ajoutée, pointant la variable dans la pile locale.
A ce point, cet espace contient des restes de données (\TT{0x6E494714}).
Maintenant. avec l'aide de l'instruction \PUSH, l'adresse de cet élément de pile
va être stockée sur la même pile à la position suivante.
Appuyons sur F8 jusqu'à la fin de l'exécution de \scanf.
Pendant l'exécution de \scanf, entrons, par exemple, 123, dans la fenêtre de la console:

\lstinputlisting{patterns/04_scanf/1_simple/console.txt}

\clearpage
\scanf a déjà fini de s'exécuter:

\begin{figure}[H]
\centering
\myincludegraphics{patterns/04_scanf/1_simple/ex1_olly_3.png}
\caption{\olly: \scanf s'est exécutée}
\label{fig:scanf_ex1_olly_3}
\end{figure}

\scanf renvoie 1 dans \EAX, ce qui indique qu'elle a lu avec succès une valeur.
Si nous regardons de nouveau l'élément de la pile correspondant à la variable
locale, il contient maintenant \TT{0x7B} (123).

\clearpage

Plus tard, cette valeur est copiée de la pile vers le registre \ECX et passée à \printf:

\begin{figure}[H]
\centering
\myincludegraphics{patterns/04_scanf/1_simple/ex1_olly_4.png}
\caption{\olly: préparation de la valeur pour la passer à \printf}
\label{fig:scanf_ex1_olly_4}
\end{figure}
}


\myparagraph{GCC}

Попробуем тоже самое скомпилировать в Linux при помощи GCC 4.4.1:

\lstinputlisting[style=customasm]{patterns/04_scanf/1_simple/ex1_GCC.asm}

\myindex{puts() вместо printf()}
GCC заменил первый вызов \printf на \puts. Почему это было сделано, 
уже было описано ранее~(\myref{puts}).

% TODO: rewrite
%\RU{Почему \scanf переименовали в \TT{\_\_\_isoc99\_scanf}, я честно говоря, пока не знаю.}
%\EN{Why \scanf is renamed to \TT{\_\_\_isoc99\_scanf}, I do not know yet.}
% 
% Apparently it has to do with the ISO c99 standard compliance. By default GCC allows specifying a standard to adhere to.
% For example if you compile with -std=c89 the outputted assmebly file will contain scanf and not __isoc99__scanf. I guess current GCC version adhares to c99 by default.
% According to my understanding the two implementations differ in the set of suported modifyers (See printf man page)


Далее всё как и прежде~--- параметры заталкиваются через стек при помощи \MOV.

\myparagraph{Кстати}

Кстати, этот простой пример иллюстрирует то обстоятельство, что компилятор преобразует
список выражений в \CCpp-блоке просто в последовательный набор инструкций.
Между выражениями в \CCpp ничего нет, и в итоговом машинном коде между ними тоже ничего нет, 
управление переходит от одной инструкции к следующей за ней.

