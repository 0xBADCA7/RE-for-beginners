\subsubsection{x86}

\myparagraph{MSVC}

Here is what we get after compiling with MSVC 2010:

\lstinputlisting[style=customasmx86]{patterns/04_scanf/1_simple/ex1_MSVC_EN.asm}

\TT{x} is a local variable.

According to the \CCpp standard it must be visible only in this function and not from any other external scope. 
Traditionally, local variables are stored on the stack. 
There are probably other ways to allocate them, but in x86 that is the way it is.

\myindex{x86!\Instructions!PUSH}
The goal of the instruction following the function prologue, \TT{PUSH ECX}, is not to save the \ECX state 
(notice the absence of corresponding \TT{POP ECX} at the function's end).

In fact it allocates 4 bytes on the stack for storing the \TT{x} variable.

\label{stack_frame}
\myindex{\Stack!Stack frame}
\myindex{x86!\Registers!EBP}
\TT{x} is to be accessed with the assistance of the \TT{\_x\$} macro (it equals to -4) and the \EBP register pointing to the current frame.

Over the span of the function's execution, \EBP is pointing to the current \gls{stack frame}
making it possible to access local variables and function arguments via \TT{EBP+offset}.

\myindex{x86!\Registers!ESP}
It is also possible to use \ESP for the same purpose, although that is not very convenient since it changes frequently.
The value of the \EBP could be perceived as a \IT{frozen state} of the value in \ESP at the start of the function's execution.

% FIXME1 это уже было в 02_stack?
Here is a typical \gls{stack frame} layout in 32-bit environment:

\begin{center}
\begin{tabular}{ | l | l | }
\hline
\dots & \dots \\
\hline
EBP-8 & local variable \#2, \MarkedInIDAAs{} \TT{var\_8} \\
\hline
EBP-4 & local variable \#1, \MarkedInIDAAs{} \TT{var\_4} \\
\hline
EBP & saved value of \EBP \\
\hline
EBP+4 & return address \\
\hline
EBP+8 & \argument \#1, \MarkedInIDAAs{} \TT{arg\_0} \\
\hline
EBP+0xC & \argument \#2, \MarkedInIDAAs{} \TT{arg\_4} \\
\hline
EBP+0x10 & \argument \#3, \MarkedInIDAAs{} \TT{arg\_8} \\
\hline
\dots & \dots \\
\hline
\end{tabular}
\end{center}

The \scanf function in our example has two arguments.

The first one is a pointer to the string containing \TT{\%d} and the second is the address of the \TT{x} variable.

\myindex{x86!\Instructions!LEA}
First, the \TT{x} variable's address is loaded into the \EAX register by the \\
\TT{lea eax, DWORD PTR \_x\$[ebp]} instruction.

\LEA stands for \IT{load effective address}, and is often used for forming an address ~(\myref{sec:LEA}).

We could say that in this case \LEA simply stores the sum of the \EBP register value and the \TT{\_x\$} macro in the \EAX register.

This is the same as \INS{lea eax, [ebp-4]}.

So, 4 is being subtracted from the \EBP register value and the result is loaded in the \EAX register.
Next the \EAX register value is pushed into the stack and \scanf is being called.

\printf is being called after that with its first argument --- a pointer to the string:
\TT{You entered \%d...\textbackslash{}n}.

The second argument is prepared with: \TT{mov ecx, [ebp-4]}.
The instruction stores the \TT{x} variable value and not its address, in the \ECX register.

Next the value in the \ECX is stored on the stack and the last \printf is being called.

\EN{\clearpage
\mysubparagraph{\olly}
\myindex{\olly}

Let's try this example in \olly.
The input value of the function (2) is loaded into \EAX: 

\begin{figure}[H]
\centering
\myincludegraphics{patterns/08_switch/2_lot/olly1.png}
\caption{\olly: function's input value is loaded in \EAX}
\label{fig:switch_lot_olly1}
\end{figure}

\clearpage
The input value is checked, is it bigger than 4? 
If not, the \q{default} jump is not taken:
\begin{figure}[H]
\centering
\myincludegraphics{patterns/08_switch/2_lot/olly2.png}
\caption{\olly: 2 is no bigger than 4: no jump is taken}
\label{fig:switch_lot_olly2}
\end{figure}

\clearpage
Here we see a jumptable:

\begin{figure}[H]
\centering
\myincludegraphics{patterns/08_switch/2_lot/olly3.png}
\caption{\olly: calculating destination address using jumptable}
\label{fig:switch_lot_olly3}
\end{figure}

Here we've clicked \q{Follow in Dump} $\rightarrow$ \q{Address constant}, so now we see the \IT{jumptable} in the data window.
These are 5 32-bit values\footnote{They are underlined by \olly because
these are also FIXUPs: \myref{subsec:relocs}, we are going to come back to them later}.
\ECX is now 2, so the third element (can be indexed as 2\footnote{About indexing, see also: \ref{arrays_at_one}}) of the table is to be used.
It's also possible to click \q{Follow in Dump} $\rightarrow$ 
\q{Memory address} and \olly will show the element addressed by the \JMP instruction. 
That's \TT{0x010B103A}.

\clearpage
After the jump we are at \TT{0x010B103A}: the code printing \q{two} will now be executed:

\begin{figure}[H]
\centering
\myincludegraphics{patterns/08_switch/2_lot/olly4.png}
\caption{\olly: now we at the \IT{case:} label}
\label{fig:switch_lot_olly4}
\end{figure}
}
\RU{\clearpage
\mysubparagraph{\olly}
\myindex{\olly}

Попробуем этот пример в \olly.
Входное значение функции (2) загружается в \EAX: 

\begin{figure}[H]
\centering
\myincludegraphics{patterns/08_switch/2_lot/olly1.png}
\caption{\olly: входное значение функции загружено в \EAX}
\label{fig:switch_lot_olly1}
\end{figure}

\clearpage
Входное значение проверяется, не больше ли оно чем 4? 
Нет, переход по умолчанию (\q{default}) не будет исполнен:

\begin{figure}[H]
\centering
\myincludegraphics{patterns/08_switch/2_lot/olly2.png}
\caption{\olly: 2 не больше чем 4: переход не сработает}
\label{fig:switch_lot_olly2}
\end{figure}

\clearpage
Здесь мы видим jumptable:

\begin{figure}[H]
\centering
\myincludegraphics{patterns/08_switch/2_lot/olly3.png}
\caption{\olly: вычисляем адрес для перехода используя jumptable}
\label{fig:switch_lot_olly3}
\end{figure}

Кстати, щелкнем по \q{Follow in Dump} $\rightarrow$ \q{Address constant}, так что теперь \IT{jumptable} видна в окне данных.

Это 5 32-битных значений\footnote{Они подчеркнуты в \olly, потому что это также и FIXUP-ы: \myref{subsec:relocs}, мы вернемся к ним позже}.
\ECX сейчас содержит 2, так что третий элемент (либо второй, если считать с нулевого) таблицы будет использован.
Кстати, можно также щелкнуть \q{Follow in Dump} $\rightarrow$ \q{Memory address} и \olly покажет элемент, который сейчас адресуется в инструкции \JMP. 
Это \TT{0x010B103A}.

\clearpage
Переход сработал и мы теперь на \TT{0x010B103A}: сейчас будет исполнен код, выводящий строку \q{two}:

\begin{figure}[H]
\centering
\myincludegraphics{patterns/08_switch/2_lot/olly4.png}
\caption{\olly: теперь мы на соответствующей метке \IT{case:}}
\label{fig:switch_lot_olly4}
\end{figure}
}
\ITA{\clearpage
\mysubparagraph{\olly}
\myindex{\olly}

Esaminiamo questo esempio con \olly.
Il valore di input della funzione (2) viene caricato \EAX: 

\begin{figure}[H]
\centering
\myincludegraphics{patterns/08_switch/2_lot/olly1.png}
\caption{\olly: il valore di input è caricato in \EAX}
\label{fig:switch_lot_olly1}
\end{figure}

\clearpage
Il valore viene controllato, è maggiore di 4?
Se no, il \q{default} jump non viene innescato:
\begin{figure}[H]
\centering
\myincludegraphics{patterns/08_switch/2_lot/olly2.png}
\caption{\olly: 2 non è maggiore di 4: il salto non viene fatto}
\label{fig:switch_lot_olly2}
\end{figure}

\clearpage
Qui vediamo un jumptable:

\begin{figure}[H]
\centering
\myincludegraphics{patterns/08_switch/2_lot/olly3.png}
\caption{\olly: calcolo dell'indirizzo di destinazione mediante jumptable}
\label{fig:switch_lot_olly3}
\end{figure}

Qui abbiamo cliccato \q{Follow in Dump} $\rightarrow$ \q{Address constant}, così da vedere la \IT{jumptable} nella data window.
Sono 5 valori a 32-bit \footnote{Sono sottolineati da \olly poiché
sono anche FIXUPs: \myref{subsec:relocs}, torneremo su questo argomento più avanti}.
\ECX adesso è 2, quindi il terzo elemento (avente indice 2\footnote{Per l'indicizzazione, vedi anche: \ref{arrays_at_one}}) della tabella.
E' anche possibile cliccare su \q{Follow in Dump} $\rightarrow$ 
\q{Memory address} e \olly mostrerà l'elemento a cui punta l'istruzione \JMP. 
In questo caso è \TT{0x010B103A}.

\clearpage
Dopo il salto ci troviamo a \TT{0x010B103A}: il codice che stampa \q{two} sarà ora eseguito:

\begin{figure}[H]
\centering
\myincludegraphics{patterns/08_switch/2_lot/olly4.png}
\caption{\olly: ora ci troviamo alla label \IT{case:}}
\label{fig:switch_lot_olly4}
\end{figure}
}
\DE{\clearpage
\subsubsection{MSVC + \olly}
\myindex{\olly}
Schauen wir uns diese Beispiel in \olly an.
Wir laden es und dr�cken F8 (\stepover) bis wir unsere ausf�hrbare Datei anstelle von \TT{ntdll.dll} erreicht haben. Wir scrollen nach oben bis \main erscheint.

Wir klicken auf den ersten Befehl (\TT{PUSH EBO}), dr�cken F2 (\IT{set a breakpoint}), dann F9 (\IT{Run}). Der Breakpoint wird ausgel�st, wenn die Funktion \main beginnt.

Verfolgen wir den Ablauf bis zu der Stelle, an der die Adresse der Variablen $x$ berechnet wird:

\begin{figure}[H]
\centering
\myincludegraphics{patterns/04_scanf/1_simple/ex1_olly_1.png}
\caption{\olly: Die Adresse der lokalen Variable wird berechnet.}
\label{fig:scanf_ex1_olly_1}
\end{figure}

Wir machen einen Rechtsklick auf \EAX in Registerfenster und w�hlen \q{Follow in stack}. 

Diese Adresse wird im Stackfenster erscheinen. Der rote Pfeil wurde nachtr�glich hinzugef�gt; er zeigt auf die Variable im lokalen Stack. Im Moment enth�lt diese Speicherstelle Zufallswerte (\TT{0x6E494714}). Jetzt wird mithilfe des \PUSH Befehls die Adresse dieses Stackelements auf demselben Stack an der folgenden Position gespeichert. 
Verfolgen wir den Ablauf mit F8 bis die Ausf�hrung von \scanf abgeschlossen ist. W�hrend der Ausf�hrung von \scanf geben wir beispielsweise 123 in der Konsole ein:

\lstinputlisting{patterns/04_scanf/1_simple/console.txt}

\clearpage
\scanf ist bereits beendet:

\begin{figure}[H]
\centering
\myincludegraphics{patterns/04_scanf/1_simple/ex1_olly_3.png}
\caption{\olly: \scanf wurde ausgef�hrt}
\label{fig:scanf_ex1_olly_3}
\end{figure}

\scanf liefert 1 im \EAX Register zur�ck, was aussagt, dass die Funktion einen Wert erfolgreich eingelesen hat. Wenn wir wiederum auf das zugeh�rige Stackelement f�r die lokale Variable schauen, enth�lt diese nun den Wert \TT{0x7B} (dez. 123).

\clearpage
Im weiteren Verlauf wird dieser Wert vom Stack in das \ECX Register kopiert und an \printf �bergeben:

\begin{figure}[H]
\centering
\myincludegraphics{patterns/04_scanf/1_simple/ex1_olly_4.png}
\caption{\olly: Wert f�r �bergabe an \printf vorbereiten.}
\label{fig:scanf_ex1_olly_4}
\end{figure}

}
\FR{\clearpage
\subsubsection{MSVC + \olly}
\myindex{\olly}

% TODO look in French olly for text translation, if exists?
Essayons cet exemple dans \olly.
Chargeons-le et appuyons sur F8 (\stepover) jusqu'à ce que nous atteignons notre
exécutable au lieu de \TT{ntdll.dll}.
Défiler vers le haut jusqu'à ce que \main apparaisse.

Cliquer sur la première instruction  (\TT{PUSH EBP}), appuyer sur F2 (\IT{set a
breakpoint}), puis F9 (\IT{Run}).
Le point d'arrêt sera déclenché lorsque \main commencera.

Continuons jusqu'au point où la variable $x$ est calculée:

\begin{figure}[H]
\centering
\myincludegraphics{patterns/04_scanf/1_simple/ex1_olly_1.png}
\caption{\olly: L'adresse de la variable locale est calculée}
\label{fig:scanf_ex1_olly_1}
\end{figure}

Cliquer droit sur \EAX dans la fenêtre des registres et choisir \q{Follow in stack}.

Cette adresse va apparaître dans la fenêtre de la pile.
La flèche rouge a été ajoutée, pointant la variable dans la pile locale.
A ce point, cet espace contient des restes de données (\TT{0x6E494714}).
Maintenant. avec l'aide de l'instruction \PUSH, l'adresse de cet élément de pile
va être stockée sur la même pile à la position suivante.
Appuyons sur F8 jusqu'à la fin de l'exécution de \scanf.
Pendant l'exécution de \scanf, entrons, par exemple, 123, dans la fenêtre de la console:

\lstinputlisting{patterns/04_scanf/1_simple/console.txt}

\clearpage
\scanf a déjà fini de s'exécuter:

\begin{figure}[H]
\centering
\myincludegraphics{patterns/04_scanf/1_simple/ex1_olly_3.png}
\caption{\olly: \scanf s'est exécutée}
\label{fig:scanf_ex1_olly_3}
\end{figure}

\scanf renvoie 1 dans \EAX, ce qui indique qu'elle a lu avec succès une valeur.
Si nous regardons de nouveau l'élément de la pile correspondant à la variable
locale, il contient maintenant \TT{0x7B} (123).

\clearpage

Plus tard, cette valeur est copiée de la pile vers le registre \ECX et passée à \printf:

\begin{figure}[H]
\centering
\myincludegraphics{patterns/04_scanf/1_simple/ex1_olly_4.png}
\caption{\olly: préparation de la valeur pour la passer à \printf}
\label{fig:scanf_ex1_olly_4}
\end{figure}
}


\myparagraph{GCC}

Let's try to compile this code in GCC 4.4.1 under Linux:

\lstinputlisting[style=customasmx86]{patterns/04_scanf/1_simple/ex1_GCC.asm}

\myindex{puts() instead of printf()}
GCC replaced the \printf call with call to \puts. The reason for this was explained in ~(\myref{puts}).

% TODO: rewrite
%\RU{Почему \scanf переименовали в \TT{\_\_\_isoc99\_scanf}, я честно говоря, пока не знаю.}
%\EN{Why \scanf is renamed to \TT{\_\_\_isoc99\_scanf}, I do not know yet.}
% 
% Apparently it has to do with the ISO c99 standard compliance. By default GCC allows specifying a standard to adhere to.
% For example if you compile with -std=c89 the outputted assmebly file will contain scanf and not __isoc99__scanf. I guess current GCC version adhares to c99 by default.
% According to my understanding the two implementations differ in the set of suported modifyers (See printf man page)

As in the MSVC example---the arguments are placed on the stack using the \MOV instruction.

\myparagraph{By the way}

By the way, this simple example is a demonstration of the fact that compiler translates
list of expressions in \CCpp-block into sequential list of instructions.
There are nothing between expressions in \CCpp, and so in resulting machine code, 
there are nothing between, control flow slips from one expression to the next one.

