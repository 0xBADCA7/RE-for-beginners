\subsubsection{ARM: \OptimizingKeilVI (\ThumbMode)}

\lstinputlisting[caption=IDA,style=customasmARM]{patterns/04_scanf/2_global/ARM.lst}

Donc, la variable \TT{x} est maintenant globale, et pour cette raison, elle se trouve
dans un autre segment, appelé le segment de données (\IT{.data}).
On pourrait demander pour quoi les chaînes de textes sont dans le segment de code
(\IT{.text}) et \TT{x} là.
C'est parque c'est une variable et que par définition sa valeur peut changer. En
outre, elle peut même changer souvent.
Alors que les chaînes de texte ont un type constant, elles ne changent pas, donc
elles sont dans le segment \IT{.text}.
\myindex{\RAM}
\myindex{\ROM}

Le segment de code peut parfois se trouver dans la \ac{ROM} d'un circuit (gardez
à l'esprit que nous avons maintenant affaire avec de l'électronique embarquée, et
que la pénurie de mémoire y est courante), et les variables~---en \ac{RAM}.

Il n'est pas très économique de stocker des constantes en RAM quand vous avez de
la ROM.

En outre, les variables en RAM doivent être initialisées, car après le démarrage,
la RAM, évidemment, contient des données aléatoires.

\myindex{Linker}

En avançant, nous voyons un pointeur sur la variable \TT{x} (\TT{off\_2C}) dans
le segment de code, et que toutes les opérations avec cette variable s'effectuent
via ce pointeur.

Car la variable \TT{x} peut se trouver loin de ce morceau de code, donc son adresse
doit être sauvée proche du code.
\myindex{ARM!\Instructions!LDR}

L'instruction \INS{LDR} en mode Thumb ne peut adresser des variables que dans un
intervalle de 1020 octets de son emplacement.

et en mode ARM~---l'intervalle des variables est de $\pm{}4095$ octets.

Et donc l'adresse de la variable \TT{x} doit se trouver quelque part de très proche,
car il n'y a pas de garantie que l'éditeur de liens pourra stocker la variable proche
du code, elle peut même se trouver sur un module de mémoire externe.

\myindex{\CLanguageElements!const}
\myindex{\ROM}

Encore une chose: si une variable est déclarée comme \IT{const}, le compilateur
Keil va l'allouer dans le segment \TT{.constdata}.

Peut-être qu'après, l'éditeur de liens mettra ce segment en ROM aussi, à côté du segment de code.

\subsubsection{ARM64}

\lstinputlisting[caption=GCC 4.9.1 ARM64 \NonOptimizing,numbers=left,style=customasmARM]{patterns/04_scanf/2_global/ARM64_GCC491_O0_FR.s}

\myindex{ARM!\Instructions!ADRP/ADD pair}

Dans ce car la variable $x$ est déclarée comme étant globale et son adresse est
calculée en utilisant la paire d'instructions \INS{ADRP}/\INS{ADD} (lignes 21 et 25).

