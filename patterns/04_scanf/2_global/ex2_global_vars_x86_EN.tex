\subsubsection{MSVC: x86}

\lstinputlisting[style=customasm]{patterns/04_scanf/2_global/ex2_MSVC.asm}

In this case the \TT{x} variable is defined in the \TT{\_DATA} segment and no memory is allocated in the local stack. It is accessed directly, not through the stack. 
Uninitialized global variables take no space in the executable file
(indeed, why one needs to allocate space for variables initially set to zero?), 
but when someone accesses their address, 
the \ac{OS} will allocate a block of zeros there\footnote{That is how a \ac{VM} behaves}.

Now let's explicitly assign a value to the variable:

\lstinputlisting[style=customc]{patterns/04_scanf/2_global/default_value_EN.c}

We got:

\begin{lstlisting}
_DATA	SEGMENT
_x	DD	0aH

...
\end{lstlisting}

Here we see a value \TT{0xA} of DWORD type (DD stands for DWORD = 32 bit) for this variable.

If you open the compiled .exe in \IDA, you can see the \IT{x} variable placed at the beginning of 
the \TT{\_DATA} segment, and after it you can see text strings.

If you open the compiled .exe from the previous example in \IDA, where the value of \IT{x} hasn't been set, you would see something like this:

\lstinputlisting[caption=\IDA,style=customasm]{patterns/04_scanf/2_global/IDA.lst}

\TT{\_x} is marked with \TT{?} with the rest of the variables that do not need to be initialized. 
This implies that after loading the .exe to the memory, a space for all these variables is to be 
allocated and filled with zeros \InSqBrackets{\CNineNineStd 6.7.8p10}.
But in the .exe file these uninitialized variables do not occupy anything.
This is convenient for large arrays, for example.

\EN{\clearpage
\mysubparagraph{\olly}
\myindex{\olly}

Let's try this example in \olly.
The input value of the function (2) is loaded into \EAX: 

\begin{figure}[H]
\centering
\myincludegraphics{patterns/08_switch/2_lot/olly1.png}
\caption{\olly: function's input value is loaded in \EAX}
\label{fig:switch_lot_olly1}
\end{figure}

\clearpage
The input value is checked, is it bigger than 4? 
If not, the \q{default} jump is not taken:
\begin{figure}[H]
\centering
\myincludegraphics{patterns/08_switch/2_lot/olly2.png}
\caption{\olly: 2 is no bigger than 4: no jump is taken}
\label{fig:switch_lot_olly2}
\end{figure}

\clearpage
Here we see a jumptable:

\begin{figure}[H]
\centering
\myincludegraphics{patterns/08_switch/2_lot/olly3.png}
\caption{\olly: calculating destination address using jumptable}
\label{fig:switch_lot_olly3}
\end{figure}

Here we've clicked \q{Follow in Dump} $\rightarrow$ \q{Address constant}, so now we see the \IT{jumptable} in the data window.
These are 5 32-bit values\footnote{They are underlined by \olly because
these are also FIXUPs: \myref{subsec:relocs}, we are going to come back to them later}.
\ECX is now 2, so the third element (can be indexed as 2\footnote{About indexing, see also: \ref{arrays_at_one}}) of the table is to be used.
It's also possible to click \q{Follow in Dump} $\rightarrow$ 
\q{Memory address} and \olly will show the element addressed by the \JMP instruction. 
That's \TT{0x010B103A}.

\clearpage
After the jump we are at \TT{0x010B103A}: the code printing \q{two} will now be executed:

\begin{figure}[H]
\centering
\myincludegraphics{patterns/08_switch/2_lot/olly4.png}
\caption{\olly: now we at the \IT{case:} label}
\label{fig:switch_lot_olly4}
\end{figure}
}
\RU{\clearpage
\mysubparagraph{\olly}
\myindex{\olly}

Попробуем этот пример в \olly.
Входное значение функции (2) загружается в \EAX: 

\begin{figure}[H]
\centering
\myincludegraphics{patterns/08_switch/2_lot/olly1.png}
\caption{\olly: входное значение функции загружено в \EAX}
\label{fig:switch_lot_olly1}
\end{figure}

\clearpage
Входное значение проверяется, не больше ли оно чем 4? 
Нет, переход по умолчанию (\q{default}) не будет исполнен:

\begin{figure}[H]
\centering
\myincludegraphics{patterns/08_switch/2_lot/olly2.png}
\caption{\olly: 2 не больше чем 4: переход не сработает}
\label{fig:switch_lot_olly2}
\end{figure}

\clearpage
Здесь мы видим jumptable:

\begin{figure}[H]
\centering
\myincludegraphics{patterns/08_switch/2_lot/olly3.png}
\caption{\olly: вычисляем адрес для перехода используя jumptable}
\label{fig:switch_lot_olly3}
\end{figure}

Кстати, щелкнем по \q{Follow in Dump} $\rightarrow$ \q{Address constant}, так что теперь \IT{jumptable} видна в окне данных.

Это 5 32-битных значений\footnote{Они подчеркнуты в \olly, потому что это также и FIXUP-ы: \myref{subsec:relocs}, мы вернемся к ним позже}.
\ECX сейчас содержит 2, так что третий элемент (либо второй, если считать с нулевого) таблицы будет использован.
Кстати, можно также щелкнуть \q{Follow in Dump} $\rightarrow$ \q{Memory address} и \olly покажет элемент, который сейчас адресуется в инструкции \JMP. 
Это \TT{0x010B103A}.

\clearpage
Переход сработал и мы теперь на \TT{0x010B103A}: сейчас будет исполнен код, выводящий строку \q{two}:

\begin{figure}[H]
\centering
\myincludegraphics{patterns/08_switch/2_lot/olly4.png}
\caption{\olly: теперь мы на соответствующей метке \IT{case:}}
\label{fig:switch_lot_olly4}
\end{figure}
}
\ITA{\clearpage
\mysubparagraph{\olly}
\myindex{\olly}

Esaminiamo questo esempio con \olly.
Il valore di input della funzione (2) viene caricato \EAX: 

\begin{figure}[H]
\centering
\myincludegraphics{patterns/08_switch/2_lot/olly1.png}
\caption{\olly: il valore di input è caricato in \EAX}
\label{fig:switch_lot_olly1}
\end{figure}

\clearpage
Il valore viene controllato, è maggiore di 4?
Se no, il \q{default} jump non viene innescato:
\begin{figure}[H]
\centering
\myincludegraphics{patterns/08_switch/2_lot/olly2.png}
\caption{\olly: 2 non è maggiore di 4: il salto non viene fatto}
\label{fig:switch_lot_olly2}
\end{figure}

\clearpage
Qui vediamo un jumptable:

\begin{figure}[H]
\centering
\myincludegraphics{patterns/08_switch/2_lot/olly3.png}
\caption{\olly: calcolo dell'indirizzo di destinazione mediante jumptable}
\label{fig:switch_lot_olly3}
\end{figure}

Qui abbiamo cliccato \q{Follow in Dump} $\rightarrow$ \q{Address constant}, così da vedere la \IT{jumptable} nella data window.
Sono 5 valori a 32-bit \footnote{Sono sottolineati da \olly poiché
sono anche FIXUPs: \myref{subsec:relocs}, torneremo su questo argomento più avanti}.
\ECX adesso è 2, quindi il terzo elemento (avente indice 2\footnote{Per l'indicizzazione, vedi anche: \ref{arrays_at_one}}) della tabella.
E' anche possibile cliccare su \q{Follow in Dump} $\rightarrow$ 
\q{Memory address} e \olly mostrerà l'elemento a cui punta l'istruzione \JMP. 
In questo caso è \TT{0x010B103A}.

\clearpage
Dopo il salto ci troviamo a \TT{0x010B103A}: il codice che stampa \q{two} sarà ora eseguito:

\begin{figure}[H]
\centering
\myincludegraphics{patterns/08_switch/2_lot/olly4.png}
\caption{\olly: ora ci troviamo alla label \IT{case:}}
\label{fig:switch_lot_olly4}
\end{figure}
}
\DE{\clearpage
\subsubsection{MSVC: x86 + \olly}
\myindex{\olly}

Hier sehen die Dinge noch einfacher aus:

\begin{figure}[H]
\centering
\myincludegraphics{patterns/04_scanf/2_global/ex2_olly_1.png}
\caption{\olly: nach Ausführung von \scanf}
\label{fig:scanf_ex2_olly_1}
\end{figure}

Die Variable befindet sich im Datensegment.
Nachdem der \PUSH Befehl (der die Adresse von $x$ speichert) ausgeführt worden ist,
erscheint die Adresse im Stackfenster. Wir machen einen Rechtsklick auf die Zeile und wählen \q{Follow in dump}.
Die Variable erscheint nun im Speicherfenster auf der linken Seite. 
Nachdem wir in der Konsole 123 eingegeben haben, erscheint \TT{0x7B} im Speicherfenster (siehe markiertes Feld im
Screenshot).

Warum ist das erste Byte \TT{7B}?
Logisch gedacht müsste dort \TT{00 00 00 7B} sein. 
Der Grund dafür ist die sogenannte \gls{Endianess} und x86 verwendet \IT{litte Endian}. 
Dies bedeutet, dass das niederwertigste Byte zuerst und das höchstwertigste zuletzt geschrieben werden.
Für mehr Informationen dazu siehe: \myref{sec:endianness}.
Zurück zu Beispiel: der 32-Bit-Wert wird von dieser Speicheradresse nach \EAX geladen und an \printf übergeben. 

Die Speicheradresse von $x$ ist \TT{0x00C53394}.

\clearpage
In \olly können wir die Speicherzuordnung des Prozesses nachvollziehen (Alt-M) und wir erkennen, dass sich diese Adresse
innerhalb des \TT{.data} PE-Segments von unserem Programm befindet:

\label{olly_memory_map_example}
\begin{figure}[H]
\centering
\myincludegraphics{patterns/04_scanf/2_global/ex2_olly_2.png}
\caption{\olly: Speicherzuordnung}
\label{fig:scanf_ex2_olly_2}
\end{figure}

}


\subsubsection{GCC: x86}

\myindex{ELF}
The picture in Linux is near the same, with the difference that the uninitialized variables are located in the \TT{\_bss} segment. 
In \ac{ELF} file this segment has the following attributes:

\begin{lstlisting}
; Segment type: Uninitialized
; Segment permissions: Read/Write
\end{lstlisting}

If you, however, initialize the variable with some value e.g. 10, 
it is to be placed in the \TT{\_data} segment, which has the following attributes:

\begin{lstlisting}
; Segment type: Pure data
; Segment permissions: Read/Write
\end{lstlisting}

\subsubsection{MSVC: x64}

\lstinputlisting[caption=MSVC 2012 x64,style=customasm]{patterns/04_scanf/2_global/ex2_MSVC_x64_EN.asm}

The code is almost the same as in x86.
Please note that the address of the $x$ variable is passed to \TT{scanf()} using a \LEA instruction,
while the variable's value is passed to the second \printf using a \MOV instruction.
\TT{DWORD PTR}---is a part of the assembly language (no relation to the machine code),
indicating that the variable data size is 32-bit and the \MOV instruction has to be encoded accordingly.

