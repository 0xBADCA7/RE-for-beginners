\subsubsection{MSVC: x86}

\lstinputlisting[style=customasmx86]{patterns/04_scanf/2_global/ex2_MSVC.asm}

Dans ce cas, la variable \TT{x} est définie dans la section \TT{\_DATA} et il n'y
a pas de mémoire allouée sur la pile locale. Elle est accédée directement, pas par
la pile.
Les variables globales non initialisées ne prennent pas de place dans le fichier
exécutable (en effet, pourquoi aurait-on besoin d'allouer de l'espace pour des variables
initialement mises à zéro ?), mais lorsque quelqu'un accède à leur adresse, l'\ac{OS}
va y allouer un bloc de zéros\footnote{C'est comme ça que se comportent les \ac{VM}}.

Maintenant, assignons explicitement une valeur à la variable:

\lstinputlisting[style=customc]{patterns/04_scanf/2_global/default_value_FR.c}

Nous obtenons:

\begin{lstlisting}[style=customasmx86]
_DATA	SEGMENT
_x	DD	0aH

...
\end{lstlisting}

Ici nous voyons une valeur \TT{0xA} de type DWORD (DD signifie DWORD = 32 bit) pour
cette variable.

Si vous ouvrez le .exe compilé dans \IDA, vous pouvez voir la variable \IT{x} placée
au début du segment \TT{\_DATA}, et après elle vous pouvez voir la chaîne de texte.

Si vous ouvrez le .exe compilé de l'exemple précédent dans \IDA, où la valeur de
\IT{x} n'était pas mise, vous verrez quelque chose comme ça:

\lstinputlisting[caption=\IDA,style=customasmx86]{patterns/04_scanf/2_global/IDA.lst}

\TT{\_x} est marquée avec \TT{?} avec le reste des variables qui ne doivent pas être
initialisées.
Ceci implique qu'après avoir chargé le .exe en mémoire, de l'espace pour toutes ces
variables doit être alloué et rempli avec des zéros \InSqBrackets{\CNineNineStd 6.7.8p10}.
Mais dans le fichier .exe, ces variables non initialisées n'occupent rien du tout.
C'est pratique pour les gros tableaux, par exemple.

\EN{\clearpage
\mysubparagraph{\olly}
\myindex{\olly}

Let's try this example in \olly.
The input value of the function (2) is loaded into \EAX: 

\begin{figure}[H]
\centering
\myincludegraphics{patterns/08_switch/2_lot/olly1.png}
\caption{\olly: function's input value is loaded in \EAX}
\label{fig:switch_lot_olly1}
\end{figure}

\clearpage
The input value is checked, is it bigger than 4? 
If not, the \q{default} jump is not taken:
\begin{figure}[H]
\centering
\myincludegraphics{patterns/08_switch/2_lot/olly2.png}
\caption{\olly: 2 is no bigger than 4: no jump is taken}
\label{fig:switch_lot_olly2}
\end{figure}

\clearpage
Here we see a jumptable:

\begin{figure}[H]
\centering
\myincludegraphics{patterns/08_switch/2_lot/olly3.png}
\caption{\olly: calculating destination address using jumptable}
\label{fig:switch_lot_olly3}
\end{figure}

Here we've clicked \q{Follow in Dump} $\rightarrow$ \q{Address constant}, so now we see the \IT{jumptable} in the data window.
These are 5 32-bit values\footnote{They are underlined by \olly because
these are also FIXUPs: \myref{subsec:relocs}, we are going to come back to them later}.
\ECX is now 2, so the third element (can be indexed as 2\footnote{About indexing, see also: \ref{arrays_at_one}}) of the table is to be used.
It's also possible to click \q{Follow in Dump} $\rightarrow$ 
\q{Memory address} and \olly will show the element addressed by the \JMP instruction. 
That's \TT{0x010B103A}.

\clearpage
After the jump we are at \TT{0x010B103A}: the code printing \q{two} will now be executed:

\begin{figure}[H]
\centering
\myincludegraphics{patterns/08_switch/2_lot/olly4.png}
\caption{\olly: now we at the \IT{case:} label}
\label{fig:switch_lot_olly4}
\end{figure}
}
\RU{\clearpage
\mysubparagraph{\olly}
\myindex{\olly}

Попробуем этот пример в \olly.
Входное значение функции (2) загружается в \EAX: 

\begin{figure}[H]
\centering
\myincludegraphics{patterns/08_switch/2_lot/olly1.png}
\caption{\olly: входное значение функции загружено в \EAX}
\label{fig:switch_lot_olly1}
\end{figure}

\clearpage
Входное значение проверяется, не больше ли оно чем 4? 
Нет, переход по умолчанию (\q{default}) не будет исполнен:

\begin{figure}[H]
\centering
\myincludegraphics{patterns/08_switch/2_lot/olly2.png}
\caption{\olly: 2 не больше чем 4: переход не сработает}
\label{fig:switch_lot_olly2}
\end{figure}

\clearpage
Здесь мы видим jumptable:

\begin{figure}[H]
\centering
\myincludegraphics{patterns/08_switch/2_lot/olly3.png}
\caption{\olly: вычисляем адрес для перехода используя jumptable}
\label{fig:switch_lot_olly3}
\end{figure}

Кстати, щелкнем по \q{Follow in Dump} $\rightarrow$ \q{Address constant}, так что теперь \IT{jumptable} видна в окне данных.

Это 5 32-битных значений\footnote{Они подчеркнуты в \olly, потому что это также и FIXUP-ы: \myref{subsec:relocs}, мы вернемся к ним позже}.
\ECX сейчас содержит 2, так что третий элемент (либо второй, если считать с нулевого) таблицы будет использован.
Кстати, можно также щелкнуть \q{Follow in Dump} $\rightarrow$ \q{Memory address} и \olly покажет элемент, который сейчас адресуется в инструкции \JMP. 
Это \TT{0x010B103A}.

\clearpage
Переход сработал и мы теперь на \TT{0x010B103A}: сейчас будет исполнен код, выводящий строку \q{two}:

\begin{figure}[H]
\centering
\myincludegraphics{patterns/08_switch/2_lot/olly4.png}
\caption{\olly: теперь мы на соответствующей метке \IT{case:}}
\label{fig:switch_lot_olly4}
\end{figure}
}
\ITA{\clearpage
\mysubparagraph{\olly}
\myindex{\olly}

Esaminiamo questo esempio con \olly.
Il valore di input della funzione (2) viene caricato \EAX: 

\begin{figure}[H]
\centering
\myincludegraphics{patterns/08_switch/2_lot/olly1.png}
\caption{\olly: il valore di input è caricato in \EAX}
\label{fig:switch_lot_olly1}
\end{figure}

\clearpage
Il valore viene controllato, è maggiore di 4?
Se no, il \q{default} jump non viene innescato:
\begin{figure}[H]
\centering
\myincludegraphics{patterns/08_switch/2_lot/olly2.png}
\caption{\olly: 2 non è maggiore di 4: il salto non viene fatto}
\label{fig:switch_lot_olly2}
\end{figure}

\clearpage
Qui vediamo un jumptable:

\begin{figure}[H]
\centering
\myincludegraphics{patterns/08_switch/2_lot/olly3.png}
\caption{\olly: calcolo dell'indirizzo di destinazione mediante jumptable}
\label{fig:switch_lot_olly3}
\end{figure}

Qui abbiamo cliccato \q{Follow in Dump} $\rightarrow$ \q{Address constant}, così da vedere la \IT{jumptable} nella data window.
Sono 5 valori a 32-bit \footnote{Sono sottolineati da \olly poiché
sono anche FIXUPs: \myref{subsec:relocs}, torneremo su questo argomento più avanti}.
\ECX adesso è 2, quindi il terzo elemento (avente indice 2\footnote{Per l'indicizzazione, vedi anche: \ref{arrays_at_one}}) della tabella.
E' anche possibile cliccare su \q{Follow in Dump} $\rightarrow$ 
\q{Memory address} e \olly mostrerà l'elemento a cui punta l'istruzione \JMP. 
In questo caso è \TT{0x010B103A}.

\clearpage
Dopo il salto ci troviamo a \TT{0x010B103A}: il codice che stampa \q{two} sarà ora eseguito:

\begin{figure}[H]
\centering
\myincludegraphics{patterns/08_switch/2_lot/olly4.png}
\caption{\olly: ora ci troviamo alla label \IT{case:}}
\label{fig:switch_lot_olly4}
\end{figure}
}
\DE{\clearpage
\subsubsection{MSVC: x86 + \olly}
\myindex{\olly}

Hier sehen die Dinge noch einfacher aus:

\begin{figure}[H]
\centering
\myincludegraphics{patterns/04_scanf/2_global/ex2_olly_1.png}
\caption{\olly: nach Ausführung von \scanf}
\label{fig:scanf_ex2_olly_1}
\end{figure}

Die Variable befindet sich im Datensegment.
Nachdem der \PUSH Befehl (der die Adresse von $x$ speichert) ausgeführt worden ist,
erscheint die Adresse im Stackfenster. Wir machen einen Rechtsklick auf die Zeile und wählen \q{Follow in dump}.
Die Variable erscheint nun im Speicherfenster auf der linken Seite. 
Nachdem wir in der Konsole 123 eingegeben haben, erscheint \TT{0x7B} im Speicherfenster (siehe markiertes Feld im
Screenshot).

Warum ist das erste Byte \TT{7B}?
Logisch gedacht müsste dort \TT{00 00 00 7B} sein. 
Der Grund dafür ist die sogenannte \gls{Endianess} und x86 verwendet \IT{litte Endian}. 
Dies bedeutet, dass das niederwertigste Byte zuerst und das höchstwertigste zuletzt geschrieben werden.
Für mehr Informationen dazu siehe: \myref{sec:endianness}.
Zurück zu Beispiel: der 32-Bit-Wert wird von dieser Speicheradresse nach \EAX geladen und an \printf übergeben. 

Die Speicheradresse von $x$ ist \TT{0x00C53394}.

\clearpage
In \olly können wir die Speicherzuordnung des Prozesses nachvollziehen (Alt-M) und wir erkennen, dass sich diese Adresse
innerhalb des \TT{.data} PE-Segments von unserem Programm befindet:

\label{olly_memory_map_example}
\begin{figure}[H]
\centering
\myincludegraphics{patterns/04_scanf/2_global/ex2_olly_2.png}
\caption{\olly: Speicherzuordnung}
\label{fig:scanf_ex2_olly_2}
\end{figure}

}


\subsubsection{GCC: x86}

\myindex{ELF}
Le schéma sous Linux est presque le même, avec la différence que les variables
non initialisées se trouvent dans le segment \TT{\_bss}.
Dans un fichier \ac{ELF} ce segment possède les attributs suivants:

\begin{lstlisting}
; Segment type: Uninitialized
; Segment permissions: Read/Write
\end{lstlisting}

Si toutefois vous initialisez la variable avec une valeur quelconque, e.g. 10,
elle sera placée dans le segment \TT{\_data}, qui possède les attributs suivants:

\begin{lstlisting}
; Segment type: Pure data
; Segment permissions: Read/Write
\end{lstlisting}

\subsubsection{MSVC: x64}

\lstinputlisting[caption=MSVC 2012 x64,style=customasmx86]{patterns/04_scanf/2_global/ex2_MSVC_x64_FR.asm}

Le code est presque le même qu'en x86.
Notez toutefois que l'adresse de la variable $x$ est passée à \TT{scanf()} en utilisant
une instruction \LEA, tandis que la valeur de la variable est passée au second \printf
en utilisant une instruction \MOV.
\TT{DWORD PTR}---fait partie du langage d'assemblage (aucune relation avec le code machine),
indique que la taille de la variable est 32-bit et que l'instruction \MOV doit être
encodée en conséquence.

