\subsubsection{MIPS}

\myparagraph{Неинициализированная глобальная переменная}

Так что теперь переменная $x$ глобальная.
Сделаем исполняемый файл вместо объектного и загрузим его в \IDA.
IDA показывает присутствие переменной $x$ в ELF-секции .sbss (помните о \q{Global Pointer}? \myref{MIPS_GP}),
так как переменная не инициализируется в самом начале.

\lstinputlisting[caption=\Optimizing GCC 4.4.5 (IDA)]{patterns/04_scanf/2_global/MIPS/O3_IDA_RU.lst}

IDA уменьшает количество информации, так что сделаем также листинг используя objdump и добавим туда свои комментарии:

\lstinputlisting[caption=\Optimizing GCC 4.4.5 (objdump),numbers=left]{patterns/04_scanf/2_global/MIPS/O3_objdump_RU.txt}

Теперь мы видим, как адрес переменной $x$ берется из буфера 64KiB, используя GP и прибавление к нему отрицательного смещения (строка 18).

И даже более того: адреса трех внешних функций, используемых в нашем примере (\puts, \scanf, \printf)
также берутся из буфера 64KiB используя GP (строки 9, 16 и 26).

GP указывает на середину буфера, так что такие смещения могут нам подсказать, что адреса всех трех функций,
а также адрес переменной $x$ расположены где-то в самом начале буфера.
Действительно, ведь наш пример крохотный.

\myindex{MIPS!\Pseudoinstructions!MOVE}
\myindex{MIPS!\Pseudoinstructions!NOP}
Ещё нужно отметить что функция заканчивается двумя \ac{NOP}-ами (\TT{MOVE \$AT,\$AT}~--- 
это холостая инструкция), чтобы выровнять начало следующей функции по 16-байтной границе.

\myparagraph{Инициализированная глобальная переменная}

Немного изменим наш пример и сделаем, чтобы у $x$ было значение по умолчанию:

\lstinputlisting[style=customc]{patterns/04_scanf/2_global/default_value_RU.c}

Теперь IDA показывает что переменная $x$ располагается в секции .data:

\lstinputlisting[caption=\Optimizing GCC 4.4.5 (IDA)]{patterns/04_scanf/2_global/MIPS/O3_IDA_init_RU.lst}

Почему не .sdata? Может быть, нужно было указать какую-то опцию в GCC?
Тем не менее, $x$ теперь в .data, а это уже общая память и мы можем посмотреть как происходит
работа с переменными там.

\myindex{MIPS!\Instructions!LUI}
\myindex{MIPS!\Instructions!ADDIU}
Адрес переменной должен быть сформирован парой инструкций.
В нашем случае это \INS{LUI} (\q{Load Upper Immediate}~--- загрузить старшие 16 бит) и 
\INS{ADDIU} (\q{Add Immediate Unsigned Word}~--- прибавить значение).
Вот так же листинг сгенерированный objdump-ом для лучшего рассмотрения:

\lstinputlisting[caption=\Optimizing GCC 4.4.5 (objdump)]{patterns/04_scanf/2_global/MIPS/O3_objdump_init_RU.txt}

\myindex{MIPS!\Instructions!LUI}
\myindex{MIPS!\Instructions!ADDIU}
\myindex{MIPS!\Instructions!LW}
Адрес формируется используя \INS{LUI} и \INS{ADDIU}, но старшая часть адреса
всё ещё в регистре \$S0, и можно закодировать смещение в инструкции \INS{LW} (\q{Load Word}), так что одной
\INS{LW} достаточно для загрузки значения из переменной и передачи его в \printf.
Регистры хранящие временные данные имеют префикс T-, но здесь есть также регистры с префиксом S-,
содержимое которых должно быть сохранено в других функциях (т.е. \q{saved}).

% FIXME:
% This needs to be clarified a bit, e.g. "the registers need to be preserved if a function is called and it wants to use them
Вот почему \$S0 был установлен по адресу 0x4006cc и затем был использован по адресу 0x4006e8
после вызова \scanf.

Функция \scanf не изменяет это значение.

% TODO non-optimized example?
