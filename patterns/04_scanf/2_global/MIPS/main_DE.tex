\subsubsection{MIPS}

\myparagraph{Nicht initialisierte globale Variable}
Nun ist $x$ eine globale Variable.
Wir kompilieren zu einer ausführbaren Datei anstelle eines Objectfiles und laden diese in \IDA.
\IDA stellt die Variable $x$ in der ELF Section \IT{.sbss} dar (erinnern Sie sich an \q{Global
Pointer}?\myref{MIPS_GP}), denn die Variable ist zu Beginn nicht initialisiert.

\lstinputlisting[caption=\Optimizing GCC 4.4.5 (IDA),style=customasmMIPS]{patterns/04_scanf/2_global/MIPS/O3_IDA_DE.lst}
\IDA reduziert die Anzahl der Informationen; also erzeugen wir auch ein Listing mit objdump und kommentieren es:

\lstinputlisting[caption=\Optimizing GCC 4.4.5
(objdump),numbers=left,style=customasmMIPS]{patterns/04_scanf/2_global/MIPS/O3_objdump_DE.txt}
Wir erkennen, dass die Adresse der Variablen $x$ mit GP aus einem 64KiB Datenpuffer gelesen wird und ein negativer
Offset hinzugefügt wird (Zeile 18). 
Mehr noch, die Adressen der drei externen Funktionen, die in unserem Beispiel verwendet werden (\puts, \scanf, \printf)
wwerden auch mithilfe von GP aus einem 64KiB globalen Datenpuffer gelesen (Zeile 9, 16 und 26).
GP zeigt auf die Mitte des Puffers und ein solches Offset gibt an, dass die Adressen aller drei Funktionen genau wie die
Adresse von $x$ irgendwo am Anfang des Puffers gespeichert werden. 
Das ergibt Sinn, denn unser Beispiel ist winzig.

\myindex{MIPS!\Pseudoinstructions!MOVE}
\myindex{MIPS!\Pseudoinstructions!NOP}
Eine andere bemerkenswerte Sache ist, dass die Funktion mit zwei \ac{NOP}s (\TT{MOVE \$AT,\$AT} --- Befehle ohne
Auswirking) schließt, damit der Beginn der nächsten Funktion auf eine 16-Byte-Grenze gelegt wird.

\myparagraph{Initialisierte globale Variable}

Verändern wir unser Beispiel dadurch, dass wir $x$ einen Defaultwert geben:

\lstinputlisting[style=customc]{patterns/04_scanf/2_global/default_value_DE.c}

\IDA zeigt hier, dass die Variable $x$ im .data Segment liegt:

\lstinputlisting[caption=\Optimizing GCC 4.4.5
(IDA),style=customasmMIPS]{patterns/04_scanf/2_global/MIPS/O3_IDA_init_DE.lst}
Warum nicht .sdata? Vielleicht hängt es mit einer Option von GCC zusammen?

Nichtsdestotrotz befindet sich $x$ jetzt in .data, also dem allgemeinen Speicherbereich, und wir können uns anschauen
wie hier mit den Variablen gearbeitet wird.


\myindex{MIPS!\Instructions!LUI}
\myindex{MIPS!\Instructions!ADDIU}
Die Adresse der Variablen muss mithilfe eines Instruktionspaares gebildet werden.
In unserem Fall sind dies \INS{LUI} (\q{Load Upper Immediate}) und \INS{ADDIU} (\q{Add Immediate Unsigned Word}).

Hier ist also das objdump Listing für eine genaue Untersuchung:

\lstinputlisting[caption=\Optimizing GCC 4.4.5
(objdump),style=customasmMIPS]{patterns/04_scanf/2_global/MIPS/O3_objdump_init_DE.txt}

\myindex{MIPS!\Instructions!LUI}
\myindex{MIPS!\Instructions!ADDIU}
\myindex{MIPS!\Instructions!LW}
Wir erkennen, dass die Adresse mit \INS{LUI} und \INS{ADDIU} gebildet wird, aber der höherwertige Teil der Adresse sich
immer noch im \$S0 Register befindet und es möglich ist, den Offset durch einen \INS{LW} (\q{Load Word}) Befehl
anzugeben, sodass ein einzelnes \INS{LW} ausreicht um den Wert aus der Variablen zu lesen und an \printf zu übergeben.

Register, die temporäre Daten halten, beginnen mit T-, aber wir sehen hier auch einige, die mit dem Präfix S- beginnen.
Die Inhalte dieser Register müssen gespeichert werden, bevor sie in anderen Funktionen verwendet werden können.

% FIXME:
% This needs to be clarified a bit, e.g. "the registers need to be preserved if a function is called and it wants to use them
Das ist der Grund warum der Wert von \$S0 an der Adresse 0x4006cc gesetzt und dann wieder an der Adresse 0x4006e8
verwendet wurde, nachdem \scanf aufgerufen wurde. Die Funktion \scanf verändert den Wert nicht.


% TODO non-optimized example?
