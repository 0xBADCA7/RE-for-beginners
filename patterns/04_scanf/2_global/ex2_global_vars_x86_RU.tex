\subsubsection{MSVC: x86}

\lstinputlisting[style=customasm]{patterns/04_scanf/2_global/ex2_MSVC.asm}

В целом ничего особенного. Теперь \TT{x} объявлена в сегменте \TT{\_DATA}. 
Память для неё в стеке более не выделяется.
Все обращения к ней происходит не через стек, а уже напрямую. 
Неинициализированные глобальные переменные не занимают места в исполняемом файле
(и действительно, зачем в исполняемом файле
нужно выделять место под изначально нулевые переменные?), но тогда, когда к этому месту в памяти
кто-то обратится, \ac{OS} подставит туда блок, состоящий из нулей\footnote{Так работает \ac{VM}}.

Попробуем изменить объявление этой переменной:

\lstinputlisting[style=customc]{patterns/04_scanf/2_global/default_value_RU.c}

Выйдет в итоге:

\begin{lstlisting}[style=customasmx86]
_DATA	SEGMENT
_x	DD	0aH

...
\end{lstlisting}

Здесь уже по месту этой переменной записано \TT{0xA} с типом DD (dword = 32 бита).

Если вы откроете скомпилированный .exe-файл в \IDA, то увидите, что \IT{x} 
находится в начале сегмента \TT{\_DATA}, после этой переменной будут текстовые строки.

А вот если вы откроете в \IDA .exe скомпилированный в прошлом примере, где значение \IT{x} не определено, то вы увидите:

\lstinputlisting[caption=\IDA,style=customasm]{patterns/04_scanf/2_global/IDA.lst}

\TT{\_x} обозначен как \TT{?}, наряду с другими переменными не требующими инициализации. 
Это означает, что при загрузке .exe в память, место под всё это выделено будет и будет заполнено
нулевыми байтами \InSqBrackets{\CNineNineStd 6.7.8p10}.
Но в самом .exe ничего этого нет. Неинициализированные переменные не занимают места в исполняемых файлах. 
Это удобно для больших массивов, например.

\EN{\clearpage
\mysubparagraph{\olly}
\myindex{\olly}

Let's try this example in \olly.
The input value of the function (2) is loaded into \EAX: 

\begin{figure}[H]
\centering
\myincludegraphics{patterns/08_switch/2_lot/olly1.png}
\caption{\olly: function's input value is loaded in \EAX}
\label{fig:switch_lot_olly1}
\end{figure}

\clearpage
The input value is checked, is it bigger than 4? 
If not, the \q{default} jump is not taken:
\begin{figure}[H]
\centering
\myincludegraphics{patterns/08_switch/2_lot/olly2.png}
\caption{\olly: 2 is no bigger than 4: no jump is taken}
\label{fig:switch_lot_olly2}
\end{figure}

\clearpage
Here we see a jumptable:

\begin{figure}[H]
\centering
\myincludegraphics{patterns/08_switch/2_lot/olly3.png}
\caption{\olly: calculating destination address using jumptable}
\label{fig:switch_lot_olly3}
\end{figure}

Here we've clicked \q{Follow in Dump} $\rightarrow$ \q{Address constant}, so now we see the \IT{jumptable} in the data window.
These are 5 32-bit values\footnote{They are underlined by \olly because
these are also FIXUPs: \myref{subsec:relocs}, we are going to come back to them later}.
\ECX is now 2, so the third element (can be indexed as 2\footnote{About indexing, see also: \ref{arrays_at_one}}) of the table is to be used.
It's also possible to click \q{Follow in Dump} $\rightarrow$ 
\q{Memory address} and \olly will show the element addressed by the \JMP instruction. 
That's \TT{0x010B103A}.

\clearpage
After the jump we are at \TT{0x010B103A}: the code printing \q{two} will now be executed:

\begin{figure}[H]
\centering
\myincludegraphics{patterns/08_switch/2_lot/olly4.png}
\caption{\olly: now we at the \IT{case:} label}
\label{fig:switch_lot_olly4}
\end{figure}
}
\RU{\clearpage
\mysubparagraph{\olly}
\myindex{\olly}

Попробуем этот пример в \olly.
Входное значение функции (2) загружается в \EAX: 

\begin{figure}[H]
\centering
\myincludegraphics{patterns/08_switch/2_lot/olly1.png}
\caption{\olly: входное значение функции загружено в \EAX}
\label{fig:switch_lot_olly1}
\end{figure}

\clearpage
Входное значение проверяется, не больше ли оно чем 4? 
Нет, переход по умолчанию (\q{default}) не будет исполнен:

\begin{figure}[H]
\centering
\myincludegraphics{patterns/08_switch/2_lot/olly2.png}
\caption{\olly: 2 не больше чем 4: переход не сработает}
\label{fig:switch_lot_olly2}
\end{figure}

\clearpage
Здесь мы видим jumptable:

\begin{figure}[H]
\centering
\myincludegraphics{patterns/08_switch/2_lot/olly3.png}
\caption{\olly: вычисляем адрес для перехода используя jumptable}
\label{fig:switch_lot_olly3}
\end{figure}

Кстати, щелкнем по \q{Follow in Dump} $\rightarrow$ \q{Address constant}, так что теперь \IT{jumptable} видна в окне данных.

Это 5 32-битных значений\footnote{Они подчеркнуты в \olly, потому что это также и FIXUP-ы: \myref{subsec:relocs}, мы вернемся к ним позже}.
\ECX сейчас содержит 2, так что третий элемент (либо второй, если считать с нулевого) таблицы будет использован.
Кстати, можно также щелкнуть \q{Follow in Dump} $\rightarrow$ \q{Memory address} и \olly покажет элемент, который сейчас адресуется в инструкции \JMP. 
Это \TT{0x010B103A}.

\clearpage
Переход сработал и мы теперь на \TT{0x010B103A}: сейчас будет исполнен код, выводящий строку \q{two}:

\begin{figure}[H]
\centering
\myincludegraphics{patterns/08_switch/2_lot/olly4.png}
\caption{\olly: теперь мы на соответствующей метке \IT{case:}}
\label{fig:switch_lot_olly4}
\end{figure}
}
\ITA{\clearpage
\mysubparagraph{\olly}
\myindex{\olly}

Esaminiamo questo esempio con \olly.
Il valore di input della funzione (2) viene caricato \EAX: 

\begin{figure}[H]
\centering
\myincludegraphics{patterns/08_switch/2_lot/olly1.png}
\caption{\olly: il valore di input è caricato in \EAX}
\label{fig:switch_lot_olly1}
\end{figure}

\clearpage
Il valore viene controllato, è maggiore di 4?
Se no, il \q{default} jump non viene innescato:
\begin{figure}[H]
\centering
\myincludegraphics{patterns/08_switch/2_lot/olly2.png}
\caption{\olly: 2 non è maggiore di 4: il salto non viene fatto}
\label{fig:switch_lot_olly2}
\end{figure}

\clearpage
Qui vediamo un jumptable:

\begin{figure}[H]
\centering
\myincludegraphics{patterns/08_switch/2_lot/olly3.png}
\caption{\olly: calcolo dell'indirizzo di destinazione mediante jumptable}
\label{fig:switch_lot_olly3}
\end{figure}

Qui abbiamo cliccato \q{Follow in Dump} $\rightarrow$ \q{Address constant}, così da vedere la \IT{jumptable} nella data window.
Sono 5 valori a 32-bit \footnote{Sono sottolineati da \olly poiché
sono anche FIXUPs: \myref{subsec:relocs}, torneremo su questo argomento più avanti}.
\ECX adesso è 2, quindi il terzo elemento (avente indice 2\footnote{Per l'indicizzazione, vedi anche: \ref{arrays_at_one}}) della tabella.
E' anche possibile cliccare su \q{Follow in Dump} $\rightarrow$ 
\q{Memory address} e \olly mostrerà l'elemento a cui punta l'istruzione \JMP. 
In questo caso è \TT{0x010B103A}.

\clearpage
Dopo il salto ci troviamo a \TT{0x010B103A}: il codice che stampa \q{two} sarà ora eseguito:

\begin{figure}[H]
\centering
\myincludegraphics{patterns/08_switch/2_lot/olly4.png}
\caption{\olly: ora ci troviamo alla label \IT{case:}}
\label{fig:switch_lot_olly4}
\end{figure}
}
\DE{\clearpage
\subsubsection{MSVC: x86 + \olly}
\myindex{\olly}

Hier sehen die Dinge noch einfacher aus:

\begin{figure}[H]
\centering
\myincludegraphics{patterns/04_scanf/2_global/ex2_olly_1.png}
\caption{\olly: nach Ausführung von \scanf}
\label{fig:scanf_ex2_olly_1}
\end{figure}

Die Variable befindet sich im Datensegment.
Nachdem der \PUSH Befehl (der die Adresse von $x$ speichert) ausgeführt worden ist,
erscheint die Adresse im Stackfenster. Wir machen einen Rechtsklick auf die Zeile und wählen \q{Follow in dump}.
Die Variable erscheint nun im Speicherfenster auf der linken Seite. 
Nachdem wir in der Konsole 123 eingegeben haben, erscheint \TT{0x7B} im Speicherfenster (siehe markiertes Feld im
Screenshot).

Warum ist das erste Byte \TT{7B}?
Logisch gedacht müsste dort \TT{00 00 00 7B} sein. 
Der Grund dafür ist die sogenannte \gls{Endianess} und x86 verwendet \IT{litte Endian}. 
Dies bedeutet, dass das niederwertigste Byte zuerst und das höchstwertigste zuletzt geschrieben werden.
Für mehr Informationen dazu siehe: \myref{sec:endianness}.
Zurück zu Beispiel: der 32-Bit-Wert wird von dieser Speicheradresse nach \EAX geladen und an \printf übergeben. 

Die Speicheradresse von $x$ ist \TT{0x00C53394}.

\clearpage
In \olly können wir die Speicherzuordnung des Prozesses nachvollziehen (Alt-M) und wir erkennen, dass sich diese Adresse
innerhalb des \TT{.data} PE-Segments von unserem Programm befindet:

\label{olly_memory_map_example}
\begin{figure}[H]
\centering
\myincludegraphics{patterns/04_scanf/2_global/ex2_olly_2.png}
\caption{\olly: Speicherzuordnung}
\label{fig:scanf_ex2_olly_2}
\end{figure}

}


\subsubsection{GCC: x86}

\myindex{ELF}
В Linux всё почти также. За исключением того, что если значение \TT{x} не определено, 
то эта переменная будет находится в сегменте \TT{\_bss}.
В \ac{ELF} этот сегмент имеет такие атрибуты:

\begin{lstlisting}
; Segment type: Uninitialized
; Segment permissions: Read/Write
\end{lstlisting}

Ну а если сделать статическое присвоение этой переменной какого-либо
значения, например, 10, то она будет находится 
в сегменте \TT{\_data},
это сегмент с такими атрибутами:

\begin{lstlisting}
; Segment type: Pure data
; Segment permissions: Read/Write
\end{lstlisting}

\subsubsection{MSVC: x64}

\lstinputlisting[caption=MSVC 2012 x64,style=customasm]{patterns/04_scanf/2_global/ex2_MSVC_x64_RU.asm}

Почти такой же код как и в x86.
Обратите внимание что для \TT{scanf()} адрес переменной $x$ передается
при помощи инструкции \LEA, а во второй \printf передается само значение переменной при помощи \MOV.
\TT{DWORD PTR} --- это часть языка ассемблера (не имеющая отношения к машинным кодам) показывающая, что тип переменной в памяти именно 32-битный, 
и инструкция \MOV должна быть здесь закодирована соответственно.

