\section{Opérateur GOTO}

L'opérateur GOTO est en général considéré comme un anti-pattern, voir
[Edgar Dijkstra, \IT{Go To Statement Considered Harmful} (1968)\footnote{\url{http://yurichev.com/mirrors/Dijkstra68.pdf}}].
Néanmoins, il peut être utilisé raisonnablement, voir
[Donald E. Knuth, \IT{Structured Programming with go to Statements} (1974)\footnote{\url{http://yurichev.com/mirrors/KnuthStructuredProgrammingGoTo.pdf}}]
\footnote{\InSqBrackets{\CNotes} a aussi quelques exemples.}.

Voici un exemple très simple:

\lstinputlisting[style=customc]{patterns/065_GOTO/goto.c}

Voici ce que nous obtenons avec MXVC 2012:

\lstinputlisting[caption=MSVC 2012,style=customasmx86]{patterns/065_GOTO/MSVC_goto.asm}

L'instruction \IT{goto} a simplement été remplacée par une instruction \JMP, qui
a le même effet: saut inconditionnel à un autre endroit.
Le second \printf peut seulement être exécuté avec une intervention humaine, en
utilisant un débogueur or en modifiant le code.

\par

\clearpage

Cela peut-être utile comme exercice simple de patching. Ouvrons l'exécutable généré
dans Hiew:

\begin{figure}[H]
\centering
\myincludegraphics{patterns/065_GOTO/hiew1.png}
\caption{Hiew}
\label{fig:goto_hiew1}
\end{figure}

\clearpage
Placez le curseur à l'adresse du \JMP (\TT{0x410}),
pressez F3 (edit), pressez deux fois zéro, donc l'opcode devient \TT{EB 00}:

\begin{figure}[H]
\centering
\myincludegraphics{patterns/065_GOTO/hiew2.png}
\caption{Hiew}
\label{fig:goto_hiew2}
\end{figure}

Le second octet de l'opcode de \JMP indique l'offset relatif du saut, 0 signifie
le point juste après l'instruction courante.

Donc maintenant \JMP n'évite plus le second \printf.

Pressez F9 (save) et quittez. Maintenant, si nous lançons l'exécutable, nous verrons
ceci:

\lstinputlisting[caption=Sortie de l'exécutable modifié]{patterns/065_GOTO/console.txt}

Le même résultat peut être obtenu en remplaçant l'instruction \JMP par 2 instructions
\NOP.

\NOP a un opcode de \TT{0x90} et une longueur de 1 octet, donc nous en avons besoin
de 2 pour remplacer \JMP (qui a une taille de 2 octets).

\subsection{Code mort}

Le second appel à \printf est aussi appelé \q{code mort} en terme de compilateur.

Cela signifie que le code ne sera jamais exécuté.
Donc lorsque vous compilez cet exemple avec les optimisations, le compilateur supprime
le \q{code mort}, n'en laissant aucune trace:

\lstinputlisting[caption=MSVC 2012 \Optimizing,style=customasmx86]{patterns/065_GOTO/MSVC_goto_Ox.asm}

Toutefois, le compilateur a oublié de supprimer la chaîne \q{skip me!}.

%Note: cl "/Ox" option for maximum optimisation does get rid of "skip me" string as well

\subsection{\Exercise}

% TODO debugger example can fit here

Essayez d'obtenir le même résultat en utilisant votre compilateur et votre débogueur
favorits.

