\subsection{Векторизация}

\newcommand{\URLVEC}{\href{http://go.yurichev.com/17080}{Wikipedia: vectorization}}

Векторизация\footnote{\URLVEC} это когда у вас есть цикл, который берет на вход несколько массивов и выдает, 
например, один массив данных. 
Тело цикла берет некоторые элементы из входных массивов, что-то делает с ними и помещает в выходной. 
%Важно, что операция, применяемая ко всем элементам одна и та же. 
Векторизация ~--- это обрабатывать несколько элементов одновременно.

Векторизация ~--- это не самая новая технология: автор сих строк видел её по крайней мере на 
линейке суперкомпьютеров Cray Y-MP от 1988, когда работал на его версии-\q{лайт} Cray Y-MP EL
\footnote{Удаленно. Он находится в музее суперкомпьютеров: \url{http://go.yurichev.com/17081}}.

% FIXME! add assembly listing!
Например:

\begin{lstlisting}[style=customc]
for (i = 0; i < 1024; i++)
{
    C[i] = A[i]*B[i];
}
\end{lstlisting}

Этот фрагмент кода берет элементы из A и B, перемножает и сохраняет результат в C.

\myindex{x86!\Instructions!PLMULLD}
\myindex{x86!\Instructions!PLMULHW}
\newcommand{\PMULLD}{\IT{PMULLD} (\IT{Перемножить запакованные знаковые DWORD и сохранить младшую часть результата})}
\newcommand{\PMULHW}{\TT{PMULHW} (\IT{Перемножить запакованные знаковые DWORD и сохранить старшую часть результата})}

Если представить, что каждый элемент массива ~--- это 32-битный \Tint, то их можно загружать сразу 
по 4 из А в 128-битный XMM-регистр, 
из B в другой XMM-регистр и выполнив инструкцию \PMULLD{} и \PMULHW{}, можно получить 4 64-битных 
\glslink{product}{произведения} сразу.

Таким образом, тело цикла исполняется $1024/4$ раза вместо 1024, что в 4 раза меньше, и, конечно, быстрее.

\newcommand{\URLINTELVEC}{\href{http://go.yurichev.com/17082}{Excerpt: Effective Automatic Vectorization}}

\subsubsection{Пример сложения}

\myindex{Intel C++}
Некоторые компиляторы умеют делать автоматическую векторизацию в простых случаях, 
например, Intel C++\footnote{Еще о том, как Intel C++ умеет автоматически векторизовать циклы: \URLINTELVEC}.

Вот очень простая функция:

\begin{lstlisting}[style=customc]
int f (int sz, int *ar1, int *ar2, int *ar3)
{
	for (int i=0; i<sz; i++)
		ar3[i]=ar1[i]+ar2[i];

	return 0;
};
\end{lstlisting}

\myparagraph{Intel C++}

Компилируем её при помощи Intel C++ 11.1.051 win32:

\begin{verbatim}
icl intel.cpp /QaxSSE2 /Faintel.asm /Ox
\end{verbatim}

Имеем такое (в \IDA):

\lstinputlisting[style=customasmx86]{patterns/19_SIMD/18_1_RU.asm}

Инструкции, имеющие отношение к SSE2 это:
\myindex{x86!\Instructions!MOVDQA}
\myindex{x86!\Instructions!MOVDQU}
\myindex{x86!\Instructions!PADDD}
\begin{itemize}
\item
\MOVDQU (\IT{Move Unaligned Double Quadword}) --- она просто загружает 16 байт из памяти в XMM-регистр.

\item
\PADDD (\IT{Add Packed Integers}) --- складывает сразу 4 пары 32-битных чисел и оставляет в первом операнде результат. 
Кстати, если произойдет переполнение, то исключения не произойдет и никакие флаги не установятся, 
запишутся просто младшие 32 бита результата. 
Если один из операндов \PADDD ~--- адрес значения в памяти, 
то требуется чтобы адрес был выровнен по 16-байтной границе. Если он не выровнен, произойдет исключение
\footnote{О выравнивании данных см. также: \URLWPDA}.

\item
\MOVDQA (\IT{Move Aligned Double Quadword}) --- тоже что и \MOVDQU, только подразумевает 
что адрес в памяти выровнен по 16-байтной границе. 
Если он не выровнен, произойдет исключение. 
\MOVDQA работает быстрее чем \MOVDQU, но требует вышеозначенного.

\end{itemize}

Итак, эти SSE2-инструкции исполнятся только в том случае если еще осталось просуммировать 
4 пары переменных типа \Tint плюс если указатель \TT{ar3} выровнен по 16-байтной границе.

Более того, если еще и \TT{ar2} выровнен по 16-байтной границе, 
то будет выполняться этот фрагмент кода:

\begin{lstlisting}[style=customasmx86]
movdqu  xmm0, xmmword ptr [ebx+edi*4] ; ar1+i*4
paddd   xmm0, xmmword ptr [esi+edi*4] ; ar2+i*4
movdqa  xmmword ptr [eax+edi*4], xmm0 ; ar3+i*4
\end{lstlisting}

А иначе, значение из \TT{ar2} загрузится в \XMM{0} используя инструкцию \MOVDQU, 
которая не требует выровненного указателя, зато может работать чуть медленнее:

\lstinputlisting[style=customasmx86]{patterns/19_SIMD/18_1_excerpt_RU.asm}

А во всех остальных случаях, будет исполняться код, который был бы, как если бы не была 
включена поддержка SSE2.

\myparagraph{GCC}

\newcommand{\URLGCCVEC}{\url{http://go.yurichev.com/17083}}

Но и GCC умеет кое-что векторизировать\footnote{Подробнее о векторизации в GCC: \URLGCCVEC}, 
если компилировать с опциями \Othree и включить поддержку SSE2: \TT{-msse2}.

Вот что вышло (GCC 4.4.1):

\lstinputlisting[style=customasmx86]{patterns/19_SIMD/18_2_gcc_O3.asm}

Почти то же самое, хотя и не так дотошно, как Intel C++.

\subsubsection{Пример копирования блоков}
\label{vec_memcpy}

Вернемся к простому примеру memcpy() (\myref{loop_memcpy}):

\lstinputlisting[style=customc]{memcpy.c}

И вот что делает оптимизирующий GCC 4.9.1:

\lstinputlisting[caption=\Optimizing GCC 4.9.1 x64,style=customasmx86]{patterns/19_SIMD/memcpy_GCC49_x64_O3_RU.s}
