% !TEX program = XeLaTeX
% !TEX encoding = UTF-8
\documentclass[UTF8,nofonts]{ctexart}
\setCJKmainfont[BoldFont=STHeiti,ItalicFont=STKaiti]{STSong}
\setCJKsansfont[BoldFont=STHeiti]{STXihei}
\setCJKmonofont{STFangsong}

\begin{document}

%daveti: translated on Dec 30, 2016
%NOTE: above is needed for MacTex.

\subsection{CPU简介}

\ac{CPU}是执行构成程序的机器代码的设备。

\textbf{简单汇总:}

\begin{description}
\item[指令 Instruction]: 一个基本的\ac{CPU}命令。
简单的例子包括在寄存器之间移动数据,读写内存,基本的算术运算。
每个\ac{CPU}通常都有自己的指令集(\ac{ISA})。

\item[机器代码 Machine code]: \ac{CPU}可以直接识别处理的代码。
一个指令一般含有几个字节的机器代码。
\item[汇编语言 Assembly language]: 机器语言的助记符和宏等扩展,主要用来帮助程序员摆脱机器代码的困扰。
\item[寄存器 CPU register]: 每个\ac{CPU}都有一组数量固定的通用寄存器(\ac{GPR})。
x86有$\approx 8$个, x86-64有$\approx 16$个, ARM有$\approx 16$个。
理解寄存器的最简单方式是把它们当成没有类型的临时变量。
The easiest way to understand a register is to think of it as an untyped temporary variable.
假设你是用某种高级语言\ac{PL}而且只能使用8个32位或者64位的变量。
事实就是即便只是使用这几个寄存器,也已经足够了。
\end{description}

% TODO1 add about linker: "кПЌпПМПвщОк" О "реЎактПр связей" в русскПязычМПй лОт-ре

% A note on the experiments in this area (like the LISP machines http://en.wikipedia.org/wiki/Lisp_machine
% might be useful
你可能会好奇为什么要创造出高级编程语言\ac{PL}而不是使用机器语言。答案的关键在于人并不是\ac{CPU}---%
对人来说,高级语言像\ac{PL} like \CCpp, Java, Python等更自然,而对\ac{CPU}来说,简单的机器语言更符合其运行效率。
或者发明一个可以直接运行高级语言\ac{PL}代码的\ac{CPU}并不是不肯能,但是相对于目前的\ac{CPU}架构,这个\ac{CPU}一定会更复杂。
同样,因为汇编语言的底层操作和面向\ac{CPU},汇编编程通常是写几行就要调试几行,因为太容易出错,所以对程序员来说,汇编语言还是不方便。
把高级语言\ac{PL}代码转换成汇编语言的程序叫做编译器\IT{compiler}。
\footnote{传统俄语文献又叫做翻译器\q{translator}。}。

\myindex{ARM!\ARMMode}%
\myindex{ARM!\ThumbMode}%
\myindex{ARM!\ThumbTwoMode}%

\subsubsection{关于不同的指令集架构 \ac{ISA}s}
x86指令集\ac{ISA}中,指令长度是变长的,所以当加入x64扩展事,指令集本身没有太大变化。
事实上,x86指令姐\ac{ISA}依然含有大量的16位8086 CPU的指令。
ARM是精简指令集架构\ac{RISC}的\ac{CPU}指令集,指令长度是固定的。
在最初版本中,所有的ARM指令都是4个字节。
\footnote{
顺便提一句,固定长度指令之所以方便是因为在译码时CPU可以快速找到下条指令的位置。
这个特性将在switch()操作符~(\myref{sec:SwitchARMLot})单元进一步讨论。
}.
传统ARM指令集又被称作\q{ARM mode}。
然后人们意识到定长指令并不是想象中那么有优势。
实际上,实际应用程序使用的大部分的\ac{CPU}指令\footnote{包括MOV/PUSH/CALL/Jcc}都可以用更少的字节来编码。
由此,Thumb指令集\ac{ISA}由此而生,每个指令的长度只是2个字节。
这种指令集又被称作\q{Thumb mode}。
但是并不是所有的ARM指令都能用2个字节来表示,所以Thumb指令集是ARM指令集的一个子集。
值得注意的是ARM模式的代码和Thumb模式的代码可能共存于同一个程序的编译代码中。
ARM觉得Thumb可以扩展,所以在ARMv7架构中提出了Thumb-2。
Thumb-2的指令依然是2个字节长,但是加入了某些4个字节长的指令。
有种观点认为Thumb-2是ARM和Thumb的混合指令集,这是错的。
Thumb-2应该是Thumb的扩展,来支持所有的处理器功能,和取代传统ARM模式---
这个目标已经基本实现,因为大部分设备的应用程序都使用Thumb-2指令集(很大原因是Xcode的默认编译模式就是Thumb-2)。
然后64位ARM诞生了。
该指令集都是4字节指令,因此不需要支持Thumb模式。
所以,因为64位的影响,我们现在有3个ARM指令集,ARM模式,Thumb模式和ARM64。
这些指令集有部分交叉,但是基本上是不同的指令集,而是不是同一个的变种。
所以,在本书中,我们将尝试罗列不同指令集下的编译代码。
\myindex{PowerPC}%
\myindex{MIPS}%
\myindex{Alpha AXP}%
顺便一提,还有其他各种精简指令集架构\ac{RISC}的指令集\ac{ISA}s是使用定长4字节的指令,例如MIPS, PowerPC和Alpha AXP。


\end{document}