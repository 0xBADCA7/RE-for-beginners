\subsection{Échanger les valeurs en entrée}

Ceci fait ce que l'on veut:

\lstinputlisting[style=customc]{patterns/061_pointers/swap/5.c}

Comme on le voit, les octets sont chargés dans la partie 8-bit basse de \TT{ECX}
et \TT{EBX} en utliisant \INS{MOVZX} (donc les parties hautes de ces registres vont
être effacées) et ensuite les octets échangés sont récrits.

\lstinputlisting[style=customasmx86,caption=GCC 5.4 \Optimizing]{patterns/061_pointers/swap/5_GCC_O3_x86.s}

Les adresses des deux octets sont lues depuis les arguments et durant l'exécution
de la fonctions sont copiés dans \TT{EDX} et \TT{EAX}.

Donc nous utilisons des pointeurs, il n'y a sans doute pas de meilleure façon de
réaliser cette tâche sans eux.

