\subsection{戻り値}

ポインターは関数から値を返すためによく使用されます( \scanf 関数の呼び出し(\myref{label_scanf}))。 

たとえば、関数が2つの値を返す必要がある場合などです。

\subsubsection{グローバル変数の例}

\lstinputlisting[style=customc]{patterns/061_pointers/2/global.c}

次のようにコンパイルされます。

\lstinputlisting[caption=\Optimizing MSVC 2010 (/Ob0),style=customasmx86]{patterns/061_pointers/2/global.asm}

\myindex{\olly}
\clearpage
\olly で見てみましょう。

\begin{figure}[H]
\centering
\myincludegraphics{patterns/061_pointers/2/olly_global1.png}
\caption{\olly: 
グローバル変数のアドレスは \ttfone に渡されます}
\label{fig:pointers_olly_global_1}
\end{figure}

まず、グローバル変数のアドレスが \ttfone に渡されます。
スタック要素に\q{Follow in dump}をクリックすると、
2つの変数に割り当てられたデータセグメント内の場所を見ることができます。

\clearpage
これらの変数は、初期化されていないデータ(\ac{BSS}から)が実行開始前にクリアされるため、
ゼロにされます。[see \CNineNineStd{} 6.7.8p10]

それらはデータセグメントにあり、Alt-Mを押してメモリマップを確認することで確認できます。

\begin{figure}[H]
\centering
\myincludegraphics{patterns/061_pointers/2/olly_global5.png}
\caption{\olly: メモリマップ}
\label{fig:pointers_olly_global_5}
\end{figure}

\clearpage
\ttfone の先頭にをトレース(F7)しましょう。

\begin{figure}[H]
\centering
\myincludegraphics{patterns/061_pointers/2/olly_global2.png}
\caption{\olly: \ttfone が開始}
\label{fig:pointers_olly_global_2}
\end{figure}

2つの値がスタック456(\TT{0x1C8})と
123(\TT{0x7B})に表示され、2つのグローバル変数のアドレスも表示されます。

\clearpage
\ttfone の終わりまでトレースしましょう。
左下のウィンドウで、計算結果がグローバル変数にどのように表示されるかを確認します。

\begin{figure}[H]
\centering
\myincludegraphics{patterns/061_pointers/2/olly_global3.png}
\caption{\olly: \ttfone 実行完了}
\label{fig:pointers_olly_global_3}
\end{figure}

\clearpage

グローバル変数の値は、(スタックを介して) \printf に渡す準備が整ったレジスタにロードされます。

\begin{figure}[H]
\centering
\myincludegraphics{patterns/061_pointers/2/olly_global4.png}
\caption{\olly: 
グローバル変数の値は \printf に渡されます}
\label{fig:pointers_olly_global_4}
\end{figure}

\subsubsection{ローカル変数の例}

私たちの例を少し修正しましょう。

\lstinputlisting[caption=now the \TT{sum} and \TT{product} variables are local,style=customc]{patterns/061_pointers/2/local_JPN.c}

\ttfone コードは変更されません。 
\main のコードだけが行います:

\lstinputlisting[caption=\Optimizing MSVC 2010 (/Ob0),style=customasmx86]{patterns/061_pointers/2/local.asm}

\newcommand{\PtrsAddresses}{\TT{0x2EF854} and \TT{0x2EF858}\xspace}

\clearpage
\olly をもう一度見てみましょう。
スタック内のローカル変数のアドレスは \PtrsAddresses です。
これらがスタックにどのようにプッシュされるのかを確認します。

\begin{figure}[H]
\centering
\myincludegraphics{patterns/061_pointers/2/olly_stk1.png}
\caption{\olly: ローカル変数のアドレスがスタックにプッシュされる}
\label{fig:pointers_olly_stk_1}
\end{figure}

\clearpage
\ttfone が開始します。
今のところ \PtrsAddresses のスタックにランダムなゴミだけがあります。

\begin{figure}[H]
\centering
\myincludegraphics{patterns/061_pointers/2/olly_stk2.png}
\caption{\olly: \ttfone 開始}
\label{fig:pointers_olly_stk_2}
\end{figure}

\clearpage
\ttfone が完了。

\begin{figure}[H]
\centering
\myincludegraphics{patterns/061_pointers/2/olly_stk3.png}
\caption{\olly: \ttfone 実行完了}
\label{fig:pointers_olly_stk_3}
\end{figure}

ここで、アドレス \PtrsAddresses に\TT{0xDB18} と \TT{0x243}が見つかります。
これらの値は \ttfone の結果です。

\subsubsection{\Conclusion{}}
 
\ttfone は、メモリ内の任意の場所にポインタを返すことができます。

これは本質的にポインタの有用性です。

ところで、\Cpp リファレンスはまったく同じように動作します。
それらの詳細については、(\myref{cpp_references})を参照してください。
