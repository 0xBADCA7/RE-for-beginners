\subsubsection{x86}

\myparagraph{MSVC \NonOptimizing}

Nous obtenons (MSVC 2010):

\lstinputlisting[caption=MSVC 2010,style=customasmx86]{patterns/14_bitfields/2_set_reset/set_reset_msvc.asm}

\myindex{x86!\Instructions!OR}

L'instruction \OR met un bit à la valeur 1 tout en ignorant les autres bits.

\myindex{x86!\Instructions!AND}

\AND annule un bit. On peut dire que \AND copie simplement tous les bits sauf un.
En effet, dans le second opérande du \AND seuls les bits qui doivent être sauvés
sont mis (à 1), seul celui qu'on ne veut pas copier ne l'est pas (il est à 0 dans
le bitmask).
C'est la manière la plus facile de mémoriser la logique.

\clearpage
\mysubparagraph{\olly}

Essayons cet exemple dans \olly.

Tout d'abord, regardons la forme binaire de la constante que nous allons utiliser:

\TT{0x200} (0b0000000000000000000{\color{red}1}000000000) (i.e., le 10ème bit (en
comptant depuis le 1er)).

\TT{0x200} inversé est \TT{0xFFFFFDFF} (0b1111111111111111111{\color{red}0}111111111).

\TT{0x4000} (0b00000000000000{\color{red}1}00000000000000) (i.e., le 15ème bit).

La valeur d'entrée est: \TT{0x12340678} (0b10010001101000000011001111000).
Nous voyons comment elle est chargée:

\begin{figure}[H]
\centering
\myincludegraphics{patterns/14_bitfields/2_set_reset/olly1.png}
\caption{\olly: valeur chargée dans \ECX}
\label{fig:set_reset_olly1}
\end{figure}

\clearpage
\OR exécuté:

\begin{figure}[H]
\centering
\myincludegraphics{patterns/14_bitfields/2_set_reset/olly2.png}
\caption{\olly: \OR exécuté}
\label{fig:set_reset_olly2}
\end{figure}

Le 15ème bit est mis: \TT{0x1234{\color{red}4}678} 
(0b10010001101000{\color{red}1}00011001111000).

\clearpage
La valeur est encore rechargée (car le compilateur n'est pas en mode avec optimisation):

\begin{figure}[H]
\centering
\myincludegraphics{patterns/14_bitfields/2_set_reset/olly3.png}
\caption{\olly: valeur rechargée dans \EDX}
\label{fig:set_reset_olly3}
\end{figure}

\clearpage
\AND exécuté:

\begin{figure}[H]
\centering
\myincludegraphics{patterns/14_bitfields/2_set_reset/olly4.png}
\caption{\olly: \AND exécuté}
\label{fig:set_reset_olly4}
\end{figure}

Le 10ème bit a été mis à 0 (ou, en d'autres mots, tous les bits ont été laissés sauf
le 10ème) et la valeur finale est maintenant
\TT{0x12344{\color{red}4}78} (0b1001000110100010001{\color{red}0}001111000).

\myparagraph{MSVC \Optimizing}

Si nous le compilons dans MSVC avec l'option d'optimisation (\Ox), le code est même
plus court:

\lstinputlisting[caption=MSVC \Optimizing,style=customasmx86]{patterns/14_bitfields/2_set_reset/set_reset_msvc_Ox.asm}

\myparagraph{GCC \NonOptimizing}

Essayons avec GCC 4.4.1 sans optimisation:

\lstinputlisting[caption=GCC \NonOptimizing,style=customasmx86]{patterns/14_bitfields/2_set_reset/set_reset_gcc.asm}

Il y a du code redondant, toutefois, c'est plus court que la version MSVC sans optimisation.

Maintenant, essayons GCC avec l'option d'optimisation \Othree:

\myparagraph{GCC \Optimizing}

\lstinputlisting[caption=GCC \Optimizing,style=customasmx86]{patterns/14_bitfields/2_set_reset/set_reset_gcc_O3.asm}

C'est plus court.
Il est intéressant de noter que le compilateur travaille avec une partie du registre
\EAX via le registre \AH---qui est la partie du registre \EAX située entre les 8ème
et 15ème bits inclus.

\RegTableOne{RAX}{EAX}{AX}{AH}{AL}

\myindex{Intel!8086}
\myindex{Intel!80386}
N.B.  L'accumulateur du CPU 16-bit 8086 était appelé \AX et consistait en deux moitiés
de 8-bit---\AL (octet bas) et \AH (octet haut).
Dans le 80386, presque tous les registres ont été étendus à 32-bit, l'accumulateur
a été appelé \EAX, mais pour des raisons de compatibilité, ses \IT{anciennes parties}
pouvent toujours être accédées par \AX/\AH/\AL.

Puisque tous les CPUs x86 sont des descendants du CPU 16-bit 8086, ces \IT{anciens}
opcodes 16-bit sont plus courts que les nouveaux sur 32-bit.
C'est pourquoi l'instruction \INS{or ah, 40h} occupe seulement 3 octets.
Il serait plus logique de générer ici \INS{or eax, 04000h}  mais ça fait 5 octets,
ou même 6 (dans le cas où le registre du premier opérande n'est pas \EAX).

\myparagraph{GCC \Optimizing et regparm}

Il serait encore plus court en mettant le flag d'optimisation \Othree et aussi \TT{regparm=3}.

\lstinputlisting[caption=GCC \Optimizing,style=customasmx86]{patterns/14_bitfields/2_set_reset/set_reset_gcc_O3_regparm3.asm}

\myindex{Code inline}

En effet, le premier argument est déjà chargé dans \EAX, donc il est possible de
travailler avec directement.
Il est intéressant de noter qu'à la fois le prologue (\INS{push ebp / mov ebp,esp})
et l'épilogue (\INS{pop ebp}) de la fonction peuvent être facilement omis ici, mais
sans doute que GCC n'est pas assez bon pour effectuer une telle optimisation de la
taille du code.
Toutefois, il est préférable que de telles petites fonctions soient des \IT{fonctions
inlined} (\myref{inline_code}).
