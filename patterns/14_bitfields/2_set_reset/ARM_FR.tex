\subsubsection{ARM + \OptimizingKeilVI (\ARMMode)}

\begin{lstlisting}[caption=\OptimizingKeilVI (\ARMMode),style=customasmARM]
02 0C C0 E3          BIC     R0, R0, #0x200
01 09 80 E3          ORR     R0, R0, #0x4000
1E FF 2F E1          BX      LR
\end{lstlisting}

\myindex{ARM!\Instructions!BIC}
L'instruction \INS{BIC} (\IT{BItwise bit Clear}) est une instruction pour mettre
à zéro des bits spécifiques. Ceci est comme l'instruction \AND, mais avec un opérande
inversé. I.e., c'est analogue à la paire d'instructions \NOT+\AND.

\myindex{ARM!\Instructions!ORR}
\INS{ORR} est le \q{ou logique}, analogue à \OR en x86.

Jusqu'ici, c'est facile.

\subsubsection{ARM + \OptimizingKeilVI (\ThumbMode)}

\begin{lstlisting}[caption=\OptimizingKeilVI (\ThumbMode),style=customasmARM]
01 21 89 03          MOVS    R1, 0x4000
08 43                ORRS    R0, R1
49 11                ASRS    R1, R1, #5   ; génère 0x200 et le met dans R1
88 43                BICS    R0, R1
70 47                BX      LR
\end{lstlisting}

Il semble que Keil a décidé que le code en mode Thumb, pour générer \TT{0x200} à
partir de \TT{0x4000}, est plus compact que celui pour écrire \TT{0x200} dans un
registre arbitraire.
% TODO1 пример, как компилятор при помощи сдвигов оптизирует такое: a=0x1000; b=0x2000; c=0x4000, etc

\myindex{ARM!\Instructions!ASRS}

C'est pourquoi, avec l'aide de \INS{ASRS} (\ASRdesc), cette valeur est calculée comme
$\TT{0x4000} \gg 5$.

\subsubsection{ARM + \OptimizingXcodeIV (\ARMMode)}
\label{anomaly:LLVM}
\myindex{\CompilerAnomaly}

\begin{lstlisting}[caption=\OptimizingXcodeIV (\ARMMode),label=ARM_leaf_example3,style=customasmARM]
42 0C C0 E3          BIC             R0, R0, #0x4200
01 09 80 E3          ORR             R0, R0, #0x4000
1E FF 2F E1          BX              LR
\end{lstlisting}

Le code qui a été généré par LLVM, pourrait être quelque chose  comme ça sous la
forme de code source:

\begin{lstlisting}[style=customc]
    REMOVE_BIT (rt, 0x4200);
    SET_BIT (rt, 0x4000);
\end{lstlisting}

Et c'est exactement ce dont nous avons besoin.
Mais pourquoi \TT{0x4200}?
Peut-être que c'est un artéfact de l'optimiseur de LLVM\footnote{C'était LLVM build
2410.2.00 fourni avec Apple Xcode 4.6.3}.

Probablement une erreur de l'optimiseur du compilateur, mais le code généré fonctionne
malgré tout correctement.

Vous pouvez en savoir plus sur les anomalies de compilateur ici~(\myref{anomaly:Intel}).

\OptimizingXcodeIV pour le mode Thumb génère le même code.

\subsubsection{ARM: plus d'informations sur l'instruction \INS{BIC}}
\myindex{ARM!\Instructions!BIC}

Retravaillons légèrement l'exemple:

\begin{lstlisting}[style=customc]
int f(int a)
{
    int rt=a;

    REMOVE_BIT (rt, 0x1234);

    return rt;
};
\end{lstlisting}

Ensuite Keil 5.03 pour mode ARM avec optimisation fait:

\begin{lstlisting}[style=customasmARM]
f PROC
        BIC      r0,r0,#0x1000
        BIC      r0,r0,#0x234
        BX       lr
        ENDP
\end{lstlisting}

Il y a deux instructions \INS{BIC}, i.e., les bits \TT{0x1234} sont mis à zéro en
deux temps.

C'est parce qu'il n'est pas possible d'encoder \TT{0x1234} dans une instruction \INS{BIC},
mais il est possible d'encoder \TT{0x1000} et \TT{0x234}.

\subsubsection{ARM64: GCC (Linaro) 4.9 \Optimizing}

GCC en compilant \Optimizing pour ARM64 peut utiliser l'instruction \AND au
lieu de \INS{BIC}:

\begin{lstlisting}[caption=GCC (Linaro) 4.9 \Optimizing,style=customasmARM]
f:
	and	w0, w0, -513	; 0xFFFFFFFFFFFFFDFF
	orr	w0, w0, 16384	; 0x4000
	ret
\end{lstlisting}

\subsubsection{ARM64: GCC (Linaro) 4.9 \NonOptimizing}

GCC \NonOptimizing génère plus de code redondant, mais fonctionne comme celui optimisé:

\begin{lstlisting}[caption=GCC (Linaro) 4.9 \NonOptimizing,style=customasmARM]
f:
	sub	sp, sp, #32
	str	w0, [sp,12]
	ldr	w0, [sp,12]
	str	w0, [sp,28]
	ldr	w0, [sp,28]
	orr	w0, w0, 16384	; 0x4000
	str	w0, [sp,28]
	ldr	w0, [sp,28]
	and	w0, w0, -513	; 0xFFFFFFFFFFFFFDFF
	str	w0, [sp,28]
	ldr	w0, [sp,28]
	add	sp, sp, 32
	ret
\end{lstlisting}
