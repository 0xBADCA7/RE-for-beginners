\subsection{Setting and clearing specific bits: \ac{FPU} example}

\myindex{IEEE 754}

Here is how bits are located in the \Tfloat type in IEEE 754 form:

\bigskip
% a hack used here! http://tex.stackexchange.com/questions/73524/bytefield-package
\begin{center}
\begin{bytefield}{32}
	\bitheader[endianness=big]{0,22,23,30,31} \\
	\bitbox{1}{S} & 
	\bitbox{8}{%
		\RU{экспонента}%
		\EN{exponent}%
		\ES{exponente}%
		\PTBRph{}%
		\DEph{}\PLph{}%
		\ITAph{}%
		\FR{exposant}%
		\JPN{指数}
	} & 
	\bitbox{23}{%
		\RU{мантисса}%
		\EN{mantissa or fraction}%
		\ES{mantisa o fracci\'on}%
		\PTBRph{}%
		\DEph{}\PLph{}%
		\ITAph{}%
		\FR{mantisse ou fraction}%
		\JPN{対数または比}
	}
\end{bytefield}
\end{center}

\begin{center}
( S\EMDASH{}%
	\RU{знак}%
	\EN{sign}%
	\ES{signo}%
	\PTBRph{}%
	\DEph{}\PLph{}%
	\ITAph{}%
	\FR{signe}%
	\JPN{記号}
)
\end{center}


The sign of number is in the \ac{MSB}. 
Will it be possible to change the sign of a floating point number without any FPU instructions?

\lstinputlisting[style=customc]{patterns/14_bitfields/35_set_reset_FPU/abs.c}

We need this trickery in \CCpp to copy to/from \Tfloat value without actual conversion.
So there are three functions: my\_abs() resets \ac{MSB}; \TT{set\_sign()} sets \ac{MSB} and negate() flips it.

\XOR can be used to flip a bit: \myref{XOR_property}.

\subsubsection{x86}

The code is pretty straightforward:

\lstinputlisting[caption=\Optimizing MSVC 2012,style=customasmx86]{patterns/14_bitfields/35_set_reset_FPU/abs_MSVC2012_Ox.asm}

An input value of type \Tfloat is taken from the stack, but treated as an integer value.

\AND and \OR reset and set the desired bit.
\XOR flips it.

Finally, the modified value is loaded into \TT{ST0}, because floating-point numbers are returned in this register.

Now let's try optimizing MSVC 2012 for x64:

\lstinputlisting[caption=\Optimizing MSVC 2012 x64,style=customasmx86]{patterns/14_bitfields/35_set_reset_FPU/abs_MSVC2012_x64_Ox.asm}

\myindex{x86!\Instructions!BTR}
\myindex{x86!\Instructions!BTS}
\myindex{x86!\Instructions!BTC}

The input value is passed in \TT{XMM0}, then it is copied into the local stack and then we see 
some instructions that are new to us: \BTR, \BTS, \BTC.

These instructions are used for resetting (\BTR), setting (\BTS) and inverting (or complementing: \BTC) 
specific bits.
The 31st bit is \ac{MSB}, counting from 0.

Finally, the result is copied into \TT{XMM0}, because floating point values are returned through \TT{XMM0} in Win64
environment.

\subsubsection{MIPS}

GCC 4.4.5 for MIPS does mostly the same:

\lstinputlisting[caption=\Optimizing GCC 4.4.5 (IDA),style=customasmMIPS]{patterns/14_bitfields/35_set_reset_FPU/MIPS_O3_IDA_EN.lst}

\myindex{MIPS!\Instructions!LUI}

One single \LUI instruction is used to load 0x80000000 into a register, because 
\LUI is clearing the low 16 bits and these are zeros in the constant, so one \LUI without subsequent \ORI is enough.

\subsubsection{ARM}

\myparagraph{\OptimizingKeilVI (\ARMMode)}

\lstinputlisting[caption=\OptimizingKeilVI (\ARMMode),style=customasmARM]{patterns/14_bitfields/35_set_reset_FPU/abs_Keil_ARM_O3_EN.s}

So far so good.
\myindex{ARM!\Instructions!BIC}
\myindex{ARM!\Instructions!EOR}

ARM has the \BIC instruction, which explicitly clears specific bit(s).
\EOR is the ARM instruction name for \XOR 
(\q{Exclusive OR}).

\myparagraph{\OptimizingKeilVI (\ThumbMode)}

\lstinputlisting[caption=\OptimizingKeilVI (\ThumbMode),style=customasmx86]{patterns/14_bitfields/35_set_reset_FPU/abs_Keil_thumb_O3.s}

Thumb mode in ARM offers 16-bit instructions and not much data can be encoded in them, so here a 
\INS{MOVS}/\INS{LSLS} instruction pair is used for forming the 0x80000000 constant.
It works like this: $1<<31 = 0x80000000$.

\myindex{ARM!\Instructions!LSLS}
\myindex{ARM!\Instructions!LSRS}

The code of \TT{my\_abs} is weird and it effectively works like this expression: $(i<<1)>>1$.
This statement looks meaningless.
But nevertheless, when $input<<1$ is executed, the \ac{MSB} (sign bit) is just dropped.
When the subsequent $result>>1$ statement is executed, all bits are now in their own places,
but \ac{MSB} is zero, because all \q{new} bits appearing from the shift operations are always zeros.
That is how the \LSLS/\LSRS instruction pair clears \ac{MSB}.

\myparagraph{\Optimizing GCC 4.6.3 (Raspberry Pi, \ARMMode)}

\lstinputlisting[caption=\Optimizing GCC 4.6.3 for Raspberry Pi (\ARMMode),style=customasmARM]{patterns/14_bitfields/35_set_reset_FPU/raspberry_GCC_O3_ARM_mode_EN.lst}

Let's run Raspberry Pi Linux in QEMU and it emulates an ARM FPU, so S-registers are used here for floating point
numbers instead of R-registers.

\myindex{ARM!\Instructions!FMRS}

The \FMRS instruction copies data from \ac{GPR} to the FPU and back.

\TT{my\_abs()} and \TT{set\_sign()} looks as expected, but negate()?
Why is there \ADD instead of \XOR?

\myindex{ARM!\Instructions!XOR}
\myindex{ARM!\Instructions!ADD}
It's hard to believe, but the instruction 
\INS{ADD register, 0x80000000} works just like \\
\INS{XOR register, 0x80000000}.
First of all, what's our goal?
The goal is to flip the \ac{MSB}, so let's forget about the \XOR operation.
From school-level mathematics we may recall that adding values like 1000 to other values never affects
the last 3 digits.
For example: $1234567 + 10000 = 1244567$ (last 4 digits are never affected).

But here we operate in binary base and\\
0x80000000 is 0b100000000000000000000000000000000, i.e., only the highest bit is set.

Adding 0x80000000 to any value never affects the lowest 31 bits, but affects only the \ac{MSB}.
Adding 1 to 0 is resulting in 1.

Adding 1 to 1 is resulting in 0b10 in binary form, but the 32th bit (counting from zero) gets dropped, 
because our registers are 32 bit wide, so the result is 0.
That's why \XOR can be replaced by \ADD here.

It's hard to say why GCC decided to do this, but it works correctly.
