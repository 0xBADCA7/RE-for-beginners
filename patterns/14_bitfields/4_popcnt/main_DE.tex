\subsection{Gesetzte Bits zählen}
Hier ist ein einfaches Beispiel einer Funktion, die die Anzahl der gesetzten
Bits in einem Eingabewert zählt.

Diese Operation wird auch \q{population count}\footnote{moderne x86 CPUs
(die SSE4 unterstützen) haben zu diesem Zweck sogar einen eigenen POPCNT Befehl}
genannt.

\lstinputlisting[style=customc]{patterns/14_bitfields/4_popcnt/shifts.c}
In dieser Schleife wird der Wert von $i$ schrittweise von 0 bis 31 erhöht,
sodass der Ausdruck $1 \ll i$ von 1 bis \TT{0x80000000} zählt.
In natürlicher Sprache würden wir diese Operation als \IT{verschiebe 1 um n
Bits nach links} beschreiben.
Mit anderen Worten: Der Ausdruck $1 \ll i$ erzeugt alle möglichen Bitpositionen
in einer 32-Bit-Zahl.
Das freie Bit auf der rechten Seite wird jeweils gelöscht.

\label{2n_numbers_table}
Hier ist eine Tabelle mit allen Werten von $1 \ll i$ 
für $i=0 \ldots 31$:

\small
\begin{center}
\begin{tabular}{ | l | l | l | l | }
\hline
\HeaderColor \CCpp Ausdruck & 
\HeaderColor Zweierpotenz & 
\HeaderColor Dezimalzahl & 
\HeaderColor Hexadezimalzahl \\
\hline
$1 \ll 0$ & 1 & 1 & 1 \\
\hline
$1 \ll 1$ & $2^{1}$ & 2 & 2 \\
\hline
$1 \ll 2$ & $2^{2}$ & 4 & 4 \\
\hline
$1 \ll 3$ & $2^{3}$ & 8 & 8 \\
\hline
$1 \ll 4$ & $2^{4}$ & 16 & 0x10 \\
\hline
$1 \ll 5$ & $2^{5}$ & 32 & 0x20 \\
\hline
$1 \ll 6$ & $2^{6}$ & 64 & 0x40 \\
\hline
$1 \ll 7$ & $2^{7}$ & 128 & 0x80 \\
\hline
$1 \ll 8$ & $2^{8}$ & 256 & 0x100 \\
\hline
$1 \ll 9$ & $2^{9}$ & 512 & 0x200 \\
\hline
$1 \ll 10$ & $2^{10}$ & 1024 & 0x400 \\
\hline
$1 \ll 11$ & $2^{11}$ & 2048 & 0x800 \\
\hline
$1 \ll 12$ & $2^{12}$ & 4096 & 0x1000 \\
\hline
$1 \ll 13$ & $2^{13}$ & 8192 & 0x2000 \\
\hline
$1 \ll 14$ & $2^{14}$ & 16384 & 0x4000 \\
\hline
$1 \ll 15$ & $2^{15}$ & 32768 & 0x8000 \\
\hline
$1 \ll 16$ & $2^{16}$ & 65536 & 0x10000 \\
\hline
$1 \ll 17$ & $2^{17}$ & 131072 & 0x20000 \\
\hline
$1 \ll 18$ & $2^{18}$ & 262144 & 0x40000 \\
\hline
$1 \ll 19$ & $2^{19}$ & 524288 & 0x80000 \\
\hline
$1 \ll 20$ & $2^{20}$ & 1048576 & 0x100000 \\
\hline
$1 \ll 21$ & $2^{21}$ & 2097152 & 0x200000 \\
\hline
$1 \ll 22$ & $2^{22}$ & 4194304 & 0x400000 \\
\hline
$1 \ll 23$ & $2^{23}$ & 8388608 & 0x800000 \\
\hline
$1 \ll 24$ & $2^{24}$ & 16777216 & 0x1000000 \\
\hline
$1 \ll 25$ & $2^{25}$ & 33554432 & 0x2000000 \\
\hline
$1 \ll 26$ & $2^{26}$ & 67108864 & 0x4000000 \\
\hline
$1 \ll 27$ & $2^{27}$ & 134217728 & 0x8000000 \\
\hline
$1 \ll 28$ & $2^{28}$ & 268435456 & 0x10000000 \\
\hline
$1 \ll 29$ & $2^{29}$ & 536870912 & 0x20000000 \\
\hline
$1 \ll 30$ & $2^{30}$ & 1073741824 & 0x40000000 \\
\hline
$1 \ll 31$ & $2^{31}$ & 2147483648 & 0x80000000 \\
\hline
\end{tabular}
\end{center}
\normalsize
Diese Konstanten (Bismasken) tauchen im Code oft auf und ein Reverse Engineer
muss in der Lage sein, sie schnell und sicher zu erkennen.

Es dazu jedoch nicht notwendig, die Dezimalzahlen (Zweipotenzen) größer
65535 auswendig zu kennen. Die hexadezimalen Zahlen sind leicht zu merken.

Die Konstanten werden häufig verwendet um Flags einzelnen Bits zuzuordnen. 
Hier ist zum Beispiel ein Auszug aus \TT{ssl\_private.h} aus dem Quellcode von
Apache 2.4.6:

\begin{lstlisting}[style=customc]
/**
 * Define the SSL options
 */
#define SSL_OPT_NONE           (0)
#define SSL_OPT_RELSET         (1<<0)
#define SSL_OPT_STDENVVARS     (1<<1)
#define SSL_OPT_EXPORTCERTDATA (1<<3)
#define SSL_OPT_FAKEBASICAUTH  (1<<4)
#define SSL_OPT_STRICTREQUIRE  (1<<5)
#define SSL_OPT_OPTRENEGOTIATE (1<<6)
#define SSL_OPT_LEGACYDNFORMAT (1<<7)
\end{lstlisting}

Zurück zu unserem Beispiel.

Das Makro \TT{IS\_SET} prüft auf Anwesenheit von Bits in $a$.
\myindex{x86!\Instructions!AND}

Das Makro \TT{IS\_SET} entspricht dabei dem logischen (\IT{AND})
und gibt 0 zurück, wenn das entsprechende Bit nicht gesetzt ist, oder die
Bitmaske, wenn das Bit gesetzt ist.
Der Operator \IT{if()} wird in \CCpp ausgeführt, wenn der boolesche Ausdruck
nicht null ist (er könnte sogar 123456 sein), weshalb es meistens richtig
funktioniert.


% subsections
\subsubsection{x86}

\myparagraph{MSVC}

Kompilieren wir (MSVC 2010):

\lstinputlisting[caption=MSVC
2010,style=customasmx86]{patterns/14_bitfields/4_popcnt/shifts_MSVC_DE.asm}

\clearpage
\mysubparagraph{\olly}
\myindex{\olly}

Betrachten wir dieses Beispiel in \olly. 
Sei der Eingabewert dabei \TT{0x12345678}.

Für $i=1$ sehen wir, wie $i$ nach \ECX geladen wird: 

\begin{figure}[H]
\centering
\myincludegraphics{patterns/14_bitfields/4_popcnt/olly1_1.png}
\caption{\olly: $i=1$, $i$ wird nach \ECX geladen}
\label{fig:shifts_olly1_1}
\end{figure}

\EDX ist 1. \SHL wird jetzt ausgeführt.

\clearpage
\SHL wurde ausgeführt:

\begin{figure}[H]
\centering
\myincludegraphics{patterns/14_bitfields/4_popcnt/olly1_2.png}
\caption{\olly: $i=1$, \EDX=$1 \ll 1=2$}
\label{fig:shifts_olly1_2}
\end{figure}

\EDX enthält $1 \ll 1$ (oder 2). Hierbei handelt es sich um eine Bitmaske.

\clearpage
\AND setzt \ZF auf 1, was bedeutet, dass der Eingabewert (\TT{0x12345678}) 
mit 2 verUNDet wird. Das Ergebnis ist 0:

\begin{figure}[H]
\centering
\myincludegraphics{patterns/14_bitfields/4_popcnt/olly1_3.png}
\caption{\olly: $i=1$, 
ist hier das Bit im Eingabewert gesetzt? Nein. (\ZF=1)}
\label{fig:shifts_olly1_3}
\end{figure}
Es gibt hier also kein entsprechendes Bit im Eingabewert.

Das Codestück, welches den Zähler erhöht, wird also nicht ausgeführt:
Der \JZ Befehl überspringt es.

\clearpage
Verfolgen wir den Ablauf ein bisschen weiter bis $i$ den Wert 4 hat.
\SHL wird jetzt ausgeführt:

\begin{figure}[H]
\centering
\myincludegraphics{patterns/14_bitfields/4_popcnt/olly4_1.png}
\caption{\olly: $i=4$, $i$ wird nach \ECX geladen}
\label{fig:shifts_olly4_1}
\end{figure}

\clearpage
\EDX=$1 \ll 4$ (oder \TT{0x10} oder 16): 

\begin{figure}[H]
\centering
\myincludegraphics{patterns/14_bitfields/4_popcnt/olly4_2.png}
\caption{\olly: $i=4$, \EDX=$1 \ll 4=0x10$}
\label{fig:shifts_olly4_2}
\end{figure}

Hierbei handelt es sich um eine weitere Bitmaske.

\clearpage
\AND wird ausgeführt:

\begin{figure}[H]
\centering
\myincludegraphics{patterns/14_bitfields/4_popcnt/olly4_3.png}
\caption{\olly: $i=4$, 
ist hier das Bit im Eingabewert gesetzt? Ja. (\ZF=0)}
\label{fig:shifts_olly4_3}
\end{figure}

\ZF ist 0 , da das Bit im Eingabewert gesetzt ist.\\
Tatsächlich gilt \TT{0x12345678 \& 0x10 = 0x10}. 

Das Bit wird gezählt: der Sprung wird nicht ausgeführt und der Zähler wird
erhöht.

Die Funktion liefert den Wert 13 zurück.
Dies entspricht der Anzahl der in der binären Darstellung von \TT{0x12345678}
gesetzten Bits.



\myparagraph{GCC}

Kompilieren wir das Beispiel mit GCC 4.4.1:

\lstinputlisting[caption=GCC 4.4.1,style=customasmx86]{patterns/14_bitfields/4_popcnt/shifts_gcc.asm}


\subsubsection{x64}
\label{subsec:popcnt}
Verändern wir das Beispiel ein wenig um es auf 64 Bit zu erweitern:

\lstinputlisting[label=popcnt_x64_example,style=customc]{patterns/14_bitfields/4_popcnt/shifts64.c}

\myparagraph{\NonOptimizing GCC 4.8.2}

So weit, so einfach.

\lstinputlisting[caption=\NonOptimizing GCC
4.8.2,style=customasmx86]{patterns/14_bitfields/4_popcnt/shifts64_GCC_O0_DE.s}

\myparagraph{\Optimizing GCC 4.8.2}

\lstinputlisting[caption=\Optimizing GCC
4.8.2,numbers=left,label=shifts64_GCC_O3,style=customasmx86]{patterns/14_bitfields/4_popcnt/shifts64_GCC_O3_DE.s}
Dieser Code ist kürzer, birgt aber eine Besonderheit.

In allen bisher betrachteten Beispieln haben wir den Wert von \q{rt} nach dem
Vergleich mit einem speziellen Bit erhöht, aber dieser Code erhöht \q{rt} vorher
(Zeile 6) und schreibt den neuen Wert in das Register \EDX.
Dadurch überträgt der Befehl \CMOVNE\footnote{Conditional MOVe if Not Equal}
(der ein Synonym für \CMOVNZ\footnote{Conditional MOVe if Not Zero} ist) den
neuen Wert von \q{rt} durch Verschieben des Wertes in \EDX (vorgeschlagener
Wert von \q{rt}) nach \EAX (\q{aktueller Wert von rt}). Der in \EAX befindliche
Wert wird schließlich zurückgegeben.

Deshalb wird die Erhöhung des Zählers in jedem Durchlauf der Schleife
durchgeführt, d.h. 64 mal, ohne dass eine Abhängigkeit vom Eingabewert
besteht.

Der Vorteil dieses Code ist, dass er nur einen bedingten Sprung enthält (am
Ende der Schleife) anstatt zwei Sprüngen (Überspringen des Erhöhens von \q{rt}
und Ende der Schleife). 
Der Code ist somit auf modernen CPUs mit Branch Pedictors möglicherweise
schneller:\myref{branch_predictors}.

\label{FATRET}
\myindex{x86!\Instructions!FATRET}
Der letzte Befehl hier ist \INS{REP RET} (Opcode \TT{F3 C3}), der von MSVC auch
\INS{FATRET} genannt wird.
Hierbei handelt es sich um eine optimierte Version von \RET, die von ARM
bevorzugt am Ende der Funktion verwendet wird, wenn \RET direkt nach einem
bedingten Sprung folg:.
\InSqBrackets{\AMDOptimization p.15}
\footnote{Mehr Informationen dazu: \url{http://go.yurichev.com/17328}}.

\myparagraph{\Optimizing MSVC 2010}

\lstinputlisting[caption=\Optimizing MSVC
2010,style=customasmx86]{patterns/14_bitfields/4_popcnt/MSVC_2010_x64_Ox_DE.asm}

\myindex{x86!\Instructions!ROL}
Hier wird der Befehl \ROL anstelle von \SHL verwendet, welches einer
\q{Linksrotation} anstatt einer \q{Linksverschiebung} entspricht.
In diesem Beispiel entspricht \ROL einem \TT{SHL}.

Für mehr Informationen zu Rotationsbefehlen siehe: \myref{ROL_ROR}. 

\Reg{8} zählt hier von 64 auf 0 herunter. 
Dies entspricht dem invertierten $i$.

Hier ist eine Tabelle einiger Register während der Ausführung des Programms:

\begin{center}
\begin{tabular}{ | l | l | }
\hline
\HeaderColor RDX & \HeaderColor R8 \\
\hline
0x0000000000000001 & 64 \\
\hline
0x0000000000000002 & 63 \\
\hline
0x0000000000000004 & 62 \\
\hline
0x0000000000000008 & 61 \\
\hline
... & ... \\
\hline
0x4000000000000000 & 2 \\
\hline
0x8000000000000000 & 1 \\
\hline
\end{tabular}
\end{center}

\myindex{x86!\Instructions!FATRET}
Am Ende finden wir den Befehl \INS{FATRET}, der hier schon erklärt
wurde:\myref{FATRET}.

\myparagraph{\Optimizing MSVC 2012}

\lstinputlisting[caption=\Optimizing MSVC
2012,style=customasmx86]{patterns/14_bitfields/4_popcnt/MSVC_2012_x64_Ox_DE.asm}

\myindex{\CompilerAnomaly}
\label{MSVC2012_anomaly}
Der optimierende MSVC 2012 erzeugt fast den gleichen Code wie MSVC 2012,
generiert aber aus irgendeinem Grund zwei identischen Rümpfe für die Schleifen
und die Schleife zählt nun bis 32 anstatt 64.

Ehrlich gesagt, kann man nicht genau erklären warum. Es könnte sich um einen
Optimierungstrick handeln. Vielleicht ist es für den Rumpf der Schleife besser
ein wenig länger zu sein.

Trotzdem ist solcher Code relevant um zu zeigen, dass der Output des Compilers
manchmal sehr merkwürdig und unlogisch sein kann und dennoch tadellos
funktioniert.

\subsubsection{ARM + \OptimizingXcodeIV (\ARMMode)}

\lstinputlisting[caption=\OptimizingXcodeIV
(\ARMMode),label=ARM_leaf_example4,style=customasmARM]{patterns/14_bitfields/4_popcnt/ARM_Xcode_O3_DE.lst}

\myindex{ARM!\Instructions!TST}
\TST entspricht dem Befehl \TEST in x86.

\myindex{ARM!Optional operators!LSL}
\myindex{ARM!Optional operators!LSR}
\myindex{ARM!Optional operators!ASR}
\myindex{ARM!Optional operators!ROR}
\myindex{ARM!Optional operators!RRX}
\myindex{ARM!\Instructions!MOV}
\myindex{ARM!\Instructions!TST}
\myindex{ARM!\Instructions!CMP}
\myindex{ARM!\Instructions!ADD}
\myindex{ARM!\Instructions!SUB}
\myindex{ARM!\Instructions!RSB}
Wie bereits in~(\myref{shifts_in_ARM_mode}) besprochen gibt es zwei verschiedene
Schiebebefehle im ARM mode.
Zusätzlich gibt es aber noch die Suffixe
LSL (\IT{Logical Shift Left}), 
LSR (\IT{Logical Shift Right}), 
ASR (\IT{Arithmetic Shift Right}), 
ROR (\IT{Rotate Right}) und
RRX (\IT{Rotate Right with Extend}), die an Befehle wie \MOV, \TST,
\CMP, \ADD, \SUB, \RSB\footnote{\DataProcessingInstructionsFootNote} angehängt
werden können.

Diese Suffixe legen fest, wie und um wie viele Bits der zweite Operand
verschoben werden soll.

\myindex{ARM!\Instructions!TST}
\myindex{ARM!Optional operators!LSL}
Dadurch entspricht der Befehl \TT{\q{TST R1, R2,LSL R3}} hier 
$R1 \land (R2 \ll R3)$.

\subsubsection{ARM + \OptimizingXcodeIV (\ThumbTwoMode)}

\myindex{ARM!\Instructions!LSL.W}
\myindex{ARM!\Instructions!LSL}
Fast das gleiche, aber hier werden zwei \INS{LSL.W}/\TST Befehle anstelle eines
einzelnen \TST verwendet, da es im Thumb mode nicht möglich ist, den Suffix \LSL
direkt in \TST zu definieren.

\begin{lstlisting}[label=ARM_leaf_example5,style=customasmARM]
                MOV             R1, R0
                MOVS            R0, #0
                MOV.W           R9, #1
                MOVS            R3, #0
loc_2F7A
                LSL.W           R2, R9, R3
                TST             R2, R1
                ADD.W           R3, R3, #1
                IT NE
                ADDNE           R0, #1
                CMP             R3, #32
                BNE             loc_2F7A
                BX              LR
\end{lstlisting}

\subsubsection{ARM64 + \Optimizing GCC 4.9}
Betrachten wir ein 64-Bit-Beispiel, das wir bereits
kennen:\myref{popcnt_x64_example}.

\lstinputlisting[caption=\Optimizing GCC (Linaro)
4.8,style=customasmARM]{patterns/14_bitfields/4_popcnt/ARM64_GCC_O3_DE.s}
Das Ergebnis ist ähnlich dem was GCC für x64 erzeugt:\myref{shifts64_GCC_O3}.

\myindex{ARM!\Instructions!CSEL}
Der Befehl \CSEL steht für \q{Conditional SELect}.
Er wählt eine von zwei Variablen abhängig von den durch \TST gesetzen Flags aus
und kopiert deren Wert nach \RegW{2}, wo die Variable \q{rt} gespeichert wird.

\subsubsection{ARM64 + \NonOptimizing GCC 4.9}
Wieder werden wir hier mit dem bereits bekannten 64-Bit-Beispiel arbeiten:
\myref{popcnt_x64_example}.
Der Code ist umfangreicher als gewöhnlich.

\lstinputlisting[caption=\NonOptimizing GCC (Linaro)
4.8,style=customasmARM]{patterns/14_bitfields/4_popcnt/ARM64_GCC_O0_DE.s}


\subsubsection{MIPS}

\myparagraph{\NonOptimizing GCC}

\lstinputlisting[caption=\NonOptimizing GCC 4.4.5
(IDA),style=customasmMIPS]{patterns/14_bitfields/4_popcnt/MIPS_O0_IDA_DE.lst}

\myindex{MIPS!\Instructions!SLL}
\myindex{MIPS!\Instructions!SLLV}
Umständlich: alle lokalen Variablen liegen auf dem lokalen Stack und werden bei
jedem Zugriff neu geladen.

Der Befehl \SLLV bedeutet \q{Shift Word Left Logical Variable}. Er unterscheidet
sich von \SLL nur dadurch, dass die Anzahl der Verschiebungen im \SLL Befehl
kodiert sind (und dadurch nicht veränderbar). \SLLV hingegen erhält die Anzahl
der Verschiebungen aus einem Register.

\myparagraph{\Optimizing GCC}
Hier ist es knapper gehalten.
Warum aber gibt es zwei Schiebebefehle anstatt eines?

Es ist möglich, den ersten \SLLV Befehl durch einen unbedingte
Verzweigungsbefehl zu ersetzen, der direkt zum zweiten \SLLV springt. 
Das zieht aber einen zweiten Verzweigungsbefehl nach sich und es ist stets
vorteilhaft sich dieser zu entledigen:\myref{branch_predictors}.

\lstinputlisting[caption=\Optimizing GCC 4.4.5
(IDA),style=customasmMIPS]{patterns/14_bitfields/4_popcnt/MIPS_O3_IDA_DE.lst}


