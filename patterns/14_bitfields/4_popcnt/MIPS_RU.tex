\subsubsection{MIPS}

\myparagraph{\NonOptimizing GCC}

\lstinputlisting[caption=\NonOptimizing GCC 4.4.5 (IDA),style=customasmMIPS]{patterns/14_bitfields/4_popcnt/MIPS_O0_IDA_RU.lst}

\myindex{MIPS!\Instructions!SLL}
\myindex{MIPS!\Instructions!SLLV}
Это многословно: все локальные переменные расположены в локальном стеке и перезагружаются каждый раз,
когда нужны.
Инструкция \SLLV это \q{Shift Word Left Logical Variable}, она отличается от \SLL только тем что
количество бит для сдвига кодируется в \SLL (и, следовательно, фиксировано), а \SLL берет количество из регистра.

\myparagraph{\Optimizing GCC}

Это более сжато.
Здесь две инструкции сдвигов вместо одной.
Почему?
Можно заменить первую инструкцию \SLLV на инструкцию безусловного перехода, передав управление прямо
на вторую \SLLV.

Но это ещё одна инструкция перехода в функции, а от них избавляться всегда выгодно: \myref{branch_predictors}.

\lstinputlisting[caption=\Optimizing GCC 4.4.5 (IDA),style=customasmMIPS]{patterns/14_bitfields/4_popcnt/MIPS_O3_IDA_RU.lst}
