\subsubsection{ARM + \OptimizingXcodeIV (\ARMMode)}

\lstinputlisting[caption=\OptimizingXcodeIV
(\ARMMode),label=ARM_leaf_example4,style=customasmARM]{patterns/14_bitfields/4_popcnt/ARM_Xcode_O3_DE.lst}

\myindex{ARM!\Instructions!TST}
\TST entspricht dem Befehl \TEST in x86.

\myindex{ARM!Optional operators!LSL}
\myindex{ARM!Optional operators!LSR}
\myindex{ARM!Optional operators!ASR}
\myindex{ARM!Optional operators!ROR}
\myindex{ARM!Optional operators!RRX}
\myindex{ARM!\Instructions!MOV}
\myindex{ARM!\Instructions!TST}
\myindex{ARM!\Instructions!CMP}
\myindex{ARM!\Instructions!ADD}
\myindex{ARM!\Instructions!SUB}
\myindex{ARM!\Instructions!RSB}
Wie bereits in~(\myref{shifts_in_ARM_mode}) besprochen gibt es zwei verschiedene
Schiebebefehle im ARM mode.
Zusätzlich gibt es aber noch die Suffixe
LSL (\IT{Logical Shift Left}), 
LSR (\IT{Logical Shift Right}), 
ASR (\IT{Arithmetic Shift Right}), 
ROR (\IT{Rotate Right}) und
RRX (\IT{Rotate Right with Extend}), die an Befehle wie \MOV, \TST,
\CMP, \ADD, \SUB, \RSB\footnote{\DataProcessingInstructionsFootNote} angehängt
werden können.

Diese Suffixe legen fest, wie und um wie viele Bits der zweite Operand
verschoben werden soll.

\myindex{ARM!\Instructions!TST}
\myindex{ARM!Optional operators!LSL}
Dadurch entspricht der Befehl \TT{\q{TST R1, R2,LSL R3}} hier 
$R1 \land (R2 \ll R3)$.

\subsubsection{ARM + \OptimizingXcodeIV (\ThumbTwoMode)}

\myindex{ARM!\Instructions!LSL.W}
\myindex{ARM!\Instructions!LSL}
Fast das gleiche, aber hier werden zwei \INS{LSL.W}/\TST Befehle anstelle eines
einzelnen \TST verwendet, da es im Thumb mode nicht möglich ist, den Suffix \LSL
direkt in \TST zu definieren.

\begin{lstlisting}[label=ARM_leaf_example5,style=customasmARM]
                MOV             R1, R0
                MOVS            R0, #0
                MOV.W           R9, #1
                MOVS            R3, #0
loc_2F7A
                LSL.W           R2, R9, R3
                TST             R2, R1
                ADD.W           R3, R3, #1
                IT NE
                ADDNE           R0, #1
                CMP             R3, #32
                BNE             loc_2F7A
                BX              LR
\end{lstlisting}

\subsubsection{ARM64 + \Optimizing GCC 4.9}
Betrachten wir ein 64-Bit-Beispiel, das wir bereits
kennen:\myref{popcnt_x64_example}.

\lstinputlisting[caption=\Optimizing GCC (Linaro)
4.8,style=customasmARM]{patterns/14_bitfields/4_popcnt/ARM64_GCC_O3_DE.s}
Das Ergebnis ist ähnlich dem was GCC für x64 erzeugt:\myref{shifts64_GCC_O3}.

\myindex{ARM!\Instructions!CSEL}
Der Befehl \CSEL steht für \q{Conditional SELect}.
Er wählt eine von zwei Variablen abhängig von den durch \TST gesetzen Flags aus
und kopiert deren Wert nach \RegW{2}, wo die Variable \q{rt} gespeichert wird.

\subsubsection{ARM64 + \NonOptimizing GCC 4.9}
Wieder werden wir hier mit dem bereits bekannten 64-Bit-Beispiel arbeiten:
\myref{popcnt_x64_example}.
Der Code ist umfangreicher als gewöhnlich.

\lstinputlisting[caption=\NonOptimizing GCC (Linaro)
4.8,style=customasmARM]{patterns/14_bitfields/4_popcnt/ARM64_GCC_O0_DE.s}

