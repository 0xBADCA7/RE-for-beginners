\subsubsection{ARM + \OptimizingXcodeIV (\ARMMode)}

\lstinputlisting[caption=\OptimizingXcodeIV (\ARMMode),label=ARM_leaf_example4]{patterns/14_bitfields/4_popcnt/ARM_Xcode_O3_RU.lst}

\myindex{ARM!\Instructions!TST}
\TST это то же что и \TEST в x86.

\myindex{ARM!Optional operators!LSL}
\myindex{ARM!Optional operators!LSR}
\myindex{ARM!Optional operators!ASR}
\myindex{ARM!Optional operators!ROR}
\myindex{ARM!Optional operators!RRX}
\myindex{ARM!\Instructions!MOV}
\myindex{ARM!\Instructions!TST}
\myindex{ARM!\Instructions!CMP}
\myindex{ARM!\Instructions!ADD}
\myindex{ARM!\Instructions!SUB}
\myindex{ARM!\Instructions!RSB}
Как уже было указано~(\myref{shifts_in_ARM_mode}),
в режиме ARM нет отдельной инструкции для сдвигов.

Однако, модификаторами 
LSL (\IT{Logical Shift Left}), 
LSR (\IT{Logical Shift Right}), 
ASR (\IT{Arithmetic Shift Right}), 
ROR (\IT{Rotate Right}) и
RRX (\IT{Rotate Right with Extend}) можно дополнять некоторые инструкции, такие как \MOV, \TST,
\CMP, \ADD, \SUB, \RSB\footnote{\DataProcessingInstructionsFootNote}.

Эти модификаторы указывают, как сдвигать второй операнд, и на сколько.

\myindex{ARM!\Instructions!TST}
\myindex{ARM!Optional operators!LSL}
Таким образом, инструкция  \TT{\q{TST R1, R2,LSL R3}} здесь работает как $R1 \land (R2 \ll R3)$.

\subsubsection{ARM + \OptimizingXcodeIV (\ThumbTwoMode)}

\myindex{ARM!\Instructions!LSL.W}
\myindex{ARM!\Instructions!LSL}
Почти такое же, только здесь применяется пара инструкций \INS{LSL.W}/\TST вместо одной \TST,
ведь в режиме Thumb нельзя добавлять модификатор \LSL прямо в \TST.

\begin{lstlisting}[label=ARM_leaf_example5]
                MOV             R1, R0
                MOVS            R0, #0
                MOV.W           R9, #1
                MOVS            R3, #0
loc_2F7A
                LSL.W           R2, R9, R3
                TST             R2, R1
                ADD.W           R3, R3, #1
                IT NE
                ADDNE           R0, #1
                CMP             R3, #32
                BNE             loc_2F7A
                BX              LR
\end{lstlisting}

\subsubsection{ARM64 + \Optimizing GCC 4.9}

Возьмем 64-битный пример, который уже был здесь использован: \myref{popcnt_x64_example}.

\lstinputlisting[caption=\Optimizing GCC (Linaro) 4.8,style=customasm]{patterns/14_bitfields/4_popcnt/ARM64_GCC_O3_RU.s}
Результат очень похож на тот, что GCC сгенерировал для x64: \myref{shifts64_GCC_O3}.

\myindex{ARM!\Instructions!CSEL}
Инструкция \CSEL это \q{Conditional SELect} (выбор при условии). 
Она просто выбирает одну из переменных, в зависимости от флагов выставленных \TST и копирует значение в регистр \RegW{2}, содержащий переменную \q{rt}.

\subsubsection{ARM64 + \NonOptimizing GCC 4.9}

И снова будем использовать 64-битный пример, который мы использовали ранее: \myref{popcnt_x64_example}.
Код более многословный, как обычно.

\lstinputlisting[caption=\NonOptimizing GCC (Linaro) 4.8,style=customasm]{patterns/14_bitfields/4_popcnt/ARM64_GCC_O0_RU.s}

