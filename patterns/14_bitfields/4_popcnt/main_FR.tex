\subsection{Compter les bits mis à 1}

Voici un exemple simple d'une fonction qui compte le nombre de bits mis à 1 dans
la valeur en entrée.

Cette opération est aussi appelée \q{population count}\footnote{les CPUs x86 modernes
(qui supportent SSE4) ont même une instruction POPCNT pour cela}.

\lstinputlisting[style=customc]{patterns/14_bitfields/4_popcnt/shifts.c}

Dans cette boucle, la variable d'itération $i$ prend les valeurs de 0 à 31, donc
la déclaration $1 \ll i$ prend les valeurs de 1 à \TT{0x80000000}.
Pour décrire cette opération en langage naturel, nous dirions \IT{décaler 1 par n bits à gauche}.
En d'autres mots, la déclaration $1 \ll i$ produit consécutivement toutes les positions
possible pour un bit dans un nombre de 32-bit.
Le bit libèré à droite est toujours à 0.

\label{2n_numbers_table}
Voici une table de tous les $1 \ll i$ possible
for $i=0 \ldots 31$:

\small
\begin{center}
\begin{tabular}{ | l | l | l | l | }
\hline
\HeaderColor \CCpp expression & 
\HeaderColor Puissance de deux & 
\HeaderColor Forme décimale & 
\HeaderColor Forme hexadécimale \\
\hline
$1 \ll 0$ & 1 & 1 & 1 \\
\hline
$1 \ll 1$ & $2^{1}$ & 2 & 2 \\
\hline
$1 \ll 2$ & $2^{2}$ & 4 & 4 \\
\hline
$1 \ll 3$ & $2^{3}$ & 8 & 8 \\
\hline
$1 \ll 4$ & $2^{4}$ & 16 & 0x10 \\
\hline
$1 \ll 5$ & $2^{5}$ & 32 & 0x20 \\
\hline
$1 \ll 6$ & $2^{6}$ & 64 & 0x40 \\
\hline
$1 \ll 7$ & $2^{7}$ & 128 & 0x80 \\
\hline
$1 \ll 8$ & $2^{8}$ & 256 & 0x100 \\
\hline
$1 \ll 9$ & $2^{9}$ & 512 & 0x200 \\
\hline
$1 \ll 10$ & $2^{10}$ & 1024 & 0x400 \\
\hline
$1 \ll 11$ & $2^{11}$ & 2048 & 0x800 \\
\hline
$1 \ll 12$ & $2^{12}$ & 4096 & 0x1000 \\
\hline
$1 \ll 13$ & $2^{13}$ & 8192 & 0x2000 \\
\hline
$1 \ll 14$ & $2^{14}$ & 16384 & 0x4000 \\
\hline
$1 \ll 15$ & $2^{15}$ & 32768 & 0x8000 \\
\hline
$1 \ll 16$ & $2^{16}$ & 65536 & 0x10000 \\
\hline
$1 \ll 17$ & $2^{17}$ & 131072 & 0x20000 \\
\hline
$1 \ll 18$ & $2^{18}$ & 262144 & 0x40000 \\
\hline
$1 \ll 19$ & $2^{19}$ & 524288 & 0x80000 \\
\hline
$1 \ll 20$ & $2^{20}$ & 1048576 & 0x100000 \\
\hline
$1 \ll 21$ & $2^{21}$ & 2097152 & 0x200000 \\
\hline
$1 \ll 22$ & $2^{22}$ & 4194304 & 0x400000 \\
\hline
$1 \ll 23$ & $2^{23}$ & 8388608 & 0x800000 \\
\hline
$1 \ll 24$ & $2^{24}$ & 16777216 & 0x1000000 \\
\hline
$1 \ll 25$ & $2^{25}$ & 33554432 & 0x2000000 \\
\hline
$1 \ll 26$ & $2^{26}$ & 67108864 & 0x4000000 \\
\hline
$1 \ll 27$ & $2^{27}$ & 134217728 & 0x8000000 \\
\hline
$1 \ll 28$ & $2^{28}$ & 268435456 & 0x10000000 \\
\hline
$1 \ll 29$ & $2^{29}$ & 536870912 & 0x20000000 \\
\hline
$1 \ll 30$ & $2^{30}$ & 1073741824 & 0x40000000 \\
\hline
$1 \ll 31$ & $2^{31}$ & 2147483648 & 0x80000000 \\
\hline
\end{tabular}
\end{center}
\normalsize

Ces constantes (masques de bit) apparaissent très souvent le code et un rétro-ingénieur
pratiquant doit pouvoir les repérer rapidement.

Les nombres décimaux avant 65536 et les hexadécimaux sont faciles à mémoriser.
Tandis que les nombres décimaux après 65536 ne valent probablement pas la peine de
l'être.

Ces constantes sont utilisées très souvent pour mapper des flags sur des bits spécifiques.
Par exemple, voici un extrait de \TT{ssl\_private.h} du code source d'Apache 2.4.6:

\begin{lstlisting}[style=customc]
/**
 * Define the SSL options
 */
#define SSL_OPT_NONE           (0)
#define SSL_OPT_RELSET         (1<<0)
#define SSL_OPT_STDENVVARS     (1<<1)
#define SSL_OPT_EXPORTCERTDATA (1<<3)
#define SSL_OPT_FAKEBASICAUTH  (1<<4)
#define SSL_OPT_STRICTREQUIRE  (1<<5)
#define SSL_OPT_OPTRENEGOTIATE (1<<6)
#define SSL_OPT_LEGACYDNFORMAT (1<<7)
\end{lstlisting}

Revenons à notre exemple.

La macro \TT{IS\_SET} teste la présence d'un bit dans $a$.
\myindex{x86!\Instructions!AND}

La macro \TT{IS\_SET} est en fait l'opération logique AND (\IT{AND}) et elle renvoie
0 si le bit testé est absent (à 0), ou le masque de bit, si le bit est présent (à 1).
L'opérateur \IT{if()} en \CCpp exécute son code si l'expression n'est pas zéro,
cela peut même être 123456, c'est pourquoi il fonctionne toujours correctement.

% subsections
\subsubsection{x86}

\myparagraph{MSVC}

Compilons-le (MSVC 2010):

\lstinputlisting[caption=MSVC 2010,style=customasmx86]{patterns/14_bitfields/4_popcnt/shifts_MSVC_FR.asm}

\clearpage
\mysubparagraph{\olly}
\myindex{\olly}

Chargeons cet exemple dans \olly.
Définissons la valeur d'entrée à \TT{0x12345678}.

Pour $i=1$, nous voyons comment $i$ est chargé dans \ECX:

\begin{figure}[H]
\centering
\myincludegraphics{patterns/14_bitfields/4_popcnt/olly1_1.png}
\caption{\olly: $i=1$, $i$ est chargé dans \ECX}
\label{fig:shifts_olly1_1}
\end{figure}

\EDX contient 1. \SHL va être exécuté maintenant.

\clearpage
\SHL a été exécuté:

\begin{figure}[H]
\centering
\myincludegraphics{patterns/14_bitfields/4_popcnt/olly1_2.png}
\caption{\olly: $i=1$, \EDX=$1 \ll 1=2$}
\label{fig:shifts_olly1_2}
\end{figure}

\EDX contient $1 \ll 1$ (ou 2). Ceci est un masque de bit.

\clearpage
\AND met \ZF à 1, ca qui implique que la valeur en entrée (\TT{0x12345678}) ANDée
avec 2 donne 0:

\begin{figure}[H]
\centering
\myincludegraphics{patterns/14_bitfields/4_popcnt/olly1_3.png}
\caption{\olly: $i=1$, 
y a-t-il ce bit dans la valeur en entrée? Non. (\ZF=1)}
\label{fig:shifts_olly1_3}
\end{figure}

Donc, il n'y a pas le bit correspondant dans la valeur en entrée.

Le morceau de code, qui \glslink{increment}{incrémente} le compteur ne va pas être
exécuté:
l'instruction \JZ l'évite.

\clearpage
Avançons un peu plus et $i$ vaut maintenant 4.
\SHL va être exécuté maintenant:

\begin{figure}[H]
\centering
\myincludegraphics{patterns/14_bitfields/4_popcnt/olly4_1.png}
\caption{\olly: $i=4$, $i$ est chargée dans \ECX}
\label{fig:shifts_olly4_1}
\end{figure}

\clearpage
\EDX=$1 \ll 4$ (ou \TT{0x10} ou 16):

\begin{figure}[H]
\centering
\myincludegraphics{patterns/14_bitfields/4_popcnt/olly4_2.png}
\caption{\olly: $i=4$, \EDX=$1 \ll 4=0x10$}
\label{fig:shifts_olly4_2}
\end{figure}

Ceci est un autre masque de bit.

\clearpage
\AND est exécuté:

\begin{figure}[H]
\centering
\myincludegraphics{patterns/14_bitfields/4_popcnt/olly4_3.png}
\caption{\olly: $i=4$, 
y a-t-il ce bit dans la valeur en entrée? Oui. (\ZF=0)}
\label{fig:shifts_olly4_3}
\end{figure}

\ZF est à 0 car ce bit est présent dans la valeur en entrée.\\
En effet, \TT{0x12345678 \& 0x10 = 0x10}.

Ce bit compte: le saut n'est pas effectué et le compteur de bit est
\glslink{increment}{incrémenté}.

La fonction renvoie 13.
C'est le nombre total de bits à 1 dans \TT{0x12345678}.



\myparagraph{GCC}

Compilons-le avec GCC 4.4.1:

\lstinputlisting[caption=GCC 4.4.1,style=customasmx86]{patterns/14_bitfields/4_popcnt/shifts_gcc.asm}


\subsubsection{x64}
\label{subsec:popcnt}

Modifions légèrement l'exemple pour l'étendre à 64-bit:

\lstinputlisting[label=popcnt_x64_example,style=customc]{patterns/14_bitfields/4_popcnt/shifts64.c}

\myparagraph{GCC 4.8.2 \NonOptimizing}

Jusqu'ici, c'est facile.

\lstinputlisting[caption=GCC 4.8.2 \NonOptimizing,style=customasmx86]{patterns/14_bitfields/4_popcnt/shifts64_GCC_O0_FR.s}

\myparagraph{GCC 4.8.2 \Optimizing}

\lstinputlisting[caption=GCC 4.8.2 \Optimizing,numbers=left,label=shifts64_GCC_O3,style=customasmx86]{patterns/14_bitfields/4_popcnt/shifts64_GCC_O3_FR.s}

Ce code est plus concis, mais a une particularité.

% TODO: comment traduire commits ?
Dans tous les exemples que nous avons vu jusqu'ici, nous incrémentions la valeur
de \q{rt} après la comparaison d'un bit spécifique, mais le code ici incrémente \q{rt}
avant (ligne 6), écrivant la nouvelle valeur dans le registre \EDX.
Donc, si le dernier bit est à 1, l'instruction \CMOVNE\footnote{Conditional MOVe if Not Equal}
(qui est un synonyme pour \CMOVNZ\footnote{Conditional MOVe if Not Zero}) \IT{commits}
la nouvelle valeur de \q{rt} en déplaçant  \EDX (\q{valeur proposée de rt}) dans
\EAX (\q{rt courant} qui va être retourné à la fin).

C'est pourquoi l'incrémentation est effectuée à chaque étape de la boucle, i.e.,
64 fois, sans relation avec la valeur en entrée.

L'avantage de ce code est qu'il contient seulement un saut conditionnel (à la fin
de la boucle) au lieu de deux sauts (évitant l'incrément de la valeur de \q{rt} et
à la fin de la boucle).
Et cela doit s'exécuter plus vite sur les CPUs modernes avec des prédicteurs de branchement:
\myref{branch_predictors}.

\label{FATRET}
\myindex{x86!\Instructions!FATRET}
La dernière instruction est \INS{REP RET} (opcode \TT{F3 C3}) qui est aussi appelée
\INS{FATRET} par MSVC.
C'est en quelque sorte une version optimisée de \RET, qu'AMD recommande de mettre
en fin de fonction, si \RET se trouve juste après un saut conditionnel:
\InSqBrackets{\AMDOptimization p.15}
\footnote{Lire aussi à ce propos: \url{http://go.yurichev.com/17328}}.

\myparagraph{MSVC 2010 \Optimizing}

\lstinputlisting[caption=MSVC 2010 \Optimizing,style=customasmx86]{patterns/14_bitfields/4_popcnt/MSVC_2010_x64_Ox_FR.asm}

\myindex{x86!\Instructions!ROL}
Ici l'instruction \ROL est utilisée au lieu de \SHL, qui est en fait \q{rotate left /
pivoter à gauche} au lieu de \q{shift left / décaler à gauche}, mais dans cet exemple
elle fonctionne tout comme \TT{SHL}.

Vous pouvez en lire plus sur l'instruction de rotation ici: \myref{ROL_ROR}.

\Reg{8} ici est compté de 64 à 0.
C'est tout comme un $i$ inversé.

Voici une table de quelques registres pendant l'exécution:

\begin{center}
\begin{tabular}{ | l | l | }
\hline
\HeaderColor RDX & \HeaderColor R8 \\
\hline
0x0000000000000001 & 64 \\
\hline
0x0000000000000002 & 63 \\
\hline
0x0000000000000004 & 62 \\
\hline
0x0000000000000008 & 61 \\
\hline
... & ... \\
\hline
0x4000000000000000 & 2 \\
\hline
0x8000000000000000 & 1 \\
\hline
\end{tabular}
\end{center}

\myindex{x86!\Instructions!FATRET}
À la fin, nous voyons l'instruction \INS{FATRET}, qui a été expliquée ici: \myref{FATRET}.

\myparagraph{MSVC 2012 \Optimizing}

\lstinputlisting[caption=MSVC 2012 \Optimizing,style=customasmx86]{patterns/14_bitfields/4_popcnt/MSVC_2012_x64_Ox_FR.asm}

\myindex{\CompilerAnomaly}
\label{MSVC2012_anomaly}
MSVC 2012 \Optimizing fait presque le même job que MSVC 2010 \Optimizing, mais en
quelque sorte, il génère deux corps de boucles identiques et le nombre de boucles
est maintenant 32 au lieu de 64.

Pour être honnête, il n'est pas possible de dire pourquoi. Une ruse d'optimisation?
Peut-être est-il meilleur pour le corps de la boucle d'être légèrement plus long?

De toute façon, ce genre de code est pertinent ici pour montrer que parfois la sortie
du compilateur peut être vraiment bizarre et illogique, mais fonctionner parfaitement.


\subsubsection{ARM + \OptimizingXcodeIV (\ARMMode)}

\lstinputlisting[caption=\OptimizingXcodeIV (\ARMMode),label=ARM_leaf_example4,style=customasmARM]{patterns/14_bitfields/4_popcnt/ARM_Xcode_O3_FR.lst}

\myindex{ARM!\Instructions!TST}
\TST est la même chose que \TEST en x86.

\myindex{ARM!Optional operators!LSL}
\myindex{ARM!Optional operators!LSR}
\myindex{ARM!Optional operators!ASR}
\myindex{ARM!Optional operators!ROR}
\myindex{ARM!Optional operators!RRX}
\myindex{ARM!\Instructions!MOV}
\myindex{ARM!\Instructions!TST}
\myindex{ARM!\Instructions!CMP}
\myindex{ARM!\Instructions!ADD}
\myindex{ARM!\Instructions!SUB}
\myindex{ARM!\Instructions!RSB}
Comme noté précédemment~(\myref{shifts_in_ARM_mode}), il n'y a pas d'instruction
de décalage séparée en mode ARM
Toutefois, il y a ces modificateurs
LSL (\IT{Logical Shift Left / décalage logique à gauche}), 
LSR (\IT{Logical Shift Right / décalage logique à droite}), 
ASR (\IT{Arithmetic Shift Right décalage arithmétique à droite}), 
ROR (\IT{Rotate Right / rotation à droite}) et
RRX (\IT{Rotate Right with Extend / rotation à droite avec extension}),
qui peuvent être ajoutés à des instructions comme \MOV, \TST,
\CMP, \ADD, \SUB, \RSB\footnote{\DataProcessingInstructionsFootNote}.

Ces modificateurs définissent comment décaler le second opérande et de combien de
bits.

\myindex{ARM!\Instructions!TST}
\myindex{ARM!Optional operators!LSL}
Ainsi l'instruction \TT{\q{TST R1, R2,LSL R3}} fonctionne ici comme $R1 \land (R2
\ll R3)$.

\subsubsection{ARM + \OptimizingXcodeIV (\ThumbTwoMode)}

\myindex{ARM!\Instructions!LSL.W}
\myindex{ARM!\Instructions!LSL}
Presque la même, mais ici il y a deux instructions utilisées, \INS{LSL.W}/\TST, au
lieu d'une seule \TST, car en mode Thumb il n'est pas possible de définir le modificateur
\LSL directement dans \TST.

\begin{lstlisting}[label=ARM_leaf_example5,style=customasmARM]
                MOV             R1, R0
                MOVS            R0, #0
                MOV.W           R9, #1
                MOVS            R3, #0
loc_2F7A
                LSL.W           R2, R9, R3
                TST             R2, R1
                ADD.W           R3, R3, #1
                IT NE
                ADDNE           R0, #1
                CMP             R3, #32
                BNE             loc_2F7A
                BX              LR
\end{lstlisting}

\subsubsection{ARM64 + GCC 4.9 \Optimizing}

Prenons un exemple en 64.bit qui a déjà été utilisé: \myref{popcnt_x64_example}.

\lstinputlisting[caption=GCC (Linaro) 4.8 \Optimizing,style=customasmARM]{patterns/14_bitfields/4_popcnt/ARM64_GCC_O3_FR.s}

Le résultat est très semblable à ce que GCC génère pour x64: \myref{shifts64_GCC_O3}.

\myindex{ARM!\Instructions!CSEL}
L'instruction \CSEL signifie \q{Conditional SELect / sélection conditionnelle}.
Elle choisi une des deux variables en fonction des flags mis par \TST et copie la
valeur dans \RegW{2}, qui contient la variable \q{rt}.

\subsubsection{ARM64 + GCC 4.9 \NonOptimizing}

De nouveau, nous travaillons sur un exemple 64-bit qui a déjà été utilisé: \myref{popcnt_x64_example}.
Le code est plus verbeux, comme d'habitude.

\lstinputlisting[caption=\NonOptimizing GCC (Linaro) 4.8,style=customasmARM]{patterns/14_bitfields/4_popcnt/ARM64_GCC_O0_FR.s}


\subsubsection{MIPS}

\myparagraph{GCC \NonOptimizing}

\lstinputlisting[caption=GCC 4.4.5 \NonOptimizing (IDA),style=customasmMIPS]{patterns/14_bitfields/4_popcnt/MIPS_O0_IDA_FR.lst}

\myindex{MIPS!\Instructions!SLL}
\myindex{MIPS!\Instructions!SLLV}

C'est très verbeux: toutes les variables locales sont situées dans la pile locale
et rechargées à chaque fois que l'on en a besoin.

L'instruction \SLLV est \q{Shift Word Left Logical Variable}, elle diffère de \SLL
seulement de ce que la valeur du décalage est encodée dans l'instruction \SLL (et
par conséquent fixée) mais \SLLV lit cette valeur depuis un registre.

\myparagraph{GCC \Optimizing}

C'est plus concis.
Il y a deux instructions de décalage au lieu d'une.
Pourquoi?

Il est possible de remplacer la première instruction \SLLV avec une instruction de
branchement inconditionnel qui saute directement au second \SLLV.
Mais cela ferait une autre instruction de branchement dans la fonction, et il est
toujours favorable de s'en passer: \myref{branch_predictors}.

\lstinputlisting[caption=GCC 4.4.5 \Optimizing (IDA),style=customasmMIPS]{patterns/14_bitfields/4_popcnt/MIPS_O3_IDA_FR.lst}


