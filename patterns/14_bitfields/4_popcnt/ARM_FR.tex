\subsubsection{ARM + \OptimizingXcodeIV (\ARMMode)}

\lstinputlisting[caption=\OptimizingXcodeIV (\ARMMode),label=ARM_leaf_example4,style=customasmARM]{patterns/14_bitfields/4_popcnt/ARM_Xcode_O3_FR.lst}

\myindex{ARM!\Instructions!TST}
\TST est la même chose que \TEST en x86.

\myindex{ARM!Optional operators!LSL}
\myindex{ARM!Optional operators!LSR}
\myindex{ARM!Optional operators!ASR}
\myindex{ARM!Optional operators!ROR}
\myindex{ARM!Optional operators!RRX}
\myindex{ARM!\Instructions!MOV}
\myindex{ARM!\Instructions!TST}
\myindex{ARM!\Instructions!CMP}
\myindex{ARM!\Instructions!ADD}
\myindex{ARM!\Instructions!SUB}
\myindex{ARM!\Instructions!RSB}
Comme noté précedemment~(\myref{shifts_in_ARM_mode}), il n'y a pas d'instruction
de décalage séparée en mode ARM
Toutfois, il y a ces modificateurs
LSL (\IT{Logical Shift Left / décalage logique à gauche}), 
LSR (\IT{Logical Shift Right / décalage logique à droite}), 
ASR (\IT{Arithmetic Shift Right décalage arithmétique à droite}), 
ROR (\IT{Rotate Right / rotation à droite}) et
RRX (\IT{Rotate Right with Extend / rotation à droite avec extension}),
qui peuvent être ajoutés à des instructions comme \MOV, \TST,
\CMP, \ADD, \SUB, \RSB\footnote{\DataProcessingInstructionsFootNote}.

Ces modificateurs définissent comment décaler le second opéreande et de combien de
bits.

\myindex{ARM!\Instructions!TST}
\myindex{ARM!Optional operators!LSL}
Ainsi l'instruction \TT{\q{TST R1, R2,LSL R3}} fonctionne ici comme $R1 \land (R2
\ll R3)$.

\subsubsection{ARM + \OptimizingXcodeIV (\ThumbTwoMode)}

\myindex{ARM!\Instructions!LSL.W}
\myindex{ARM!\Instructions!LSL}
Presque la même, mais ici il y a deux instructions utilisées, \INS{LSL.W}/\TST, au
lieu d'une seule \TST, car en mode Thumb il n'est pas possible de définir le modificateur
\LSL directement dans \TST.

\begin{lstlisting}[label=ARM_leaf_example5,style=customasmARM]
                MOV             R1, R0
                MOVS            R0, #0
                MOV.W           R9, #1
                MOVS            R3, #0
loc_2F7A
                LSL.W           R2, R9, R3
                TST             R2, R1
                ADD.W           R3, R3, #1
                IT NE
                ADDNE           R0, #1
                CMP             R3, #32
                BNE             loc_2F7A
                BX              LR
\end{lstlisting}

\subsubsection{ARM64 + GCC 4.9 \Optimizing}

Prenons un exemple en 64.bit qui a déjà été utilisé: \myref{popcnt_x64_example}.

\lstinputlisting[caption=GCC (Linaro) 4.8 \Optimizing,style=customasmARM]{patterns/14_bitfields/4_popcnt/ARM64_GCC_O3_FR.s}

Le résultat est très semblable à ce que GCC génère pour x64: \myref{shifts64_GCC_O3}.

\myindex{ARM!\Instructions!CSEL}
L'instruction \CSEL signifie \q{Conditional SELect / sélection conditionnelle}.
Elle choisi une des deux variables en fonction des flags mis par \TST et copie la
valeur dans \RegW{2}, qui contient la variable \q{rt}.

\subsubsection{ARM64 + GCC 4.9 \NonOptimizing}

De nouveau, nous travaillons sur un exemple 64-bit qui a déjà été utilisé: \myref{popcnt_x64_example}.
Le code est plus verbeux, comme d'habitude.

\lstinputlisting[caption=\NonOptimizing GCC (Linaro) 4.8,style=customasmARM]{patterns/14_bitfields/4_popcnt/ARM64_GCC_O0_FR.s}

