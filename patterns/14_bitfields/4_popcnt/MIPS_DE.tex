\subsubsection{MIPS}

\myparagraph{\NonOptimizing GCC}

\lstinputlisting[caption=\NonOptimizing GCC 4.4.5
(IDA),style=customasmMIPS]{patterns/14_bitfields/4_popcnt/MIPS_O0_IDA_DE.lst}

\myindex{MIPS!\Instructions!SLL}
\myindex{MIPS!\Instructions!SLLV}
Umständlich: alle lokalen Variablen liegen auf dem lokalen Stack und werden bei
jedem Zugriff neu geladen.

Der Befehl \SLLV bedeutet \q{Shift Word Left Logical Variable}. Er unterscheidet
sich von \SLL nur dadurch, dass die Anzahl der Verschiebungen im \SLL Befehl
kodiert sind (und dadurch nicht veränderbar). \SLLV hingegen erhält die Anzahl
der Verschiebungen aus einem Register.

\myparagraph{\Optimizing GCC}
Hier ist es knapper gehalten.
Warum aber gibt es zwei Schiebebefehle anstatt eines?

Es ist möglich, den ersten \SLLV Befehl durch einen unbedingte
Verzweigungsbefehl zu ersetzen, der direkt zum zweiten \SLLV springt. 
Das zieht aber einen zweiten Verzweigungsbefehl nach sich und es ist stets
vorteilhaft sich dieser zu entledigen:\myref{branch_predictors}.

\lstinputlisting[caption=\Optimizing GCC 4.4.5
(IDA),style=customasmMIPS]{patterns/14_bitfields/4_popcnt/MIPS_O3_IDA_DE.lst}

