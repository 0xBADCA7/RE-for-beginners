\subsubsection{x64}
\label{subsec:popcnt}

Modifions légèrement l'exemple pour l'étendre à 64-bit:

\lstinputlisting[label=popcnt_x64_example,style=customc]{patterns/14_bitfields/4_popcnt/shifts64.c}

\myparagraph{GCC 4.8.2 \NonOptimizing}

Jusqu'ici, c'est facile.

\lstinputlisting[caption=GCC 4.8.2 \NonOptimizing,style=customasmx86]{patterns/14_bitfields/4_popcnt/shifts64_GCC_O0_FR.s}

\myparagraph{GCC 4.8.2 \Optimizing}

\lstinputlisting[caption=GCC 4.8.2 \Optimizing,numbers=left,label=shifts64_GCC_O3,style=customasmx86]{patterns/14_bitfields/4_popcnt/shifts64_GCC_O3_FR.s}

Ce code est plus concis, mais a une particularité.

% TODO: comment traduire commits ?
Dans tous les exemples que nous avons vu jusqu'ici, nous incrémentions la valeur
de \q{rt} après la comparaison d'un bit spécifique, mais le code ici incrémente \q{rt}
avant (ligne 6), écrivant la nouvelle valeur dans le registre \EDX.
Donc, si le dernier bit est à 1, l'instruction \CMOVNE\footnote{Conditional MOVe if Not Equal}
(qui est un synonyme pour \CMOVNZ\footnote{Conditional MOVe if Not Zero}) \IT{commits}
la nouvelle valeur de \q{rt} en déplaçant  \EDX (\q{valeur proposée de rt}) dans
\EAX (\q{rt courant} qui va être retourné à la fin).

C'est pourquoi l'incrémentation est effectuée à chaque étape de la boucle, i.e.,
64 fois, sans relation avec la valeur en entrée.

L'avantage de ce code est qu'il contient seulement un saut conditionnel (à la fin
de la boucle) au lieu de deux sauts (évitant l'incrément de la valeur de \q{rt} et
à la fin de la boucle).
Et cela doit s'exécuter plus vite sur les CPUs modernes avec des prédicteurs de branchement:
\myref{branch_predictors}.

\label{FATRET}
\myindex{x86!\Instructions!FATRET}
La dernière instruction est \INS{REP RET} (opcode \TT{F3 C3}) qui est aussi appelée
\INS{FATRET} par MSVC.
C'est en quelque sorte une version optimisée de \RET, qu'AMD recommande de mettre
en fin de fonction, si \RET se trouve juste après un saut conditionnel:
\InSqBrackets{\AMDOptimization p.15}
\footnote{Lire aussi à ce propos: \url{http://go.yurichev.com/17328}}.

\myparagraph{MSVC 2010 \Optimizing}

\lstinputlisting[caption=MSVC 2010 \Optimizing,style=customasmx86]{patterns/14_bitfields/4_popcnt/MSVC_2010_x64_Ox_FR.asm}

\myindex{x86!\Instructions!ROL}
Ici l'instruction \ROL est utilisée au lieu de \SHL, qui est en fait \q{rotate left /
pivoter à gauche} au lieu de \q{shift left / décaler à gauche}, mais dans cet exemple
elle fonctionne tout comme \TT{SHL}.

Vous pouvez en lire plus sur l'instruction de rotation ici: \myref{ROL_ROR}.

\Reg{8} ici est compté de 64 à 0.
C'est tout comme un $i$ inversé.

Voici une table de quelques registres pendant l'exécution:

\begin{center}
\begin{tabular}{ | l | l | }
\hline
\HeaderColor RDX & \HeaderColor R8 \\
\hline
0x0000000000000001 & 64 \\
\hline
0x0000000000000002 & 63 \\
\hline
0x0000000000000004 & 62 \\
\hline
0x0000000000000008 & 61 \\
\hline
... & ... \\
\hline
0x4000000000000000 & 2 \\
\hline
0x8000000000000000 & 1 \\
\hline
\end{tabular}
\end{center}

\myindex{x86!\Instructions!FATRET}
Á la fin, nous voyons l'instruction \INS{FATRET}, qui a été expliquée ici: \myref{FATRET}.

\myparagraph{MSVC 2012 \Optimizing}

\lstinputlisting[caption=MSVC 2012 \Optimizing,style=customasmx86]{patterns/14_bitfields/4_popcnt/MSVC_2012_x64_Ox_FR.asm}

\myindex{\CompilerAnomaly}
\label{MSVC2012_anomaly}
MSVC 2012 \Optimizing fait presque le même job que MSVC 2010 \Optimizing, mais en
quelque sorte, il génère deux corps de boucles identiques et le nombre de boucles
est maintenant 32 au lieu de 64.

Pour être honnête, il n'est pas possible de dire pourquoi. Une ruse d'optimisation?
Peut-être est-il meilleur pour le corps de la boucle d'être légèrement plus long?

De toute façon, ce genre de code est pertinent ici pour montrer que parfois la sortie
du compilateur peut être vraiment bizarre et illogique, mais fonctionner parfaitement.

