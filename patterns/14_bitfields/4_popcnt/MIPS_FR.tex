\subsubsection{MIPS}

\myparagraph{GCC \NonOptimizing}

\lstinputlisting[caption=GCC 4.4.5 \NonOptimizing (IDA),style=customasmMIPS]{patterns/14_bitfields/4_popcnt/MIPS_O0_IDA_FR.lst}

\myindex{MIPS!\Instructions!SLL}
\myindex{MIPS!\Instructions!SLLV}

C'est très verbeux: toutes les varaibles locales sont situées dans la pile locale
et rechargées à chaque fois que l'on en a besoin.

L'instruction \SLLV est \q{Shift Word Left Logical Variable}, elle diffère de \SLL
seulement de ce que la valeur du décalage est encodée dans l'instruction \SLL (et
par conséquent fixée) mais \SLLV lit cette valeur depuis un registre.

\myparagraph{GCC \Optimizing}

C'est plus concis.
Il y a deux instructions de décalage au lieu d'une.
Pourquoi?

Il est possible de remplacer la première instruction \SLLV avec une instruction de
branchement inconditionnel qui saute directement au second \SLLV.
Mais cela ferait une autre instruction de branchement dans la fonction, et il est
toujours favorable de s'en passer: \myref{branch_predictors}.

\lstinputlisting[caption=GCC 4.4.5 \Optimizing (IDA),style=customasmMIPS]{patterns/14_bitfields/4_popcnt/MIPS_O3_IDA_FR.lst}

