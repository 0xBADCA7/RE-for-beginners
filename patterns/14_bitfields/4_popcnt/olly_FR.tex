\clearpage
\mysubparagraph{\olly}
\myindex{\olly}

Chargeons cet exemple dans \olly.
Définissons la valeur d'entrée à \TT{0x12345678}.

Pour $i=1$, nous voyons comment $i$ est chargé dans \ECX:

\begin{figure}[H]
\centering
\myincludegraphics{patterns/14_bitfields/4_popcnt/olly1_1.png}
\caption{\olly: $i=1$, $i$ est chargé dans \ECX}
\label{fig:shifts_olly1_1}
\end{figure}

\EDX contient 1. \SHL va être exécuté maintenant.

\clearpage
\SHL a été exécuté:

\begin{figure}[H]
\centering
\myincludegraphics{patterns/14_bitfields/4_popcnt/olly1_2.png}
\caption{\olly: $i=1$, \EDX=$1 \ll 1=2$}
\label{fig:shifts_olly1_2}
\end{figure}

\EDX contient $1 \ll 1$ (ou 2). Ceci est un masque de bit.

\clearpage
\AND met \ZF à 1, ca qui implique que la valeur en entrée (\TT{0x12345678}) ANDée
avec 2 donne 0:

\begin{figure}[H]
\centering
\myincludegraphics{patterns/14_bitfields/4_popcnt/olly1_3.png}
\caption{\olly: $i=1$, 
y a-t-il ce bit dans la valeur en entrée? Non. (\ZF=1)}
\label{fig:shifts_olly1_3}
\end{figure}

Donc, il n'y a pas le bit correspondant dans la valeur en entrée.

Le morceau de code, qui \glslink{increment}{incrémente} le compteur ne va pas être
exécuté:
l'instruction \JZ l'évite.

\clearpage
Avançons un peu plus et $i$ vaut maintenant 4.
\SHL va être exécuté maintenant:

\begin{figure}[H]
\centering
\myincludegraphics{patterns/14_bitfields/4_popcnt/olly4_1.png}
\caption{\olly: $i=4$, $i$ est chargée dans \ECX}
\label{fig:shifts_olly4_1}
\end{figure}

\clearpage
\EDX=$1 \ll 4$ (ou \TT{0x10} ou 16):

\begin{figure}[H]
\centering
\myincludegraphics{patterns/14_bitfields/4_popcnt/olly4_2.png}
\caption{\olly: $i=4$, \EDX=$1 \ll 4=0x10$}
\label{fig:shifts_olly4_2}
\end{figure}

Ceci est un autre masque de bit.

\clearpage
\AND est exécuté:

\begin{figure}[H]
\centering
\myincludegraphics{patterns/14_bitfields/4_popcnt/olly4_3.png}
\caption{\olly: $i=4$, 
y a-t-il ce bit dans la valeur en entrée? Oui. (\ZF=0)}
\label{fig:shifts_olly4_3}
\end{figure}

\ZF est à 0 car ce bit est présent dans la valeur en entrée.\\
En effet, \TT{0x12345678 \& 0x10 = 0x10}.

Ce bit compte: le saut n'est pas effectué et le compteur de bit est
\glslink{increment}{incrémenté}.

La fonction renvoie 13.
C'est le nombre total de bits à 1 dans \TT{0x12345678}.

