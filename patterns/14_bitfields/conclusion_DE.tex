% TODO: add ROL/ROR
\subsection{\Conclusion{}}

\myindex{x86!\Instructions!SHR}
\myindex{x86!\Instructions!SHL}
\myindex{x86!\Instructions!SAR}
Analog zu den Schiebebefehlen \TT{$\ll$} und \TT{$\gg$} in \CCpp gibt es in
x86 die Befehle \SHR/\SHL (für vorzeichenlose Werte) und \SAR/\SHL (für
vorzeichenbehaftete Werte).

\myindex{ARM!\Instructions!LSR}
\myindex{ARM!\Instructions!LSL}
\myindex{ARM!\Instructions!ASR}
Die Schiebebefehle in ARM sind \LSR/\LSL (für vorzeichenlose Werte) und
\ASR/\LSL (für vorzeichenbehaftete Werte).

Es sind bei manchen Befehlen auch mögliche Suffixe für die Verschiebung
anzuhängen (diese heiße \q{data processing instructions}).
% FIXME: which instructions?

\subsubsection{Prüfung auf spezifisches Bit (zur Compilezeit bekannt)}
Prüfung, ob das Bit 0b10000000 (0x40) sich im Registerwert befindet:

\lstinputlisting[caption=\CCpp,style=customc]{patterns/14_bitfields/c_snippet0.c}

\lstinputlisting[caption=x86,style=customasmx86]{patterns/14_bitfields/TEST_JNZ_DE.lst}

\lstinputlisting[caption=x86,style=customasmx86]{patterns/14_bitfields/TEST_JZ_DE.lst}

\lstinputlisting[caption=ARM
(\ARMMode),style=customasmARM]{patterns/14_bitfields/TST_BNE_DE.lst}

\myindex{x86!\Instructions!AND}
\myindex{x86!\Instructions!TEST}
Manchmal wird \AND anstelle von \TEST verwendet, aber die gesetzten Flags sind
die gleichen.

\subsubsection{Prüfung auf spezifisches Bit (zur Laufzeit angegeben)}
Dies wird normalerweise durch den folgenden \CCpp Code gelöst (verschiebe Wert
um $n$ Bits nach rechts und schneide dann niederwertigstes Bit ab):

\lstinputlisting[caption=\CCpp,style=customc]{patterns/14_bitfields/c_snippet1.c}
In x86 Code wird dies gewöhnlich wie folgt implementiert:

\begin{lstlisting}[caption=x86,style=customasmx86]
; REG=input_value
; CL=n
SHR REG, CL
AND REG, 1
\end{lstlisting}
Eine andere Möglichkeit: (verschiebe 1 Bit $n$-mal nach links, isoliere dieses
Bit im Eingabewert und prüfe, ob es nicht 0 ist):

\lstinputlisting[caption=\CCpp,style=customc]{patterns/14_bitfields/c_snippet2.c}

In x86 Code wird dies gewöhnlich wie folgt implementiert:

\begin{lstlisting}[caption=x86,style=customasmx86]
; CL=n
MOV REG, 1
SHL REG, CL
AND input_value, REG
\end{lstlisting}

\subsubsection{Setzen eines spezifischen Bits (zur Compilerzeit bekannt)}

\begin{lstlisting}[caption=\CCpp]
value=value|0x40;
\end{lstlisting}

\begin{lstlisting}[caption=x86,style=customasmx86]
OR REG, 40h
\end{lstlisting}

\begin{lstlisting}[caption=ARM (\ARMMode) and ARM64,style=customasmARM]
ORR R0, R0, #0x40
\end{lstlisting}

\subsubsection{Setzen eines spezifischen Bits (zur Laufzeit angegeben)}

\lstinputlisting[caption=\CCpp,style=customc]{patterns/14_bitfields/c_snippet3.c}

In x86 Code wird dies gewöhnlich wie folgt implementiert:

\begin{lstlisting}[caption=x86,style=customasmx86]
; CL=n
MOV REG, 1
SHL REG, CL
OR input_value, REG
\end{lstlisting}

\subsubsection{Löschen eines spezifischen Bits (zur Compilezeit bekannt)}
Man verwendet einfach den \AND Befehl mit dem invertierten Wert:

\begin{lstlisting}[caption=\CCpp,style=customc]
value=value&(~0x40);
\end{lstlisting}

\begin{lstlisting}[caption=x86,style=customasmx86]
AND REG, 0FFFFFFBFh
\end{lstlisting}

\begin{lstlisting}[caption=x64,style=customasmx86]
AND REG, 0FFFFFFFFFFFFFFBFh
\end{lstlisting}

Dies sorgt dafür, dass alle Bits bis auf eines gesetzt werden.

\myindex{ARM!\Instructions!BIC}

ARM im ARM mode verfügt über den Befehl \BIC, der wie ein \NOT+\AND Befehlspaar
arbeitet:

\begin{lstlisting}[caption=ARM (\ARMMode),style=customasmARM]
BIC R0, R0, #0x40
\end{lstlisting}

\subsubsection{Löschen eines spezifischen Bits (zur Laufzeit angegeben)}

\lstinputlisting[caption=\CCpp,style=customc]{patterns/14_bitfields/c_snippet4.c}

\begin{lstlisting}[caption=x86,style=customasmx86]
; CL=n
MOV REG, 1
SHL REG, CL
NOT REG
AND input_value, REG
\end{lstlisting}
