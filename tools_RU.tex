% TODO sync with English version
\chapter{Инструменты}

\section{Дизассемблеры}

\subsection{IDA}

\label{IDA}
Старая бесплатная версия доступна для скачивания
\footnote{\href{http://go.yurichev.com/17031}{hex-rays.com/products/ida/support/download\_freeware.shtml}}.

\ShortHotKeyCheatsheet: \myref{sec:IDA_cheatsheet}

\section{Отладчики}

\subsection{\olly}
\myindex{\olly}

Очень популярный отладчик пользовательской среды win32: \href{http://go.yurichev.com/17032}{ollydbg.de}.

\ShortHotKeyCheatsheet: \myref{sec:Olly_cheatsheet}

\subsection{GDB}
\myindex{GDB}

Не очень популярный отладчик у реверсеров, тем не менее, крайне удобный.

Некоторые команды: \myref{sec:GDB_cheatsheet}.

\subsection{tracer}

\myindex{tracer}
\label{tracer}
Автор часто использует \IT{tracer}
\footnote{\href{http://go.yurichev.com/17339}{yurichev.com}}
вместо отладчика.

Со временем, автор этих строк отказался использовать отладчик, потому что всё что ему нужно от него это иногда подсмотреть 
какие-либо аргументы какой-либо функции во время исполнения или состояние регистров в определенном месте. 
Каждый раз загружать отладчик для этого это слишком, поэтому родилась очень простая утилита \IT{tracer}. 
Она консольная, запускается из командной строки, позволяет перехватывать исполнение функций, 
ставить точки останова на произвольные места, смотреть состояние регистров, модифицировать их, итд.

Но для учебы очень полезно трассировать код руками в отладчике, наблюдать как меняются значения регистров 
(например, как минимум классический SoftICE, OllyDbg, WinDbg подсвечивают измененные регистры), 
флагов, данные, менять их самому, смотреть реакцию, итд.

\section{Трассировка системных вызовов}

\label{strace}
\myindex{strace}
\myindex{dtruss}
\subsubsection{strace / dtruss}

\myindex{syscall}
Позволяет показать, какие системные вызовы (syscalls(\myref{syscalls})) прямо сейчас вызывает процесс.

Например:

\begin{lstlisting}
# strace df -h

...

access("/etc/ld.so.nohwcap", F_OK)      = -1 ENOENT (No such file or directory)
open("/lib/i386-linux-gnu/libc.so.6", O_RDONLY|O_CLOEXEC) = 3
read(3, "\177ELF\1\1\1\0\0\0\0\0\0\0\0\0\3\0\3\0\1\0\0\0\220\232\1\0004\0\0\0"..., 512) = 512
fstat64(3, {st_mode=S_IFREG|0755, st_size=1770984, ...}) = 0
mmap2(NULL, 1780508, PROT_READ|PROT_EXEC, MAP_PRIVATE|MAP_DENYWRITE, 3, 0) = 0xb75b3000
\end{lstlisting}

\myindex{\MacOSX}
В \MacOSX для этого же имеется dtruss.

\myindex{Cygwin}
В Cygwin также есть strace, впрочем, насколько известно, 
он показывает результаты только для .exe-файлов скомпилированных для среды самого cygwin.

\section{Декомпиляторы}

Пока существует только один публично доступный декомпилятор в Си высокого качества: Hex-Rays:
\href{http://go.yurichev.com/17033}{hex-rays.com/products/decompiler/}

% TODO Java, .NET, VB, etc

\section{Прочие инструменты}

\begin{itemize}
\item
Microsoft Visual Studio Express\footnote{\href{http://go.yurichev.com/17034}{visualstudio.com/en-US/products/visual-studio-express-vs}}:
Усеченная бесплатная версия Visual Studio, пригодная для простых экспериментов.
Некоторые полезные опции: \myref{sec:MSVC_options}.

\item
\label{Hiew}
Hiew\footnote{\href{http://go.yurichev.com/17035}{hiew.ru}}:
для мелкой модификации кода в исполняемых файлах.
	
\item
\myindex{binary grep}
binary grep: 
небольшая утилита для поиска констант (либо просто последовательности байт)
в большом количестве файлов, включая неисполняемые: \BGREPURL.
\index{rafind2}
В rada.re имеется также rafind2 для тех же целей.

\end{itemize}

