% TODO split
\part*{%
	\RU{Список принятых сокращений}%
	\EN{Acronyms used}%
	\NL{Gebruikte afkortingen}%
	\ES{Acr\'onimos utilizados}%
	\PTBRph{}%
	\DE{Verwendete Abkürzungen}%
	\PLph{}%
	\ITAph{}%
	\FR{Acronymes utilisés}%
	\JPNph{}
}
\addcontentsline{toc}{part}{%
	\RU{Список принятых сокращений}%
	\EN{Acronyms used}%
	\ES{Acr\'onimos utilizados}%
	\NL{Gebruikte afkortingen}%
	\PTBRph{}%
	\DE{Verwendete Abkürzungen}%
	\PLph{}%
	\ITAph{}%
	\FR{Acronymes utilisés}%
	\JPNph{}
}
\begin{acronym}
\RU{
	\acro{OS}[ОС]{Операционная Система}
	\acro{FAQ}[ЧаВО]{Часто задаваемые вопросы}
	\acro{OOP}[ООП]{Объектно-Ориентированное Программирование}
	\acro{PL}[ЯП]{Язык Программирования}
	\acro{PRNG}[ГПСЧ]{Генератор псевдослучайных чисел}
	\acro{ROM}[ПЗУ]{Постоянное запоминающее устройство}
	\acro{ALU}[АЛУ]{Арифметико-логическое устройство}
	\acro{PID}{ID программы/процесса}
	\acro{LF}{Line feed (подача строки) (10 или '\textbackslash{}n' в \CCpp)}
	\acro{CR}{Carriage return (возврат каретки) (13 или '\textbackslash{}r' в \CCpp)}
	\acro{LIFO}{Last In First Out (последним вошел, первым вышел)}
	\acro{MSB}{Most significant bit (самый старший бит)} % NOT BYTE!
	\acro{LSB}{Least significant bit (самый младший бит)} % NOT BYTE!
	\acro{NSA}[АНБ]{Агентство национальной безопасности}
	\acro{CFB}{Режим обратной связи по шифротексту (Cipher Feedback)}
	\acro{CSPRNG}{Криптографически стойкий генератор псевдослучайных чисел (cryptographically secure pseudorandom number generator)}
	\acro{SICP}{Структура и интерпретация компьютерных программ (Structure and Interpretation of Computer Programs)}
}%
\EN{
	\acro{OS}{Operating System}
	\acro{FAQ}{Frequently Asked Questions}
	\acro{OOP}{Object-Oriented Programming}
	\acro{PL}{Programming Language}
	\acro{PRNG}{Pseudorandom Number Generator}
	\acro{ROM}{Read-Only Memory}
	\acro{ALU}{Arithmetic Logic Unit}
	\acro{PID}{Program/process ID}
	\acro{LF}{Line Feed (10 or '\textbackslash{}n' in \CCpp)}
	\acro{CR}{Carriage Return (13 or '\textbackslash{}r' in \CCpp)}
	\acro{LIFO}{Last In First Out}
	\acro{MSB}{Most Significant Bit} % NOT BYTE!
	\acro{LSB}{Least Significant Bit} % NOT BYTE!
	\acro{NSA}{National Security Agency}
	\acro{CFB}{Cipher Feedback}
	\acro{CSPRNG}{Cryptographically Secure Pseudorandom Number Generator}
	\acro{SICP}{Structure and Interpretation of Computer Programs}
	\acro{ABI}{Application Binary Interface}
}%
\ES{
	\acro{OS}[SO]{Sistema Operativo}
	\acro{FAQ}{Preguntas Frecuentes}
	\acro{OOP}[POO]{Programaci\'on Orientada a Objetos}
	\acro{PL}[LP]{Lenguaje de Programaci\'on}
	\acro{PRNG}[GPAN]{Generador Pseudo-Aleatorio de N\'umeros}
	\acro{ROM}{Memoria de Solo Lectura}
	\acro{ALU}{Unidad Aritm\'etica L\'ogica}
	\acro{NSA}{\ESph{}}
}%
\PTBR{
	\acro{OS}{\PTBRph{}}
	\acro{FAQ}{\PTBRph{}}
	\acro{OOP}{\PTBRph{}}
	\acro{PL}{\PTBRph{}}
	\acro{PRNG}{\PTBRph{}}
	\acro{ROM}{\PTBRph{}}
	\acro{ALU}{\PTBRph{}}
	\acro{NSA}{\PTBRph{}}
}%
\PL{
	\acro{OS}{\PLph{}}
	\acro{FAQ}{\PLph{}}
	\acro{OOP}{\PLph{}}
	\acro{PL}{\PLph{}}
	\acro{PRNG}{\PLph{}}
	\acro{ROM}{\PLph{}}
	\acro{ALU}{\PLph{}}
}%
\DE{
	\acro{OS}[BS]{Betriebssystem}
	\acro{FAQ}{Häufig gestellte Fragen}
	\acro{OOP}{Objektorientierte Programmierung}
	\acro{PL}[PS]{Programmiersprache}
	\acro{PRNG}{Pseudozufallszahlen-Generator}
	\acro{ROM}{\DEph{}}
	\acro{ALU}[ALE]{Arithmetisch-logische Einheit}
	\acro{NSA}{\DEph{}}
}%
\ITA{
	\acro{OS}{\ITAph{}}
	\acro{FAQ}{\ITAph{}}
	\acro{OOP}{\ITAph{}}
	\acro{PL}{\ITAph{}}
	\acro{PRNG}{\ITAph{}}
	\acro{ROM}{\ITAph{}}
	\acro{ALU}{\ITAph{}}
}%
\THA{
	\acro{OS}{\THAph{}}
	\acro{FAQ}{\THAph{}}
	\acro{OOP}{\THAph{}}
	\acro{PL}{\THAph{}}
	\acro{PRNG}{\THAph{}}
	\acro{ROM}{\THAph{}}
	\acro{ALU}{\THAph{}}
}%
\NL{
	\acro{OS}{\NL{Operating System}}
	\acro{FAQ}{\NL{Veelvoorkomende vragen}}
	\acro{OOP}{\NL{Object-Oriented Programmeren}}
	\acro{PL}[PT]{\NL{Programmeertaal}}
	\acro{PRNG}{\NL{Pseudorandom number generator}}
	\acro{ROM}{\NL{Read-only memory}}
	\acro{ALU}{\NL{Arithmetic logic unit}}
}%
\FR{
	\acro{OS}[OS]{Système d'exploitation (Operating System)}
	\acro{FAQ}{Foire Aux Questions}
	\acro{OOP}[POO]{Programmation orientée objet}
	\acro{PL}[LP]{Langage de programmation}
	\acro{PRNG}{Nombre généré pseudo-aléatoirement}
	\acro{ROM}{Mémoire morte}
	\acro{ALU}[UAL]{Unité arithmétique et logique}
	\acro{PID}{ID d'un processus}
	\acro{LF}{Line feed (10 ou '\textbackslash{}n' en \CCpp)}
	\acro{CR}{Carriage return (13 ou '\textbackslash{}r' en \CCpp)}
	\acro{LIFO}{Dernier entré, premier sorti}
	\acro{MSB}{Bit le plus significatif} % NOT BYTE!
	\acro{LSB}{Bit le moins significatif} % NOT BYTE!
	\acro{NSA}{National Security Agency (Agence Nationale de la Sécurité)} % translation not used in French
	\acro{CSPRNG}{Cryptographically Secure Pseudorandom Number Generator (générateur de nombres pseudo-aléatoire cryptographiquement sûr)}
	\acro{SICP}{Structure and Interpretation of Computer Programs}
	\acro{ABI}{Application Binary Interface}
}%
\JPN{
	\acro{OS}{\JPNph{}}
	\acro{FAQ}{\JPNph{}}
	\acro{OOP}{\JPNph{}}
	\acro{PL}{\JPNph{}}
	\acro{PRNG}{\JPNph{}}
	\acro{ROM}{\JPNph{}}
	\acro{ALU}{\JPNph{}}
}%
\acro{RA}{\ReturnAddress}
\acro{PE}{Portable Executable}
\acro{SP}{\gls{stack pointer}. SP/ESP/RSP \InENRU x86/x64. SP \InENRU ARM.}
\acro{DLL}{Dynamic-Link Library}
\acro{PC}{Program Counter. IP/EIP/RIP \InENRU x86/64. PC \InENRU ARM.}
\acro{LR}{Link Register}
\acro{IDA}{
	\RU{Интерактивный дизассемблер и отладчик, разработан \href{https://hex-rays.com/}{Hex-Rays}}%
	\EN{Interactive Disassembler and Debugger developed by \href{https://hex-rays.com/}{Hex-Rays}}%
	\ES{Desensamblador Interactivo y depurador desarrollado por \href{https://hex-rays.com/}{Hex-Rays}}%
	\NL{Interactive Disassembler en debugger ontwikkeld door \href{https://hex-rays.com}{Hex-Rays}}
	\PTBRph{}%
	\PLph{}%
	\DE{Interaktiver Disassembler und Debugger entwickelt von \href{https://hex-rays.com/}{Hex-Rays}}%
	\ITAph{}%
	\THAph{}%
	\FR{Désassembleur interactif et débuggueur développé par \href{https://hex-rays.com/}{Hex-Rays}}%
	\JPNph{}%
}
\acro{IAT}{Import Address Table}
\acro{INT}{Import Name Table}
\acro{RVA}{Relative Virtual Address}
\acro{VA}{Virtual Address}
\acro{OEP}{Original Entry Point}
\acro{MSVC}{Microsoft Visual C++}
\acro{MSVS}{Microsoft Visual Studio}
\acro{ASLR}{Address Space Layout Randomization}
\acro{MFC}{Microsoft Foundation Classes}
\acro{TLS}{Thread Local Storage}
\acro{AKA}{
        \EN{Also Known As}%
	\FR{Aussi connu sous le nom de}%
	\RU{ - (Также известный как)}%
	\ES{ - (Tambi\'en Conocido Como)}%
	\NL{ - (Ook gekend als)}%
	\PTBRph{}%
	\PLph{}%
	\DEph{}%
	\ITAph{}%
	\THAph{}%
	\JPNph{}%
}
\acro{CRT}{C Runtime library}
\acro{CPU}{Central Processing Unit}
\acro{GPU}{Graphics Processing Unit}
\acro{FPU}{Floating-Point Unit}
\acro{CISC}{Complex Instruction Set Computing}
\acro{RISC}{Reduced Instruction Set Computing}
\acro{GUI}{Graphical User Interface}
\acro{RTTI}{Run-Time Type Information}
\acro{BSS}{Block Started by Symbol}
\acro{SIMD}{Single Instruction, Multiple Data}
\acro{BSOD}{Blue Screen of Death}
\acro{DBMS}{Database Management Systems}
\acro{ISA}{Instruction Set Architecture\RU{ (Архитектура набора команд)}}
\acro{CGI}{Common Gateway Interface}
\acro{HPC}{High-Performance Computing}
\acro{SOC}{System on Chip}
\acro{SEH}{Structured Exception Handling}
\acro{ELF}{\RU{Формат исполняемых файлов, использующийся в Linux и некоторых других *NIX}
\EN{Executable File format widely used in *NIX systems including Linux}\ESph{}\PTBRph{}\PLph{}\ITAph{}\DEph{}\NLph{}
\FR{Format de fichier exécutable couramment utilisé sur les systèmes *NIX, Linux inclus}
\JPN{Linuxを含め*NIXシステムで広く使用される実行ファイルフォーマット}}
\acro{TIB}{Thread Information Block}
\acro{TEA}{Tiny Encryption Algorithm}
\acro{PIC}{Position Independent Code}
\acro{NAN}{Not a Number}
\acro{NOP}{No Operation}
\acro{BEQ}{(PowerPC, ARM) Branch if Equal}
\acro{BNE}{(PowerPC, ARM) Branch if Not Equal}
\acro{BLR}{(PowerPC) Branch to Link Register}
\acro{XOR}{eXclusive OR\RU{ (исключающее \q{ИЛИ})}\FR{ (OU exclusif)}}
\acro{MCU}{Microcontroller Unit}
\acro{RAM}{Random-Access Memory}
\acro{GCC}{GNU Compiler Collection}
\acro{EGA}{Enhanced Graphics Adapter}
\acro{VGA}{Video Graphics Array}
\acro{API}{Application Programming Interface}
\acro{ASCII}{American Standard Code for Information Interchange}
\acro{ASCIIZ}{ASCII Zero (\RU{ASCII-строка заканчивающаяся нулем}\EN{null-terminated ASCII string}
\FR{chaîne ASCII terminée par un octet nul (à zéro)}\ESph{}\PTBRph{}\PLph{}\ITAph{}\DEph{}\NLph{})}
\acro{IA64}{Intel Architecture 64 (Itanium)}
\acro{EPIC}{Explicitly Parallel Instruction Computing}
\acro{OOE}{Out-of-Order Execution}
\acro{MSDN}{Microsoft Developer Network}
\acro{STL}{(\Cpp) Standard Template Library: \myref{sec:STL}}
\acro{PODT}{(\Cpp) Plain Old Data Type}
\acro{HDD}{Hard Disk Drive}
\acro{VM}{Virtual Memory\RU{ (виртуальная память)}\FR{ (mémoire virtuelle)}}
\acro{WRK}{Windows Research Kernel}
\acro{GPR}{General Purpose Registers\RU{ (регистры общего пользования)}}
\acro{SSDT}{System Service Dispatch Table}
\acro{RE}{Reverse Engineering}
\acro{RAID}{Redundant Array of Independent Disks}
\acro{SSE}{Streaming SIMD Extensions}
\acro{BCD}{Binary-Coded Decimal}
\acro{BOM}{Byte Order Mark}
\acro{GDB}{GNU Debugger}
\acro{FP}{Frame Pointer}
\acro{MBR}{Master Boot Record}
\acro{JPE}{Jump Parity Even (\RU{инструкция x86}\EN{x86 instruction}\FR{instruction x86})}
\acro{CIDR}{Classless Inter-Domain Routing}
\acro{STMFD}{Store Multiple Full Descending (\RU{инструкция ARM}\EN{ARM instruction}\FR{instruction ARM})}
\acro{LDMFD}{Load Multiple Full Descending (\RU{инструкция ARM}\EN{ARM instruction}\FR{instruction ARM})}
\acro{STMED}{Store Multiple Empty Descending (\RU{инструкция ARM}\EN{ARM instruction}\FR{instruction ARM})}
\acro{LDMED}{Load Multiple Empty Descending (\RU{инструкция ARM}\EN{ARM instruction}\FR{instruction ARM})}
\acro{STMFA}{Store Multiple Full Ascending (\RU{инструкция ARM}\EN{ARM instruction}\FR{instruction ARM})}
\acro{LDMFA}{Load Multiple Full Ascending (\RU{инструкция ARM}\EN{ARM instruction}\FR{instruction ARM})}
\acro{STMEA}{Store Multiple Empty Ascending (\RU{инструкция ARM}\EN{ARM instruction}\FR{instruction ARM})}
\acro{LDMEA}{Load Multiple Empty Ascending (\RU{инструкция ARM}\EN{ARM instruction}\FR{instruction ARM})}
\acro{APSR}{(ARM) Application Program Status Register}
\acro{FPSCR}{(ARM) Floating-Point Status and Control Register}
\acro{RFC}{Request for Comments}
\acro{TOS}{Top of Stack\RU{ (вершина стека)}}
\acro{LVA}{(Java) Local Variable Array\RU{ (массив локальных переменных)}}
\acro{JVM}{Java Virtual Machine}
\acro{JIT}{Just-In-Time compilation}
\acro{CDFS}{Compact Disc File System}
\acro{CD}{Compact Disc}
\acro{ADC}{Analog-to-Digital Converter}
\acro{EOF}{End of File\RU{ (конец файла)}\FR{ (fin de fichier)}}
\acro{TBT}{To be Translated. The presence of this acronym in this place means that the English version has some new/modified content which is to be translated and placed right here.}
\acro{DIY}{Do It Yourself}
\acro{MMU}{Memory Management Unit}
\acro{DES}{Data Encryption Standard}
\acro{MIME}{Multipurpose Internet Mail Extensions}
\acro{DBI}{Dynamic Binary Instrumentation}
\acro{XML}{Extensible Markup Language}
\acro{JSON}{JavaScript Object Notation}
\acro{URL}{Uniform Resource Locator}
\acro{ISP}{Internet Service Provider}
\acro{IV}{Initialization Vector}
\acro{RSA}{Rivest Shamir Adleman}
\acro{CPRNG}{Cryptographically secure PseudoRandom Number Generator}
\acro{GiB}{Gibibyte}
\end{acronym}
