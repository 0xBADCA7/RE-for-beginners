\subsection{Win32 PE}
\label{win32_pe}
\myindex{Windows!Win32}

\acs{PE} ist ein Dateiformat für ausführbare Dateien unter Windows.
Der Unterschied zwischen .exe, .dll und .sys ist, dass .exe und .sys in der Regel
nur Imports und keine Exports haben.

\myindex{OEP}

Eine \ac{DLL} hat wie jede andere PE-Datei einen Eintrittspunkt (\ac{OEP}) (die
Funktion DllMain() befindet sich hier), allerdings macht diese Funktion in der
Regel nichts.
.sys ist normalerweise ein Gerätetreiber.
Wie auch bei Treibern, erwartet Windows eine Prüfsumme in der PE-Datei, die korrekt
sein muss\footnote{Hiew(\myref{Hiew}) kann diese berechnen}.

\myindex{Windows!Windows Vista}
Ab Windows Vista, muss eine Treiberdatei auch mit einer digitalen Signatur versehen
sein. Anderseits wird das Laden des Treibers fehlschlagen.

\myindex{MS-DOS}
Jede PE-Datei beginnt mit einem kleinen DOS-Programm, welches eine Nachricht in der
Art wie folgt ausgibt:
\q{Dieses Programm kann nicht im MS-DOS-Modus gestartet werden.}---wenn versucht wird
das Programm unter DOS oder Windows 3.1 zu starten (\ac{OS}e die das PE-Format
nicht kennen).

\subsubsection{Terminologie}

\myindex{VA}
\myindex{Base address}
\myindex{RVA}
\myindex{Windows!IAT}
\myindex{Windows!INT}

\begin{itemize}
\item Modul: eine separate Datei, .exe oder .dll

\item Prozess: ein Programm das in den Speicher geladen wurde und gerade ausgeführt
wird. Besteht meist aus einer .exe-Datei und einer Reihe von .dll-Dateien.

\item Prozessspeicher: der Speicher, mit dem der Prozess arbeitet. Jeder Prozess
hat seinen eigenen Bereich. Hier befinden sich in der Regel Module, der Stack,
\gls{heap}(s) usw.

\item \ac{VA}: Eine Adresse welche zur Laufzeit in einem Programm verwendet wird.

\item Basisadresse (eines Moduls): die Adresse im Prozessspeicher an der das Modul
geladen wird. Der \ac{OS}-Lader kann diese ändern wenn die Basisadresse bereits
von einem vorher geladenen Modul verwendet wird.

\item \ac{RVA}: die \ac{VA}-Adresse minus der Basisadresse.

Viele Adressen in PE-Dateien-Tabellen nutzen \ac{RVA}-Adressen.

%\item
%Data directory\EMDASH{}...

\item \ac{IAT}: Ein Array von Adressen importierter Module\footnote{\PietrekPE}.
Manchmal zeigt das \TT{IMAGE\_DIRECTORY\_ENTRY\_IAT}-Daten-Verzeichnis auf \ac{IAT}.
\label{IDA_idata}
Erwähnenswert ist es dass \ac{IDA} (v6.1) eine Pseudosektion namens \TT{.idata}
für \ac{IAT} allozieren kann, selbst wenn \ac{IAT} Teil einer anderen Sektion ist.

\item \ac{INT}: Ein Array von Namen zu importierender Symbole\footnote{\PietrekPE}.
\end{itemize}

\subsubsection{Basisadresse}

Das Problem ist, dass mehrere Modulprogrammierer DLL-Dateien für die Nutzung für andere
vorbereiten können, es jedoch nicht möglich ist festzulegen welche Adressen für diese
Module verwendet werden.

Das ist der Grund, warum in dem Fall wenn zwei für einen Prozess notwendige DLLs
dieselbe Basisadresse haben, eine davon an die Basisadresse geladen wird und die
andere an eine andere freie Stelle im Prozessspeicher. Jede virtuelle Adresse der
zweiten DLL wird korrigiert.

\par Mit \ac{MSVC} generiert der Linker oft .exe-Dateien mit der Basisadresse
\TT{0x400000}\footnote{der Ursprung dieser Adress-Auswahl ist hier: \href{http://go.yurichev.com/17041}{MSDN}},
und mit der Code-Sektion die bei \TT{0x401000} beginnt.
Dies bedeutet, dass die \ac{RVA} des Beginns der Code-Sektion \TT{0x1000} ist.

DLLs werden vom MSVC-Linker oft mit der Basisadresse \TT{0x10000000}
erzeugt\footnote{Dies kann mit der Linker-Option /BASE geändert werden}.

\myindex{ASLR}

Es gibt einen weiteren Grund warum Module an verschiedenen Basisadressen, in diesem
Fall an zufälligen Adressen, geladen werden: \ac{ASLR}\footnote{\href{http://go.yurichev.com/17140}{Wikipedia}}.

\myindex{Shellcode}

Ein Shellcode der auf einem kompromittierten System ausgeführt werden soll, muss
Systemfunktionen aufrufen und dementsprechend deren Adressen kennen.

In älteren \ac{OS} (in der \gls{Windows NT}-Reihe: bevor Windows Vista) wurden
System-DLL (wie kernel32.dll, user32.dll) immer an bekannte Adressen geladen und
mit dem Wissen, dass deren Versionen selten wechseln, waren die Adressen der Funktionen
fest und konnten direkt aufgerufen werden.

Um dies zu verhindern läd \ac{ASLR} das Programm und alle benötigten Module jedes
Mal an zufällige Basisadressen.

\ac{ASLR}-Unterstützung ist in der PE-Datei mit dem Flag
\TT{IMAGE\_DLL\_CHARACTERISTICS\_DYNAMIC\_BASE} markiert \InSqBrackets{siehe \Russinovich}.

\subsubsection{Subsystem}

Es existiert auch ein \IT{Subsystem}-Feld, dass normalerweise wie folgt ist:

\myindex{Native API}

\begin{itemize}
\item native\footnote{bedeutet, dass das Modul eine eigene API statt Win32 nutzt} (.sys-Treiber),

\item console (Konsolenanwendung) oder

\item \ac{GUI} (Anwendung mit grafischer Oberfläche).
\end{itemize}

\subsubsection{Betriebssystem-Version}

Eine PE-Datei spezifiziert auch die minimale Windows-Version die sie benötigt um
ladbar zu sein.

Die Tabelle mit Versionsnummern in der PE-Datei und die entsprechenden Windows-Codenamen
ist hier\footnote{\href{http://go.yurichev.com/17044}{Wikipedia}}.

\myindex{Windows!Windows NT4}
\myindex{Windows!Windows 2000}
Beispielsweise kompiliert \ac{MSVC} 2005 .exe-Dateien für Windows NT4 (Version 4.00),
\ac{MSVC} 2008 jedoch nicht (die erzeugten Dateien haben die Version 5.00, es ist
mindestens Windows 2000 notwendig um sie ausführen zu können).

\myindex{Windows!Windows XP}

\ac{MSVC} 2012 erzeugt standardmäßig .exe-Dateien mit der Version 6.00, die mindestens
Windows Vista benötigen.
Mit Änderungen an den Compiler-Option\footnote{\href{http://go.yurichev.com/17045}{MSDN}}
ist es jedoch möglich eine Kompilierung für Windows XP zu erzwingen.

\subsubsection{Sektionen}

Abschnitte in Sektionen sind in allen Formaten für ausführbare Dateien vorhanden.

Dies wurde entwickelt um Code von Daten und Daten von Konstanten zu trennen.

\begin{itemize}
\item Entweder das \IT{IMAGE\_SCN\_CNT\_CODE}- oder \IT{IMAGE\_SCN\_MEM\_EXECUTE}-Flag
ist in der Code-Sektion gesetzt. Dies kennzeichnet ausführbaren Code.

\item In der Daten-Sektion sind die Flags \IT{IMAGE\_SCN\_CNT\_INITIALIZED\_DATA},
\IT{IMAGE\_SCN\_MEM\_READ} und \IT{IMAGE\_SCN\_MEM\_WRITE} gesetzt.

\item In einer leeren Sektion mit uninitialisierten Daten sind die Flags
\IT{IMAGE\_SCN\_CNT\_UNINITIALIZED\_DATA}, \IT{IMAGE\_SCN\_MEM\_READ} und
\IT{IMAGE\_SCN\_MEM\_WRITE} gesetzt.

\item In der Sektion mit konstanten Daten (die schreibgeschützt ist), sind die
Flags \IT{IMAGE\_SCN\_CNT\_INITIALIZED\_DATA} und \IT{IMAGE\_SCN\_MEM\_READ}
gesetzt, nicht jedoch \IT{IMAGE\_SCN\_MEM\_WRITE}.
Ein Prozess wird abstürzen, wenn er versucht hier schreibend zuzugreifen.

\end{itemize}

\myindex{TLS}
\myindex{BSS}
Jede Sektion in einer PE-Datei kann einen Namen haben, auch wenn dieser nicht
unbedingt wichtig ist.
Oft (aber nicht immer) ist die Code-Sektion mit \TT{.text} benannt, die Datensektion
mit \TT{.data}, und die Sektion mit konstanten Daten \TT{.rdata} für \IT{(readable data)}.

Andere verbreitete Sektionsnamen sind:

\myindex{MIPS}
\begin{itemize}
\item \TT{.idata}: Import-Sektion
\ac{IDA} kann eine Pseudosektion mit dem Namen \myref{IDA_idata} erzeugen.
\item \TT{.edata}: Export-Sektion (selten)
\item \TT{.pdata}: Sektion, die alle Informationen über Ausnahmen in Windows NT
für MIPS, \ac{IA64} und x64 enthält: \myref{SEH_win64}
\item \TT{.reloc}: Reloc-Sektion
\item \TT{.bss}: uninitialisierte Daten (\ac{BSS})
\item \TT{.tls}: Thread-lokaler Speicher (\ac{TLS})
\item \TT{.rsrc}: Ressourcen
\item \TT{.CRT}: kann in Binärdateien vorhanden sein die mit alten MSVC-Compilern
erzeugt wurden
\end{itemize}

Pack- und Verschlüsselungsprogramme für PE-Dateien verändern häufig die Sektionsnamen
oder ersetzen sie durch eigene Namen.

\ac{MSVC} ermöglicht es Daten in beliebigen benannte Sektion zu
deklarieren\footnote{\href{http://go.yurichev.com/17047}{MSDN}}.

Einige Compiler und Linker können eine Sektion mit Debug-Symbolen und anderen
Debug-Informationen hinzufügen (MinGW zum Beispiel).
\myindex{Windows!PDB}
Nichtsdestotrotz ist dies in der aktuellen Version von \ac{MSVC} möglich.
Hier gibt es separate \gls{PDB}-Dateien für diesen Zweck.

Nachfolgend der Aufbau der PE-Sektion in dieser Datei:

\begin{lstlisting}
typedef struct _IMAGE_SECTION_HEADER {
  BYTE  Name[IMAGE_SIZEOF_SHORT_NAME];
  union {
    DWORD PhysicalAddress;
    DWORD VirtualSize;
  } Misc;
  DWORD VirtualAddress;
  DWORD SizeOfRawData;
  DWORD PointerToRawData;
  DWORD PointerToRelocations;
  DWORD PointerToLinenumbers;
  WORD  NumberOfRelocations;
  WORD  NumberOfLinenumbers;
  DWORD Characteristics;
} IMAGE_SECTION_HEADER, *PIMAGE_SECTION_HEADER;
\end{lstlisting}
\footnote{\href{http://go.yurichev.com/17048}{MSDN}}

\myindex{Hiew}
Ein Wort zur Terminologie: \IT{PointerToRawData} wird in Hiew \q{Offset}
und \IT{VirtualAddress} \q{RVA} genannt.

\subsubsection{Datensektion}
Die Datensektion in der Datei kann kleiner sein als im Arbeitsspeicher. Zum
Beispiel können einige Variablen initialisiert sein und andere nicht.
Compiler und Linker sammeln diese alle in einer einzelnen Sektion, aber
der erste Teil davon ist initialisiert und in der Datei enthalten, während
ein anderer in der Datei fehlt um diese kleiner zu machen.
\IT{VirtualSize} wird gleich groß sein wie die Sektion im Speicher und
\IT{SizeOfRawData} wie die Sektion in der Datei.

IDA kann die Grenze der initialisierten und nicht initialisierten Teile
wie folgt anzeigen:

\begin{lstlisting}
...

.data:10017FFA                 db    0
.data:10017FFB                 db    0
.data:10017FFC                 db    0
.data:10017FFD                 db    0
.data:10017FFE                 db    0
.data:10017FFF                 db    0
.data:10018000                 db    ? ;
.data:10018001                 db    ? ;
.data:10018002                 db    ? ;
.data:10018003                 db    ? ;
.data:10018004                 db    ? ;
.data:10018005                 db    ? ;

...
\end{lstlisting}

\subsubsection{Relocations (relocs)}
\label{subsec:relocs}

\ac{AKA} FIXUPs (zumindest in Hiew).
Diese sind ebenfalls in den meisten Formaten für ausführbare Dateien
vorhanden\footnote{Sogar in .exe-Dateien für MS-DOS.}.
Ausnahmen sind gemeinsam genutzte (dynamische) Bibliotheken, die \ac{PIC} enthalten

Wofür dienen die relocs?

Offensichtlich können Module an verschiedene Basisadressen geladen werden. Wie
wird jedoch zum Beispiel mit globalen Variablen umgegangen?
Auf diese muss anhand der Adresse zugegriffen werden. Eine Möglichkeit dazu ist
\PICcode{} (\myref{sec:PIC}), was aber nicht immer komfortabel ist

Aus diesem Grund existieren Relocation-Tabellen. Hier sind die Adressen die korrigiert
werden müssen aufgelistet, falls an eine andere Basisadresse geladen wird.

% TODO тут бы пример с HIEW или objdump..
Beispielsweise ist eine globale Variable an Adresse \TT{0x410000}. Das folgende
Listing zeigt wie auf diese zugegriffen wird:

\begin{lstlisting}
A1 00 00 41 00         mov         eax,[000410000]
\end{lstlisting}

Die Basisadresse des Moduls ist \TT{0x400000}, die \ac{RVA} der globalen Variablen
ist \TT{0x10000}.

Falls das Modul an die Basisadresse \TT{0x500000} geladen wird, muss die reale
Adresse der globalen Variablen auf \TT{0x510000} geändert werden.

\myindex{x86!\Instructions!MOV}

Wie man sieht ist die Adresse der Variablen in der Anweisung \TT{MOV}, nach dem
\TT{0xA1} kodiert.

Aus diesem Grund wird die Adresse der vier Byte nach \TT{0xA1} in die
Relocation-Tabelle geschrieben.

Wenn das Modul an einer anderen Basisadresse geladen wird, listet der \ac{OS}-Lader
alle Adressen in der Tabelle auf, findet jedes 32-Bit-Word auf das die Adresse zeigt,
subtrahiert die Basisadresse davon (das ergibt hier die \ac{RVA}) und addiert die
neue Basisadresse hinzu.

Wird das Modul an der originalen Basisadresse geladen passiert nichts.

Alle globalen Variablen können auf diese Weise behandelt werden.

Relocs können verschiedene Typen haben. In Windows für x86-Prozessoren ist dieser
üblicherweise \IT{IMAGE\_REL\_BASED\_HIGHLOW}.

\myindex{Hiew}

Übrigens sind Relocs in Hiew abgedunkelt, wie hier zu sehen: \figref{fig:scanf_ex3_hiew_1}.

\myindex{\olly}
\olly unterstreicht die Orte im Speicher auf die Relocs angewendet wurden,
beispielsweise: \figref{fig:switch_lot_olly3}.

%\subsubsection{Exports and imports}
%
%\label{PE_exports_imports}
%As we all know, any executable program must use the \ac{OS}'s services and other DLL-libraries somehow.
%
%It can be said that functions from one module (usually DLL) must be connected somehow to the points of their
%calls in other modules (.exe-file or another DLL).
%
%For this, each DLL has an \q{exports} table, which consists of functions plus their addresses in a module.
%
%And every .exe file or DLL has \q{imports}, a table of functions it needs for execution including
%list of DLL filenames.
%
%After loading the main .exe-file, the \ac{OS} loader processes imports table:
%it loads the additional DLL-files, finds function names
%among the DLL exports and writes their addresses down in the \ac{IAT} of the main .exe-module.
%
%\myindex{Windows!Win32!Ordinal}
%
%As we can see, during loading the loader must compare a lot of function names, but string comparison is not a very
%fast procedure, so there is a support for \q{ordinals} or \q{hints},
%which are function numbers stored in the table, instead of their names.
%
%That is how they can be located faster when loading a DLL.
%Ordinals are always present in the \q{export} table.
%
%\myindex{MFC}
%For example, a program using the \ac{MFC} library usually loads mfc*.dll by ordinals,
%and in such programs there are no \ac{MFC} function names in \ac{INT}.
%
%% TODO example!
%When loading such programs in \IDA, it will ask for a path to the mfc*.dll files
%in order to determine the function names.
%
%If you don't tell \IDA the path to these DLLs, there will be \IT{mfc80\_123} instead of function names.
%
%\myparagraph{Imports section}
%
%Often a separate section is allocated for the imports table and everything related to it (with name like \TT{.idata}),
%however, this is not a strict rule.
%
%Imports are also a confusing subject because of the terminological mess. Let's try to collect all information in one place.
%
%\begin{figure}[H]
%\centering
%\myincludegraphics{OS/PE/unnamed0.png}
%\caption{
%A scheme that unites all PE-file structures related to imports}
%\end{figure}
%
%The main structure is the array \IT{IMAGE\_IMPORT\_DESCRIPTOR}.
%Each element for each DLL being imported.
%
%Each element holds the \ac{RVA} address of the text string (DLL name) (\IT{Name}).
%
%\IT{OriginalFirstThunk} is the \ac{RVA} address of the \ac{INT} table.
%This is an array of \ac{RVA} addresses, each of which points to a text string with a function name.
%Each string is prefixed by a 16-bit integer
%(\q{hint})---\q{ordinal} of function.
%
%While loading, if it is possible to find a function by ordinal,
%then the strings comparison will not occur. The array is terminated by zero.
%
%There is also a pointer to the \ac{IAT} table named \IT{FirstThunk}, it is just the \ac{RVA} address
%of the place where the loader writes the addresses of the resolved functions.
%
%The points where the loader writes addresses are marked by \IDA like this: \IT{\_\_imp\_CreateFileA}, etc.
%
%There are at least two ways to use the addresses written by the loader.
%
%\myindex{x86!\Instructions!CALL}
%\begin{itemize}
%\item The code will have instructions like \IT{call \_\_imp\_CreateFileA},
%and since the field with the address of the imported function is a global variable in some sense,
%the address of the \IT{call} instruction (plus 1 or 2) is to be added to the relocs table,
%for the case when the module is loaded at a different base address.
%
%But, obviously, this may enlarge relocs table significantly.
%
%Because there are might be a lot of calls to imported functions in the module.
%
%Furthermore, large relocs table slows down the process of loading modules.
%
%\myindex{x86!\Instructions!JMP}
%\myindex{thunk-functions}
%\item For each imported function, there is only one jump allocated, using the \JMP instruction
%plus a reloc to it.
%Such points are also called \q{thunks}.
%
%All calls to the imported functions are just \CALL instructions to the corresponding \q{thunk}.
%In this case, additional relocs are not necessary because these \CALL{}-s
%have relative addresses and do not need to be corrected.
%\end{itemize}
%
%These two methods can be combined.
%
%Possible, the linker creates individual \q{thunk}s if there are too many calls to the function,
%but not done by default. \\
%\\
%By the way, the array of function addresses to which FirstThunk is pointing is not necessary to be located in the \ac{IAT} section.
%For example, the author of these lines once wrote the PE\_add\_import\footnote{\href{http://go.yurichev.com/17049}{yurichev.com}}
%utility for adding imports to an existing .exe-file.
%
%Some time earlier, in the previous versions of the utility,
%at the place of the function you want to substitute with a call to another DLL,
%my utility wrote the following code:
%
%\begin{lstlisting}
%MOV EAX, [yourdll.dll!function]
%JMP EAX
%\end{lstlisting}
%
%FirstThunk points to the first instruction. In other words, when loading yourdll.dll,
%the loader writes the address of the \IT{function} function right in the code.
%
%It also worth noting that a code section is usually write-protected, so my utility adds the \\
%\IT{IMAGE\_SCN\_MEM\_WRITE}
%flag for code section. Otherwise, the program to crash while loading with error code
%5 (access denied). \\
%\\
%One might ask: what if I supply a program with a set of DLL files which is not supposed to change (including addresses of all DLL functions),
%is it possible to speed up the loading process?
%
%Yes, it is possible to write the addresses of the functions to be imported into the FirstThunk arrays in advance.
%The \IT{Timestamp} field is present in the \\
%\IT{IMAGE\_IMPORT\_DESCRIPTOR} structure.
%
%If a value is present there, then the loader compares this value with the date-time of the DLL file.
%
%If the values are equal, then the loader does not do anything, and the loading of the process can be faster.
%This is called \q{old-style binding}
%\footnote{\href{http://go.yurichev.com/17050}{MSDN}. There is also the \q{new-style binding}.}.
%\myindex{BIND.EXE}
%
%The BIND.EXE utility in Windows SDK is for for this.
%For speeding up the loading of your program, Matt Pietrek in \PietrekPEURL, suggests to do the binding shortly after your program
%installation on the computer of the end user. \\
%\\
%PE-files packers/encryptors may also compress/encrypt imports table.
%
%In this case, the Windows loader, of course, will not load all necessary DLLs.
%\myindex{Windows!Win32!LoadLibrary}
%\myindex{Windows!Win32!GetProcAddress}
%
%Therefore, the packer/encryptor does this on its own, with the help of
%\IT{LoadLibrary()} and the \IT{GetProcAddress()} functions.
%
%That is why these two functions are often present in \ac{IAT} in packed files.\\
%\\
%In the standard DLLs from the Windows installation, \ac{IAT} often is located right at the beginning of the PE file.
%Supposedly, it is made so for optimization.
%
%While loading, the .exe file is not loaded into memory as a whole (recall huge install programs which are
%started suspiciously fast), it is \q{mapped}, and loaded into memory in parts as they are accessed.
%
%Probably, Microsoft developers decided it will be faster.
%
%\subsubsection{Resources}
%
%\label{PEresources}
%
%Resources in a PE file are just a set of icons, pictures, text strings, dialog descriptions.
%
%Perhaps they were separated from the main code, so all these things could be multilingual,
%and it would be simpler to pick text or picture for the language that is currently set in the \ac{OS}. \\
%\\
%As a side effect, they can be edited easily and saved back to the executable file, even if one does not have special knowledge,
%by using the ResHack editor, for example (\myref{ResHack}).
%
%\subsubsection{.NET}
%
%\myindex{.NET}
%
%.NET programs are not compiled into machine code but into a special bytecode.
%\myindex{OEP}
%Strictly speaking, there is bytecode instead of the usual x86 code
%in the .exe file, however, the entry point (\ac{OEP}) points to this tiny fragment of x86 code:
%
%\begin{lstlisting}
%jmp         mscoree.dll!_CorExeMain
%\end{lstlisting}
%
%The .NET loader is located in mscoree.dll, which processes the PE file.
%\myindex{Windows!Windows XP}
%
%It was so in all pre-Windows XP \ac{OS}es. Starting from XP, the \ac{OS} loader is able to detect the .NET file
%and run it without executing that \JMP instruction
%\footnote{\href{http://go.yurichev.com/17051}{MSDN}}.
%
%\myindex{TLS}
%\subsubsection{TLS}
%
%This section holds initialized data for the \ac{TLS}(\myref{TLS}) (if needed).
%When a new thread start, its \ac{TLS} data is initialized using the data from this section. \\
%\\
%\myindex{TLS!Callbacks}
%
%Aside from that, the PE file specification also provides initialization of the
%\ac{TLS} section, the so-called TLS callbacks.
%
%If they are present, they are to be called before the control is passed to the main entry point (\ac{OEP}).
%
%This is used widely in the PE file packers/encryptors.
%
%\subsubsection{Tools}
%
%\myindex{objdump}
%\myindex{Cygwin}
%\myindex{Hiew}
%\label{ResHack}
%
%\begin{itemize}
%\item objdump (present in cygwin) for dumping all PE-file structures.
%
%\item Hiew(\myref{Hiew}) as editor.
%
%\item pefile---Python-library for PE-file processing \footnote{\url{http://go.yurichev.com/17052}}.
%
%\item ResHack \acs{AKA} Resource Hacker---resources editor\footnote{\url{http://go.yurichev.com/17052}}.
%
%\item PE\_add\_import\footnote{\url{http://go.yurichev.com/17049}}---
%simple tool for adding symbol(s) to PE executable import table.
%
%\item PE\_patcher\footnote{\href{http://go.yurichev.com/17054}{yurichev.com}}---simple tool for patching PE executables.
%
%\item PE\_search\_str\_refs\footnote{\href{http://go.yurichev.com/17055}{yurichev.com}}---simple tool for searching for a function in PE executables which use some text string.
%\end{itemize}
%
%\subsubsection{Further reading}
%
%% FIXME: bibliography per chapter or section
%\begin{itemize}
%\item Daniel Pistelli---The .NET File Format \footnote{\url{http://go.yurichev.com/17056}}
%\end{itemize}
%
