\subsubsection{Win32}

\myparagraph{Uninitialized \ac{TLS} data}

One solution is to add \TT{\_\_declspec( thread )} modifier to the global variable, 
then it will be allocated in the \ac{TLS} (line 9):

\lstinputlisting[numbers=left,style=customc]{OS/TLS/win32/rand_uninit.c}

Hiew shows us that there is a new PE section in the executable file: \TT{.tls}.
% TODO1 hiew screenshot?

\lstinputlisting[caption=\Optimizing MSVC 2013 x86,style=customasm]{OS/TLS/win32/rand_x86_uninit.asm}

\TT{rand\_state} is now in the \ac{TLS} segment, and each thread has its own version of this variable.

Here is how it's accessed: load the address of the \ac{TIB} from FS:2Ch, then add an additional index (if needed),
then calculate the address of the \ac{TLS} segment.

Then it's possible to access the \TT{rand\_state} variable through the ECX register, which points to an unique area in each thread.

\myindex{x86!\Registers!FS}

The \TT{FS:} selector is familiar to every reverse engineer, it is specially used to always point to \ac{TIB},
so it would be fast to load the thread-specific data.

\myindex{x86!\Registers!GS}

The \TT{GS:} selector is used in Win64 and the address of the \ac{TLS} is 0x58:

\lstinputlisting[caption=\Optimizing MSVC 2013 x64,style=customasm]{OS/TLS/win32/rand_x64_uninit.asm}

\myparagraph{Initialized \ac{TLS} data}

Let's say, we want to set some fixed value to \TT{rand\_state}, so in case the programmer forgets to,
the \TT{rand\_state} variable would be initialized to some constant anyway (line 9):

\lstinputlisting[numbers=left,style=customc]{OS/TLS/win32/rand_init.c}

The code is no differ from what we already saw, but in IDA we see:

\lstinputlisting[style=customasm]{OS/TLS/win32/rand_init_IDA.lst}

1234 is there and every time a new thread starts, a new \ac{TLS} is allocated for it, 
and all this data, including 1234, will be copied there.

This is a typical scenario:

\begin{itemize}
\item Thread A is started. A \ac{TLS} is created for it, 1234 is copied to \TT{rand\_state}.

\item The \TT{my\_rand()} function is called several times in thread A.\\
\TT{rand\_state} is different from 1234.

\item Thread B is started. A \ac{TLS} is created for it, 1234 is copied to \TT{rand\_state}, 
while thread A has a different value in the same variable.
\end{itemize}

\myparagraph{\ac{TLS} callbacks}
\myindex{TLS!Callbacks}

But what if the variables in the \ac{TLS} have to be filled with some data that must be prepared in some unusual way?

Let's say, we've got the following task:
the programmer can forget to call the \TT{my\_srand()} function to initialize the \ac{PRNG}, but the generator has to be 
initialized at start with something truly random, instead of 1234.
This is a case in which \ac{TLS} callbacks can be used.

The following code is not very portable due to the hack, but nevertheless, you get the idea.

What we do here is define a function (\TT{tls\_callback()}) which is to be called \IT{before} 
the process and/or thread start.

The function initializes the \ac{PRNG} with the value returned by \TT{GetTickCount()} function.

\lstinputlisting[style=customc]{OS/TLS/win32/rand_TLS_callback.c}

Let's see it in IDA:

\lstinputlisting[caption=\Optimizing MSVC 2013,style=customasm]{OS/TLS/win32/rand_TLS_callback.lst}

TLS callback functions are sometimes used in unpacking routines to obscure their processing.

Some people may be confused and be in the dark that some code executed right before the \ac{OEP}.
