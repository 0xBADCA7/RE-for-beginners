\section{Thread Local Storage}
\label{TLS}
\myindex{TLS}

TLS est un espace de données propre à chaque thread, qui peut y conserver ce qu'il estime utile.
Un exemple d'utilisation bien connu en est la variable globale \IT{errno} du standard C.

Plusieurs threads peuvent invoquer simultanément différentes fonctions qui retournent toutes un code 
erreur dans la variable \IT{errno}. L'utilisation d'une variable globale dans le contexte d'un 
programme comportant plusieurs threads serait donc inadaptée dans ce cas. C'est pourquoi la variable 
\IT{errno} est conservée dans l'espace \ac{TLS}.\\
\\
\myindex{\Cpp!C++11}
La version 11 du standard C++ a ajouté un nouveau modificateur \IT{thread\_local} qui indique que 
chaque thread possède sa propre copie de la variable décorée par ce modificateur. La variable en 
question est alors conservée dans l'espace \ac{TLS}
\footnote{\myindex{C11} C11 also has thread support, optional though}:

\begin{lstlisting}[caption=C++11,style=customc]
#include <iostream>
#include <thread>

thread_local int tmp=3;

int main()
{
	std::cout << tmp << std::endl;
};
\end{lstlisting}

Compilé avec MinGW GCC 4.8.1, mais pas avec MSVC 2012.

Dans le contexte des fichiers au format PE, la variable \IT{tmp} sera allouée dans la section dédiée 
au \ac{TLS} du fichier exécutable résultant de la compilation.

\subsection{Amélioration du générateur linéaire congruent}
\label{LCG_TLS}

Le générateur de nombres pseudo-aléatoires \myref{LCG_simple} que nous avons étudié précédemment 
contient une faiblesse. Son comportement est buggé dans un environnement multi-thread. Il utilise 
en effet une variable d'état dont la valeur peut être lue et/ou modifiée par plusieurs threads 
simultanément.

% subsections
\subsubsection{Win32}

\myparagraph{Données \ac{TLS} non initialisées}

Une solution consiste à décorer la variable globale avec le modificateur \TT{\_\_declspec( thread )}. 
Elle sera alors allouée dans le \ac{TLS} (ligne 9):

\lstinputlisting[numbers=left,style=customc]{OS/TLS/win32/rand_uninit.c}

Hiew nous montre alors qu'il existe une nouvelle section nommée \TT{.tls} dans le fichier PE.
% TODO1 hiew screenshot?

\lstinputlisting[caption=\Optimizing MSVC 2013 x86,style=customasmx86]{OS/TLS/win32/rand_x86_uninit.asm}

La variable \TT{rand\_state} se trouve donc maintenant dans le segment \ac{TLS} et chaque thread en 
possède sa propre copie.

Voyons comment elle est accédée. Pour cela, chargeons l'adresse du \ac{TIB} depuis FS:2Ch. 
Ajoutons-y (si nécessaire) un index puis calculons l'adresse du segment \ac{TLS}.

Il est ainsi possible d'accéder la valeur de la variable \TT{rand\_state} au travers du registre 
ECX qui contient une adresse propre à chaque thread.

\myindex{x86!\Registers!FS}

Le sélecteur \TT{FS:} est connu de tous les reverse engineer. Il est spécifiquement utilisé pour 
toujours contenir l'adresse du \ac{TIB} du thread en cours d'exécution. L'accès aux données propres 
à chaque thread est donc une opération performante.

\myindex{x86!\Registers!GS}

En environnement Win64, c'est le sélecteur \TT{GS:} qui est utilisé pour ce faire. L'adresse de 
l'espace \ac{TLS} y est conservé à l'offset 0x58 :

\lstinputlisting[caption=\Optimizing MSVC 2013 x64,style=customasmx86]{OS/TLS/win32/rand_x64_uninit.asm}

\myparagraph{Initialisation des données \ac{TLS}}

Imaginons maintenant que nous voulons nous prémunir des erreurs de programmation en initialisant 
systématiquement la variable \TT{rand\_state} avec une valeur constante (ligne 9):

\lstinputlisting[numbers=left,style=customc]{OS/TLS/win32/rand_init.c}

Le code ne semble pas différent de celui que nous avons étudié. Pourtant dans IDA nous constatons:

\lstinputlisting[style=customasmx86]{OS/TLS/win32/rand_init_IDA.lst}

La valeur 1234 est bien présente. Chaque fois qu'un nouveau thread est créé, un nouvel espace \ac{TLS} 
est alloué pour ses besoins et toutes les données - y compris 1234 - y sont copiées.

Considérons le scénario hypothétique suivant:

\begin{itemize}
\item Le thread A démarre. Un espace \ac{TLS} est créé pour ses besoins et la valeur 1234 est copiée 
dans \TT{rand\_state}.

\item La fonction \TT{my\_rand()} est invoquée plusieurs fois par le thread A.\\
La valeur de la variable \TT{rand\_state} est maintenant différente de 1234.

\item Le thread B démarre. Un espace \ac{TLS} est créé pour ses besoins et la valeur 1234 est copiée 
dans \TT{rand\_state}. Dans le thread A, la valeur de \TT{rand\_state} demeure différente de 1234.
\end{itemize}

\myparagraph{Fonctions de rappel \ac{TLS}}
\myindex{TLS!Callbacks}

Mais comment procédons-nous si les variables conservées dans l'environnement \ac{TLS} doivent être 
initialisées avec des valeurs qui ne sont pas constantes ?

Imaginons le scénario suivant:
Il se peut que le programmeur oublie d'invoquer la fonction \TT{my\_srand()} pour initialiser le 
\ac{PRNG}. Malgré cela, le générateur doit être initialisé avec une valeur réellement aléatoire et 
non pas avec 1234. C'est précisément dans ce genre de cas que les fonctions de rappel \ac{TLS} 
sont utilisées.

Le code ci-dessous n'est pas vraiment portable du fait du bricolage, mais vous devriez comprendre 
l'idée.

Nous définissons une fonction (\TT{tls\_callback()}) qui doit être invoquée avant le démarrage du 
processus et/ou d'un thread.

Cette fonction initialise le \ac{PRNG} avec la valeur retournée par la fonction \TT{GetTickCount()}.

\lstinputlisting[style=customc]{OS/TLS/win32/rand_TLS_callback.c}

Voyons cela dans IDA:

\lstinputlisting[caption=\Optimizing MSVC 2013,style=customasmx86]{OS/TLS/win32/rand_TLS_callback.lst}

Les fonctions de rappel TLS sont parfois utilisées par les mécanismes de décompression d'exécutable 
afin d'en rendre le fonctionnement plus obscure.

Cette pratique peut en laisser certains dans le noir parce qu'ils auront omis de considérer qu'un 
fragment de code a pu s'exécuter avant l'\ac{OEP}.

\subsubsection{Linux}

Voyons maintenant comment une variable globale conservée dans l'espace de stockage propre au thread 
est déclarée avec GCC:

\begin{lstlisting}
__thread uint32_t rand_state=1234;
\end{lstlisting}

Il ne s'agit pas du modificateur standard \CCpp modifier, mais bien d'un modificateur spécifique à 
GCC
\footnote{\url{http://go.yurichev.com/17062}}.

\myindex{x86!\Registers!GS}

Le sélecteur \TT{GS:} est utilisé lui aussi pour accéder au \ac{TLS}, mais d'une manière un peu 
différente:

\lstinputlisting[caption=\Optimizing GCC 4.8.1 x86,style=customasmx86]{OS/TLS/linux/rand.lst}

% FIXME (to be checked) Uninitialized data is allocated in \TT{.tbss} section, initialized --- in \TT{.tdata} section.

Pour en savoir plus: \DrepperTLS.



