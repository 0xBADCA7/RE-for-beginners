\section{Arguments passing methods (calling conventions)}
\label{sec:callingconventions}

\subsection{cdecl}
\myindex{cdecl}
\label{cdecl}

This is the most popular method for passing arguments to functions in the \CCpp languages.

The gls{caller} also must return the value of the \gls{stack pointer} (\ESP) to its initial state after the \gls{callee} function exits.

\begin{lstlisting}[caption=cdecl,style=customasm]
push arg3
push arg2
push arg1
call function
add esp, 12 ; returns ESP
\end{lstlisting}

\subsection{stdcall}
\label{sec:stdcall}
\myindex{stdcall}

\newcommand{\SIZEOFINT}{The size of an \Tint type variable is 4 in x86 systems and 8 in x64 systems}

It's almost the same as \IT{cdecl}, with the exception that the \gls{callee} must set \ESP to the initial state by executing the \TT{RET x} instruction instead of \RET, \\
where \TT{x = arguments number * sizeof(int)\footnote{\SIZEOFINT}}.
The \gls{caller} is not adjusting the \gls{stack pointer}, 
there are no \TT{add esp, x} instruction.

\begin{lstlisting}[caption=stdcall,style=customasm]
push arg3
push arg2
push arg1
call function

function:
... do something ...
ret 12
\end{lstlisting}

The method is ubiquitous in win32 standard libraries, but not in win64 (see below about win64).

\par For example, we can take the function from \myref{src:passing_arguments_ex} and change it slightly by adding the \TT{\_\_stdcall} modifier:

\begin{lstlisting}[style=customc]
int __stdcall f2 (int a, int b, int c)
{
	return a*b+c;
};
\end{lstlisting}

It is to be compiled in almost the same way as \myref{src:passing_arguments_ex_MSVC_cdecl}, but you will see \TT{RET 12} instead of \TT{RET}.
\ac{SP} is not update in the \gls{caller}.

As a consequence, 
the number of function arguments can be easily deduced from the \TT{RETN n} instruction: just divide $n$ by 4.

\lstinputlisting[caption=MSVC 2010,style=customasm]{OS/calling_conventions/stdcall_ex.asm}

\subsubsection{Functions with variable number of arguments}

\printf-like functions are, probably, the only case of functions with a variable number of arguments in \CCpp,
but it is easy to illustrate an important difference between \IT{cdecl} and \IT{stdcall} with their help.
Let's start with the idea that the compiler knows the argument count of each \printf function call.

However, the called \printf, which is already compiled and located in MSVCRT.DLL (if we talk about Windows),
does not have any information about how much arguments were passed, however it can determine it from the format string.

Thus, if \printf would be a \IT{stdcall} function and restored \gls{stack pointer} to its initial state by counting
the number of arguments in the format string, this could be a dangerous situation, when one programmer's typo can
provoke a sudden program crash.
Thus it is not suitable for such functions to use \IT{stdcall}, \IT{cdecl} is better.

\subsection{fastcall}
\label{fastcall}
\myindex{fastcall}

That's the general naming for the method of passing some arguments via registers and the 
rest via the stack. It worked faster than \IT{cdecl}/\IT{stdcall} on older CPUs 
(because of smaller stack pressure).
It may not help to gain any significant performance on latest (much more complex) \ac{CPU}s, however.

It is not standardized, so the various compilers can do it differently.
It's a well known caveat: if you have two DLLs and the one uses another one, and they are built by different compilers with 
different \IT{fastcall} calling conventions, you can expect problems.

Both MSVC and GCC pass the first and second arguments via \ECX and \EDX and the rest of the arguments via the stack.

The \gls{stack pointer} must be restored to its initial state by the \gls{callee} (like in \IT{stdcall}).

\begin{lstlisting}[caption=fastcall,style=customasm]
push arg3
mov edx, arg2
mov ecx, arg1
call function

function:
.. do something ..
ret 4
\end{lstlisting}

For example, we may take the function from \myref{src:passing_arguments_ex} and change it slightly by adding a \TT{\_\_fastcall} modifier:

\begin{lstlisting}[style=customc]
int __fastcall f3 (int a, int b, int c)
{
	return a*b+c;
};
\end{lstlisting}

Here is how it is to be compiled:

\lstinputlisting[caption=\Optimizing MSVC 2010 /Ob0,style=customasm]{OS/calling_conventions/fastcall_ex.asm}

We see that the \gls{callee} returns \ac{SP} by using the \TT{RETN} instruction with an operand.

Which implies that the number of arguments can be deduced easily here as well.

\subsubsection{GCC regparm}

\newcommand{\URLREGPARMM}{\url{http://go.yurichev.com/17040}}

It is the evolution of \IT{fastcall}\footnote{\URLREGPARMM} in some sense.
With the \TT{-mregparm} option it is possible to set how many arguments are to be passed via registers (3 is the maximum).
Thus, the \EAX, \EDX and \ECX registers are to be used.

Of course, if the number the of arguments is less than 3, not all 3 registers are to be used.

The \gls{caller} restores the \gls{stack pointer} to its initial state.

For example, see (\myref{regparm}).

\subsubsection{Watcom/OpenWatcom}
\myindex{OpenWatcom}

Here it is called \q{register calling convention}.
The first 4 arguments are passed via the \EAX, \EDX, \EBX and \ECX registers.
All the rest---via the stack.

These functions has an underscore appended to the function name in order to distinguish them from 
those having a different calling convention.

\subsection{thiscall}
\myindex{thiscall}

This is passing the object's \ITthis pointer to the function-method, in \Cpp.

In MSVC, \ITthis is usually passed in the \ECX register.

In GCC, the \ITthis pointer is passed as the first function-method argument.
Thus it will be very visible that internally: all function-methods have an extra argument.

For an example, see (\myref{thiscall}).

\subsection{x86-64}
\myindex{x86-64}

\subsubsection{Windows x64}
\label{sec:callingconventions_win64}

The method of for passing arguments in Win64 somewhat resembles \TT{fastcall}.
The first 4 arguments are passed via \RCX, \RDX, \Reg{8} and \Reg{9}, the rest---via the stack.
The \gls{caller} also must prepare space for 32 bytes or 4 64-bit values,
so then the \gls{callee} can save there the first 4 arguments.
Short functions may use the arguments' values just from the registers,
but larger ones may save their values for further use.

The \gls{caller} also must return the \gls{stack pointer} into its initial state.

This calling convention is also used in Windows x86-64 system DLLs 
(instead of \IT{stdcall} in win32).

Example:

\lstinputlisting[style=customc]{OS/calling_conventions/x64.c}

\lstinputlisting[caption=MSVC 2012 /0b,style=customasm]{OS/calling_conventions/x64_MSVC_Ob.asm}

\myindex{Scratch space}

Here we clearly see how 7 arguments are passed: 4 via registers and the remaining 3 via the stack.

The code of the f1() function's prologue saves the arguments in the \q{scratch space}---a space in the stack
intended exactly for this purpose.

This is arranged so because the compiler cannot be sure that there will be enough registers to use without these 4,
which will otherwise be occupied by the arguments until the function's execution end.

The \q{scratch space} allocation in the stack is the caller's duty.

\lstinputlisting[caption=\Optimizing MSVC 2012 /0b,style=customasm]{OS/calling_conventions/x64_MSVC_Ox_Ob.asm}

If we compile the example with optimizations, it is to be almost the same, 
but the \q{scratch space} will not be used, because it won't be needed.

\myindex{x86!\Instructions!LEA}
\label{using_MOV_and_pack_of_LEA_to_load_values}

Also take a look on how MSVC 2012 optimizes the loading of primitive values into registers by using \LEA (\myref{sec:LEA}).
\INS{MOV} would be 1 byte longer here (5 instead of 4).

Another example of such thing is: \myref{TaskMgr_LEA}.

\myparagraph{Windows x64: Passing \ITthis (\CCpp)}

The \ITthis pointer is passed in \RCX, the first argument of the method is in \RDX, etc.
For an example see: \myref{simple_CPP_MSVC_x64}.
 
\subsubsection{Linux x64}

The way arguments are passed in Linux for x86-64 is almost the same as in Windows, but 6 registers are
used instead of 4 (\RDI, \RSI, \RDX, \RCX, \Reg{8}, \Reg{9}) and there is no \q{scratch space}, 
although the \gls{callee} may save the register values in the stack, if it needs/wants to.

\lstinputlisting[caption=\Optimizing GCC 4.7.3,style=customasm]{OS/calling_conventions/x64_linux_O3.s}

\myindex{AMD}

N.B.: here the values are written into the 32-bit parts of the registers (e.g., EAX) but not in the whole 64-bit 
register (RAX).
This is because each write to the low 32-bit part of a register automatically clears the high 32 bits.
Supposedly, it was decided in AMD to do so to simplify porting code to x86-64.

\subsection{Return values of \Tfloat and \Tdouble type}
\myindex{float}
\myindex{double}

In all conventions except in Win64, the values of type \Tfloat or \Tdouble are returned via the FPU register \ST{0}.

In Win64, the values of \Tfloat and \Tdouble types are returned 
in the low 32 or 64 bits of the \XMM{0} register.

\subsection{Modifying arguments}

Sometimes, \CCpp{} programmers (not limited to these \ac{PL}s, though),
may ask, what can happen if they modify the arguments?

The answer is simple: the arguments are stored in the stack, 
that is where the modification takes place.

The calling functions is not using them after the \gls{callee}'s exit (the author of these lines has never seen any such case in his practice).

\lstinputlisting[style=customc]{OS/calling_conventions/change_arguments.c}

\lstinputlisting[caption=MSVC 2012,style=customasm]{OS/calling_conventions/change_arguments.asm}

% TODO (OllyDbg) пример как в стеке меняется $a$

So yes, one can modify the arguments easily.
Of course, if it is not \IT{references} in \Cpp{} (\myref{cpp_references}),
and if you don't modify data to which a pointer points to, 
then the effect will not propagate outside the current function.

Theoretically, after the \gls{callee}'s return, 
the \gls{caller} could get the modified argument and use it somehow.
Maybe if it is written directly in assembly language.

For example, code like this will be generated by usual \CCpp compiler:

\begin{lstlisting}[style=customasm]
	push	456	; will be b
	push	123	; will be a
	call	f	; f() modifies its first argument
	add	esp, 2*4
\end{lstlisting}

We can rewrite this code like:

\begin{lstlisting}[style=customasm]
	push	456	; will be b
	push	123	; will be a
	call	f	; f() modifies its first argument
	pop	eax
	add	esp, 4
	; EAX=1st argument of f() modified in f()
\end{lstlisting}

Hard to imagine, why anyone would need this, but this is possible in practice.
Nevertheless, the \CCpp languages standards don't offer any way to do so.

% sections
\subsection{Taking a pointer to function argument}
\label{pointer_to_argument}

\dots even more than that, it's possible to take a pointer to the function's argument and pass
it to another function:

\lstinputlisting[style=customc]{OS/calling_conventions/ptr_to_argument/10.c}

It's hard to understand how it works until we can see the code:

\lstinputlisting[caption=\Optimizing MSVC 2010,style=customasmx86]{OS/calling_conventions/ptr_to_argument/MSVC_2010_O3.asm}

The address of the place in the stack where $a$ has been passed is just passed to another function.
It modifies the value addressed by the pointer and then \printf prints the modified value.

\par The observant reader might ask, what about calling conventions where the function's arguments are
passed in registers?

That's a situation where the \IT{Shadow Space} is used.

The input value is copied from the register
to the \IT{Shadow Space} in the local stack, and then this address is passed to the other function:

\lstinputlisting[caption=\Optimizing MSVC 2012 x64,style=customasmx86]{OS/calling_conventions/ptr_to_argument/MSVC_2012_x64_O3.asm}

GCC also stores the input value in the local stack:

\lstinputlisting[caption=\Optimizing GCC 4.9.1 x64,style=customasmx86]{OS/calling_conventions/ptr_to_argument/GCC491_x64_O3.s}

GCC for ARM64 does the same, but this space is called \IT{Register Save Area} here:

\lstinputlisting[caption=\Optimizing GCC 4.9.1 ARM64,style=customasmARM]{OS/calling_conventions/ptr_to_argument/GCC49_ARM64_O3.s}

By the way, a similar usage of the \IT{Shadow Space} is also considered here: \myref{variadic_arith_registers}.



