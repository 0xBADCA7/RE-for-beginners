\section{Methoden zur Argumentenübergabe (Aufrufkonventionen)}
\label{sec:callingconventions}

\subsection{cdecl}
\myindex{cdecl}
\label{cdecl}

Hierbei handelt es sich um die am weitesten verbreitete Methode um in \CCpp-Sprachen
Argumente an Funktionen zu übergeben.

Der \gls{caller} muss den Wert des \gls{stack pointer} (\ESP) auf den ursprünglichen Stand
bringen, nachdem \gls{callee}-Funktion beendet wurde.

\begin{lstlisting}[caption=cdecl]
push Argument3
push Argument2
push Argument1
call Funktion
add esp, 12 ; gibt ESP zurueck
\end{lstlisting}

\subsection{stdcall}
\label{sec:stdcall}
\myindex{stdcall}

\newcommand{\SIZEOFINT}{Die Größe einer Variablen vom Datentyp \Tint ist 4 in x86-Systemen und 8 in x64-Systemen}

Dies ist fast gleich zu der \IT{cdecl}-Aufrufkonvention, mit Ausnahme, dass die
\gls{callee} den Wert von \ESP auf den ursprünglichen Wert setzen muss. Dies geschieht
durch die \TT{RET x} anstatt \RET, wobei gilt \TT{x = Nummer des Arguments * sizeof(int)\footnote{\SIZEOFINT}}.

Der \gls{caller} passt den \gls{stack pointer} nicht an, es sind also keine \TT{add esp, x}-Anweisungen vorhanden.

\begin{lstlisting}[caption=stdcall]
push Argument3
push Argument2
push Argument1
call Funktion

Funktion:
... tue etwas ...
ret 12
\end{lstlisting}

Diese Methode ist in Win32-Standard-Bibliotheken allgegenwärtig, fehlt jedoch in Win64 (siehe unten).

\par Beispielsweise kann die Funktion von \myref{src:passing_arguments_ex} genommen werden und durch
Hinzufügen des \TT{\_\_stdcall}-Modifizierers leicht verändert werden:

\begin{lstlisting}
int __stdcall f2 (int a, int b, int c)
{
	return a*b+c;
};
\end{lstlisting}

Das Kompilat ist fast das gleiche wie bei \myref{src:passing_arguments_ex_MSVC_cdecl},
jedoch wird \TT{RET 12} anstatt \TT{RET} genutzt. Der \ac{SP} wird im \gls{caller} nicht
aktualisiert.

Als Konsequenz daraus kann die Anzahl der Funktionsargumente einfach von der \TT{RETN n}-Anweisung
abgeleitet werden, indem $n$ durch 4 geteilt wird

%TODO Compilation of german version fails with style=customasm
%\lstinputlisting[caption=MSVC 2010,style=customasm]{OS/calling_conventions/stdcall_ex.asm}
\lstinputlisting[caption=MSVC 2010]{OS/calling_conventions/stdcall_ex.asm}

\subsubsection{Funktionen mit einer variablen Anzahl von Argumenten}

\printf-ähnliche Funktionen sind vielleicht die einzigen Funktionen in \CCpp mit einer Variablen
Anzahl von Argumenten, aber mit ihnen kann ein wichtiger Unterschied zwischen \IT{cdecl} und
\IT{stdcall} veranschaulicht werden.
Beginnen wir mit der Vorstellung, dass der Compiler die Anzahl der Argumente für jeden Aufruf
von \printf kennt.

Die aufgerufene und bereits kompilierte \printf-Funktion befindet sich in der Datei MSVCRT.DLL
(wenn über Windows geredet wird) und hat aber keinerlei Informationen darüber wie viele Argumente
übergeben wurden; dies kann jedoch über den Formatstring herausgefunden werden.

%TODO: The following sentence is much too long.
Wenn \printf eine \IT{stdcall}-Funktion wäre und den \gls{stack pointer} durch Zählen der Zahl
der Argumente im Formatstring auf den ursprünglichen Wert setzen würde, kann eine gefährliche
Situation entstehen, da ein Schreibfehler des Programmierers zu einem plötzliche Programmabsturz
führen könnte.
Aus diesem Grund ist für diese Art von Funktionen \IT{stdcall} ungeeignet und \IT{cdecl} ist
zu bevorzugen.

\subsection{fastcall}
\label{fastcall}
\myindex{fastcall}

Dies ist der allgemeine Name für eine Methode, in der einige Argumente mittels Registern und
der Rest über den Stack übergeben werden. Für ältere CPUs ist \IT{fastcall} schneller als
\IT{cdecl} und \IT{stdcall} wegen der geringeren Stack-Nutzung.
Auf neueren \ac{CPU}s wird dieser Ansatz vermutlich keine signifikante Geschwindigkeitserhöhung
nach sich ziehen.

\IT{fastcall} ist nicht standardisiert, so das verschiedene Compiler eine unterschiedliche
Umsetzung machen können.
Dies kann zu Problemen führen, wenn zwei DLLs genutzt werden, von denen eine die andere nutzt
und durch das Nutzen verschiedener Compiler unterschiedliche \IT{fastcall}-Aufrufkonventionen
genutzt werden.

Sowohl MSVC als auch GCC übergeben das erste und zweite Argument über \ECX und \EDX und den Rest
der Arguments mittels des Stacks.

Der \gls{stack pointer} muss vom \gls{callee} auf den ursprünglichen Wert gesetzt werden,
wie in \IT{stdcall} auch.

\begin{lstlisting}[caption=fastcall]
push Argument3
mov edx, Argument2
mov ecx, Argument1
call Funktion

Funktion:
.. tue etwas ..
ret 4
\end{lstlisting}

Beispielsweise kann die Funktion von \myref{src:passing_arguments_ex} genommen werden und durch
Hinzufügen des \TT{\_\_fastcall}-Modifizierers leicht verändert werden:

\begin{lstlisting}
int __fastcall f3 (int a, int b, int c)
{
	return a*b+c;
};
\end{lstlisting}

Nachfolgend das Kompilat:

%TODO Compilation of german version fails with style=customasm
%\lstinputlisting[caption=\Optimizing MSVC 2010 /Ob0,style=customasm]{OS/calling_conventions/fastcall_ex.asm}
\lstinputlisting[caption=\Optimizing MSVC 2010 /Ob0]{OS/calling_conventions/fastcall_ex.asm}

Es ist erkennbar, dass der \gls{callee} den \ac{SP} mit der \TT{RETN}-Anweisung
und einem Operanden auf den ursprünglichen Wert setzt.

Dies bedeutet, dass die Anzahl der Argumente ebenfalls einfach abgeleitet werden kann.

\subsubsection{GCC regparm}

\newcommand{\URLREGPARMM}{\url{http://go.yurichev.com/17040}}

Dies ist in gewisser Weise die Weiterentwicklung von \IT{fastcall}\footnote{\URLREGPARMM}.
%With the \TT{-mregparm} option it is possible to set how many arguments are to be passed via registers (3 is the maximum).
%Thus, the \EAX, \EDX and \ECX registers are to be used.
%
%Of course, if the number the of arguments is less than 3, not all 3 registers are to be used.
%
%The \gls{caller} restores the \gls{stack pointer} to its initial state.
%
%For example, see (\myref{regparm}).
%
%\subsubsection{Watcom/OpenWatcom}
%\myindex{OpenWatcom}
%
%Here it is called \q{register calling convention}.
%The first 4 arguments are passed via the \EAX, \EDX, \EBX and \ECX registers.
%All the rest---via the stack.
%
%These functions has an underscore appended to the function name in order to distinguish them from 
%those having a different calling convention.
%
%\subsection{thiscall}
%\myindex{thiscall}
%
%This is passing the object's \ITthis pointer to the function-method, in \Cpp.
%
%In MSVC, \ITthis is usually passed in the \ECX register.
%
%In GCC, the \ITthis pointer is passed as the first function-method argument.
%Thus it will be very visible that internally: all function-methods have an extra argument.
%
%For an example, see (\myref{thiscall}).
%
%\subsection{x86-64}
%\myindex{x86-64}
%
%\subsubsection{Windows x64}
%\label{sec:callingconventions_win64}
%
%The method of for passing arguments in Win64 somewhat resembles \TT{fastcall}.
%The first 4 arguments are passed via \RCX, \RDX, \Reg{8} and \Reg{9}, the rest---via the stack.
%The \gls{caller} also must prepare space for 32 bytes or 4 64-bit values,
%so then the \gls{callee} can save there the first 4 arguments.
%Short functions may use the arguments' values just from the registers,
%but larger ones may save their values for further use.
%
%The \gls{caller} also must return the \gls{stack pointer} into its initial state.
%
%This calling convention is also used in Windows x86-64 system DLLs 
%(instead of \IT{stdcall} in win32).
%
%Example:
%
%\lstinputlisting[style=customc]{OS/calling_conventions/x64.c}
%
%\lstinputlisting[caption=MSVC 2012 /0b,style=customasm]{OS/calling_conventions/x64_MSVC_Ob.asm}
%
%\myindex{Scratch space}
%
%Here we clearly see how 7 arguments are passed: 4 via registers and the remaining 3 via the stack.
%
%The code of the f1() function's prologue saves the arguments in the \q{scratch space}---a space in the stack
%intended exactly for this purpose.
%
%This is arranged so because the compiler cannot be sure that there will be enough registers to use without these 4,
%which will otherwise be occupied by the arguments until the function's execution end.
%
%The \q{scratch space} allocation in the stack is the caller's duty.
%
%\lstinputlisting[caption=\Optimizing MSVC 2012 /0b,style=customasm]{OS/calling_conventions/x64_MSVC_Ox_Ob.asm}
%
%If we compile the example with optimizations, it is to be almost the same, 
%but the \q{scratch space} will not be used, because it won't be needed.
%
%\myindex{x86!\Instructions!LEA}
%\label{using_MOV_and_pack_of_LEA_to_load_values}
%
%Also take a look on how MSVC 2012 optimizes the loading of primitive values into registers by using \LEA (\myref{sec:LEA}).
%\INS{MOV} would be 1 byte longer here (5 instead of 4).
%
%Another example of such thing is: \myref{TaskMgr_LEA}.
%
%\myparagraph{Windows x64: Passing \ITthis (\CCpp)}
%
%The \ITthis pointer is passed in \RCX, the first argument of the method is in \RDX, etc.
%For an example see: \myref{simple_CPP_MSVC_x64}.
% 
%\subsubsection{Linux x64}
%
%The way arguments are passed in Linux for x86-64 is almost the same as in Windows, but 6 registers are
%used instead of 4 (\RDI, \RSI, \RDX, \RCX, \Reg{8}, \Reg{9}) and there is no \q{scratch space}, 
%although the \gls{callee} may save the register values in the stack, if it needs/wants to.
%
%\lstinputlisting[caption=\Optimizing GCC 4.7.3,style=customasm]{OS/calling_conventions/x64_linux_O3.s}
%
%\myindex{AMD}
%
%N.B.: here the values are written into the 32-bit parts of the registers (e.g., EAX) but not in the whole 64-bit 
%register (RAX).
%This is because each write to the low 32-bit part of a register automatically clears the high 32 bits.
%Supposedly, it was decided in AMD to do so to simplify porting code to x86-64.
%
%\subsection{Return values of \Tfloat and \Tdouble type}
%\myindex{float}
%\myindex{double}
%
%In all conventions except in Win64, the values of type \Tfloat or \Tdouble are returned via the FPU register \ST{0}.
%
%In Win64, the values of \Tfloat and \Tdouble types are returned 
%in the low 32 or 64 bits of the \XMM{0} register.
%
%\subsection{Modifying arguments}
%
%Sometimes, \CCpp{} programmers (not limited to these \ac{PL}s, though),
%may ask, what can happen if they modify the arguments?
%
%The answer is simple: the arguments are stored in the stack, 
%that is where the modification takes place.
%
%The calling functions is not using them after the \gls{callee}'s exit (the author of these lines has never seen any such case in his practice).
%
%\lstinputlisting[style=customc]{OS/calling_conventions/change_arguments.c}
%
%\lstinputlisting[caption=MSVC 2012,style=customasm]{OS/calling_conventions/change_arguments.asm}
%
%% TODO (OllyDbg) пример как в стеке меняется $a$
%
%So yes, one can modify the arguments easily.
%Of course, if it is not \IT{references} in \Cpp{} (\myref{cpp_references}),
%and if you don't modify data to which a pointer points to, 
%then the effect will not propagate outside the current function.
%
%Theoretically, after the \gls{callee}'s return, 
%the \gls{caller} could get the modified argument and use it somehow.
%Maybe if it is written directly in assembly language.
%
%For example, code like this will be generated by usual \CCpp compiler:
%
%\begin{lstlisting}
%	push	456	; will be b
%	push	123	; will be a
%	call	f	; f() modifies its first argument
%	add	esp, 2*4
%\end{lstlisting}
%
%We can rewrite this code like:
%
%\begin{lstlisting}
%	push	456	; will be b
%	push	123	; will be a
%	call	f	; f() modifies its first argument
%	pop	eax
%	add	esp, 4
%	; EAX=1st argument of f() modified in f()
%\end{lstlisting}
%
%Hard to imagine, why anyone would need this, but this is possible in practice.
%Nevertheless, the \CCpp languages standards don't offer any way to do so.
%
%% sections
%\subsection{Taking a pointer to function argument}
\label{pointer_to_argument}

\dots even more than that, it's possible to take a pointer to the function's argument and pass
it to another function:

\lstinputlisting[style=customc]{OS/calling_conventions/ptr_to_argument/10.c}

It's hard to understand how it works until we can see the code:

\lstinputlisting[caption=\Optimizing MSVC 2010,style=customasmx86]{OS/calling_conventions/ptr_to_argument/MSVC_2010_O3.asm}

The address of the place in the stack where $a$ has been passed is just passed to another function.
It modifies the value addressed by the pointer and then \printf prints the modified value.

\par The observant reader might ask, what about calling conventions where the function's arguments are
passed in registers?

That's a situation where the \IT{Shadow Space} is used.

The input value is copied from the register
to the \IT{Shadow Space} in the local stack, and then this address is passed to the other function:

\lstinputlisting[caption=\Optimizing MSVC 2012 x64,style=customasmx86]{OS/calling_conventions/ptr_to_argument/MSVC_2012_x64_O3.asm}

GCC also stores the input value in the local stack:

\lstinputlisting[caption=\Optimizing GCC 4.9.1 x64,style=customasmx86]{OS/calling_conventions/ptr_to_argument/GCC491_x64_O3.s}

GCC for ARM64 does the same, but this space is called \IT{Register Save Area} here:

\lstinputlisting[caption=\Optimizing GCC 4.9.1 ARM64,style=customasmARM]{OS/calling_conventions/ptr_to_argument/GCC49_ARM64_O3.s}

By the way, a similar usage of the \IT{Shadow Space} is also considered here: \myref{variadic_arith_registers}.


%
