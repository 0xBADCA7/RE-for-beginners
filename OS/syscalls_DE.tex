\section{Systemaufrufe}
%
%\label{syscalls}
%\myindex{syscall}
%
%\myindex{kernel space}
%\myindex{user space}
%As we know, all running processes inside an \ac{OS} are divided into two categories:
%those having full access to the hardware (\q{kernel space}) 
%and those that do not (\q{user space}).
%
%The \ac{OS} kernel and usually the drivers are in the first category.
%
%All applications are usually in the second category.
%
%\myindex{Glibc}
%For example, Linux kernel is in \IT{kernel space}, but Glibc in \IT{user space}.
%
%This separation is crucial for the safety of the \ac{OS}: it is very important not to give to any process the possibility to screw up
%something in other processes or even in the \ac{OS} kernel.
%\myindex{kernel panic}
%\myindex{BSoD}
%On the other hand, a failing driver or error inside the \ac{OS}'s kernel usually leads to a kernel panic or \ac{BSOD}.
%
%The protection in the x86 processors allows to separate everything into 4 levels of protection (rings), but both in Linux
%and in Windows only two are used: ring0 (\q{kernel space}) and ring3 (\q{user space}).
%
%System calls (syscall-s)
%are a point where these two areas are connected.
%
%It can be said that this is the main \ac{API} provided to applications.
%
%As in \gls{Windows NT}, the syscalls table resides in the \ac{SSDT}.
%
%\myindex{Shellcode}
%
%The usage of syscalls is very popular among shellcode and computer viruses authors, 
%because it is hard to determine the addresses of
%needed functions in the system libraries, but it is easier to use syscalls. However, much more code has to be
%written due to the lower level of abstraction of the \ac{API}.
%
%It is also worth noting that the syscall numbers may be different in various OS versions.
%
%\subsection{Linux}
%\label{linux_syscall}
%
%\myindex{x86!\Instructions!INT!INT 0x80}
%In Linux, a syscall is usually called via \TT{int 0x80}.
%The call's number is passed in the \EAX register, and any other parameters~---in the other registers.
%
%\lstinputlisting[caption=A simple example of the usage of two syscalls,style=customasm]{OS/linux_syscall.s}
%
%Compilation:
%
%\begin{lstlisting}
%nasm -f elf32 1.s
%ld 1.o
%\end{lstlisting}
%
%The full list of syscalls in Linux: \url{http://go.yurichev.com/17319}.
%
%For system calls interception and tracing in Linux, strace(\myref{strace}) can be used.
%
%\subsection{Windows}
%
%\myindex{x86!\Instructions!INT!INT 0x2e}
%\myindex{x86!\Instructions!SYSENTER}
%
%Here they are called via \TT{int 0x2e} 
%or using the special x86 instruction \TT{SYSENTER}.
%
%The full list of syscalls in Windows: \url{http://go.yurichev.com/17320}.
%
%Further reading:
%
%\q{Windows Syscall Shellcode} by Piotr Bania: \url{http://go.yurichev.com/17321}.
%
