\section{Systemaufrufe}

\label{syscalls}
\myindex{syscall}

\myindex{kernel space}
\myindex{user space}
Wie bekannt werden alle laufende Prozesse in einem \ac{OS} in zwei Kategorien unterteilt:
diejenigen, die vollen Zugriff auf die Hardware haben (\q{kernel space}) und diejenigen
die dies nicht haben (\q{user space}).

Der Kernel des Betriebssystems sowie die Treiber gehören in der Regel zur ersten Kategorie.

Alle Anwendungen gehören in der Regel der zweiten Kategorie an.

\myindex{Glibc}
Beispielsweise gehört der Linux-Kernel in den \IT{kernel space} während Glibc im \IT{user space}
ausgeführt wird.

Diese Unterscheidung ist von großer Bedeutung für die Sicherheit eines \ac{OS}:
es ist sehr wichtig nicht jedem Prozess die Möglichkeit etwas in einem anderen Prozess oder
sogar im \ac{OS}-Kernel zum Absturz zu bringen.
\myindex{kernel panic}
\myindex{BSoD}
Auf der anderen Seite führt ein Fehler im Treiber oder innerhalb des \ac{OS}-Kernels in der Regel
zu einem Kernel-Panic oder \ac{BSOD}.

Der Schutz im x86-Prozessor erlaubt die Unterscheidung in vier unterschiedliche Schutzlevel (Ringe).
Sowohl von Linux als auch von Windows werden jedoch nur zwei genutzt:
Ring 0 (\q{kernel space}) und Ring 3 (\q{user space}).

Systemaufrufe (syscalls) sind die Stelle an der diese beiden Bereiche miteinander verbunden werden.

In diesem Sinne sind die Systemaufrufe die Haupt-\ac{API} für die Anwendungen.

Unter \gls{Windows NT}, befindet sich die Tabelle mit Systemaufrufen in der \ac{SSDT}.

\myindex{Shellcode}

Die Nutzung von Systemaufrufen ist sehr verbreitet bei Shellcode- und Viren-Programmierern,
weil es schwieriger ist die Adresse einer benötigten Funktion herauszufinden als einen
Systemaufruf zu nutzen.
Die Kehrseite ist, das viel mehr Code geschrieben werden muss, aufgrund dem geringeren Grad
an Abstraktion der \ac{API}.

Erwähnenswert ist es das die Anzahl der Systemaufrufe bei den unterschiedlichen Betriebssystemen
variieren kann.

\subsection{Linux}
\label{linux_syscall}

\myindex{x86!\Instructions!INT!INT 0x80}
Unter Linux wird ein Systemaufruf in der Regel mit \TT{int 0x80} aufgerufen.
Die Nummer des Aufrufs wird im \EAX-Register übergeben und die restlichen Parameter in anderen Registern.

%TODO: style=customasm does not compile in german version
%\lstinputlisting[caption=A simple example of the usage of two syscalls,style=customasm]{OS/linux_syscall.s}
\lstinputlisting[caption=Ein einfaches Beispiel zur Nutzung zweier Systemaufrufe]{OS/linux_syscall.s}

Kompilation:

\begin{lstlisting}
nasm -f elf32 1.s
ld 1.o
\end{lstlisting}

Eine vollständige Liste von Systemaufrufen unter Linux: \url{http://go.yurichev.com/17319}.

Um Systemaufrufe unter Linux zu unterbrechen und nachverfolgen zu können, kann strace(\myref{strace})
genutzt werden.

\subsection{Windows}

\myindex{x86!\Instructions!INT!INT 0x2e}
\myindex{x86!\Instructions!SYSENTER}

Hier werden die Systemaufrufe via \TT{int 0x2e} aufgerufen oder über die spezielle x86-Anweisung \TT{SYSENTER}.

Eine vollständige Liste von Systemaufrufen unter Windows: \url{http://go.yurichev.com/17320}.

Weitere Informationen:

\q{Windows Syscall Shellcode} von Piotr Bania: \url{http://go.yurichev.com/17321}.
