\subsubsection{Теперь вспомним MSVC}

\myindex{\Cpp!исключения}
Должно быть, программистам Microsoft были нужны исключения в Си, но не в \Cpp, так что они добавили нестандартное расширение Си в MSVC
\footnote{\href{http://go.yurichev.com/17057}{MSDN}}.
Оно не связано с исключениями в Си++.

% FIXME russian listing:
\begin{lstlisting}[style=customc]
__try
{
    ...
}
__except(filter code)
{
    handler code
}
\end{lstlisting}

Блок \q{finally} может присутствовать вместо код обработчика:

\begin{lstlisting}[style=customc]
__try
{
    ...
}
__finally
{
    ...
}
\end{lstlisting}

Код-фильтр --- это выражение, отвечающее на вопрос, соответствует ли код этого обработчика к поднятому исключению.
Если ваш код слишком большой и не помещается в одно выражение, отдельная функция-фильтр может быть определена.\\
\\
Таких конструкций много в ядре Windows.
Вот несколько примеров оттуда (\ac{WRK}):

\lstinputlisting[caption=WRK-v1.2/base/ntos/ob/obwait.c,style=customc]{OS/SEH/2/wrk_ex1.c}

\lstinputlisting[caption=WRK-v1.2/base/ntos/cache/cachesub.c,style=customc]{OS/SEH/2/wrk_ex2.c}

Вот пример кода-фильтра:

\lstinputlisting[caption=WRK-v1.2/base/ntos/cache/copysup.c,style=customc]{OS/SEH/2/wrk_ex3.c}

Внутри, SEH это расширение исключений поддерживаемых OS.

Но функция обработчик теперь или \TT{\_except\_handler3} (для SEH3) или \TT{\_except\_handler4} (для SEH4).
Код обработчика от MSVC, расположен в его библиотеках, или же в  msvcr*.dll.
Очень важно понимать, что SEH это специфичное для MSVC.
Другие win32-компиляторы могут предлагать что-то совершенно другое.

\myparagraph{SEH3}

SEH3 имеет \TT{\_except\_handler3} как функцию-обработчик, и расширяет структуру \\
\TT{\_EXCEPTION\_REGISTRATION} добавляя указатель на \IT{scope table}
и переменную \IT{previous try level}.
SEH4 расширяет \IT{scope table} добавляя еще 4 значения связанных с защитой от переполнения буфера.\\
\\
\IT{scope table} это таблица, состоящая из указателей на код фильтра и обработчика, для каждого уровня вложенности \IT{try/except}.

\begin{center}
\begin{tikzpicture}[thick,scale=0.85, every node/.style={scale=0.85}]
	\tikzstyle{every path}=[thick]
	\tikzstyle{undefined}=[draw,rectangle,minimum height=1cm, minimum width=3.5cm, text width=3.5cm]
	\tikzstyle{undefinedw}=[draw,rectangle,minimum height=1cm, minimum width=4.5cm, text width=4.5cm]
	\tikzstyle{node}=[draw,rectangle,minimum height=1cm, minimum width=3.5cm, text width=3.5cm, fill=gray!20]
	\tikzstyle{nodew}=[draw,rectangle,minimum height=1cm, minimum width=4.5cm, text width=4.5cm, fill=gray!20]
	\tikzstyle{text_as_rect}=[rectangle,minimum height=1cm, minimum width=3.5cm, text width=3.5cm]
	\tikzstyle{point}=[inner sep=0pt]

	\node[node] (fs) [minimum width=1.5cm, text width=1.5cm] {FS:0};

\begin{scope}[start chain=1 going below, node distance=0cm]
	\node[on chain=1,node] (tib1) [right=1.5cm of fs] {+0: \_\_except\_list};
	\node[on chain=1,undefined] {+4: \dots};
	\node[on chain=1,undefined] {+8: \dots};
\end{scope}
	\node (tib_text) [above of=tib1] {TIB};
	
	\draw [->] (fs.east) -- (tib1.west);

\begin{scope}[start chain=3 going below, node distance=0cm]
	\node[on chain=3,undefined] (u1) [text centered, right=2.5cm of tib1] {\dots};
	%\node[on chain=3,node] (n1prev) {Prev=0xFFFFFFFF (1)};
	\node[on chain=3,node] (n1prev) {Prev=0xFFFFFFFF};
	\node[on chain=3,node] (n1handler) {Handle};
	\node[on chain=3,undefined] [text centered] {\dots};
	\node[on chain=3,node] (n2prev) {Prev};
	\node[on chain=3,node] (n2handler) {Handle};
	\node[on chain=3,undefined] [text centered] {\dots};
	\node[on chain=3,node] (n3prev) {Prev};
	\node[on chain=3,node] (n3handler) {Handle};
	\ifx\SEHfour\undefined	
	\node[on chain=3,node] (n3scopetable) {\ScopeTable};
	\else
	\node[on chain=3,node] (n3scopetable) {\ScopeTable$\oplus$security\_cookie};
	\fi
	\node[on chain=3,node] {\RU{предыдущий уровень try}\EN{previous try level}};
	\node[on chain=3,node] (n3ebp) {EBP};
	\ifx\SEHfour\undefined
	\else
	\node[on chain=3,undefined] [text centered] {\dots};
	\node[on chain=3,node] (AnotherCookie) {EBP$\oplus$security\_cookie};
	\node[on chain=3,undefined] [text centered] {\dots};
	\fi
\end{scope}
	
	\node [node] (n1handler_text) [right=1cm of n1handler] {\HandlerFunction};
	\draw [->] (n1handler.east) -- (n1handler_text.west);
	\node [node] (n2handler_text) [right=1cm of n2handler] {\HandlerFunction};
	\draw [->] (n2handler.east) -- (n2handler_text.west);

	\ifx\SEHfour\undefined	
	\node [node] (n3handler_text) [right=1cm of n3handler] {\_except\_handler3};
	\else
	\node [node] (n3handler_text) [right=1cm of n3handler] {\_except\_handler4};
	\fi

	\draw [->] (n3handler.east) -- (n3handler_text.west);
	\node[undefined] (u4) [text centered, below of=n3ebp] {\dots};
	\node (stack_text) [above of=u1] {Stack};
	
	\node[point] (n2block_pt2) [above left=0cm and 0.8cm of n2prev] {};
	\draw [->] (n3prev.west) .. controls +(left:0.8cm) and (n2block_pt2) .. (n2prev.north west);

	\node[point] (n1block_pt2) [above left=0cm and 0.8cm of n1prev] {};
	\draw [->] (n2prev.west) .. controls +(left:0.8cm) and (n1block_pt2) .. (n1prev.north west);
	
	\node[point] (n3block_pt2) [above left=0cm and 1.25cm of n3prev] {};
	\draw [->] (tib1.east) .. controls +(right:1.25cm) and (n3block_pt2) .. (n3prev.north west);

\begin{scope}[start chain=2 going below, node distance=0cm]
	% SEH3
	\ifx\SEHfour\undefined	
	\node[on chain=2,nodew] (scope_tbl) [left=3.5cm of n2handler] {0xFFFFFFFF (-1)};
	\else
	% SEH4 header
	\node[on chain=2,nodew] (scope_tbl) [left=3.5 cm of n2handler] {\RU{смещение GS Cookie}\EN{GS Cookie Offset}};
	\node[on chain=2,nodew] {\RU{смещение GS Cookie XOR}\EN{GS Cookie XOR Offset}};
	\node[on chain=2,nodew] (EHCookieOffset) {\RU{смещение EH Cookie}\EN{EH Cookie Offset}};
	\node[on chain=2,nodew] {\RU{смещение EH Cookie XOR}\EN{EH Cookie XOR Offset}};
	\node[on chain=2,minimum height=0.3cm] {};
	\node[on chain=2,nodew] {0xFFFFFFFF (-1)};
	\fi

	\node[on chain=2,nodew] (scope_tbl_filter1) {\FilterFunction};
	\node[on chain=2,nodew] {\HandlerFinallyFunction};
	\node[on chain=2,minimum height=0.3cm] {};
	\node[on chain=2,nodew] {0};
	\node[on chain=2,nodew] (scope_tbl_filter2) {\FilterFunction};
	\node[on chain=2,nodew] {\HandlerFinallyFunction};
	\node[on chain=2,minimum height=0.3cm] {};
	\node[on chain=2,nodew] {1};
	\node[on chain=2,nodew] (scope_tbl_filter3) {\FilterFunction};
	\node[on chain=2,nodew] {\HandlerFinallyFunction};
	\node[on chain=2,minimum height=0.3cm] {};
	\node[on chain=2,undefinedw] [text centered] {\dots \MoreEntries \dots};

\end{scope}
	
	\node[text_as_rect,left=0.5cm of scope_tbl_filter1] {\RU{информация для первого блока try/except}\EN{information about first try/except block}};
	\node[text_as_rect,left=0.5cm of scope_tbl_filter2] {\RU{информация для второго блока try/except}\EN{information about second try/except block}};
	\node[text_as_rect,left=0.5cm of scope_tbl_filter3] {\RU{информация для третьего блока try/except}\EN{information about third try/except block}};

	\node (scope_text) [above of=scope_tbl] {\ScopeTable};
	\node[point] (scope_table_pt2) [right=1.75cm of scope_tbl] {};
	\draw [->] (n3scopetable.west) .. controls +(left:1.75cm) and (scope_table_pt2) .. (scope_tbl.north east);

	\ifx\SEHfour\undefined
	\else
	\node[point] (AnotherCookie_pt2) [left=1.75cm of AnotherCookie] {};
	\draw [->] (EHCookieOffset.east) .. controls +(right:1.75cm) and (AnotherCookie_pt2) .. (AnotherCookie.west);
	\fi

\end{tikzpicture}
\end{center}


И снова, очень важно понимать, что OS заботится только о полях \IT{prev/handle}, и больше ничего.
Это работа функции \TT{\_except\_handler3} читать другие поля, читать \IT{scope table} и решать, какой обработчик исполнять и когда.\\
\\
\myindex{Wine}
\myindex{ReactOS}
Исходный код функции \TT{\_except\_handler3} закрыт.
Хотя, Sanos OS, имеющая слой совместимости с win32, имеет некоторые функции написанные заново, которые
в каком-то смысле эквивалентны тем что в Windows
\footnote{\url{http://go.yurichev.com/17058}}.
Другие попытки реализации имеются в 
Wine\footnote{\href{http://go.yurichev.com/17059}{GitHub}}
и ReactOS\footnote{\url{http://go.yurichev.com/17060}}.\\
\\
Если указатель \IT{filter} ноль, \IT{handler} указывает на код \IT{finally} .\\
\\
Во время исполнения, значение \IT{previous try level} в стеке меняется, чтобы функция
\TT{\_except\_handler3} знала о текущем уровне вложенности, чтобы знать, какой элемент таблицы
\IT{scope table} использовать.

\myparagraph{SEH3: пример с одним блоком try/except}

\lstinputlisting[style=customc]{OS/SEH/2/2.c}

\lstinputlisting[caption=MSVC 2003,style=customasmx86]{OS/SEH/2/2_SEH3.asm}

Здесь мы видим, как структура SEH конструируется в стеке.
\IT{Scope table} расположена в сегменте \TT{CONST} --- действительно, эти поля не будут меняться.
Интересно, как меняется переменная \IT{previous try level}.
Исходное значение \TT{0xFFFFFFFF} ($-1$).
Момент, когда тело \TT{try} открывается, обозначен инструкцией, записывающей 0 в эту переменную.
В момент, когда тело \TT{try} закрывается, $-1$ возвращается в нее назад.
Мы также видим адреса кода фильтра и обработчика.
Так мы можем легко увидеть структуру конструкций \IT{try/except} в функции.\\
\\
Так как код инициализации SEH-структур в прологе функций может быть общим для нескольких функций, иногда компилятор
вставляет в прологе вызов функции \TT{SEH\_prolog()}, которая всё это делает.
А код для деинициализации SEH в функции \TT{SEH\_epilog()}.\\
\\
Запустим этот пример в \tracer{}:
\myindex{tracer}

\begin{lstlisting}
tracer.exe -l:2.exe --dump-seh
\end{lstlisting}

\begin{lstlisting}[caption=tracer.exe output]
EXCEPTION_ACCESS_VIOLATION at 2.exe!main+0x44 (0x401054) ExceptionInformation[0]=1
EAX=0x00000000 EBX=0x7efde000 ECX=0x0040cbc8 EDX=0x0008e3c8
ESI=0x00001db1 EDI=0x00000000 EBP=0x0018feac ESP=0x0018fe80
EIP=0x00401054
FLAGS=AF IF RF
* SEH frame at 0x18fe9c prev=0x18ff78 handler=0x401204 (2.exe!_except_handler3)
SEH3 frame. previous trylevel=0
scopetable entry[0]. previous try level=-1, filter=0x401070 (2.exe!main+0x60) handler=0x401088 (2.exe!main+0x78)
* SEH frame at 0x18ff78 prev=0x18ffc4 handler=0x401204 (2.exe!_except_handler3)
SEH3 frame. previous trylevel=0
scopetable entry[0]. previous try level=-1, filter=0x401531 (2.exe!mainCRTStartup+0x18d) handler=0x401545 (2.exe!mainCRTStartup+0x1a1)
* SEH frame at 0x18ffc4 prev=0x18ffe4 handler=0x771f71f5 (ntdll.dll!__except_handler4)
SEH4 frame. previous trylevel=0
SEH4 header:	GSCookieOffset=0xfffffffe GSCookieXOROffset=0x0
		EHCookieOffset=0xffffffcc EHCookieXOROffset=0x0
scopetable entry[0]. previous try level=-2, filter=0x771f74d0 (ntdll.dll!___safe_se_handler_table+0x20) handler=0x771f90eb (ntdll.dll!_TppTerminateProcess@4+0x43)
* SEH frame at 0x18ffe4 prev=0xffffffff handler=0x77247428 (ntdll.dll!_FinalExceptionHandler@16)
\end{lstlisting}

Мы видим, что цепочка SEH состоит из 4-х обработчиков.\\
\\
\myindex{CRT}
Первые два расположены в нашем примере. Два?
Но ведь мы же сделали только один?
Да, второй был установлен в \ac{CRT}-функции 
\TT{\_mainCRTStartup()}, и судя по всему, он обрабатывает как минимум исключения связанные с \ac{FPU}.
Его код можно посмотреть в инсталляции MSVC: \TT{crt/src/winxfltr.c}.\\
\\
Третий это SEH4 в ntdll.dll, и четвертый это обработчик, не имеющий отношения к MSVC, расположенный в ntdll.dll, имеющий \q{говорящее} название функции.\\
\\
Как видно, в цепочке присутствуют обработчики трех типов:
один не связан с MSVC вообще (последний) и два связанных с MSVC: SEH3 и SEH4.

\myparagraph{SEH3: пример с двумя блоками try/except}

\lstinputlisting[style=customc]{OS/SEH/2/3.c}

Теперь здесь два блока \TT{try}.
Так что \IT{scope table} теперь содержит два элемента, один элемент на каждый блок.
\IT{Previous try level} меняется вместе с тем, как исполнение доходит до очередного \TT{try}-блока, либо выходит из него.

\lstinputlisting[caption=MSVC 2003,style=customasmx86]{OS/SEH/2/3_SEH3.asm}

Если установить точку останова на функцию \printf{} вызываемую из обработчика,
мы можем увидеть, что добавился еще один SEH-обработчик.
Наверное, это еще какая-то дополнительная механика, скрытая внутри процесса обработки исключений.
Тут мы также видим \IT{scope table} состоящую из двух элементов.

\begin{lstlisting}
tracer.exe -l:3.exe bpx=3.exe!printf --dump-seh
\end{lstlisting}

\begin{lstlisting}[caption=tracer.exe output]
(0) 3.exe!printf
EAX=0x0000001b EBX=0x00000000 ECX=0x0040cc58 EDX=0x0008e3c8
ESI=0x00000000 EDI=0x00000000 EBP=0x0018f840 ESP=0x0018f838
EIP=0x004011b6
FLAGS=PF ZF IF
* SEH frame at 0x18f88c prev=0x18fe9c handler=0x771db4ad (ntdll.dll!ExecuteHandler2@20+0x3a)
* SEH frame at 0x18fe9c prev=0x18ff78 handler=0x4012e0 (3.exe!_except_handler3)
SEH3 frame. previous trylevel=1
scopetable entry[0]. previous try level=-1, filter=0x401120 (3.exe!main+0xb0) handler=0x40113b (3.exe!main+0xcb)
scopetable entry[1]. previous try level=0, filter=0x4010e8 (3.exe!main+0x78) handler=0x401100 (3.exe!main+0x90)
* SEH frame at 0x18ff78 prev=0x18ffc4 handler=0x4012e0 (3.exe!_except_handler3)
SEH3 frame. previous trylevel=0
scopetable entry[0]. previous try level=-1, filter=0x40160d (3.exe!mainCRTStartup+0x18d) handler=0x401621 (3.exe!mainCRTStartup+0x1a1)
* SEH frame at 0x18ffc4 prev=0x18ffe4 handler=0x771f71f5 (ntdll.dll!__except_handler4)
SEH4 frame. previous trylevel=0
SEH4 header:	GSCookieOffset=0xfffffffe GSCookieXOROffset=0x0
		EHCookieOffset=0xffffffcc EHCookieXOROffset=0x0
scopetable entry[0]. previous try level=-2, filter=0x771f74d0 (ntdll.dll!___safe_se_handler_table+0x20) handler=0x771f90eb (ntdll.dll!_TppTerminateProcess@4+0x43)
* SEH frame at 0x18ffe4 prev=0xffffffff handler=0x77247428 (ntdll.dll!_FinalExceptionHandler@16)
\end{lstlisting}

\myparagraph{SEH4}

\myindex{\BufferOverflow}
\myindex{Security cookie}
Во время атаки переполнения буфера (\myref{subsec:bufferoverflow})
адрес \IT{scope table} может быть перезаписан, так что начиная с MSVC 2005, SEH3 был дополнен защитой от переполнения буфера, до SEH4.
Указатель на \IT{scope table} теперь \glslink{xoring}{про-XOR-ен} с \gls{security cookie}.

\IT{Scope table} расширена, теперь имеет заголовок, содержащий 2 указателя на \IT{security cookies}.
Каждый элемент имеет смещение внутри стека на другое значение: это адрес \glslink{stack frame}{фрейма} (\EBP) также \glslink{xoring}{про-XOR-еный} с 
\TT{security\_cookie} расположенный в стеке.
Это значение будет прочитано во время обработки исключения и проверено на правильность.

\IT{Security cookie} в стеке случайное каждый раз, так что атакующий, как мы надеемся, не может предсказать его.\\
\\
Изначальное значение \IT{previous try level} это $-2$ в SEH4 вместо $-1$.

\def\SEHfour{1}
\begin{center}
\begin{tikzpicture}[thick,scale=0.85, every node/.style={scale=0.85}]
	\tikzstyle{every path}=[thick]
	\tikzstyle{undefined}=[draw,rectangle,minimum height=1cm, minimum width=3.5cm, text width=3.5cm]
	\tikzstyle{undefinedw}=[draw,rectangle,minimum height=1cm, minimum width=4.5cm, text width=4.5cm]
	\tikzstyle{node}=[draw,rectangle,minimum height=1cm, minimum width=3.5cm, text width=3.5cm, fill=gray!20]
	\tikzstyle{nodew}=[draw,rectangle,minimum height=1cm, minimum width=4.5cm, text width=4.5cm, fill=gray!20]
	\tikzstyle{text_as_rect}=[rectangle,minimum height=1cm, minimum width=3.5cm, text width=3.5cm]
	\tikzstyle{point}=[inner sep=0pt]

	\node[node] (fs) [minimum width=1.5cm, text width=1.5cm] {FS:0};

\begin{scope}[start chain=1 going below, node distance=0cm]
	\node[on chain=1,node] (tib1) [right=1.5cm of fs] {+0: \_\_except\_list};
	\node[on chain=1,undefined] {+4: \dots};
	\node[on chain=1,undefined] {+8: \dots};
\end{scope}
	\node (tib_text) [above of=tib1] {TIB};
	
	\draw [->] (fs.east) -- (tib1.west);

\begin{scope}[start chain=3 going below, node distance=0cm]
	\node[on chain=3,undefined] (u1) [text centered, right=2.5cm of tib1] {\dots};
	%\node[on chain=3,node] (n1prev) {Prev=0xFFFFFFFF (1)};
	\node[on chain=3,node] (n1prev) {Prev=0xFFFFFFFF};
	\node[on chain=3,node] (n1handler) {Handle};
	\node[on chain=3,undefined] [text centered] {\dots};
	\node[on chain=3,node] (n2prev) {Prev};
	\node[on chain=3,node] (n2handler) {Handle};
	\node[on chain=3,undefined] [text centered] {\dots};
	\node[on chain=3,node] (n3prev) {Prev};
	\node[on chain=3,node] (n3handler) {Handle};
	\ifx\SEHfour\undefined	
	\node[on chain=3,node] (n3scopetable) {\ScopeTable};
	\else
	\node[on chain=3,node] (n3scopetable) {\ScopeTable$\oplus$security\_cookie};
	\fi
	\node[on chain=3,node] {\RU{предыдущий уровень try}\EN{previous try level}};
	\node[on chain=3,node] (n3ebp) {EBP};
	\ifx\SEHfour\undefined
	\else
	\node[on chain=3,undefined] [text centered] {\dots};
	\node[on chain=3,node] (AnotherCookie) {EBP$\oplus$security\_cookie};
	\node[on chain=3,undefined] [text centered] {\dots};
	\fi
\end{scope}
	
	\node [node] (n1handler_text) [right=1cm of n1handler] {\HandlerFunction};
	\draw [->] (n1handler.east) -- (n1handler_text.west);
	\node [node] (n2handler_text) [right=1cm of n2handler] {\HandlerFunction};
	\draw [->] (n2handler.east) -- (n2handler_text.west);

	\ifx\SEHfour\undefined	
	\node [node] (n3handler_text) [right=1cm of n3handler] {\_except\_handler3};
	\else
	\node [node] (n3handler_text) [right=1cm of n3handler] {\_except\_handler4};
	\fi

	\draw [->] (n3handler.east) -- (n3handler_text.west);
	\node[undefined] (u4) [text centered, below of=n3ebp] {\dots};
	\node (stack_text) [above of=u1] {Stack};
	
	\node[point] (n2block_pt2) [above left=0cm and 0.8cm of n2prev] {};
	\draw [->] (n3prev.west) .. controls +(left:0.8cm) and (n2block_pt2) .. (n2prev.north west);

	\node[point] (n1block_pt2) [above left=0cm and 0.8cm of n1prev] {};
	\draw [->] (n2prev.west) .. controls +(left:0.8cm) and (n1block_pt2) .. (n1prev.north west);
	
	\node[point] (n3block_pt2) [above left=0cm and 1.25cm of n3prev] {};
	\draw [->] (tib1.east) .. controls +(right:1.25cm) and (n3block_pt2) .. (n3prev.north west);

\begin{scope}[start chain=2 going below, node distance=0cm]
	% SEH3
	\ifx\SEHfour\undefined	
	\node[on chain=2,nodew] (scope_tbl) [left=3.5cm of n2handler] {0xFFFFFFFF (-1)};
	\else
	% SEH4 header
	\node[on chain=2,nodew] (scope_tbl) [left=3.5 cm of n2handler] {\RU{смещение GS Cookie}\EN{GS Cookie Offset}};
	\node[on chain=2,nodew] {\RU{смещение GS Cookie XOR}\EN{GS Cookie XOR Offset}};
	\node[on chain=2,nodew] (EHCookieOffset) {\RU{смещение EH Cookie}\EN{EH Cookie Offset}};
	\node[on chain=2,nodew] {\RU{смещение EH Cookie XOR}\EN{EH Cookie XOR Offset}};
	\node[on chain=2,minimum height=0.3cm] {};
	\node[on chain=2,nodew] {0xFFFFFFFF (-1)};
	\fi

	\node[on chain=2,nodew] (scope_tbl_filter1) {\FilterFunction};
	\node[on chain=2,nodew] {\HandlerFinallyFunction};
	\node[on chain=2,minimum height=0.3cm] {};
	\node[on chain=2,nodew] {0};
	\node[on chain=2,nodew] (scope_tbl_filter2) {\FilterFunction};
	\node[on chain=2,nodew] {\HandlerFinallyFunction};
	\node[on chain=2,minimum height=0.3cm] {};
	\node[on chain=2,nodew] {1};
	\node[on chain=2,nodew] (scope_tbl_filter3) {\FilterFunction};
	\node[on chain=2,nodew] {\HandlerFinallyFunction};
	\node[on chain=2,minimum height=0.3cm] {};
	\node[on chain=2,undefinedw] [text centered] {\dots \MoreEntries \dots};

\end{scope}
	
	\node[text_as_rect,left=0.5cm of scope_tbl_filter1] {\RU{информация для первого блока try/except}\EN{information about first try/except block}};
	\node[text_as_rect,left=0.5cm of scope_tbl_filter2] {\RU{информация для второго блока try/except}\EN{information about second try/except block}};
	\node[text_as_rect,left=0.5cm of scope_tbl_filter3] {\RU{информация для третьего блока try/except}\EN{information about third try/except block}};

	\node (scope_text) [above of=scope_tbl] {\ScopeTable};
	\node[point] (scope_table_pt2) [right=1.75cm of scope_tbl] {};
	\draw [->] (n3scopetable.west) .. controls +(left:1.75cm) and (scope_table_pt2) .. (scope_tbl.north east);

	\ifx\SEHfour\undefined
	\else
	\node[point] (AnotherCookie_pt2) [left=1.75cm of AnotherCookie] {};
	\draw [->] (EHCookieOffset.east) .. controls +(right:1.75cm) and (AnotherCookie_pt2) .. (AnotherCookie.west);
	\fi

\end{tikzpicture}
\end{center}


Оба примера скомпилированные в MSVC 2012 с SEH4:

\lstinputlisting[caption=MSVC 2012: one try block example,style=customasmx86]{OS/SEH/2/2_SEH4.asm}

\lstinputlisting[caption=MSVC 2012: two try blocks example,style=customasmx86]{OS/SEH/2/3_SEH4.asm}

Вот значение \IT{cookies}: \TT{Cookie Offset} 
это разница между адресом записанного в стеке значения EBP и значения $EBP \oplus security\_cookie$ в стеке.
\TT{Cookie XOR Offset} это дополнительная разница между значением $EBP \oplus security\_cookie$ и тем что записано в стеке.
Если это уравнение не верно, то процесс остановится из-за разрушения стека:

\begin{center}
$security\_cookie \oplus (CookieXOROffset + address\_of\_saved\_EBP) == stack[address\_of\_saved\_EBP + CookieOffset]$
\end{center}

Если \TT{Cookie Offset} равно $-2$, это значит, что оно не присутствует.

\myindex{tracer}
Проверка \IT{cookies} также реализована в моем \tracer{},
смотрите \href{http://go.yurichev.com/17061}{GitHub} для деталей.\\
\\
Возможность переключиться назад на SEH3 все еще присутствует в компиляторах после (и включая) MSVC 2005, нужно включить
опцию \TT{/GS-}, впрочем, \ac{CRT}-код будет продолжать использовать SEH4.

