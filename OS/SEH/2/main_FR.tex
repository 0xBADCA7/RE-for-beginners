\subsubsection{Retour à MSVC}

\myindex{\Cpp!exceptions}

Microsoft a ajouté un mécanisme non standard de gestion d'exceptions à 
MSVC\footnote{\href{http://go.yurichev.com/17057}{MSDN}} essentiellement à l'usage des programmeurs C.
Ce mécanisme est totalement distinct de celui définit par le standard ISO du langage C++.

% FIXME russian listing:
\begin{lstlisting}[style=customc]
__try
{
    ...
}
__except(filter code)
{
    handler code
}
\end{lstlisting}

A la place du gestionnaire d'exception, on peut trouver un block \q{Finally}:

\begin{lstlisting}[style=customc]
__try
{
    ...
}
__finally
{
    ...
}
\end{lstlisting}

Le code de filtrage est une expression. L'évaluation de celle-ci permet de définir si le gestionnaire 
reconnaît l'exception qui a été déclenchée.

Si votre filtre est trop complexe pour tenir dans une seule expression, une fonction de filtrage 
séparée peut être définie.\\
\\
Il existe de nombreuses constructions de ce type dans le noyau Windows.
En voici quelques exemples (\ac{WRK}):

\lstinputlisting[caption=WRK-v1.2/base/ntos/ob/obwait.c,style=customc]{OS/SEH/2/wrk_ex1.c}

\lstinputlisting[caption=WRK-v1.2/base/ntos/cache/cachesub.c,style=customc]{OS/SEH/2/wrk_ex2.c}

Voici aussi un exemple de code de filtrage:

\lstinputlisting[caption=WRK-v1.2/base/ntos/cache/copysup.c,style=customc]{OS/SEH/2/wrk_ex3.c}

En interne, SEH est une extension du mécanisme de gestion des exceptions implémenté par l'OS.
La fonction de gestion d'exceptions est \TT{\_except\_handler3} (pour SEH3) ou 
\TT{\_except\_handler4} (pour SEH4).

Le code de ce gestionnaire est propre à MSVC et est situé dans ses librairies, ou dans msvcr*.dll.
Il est essentiel de comprendre que SEH est purement lié à MSVC.

D'autres compilateurs win32 peuvent choisir un modèle totalement différent.

\myparagraph{SEH3}

SEH3 est géré par la fonction \TT{\_except\_handler3}. Il ajoute à la structure \TT{\_EXCEPTION\_REGISTRATION} 
un pointeur vers une \IT{scope table} et une variable \IT{previous try level}.
SEH4 de son côté ajoute 4 valeurs à la structure \IT{scope table} pour la gestion des dépassements 
de buffer.\\
\\
La structure \IT{scope table} est un ensemble de pointeurs vers les blocs de code du filtre et du 
gestionnaire de chaque niveau \IT{try/except} imbriqué.

\begin{center}
\begin{tikzpicture}[thick,scale=0.85, every node/.style={scale=0.85}]
	\tikzstyle{every path}=[thick]
	\tikzstyle{undefined}=[draw,rectangle,minimum height=1cm, minimum width=3.5cm, text width=3.5cm]
	\tikzstyle{undefinedw}=[draw,rectangle,minimum height=1cm, minimum width=4.5cm, text width=4.5cm]
	\tikzstyle{node}=[draw,rectangle,minimum height=1cm, minimum width=3.5cm, text width=3.5cm, fill=gray!20]
	\tikzstyle{nodew}=[draw,rectangle,minimum height=1cm, minimum width=4.5cm, text width=4.5cm, fill=gray!20]
	\tikzstyle{text_as_rect}=[rectangle,minimum height=1cm, minimum width=3.5cm, text width=3.5cm]
	\tikzstyle{point}=[inner sep=0pt]

	\node[node] (fs) [minimum width=1.5cm, text width=1.5cm] {FS:0};

\begin{scope}[start chain=1 going below, node distance=0cm]
	\node[on chain=1,node] (tib1) [right=1.5cm of fs] {+0: \_\_except\_list};
	\node[on chain=1,undefined] {+4: \dots};
	\node[on chain=1,undefined] {+8: \dots};
\end{scope}
	\node (tib_text) [above of=tib1] {TIB};
	
	\draw [->] (fs.east) -- (tib1.west);

\begin{scope}[start chain=3 going below, node distance=0cm]
	\node[on chain=3,undefined] (u1) [text centered, right=2.5cm of tib1] {\dots};
	%\node[on chain=3,node] (n1prev) {Prev=0xFFFFFFFF (1)};
	\node[on chain=3,node] (n1prev) {Prev=0xFFFFFFFF};
	\node[on chain=3,node] (n1handler) {Handle};
	\node[on chain=3,undefined] [text centered] {\dots};
	\node[on chain=3,node] (n2prev) {Prev};
	\node[on chain=3,node] (n2handler) {Handle};
	\node[on chain=3,undefined] [text centered] {\dots};
	\node[on chain=3,node] (n3prev) {Prev};
	\node[on chain=3,node] (n3handler) {Handle};
	\ifx\SEHfour\undefined	
	\node[on chain=3,node] (n3scopetable) {\ScopeTable};
	\else
	\node[on chain=3,node] (n3scopetable) {\ScopeTable$\oplus$security\_cookie};
	\fi
	\node[on chain=3,node] {\RU{предыдущий уровень try}\EN{previous try level}};
	\node[on chain=3,node] (n3ebp) {EBP};
	\ifx\SEHfour\undefined
	\else
	\node[on chain=3,undefined] [text centered] {\dots};
	\node[on chain=3,node] (AnotherCookie) {EBP$\oplus$security\_cookie};
	\node[on chain=3,undefined] [text centered] {\dots};
	\fi
\end{scope}
	
	\node [node] (n1handler_text) [right=1cm of n1handler] {\HandlerFunction};
	\draw [->] (n1handler.east) -- (n1handler_text.west);
	\node [node] (n2handler_text) [right=1cm of n2handler] {\HandlerFunction};
	\draw [->] (n2handler.east) -- (n2handler_text.west);

	\ifx\SEHfour\undefined	
	\node [node] (n3handler_text) [right=1cm of n3handler] {\_except\_handler3};
	\else
	\node [node] (n3handler_text) [right=1cm of n3handler] {\_except\_handler4};
	\fi

	\draw [->] (n3handler.east) -- (n3handler_text.west);
	\node[undefined] (u4) [text centered, below of=n3ebp] {\dots};
	\node (stack_text) [above of=u1] {Stack};
	
	\node[point] (n2block_pt2) [above left=0cm and 0.8cm of n2prev] {};
	\draw [->] (n3prev.west) .. controls +(left:0.8cm) and (n2block_pt2) .. (n2prev.north west);

	\node[point] (n1block_pt2) [above left=0cm and 0.8cm of n1prev] {};
	\draw [->] (n2prev.west) .. controls +(left:0.8cm) and (n1block_pt2) .. (n1prev.north west);
	
	\node[point] (n3block_pt2) [above left=0cm and 1.25cm of n3prev] {};
	\draw [->] (tib1.east) .. controls +(right:1.25cm) and (n3block_pt2) .. (n3prev.north west);

\begin{scope}[start chain=2 going below, node distance=0cm]
	% SEH3
	\ifx\SEHfour\undefined	
	\node[on chain=2,nodew] (scope_tbl) [left=3.5cm of n2handler] {0xFFFFFFFF (-1)};
	\else
	% SEH4 header
	\node[on chain=2,nodew] (scope_tbl) [left=3.5 cm of n2handler] {\RU{смещение GS Cookie}\EN{GS Cookie Offset}};
	\node[on chain=2,nodew] {\RU{смещение GS Cookie XOR}\EN{GS Cookie XOR Offset}};
	\node[on chain=2,nodew] (EHCookieOffset) {\RU{смещение EH Cookie}\EN{EH Cookie Offset}};
	\node[on chain=2,nodew] {\RU{смещение EH Cookie XOR}\EN{EH Cookie XOR Offset}};
	\node[on chain=2,minimum height=0.3cm] {};
	\node[on chain=2,nodew] {0xFFFFFFFF (-1)};
	\fi

	\node[on chain=2,nodew] (scope_tbl_filter1) {\FilterFunction};
	\node[on chain=2,nodew] {\HandlerFinallyFunction};
	\node[on chain=2,minimum height=0.3cm] {};
	\node[on chain=2,nodew] {0};
	\node[on chain=2,nodew] (scope_tbl_filter2) {\FilterFunction};
	\node[on chain=2,nodew] {\HandlerFinallyFunction};
	\node[on chain=2,minimum height=0.3cm] {};
	\node[on chain=2,nodew] {1};
	\node[on chain=2,nodew] (scope_tbl_filter3) {\FilterFunction};
	\node[on chain=2,nodew] {\HandlerFinallyFunction};
	\node[on chain=2,minimum height=0.3cm] {};
	\node[on chain=2,undefinedw] [text centered] {\dots \MoreEntries \dots};

\end{scope}
	
	\node[text_as_rect,left=0.5cm of scope_tbl_filter1] {\RU{информация для первого блока try/except}\EN{information about first try/except block}};
	\node[text_as_rect,left=0.5cm of scope_tbl_filter2] {\RU{информация для второго блока try/except}\EN{information about second try/except block}};
	\node[text_as_rect,left=0.5cm of scope_tbl_filter3] {\RU{информация для третьего блока try/except}\EN{information about third try/except block}};

	\node (scope_text) [above of=scope_tbl] {\ScopeTable};
	\node[point] (scope_table_pt2) [right=1.75cm of scope_tbl] {};
	\draw [->] (n3scopetable.west) .. controls +(left:1.75cm) and (scope_table_pt2) .. (scope_tbl.north east);

	\ifx\SEHfour\undefined
	\else
	\node[point] (AnotherCookie_pt2) [left=1.75cm of AnotherCookie] {};
	\draw [->] (EHCookieOffset.east) .. controls +(right:1.75cm) and (AnotherCookie_pt2) .. (AnotherCookie.west);
	\fi

\end{tikzpicture}
\end{center}


Il est essentiel de comprendre que l'\ac{OS} ne se préoccupe que des champs \IT{prev/handle} et de 
rien d'autre.\\
Les autres champs sont exploités par la fonction \TT{\_except\_handler3}, de même que le contenu 
de la structure \IT{scope table} afin de décider quel gestionnaire exécuter et quand.\\
\\
\myindex{Wine}
\myindex{ReactOS}
Le code source de la fonction \TT{\_except\_handler3} n'est pas public.

Cependant, le système d'exploitation Sanos, possède un mode de compatibilité win32.
Celui-ci réimplémente les mêmes fonctions d'une manière quasi équivalente à celle de Windows
\footnote{\url{http://go.yurichev.com/17058}}.
On trouve une autre réimplementation dans Wine\footnote{\href{http://go.yurichev.com/17059}{GitHub}}
ainsi que dans ReactOS\footnote{\url{http://go.yurichev.com/17060}}.\\
\\
Lorsque le champ \IT{filter} est un pointeur NULL, le champ \IT{handler} est un pointeur vers un 
bloc de code \IT{finally}.\\
\\
Au cours de l'exécution, la valeur du champe \IT{previous try level} change. Ceci permet à la 
fonction \TT{\_except\_handler3} de connaître le niveau d'imbrication et donc de savoir quelle 
entrée de la table \IT{scope table} utiliser en cas d'exception.

\myparagraph{SEH3: exemple de bloc try/except}

\lstinputlisting[style=customc]{OS/SEH/2/2.c}

\lstinputlisting[caption=MSVC 2003,style=customasmx86]{OS/SEH/2/2_SEH3.asm}

Nous voyons ici la manière dont le bloc SEH est construit sur la pile.
La structure \IT{scope table} est présente dans le segment \TT{CONST} du programme--- ce qui est 
normal puisque son contenu n'a jamais besoin d'être changé.
Un point intéressant est la manière dont la valeur de la variable \IT{previous try level} évolue.
Sa valeur initiale est \TT{0xFFFFFFFF} ($-1$).
L'entrée dans le bloc \TT{try} débute par l'écriture de la valeur 0 dans la variable.
La sortie du bloc \TT{try} est marquée par la restauration de la valeur $-1$.
Nous voyons également l'adresse du bloc de filtrage et de celui du gestionnaire.

Nous pouvons donc observer facilement la présence de blocs \IT{try/except} dans la fonction.\\
\\
Le code d'initialisation des structures SEH dans le prologue de la fonction peut être partagé par 
de nombreuses fonctions. Le compilateur choisi donc parfois d'insérer dans le prologue d'une fonction 
un appel à la fonction \TT{SEH\_prolog()} qui assure cette initialisation.

Le code de nettoyage des structures SEH se trouve quant à lui dans la fonction \TT{SEH\_epilog()}.\\
\\
Tentons d'exécuter cet exemple dans \tracer{}:
\myindex{tracer}

\begin{lstlisting}
tracer.exe -l:2.exe --dump-seh
\end{lstlisting}

\begin{lstlisting}[caption=tracer.exe output]
EXCEPTION_ACCESS_VIOLATION at 2.exe!main+0x44 (0x401054) ExceptionInformation[0]=1
EAX=0x00000000 EBX=0x7efde000 ECX=0x0040cbc8 EDX=0x0008e3c8
ESI=0x00001db1 EDI=0x00000000 EBP=0x0018feac ESP=0x0018fe80
EIP=0x00401054
FLAGS=AF IF RF
* SEH frame at 0x18fe9c prev=0x18ff78 handler=0x401204 (2.exe!_except_handler3)
SEH3 frame. previous trylevel=0
scopetable entry[0]. previous try level=-1, filter=0x401070 (2.exe!main+0x60) handler=0x401088 (2.exe!main+0x78)
* SEH frame at 0x18ff78 prev=0x18ffc4 handler=0x401204 (2.exe!_except_handler3)
SEH3 frame. previous trylevel=0
scopetable entry[0]. previous try level=-1, filter=0x401531 (2.exe!mainCRTStartup+0x18d) handler=0x401545 (2.exe!mainCRTStartup+0x1a1)
* SEH frame at 0x18ffc4 prev=0x18ffe4 handler=0x771f71f5 (ntdll.dll!__except_handler4)
SEH4 frame. previous trylevel=0
SEH4 header:	GSCookieOffset=0xfffffffe GSCookieXOROffset=0x0
		EHCookieOffset=0xffffffcc EHCookieXOROffset=0x0
scopetable entry[0]. previous try level=-2, filter=0x771f74d0 (ntdll.dll!___safe_se_handler_table+0x20) handler=0x771f90eb (ntdll.dll!_TppTerminateProcess@4+0x43)
* SEH frame at 0x18ffe4 prev=0xffffffff handler=0x77247428 (ntdll.dll!_FinalExceptionHandler@16)
\end{lstlisting}

Nous constatons que la chaîne SEH est constituée de 4 gestionnaires.\\
\\
\myindex{CRT}
Le deux premiers sont situés dans le code de notre exemple. Deux? Mais nous n'en avons défini qu'un!
Effectivement, mais un second a été initialisé dans la fonction \TT{\_mainCRTStartup()} du \ac{CRT}.
Il semble que celui-ci gère au moins les exceptions \ac{FPU}.
Son code source figure dans le fichier \TT{crt/src/winxfltr.c} fournit avec l'installation de MSVC.\\
\\
Le troisième est le gestionnaire SEH4 dans ntdll.dll.
Le quatrième n'est pas lié à MSVC et se situe dans ntdll.dll. Son nom suffit à en décrire l'utilité.\\
\\
Comme vous le constatez, nous avons 3 types de gestionnaire dans la même chaîne:

L'un n'a rien à voir avec MSVC (le dernier) et deux autres sont liés à MSVC: SEH3 et SEH4.

\myparagraph{SEH3: exemple de deux blocs try/except}

\lstinputlisting[style=customc]{OS/SEH/2/3.c}

Nous avons maintenant deux blocs \TT{try}.
La structure \IT{scope table} possède donc deux entrées, une pour chaque bloc.
La valeur de \IT{Previous try level} change selon que l'exécution entre ou sort des blocs \TT{try}.

\lstinputlisting[caption=MSVC 2003,style=customasmx86]{OS/SEH/2/3_SEH3.asm}

Si nous positionnons un point d'arrêt sur la fonction \printf{} qui est appelée par le gestionnaire, 
nous pouvons constater comment un nouveau gestionnaire SEH est ajouté.

Il s'agit peut-être d'un autre mécanisme interne de la gestion SEH.
Nous constatons aussi que notre structure \IT{scope table} contient 2 entrées.

\begin{lstlisting}
tracer.exe -l:3.exe bpx=3.exe!printf --dump-seh
\end{lstlisting}

\begin{lstlisting}[caption=tracer.exe output]
(0) 3.exe!printf
EAX=0x0000001b EBX=0x00000000 ECX=0x0040cc58 EDX=0x0008e3c8
ESI=0x00000000 EDI=0x00000000 EBP=0x0018f840 ESP=0x0018f838
EIP=0x004011b6
FLAGS=PF ZF IF
* SEH frame at 0x18f88c prev=0x18fe9c handler=0x771db4ad (ntdll.dll!ExecuteHandler2@20+0x3a)
* SEH frame at 0x18fe9c prev=0x18ff78 handler=0x4012e0 (3.exe!_except_handler3)
SEH3 frame. previous trylevel=1
scopetable entry[0]. previous try level=-1, filter=0x401120 (3.exe!main+0xb0) handler=0x40113b (3.exe!main+0xcb)
scopetable entry[1]. previous try level=0, filter=0x4010e8 (3.exe!main+0x78) handler=0x401100 (3.exe!main+0x90)
* SEH frame at 0x18ff78 prev=0x18ffc4 handler=0x4012e0 (3.exe!_except_handler3)
SEH3 frame. previous trylevel=0
scopetable entry[0]. previous try level=-1, filter=0x40160d (3.exe!mainCRTStartup+0x18d) handler=0x401621 (3.exe!mainCRTStartup+0x1a1)
* SEH frame at 0x18ffc4 prev=0x18ffe4 handler=0x771f71f5 (ntdll.dll!__except_handler4)
SEH4 frame. previous trylevel=0
SEH4 header:	GSCookieOffset=0xfffffffe GSCookieXOROffset=0x0
		EHCookieOffset=0xffffffcc EHCookieXOROffset=0x0
scopetable entry[0]. previous try level=-2, filter=0x771f74d0 (ntdll.dll!___safe_se_handler_table+0x20) handler=0x771f90eb (ntdll.dll!_TppTerminateProcess@4+0x43)
* SEH frame at 0x18ffe4 prev=0xffffffff handler=0x77247428 (ntdll.dll!_FinalExceptionHandler@16)
\end{lstlisting}

\myparagraph{SEH4}

\myindex{\BufferOverflow}
\myindex{Security cookie}
Lors d'une attaque par dépassement de buffer (\myref{subsec:bufferoverflow}), l'adresse de la 
\IT{scope table} peut être modifiée. C'est pourquoi à partir de MSVC 2005 SEH3 a été amélioré vers 
SEH4 pour ajouter une protection contre ce type d'attaque.
Le pointeur vers la structure \IT{scope table} est désormais \glslink{xoring}{xored} avec la valeur 
d'un \gls{security cookie}.
Par ailleurs, la structure \IT{scope table} a été étendue avec une en-tête contenant deux pointeurs 
vers des \IT{security cookies}.

Chaque élément contient un offset dans la pile d'une valeur correspondant à: 
addresse du \gls{stack frame} (\EBP) \glslink{xoring}{xored} avec la valeur du \TT{security\_cookie} 
lui aussi situé sur la pile.

Durant la gestion d'exception, l'intégrité de cette valeur est vérifiée.
La valeur de chaque \IT{security cookie} situé sur la pile est aléatoire. Une attaque à distance ne 
pourra donc pas la deviner. \\
\\
Avec SEH4, la valeur initiale de \IT{previous try level} est de $-2$ et non de $-1$.

\def\SEHfour{1}
\begin{center}
\begin{tikzpicture}[thick,scale=0.85, every node/.style={scale=0.85}]
	\tikzstyle{every path}=[thick]
	\tikzstyle{undefined}=[draw,rectangle,minimum height=1cm, minimum width=3.5cm, text width=3.5cm]
	\tikzstyle{undefinedw}=[draw,rectangle,minimum height=1cm, minimum width=4.5cm, text width=4.5cm]
	\tikzstyle{node}=[draw,rectangle,minimum height=1cm, minimum width=3.5cm, text width=3.5cm, fill=gray!20]
	\tikzstyle{nodew}=[draw,rectangle,minimum height=1cm, minimum width=4.5cm, text width=4.5cm, fill=gray!20]
	\tikzstyle{text_as_rect}=[rectangle,minimum height=1cm, minimum width=3.5cm, text width=3.5cm]
	\tikzstyle{point}=[inner sep=0pt]

	\node[node] (fs) [minimum width=1.5cm, text width=1.5cm] {FS:0};

\begin{scope}[start chain=1 going below, node distance=0cm]
	\node[on chain=1,node] (tib1) [right=1.5cm of fs] {+0: \_\_except\_list};
	\node[on chain=1,undefined] {+4: \dots};
	\node[on chain=1,undefined] {+8: \dots};
\end{scope}
	\node (tib_text) [above of=tib1] {TIB};
	
	\draw [->] (fs.east) -- (tib1.west);

\begin{scope}[start chain=3 going below, node distance=0cm]
	\node[on chain=3,undefined] (u1) [text centered, right=2.5cm of tib1] {\dots};
	%\node[on chain=3,node] (n1prev) {Prev=0xFFFFFFFF (1)};
	\node[on chain=3,node] (n1prev) {Prev=0xFFFFFFFF};
	\node[on chain=3,node] (n1handler) {Handle};
	\node[on chain=3,undefined] [text centered] {\dots};
	\node[on chain=3,node] (n2prev) {Prev};
	\node[on chain=3,node] (n2handler) {Handle};
	\node[on chain=3,undefined] [text centered] {\dots};
	\node[on chain=3,node] (n3prev) {Prev};
	\node[on chain=3,node] (n3handler) {Handle};
	\ifx\SEHfour\undefined	
	\node[on chain=3,node] (n3scopetable) {\ScopeTable};
	\else
	\node[on chain=3,node] (n3scopetable) {\ScopeTable$\oplus$security\_cookie};
	\fi
	\node[on chain=3,node] {\RU{предыдущий уровень try}\EN{previous try level}};
	\node[on chain=3,node] (n3ebp) {EBP};
	\ifx\SEHfour\undefined
	\else
	\node[on chain=3,undefined] [text centered] {\dots};
	\node[on chain=3,node] (AnotherCookie) {EBP$\oplus$security\_cookie};
	\node[on chain=3,undefined] [text centered] {\dots};
	\fi
\end{scope}
	
	\node [node] (n1handler_text) [right=1cm of n1handler] {\HandlerFunction};
	\draw [->] (n1handler.east) -- (n1handler_text.west);
	\node [node] (n2handler_text) [right=1cm of n2handler] {\HandlerFunction};
	\draw [->] (n2handler.east) -- (n2handler_text.west);

	\ifx\SEHfour\undefined	
	\node [node] (n3handler_text) [right=1cm of n3handler] {\_except\_handler3};
	\else
	\node [node] (n3handler_text) [right=1cm of n3handler] {\_except\_handler4};
	\fi

	\draw [->] (n3handler.east) -- (n3handler_text.west);
	\node[undefined] (u4) [text centered, below of=n3ebp] {\dots};
	\node (stack_text) [above of=u1] {Stack};
	
	\node[point] (n2block_pt2) [above left=0cm and 0.8cm of n2prev] {};
	\draw [->] (n3prev.west) .. controls +(left:0.8cm) and (n2block_pt2) .. (n2prev.north west);

	\node[point] (n1block_pt2) [above left=0cm and 0.8cm of n1prev] {};
	\draw [->] (n2prev.west) .. controls +(left:0.8cm) and (n1block_pt2) .. (n1prev.north west);
	
	\node[point] (n3block_pt2) [above left=0cm and 1.25cm of n3prev] {};
	\draw [->] (tib1.east) .. controls +(right:1.25cm) and (n3block_pt2) .. (n3prev.north west);

\begin{scope}[start chain=2 going below, node distance=0cm]
	% SEH3
	\ifx\SEHfour\undefined	
	\node[on chain=2,nodew] (scope_tbl) [left=3.5cm of n2handler] {0xFFFFFFFF (-1)};
	\else
	% SEH4 header
	\node[on chain=2,nodew] (scope_tbl) [left=3.5 cm of n2handler] {\RU{смещение GS Cookie}\EN{GS Cookie Offset}};
	\node[on chain=2,nodew] {\RU{смещение GS Cookie XOR}\EN{GS Cookie XOR Offset}};
	\node[on chain=2,nodew] (EHCookieOffset) {\RU{смещение EH Cookie}\EN{EH Cookie Offset}};
	\node[on chain=2,nodew] {\RU{смещение EH Cookie XOR}\EN{EH Cookie XOR Offset}};
	\node[on chain=2,minimum height=0.3cm] {};
	\node[on chain=2,nodew] {0xFFFFFFFF (-1)};
	\fi

	\node[on chain=2,nodew] (scope_tbl_filter1) {\FilterFunction};
	\node[on chain=2,nodew] {\HandlerFinallyFunction};
	\node[on chain=2,minimum height=0.3cm] {};
	\node[on chain=2,nodew] {0};
	\node[on chain=2,nodew] (scope_tbl_filter2) {\FilterFunction};
	\node[on chain=2,nodew] {\HandlerFinallyFunction};
	\node[on chain=2,minimum height=0.3cm] {};
	\node[on chain=2,nodew] {1};
	\node[on chain=2,nodew] (scope_tbl_filter3) {\FilterFunction};
	\node[on chain=2,nodew] {\HandlerFinallyFunction};
	\node[on chain=2,minimum height=0.3cm] {};
	\node[on chain=2,undefinedw] [text centered] {\dots \MoreEntries \dots};

\end{scope}
	
	\node[text_as_rect,left=0.5cm of scope_tbl_filter1] {\RU{информация для первого блока try/except}\EN{information about first try/except block}};
	\node[text_as_rect,left=0.5cm of scope_tbl_filter2] {\RU{информация для второго блока try/except}\EN{information about second try/except block}};
	\node[text_as_rect,left=0.5cm of scope_tbl_filter3] {\RU{информация для третьего блока try/except}\EN{information about third try/except block}};

	\node (scope_text) [above of=scope_tbl] {\ScopeTable};
	\node[point] (scope_table_pt2) [right=1.75cm of scope_tbl] {};
	\draw [->] (n3scopetable.west) .. controls +(left:1.75cm) and (scope_table_pt2) .. (scope_tbl.north east);

	\ifx\SEHfour\undefined
	\else
	\node[point] (AnotherCookie_pt2) [left=1.75cm of AnotherCookie] {};
	\draw [->] (EHCookieOffset.east) .. controls +(right:1.75cm) and (AnotherCookie_pt2) .. (AnotherCookie.west);
	\fi

\end{tikzpicture}
\end{center}


Voici deux exemples compilés avec MSVC 2012 et SEH4:

\lstinputlisting[caption=MSVC 2012: exemple bloc try unique,style=customasmx86]{OS/SEH/2/2_SEH4.asm}

\lstinputlisting[caption=MSVC 2012: exemple de deux blocs try,style=customasmx86]{OS/SEH/2/3_SEH4.asm}

La signification de \IT{cookies} est la suivante: \TT{Cookie Offset} est la différence entre l'adresse 
dans la pile de la dernière valeur sauvegardée du registre EBP et de l'adresse dans la pile du 
résultat de l'addition $EBP \oplus security\_cookie$.
\TT{Cookie XOR Offset} est quant à lui la différence entre $EBP \oplus security\_cookie$ et la valeur 
conservée sur la pile.

Si le prédicat ci-dessous n'est pas respecté, le processus est arrêté du fait d'une corruption 
de la pile:

\begin{center}
$security\_cookie \oplus (CookieXOROffset + address\_of\_saved\_EBP) == stack[address\_of\_saved\_EBP + CookieOffset]$
\end{center}

Lorsque \TT{Cookie Offset} vaut $-2$, ceci indique qu'il n'est pas présent.

\myindex{tracer}
La vérification des \IT{Cookies} est aussi implémentée dans mon \tracer{},
voir \href{http://go.yurichev.com/17061}{GitHub} pour les détails.\\
\\
Pour les versions à partir de MSVC 2005, il est toujours possible de revenir à la version SEH3 
en utilisant l'option \TT{/GS-}. Toutefois, le \ac{CRT} continue à utiliser SEH4.

