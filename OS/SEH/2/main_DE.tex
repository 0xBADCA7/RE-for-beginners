\subsubsection{Zurück zu MSVC}

\myindex{\Cpp!exceptions}
Offensichtlich benötigten die Microsoft-Entwickler Ausnahmen in C aber nicht in
\Cpp und führten eine nicht-standardisierte C-Erweiterung ein \footnote{\href{http://go.yurichev.com/17057}{MSDN}}.
Diese hat aber keinen Zusammenhang zu  C++ \ac{PL}-Ausnahmen.

% FIXME russian listing:
\begin{lstlisting}[style=customc]
__try
{
    ...
}
__except(filter code)
{
    handler code
}
\end{lstlisting}

Der \q{Finally}-Block kann anstelle des Handler-Codes stehen:

\begin{lstlisting}[style=customc]
__try
{
    ...
}
__finally
{
    ...
}
\end{lstlisting}

Der Filter-Code ist ein Ausdruck, der anzeigt, ob dieser Handler-Code zu der
geworfenen Ausnahme passt.

If der Code zu groß ist und nicht in einen Ausdruck passt, kann eine separate
Filter-Funktion definiert werden.

Im Windows-Kernel existieren eine Reihe solcher Konstrukte.
Nachfolgend einige Beispiel von dort (\ac{WRK}):

\lstinputlisting[caption=WRK-v1.2/base/ntos/ob/obwait.c,style=customc]{OS/SEH/2/wrk_ex1.c}

\lstinputlisting[caption=WRK-v1.2/base/ntos/cache/cachesub.c,style=customc]{OS/SEH/2/wrk_ex2.c}

Hier ist ein Filter-Code-Beispiel:

\lstinputlisting[caption=WRK-v1.2/base/ntos/cache/copysup.c,style=customc]{OS/SEH/2/wrk_ex3.c}

Intern ist SEH eine Erweiterung der vom \ac{OS}-unterstützten Ausnahmen,
aber die Handler-Funktion ist \TT{\_except\_handler3} (für SEH3) oder \TT{\_except\_handler4} (für SEH4).

Der Code dieses Handlers ist MSVC-spezifisch und befindet sich in dessen Bibliotheken
oder in der msvcr*.dll.
Es ist wichtig zu wissen, dass SEH eine MSVC-spezifische Sache ist.

Andere Win32-Compiler bieten möglicherweise etwas völlig anderes an.

\myparagraph{SEH3}

SEH3 hat \TT{\_except\_handler3} als Handler-Funktion und erweitert die
\TT{\_EXCEPTION\_REGISTRATION}-Tabelle indem ein Zeiger zur \IT{Scope-Tabelle} und
der \IT{previous try level}-Variablen hinzugefügt wird.
SEH4 erweitert die \IT{Scope-Tabelle} um vier Werte für Schutz vor Speicherüberläufen.

Die \IT{Scope-Tabelle} ist eine Tabelle die aus Zeigern auf Filter und Handler-Code-Blöcken
für jede verschachtelte Ebene für \IT{try/except} besteht.

\begin{center}
\begin{tikzpicture}[thick,scale=0.85, every node/.style={scale=0.85}]
	\tikzstyle{every path}=[thick]
	\tikzstyle{undefined}=[draw,rectangle,minimum height=1cm, minimum width=3.5cm, text width=3.5cm]
	\tikzstyle{undefinedw}=[draw,rectangle,minimum height=1cm, minimum width=4.5cm, text width=4.5cm]
	\tikzstyle{node}=[draw,rectangle,minimum height=1cm, minimum width=3.5cm, text width=3.5cm, fill=gray!20]
	\tikzstyle{nodew}=[draw,rectangle,minimum height=1cm, minimum width=4.5cm, text width=4.5cm, fill=gray!20]
	\tikzstyle{text_as_rect}=[rectangle,minimum height=1cm, minimum width=3.5cm, text width=3.5cm]
	\tikzstyle{point}=[inner sep=0pt]

	\node[node] (fs) [minimum width=1.5cm, text width=1.5cm] {FS:0};

\begin{scope}[start chain=1 going below, node distance=0cm]
	\node[on chain=1,node] (tib1) [right=1.5cm of fs] {+0: \_\_except\_list};
	\node[on chain=1,undefined] {+4: \dots};
	\node[on chain=1,undefined] {+8: \dots};
\end{scope}
	\node (tib_text) [above of=tib1] {TIB};
	
	\draw [->] (fs.east) -- (tib1.west);

\begin{scope}[start chain=3 going below, node distance=0cm]
	\node[on chain=3,undefined] (u1) [text centered, right=2.5cm of tib1] {\dots};
	%\node[on chain=3,node] (n1prev) {Prev=0xFFFFFFFF (1)};
	\node[on chain=3,node] (n1prev) {Prev=0xFFFFFFFF};
	\node[on chain=3,node] (n1handler) {Handle};
	\node[on chain=3,undefined] [text centered] {\dots};
	\node[on chain=3,node] (n2prev) {Prev};
	\node[on chain=3,node] (n2handler) {Handle};
	\node[on chain=3,undefined] [text centered] {\dots};
	\node[on chain=3,node] (n3prev) {Prev};
	\node[on chain=3,node] (n3handler) {Handle};
	\ifx\SEHfour\undefined	
	\node[on chain=3,node] (n3scopetable) {\ScopeTable};
	\else
	\node[on chain=3,node] (n3scopetable) {\ScopeTable$\oplus$security\_cookie};
	\fi
	\node[on chain=3,node] {\RU{предыдущий уровень try}\EN{previous try level}};
	\node[on chain=3,node] (n3ebp) {EBP};
	\ifx\SEHfour\undefined
	\else
	\node[on chain=3,undefined] [text centered] {\dots};
	\node[on chain=3,node] (AnotherCookie) {EBP$\oplus$security\_cookie};
	\node[on chain=3,undefined] [text centered] {\dots};
	\fi
\end{scope}
	
	\node [node] (n1handler_text) [right=1cm of n1handler] {\HandlerFunction};
	\draw [->] (n1handler.east) -- (n1handler_text.west);
	\node [node] (n2handler_text) [right=1cm of n2handler] {\HandlerFunction};
	\draw [->] (n2handler.east) -- (n2handler_text.west);

	\ifx\SEHfour\undefined	
	\node [node] (n3handler_text) [right=1cm of n3handler] {\_except\_handler3};
	\else
	\node [node] (n3handler_text) [right=1cm of n3handler] {\_except\_handler4};
	\fi

	\draw [->] (n3handler.east) -- (n3handler_text.west);
	\node[undefined] (u4) [text centered, below of=n3ebp] {\dots};
	\node (stack_text) [above of=u1] {Stack};
	
	\node[point] (n2block_pt2) [above left=0cm and 0.8cm of n2prev] {};
	\draw [->] (n3prev.west) .. controls +(left:0.8cm) and (n2block_pt2) .. (n2prev.north west);

	\node[point] (n1block_pt2) [above left=0cm and 0.8cm of n1prev] {};
	\draw [->] (n2prev.west) .. controls +(left:0.8cm) and (n1block_pt2) .. (n1prev.north west);
	
	\node[point] (n3block_pt2) [above left=0cm and 1.25cm of n3prev] {};
	\draw [->] (tib1.east) .. controls +(right:1.25cm) and (n3block_pt2) .. (n3prev.north west);

\begin{scope}[start chain=2 going below, node distance=0cm]
	% SEH3
	\ifx\SEHfour\undefined	
	\node[on chain=2,nodew] (scope_tbl) [left=3.5cm of n2handler] {0xFFFFFFFF (-1)};
	\else
	% SEH4 header
	\node[on chain=2,nodew] (scope_tbl) [left=3.5 cm of n2handler] {\RU{смещение GS Cookie}\EN{GS Cookie Offset}};
	\node[on chain=2,nodew] {\RU{смещение GS Cookie XOR}\EN{GS Cookie XOR Offset}};
	\node[on chain=2,nodew] (EHCookieOffset) {\RU{смещение EH Cookie}\EN{EH Cookie Offset}};
	\node[on chain=2,nodew] {\RU{смещение EH Cookie XOR}\EN{EH Cookie XOR Offset}};
	\node[on chain=2,minimum height=0.3cm] {};
	\node[on chain=2,nodew] {0xFFFFFFFF (-1)};
	\fi

	\node[on chain=2,nodew] (scope_tbl_filter1) {\FilterFunction};
	\node[on chain=2,nodew] {\HandlerFinallyFunction};
	\node[on chain=2,minimum height=0.3cm] {};
	\node[on chain=2,nodew] {0};
	\node[on chain=2,nodew] (scope_tbl_filter2) {\FilterFunction};
	\node[on chain=2,nodew] {\HandlerFinallyFunction};
	\node[on chain=2,minimum height=0.3cm] {};
	\node[on chain=2,nodew] {1};
	\node[on chain=2,nodew] (scope_tbl_filter3) {\FilterFunction};
	\node[on chain=2,nodew] {\HandlerFinallyFunction};
	\node[on chain=2,minimum height=0.3cm] {};
	\node[on chain=2,undefinedw] [text centered] {\dots \MoreEntries \dots};

\end{scope}
	
	\node[text_as_rect,left=0.5cm of scope_tbl_filter1] {\RU{информация для первого блока try/except}\EN{information about first try/except block}};
	\node[text_as_rect,left=0.5cm of scope_tbl_filter2] {\RU{информация для второго блока try/except}\EN{information about second try/except block}};
	\node[text_as_rect,left=0.5cm of scope_tbl_filter3] {\RU{информация для третьего блока try/except}\EN{information about third try/except block}};

	\node (scope_text) [above of=scope_tbl] {\ScopeTable};
	\node[point] (scope_table_pt2) [right=1.75cm of scope_tbl] {};
	\draw [->] (n3scopetable.west) .. controls +(left:1.75cm) and (scope_table_pt2) .. (scope_tbl.north east);

	\ifx\SEHfour\undefined
	\else
	\node[point] (AnotherCookie_pt2) [left=1.75cm of AnotherCookie] {};
	\draw [->] (EHCookieOffset.east) .. controls +(right:1.75cm) and (AnotherCookie_pt2) .. (AnotherCookie.west);
	\fi

\end{tikzpicture}
\end{center}


%Again, it is very important to understand that the \ac{OS} takes care only of the \IT{prev/handle} fields, and nothing more.\\
%It is the job of the \TT{\_except\_handler3} function to read the other fields and \IT{scope table}, and decide
%which handler to execute and when.\\
%\\
%\myindex{Wine}
%\myindex{ReactOS}
%The source code of the \TT{\_except\_handler3} function is closed.
%
%However, Sanos OS, which has a win32 compatibility layer, has the same
%functions reimplemented, which are somewhat equivalent to those in Windows
%\footnote{\url{http://go.yurichev.com/17058}}.
%Another reimplementation is present in 
%Wine\footnote{\href{http://go.yurichev.com/17059}{GitHub}}
%and ReactOS\footnote{\url{http://go.yurichev.com/17060}}.\\
%\\
%If the \IT{filter} pointer is NULL, the \IT{handler} 
%pointer is the pointer to the \IT{finally} code block.\\
%\\
%During execution, the \IT{previous try level} value in the stack changes, so
%\TT{\_except\_handler3} can get information about the current level of nestedness, 
%in order to know which \IT{scope table} entry to use.
%
%\myparagraph{SEH3: one try/except block example}
%
%\lstinputlisting[style=customc]{OS/SEH/2/2.c}
%
%\lstinputlisting[caption=MSVC 2003,style=customasmx86]{OS/SEH/2/2_SEH3.asm}
%
%Here we see how the SEH frame is constructed in the stack.
%The \IT{scope table} is located in the \TT{CONST} segment---indeed, these fields are not to be changed.
%An interesting thing is how the \IT{previous try level} variable has changed.
%The initial value is \TT{0xFFFFFFFF} ($-1$).
%The moment when the body of the \TT{try} 
%statement is opened is marked with an instruction that writes 0 to the variable.
%The moment when the body of the \TT{try} statement is closed, $-1$ 
%is written back to it.
%We also see the addresses of filter and handler code.
%
%Thus we can easily see the structure of the \IT{try/except} constructs in the function.\\
%\\
%Since the SEH setup code in the function prologue may be shared between many functions,
%sometimes the compiler inserts a call to the \TT{SEH\_prolog()} function in the prologue, which does just that.
%
%The SEH cleanup code is in the \TT{SEH\_epilog()} function.\\
%\\
%Let's try to run this example in \tracer{}:
%\myindex{tracer}
%
%\begin{lstlisting}
%tracer.exe -l:2.exe --dump-seh
%\end{lstlisting}
%
%\begin{lstlisting}[caption=tracer.exe output]
%EXCEPTION_ACCESS_VIOLATION at 2.exe!main+0x44 (0x401054) ExceptionInformation[0]=1
%EAX=0x00000000 EBX=0x7efde000 ECX=0x0040cbc8 EDX=0x0008e3c8
%ESI=0x00001db1 EDI=0x00000000 EBP=0x0018feac ESP=0x0018fe80
%EIP=0x00401054
%FLAGS=AF IF RF
%* SEH frame at 0x18fe9c prev=0x18ff78 handler=0x401204 (2.exe!_except_handler3)
%SEH3 frame. previous trylevel=0
%scopetable entry[0]. previous try level=-1, filter=0x401070 (2.exe!main+0x60) handler=0x401088 (2.exe!main+0x78)
%* SEH frame at 0x18ff78 prev=0x18ffc4 handler=0x401204 (2.exe!_except_handler3)
%SEH3 frame. previous trylevel=0
%scopetable entry[0]. previous try level=-1, filter=0x401531 (2.exe!mainCRTStartup+0x18d) handler=0x401545 (2.exe!mainCRTStartup+0x1a1)
%* SEH frame at 0x18ffc4 prev=0x18ffe4 handler=0x771f71f5 (ntdll.dll!__except_handler4)
%SEH4 frame. previous trylevel=0
%SEH4 header:	GSCookieOffset=0xfffffffe GSCookieXOROffset=0x0
%		EHCookieOffset=0xffffffcc EHCookieXOROffset=0x0
%scopetable entry[0]. previous try level=-2, filter=0x771f74d0 (ntdll.dll!___safe_se_handler_table+0x20) handler=0x771f90eb (ntdll.dll!_TppTerminateProcess@4+0x43)
%* SEH frame at 0x18ffe4 prev=0xffffffff handler=0x77247428 (ntdll.dll!_FinalExceptionHandler@16)
%\end{lstlisting}
%
%We see that the SEH chain consists of 4 handlers.\\
%\\
%\myindex{CRT}
%The first two are located in our example. Two?
%But we made only one?
%Yes, another one has been set up in the \ac{CRT} function \TT{\_mainCRTStartup()}, and as it seems that it handles at least \ac{FPU} exceptions.
%Its source code can found in the MSVC installation: \TT{crt/src/winxfltr.c}.\\
%\\
%The third is the SEH4 one in ntdll.dll, 
%and the fourth handler is not MSVC-related and is located in ntdll.dll, and has a self-describing function name.\\
%\\
%As you can see, there are 3 types of handlers in one chain:
%
%one is not related to MSVC at all (the last one) and two MSVC-related: SEH3 and SEH4.
%
%\myparagraph{SEH3: two try/except blocks example}
%
%\lstinputlisting[style=customc]{OS/SEH/2/3.c}
%
%Now there are two \TT{try} blocks.
%So the \IT{scope table} now has two entries, one for each block.
%\IT{Previous try level} changes as execution flow enters or exits the \TT{try} block.
%
%\lstinputlisting[caption=MSVC 2003,style=customasmx86]{OS/SEH/2/3_SEH3.asm}
%
%If we set a breakpoint on the \printf{} function, which is called from the handler, 
%we can also see how yet another SEH handler is added.
%
%Perhaps it's another machinery inside the SEH handling process.
%Here we also see our \IT{scope table} consisting of 2 entries.
%
%\begin{lstlisting}
%tracer.exe -l:3.exe bpx=3.exe!printf --dump-seh
%\end{lstlisting}
%
%\begin{lstlisting}[caption=tracer.exe output]
%(0) 3.exe!printf
%EAX=0x0000001b EBX=0x00000000 ECX=0x0040cc58 EDX=0x0008e3c8
%ESI=0x00000000 EDI=0x00000000 EBP=0x0018f840 ESP=0x0018f838
%EIP=0x004011b6
%FLAGS=PF ZF IF
%* SEH frame at 0x18f88c prev=0x18fe9c handler=0x771db4ad (ntdll.dll!ExecuteHandler2@20+0x3a)
%* SEH frame at 0x18fe9c prev=0x18ff78 handler=0x4012e0 (3.exe!_except_handler3)
%SEH3 frame. previous trylevel=1
%scopetable entry[0]. previous try level=-1, filter=0x401120 (3.exe!main+0xb0) handler=0x40113b (3.exe!main+0xcb)
%scopetable entry[1]. previous try level=0, filter=0x4010e8 (3.exe!main+0x78) handler=0x401100 (3.exe!main+0x90)
%* SEH frame at 0x18ff78 prev=0x18ffc4 handler=0x4012e0 (3.exe!_except_handler3)
%SEH3 frame. previous trylevel=0
%scopetable entry[0]. previous try level=-1, filter=0x40160d (3.exe!mainCRTStartup+0x18d) handler=0x401621 (3.exe!mainCRTStartup+0x1a1)
%* SEH frame at 0x18ffc4 prev=0x18ffe4 handler=0x771f71f5 (ntdll.dll!__except_handler4)
%SEH4 frame. previous trylevel=0
%SEH4 header:	GSCookieOffset=0xfffffffe GSCookieXOROffset=0x0
%		EHCookieOffset=0xffffffcc EHCookieXOROffset=0x0
%scopetable entry[0]. previous try level=-2, filter=0x771f74d0 (ntdll.dll!___safe_se_handler_table+0x20) handler=0x771f90eb (ntdll.dll!_TppTerminateProcess@4+0x43)
%* SEH frame at 0x18ffe4 prev=0xffffffff handler=0x77247428 (ntdll.dll!_FinalExceptionHandler@16)
%\end{lstlisting}
%
%\myparagraph{SEH4}
%
%\myindex{\BufferOverflow}
%\myindex{Security cookie}
%During a buffer overflow (\myref{subsec:bufferoverflow}) attack, the address of the \IT{scope table} 
%can be rewritten, so starting from MSVC 2005, SEH3 was upgraded to SEH4 in order to have buffer overflow protection.
%The pointer to the \IT{scope table} is now \glslink{xoring}{xored} with a \gls{security cookie}.
%The \IT{scope table} was extended to have a header consisting of two pointers to \IT{security cookies}.
%
%Each element has an offset inside the stack of another value: 
%the address of the \gls{stack frame} (\EBP) \glslink{xoring}{xored} with the \TT{security\_cookie} , placed in the stack.
%
%This value will be read during exception handling and checked for correctness.
%The \IT{security cookie} in the stack is random each time, so hopefully a remote attacker can't predict it. \\
%\\
%The initial \IT{previous try level} is $-2$ in SEH4 instead of $-1$.
%
%\def\SEHfour{1}
%\begin{center}
\begin{tikzpicture}[thick,scale=0.85, every node/.style={scale=0.85}]
	\tikzstyle{every path}=[thick]
	\tikzstyle{undefined}=[draw,rectangle,minimum height=1cm, minimum width=3.5cm, text width=3.5cm]
	\tikzstyle{undefinedw}=[draw,rectangle,minimum height=1cm, minimum width=4.5cm, text width=4.5cm]
	\tikzstyle{node}=[draw,rectangle,minimum height=1cm, minimum width=3.5cm, text width=3.5cm, fill=gray!20]
	\tikzstyle{nodew}=[draw,rectangle,minimum height=1cm, minimum width=4.5cm, text width=4.5cm, fill=gray!20]
	\tikzstyle{text_as_rect}=[rectangle,minimum height=1cm, minimum width=3.5cm, text width=3.5cm]
	\tikzstyle{point}=[inner sep=0pt]

	\node[node] (fs) [minimum width=1.5cm, text width=1.5cm] {FS:0};

\begin{scope}[start chain=1 going below, node distance=0cm]
	\node[on chain=1,node] (tib1) [right=1.5cm of fs] {+0: \_\_except\_list};
	\node[on chain=1,undefined] {+4: \dots};
	\node[on chain=1,undefined] {+8: \dots};
\end{scope}
	\node (tib_text) [above of=tib1] {TIB};
	
	\draw [->] (fs.east) -- (tib1.west);

\begin{scope}[start chain=3 going below, node distance=0cm]
	\node[on chain=3,undefined] (u1) [text centered, right=2.5cm of tib1] {\dots};
	%\node[on chain=3,node] (n1prev) {Prev=0xFFFFFFFF (1)};
	\node[on chain=3,node] (n1prev) {Prev=0xFFFFFFFF};
	\node[on chain=3,node] (n1handler) {Handle};
	\node[on chain=3,undefined] [text centered] {\dots};
	\node[on chain=3,node] (n2prev) {Prev};
	\node[on chain=3,node] (n2handler) {Handle};
	\node[on chain=3,undefined] [text centered] {\dots};
	\node[on chain=3,node] (n3prev) {Prev};
	\node[on chain=3,node] (n3handler) {Handle};
	\ifx\SEHfour\undefined	
	\node[on chain=3,node] (n3scopetable) {\ScopeTable};
	\else
	\node[on chain=3,node] (n3scopetable) {\ScopeTable$\oplus$security\_cookie};
	\fi
	\node[on chain=3,node] {\RU{предыдущий уровень try}\EN{previous try level}};
	\node[on chain=3,node] (n3ebp) {EBP};
	\ifx\SEHfour\undefined
	\else
	\node[on chain=3,undefined] [text centered] {\dots};
	\node[on chain=3,node] (AnotherCookie) {EBP$\oplus$security\_cookie};
	\node[on chain=3,undefined] [text centered] {\dots};
	\fi
\end{scope}
	
	\node [node] (n1handler_text) [right=1cm of n1handler] {\HandlerFunction};
	\draw [->] (n1handler.east) -- (n1handler_text.west);
	\node [node] (n2handler_text) [right=1cm of n2handler] {\HandlerFunction};
	\draw [->] (n2handler.east) -- (n2handler_text.west);

	\ifx\SEHfour\undefined	
	\node [node] (n3handler_text) [right=1cm of n3handler] {\_except\_handler3};
	\else
	\node [node] (n3handler_text) [right=1cm of n3handler] {\_except\_handler4};
	\fi

	\draw [->] (n3handler.east) -- (n3handler_text.west);
	\node[undefined] (u4) [text centered, below of=n3ebp] {\dots};
	\node (stack_text) [above of=u1] {Stack};
	
	\node[point] (n2block_pt2) [above left=0cm and 0.8cm of n2prev] {};
	\draw [->] (n3prev.west) .. controls +(left:0.8cm) and (n2block_pt2) .. (n2prev.north west);

	\node[point] (n1block_pt2) [above left=0cm and 0.8cm of n1prev] {};
	\draw [->] (n2prev.west) .. controls +(left:0.8cm) and (n1block_pt2) .. (n1prev.north west);
	
	\node[point] (n3block_pt2) [above left=0cm and 1.25cm of n3prev] {};
	\draw [->] (tib1.east) .. controls +(right:1.25cm) and (n3block_pt2) .. (n3prev.north west);

\begin{scope}[start chain=2 going below, node distance=0cm]
	% SEH3
	\ifx\SEHfour\undefined	
	\node[on chain=2,nodew] (scope_tbl) [left=3.5cm of n2handler] {0xFFFFFFFF (-1)};
	\else
	% SEH4 header
	\node[on chain=2,nodew] (scope_tbl) [left=3.5 cm of n2handler] {\RU{смещение GS Cookie}\EN{GS Cookie Offset}};
	\node[on chain=2,nodew] {\RU{смещение GS Cookie XOR}\EN{GS Cookie XOR Offset}};
	\node[on chain=2,nodew] (EHCookieOffset) {\RU{смещение EH Cookie}\EN{EH Cookie Offset}};
	\node[on chain=2,nodew] {\RU{смещение EH Cookie XOR}\EN{EH Cookie XOR Offset}};
	\node[on chain=2,minimum height=0.3cm] {};
	\node[on chain=2,nodew] {0xFFFFFFFF (-1)};
	\fi

	\node[on chain=2,nodew] (scope_tbl_filter1) {\FilterFunction};
	\node[on chain=2,nodew] {\HandlerFinallyFunction};
	\node[on chain=2,minimum height=0.3cm] {};
	\node[on chain=2,nodew] {0};
	\node[on chain=2,nodew] (scope_tbl_filter2) {\FilterFunction};
	\node[on chain=2,nodew] {\HandlerFinallyFunction};
	\node[on chain=2,minimum height=0.3cm] {};
	\node[on chain=2,nodew] {1};
	\node[on chain=2,nodew] (scope_tbl_filter3) {\FilterFunction};
	\node[on chain=2,nodew] {\HandlerFinallyFunction};
	\node[on chain=2,minimum height=0.3cm] {};
	\node[on chain=2,undefinedw] [text centered] {\dots \MoreEntries \dots};

\end{scope}
	
	\node[text_as_rect,left=0.5cm of scope_tbl_filter1] {\RU{информация для первого блока try/except}\EN{information about first try/except block}};
	\node[text_as_rect,left=0.5cm of scope_tbl_filter2] {\RU{информация для второго блока try/except}\EN{information about second try/except block}};
	\node[text_as_rect,left=0.5cm of scope_tbl_filter3] {\RU{информация для третьего блока try/except}\EN{information about third try/except block}};

	\node (scope_text) [above of=scope_tbl] {\ScopeTable};
	\node[point] (scope_table_pt2) [right=1.75cm of scope_tbl] {};
	\draw [->] (n3scopetable.west) .. controls +(left:1.75cm) and (scope_table_pt2) .. (scope_tbl.north east);

	\ifx\SEHfour\undefined
	\else
	\node[point] (AnotherCookie_pt2) [left=1.75cm of AnotherCookie] {};
	\draw [->] (EHCookieOffset.east) .. controls +(right:1.75cm) and (AnotherCookie_pt2) .. (AnotherCookie.west);
	\fi

\end{tikzpicture}
\end{center}

%
%Here are both examples compiled in MSVC 2012 with SEH4:
%
%\lstinputlisting[caption=MSVC 2012: one try block example,style=customasmx86]{OS/SEH/2/2_SEH4.asm}
%
%\lstinputlisting[caption=MSVC 2012: two try blocks example,style=customasmx86]{OS/SEH/2/3_SEH4.asm}
%
%Here is the meaning of the \IT{cookies}: \TT{Cookie Offset} 
%is the difference between the address of the saved EBP value in the stack
%and the $EBP \oplus security\_cookie$ value in the stack.
%\TT{Cookie XOR Offset} is an additional difference between the 
%$EBP \oplus security\_cookie$ value and what is
%stored in the stack.
%
%If this equation is not true, the process is to halt due to stack corruption:
%
%\begin{center}
%$security\_cookie \oplus (CookieXOROffset + address\_of\_saved\_EBP) == stack[address\_of\_saved\_EBP + CookieOffset]$
%\end{center}
%
%If \TT{Cookie Offset} is $-2$, this implies that it is not present.
%
%\myindex{tracer}
%\IT{Cookies} checking is also implemented in my \tracer{},
%see \href{http://go.yurichev.com/17061}{GitHub} for details.\\
%\\
%It is still possible to fall back to SEH3 in the compilers after 
%(and including) MSVC 2005 by setting the \TT{/GS-} option,
%however, the \ac{CRT} code use SEH4 anyway.
