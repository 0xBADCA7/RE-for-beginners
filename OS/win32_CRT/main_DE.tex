\subsection{CRT (win32)}
\label{sec:CRT}
\myindex{CRT}

Startet ein Programm genau bei der \main{}-Funktion?
Nein, tut es nicht!

Würden wir jede ausführbare Datei in \IDA oder HIEW öffnen können, würde wir sehen,
dass \ac{OEP} auf einen anderen Code-Block zeigt.

Dieser Code erledigt einige Vorbereitungen bevor die Ausführungskontrolle an unseren
Code übergeben wird.
Dies ist der sogenannte Startup- oder CRT-Code (C RunTime).

Die \main{}-Funktion nimmt ein Array mit den Argumenten entgegen, die auf der Kommandozeile
übergeben wurde, sowie eins mit den Umgebungsvariablen.
Genaugenommen wird eine normale Zeichenkette an das Programm übergeben und der CRT-Code
unterteilt diesen anhand der Leerzeichen in seine Bestandteile.
Der CRT-Code bereitet auch das Array \TT{envp} vor, welches die Umgebungsvariablen enthält.

Für \ac{GUI}-Programme unter Win32 wird \TT{WinMain} anstatt \main{} genutzt, welches eigene
Argumente hat:

\begin{lstlisting}
int CALLBACK WinMain(
  _In_  HINSTANCE hInstance,
  _In_  HINSTANCE hPrevInstance,
  _In_  LPSTR lpCmdLine,
  _In_  int nCmdShow
);
\end{lstlisting}

Der CRT-Code bereitet diese ebenfalls vor.

Die Zahl die von \main{} zurückgegeben wird ist der Exit-Code.

Diese kann in der CRT für die Funktion \TT{ExitProcess()} werden, die diesen Exit-Code als
Argument entgegennimmt.

In der Regel hat jeder Compiler seinen eigenen CRT-Code.

Nachfolgend eine typischer CRT-Code für MSVC 2008.

%TODO In german translation the customasm style can not be compiled.
%\lstinputlisting[numbers=left,style=customasm]{OS/win32_CRT/crt_msvc_2008.asm}
\lstinputlisting[numbers=left]{OS/win32_CRT/crt_msvc_2008.asm}

Wir sehen hier die Aufrufe zu \TT{GetCommandLineA()} (Zeile 62), anschließend zu \TT{setargv()}
(Zeile 66) und \TT{setenvp()} (Zeile 74), was offensichtlich die globalen Variablen \TT{argc},
\TT{argv} und  \TT{envp} initialisiert.

Zum Schluss wird \main{} mit diesen Argumenten aufgerufen (Zeile 97).

Es sind ebenfalls Aufrufe zu Funktionen mit selbsterklärenden Namen zu finden, wie \TT{heap\_init()}
(Zeile 35) und \TT{ioinit()} (Zeile 54).

Der \glslink{heap}{heap} wird jedoch vom \ac{CRT} initialisiert.
Wenn man versucht \TT{malloc()} in einem Programm ohne CRT zu nutzen, wird dieses mit dem folgenden
Fehler abstürzen:

\begin{lstlisting}
runtime error R6030
- CRT not initialized
\end{lstlisting}

Die Initialisierung von globalen Objekten in \Cpp passiert ebenfalls in der \ac{CRT} vor der
Ausführung von \main{}:
\myref{sec:std_string_as_global_variable}.

%TODO SLN32 not knwon in german translation
%Der Wert den \main{} zurückgibt wird an \TT{cexit()} übergeben oder in \TT{\SLN32}, welches
%\TT{doexit()} aufruft.

Ist es möglich die \ac{CRT} loszuwerden?
Ja, wenn man genau weiß, was man tut.

Der Linker von \ac{MSVC} hat die \TT{/ENTRY}-Option um den Einsprungpunkt festzulegen.

\begin{lstlisting}
#include <windows.h>

int main()
{
	MessageBox (NULL, "hello, world", "caption", MB_OK);
};
\end{lstlisting}

Kompilieren wir dies in MSVC 2008.

\begin{lstlisting}
cl no_crt.c user32.lib /link /entry:main
\end{lstlisting}

Wir bekommen eine lauffähige .exe-Datei mit der Größe 2560 Byte mit einem PE-Header,
Anweisungen die \TT{MessageBox} aufrufen, zwei Zeichenketten im Datensegment, die aus
\TT{user32.dll} importierte Funktion \TT{MessageBox} und sonst nichts.

Dies funktioniert, jedoch kann nicht \TT{WinMain} mit den 4 Argumenten anstatt \main{}
genutzt werden.

Um genau zu sein, wäre dies zwar möglich, allerdings wurden die Argumente nicht vorbereitet
um sie zu nutzen.

Es ist übrigens auch möglich die .exe-Datei kleiner machen indem die \ac{PE}-Sektion
an weniger als den standardmäßigen 4096 Byte auszurichten.

\begin{lstlisting}
cl no_crt.c user32.lib /link /entry:main /align:16
\end{lstlisting}

Der Linker gibt aus:

\begin{lstlisting}
LINK : warning LNK4108: /ALIGN specified without /DRIVER; image may not run
\end{lstlisting}

Die Ausgabe ist eine .exe-Datei mit 720 Byte.
Sie kann unter Windows 7 x86, jedoch nicht x64 ausgeführt werden (beim Versuch wird
eine Fehlermeldung erscheinen).

Mit mehr Aufwand ist es möglich die Datei noch weiter zu verkleinern, aber wie zu
sehen ist, bekommt man schnell Kompatibilitätsprobleme.

