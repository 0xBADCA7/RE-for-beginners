\chapter{Books/blogs worth reading}

\section{Books and other materials}

\subsection{Reverse Engineering}

\input{RE_books}

Also, Kris Kaspersky's books.

\subsection{Windows}

\begin{itemize}
\item \Russinovich
\end{itemize}

\EN{Blogs}\ES{Blogs}\RU{Блоги}\FR{Blogs}\DE{Blogs}:

\begin{itemize}
\item \href{http://go.yurichev.com/17025}{Microsoft: Raymond Chen}
\item \href{http://go.yurichev.com/17026}{nynaeve.net}
\end{itemize}



\subsection{\CCpp}

\label{CCppBooks}

\begin{itemize}

\item \KRBook

\item \CNineNineStd\footnote{\AlsoAvailableAs \url{http://go.yurichev.com/17274}}

\item \TCPPPL

\item \CppOneOneStd\footnote{\AlsoAvailableAs \url{http://www.open-std.org/jtc1/sc22/wg21/docs/papers/2013/n3690.pdf}.}

\item \AgnerFogCPP\footnote{\AlsoAvailableAs \url{http://agner.org/optimize/optimizing_cpp.pdf}.}

\item \ParashiftCPPFAQ\footnote{\AlsoAvailableAs \url{http://go.yurichev.com/17291}}

\item \CNotes\footnote{\AlsoAvailableAs \url{http://yurichev.com/C-book.html}}

\end{itemize}



\subsection{x86 / x86-64}

\label{x86_manuals}
\begin{itemize}
\item Intel manuals\footnote{\AlsoAvailableAs \url{http://www.intel.com/content/www/us/en/processors/architectures-software-developer-manuals.html}}

\item AMD manuals\footnote{\AlsoAvailableAs \url{http://developer.amd.com/resources/developer-guides-manuals/}}

\item \AgnerFog{}\footnote{\AlsoAvailableAs \url{http://agner.org/optimize/microarchitecture.pdf}}

\item \AgnerFogCC{}\footnote{\AlsoAvailableAs \url{http://www.agner.org/optimize/calling_conventions.pdf}}

\item \IntelOptimization

\item \AMDOptimization
\end{itemize}

Somewhat outdated, but still interesting to read:

\MAbrash\footnote{\AlsoAvailableAs \url{https://github.com/jagregory/abrash-black-book}}
(he is known for his work on low-level optimization for such projects as Windows NT 3.1 and id Quake).

\subsection{ARM}

\begin{itemize}
\item ARM manuals\footnote{\AlsoAvailableAs \url{http://infocenter.arm.com/help/index.jsp?topic=/com.arm.doc.subset.architecture.reference/index.html}}

\item \ARMSevenRef

\item \ARMSixFourRefURL

\item \ARMCookBook\footnote{\AlsoAvailableAs \url{http://go.yurichev.com/17273}}
\end{itemize}

\subsection{Assembly language}

Richard Blum --- Professional Assembly Language.

\subsection{Java}

\JavaBook.

\subsection{UNIX}

\TAOUP

\subsection{Programming in general}

\begin{itemize}

\item \RobPikePractice

\item \HenryWarren.
Some people say tricks and hacks from the book are not relevant today because they were good only for \ac{RISC} \ac{CPU}s,
where banrching instructions are expensive.
Nevertheless, these can help immensely to understand boolean algebra and what all the mathematics near it.

\item (For hard-core geeks with computer science and mathematical background) Donald E. Knuth, \IT{The Art of Computer Programming}.

\end{itemize}

% subsection:
\subsection{\EN{Cryptography}\ES{Criptograf\'ia}\ITA{Crittografia}\RU{Криптография}\FR{Cryptographie}\DE{Kryptografie}}
\label{crypto_books}

\begin{itemize}
\item \Schneier{}

\item (Free) lvh, \IT{Crypto 101}\footnote{\AlsoAvailableAs \url{https://www.crypto101.io/}}

\item (Free) Dan Boneh, Victor Shoup, \IT{A Graduate Course in Applied Cryptography}\footnote{\AlsoAvailableAs \url{https://crypto.stanford.edu/~dabo/cryptobook/}}.
\end{itemize}



