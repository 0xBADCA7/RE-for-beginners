\section{Using magic numbers while tracing}

Oft ist unser Hauptziel zu verstehen wie ein Programm einen Wert behandelt der entweder über eine Datei oder über das Netzwerk erhalten wurde.
Das manuelle tracen eines Wertes ist meistens ein ziemlich arbeits-intensiver Task. Eine der einfachsten Techniken um Werte zu Tracen (auch wenn nicht 100\% verlässlich)
ist eigene \IT{magic number}'s zu benutzen. 

% Often, our main goal is to understand how the program uses a value that has been either read from file or received via network. 
% The manual tracing of a value is often a very labor-intensive task. One of the simplest techniques for this (although not 100\% reliable) 
% is to use your own \IT{magic number}.

Das ähnelt ein wenig dem Vorgang beim Röntgen auf gewisser weise: ein radioaktives Kontrastmittel wird dem Patienten injeziert,
welches dann benutzt wird um die Gefässe des Patienten besser zu erkennen duch die Rönthgenstahlung. Wie das blut bei 
gesunden Menschen in den Nieren gereinigt wird wenn das Kontrastmittel im Blut ist, man kann dann sehr einfach auf dem
Bild der Tomografie erkennen ob sich Nierensteine oder Tumore in den Nierenbefinden. 

% This resembles X-ray computed tomography is some sense: a radiocontrast agent is injected into the patient's blood,
% which is then used to improve the visibility of the patient's internal structure in to the X-rays.
% It is well known how the blood of healthy humans
% percolates in the kidneys and if the agent is in the blood, it can be easily seen on tomography, how blood is percolating,
% and are there any stones or tumors.

Wir können einfach eine 32-Bit Zahl nehmen z.B \TT{0xbadf00d}, oder ein Geburtsdatum wie \TT{0x11101979}
und diese 4-Byte Zahl wird an einem bestimmten Punkt in eine Datei geschrieben welche von dem Programm 
das wir untersuchen genutzt wird. 

% We can take a 32-bit number like \TT{0x0badf00d}, or someone's birth date like \TT{0x11101979}
% and write this 4-byte number to some point in a file used by the program we investigate.

\myindex{\GrepUsage}
\myindex{tracer}

Dann während das programm getraced wird mit \tracer im \IT{code coverage} modus, mit der Hilfe von \IT{grep}
oder durch einfaches durchsuchen der Textdatei (der trace Ergebnisse), können wir ganz einfach sehen wo der 
Wert benutzt wurde und wie er benutzt wurde. 

% Then, while tracing this program with \tracer in \IT{code coverage} mode, with the help of \IT{grep}
% or just by searching in the text file (of tracing results), we can easily see where the value has been used and how.


Beispiel 
der \IT{grepable} \tracer Ergebnissen im \IT{cc} mode:

% Example 
% of \IT{grepable} \tracer results in \IT{cc} mode:

\begin{lstlisting}[style=customasmx86]
0x150bf66 (_kziaia+0x14), e=       1 [MOV EBX, [EBP+8]] [EBP+8]=0xf59c934 
0x150bf69 (_kziaia+0x17), e=       1 [MOV EDX, [69AEB08h]] [69AEB08h]=0 
0x150bf6f (_kziaia+0x1d), e=       1 [FS: MOV EAX, [2Ch]] 
0x150bf75 (_kziaia+0x23), e=       1 [MOV ECX, [EAX+EDX*4]] [EAX+EDX*4]=0xf1ac360 
0x150bf78 (_kziaia+0x26), e=       1 [MOV [EBP-4], ECX] ECX=0xf1ac360 
\end{lstlisting}
% TODO: good example!

Das gleiche verfahren kann man auch auf Netzwerkpakete anwenden.
Für die \IT{magic number} ist es wichtig das diese einzigartig ist und nicht im Programm code vorkommt.

% This can be used for network packets as well.
% It is important for the \IT{magic number} to be unique and not to be present in the program's code.

\newcommand{\DOSBOXURL}{\href{http://go.yurichev.com/17222}{blog.yurichev.com}}

\myindex{DosBox}
\myindex{MS-DOS}
Neben dem \tracer Befehl, gibt es noch den DosBox (MS-DOS emulator) im heavydebug Modus,
welcher in der Lage ist alle Informationen über alle Register zustände für jede ausgeführte Instruktion des Programmes in
eine einfache Textdatei\footnote{See also my blog post about this DosBox feature: \DOSBOXURL{} zu schreiben, so kann
diese Technik für DOS Programme nützlich sein. 

% Aside of 
% the \tracer, DosBox (MS-DOS emulator) in heavydebug mode
% is able to write information about all registers' states for each executed instruction of the program to a plain text file\footnote{See also my 
% blog post about this DosBox feature: \DOSBOXURL{}}, so this technique may be useful for DOS programs as well.

