\section{Вызовы assert()}
\myindex{\CStandardLibrary!assert()}
Может также помочь наличие \TT{assert()} в коде: обычно этот макрос оставляет название файла-исходника, 
номер строки, и условие.

Наиболее полезная информация содержится в assert-условии, по нему можно судить по именам переменных
или именам полей структур. Другая полезная информация --- это имена файлов, по их именам можно попытаться
предположить, что там за код. Также, по именам файлов можно опознать какую-либо очень известную опен-сорсную
библиотеку.

\lstinputlisting[caption=Пример информативных вызовов assert()]{digging_into_code/assert_examples.lst}

Полезно \q{гуглить} и условия и имена файлов, это может вывести вас к опен-сорсной бибилотеке.
Например, если \q{погуглить} \q{sp->lzw\_nbits <= BITS\_MAX}, 
это вполне предсказуемо выводит на опенсорсный код, что-то связанное с LZW-компрессией.

