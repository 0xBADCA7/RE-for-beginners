\section{Finding the right instructions}

If the program is utilizing FPU instructions and there are very few of them in the code,
one can try to check each one manually with a debugger.

\par For example, we may be interested how Microsoft Excel calculates the formulae entered by user.
For example, the division operation.

\myindex{\GrepUsage}
\myindex{x86!\Instructions!FDIV}

If we load excel.exe (from Office 2010) version 14.0.4756.1000 into \IDA, make a full listing
and to find every \FDIV instruction (except the ones which use constants as a second 
operand---obviously, they do not suit us):

\begin{lstlisting}
cat EXCEL.lst | grep fdiv | grep -v dbl_ > EXCEL.fdiv
\end{lstlisting}

\dots then we see that there are 144 of them.

\par We can enter a string like \TT{=(1/3)} in Excel and check each instruction.

\myindex{tracer}

\par By checking each instruction in a debugger or \tracer
(one may check 4 instruction at a time),
we get lucky and the sought-for instruction is just the 14th:

\begin{lstlisting}[style=customasmx86]
.text:3011E919 DC 33          fdiv    qword ptr [ebx]
\end{lstlisting}

\begin{lstlisting}
PID=13944|TID=28744|(0) 0x2f64e919 (Excel.exe!BASE+0x11e919)
EAX=0x02088006 EBX=0x02088018 ECX=0x00000001 EDX=0x00000001
ESI=0x02088000 EDI=0x00544804 EBP=0x0274FA3C ESP=0x0274F9F8
EIP=0x2F64E919
FLAGS=PF IF
FPU ControlWord=IC RC=NEAR PC=64bits PM UM OM ZM DM IM 
FPU StatusWord=
FPU ST(0): 1.000000
\end{lstlisting}

\ST{0} holds the first argument (1) and second one is in \TT{[EBX]}.\\
\\
\myindex{x86!\Instructions!FDIV}

The instruction after \FDIV (\TT{FSTP}) writes the result in memory:\\

\begin{lstlisting}[style=customasmx86]
.text:3011E91B DD 1E          fstp    qword ptr [esi]
\end{lstlisting}

If we set a breakpoint on it, we can see the result:

\begin{lstlisting}
PID=32852|TID=36488|(0) 0x2f40e91b (Excel.exe!BASE+0x11e91b)
EAX=0x00598006 EBX=0x00598018 ECX=0x00000001 EDX=0x00000001
ESI=0x00598000 EDI=0x00294804 EBP=0x026CF93C ESP=0x026CF8F8
EIP=0x2F40E91B
FLAGS=PF IF
FPU ControlWord=IC RC=NEAR PC=64bits PM UM OM ZM DM IM 
FPU StatusWord=C1 P 
FPU ST(0): 0.333333
\end{lstlisting}

Also as a practical joke, we can modify it on the fly:

\begin{lstlisting}
tracer -l:excel.exe bpx=excel.exe!BASE+0x11E91B,set(st0,666)
\end{lstlisting}

\begin{lstlisting}
PID=36540|TID=24056|(0) 0x2f40e91b (Excel.exe!BASE+0x11e91b)
EAX=0x00680006 EBX=0x00680018 ECX=0x00000001 EDX=0x00000001
ESI=0x00680000 EDI=0x00395404 EBP=0x0290FD9C ESP=0x0290FD58
EIP=0x2F40E91B
FLAGS=PF IF
FPU ControlWord=IC RC=NEAR PC=64bits PM UM OM ZM DM IM 
FPU StatusWord=C1 P 
FPU ST(0): 0.333333
Set ST0 register to 666.000000
\end{lstlisting}

Excel shows 666 in the cell, finally convincing us that we have found the right point.

\begin{figure}[H]
\centering
\includegraphics[width=0.6\textwidth]{digging_into_code/Excel_prank.png}
\caption{The practical joke worked}
\end{figure}

If we try the same Excel version, but in x64,
we will find only 12 \FDIV instructions there,
and the one we looking for is the third one.

\begin{lstlisting}
tracer.exe -l:excel.exe bpx=excel.exe!BASE+0x1B7FCC,set(st0,666)
\end{lstlisting}

\myindex{x86!\Instructions!DIVSD}

It seems that a lot of division operations of \Tfloat and \Tdouble types, were replaced by the compiler with SSE instructions
like \TT{DIVSD} (\TT{DIVSD} is present 268 times in total).
