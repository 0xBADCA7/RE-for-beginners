\subsubsection{Base64}
\myindex{Base64}

Die Base64 codierung ist sehr weit verbreitet für fälle in denen man Binärdaten als Textstring übertragen will.
% The base64 encoding is highly popular for the cases when you have to transfer binary data as a text string.

Im Grunde, codiert dieser Algorithmus 3 Binär Bytes in 4 druckbare Zeichen: 
Alle 26 Latin Zeichen (beides klein und groß Buchstaben), Ziffern, plus Zeichen (\q{+}) und slash Zeichen (\q{/}),
64 Zeichen insgesammt. 

% In essence, this algorithm encodes 3 binary bytes into 4 printable characters:
% all 26 Latin letters (both lower and upper case), digits, plus sign (\q{+}) and slash sign (\q{/}),
% 64 characters in total.

Ein charakteristisches Feature von Base64 Strings ist das sie of (aber nicht immer) mit 1 oder 2  \gls{padding}
gleichheitszeichen (\q{=}) Enden, zum Beispiel: 

% One distinctive feature of base64 strings is that they often (but not always) ends with 1 or 2 \gls{padding}
% equality symbol(s) (\q{=}), for example:

\begin{lstlisting}
AVjbbVSVfcUMu1xvjaMgjNtueRwBbxnyJw8dpGnLW8ZW8aKG3v4Y0icuQT+qEJAp9lAOuWs=
\end{lstlisting}

\begin{lstlisting}
WVjbbVSVfcUMu1xvjaMgjNtueRwBbxnyJw8dpGnLW8ZW8aKG3v4Y0icuQT+qEJAp9lAOuQ==
\end{lstlisting}

Das Gleichheitszeichen Symbol (q{=}) wird man niemals in der Mitte eines Base64-codierten
Strings sehen.
% The equality sign (\q{=}) is never encounter in the middle of base64-encoded strings.

Jetzt ein beispiel wie man per Hand Base64 codieren kann.
Lasst uns 0x00, 0x11 , 0x22 und 0x33 in Hexadezimalzahlen in einen Base64
String umwandeln: 

% Now example of manual encoding.
% Let's encode 0x00, 0x11, 0x22, 0x33 hexadecimal bytes into base64 string:

\lstinputlisting{digging_into_code/strings/base64_ex.sh}

Lasst uns alle 4 Bytes in Binär form bringen und dann neu gruppieren in 6-Bit Gruppen:
% Let's put all 4 bytes in binary form, then regroup them into 6-bit groups:

\begin{lstlisting}
|  00  ||  11  ||  22  ||  33  ||      ||      |
00000000000100010010001000110011????????????????
| A  || B  || E  || i  || M  || w  || =  || =  |
\end{lstlisting}

Die ersten drei Bytes (0x00, 0x11, 0x22) können in 4 Base64 Zeichen umgewandelt werden (``ABEi''),
aber nicht das letzte Byte (0x33), also wird das Byte codiert indem man Zweibuchstaben 
benutzt (``Mw'') und das \gls{padding} Symbol (``='') wird zweimal hinzugefügt um die letzte
Gruppe auf 4 Zeichen zu erweitern. Das bedeutet das die länge aller correcten base64 Strings
sich immer durch 4 Teilen lässt. 

% Three first bytes (0x00, 0x11, 0x22) can be encoded into 4 base64 characters (``ABEi''),
% but the last one (0x33) --- cannot be,
% so it's encoded using two characters (``Mw'') and \gls{padding} symbol (``='')
% is added twice to pad the last group to 4 characters.
% Hence, length of all correct base64 strings are always divisible by 4.

\myindex{XML}
\myindex{PGP}
Base64 wird oft benutzt wenn es darum geht Binärdaten in  XML Datein zu speichern.
``Armored'' (z.B, in Text Form) PGP keys und Signaturen werden codiert mit Base64.

% Base64 is often used when binary data needs to be stored in XML.
% ``Armored'' (i.e., in text form) PGP keys and signatures are encoded using base64.

Manche Leute versuchen auch base64 zu benutzen um Strings zu verschleiern. 
% Some people tries to use base64 to obfuscate strings:
\url{http://blog.sec-consult.com/2016/01/deliberately-hidden-backdoor-account-in.html}
\footnote{\url{http://archive.is/nDCas}}.

\myindex{base64scanner}
Es gibt Werkzeuge zum scannen von beliebigen Binärdatein nach Base64 Strings.
Ein solch ein Scanner ist base64scanner\footnote{\url{https://github.com/dennis714/base64scanner}}.
% There are utilities for scanning an arbitrary binary files for base64 strings.
% One such utility is base64scanner\footnote{\url{https://github.com/dennis714/base64scanner}}.

\myindex{UseNet}
\myindex{FidoNet}
\myindex{Uuencoding}
Ein weiteres codierungs System was sehr weit verbreitet und weit mehr genutzt wurde im UseNet und im 
FidoNet ist Uuencoding. Es hat eigentlich die gleichen Features, aber unterscheidet sich von Base64 
in so fern das der Dateiname auch im header gespeichert wird.

% Another encoding system which was much more popular in UseNet and FidoNet is Uuencoding.
% It offers mostly the same features, but is different from base64 in the sense that file name
% is also stored in header.


\myindex{Tor}
\myindex{base32}
By the Way: Es gibt auch einen nahen Verwandten zu Base64: Base32., ein Alphabet das ~10 Zeichen und ~26 Latin Zeichen hat. 
Eine verbreitete anwendung ist Onion Adressen zu codieren. 
\footnote{\url{https://trac.torproject.org/projects/tor/wiki/doc/HiddenServiceNames}},
like: \url{http://3g2upl4pq6kufc4m.onion/}.
\ac{URL} kann keine mixed-case Latin Zeichen beinhalten, deshalb haben Tor Entwickler sich für Base32 entschieden.

% By the way: there is also close sibling to base64: base32, alphabet of which has ~10 digits and ~26 Latin characters.
% One well-known usage of it is onion addresses
% \footnote{\url{https://trac.torproject.org/projects/tor/wiki/doc/HiddenServiceNames}},
% like: \url{http://3g2upl4pq6kufc4m.onion/}.
% \ac{URL} can't have mixed-case Latin characters, so apparently, this is why Tor developers used base32.

