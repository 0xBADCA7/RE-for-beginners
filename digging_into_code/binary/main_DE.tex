% TODO move section...

\subsection{ Muster in Bin\"ardatein finden}

Alle Beispiele hier wurden vorbereitet mit Windows mit aktiver Code Page 437
\footnote{\url{https://en.wikipedia.org/wiki/Code_page_437}} in der Konsole.
Bin\"ar Dateien sehen intern etwas anders aus wenn eine andere Code page gesetzt ist.

\clearpage
\subsubsection{Arrays}

Manchmal kann man klar ein Array von 16/32/64-Bit Werten mit bloßem Auge im hex Editor erkennen.

Hier ist ein Beispiel eines 16-Bit Wertes.
Wir sehen das das erste Byte ein paar aus 7 oder 8 ist und das zweite sieht
zuf\"allig aus:

\begin{figure}[H]
\centering
\myincludegraphics{digging_into_code/binary/16bit_array.png}
\caption{FAR: array von 16-Bit Werten}
\end{figure}

Ich habe eine Datei benutzt die ein 12 Kanal Signal digitalisiert mit 16-Bit nutzt \ac{ADC}.

\clearpage
\myindex{MIPS}
\par Und hier ist ein Beispiel von einem Typischen MIPS Code.

Wie wir uns vielleicht erinnern, jede MIPS ( also auch ARM in ARM Mode oder ARM64 ) Instruktion hat eine Gr\"oße von 32 Bits (oder 4 Bytes),
also ist solcher Code ein Array von 32-Bit Werten. 

Wenn man den Screenshot anschaut, sehen wir eine Art Muster.

Vertikale und rote Linien wurden zur besseren Lesbarkeit eingef\"ugt:

\begin{figure}[H]
\centering
\myincludegraphics{digging_into_code/binary/typical_MIPS_code.png}
\caption{Hiew: sehr typischer MIPS code}
\end{figure}

Ein weiteres Beispiel eines solchen Musters ist Buch:
\myref{Oracle_SYM_files_example}.

\clearpage
\subsubsection{Sparse Dateien} 

Diese d\"urftige Datei mit zerstreuten Daten inmitten einer fast leeren Datei.
Jedes Space Zeichen hier ist in der tat ein Zero Byte (das wie ein space aussieht). % <-- findet man sicher was besseres
Das ist eine Datei mit der ein FPGA Programmiert wird (Ein Altera Stratix GX Ger\"at).
Sicher k\"onnen Dateien wie diese einfach Komprimiert werden, aber diese Formate sind in 
der Wissenschaft und im Ingenieurs Wesen so wie in der Softwareentwicklung sehr verbreitet.
Wo es oft um effizienten Zugriff geht und weniger um die Komprimierung der Daten.

% This is sparse file with data scattered amidst almost empty file.
% Each space character here is in fact zero byte (which is looks like space).
% This is a file to program FPGA (Altera Stratix GX device).
% Of course, files like these can be compressed easily, but formats like this one are very popular in scientific and engineering software where efficient access is important while compactness is not.

\begin{figure}[H]
\centering
\myincludegraphics{digging_into_code/binary/sparse_FPGA.png}
\caption{FAR: Sparse file}
\end{figure}

\clearpage
\subsubsection{Komprimierte Dateien}

% FIXME \ref{} ->
Diese Datei ist einfach ein komprimiertes Archiv. 
Es hat eine relativ hohe Entropie und visuell betrachtet sieht es 
eher Chaotisch aus. So sehen komprimierte oder verschl\"usselte Dateien aus.

\begin{figure}[H]
\centering
\myincludegraphics{digging_into_code/binary/compressed.png}
\caption{FAR: Komprimierte Datei}
\end{figure}

\clearpage
\subsubsection{\ac{CDFS}}

\ac{OS} Installationen werden \"ublicherweise als ISO Datei bereit gestellt, die Kopien von CD/DVD Disks sind. 
Das Dateisystem das benutzt wird heißt \ac{cdfs}, hier sieht man wie Dateinamen mit zus\"atzlichen Daten vermischt sind.
Das k\"onnen Datei Gr\"oßen, Pointer auf andere Verzeichnisse, Datei Attribute und anderes sein. 
So sehen Dateisysteme typischerweise auch von innen aus.

\begin{figure}[H]
\centering
\myincludegraphics{digging_into_code/binary/cdfs.png}
\caption{FAR: ISO file: Ubuntu 15 Installation \ac{CD}}
\end{figure}

\clearpage
\subsubsection{32-bit x86 ausf\"uhrbarer Code} 

So sieht 32-Bit x86 ausf\"uhrbarer Code aus. 
Der Code hat nicht wirklich viel Entropie, weil manche Bytes \"ofters vorkommen als andere.

\begin{figure}[H]
\centering
\myincludegraphics{digging_into_code/binary/x86_32.png}
\caption{FAR: Executable 32-bit x86 code}
\end{figure}

% TODO: Read more about x86 statistics: \ref{}. % FIXME blog post about decryption...

\clearpage
\subsubsection{BMP graphics files}

% TODO: bitmap, bit, group of bits...

BMP Dateien sind nicht komprimiert, also ist jedes Byte ( oder Gruppen von Bytes ) beschrieben als
ein Pixel. Diese Bild habe ich irgendwo in meiner Windows 8.1 Installation gefunden: 

\begin{figure}[H]
\centering
\myincludegraphicsSmall{digging_into_code/binary/bmp.png}
\caption{Example picture}
\end{figure}

Man kann sehen das dieses Bild Pixel hat, die nicht wirklich gut komprimiert werden k\"onne (um das Zentrum herum),
aber es sind lange ein-Farben Linien am Anfang und am ende der Datei. Tats\"achlich Linien wie diese sehen wie Linien aus
wenn man sich die Datei anschaut:

\begin{figure}[H]
\centering
\myincludegraphics{digging_into_code/binary/bmp_FAR.png}
\caption{BMP file fragment}
\end{figure}

