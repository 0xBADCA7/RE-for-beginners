\chapter{Finden von wichtigen/ interessanten Stellen im Code}

Minimalismus ist kein beliebtes Feature von moderner Software.

\myindex{\Cpp!STL}

Aber nicht weil die Programmierer so viel Code schreiben, sondern weil die libaries allgemein statisch zu ausf\"uhrbaren
Dateien gelinkt werden. 
Wenn alle externen libraries in externe DLL Dateien verschoben werden w\"urden w\"are die Welt ein anderer Ort.
(Ein weiterer Grund f\"ur C++ sind die \ac{STL} und andere template libraries.)

\newcommand{\FOOTNOTEBOOST}{\footnote{\url{http://go.yurichev.com/17036}}}
\newcommand{\FOOTNOTELIBPNG}{\footnote{\url{http://go.yurichev.com/17037}}}

Deshalb ist es sehr wichtig den Ursprung einer Funktion zu bestimmen, wenn die Funktion aus 
einer Standard library oder aus einer sehr bekannten library stammt (wie z.B Boost\FOOTNOTEBOOST, libpng\FOOTNOTELIBPNG),
oder ob die Funktion sich auf das bezieht was wir im Code versuchen zu finden.

Es ist ein wenig absurd s\"amtlichen Code neu zu schreiben in \CCpp um das zu finden 
was wir suchen.

Eine der Hauptaufgaben eines Reverse Enigneers ist es schnell Code zu finden den er/sie sucht.

\myindex{\GrepUsage}

Der \IDA disassembler erlaubt es durch Text Strings, Byte Sequenzen und konstanten zu suchen. 
Es ist sogar m\"oglich den Code in .lst oder .asm Text Dateien zu exportieren und diese mit \TT{grep}, \TT{awk}, etc. zu untersuchen.

Wenn man versucht zu verstehen wie ein bestimmter Code funktioniert, kann es auch einfach eine open-source library wie libpng sein. % <-- kling scheisse, noch mal \"uber den Sinn nachdenken?!
Wenn man also eine Konstante oder Textstrings findet die vertraut erscheinen, ist es immer einen Versuch wert diese zu \IT{google}n .
Und wenn man ein Opensource Projekt findet in dem diese Funktion benutzt wird, 
reicht es meist aus diese Funktionen miteinander zu vergleichen.
Es k\"onnte helfen Teile des Problems zu l\"osen.

% When you try to understand what some code is doing, this easily could be some open-source library like libpng.
% So when you see some constants or text strings which look familiar, it is always worth to \IT{google} them.
% And if you find the opensource project where they are used, 
% then it's enough just to compare the functions.
% It may solve some part of the problem.

Zum Beispiel, wenn ein Programm XML Dateien benutzt, w\"are der erste Schritt zu ermitteln welche
XML library benutzt wird f\"ur die Verarbeitung, da die Standard (oder am weitesten verbreitete) libraries
normal benutzt werden anstatt selbst geschriebene librarys.

\myindex{SAP}
\myindex{Windows!PDB}

Zum Beispiel, der Autor dieser Zeilen wollte verstehen wie die Kompression/Dekompression von Netzwerkpaketen in SAP 6.0 funktioniert.
SAP ist ein gewaltiges St\"uck Software, aber detaillierte -\gls{PDB} Dateien mit Debug Informationen sind vorhanden, was sehr praktisch 
ist. Der Autor hat schließlich eine Ahnung gehabt, das eine Funktion genannt \IT{CsDecomprLZC} die Dekompression der Netzwerkpakete \"ubernahm.
Er hat nach dem Namen der Funktion auf google gesucht und ist schnell zum schluss gekommen das diese Funktion in 
MaxDB benutzt wurde (Das ist ein Open-Source SAP Projekt) \footnote{Mehr dar\"uber in der relevanten Sektion~(\myref{sec:SAPGbUI})}. 

\url{http://www.google.com/search?q=CsDecomprLZC}

Erstaunlich, das MaxDB und die SAP 6.0 Software den selben Code geteilt haben f\"ur die Kompression/Dekompression der Netzwerkpakete.

\section{Ausführbare Dateien Identifizieren}

\subsection{Microsoft Visual C++}
\label{MSVC_versions}

MSVC Versionen und DLLs die Importiert werden k\"onnen:

\small
\begin{center}
\begin{tabular}{ | l | l | l | l | l | }
\hline
\HeaderColor Marketing ver. & 
\HeaderColor Internal ver. & 
\HeaderColor CL.EXE ver. &
\HeaderColor DLLs imported &
\HeaderColor Release date \\
\hline
% 4.0, April 1995
% 97 & 5.0 & February 1997
6		&  6.0	& 12.00	& msvcrt.dll	& June 1998		\\
		&	&	& msvcp60.dll	&			\\
\hline
.NET (2002)	&  7.0	& 13.00	& msvcr70.dll	& February 13, 2002	\\
		&	&	& msvcp70.dll	&			\\
\hline
.NET 2003	&  7.1	& 13.10 & msvcr71.dll	& April 24, 2003	\\
		&	&	& msvcp71.dll	&			\\
\hline
2005		&  8.0	& 14.00 & msvcr80.dll	& November 7, 2005	\\
		&	&	& msvcp80.dll	&			\\
\hline
2008		&  9.0	& 15.00 & msvcr90.dll	& November 19, 2007	\\
		&	&	& msvcp90.dll	&			\\
\hline
2010		& 10.0	& 16.00 & msvcr100.dll	& April 12, 2010 	\\
		&	&	& msvcp100.dll	&			\\
\hline
2012		& 11.0	& 17.00 & msvcr110.dll	& September 12, 2012 	\\
		&	&	& msvcp110.dll	&			\\
\hline
2013		& 12.0	& 18.00 & msvcr120.dll	& October 17, 2013 	\\
		&	&	& msvcp120.dll	&			\\
\hline
\end{tabular}
\end{center}
\normalsize

msvcp*.dll hat \Cpp{}-bezogene Funktionen, bedeutet wenn die library importiert wird,
ist das Programm das sie importiert wahrscheinlich ein \Cpp program.

\subsubsection{Name mangling} 

Die Namen fangen normal an mit dem \TT{?} Symbol.

Hier: \myref{namemangling} kann man mehr lesen \"uber MSVC's \gls{name mangling} . 

\subsection{GCC}
\myindex{GCC}

Neben *NIX Umgebungen, ist GCC auch in win32 Umgebungen pr\"asent, in der Form von Cygwin and MinGW. 

\subsubsection{Name mangling} 

Namen fangen hier normal mit dem \TT{\_Z} Symbolen an.

Man kann mehr lesen \"uber GCC's \gls{name mangling} hier: \myref{namemangling}.

\subsubsection{Cygwin}
\myindex{Cygwin}

cygwin1.dll wird oft importiert.

\subsubsection{MinGW}
\myindex{MinGW}

msvcrt.dll wird vielleicht importiert.

\subsection{Intel Fortran}
\myindex{Fortran}


libifcoremd.dll, libifportmd.dll and libiomp5md.dll (OpenMP Support) werden vielleicht importiert.

libifcoremd.dll hat eine menge an Funktionen die das \TT{for\_} Pr\"afix haben, was \IT{Fortran} bedeutet.

\subsection{Watcom, OpenWatcom}
\myindex{Watcom}
\myindex{OpenWatcom}

\subsubsection{Name mangling}

Namen fangen normal mit dem \TT{W} Symbol an. 

Zum Beispiel wird so eine Methode benannt \q{method} der Klasse \q{class} die keine Argumente hat und \Tvoid zur\"uck gibt: % <-- Finde was besseres!
% For example, that is how the method named \q{method} of the class \q{class} that does not have any arguments and returns
% \Tvoid is encoded:

\begin{lstlisting}
W?method$_class$n__v
\end{lstlisting}

\subsection{Borland}
\myindex{Borland Delphi}
\myindex{Borland C++Builder}

Hier ist ein Beispiel f\"ur Borland Delphi's und C++Builder's \gls{name mangling}:

\lstinputlisting{digging_into_code/identification/borland_mangling.txt}

Die Namen fangen immer mit dem \TT{@} Symbol an, dann haben wir den Namen
der Klassen Namen, Methoden Namen, und codiert die Typen der Argumente der Methode.

Diese Namen k\"onnen in den .exe Imports, .dll Exports, Debug Daten und etc existieren.

Borland Visual Component Libraries (VCL) 
werden in .bpl Dateien gehalten anstatt .dll's, zum Beispiel vcl50.dll, rtl60.dll.

Eine weitere DLL die vielleicht importiert wird: BORLNDMM.DLL

\subsubsection{Delphi}

Fast alle Delphi executables haben den \q{Boolean} Text String am Anfang des Code Segments, zusammen mit den Namen anderer Typen liegen.
% Almost all Delphi executables has the \q{Boolean} text string at the beginning of the code segment, along with other type names.

Dies ist ein sehr typischer Anfang f\"ur das \TT{CODE} Segment bei einem 
Delphi Programm, dieser Block kam direkt nach dem win32 PE Datei header:

\lstinputlisting{digging_into_code/identification/delphi.txt}

Die ersten 4 Btyes des Daten Segments (\TT{DATA}) k\"onnen \TT{00 00 00 00}, \TT{32 13 8B C0} oder \TT{FF FF FF FF} sein.

Diese Informationen k\"onnen n\"utzlich sein wenn man mit gepackten oder verschl\"usselten Delphi executables arbeiten muss. 

\subsection{Other known DLLs}

\myindex{OpenMP}
\begin{itemize}
\item vcomp*.dll---Microsoft's Implementierung von OpenMP. 
\end{itemize}

 

\section{Kommunikation mit der außen Welt (Funktion Level)} 
Oft ist es empfehlenswert die Funktion Argumente und die R\"uckgabe werte im Debugger oder \ac{DBI} zu \"uberwachen.
Zum Beispiel, einmal hat der Autor versucht die Bedeutung einer obskuren Funktion zu verstehen, die einen inkorrekten
Bubble sort Algorithmus implementiert hatte\footnote{\url{https://yurichev.com/blog/weird_sort/}}
 (Sie hat funktioniert, jedoch viel langsamer als normal). Die Eingaben und Ausgaben zur laufzeit 
der Funktion zu \"uberwachen hilft instant zu verstehen was die Funktion tut.

% TBT

% sections:
\section{Communication with the outer world (win32)}

Sometimes it's enough to observe some function's inputs and outputs in order to understand what it does.
That way you can save time.

Files and registry access: 
for the very basic analysis, Process Monitor\footnote{\url{http://go.yurichev.com/17301}}
utility from SysInternals can help.

For the basic analysis of network accesses, Wireshark\footnote{\url{http://go.yurichev.com/17303}} can be useful.

But then you will have to to look inside anyway. \\
\\
The first thing to look for is which functions from the \ac{OS}'s \ac{API}s and standard libraries are used.

If the program is divided into a main executable file and a group of DLL files, sometimes the names of the functions in these DLLs can help.

If we are interested in exactly what can lead to a call to \TT{MessageBox()} with specific text, 
we can try to find this text in the data segment, find the references to it and find the points
from which the control may be passed to the \TT{MessageBox()} call we're interested in.

\myindex{\CStandardLibrary!rand()}
If we are talking about a video game and we're interested in which events are more or less random in it,
we may try to find the \rand function or its replacements (like the Mersenne twister algorithm) and find the places
from which those functions are called, and more importantly, how are the results used.
% BUG in varioref: http://tex.stackexchange.com/questions/104261/varioref-vref-or-vpageref-at-page-boundary-may-loop
One example: \ref{chap:color_lines}. 

But if it is not a game, and \rand is still used, it is also interesting to know why.
There are cases of unexpected \rand usage in data compression algorithms (for encryption imitation):
\href{http://go.yurichev.com/17221}{blog.yurichev.com}.

\subsection{Often used functions in the Windows API}

These functions may be among the imported.
It is worth to note that not every function might be used in the code that was written by the programmer.
A lot of functions might be called from library functions and \ac{CRT} code.

Some functions may have the \GTT{-A} suffix for the ASCII version and \GTT{-W} for the Unicode version.

\begin{itemize}

\item
Registry access (advapi32.dll): 
RegEnumKeyEx, RegEnumValue, RegGetValue, RegOpenKeyEx, RegQueryValueEx.

\item
Access to text .ini-files (kernel32.dll): 
GetPrivateProfileString.

\item
Dialog boxes (user32.dll): 
MessageBox, MessageBoxEx, CreateDialog, SetDlgItemText, GetDlgItemText.

\item
Resources access (\myref{PEresources}): (user32.dll): LoadMenu.

\item
TCP/IP networking (ws2\_32.dll):
WSARecv, WSASend.

\item
File access (kernel32.dll):
CreateFile, ReadFile, ReadFileEx, WriteFile, WriteFileEx.

\item
High-level access to the Internet (wininet.dll): WinHttpOpen.

\item
Checking the digital signature of an executable file (wintrust.dll):
WinVerifyTrust.

\item
The standard MSVC library (if it's linked dynamically) (msvcr*.dll):
assert, itoa, ltoa, open, printf, read, strcmp, atol, atoi, fopen, fread, fwrite, memcmp, rand,
strlen, strstr, strchr.

\end{itemize}

\subsection{Extending trial period}

Registry access functions are frequent targets for those who try to crack trial period of some software, which may save
installation date/time into registry.

Another popular target are GetLocalTime() and GetSystemTime() functions:
a trial software, at each startup, must check current date/time somehow anyway.

\subsection{Removing nag dialog box}

A popular way to find out what causing popping nag dialog box is intercepting MessageBox(), 
CreateDialog() and CreateWindow() functions.

\subsection{tracer: Intercepting all functions in specific module}
\myindex{tracer}

\myindex{x86!\Instructions!INT3}
There are INT3 breakpoints in the \tracer, that are triggered only once, however, they can be set for all functions
in a specific DLL.

\begin{lstlisting}
--one-time-INT3-bp:somedll.dll!.*
\end{lstlisting}

Or, let's set INT3 breakpoints on all functions with the \TT{xml} prefix in their name:

\begin{lstlisting}
--one-time-INT3-bp:somedll.dll!xml.*
\end{lstlisting}

On the other side of the coin, such breakpoints are triggered only once.
Tracer will show the call of a function, if it happens, but only once.
Another drawback---it is impossible to see the function's arguments.

Nevertheless, this feature is very useful when you know that the program uses a DLL,
but you do not know which functions are actually used.
And there are a lot of functions. 

\par
\myindex{Cygwin}
For example, let's see, what does the uptime utility from cygwin use:

\begin{lstlisting}
tracer -l:uptime.exe --one-time-INT3-bp:cygwin1.dll!.*
\end{lstlisting}

Thus we may see all that cygwin1.dll library functions that were called at least once, and where from:

\lstinputlisting{digging_into_code/uptime_cygwin.txt}


\section{Strings}
\label{sec:digging_strings}

\subsection{Text strings}

\subsubsection{\CCpp}

\label{C_strings}
The normal C strings are zero-terminated (\ac{ASCIIZ}-strings).

The reason why the C string format is as it is (zero-terminated) is apparently historical.
In [Dennis M. Ritchie, \IT{The Evolution of the Unix Time-sharing System}, (1979)]
we read:

\begin{framed}
\begin{quotation}
A minor difference was that the unit of I/O was the word, not the byte, because the PDP-7 was a word-addressed
machine. In practice this meant merely that all programs dealing with character streams ignored null
characters, because null was used to pad a file to an even number of characters.
\end{quotation}
\end{framed}

\myindex{Hiew}

In Hiew or FAR Manager these strings looks like this:

\begin{lstlisting}[style=customc]
int main()
{
	printf ("Hello, world!\n");
};
\end{lstlisting}

\begin{figure}[H]
\centering
\includegraphics[width=0.6\textwidth]{digging_into_code/strings/C-string.png}
\caption{Hiew}
\end{figure}

% FIXME видно \n в конце, потом пробел

\subsubsection{Borland Delphi}
\myindex{Pascal}
\myindex{Borland Delphi}

The string in Pascal and Borland Delphi is preceded by an 8-bit or 32-bit string length.

For example:

\begin{lstlisting}[caption=Delphi,style=customasmx86]
CODE:00518AC8                 dd 19h
CODE:00518ACC aLoading___Plea db 'Loading... , please wait.',0

...

CODE:00518AFC                 dd 10h
CODE:00518B00 aPreparingRun__ db 'Preparing run...',0
\end{lstlisting}

\subsubsection{Unicode}

\myindex{Unicode}

Often, what is called Unicode is a methods for encoding strings where each character occupies 2 bytes or 16 bits.
This is a common terminological mistake.
Unicode is a standard for assigning a number to each character in the many writing systems of the 
world, but does not describe the encoding method.

\myindex{UTF-8}
\myindex{UTF-16LE}
The most popular encoding methods are: UTF-8 (is widespread in Internet and *NIX systems) and UTF-16LE (is used in Windows).

\myparagraph{UTF-8}

\myindex{UTF-8}
UTF-8 is one of the most successful methods for
encoding characters.
All Latin symbols are encoded just like in ASCII,
and the symbols beyond the ASCII table are encoded using several bytes.
0 is encoded as
before, so all standard C string functions work with UTF-8 strings just like any other string.

Let's see how the symbols in various languages are encoded in UTF-8 and how it looks like in FAR, using the 437 codepage
\footnote{The example and translations was taken from here: 
\url{http://go.yurichev.com/17304}}:

\begin{figure}[H]
\centering
\includegraphics[width=0.6\textwidth]{digging_into_code/strings/multilang_sampler.png}
\end{figure}

% FIXME: cut it
\begin{figure}[H]
\centering
\myincludegraphics{digging_into_code/strings/multilang_sampler_UTF8.png}
\caption{FAR: UTF-8}
\end{figure}

As you can see, the English language string looks the same as it is in ASCII.

The Hungarian language uses some Latin symbols plus symbols with diacritic marks.

These symbols are encoded using several bytes, these are underscored with red.
It's the same story with the Icelandic and Polish languages.

There is also the \q{Euro} currency symbol at the start, which is encoded with 3 bytes.

The rest of the writing systems here have no connection with Latin.

At least in Russian, Arabic, Hebrew and Hindi we can see some recurring bytes, and that is not surprise:
all symbols from a writing system are usually located in the same Unicode table, so their code begins with
the same numbers.

At the beginning, before the \q{How much?} string we see 3 bytes, which are in fact the \ac{BOM}.
The \ac{BOM} defines the encoding system to be
used.

\myparagraph{UTF-16LE}

\myindex{UTF-16LE}
\myindex{Windows!Win32}
Many win32 functions in Windows have the suffixes \TT{-A} and \TT{-W}.
The first type of functions works
with normal strings, the other with UTF-16LE strings (\IT{wide}).

In the second case, each symbol is usually stored in a 16-bit value of type \IT{short}.

The Latin symbols in UTF-16 strings look in Hiew or FAR like they are interleaved with zero byte:

\begin{lstlisting}[style=customc]
int wmain()
{
	wprintf (L"Hello, world!\n");
};
\end{lstlisting}

\begin{figure}[H]
\centering
\includegraphics[width=0.6\textwidth]{digging_into_code/strings/UTF16-string.png}
\caption{Hiew}
\end{figure}

We can see this often in \gls{Windows NT} system files:

\begin{figure}[H]
\centering
\includegraphics[width=0.6\textwidth]{digging_into_code/strings/ntoskrnl_UTF16.png}
\caption{Hiew}
\end{figure}

\myindex{IDA}
Strings with characters that occupy exactly 2 bytes are called \q{Unicode} in \IDA:

\begin{lstlisting}[style=customasmx86]
.data:0040E000 aHelloWorld:
.data:0040E000                 unicode 0, <Hello, world!>
.data:0040E000                 dw 0Ah, 0
\end{lstlisting}

Here is how the Russian language string is encoded in UTF-16LE:

\begin{figure}[H]
\centering
\includegraphics[width=0.6\textwidth]{digging_into_code/strings/russian_UTF16.png}
\caption{Hiew: UTF-16LE}
\end{figure}

What we can easily spot is that the symbols are interleaved by the diamond character (which has the ASCII code of 4).
Indeed, the Cyrillic symbols are located in the fourth Unicode plane
\footnote{\href{http://go.yurichev.com/17003}{wikipedia}}.
Hence, all Cyrillic symbols in UTF-16LE are located in the \TT{0x400-0x4FF} range.

Let's go back to the example with the string written in multiple languages.
Here is how it looks like in UTF-16LE.

% FIXME: cut it
\begin{figure}[H]
\centering
\myincludegraphics{digging_into_code/strings/multilang_sampler_UTF16.png}
\caption{FAR: UTF-16LE}
\end{figure}

Here we can also see the \ac{BOM} at the beginning.
All Latin characters are interleaved with a zero byte.

Some characters with diacritic marks (Hungarian and Icelandic languages) are also underscored in red.

% subsection:
\subsubsection{Base64}
\myindex{Base64}

The base64 encoding is highly popular for the cases when you have to transfer binary data as a text string.

In essence, this algorithm encodes 3 binary bytes into 4 printable characters:
all 26 Latin letters (both lower and upper case), digits, plus sign (\q{+}) and slash sign (\q{/}),
64 characters in total.

One distinctive feature of base64 strings is that they often (but not always) ends with 1 or 2 \gls{padding}
equality symbol(s) (\q{=}), for example:

\begin{lstlisting}
AVjbbVSVfcUMu1xvjaMgjNtueRwBbxnyJw8dpGnLW8ZW8aKG3v4Y0icuQT+qEJAp9lAOuWs=
\end{lstlisting}

\begin{lstlisting}
WVjbbVSVfcUMu1xvjaMgjNtueRwBbxnyJw8dpGnLW8ZW8aKG3v4Y0icuQT+qEJAp9lAOuQ==
\end{lstlisting}

The equality sign (\q{=}) is never encounter in the middle of base64-encoded strings.

Now example of manual encoding.
Let's encode 0x00, 0x11, 0x22, 0x33 hexadecimal bytes into base64 string:

\lstinputlisting{digging_into_code/strings/base64_ex.sh}

Let's put all 4 bytes in binary form, then regroup them into 6-bit groups:

\begin{lstlisting}
|  00  ||  11  ||  22  ||  33  ||      ||      |
00000000000100010010001000110011????????????????
| A  || B  || E  || i  || M  || w  || =  || =  |
\end{lstlisting}

Three first bytes (0x00, 0x11, 0x22) can be encoded into 4 base64 characters (``ABEi''),
but the last one (0x33) --- cannot be,
so it's encoded using two characters (``Mw'') and \gls{padding} symbol (``='')
is added twice to pad the last group to 4 characters.
Hence, length of all correct base64 strings are always divisible by 4.

\myindex{XML}
\myindex{PGP}
Base64 is often used when binary data needs to be stored in XML.
``Armored'' (i.e., in text form) PGP keys and signatures are encoded using base64.

Some people tries to use base64 to obfuscate strings:
\url{http://blog.sec-consult.com/2016/01/deliberately-hidden-backdoor-account-in.html}
\footnote{\url{http://archive.is/nDCas}}.

\myindex{base64scanner}
There are utilities for scanning an arbitrary binary files for base64 strings.
One such utility is base64scanner\footnote{\url{https://github.com/dennis714/base64scanner}}.

\myindex{UseNet}
\myindex{FidoNet}
\myindex{Uuencoding}
Another encoding system which was much more popular in UseNet and FidoNet is Uuencoding.
It offers mostly the same features, but is different from base64 in the sense that file name
is also stored in header.

\myindex{Tor}
\myindex{base32}
By the way: there is also close sibling to base64: base32, alphabet of which has ~10 digits and ~26 Latin characters.
One well-known usage of it is onion addresses
\footnote{\url{https://trac.torproject.org/projects/tor/wiki/doc/HiddenServiceNames}},
like: \url{http://3g2upl4pq6kufc4m.onion/}.
\ac{URL} can't have mixed-case Latin characters, so apparently, this is why Tor developers used base32.





\subsection{Finding strings in binary}

\myindex{UNIX!strings}
Das Standard UNIX \IT{strings} Utility ist ein quick-n-dirty Weg um alle Strings in der 
Datei an zu schauen. Zum Beispiel, in der OpenSSH 7.2 sshd executable Datei gibt es einige Strings:

\lstinputlisting{digging_into_code/sshd_strings.txt}

Dort kann man Optionen, Fehler Meldungen, Datei Pfade, importierte dynamische Module, Funktionen und einige andere komische 
Strings (keys?) sehen. Es gibt auch nicht druckbare Zeichen---x86 Code enthält chunks von druckbaren ASCII Zeichen, bis zu ca 8 Zeichen. % <-- bessere formulierung?

Sicher, OpenSSH ist ein open-source Programm.
Aber sich die lesbaren Strings eines unbekannten Programms an zuschauen ist meist der erste Schritt bei 
der Analyse. 
\myindex{UNIX!grep}

\IT{grep} kann genauso benutzt werden.

\myindex{Hiew}
\myindex{Sysinternals}
Hiew hat die gleichen Fähigkeiten (Alt-F6), genau wie der Sysinternals ProcessMonitor.

\subsection{Error/debug messages}

Debugging Messages sind auch sehr nützlich, falls vorhanden.
Auf gewisse weise, melden die debugging messages was gerade
im Programm vorgeht. Oft schreiben diese \printf-like Funktionen, welche
in log-Dateien schreiben oder manchmal auch gar nichts schreiben aber die 
calls sind noch vorhanden, weil der build kein Debug build aber ein \IT{release} ist. % <-- nochmal über formulierung nachdenken
\myindex{\oracle}

Wenn lokale oder globale Variablen in Debug messages geschrieben werden, kann das auch 
hilfreich sein da man so an die Variablen Namen kommt. % <-- auch kacke geschrieben noch mal drüber nachdenken
Zum Beispiel, eine solche Funktion in \oracle ist \TT{ksdwrt()}.

Textstrings mit Aussage sind oft auch Hilfreich. %<-- der ist auch kacke
Der \IDA disassembler zeigt welche Funktion und von welchem Punkt aus ein spezifischer String benutzt wird.
Manchmal passieren lustige Dinge dabei\footnote{\href{http://go.yurichev.com/17223}{blog.yurichev.com}}.

Fehlermeldungen helfen uns genauso.
In \oracle, werden Fehler von einer Gruppe von Funktionen gemeldet.
Über das Thema kann man mehr hier erfahren: \href{http://go.yurichev.com/17224}{blog.yurichev.com}.

\myindex{Error messages}

Es ist Möglich heraus zu finden welche Funktionen Fehler melden und unter welchen Bedingungen.


Übrigens, das ist für Kopierschutztsysteme oft der Grund kryptische Fehlermeldungen oder einfach nur 
Fehlernummer aus zu geben. Niemand ist glücklich darüber wenn der Softwarecracker den Kopierschutz besser
versteht nur weil dieser durch eine Fehlermeldung ausgelöst wurde.

Ein Beispiel von verschlüsselten Fehlermeldungen gibt es hier: \myref{examples_SCO}.

\subsection{Suspicious magic strings}

Manche Magic Strings welche in Hintertüren benutzt werden sehen schon ziemlich verdächtig aus.


Zum Beispiel, es gab eine Hintertür im TP-Link WR740 Home Router\footnote{\url{http://sekurak.pl/tp-link-httptftp-backdoor/}}.
Die Hintertür konnte aktiviert werden wenn man folgende URL aufrief:
\url{http://192.168.0.1/userRpmNatDebugRpm26525557/start_art.html}.\\

Tatsächlich, kann man den Magic String \q{userRpmNatDebugRpm26525557} in der Firmware finden.

Der String war nicht googlebar bis die Information öffentlich über die Hintertür öffentlich verbreitet wurde.


Man würde solche Information auch nicht in irgendeinem \ac{RFC} finden. %<-- nochmal drüber nachdenken


Man würde auch keinen Informatik Algorithmus finden der solch seltsame Byte Sequenzen benutzt.


Und es sieht auch nicht nach einer Fehler- order Debugnaricht aus.


Also es ist immer eine gute Idee so seltsame Dinge genauer zu betrachten.\\
\\
\myindex{base64}

Manchmal, sind solche Strings auch mit base64 codiert.

Es ist also immer eine gute Idee diese Stings zu Decodieren und sie visuell zu durchsuchen, ein Blick
kann schon genügen.

\\
\myindex{Security through obscurity}
Präziser gesagt, diese Methode Hintertüren zu verstecken nennt man \q{security through obscurity}.



\section{assert() Aufrufe}
\myindex{\CStandardLibrary!assert()}

Manchmal ist die Präsenz des \TT{assert()} macro's ebenfalls nützlich:
allgemein erlaubt dieses Makro Rückschlüsse auf source code Dateinamen,
Zeilen nummern und die Bedienung für das Makro im Code.

Die nützlichste Informationen ist enthalten in der Bedingung von assert, wir können Variablennamen oder Namen
von Struct Feldern ableiten. Ein weiteres nützliches Stück Information sind die Datei Namen---Wir können versuchen
abzuleiten von welcher Art der Code ist. 
Es ist ebenfalls möglich bekannte open-source library-Namen von den Datei Namen abzu leiten.

\lstinputlisting[caption=Example of informative assert() calls,style=customasmx86]{digging_into_code/assert_examples.lst}

Es ist Empfehlens wert beides die Konditionen und die Datei Namen in \q{google} zu suchen, was uns zu einer open-source library führen kann. 
Zum Beispiel, wenn wir \q{sp->lzw\_nbits <= BITS\_MAX} in \q{google} suchen, ist es absehbar das wir als Ergebnis den Code aus der 
Open-Source library für die LZW Kompression bekommen. 

\section{Konstanten}

Menschen, Programmierer eingeschlossen, neigen dazu Zahlen zu Runden wie z.B 10, 100, 1000,
im realen Leben so wie in ihrem Code.

Der angehende Reverse Engineer kennt diese Werte und ihre Hexadezimale Repräsentation sehr gut:
0b10=0xA, 0b100=0x64, 0b1000=0x3E8, 0b10000=0x2710.

Die Konstanten \TT{0xAAAAAAAA} (0b10101010101010101010101010101010) und 
\TT{0x55555555} (0b01010101010101010101010101010101) sind auch sehr populär---
sie sind zusammen gesetzt aus sich verändernden Bits. % <-- Findest vielleicht noch ne besere bezeichnung

% The constants \TT{0xAAAAAAAA} (0b10101010101010101010101010101010) and \\
% \TT{0x55555555} (0b01010101010101010101010101010101)  are also popular---those
% are composed of alternating bits.

Das hilft vielleicht ein Signal von einem Signal zu unterscheiden bei dem alle Bits eingeschaltet werden  (0b1111 \dots) oder ausgeschaltet (0b0000 \dots).
Zum Beispiel die \TT{0x55AA} Konstante wird
zumindest beim Boot Sektor benutzt, \ac{MBR},
und im \ac{ROM} von IBM-Kompatiblen Erweiterung Karten.

% That may help to distinguish some signal from a signal where all bits are turned on (0b1111 \dots) or off (0b0000 \dots).
% For example, the \TT{0x55AA} constant
% is used at least in the boot sector, \ac{MBR}, 
% and in the \ac{ROM} of IBM-compatible extension cards.

Manche Algorithmen, speziell die Kryptografischen benutzen eindeutige Konstanten, die einfach zu finden 
sind im Code mit der Hilfe von \IDA.
% Some algorithms, especially cryptographical ones use distinct constants, which are easy to find
% in code using \IDA.

\myindex{MD5}
\newcommand{\URLMD}{http://go.yurichev.com/17111}

Zum Beispiel, der MD5\footnote{\href{\URLMD}{wikipedia}} Algorithmus initialisiert seine Internen Variablen wie folgt:
% For example, the MD5\footnote{\href{\URLMD}{wikipedia}} algorithm initializes its own internal variables like this:

\begin{verbatim}
var int h0 := 0x67452301
var int h1 := 0xEFCDAB89
var int h2 := 0x98BADCFE
var int h3 := 0x10325476
\end{verbatim}

Wenn man diese vier Konstanten im Code hintereinander benutzt findet, dann ist die Wahrscheinlichkeit das diese Funktion 
sich auf MD5 bezieht.
% If you find these four constants used in the code in a row, it is highly probable that this function is related to MD5.

\par Ein weiteres Beispiel sind die CRC16/CRC32 Algorithmen,
deren Berechnung Algorithmen benutzen oft vor-berechnete Tabellen wie diese:
% \par Another example are the CRC16/CRC32 algorithms, 
% whose calculation algorithms often use precomputed tables like this one:

\begin{lstlisting}[caption=linux/lib/crc16.c,style=customc]
/** CRC table for the CRC-16. The poly is 0x8005 (x^16 + x^15 + x^2 + 1) */
u16 const crc16_table[256] = {
	0x0000, 0xC0C1, 0xC181, 0x0140, 0xC301, 0x03C0, 0x0280, 0xC241,
	0xC601, 0x06C0, 0x0780, 0xC741, 0x0500, 0xC5C1, 0xC481, 0x0440,
	0xCC01, 0x0CC0, 0x0D80, 0xCD41, 0x0F00, 0xCFC1, 0xCE81, 0x0E40,
	...
\end{lstlisting}

Man beachte auch die vor berechnete Tabelle für CRC32: \myref{sec:CRC32}.
% See also the precomputed table for CRC32: \myref{sec:CRC32}.

In Tabellen losen CRC Algorithmen werden bekannte Polynome benutzt, zum Beispiel, 0xEDB88320 für CRC32.
% In tableless CRC algorithms well-known polynomials are used, for example, 0xEDB88320 for CRC32.

\subsection{Magic numbers}
\label{magic_numbers}

\newcommand{\FNURLMAGIC}{\footnote{\href{http://go.yurichev.com/17112}{wikipedia}}}

Viele Datei Formate definieren einen Standard Datei Header wo eine \IT{magic number(s)}\FNURLMAGIC{} benutzt wird, einzelne oder
sogar mehrere. 
% A lot of file formats define a standard file header where a \IT{magic number(s)}\FNURLMAGIC{} is used, single one or even several.

\myindex{MS-DOS}

Zum Beispiel, alle Win32 und MS-DOS executable starten mit zwei Zeichen \q{MZ}\footnote{\href{http://go.yurichev.com/17113}{wikipedia}}.
% For example, all Win32 and MS-DOS executables start with the two characters \q{MZ}\footnote{\href{http://go.yurichev.com/17113}{wikipedia}}.

\myindex{MIDI}

Am Anfang einer MIDI Datei muss die \q{MThd} Signatur vorhanden sein.
Wenn wir ein Programm haben das auf MIDI Dateien zugreift um sonst was zu machen,
ist es sehr wahrscheinlich das das Programm die Datei validieren muss in dem es
mindestens die ersten 4 Bytes prüft.

% At the beginning of a MIDI file the \q{MThd} signature must be present. 
% If we have a program which uses MIDI files for something,
% it's very likely that it must check the file for validity by checking at least the first 4 bytes.

Das kann wie folgt realisiert werden: % <-- 
(\IT{buf} Zeigt auf den Anfang der geladenen Datei im Speicher) 

% This could be done like this:
% (\IT{buf} points to the beginning of the loaded file in memory)

\begin{lstlisting}[style=customasmx86]
cmp [buf], 0x6468544D ; "MThd"
jnz _error_not_a_MIDI_file
\end{lstlisting}

\myindex{\CStandardLibrary!memcmp()}
\myindex{x86!\Instructions!CMPSB}

\dots oder durch das Aufrufen der Funktion für das vergleichen von speicherblöcken wie z.B \TT{memcmp()} oder 
beliebigen anderen Code bis hin zu einem \TT{CMPSB} (\myref{REPE_CMPSx}) Instruktion.

% \dots or by calling a function for comparing memory blocks like \TT{memcmp()} or any other equivalent code
% up to a \TT{CMPSB} (\myref{REPE_CMPSx}) instruction.

Wenn man so einen Punkt findet kann man bereits sagen das eine MIDI Datei geladen wird, % <-- Ändern?
wir können auch sehen wo der Buffer mit den Inhalten der MIDI Datei liegt und was/wie aus diesem
Puffer verwendet wird.

% When you find such point you already can say where the loading of the MIDI file starts,
% also, we could see the location
% of the buffer with the contents of the MIDI file, what is used from the buffer, and how.

\subsubsection{Daten}

\myindex{UFS2}
\myindex{FreeBSD}
\myindex{HASP}

Oft findet man auch nur eine Zahl wie \TT{0x19870116}, was ganz klar nach einem Jahres Datum aussieht (Tag 16,  1 Monat (Januar),  Jahr 1987).
Das ist vielleicht das Geburtsdatum von jemandem (ein Programmierer. ihre/seine bekannte, Kind), oder ein anderes wichtiges Datum.
Das Datum kann auch in umgekehrter folge auftreten, wie z.B \TT{0x16011987}. 
Datums angaben im Amerikanischen-Stil sind auch weit verbreitet wie \TT{0x01161987}.

% Often, one may encounter number like \TT{0x19870116}, which is clearly looks like a date (year 1987, 1th month (January), 16th day).
% This may be someone's birthday (a programmer, his/her relative, child), or some other important date.
% The date may also be written in a reverse order, like \TT{0x16011987}.
% American-style dates are also popular, like \TT{0x01161987}.

Ein ziemlich bekanntes Beispiel ist  \TT{0x19540119} (magic number wird in der UFS2 Superblock Struktur benutzt), das 
Geburtsdatum von Marschall Kirk McKusick ist, einem Prominenten FreeBSD Entwickler. 

% Well-known example is \TT{0x19540119} (magic number used in UFS2 superblock structure), which is a birthday of Marshall Kirk McKusick, prominent FreeBSD contributor.


\myindex{Stuxnet}
Stuxnet benutzt die Zahl ``19790509'' (nicht als 32-Bit Zahl, aber als String), was zu Spekulationen geführt hat
weil die malware Verbindungen nach Israel aufzeigt.
\footnote{Das ist das Datum der Hinrichtung von Habib Elghanian, persischer Jude.}
% Stuxnet uses the number ``19790509'' (not as 32-bit number, but as string, though), and this led to speculation
% that the malware is connected to Israel
% \footnote{This is a date of execution of Habib Elghanian, persian jew.}

Solche Zahlen sind auch sehr beliebt in Amateur Kryptografie, zum Beispiel, ein Ausschnitt aus den \IT{secret function} Interna aus dem HASP3 Dongle %  <-- Vielleicht bessere formulierung?
\footnote{\url{https://web.archive.org/web/20160311231616/http://www.woodmann.com/fravia/bayu3.htm}}:
% Also, numbers like those are very popular in amateur-grade cryptography, for example, excerpt from the \IT{secret function} internals from HASP3 dongle
% \footnote{\url{https://web.archive.org/web/20160311231616/http://www.woodmann.com/fravia/bayu3.htm}}:

\begin{lstlisting}[style=customc]
void xor_pwd(void) 
{ 
	int i; 
	
	pwd^=0x09071966;
	for(i=0;i<8;i++) 
	{ 
		al_buf[i]= pwd & 7; pwd = pwd >> 3; 
	} 
};

void emulate_func2(unsigned short seed)
{ 
	int i, j; 
	for(i=0;i<8;i++) 
	{ 
		ch[i] = 0; 
		
		for(j=0;j<8;j++)
		{ 
			seed *= 0x1989; 
			seed += 5; 
			ch[i] |= (tab[(seed>>9)&0x3f]) << (7-j); 
		}
	} 
}
\end{lstlisting}

\subsubsection{DHCP}

Das Trifft auf Netzwerk Protokolle ebenso zu. 
Zum Beispiel, die Pakete des DHCP Protokoll's beinhalten so genannte \IT{magic cookie}: \TT{0x63538263}.
Jeder Code der ein DHCP Packte generiert, muss diese Konstante in das Packte einbetten.
Wenn wir diesen Code finden, wissen wir auch wo es passiert und nicht nur was passiert.
Jedes Programm das DHCP Pakete empfangen kann muss verifizieren das der \IT{magic cookie} mit der Konstante 
übereinstimmt. 

% This applies to network protocols as well.
% For example, the DHCP protocol's network packets contains the so-called \IT{magic cookie}: \TT{0x63538263}.
% Any code that generates DHCP packets somewhere must embed this constant into the packet.
% If we find it in the code we may find where this happens and, not only that.
% Any program which can receive DHCP packet must verify the \IT{magic cookie}, comparing it with the constant.

Zum Beispiel, lasst uns die dhcpcore.dll Datei aus Windows 7 x64 analysieren die nach der Konstante suchen.
Wir können die Konstante zweimal finden:
Es sieht danach aus als wäre die Konstante in zwei Funktionen benutzt mit dem selbst redenden Namen\\
\TT{DhcpExtractOptionsForValidation()} und \TT{DhcpExtractFullOptions()}:

% For example, let's take the dhcpcore.dll file from Windows 7 x64 and search for the constant.
% And we can find it, twice:
% it seems that the constant is used in two functions with descriptive names\\
% \TT{DhcpExtractOptionsForValidation()} and \TT{DhcpExtractFullOptions()}:

\begin{lstlisting}[caption=dhcpcore.dll (Windows 7 x64),style=customasmx86]
.rdata:000007FF6483CBE8 dword_7FF6483CBE8 dd 63538263h          ; DATA XREF: DhcpExtractOptionsForValidation+79
.rdata:000007FF6483CBEC dword_7FF6483CBEC dd 63538263h          ; DATA XREF: DhcpExtractFullOptions+97
\end{lstlisting}

Und hier die (Speicher) Orte an denen auf die Konstante zugegriffen wird:
% And here are the places where these constants are accessed:

\begin{lstlisting}[caption=dhcpcore.dll (Windows 7 x64),style=customasmx86]
.text:000007FF6480875F  mov     eax, [rsi]
.text:000007FF64808761  cmp     eax, cs:dword_7FF6483CBE8
.text:000007FF64808767  jnz     loc_7FF64817179
\end{lstlisting}

Und:

\begin{lstlisting}[caption=dhcpcore.dll (Windows 7 x64),style=customasmx86]
.text:000007FF648082C7  mov     eax, [r12]
.text:000007FF648082CB  cmp     eax, cs:dword_7FF6483CBEC
.text:000007FF648082D1  jnz     loc_7FF648173AF
\end{lstlisting}

\subsection{Spezifische Konstanten}

Manchmal, gibt es spezifische Konstanten für gewissen Code % <-- Besser? 
Zum Beispiel, einmal hat der Autor sich in ein Stück Code gegraben wo die Nummer 12 verdächtig
oft vor kam. Arrays haben oft eine Größe von 12 oder ein vielfaches von 12 (24, etc). 
Wie sich raus stellte, hat der Code eine 12-Kanal Audiodatei an der Eingabe entgegen genommen und
sie verarbeitet.

% Sometimes, there is a specific constant for some type of code.
% For example, the author once dug into a code, where number 12 was encountered suspiciously often.
% Size of many arrays is 12, or multiple of 12 (24, etc).
% As it turned out, that code takes 12-channel audio file at input and process it.

Und umgekehrt: zum Beispiel, wenn ein Programm ein Textfeld verarbeitet das eine Länge von 120 Bytes hat,
dann gibt es auch eine Konstante 120 oder 119 irgendwo im Code.
Wenn UFT-16 Benutzt wird, dann $2 \cdot 120$. Wenn ein Code mit Netzwerkpaketen von fester Größe
arbeitet, ist es eine gute Idee nach dieser Konstante auch im Code zu suchen.

% And vice versa: for example, if a program works with text field which has length of 120 bytes,
% there has to be a constant 120 or 119 somewhere in the code.
% If UTF-16 is used, then $2 \cdot 120$.
% If a code works with network packets of fixed size, it's good idea to search for this constant in the code as well.

Das trifft auch auf Amateur Kryptografie zu (Lizenz Schlüssel, etc). 
Bei einem verschlüsselten Block von $n$ Bytes, will man vielleicht versuchen vorkommen dieser Nummer im Code zu suchen,
Auch, wenn man ein Stück Code sieht der sich $n$ mal während einer Schleifen Ausführung wiederholt, ist das vielleicht
eine ver-/Entschlüsselung Routine.

% This is also true for amateur cryptography (license keys, etc).
% If encrypted block has size of $n$ bytes, you may want to try to find occurences of this number throughout the code.
% Also, if you see a piece of code which is been repeated $n$ times in loop during execution,
% this may be encryption/decryption routine.

\subsection{Nach Konstanten suchen}

Das ist einfach in \IDA: Alt-B oder Alt-I.
\myindex{binär grep}
Und für das suchen von Konstanten in einem Haufen großer Dateien, oder für das suchen in nicht ausführbaren Dateien,
gibt es ein kleines Utility genannt \IT{binary grep}\footnote{\BGREPURL}.

% It is easy in \IDA: Alt-B or Alt-I.
% \myindex{binary grep}
% And for searching for a constant in a big pile of files, or for searching in non-executable files,
% there is a small utility called \IT{binary grep}\footnote{\BGREPURL}.



\section{Die richtigen Instruktionen finden}

Wenn ein Programm auf die FPU Instruktionen zugreift und der Code selber enthält nur sehr wenige
dieser Instruktionen, kann man diese einzeln mit einem Debugger überprüfen.

\par Zum Beispiel, eventuell haben wir Interesse daran wie Microsoft Excel die Formel berechnet die vom Benutzer eingegeben wurde.
Zum Beispiel die Division Operation.

\myindex{\GrepUsage}
\myindex{x86!\Instructions!FDIV}

Wenn wir excel.exe (von Office 2010) in Version 14.0.4756.1000 in \IDA laden ,ein komplettes
Listig erstellen und jede \FDIV Instruktion anschauen (ausgenommen die Instruktionen die
eine Konstante als zweiten Parameter haben---diese Instruktionen interessieren uns nicht)

\begin{lstlisting}
cat EXCEL.lst | grep fdiv | grep -v dbl_ > EXCEL.fdiv
\end{lstlisting}

\dots dann sehen wir das es 144 FPU Instruktionen gibt.

\par Wir können einen String wie z.B \TT{=(1/3)} in Excel eingeben und dann die Instruktionen überprüfen.

\myindex{tracer}

\par Beim prüfen jeder dieser Instruktionen in einem Debugger oder \tracer
( manche Prüfen 4 Instruktionen auf einmal), haben wir Glück und die
gesuchte Instruktion ist die Nummer 14:

\begin{lstlisting}[style=customasmx86]
.text:3011E919 DC 33          fdiv    qword ptr [ebx]
\end{lstlisting}

\begin{lstlisting}
PID=13944|TID=28744|(0) 0x2f64e919 (Excel.exe!BASE+0x11e919)
EAX=0x02088006 EBX=0x02088018 ECX=0x00000001 EDX=0x00000001
ESI=0x02088000 EDI=0x00544804 EBP=0x0274FA3C ESP=0x0274F9F8
EIP=0x2F64E919
FLAGS=PF IF
FPU ControlWord=IC RC=NEAR PC=64bits PM UM OM ZM DM IM 
FPU StatusWord=
FPU ST(0): 1.000000
\end{lstlisting}

\ST{0} Beinhaltet das erste Argument (1) und das zweite Argument ist in \TT{[EBX]}.\\
\\
\myindex{x86!\Instructions!FDIV}

Die Instruktion nach \FDIV (\TT{FSTP}) schreibt jedes Ergebnis in den Speicher:\\

\begin{lstlisting}[style=customasmx86]
.text:3011E91B DD 1E          fstp    qword ptr [esi]
\end{lstlisting}

Wenn wir einen Breakpoint auf diese Instruktion setzen können wir das Ergebnis betrachten:

\begin{lstlisting}
PID=32852|TID=36488|(0) 0x2f40e91b (Excel.exe!BASE+0x11e91b)
EAX=0x00598006 EBX=0x00598018 ECX=0x00000001 EDX=0x00000001
ESI=0x00598000 EDI=0x00294804 EBP=0x026CF93C ESP=0x026CF8F8
EIP=0x2F40E91B
FLAGS=PF IF
FPU ControlWord=IC RC=NEAR PC=64bits PM UM OM ZM DM IM 
FPU StatusWord=C1 P 
FPU ST(0): 0.333333
\end{lstlisting}

Auch ein netter Scherz, wir können das Ergebnis auf die schnelle ändern:

\begin{lstlisting}
tracer -l:excel.exe bpx=excel.exe!BASE+0x11E91B,set(st0,666)
\end{lstlisting}

\begin{lstlisting}
PID=36540|TID=24056|(0) 0x2f40e91b (Excel.exe!BASE+0x11e91b)
EAX=0x00680006 EBX=0x00680018 ECX=0x00000001 EDX=0x00000001
ESI=0x00680000 EDI=0x00395404 EBP=0x0290FD9C ESP=0x0290FD58
EIP=0x2F40E91B
FLAGS=PF IF
FPU ControlWord=IC RC=NEAR PC=64bits PM UM OM ZM DM IM 
FPU StatusWord=C1 P 
FPU ST(0): 0.333333
Set ST0 register to 666.000000
\end{lstlisting}

Excel zeigt nun 666 in unserer Zelle, was uns letztendlich auch bestätigt das wir das richtige Ergebnis gefunden haben.

\begin{figure}[H]
\centering
\includegraphics[width=0.6\textwidth]{digging_into_code/Excel_prank.png}
\caption{Der Scherz hat funktioniert}
\end{figure}

Wenn wir das gleiche mit der selben Excel Version versuchen, jedoch in 64-Bit Umgebungen.
Dann finden wir nur noch 12 \FDIV Instruktionen und die Instruktion nach der wir suchen ist
die dritte. 

\begin{lstlisting}
tracer.exe -l:excel.exe bpx=excel.exe!BASE+0x1B7FCC,set(st0,666)
\end{lstlisting}

\myindex{x86!\Instructions!DIVSD}

Es sieht danach aus als wären viele der Division Operationen der \Tfloat und \Tdouble Typen, vom Compiler mit SSE Instruktionen ersetzt wurden.
Wie z.B \TT{DIVSD} (\TT{DIVSD} kommt insgesamt 268 mal vor).

\section{Verd\"achtige Code muster}

\subsection{XOR Instruktionen}
\myindex{x86!\Instructions!XOR}

Instruktionen wie \TT{XOR op, op} (zum Beispiel, \TT{XOR EAX, EAX})
werden normal daf\"ur benutzt Register Werte auf Null zu setzen, wenn jedoch
einer der Operanden sich unterscheidet wird die \q{exclusive or} Operation 
ausgef\"uhrt.

Diese Operation wird allgemeinen selten benutzt beim programmieren, aber ist
weit verbreitet in der Kryptografie, besonders bei Amateuren der Kryptografie.
Sowas ist besonders Verd\"achtig wenn der zweite Operand eine große Zahl ist.

Das k\"onnte ein Hinweis sein das etwas ver-/entschl\"usselt wird oder Checksumme berechnet werden, etc.\\
\\

Eine Ausnahme dieser Beobachtung ist der \q{canary} (\myref{subsec:BO_protection}). 
Die Generierung und das pr\"ufen des \q{canary} werden oft mit Hilfe der \XOR Instruktion gemacht. \\
\\


\myindex{AWK}

Dieses AWK Skript kann benutzt werden um \IDA{} listing (.lst) Dateien zu parsen:

\begin{lstlisting}
gawk -e '$2=="xor" { tmp=substr($3, 0, length($3)-1); if (tmp!=$4) if($4!="esp") if ($4!="ebp") { print $1, $2, tmp, ",", $4 } }' filename.lst
\end{lstlisting}

Es sollte auch noch erw\"ahnt werden das diese Art von Skript in der Lage ist inkorrekt disassemblierten Code zu erkennen
(\myref{sec:incorrectly_disasmed_code}).

\subsection{Hand geschriebener Assembler code}

\myindex{Function prologue}
\myindex{Function epilogue}
\myindex{x86!\Instructions!LOOP}
\myindex{x86!\Instructions!RCL}

Moderne Compiler benutzen keine \TT{LOOP} und \TT{RCL} Instruktionen.
Auf der anderen Seite sind diese Instruktionen sehr beliebt bei Programmieren die Code direkt in Assembler schreiben.
Wenn man diese Instruktionen sieht, kann man mit hoher Sicherheit sagen das dieses Code Fragment h\"andisch geschrieben wurde.,
Diese Instruktionen sind in der Instruktionsliste im Anhang mit (M) markiert: \myref{sec:x86_instructions}.

\par

Die Funktions Prolog und Epilog sind allgemein nicht vorhanden bei handgeschriebenen Assembler Code.
\par

Tats\"achlich gibt es kein bestimmtes System um Argumente an Funktionen zu \"ubergeben wenn der Code handgeschrieben wurde. 

\par
Beispiel aus dem Windows 2003 Kernel
(ntoskrnl.exe file):

\begin{lstlisting}[style=customasmx86]
MultiplyTest proc near               ; CODE XREF: Get386Stepping
             xor     cx, cx
loc_620555:                          ; CODE XREF: MultiplyTest+E
             push    cx
             call    Multiply
             pop     cx
             jb      short locret_620563
             loop    loc_620555
             clc
locret_620563:                       ; CODE XREF: MultiplyTest+C
             retn
MultiplyTest endp

Multiply     proc near               ; CODE XREF: MultiplyTest+5
             mov     ecx, 81h
             mov     eax, 417A000h
             mul     ecx
             cmp     edx, 2
             stc
             jnz     short locret_62057F
             cmp     eax, 0FE7A000h
             stc
             jnz     short locret_62057F
             clc
locret_62057F:                       ; CODE XREF: Multiply+10
                                     ; Multiply+18
             retn
Multiply     endp
\end{lstlisting}

Tats\"achlich, wenn wir in den 
\ac{WRK} v1.2 source code schauen,
kann dieser Code einfach in der Datei
\IT{WRK-v1.2\textbackslash{}base\textbackslash{}ntos\textbackslash{}ke\textbackslash{}i386\textbackslash{}cpu.asm} gefunden werden.

\section{Using magic numbers while tracing}

Oft ist unser Hauptziel zu verstehen wie ein Programm einen Wert behandelt der entweder über eine Datei oder über das Netzwerk erhalten wurde.
Das manuelle tracen eines Wertes ist meistens ein ziemlich arbeits-intensiver Task. Eine der einfachsten Techniken um Werte zu Tracen (auch wenn nicht 100\% verlässlich)
ist eigene \IT{magic number}'s zu benutzen. 

Das ähnelt ein wenig dem Vorgang beim Röntgen auf gewisser weise: ein radioaktives Kontrastmittel wird dem Patienten injeziert,
welches dann benutzt wird um die Gefässe des Patienten besser zu erkennen duch die Rönthgenstahlung. Wie das blut bei 
gesunden Menschen in den Nieren gereinigt wird wenn das Kontrastmittel im Blut ist, man kann dann sehr einfach auf dem
Bild der Tomografie erkennen ob sich Nierensteine oder Tumore in den Nierenbefinden. 

Wir können einfach eine 32-Bit Zahl nehmen z.B \TT{0xbadf00d}, oder ein Geburtsdatum wie \TT{0x11101979}
und diese 4-Byte Zahl wird an einem bestimmten Punkt in eine Datei geschrieben welche von dem Programm 
das wir untersuchen genutzt wird. 

\myindex{\GrepUsage}
\myindex{tracer}

Dann während das programm getraced wird mit \tracer im \IT{code coverage} modus, mit der Hilfe von \IT{grep}
oder durch einfaches durchsuchen der Textdatei (der trace Ergebnisse), können wir ganz einfach sehen wo der 
Wert benutzt wurde und wie er benutzt wurde. 


Beispiel 
der \IT{grepable} \tracer Ergebnissen im \IT{cc} mode:


\begin{lstlisting}[style=customasmx86]
0x150bf66 (_kziaia+0x14), e=       1 [MOV EBX, [EBP+8]] [EBP+8]=0xf59c934 
0x150bf69 (_kziaia+0x17), e=       1 [MOV EDX, [69AEB08h]] [69AEB08h]=0 
0x150bf6f (_kziaia+0x1d), e=       1 [FS: MOV EAX, [2Ch]] 
0x150bf75 (_kziaia+0x23), e=       1 [MOV ECX, [EAX+EDX*4]] [EAX+EDX*4]=0xf1ac360 
0x150bf78 (_kziaia+0x26), e=       1 [MOV [EBP-4], ECX] ECX=0xf1ac360 
\end{lstlisting}
% TODO: good example!

Das gleiche verfahren kann man auch auf Netzwerkpakete anwenden.
Für die \IT{magic number} ist es wichtig das diese einzigartig ist und nicht im Programm code vorkommt.

\newcommand{\DOSBOXURL}{\href{http://go.yurichev.com/17222}{blog.yurichev.com}}

\myindex{DosBox}
\myindex{MS-DOS}
Neben dem \tracer Befehl, gibt es noch den DosBox (MS-DOS emulator) im heavydebug Modus,
welcher in der Lage ist alle Informationen über alle Register zustände für jede ausgeführte Instruktion des Programmes in
eine einfache Textdatei\footnote{See also my blog post about this DosBox feature: \DOSBOXURL{} zu schreiben, so kann
diese Technik für DOS Programme nützlich sein. 

% TBT \section{Schleifen}

Wann immer ein Programm mit einer Datei oder einem Puffer bestimmter Größe
zu tun hat, muss dies eine Art von Verabeitungsschleife im code haben.


Dies ist ein reales Beispiel der \tracer-Tool-Ausgabe, bei dem der Code auf
irgendeine Weise codierte Datei von 258 Byte lud.
Das Tool lief mit der Absicht die Zahl der Anweisungen zukommen
(ein \ac{DBI}-Tool würde dies heutzutage sehr viel besser machen).
Ich fand sehr schnell ein Code-Stück, welches 259/258 mal ausgeführt wurde.

\lstinputlisting{digging_into_code/crypto_loop.txt}

Wie sich herausstellte war dies auch eine Decodier-Schleife.

% TODO move section...

\subsection{Some binary file patterns} % <-- Find hier ne gute übersetzung

Alle Beispiele hier wurden vorbereitet mit Windows mit aktiver Code Page 437
\footnote{\url{https://en.wikipedia.org/wiki/Code_page_437}} in der Konsole.
Binär Dateien sehen intern etwas anders aus wenn eine andere Code page gesetzt ist.

\clearpage
\subsubsection{Arrays}

Manchmal kann man klar ein Array von 16/32/64-Bit Werten mit bloßem Auge im hex Editor erkennen.

Hier ist ein Beispiel eines 16-Bit Wertes.
Wir sehen das das erste Byte ein paar aus 7 oder 8 ist und das zweite sieht
zufällig aus:

\begin{figure}[H]
\centering
\myincludegraphics{digging_into_code/binary/16bit_array.png}
\caption{FAR: array of 16-bit values} % <-- übersetzen
\end{figure}

Ich habe eine Datei benutzt die ein 12 Kanal Signal digitalisiert benutzt mit 16-Bit \ac{ADC}. % <-- Besseres? 
% I used a file containing 12-channel signal digitized using 16-bit \ac{ADC}.

\clearpage
\myindex{MIPS}
\par Und hier ist ein Beispiel von einem Typischen MIPS Code.

Wie wir uns vielleicht erinnern, jede MIPS ( also auch ARM in ARM Mode oder ARM64 ) Instruktion hat eine Größe von 32 Bits (oder 4 Bytes),
also ist solcher Code ein Array von 32-Bit Werten. 

Wenn man den Screenshot anschaut, sehen wir eine Art Muster.

Vertikale und rote Linien wurden zur besseren Lesbarkeit eingefügt:

\begin{figure}[H]
\centering
\myincludegraphics{digging_into_code/binary/typical_MIPS_code.png}
% \caption{Hiew: very typical MIPS code}
\caption{Hiew: sehr typischer MIPS code}
\end{figure}

Ein weiteres Beispiel eines solchen Musters ist Buch:
\myref{Oracle_SYM_files_example}.

\clearpage
\subsubsection{Sparse files} % <-- übersetzen

Diese dürftige Datei mit zerstreuten Daten inmitten einer fast leeren Datei.
Jedes Space Zeichen hier ist in der tat ein Zero Byte (das wie ein space aussieht). % <-- findet man sicher was besseres
Das ist eine Datei mit der ein FPGA Programmiert wird (Ein Altera Stratix GX Gerät).
Sicher können Dateien wie diese einfach Komprimiert werden, aber diese Formate sind in 
der Wissenschaft und im Ingenieurs Wesen so wie in der Softwareentwicklung sehr verbreitet.
Wo es oft um effizienten Zugriff geht und weniger um die Komprimierung der Daten.

% This is sparse file with data scattered amidst almost empty file.
% Each space character here is in fact zero byte (which is looks like space).
% This is a file to program FPGA (Altera Stratix GX device).
% Of course, files like these can be compressed easily, but formats like this one are very popular in scientific and engineering software where efficient access is important while compactness is not.

\begin{figure}[H]
\centering
\myincludegraphics{digging_into_code/binary/sparse_FPGA.png}
\caption{FAR: Sparse file}
\end{figure}

\clearpage
\subsubsection{Compressed file} % <-- Übersetzen

% FIXME \ref{} ->
Diese Datei ist einfach ein komprimiertes Archiv. 
Es hat eine relativ hohe Entropie und visuell betrachtet sieht es 
eher Chaotisch aus. So sehen komprimierte oder verschlüsselte Dateien aus.

\begin{figure}[H]
\centering
\myincludegraphics{digging_into_code/binary/compressed.png}
\caption{FAR: Komprimierte Datei}
\end{figure}

\clearpage
\subsubsection{\ac{CDFS}}

\ac{OS} Installationen werden üblicherweise als ISO Datei bereit gestellt, die Kopien von CD/DVD Disks sind. 
Das Dateisystem das benutzt wird heißt \ac{cdfs}, hier sieht man wie Dateinamen mit zusätzlichen Daten vermischt sind.
Das können Datei Größen, Pointer auf andere Verzeichnisse, Datei Attribute und anderes sein. 
So sehen Dateisysteme typischerweise auch von innen aus.

\begin{figure}[H]
\centering
\myincludegraphics{digging_into_code/binary/cdfs.png}
\caption{FAR: ISO file: Ubuntu 15 Installation \ac{CD}}
\end{figure}

\clearpage
\subsubsection{32-bit x86 executable code} % <-- Übersetzen

So sieht 32-Bit x86 ausführbarer Code aus. 
Der Code hat nicht wirklich viel Entropie, weil manche Bytes öfters vorkommen als andere.

\begin{figure}[H]
\centering
\myincludegraphics{digging_into_code/binary/x86_32.png}
\caption{FAR: Executable 32-bit x86 code}
\end{figure}

% TODO: Read more about x86 statistics: \ref{}. % FIXME blog post about decryption...

\clearpage
\subsubsection{BMP graphics files}

% TODO: bitmap, bit, group of bits...

BMP Dateien sind nicht komprimiert, also ist jedes Byte ( oder Gruppen von Bytes ) beschrieben als
ein Pixel. Diese Bild habe ich irgendwo in meiner Windows 8.1 Installation gefunden: 

\begin{figure}[H]
\centering
\myincludegraphicsSmall{digging_into_code/binary/bmp.png}
\caption{Example picture}
\end{figure}

Man kann sehen das dieses Bild Pixel hat, die nicht wirklich gut komprimiert werden könne (um das Zentrum herum),
aber es sind lange ein-Farben Linien am Anfang und am ende der Datei. Tatsächlich Linien wie diese sehen wie Linien aus
wenn man sich die Datei anschaut:

\begin{figure}[H]
\centering
\myincludegraphics{digging_into_code/binary/bmp_FAR.png}
\caption{BMP file fragment}
\end{figure}


% FIXME comparison!
\subsection{Memory \q{snapshots} comparing}
\label{snapshots_comparing}

The technique of the straightforward comparison of two memory snapshots in order to see changes was often used to hack
8-bit computer games and for hacking \q{high score} files.

For example, if you had a loaded game on an 8-bit computer (there isn't much memory on these, but the game usually
consumes even less memory) and you know that you have now, let's say, 100 bullets, you can do a \q{snapshot}
of all memory and back it up to some place. Then shoot once, the bullet count goes to 99, do a second \q{snapshot}
and then compare both: it must be a byte somewhere which has been 100 at the beginning, and now it is 99.

Considering the fact that these 8-bit games were often written in assembly language and such variables were global,
it can be said for sure which address in memory has holding the bullet count. If you searched for all references to the
address in the disassembled game code, it was not very hard to find a piece of code \glslink{decrement}{decrementing} the bullet count,
then to write a \gls{NOP} instruction there, or a couple of \gls{NOP}-s, 
and then have a game with 100 bullets forever.
\myindex{BASIC!POKE}
Games on these 8-bit computers were commonly loaded at the constant
address, also, there were not much different versions of each game (commonly just one version was popular for a long span of time),
so enthusiastic gamers knew which bytes must be overwritten (using the BASIC's instruction \gls{POKE}) at which address in
order to hack it. This led to \q{cheat} lists that contained \gls{POKE} instructions, published in magazines related to
8-bit games. See also: \href{http://go.yurichev.com/17114}{wikipedia}.

\myindex{MS-DOS}

Likewise, it is easy to modify \q{high score} files, this does not work with just 8-bit games. Notice 
your score count and back up the file somewhere. When the \q{high score} count gets different, just compare the two files,
it can even be done with the DOS utility FC\footnote{MS-DOS utility for comparing binary files} (\q{high score} files
are often in binary form).

There will be a point where a couple of bytes are different and it is easy to see which ones are
holding the score number.
However, game developers are fully aware of such tricks and may defend the program against it.

Somewhat similar example in this book is: \myref{Millenium_DOS_game}.

% TODO: пример с какой-то простой игрушкой?

\subsubsection{Windows registry}

It is also possible to compare the Windows registry before and after a program installation.

It is a very popular method of finding which registry elements are used by the program.
Perhaps, this is the reason why the \q{windows registry cleaner} shareware is so popular.

\subsubsection{Blink-comparator}

Comparison of files or memory snapshots remind us blink-comparator
\footnote{\url{http://go.yurichev.com/17348}}:
a device used by astronomers in past, intended to find moving celestial objects.

Blink-comparator allows to switch quickly between two photographies shot in different time,
so astronomer would spot the difference visually.

By the way, Pluto was discovered by blink-comparator in 1930.

% TBT \input{digging_into_code/ISA_detect_DE}

\section{Andere Dinge}

\subsection{Die Idee}  

Ein Reverse Engineer sollte versuchen so oft wie M\"oglich in den Schuhen des Programmierers zu laufen.
Um ihren/seinen Standpunkt zu betrachten uns sich selbst zu Fragen wie man einen Task in spezifischen F\"allen l\"osen w\"urde.

\subsection{Anordnung von Funktionen in Bin\"ar Code}  

S\"amtliche Funktionen die in einer einzelnen .c oder .cpp-Datei gefunden werden, werden zu den entsprechenden Objekt Dateien (.o) kompiliert. Sp\"ater, f\"ugt der Linker alle Objektdatein die er braucht zusammen, ohne die Reihenfolge oder die Funktionen in Ihnen zu ver\"andern. Als eine Konsequenz, ergibt sich daraus wenn man zwei oder mehr aufeinander folgende Funktionen sieht, bedeutet dass das sie in der gleichen Source Code Datei platziert waren ( Ausser nat\"urlich man bewegt sich an der Grenze zwischen zwei Dateien. ).  Das bedeutet
das diese Funktionen etwas gemeinsam haben, das sie aus dem gleichen \ac{API} Level stammen oder aus der gleichen library, etc.

\subsection{kleine Funktionen} 

Sehr kleine oder leere Funktionen  (\myref{empty_func})
oder Funktionen die nur ``true'' (1) oder ``false'' (0) (\myref{ret_val_func}) sind weit verbreitet,
und fast jeder ordentlicher Compiler tendiert dazu nur solche Funktionen in den resultierenden ausf\"uhrbaren Code zu stecken,
sogar wenn es mehrere gleiche Funktionen im Source Code bereits gibt. 
Also, wann immer man solche kleinen Funktionen sieht die z.B nur aus \TT{mov eax, 1 / ret} bestehen und von mehreren 
Orten aus referenziert werden (und aufgerufen werden k\"onnen), und scheinbar keine Verbindung zu einander haben, dann 
ist das wahrscheinlich das Ergebnis einer Optimierung. 

\subsection{\Cpp}

\ac{RTTI}~(\myref{RTTI})-data ist vielleicht auch n\"utzlich f\"ur die \Cpp Klassen Identifikation.
