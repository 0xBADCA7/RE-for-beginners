\chapter{Finden von wichtigen/ interessanten Stellen im Code}

Minimalismus ist kein beliebtes feature von moderner Software.
% Minimalism it is not a prominent feature of modern software.

\myindex{\Cpp!STL}

Aber nicht weil die Programmierer so viel Code schreiben, sondern weil die libaries allgemein statisch zu ausführbaren
Dateien gelinkt werden. 
Wenn alle externen libraries in externe DLL Dateien verschoben werden würden wäre die Welt ein anderer Ort.
(Ein weiterer grund für C++ sind die \ac{STL} und andere template libraries.)

% But not because the programmers are writing a lot, but because a lot of libraries are commonly linked statically
% to executable files.
% If all external libraries were shifted into an external DLL files, the world would be different.
% (Another reason for C++ are the \ac{STL} and other template libraries.)

\newcommand{\FOOTNOTEBOOST}{\footnote{\url{http://go.yurichev.com/17036}}}
\newcommand{\FOOTNOTELIBPNG}{\footnote{\url{http://go.yurichev.com/17037}}}

Deshalb ist es sehr wichtig den ursprung einer Funktion zu bestimmen, wenn die Funktuion aus 
einer Standart library oder aus einer sehr bekannten library stammt (wie z.B Boost\FOOTNOTEBOOST, libpng\FOOTNOTELIBPNG),
oder ob die Funktion sich auf das bezieht was wir im Code versuchen zu finden.

% Thus, it is very important to determine the origin of a function, if it is from standard library or 
% well-known library (like Boost\FOOTNOTEBOOST, libpng\FOOTNOTELIBPNG),
% or if it is related to what we are trying to find in the code.

Es ist ein wenig absurd sämtlichen code neu zu schreiben in \CCpp um das zu finden 
was wir suchen.
% It is just absurd to rewrite all code in \CCpp to find what we're looking for.

Eine der Hauptaufgaben eines Reverse Enigneers ist es schnell den Code zu finden den er/sie sucht. % <-- nachbessern (formulierung) ?
% One of the primary tasks of a reverse engineer is to find quickly the code he/she needs.

\myindex{\GrepUsage}

Der \IDA disassembler erlaubt uns durch Text Strings, Byte sqeuenzen und konstanten zu suchen. % <-- noch mal? 
Es ist sogar möglich den code in .lst oder .asm Text dateien zu exporiteren und diese mit \TT{grep}, \TT{awk}, etc. zu untersuchen.

% The \IDA disassembler allow us to search among text strings, byte sequences and constants.
% It is even possible to export the code to .lst or .asm text files and then use \TT{grep}, \TT{awk}, etc.

Wenn man versucht zu verstehen wie ein bestimmter Code funtkioniert, kann es auch einfach eine open-source library wie libpng sein. % <-- kling scheisse, noch mal über den Sinn nachdenken?!
Wenn man also eine Konstante oder Textstrings findet die vertraut erscheinen, ist es immer einen Versuch wert diese zu \IT{google}n .
Und wenn man ein Opensource Project findet in dem diese Funktion benutzt wird, 
reicht es meist aus diese Funktionen miteinander zu vergleichen.
Es könnte helfen Teile des Problems zu lösen.

% When you try to understand what some code is doing, this easily could be some open-source library like libpng.
% So when you see some constants or text strings which look familiar, it is always worth to \IT{google} them.
% And if you find the opensource project where they are used, 
% then it's enough just to compare the functions.
% It may solve some part of the problem.

Zum Beispiel, wenn ein Programm XML dateien benutzt, wäre der erste Schritt zu ermitteln welche
XML library benutzt wird für die verarbeitung, da die Standart (oder am weitesten verbreitete) libraries
normal benutzt werden anstatt selbst geschriebene librarys.

% For example, if a program uses XML files, the first step may be determining which
% XML library is used for processing, since the standard (or well-known) libraries are usually used
% instead of self-made one.

\myindex{SAP}
\myindex{Windows!PDB}

Zum Beispiel, der Author dieser Zeilen wollte verstehen wie die kompression/dekomprission von Netzwerkpaketen in SAP 6.0 funktioniert.
SAP ist ein gewaltiges Stück Software, aber detailierte -\gls{PDB} Dateien mit debugging Informationen sind vorhanden, was sehr praktisch 
ist. Der Author hat schliesslich eine Ahnung gehabt, das eine Funktion genannt \IT{CsDecomprLZC} die dekompression der Netzwerkpakete übernahm.
Er hat den Namen der Funtkion gegoogled und ist schnell darauf gekommen das diese Funktion in MaxDB benutzt wurde
(Das ist ein Open-Source SAP Projekt) \footnote{Mehr darüber in der relevanten Sektion~(\myref{sec:SAPGbUI})}. 

% For example, the author of these lines once tried to understand how the compression/decompression of network packets works in SAP 6.0. 
% It is a huge software, but a detailed .\gls{PDB} with debugging information is present, 
% and that is convenient.
% He finally came to the idea that one of the functions, that was called \IT{CsDecomprLZC}, was doing the decompression of network packets.
% Immediately he tried to google its name and he quickly found the function was used in MaxDB
% (it is an open-source SAP project) \footnote{More about it in relevant section~(\myref{sec:SAPGbUI})}.

\url{http://www.google.com/search?q=CsDecomprLZC}

Erstaunlich, das MaxDB und die SAP 6.0 Software den selben code geteilt haben für die Kompression/Dekompression der Netzwerkpakete.
% Astoundingly, MaxDB and SAP 6.0 software shared likewise code for the compression/decompression of network packets.

\input{digging_into_code/identification/exec_EN}

% binary files might be also here

\section{Communication with outer world (function level)}
Oft ist es empfehlenswert die Funktions Argumente und die Rückgabewerte im Debugger oder \ac{DBI} zu überwachen.
Zum Beispiel, einmal hat der Author versucht die bedeutung einer obscuren Funktion zu verstehen, die einen inkorrekten
Bubble sort algorithmus implementiert hatte. (Sie hat funktioniert, jedoch langsamer.) 
Zwischenzeitlich, die Eingaben und Ausgaben der Funktion zu überwachen hilft instant zu verstehen was die Funktion tut.

% It's often advisable to track function arguments and return values in debugger or \ac{DBI}.
% For example, the author once tried to understand meaning of some obscure function, which happens to be incorrectly
% implemented bubble sort.
% (It worked correctly, but slower.)
% Meanwhile, watching inputs and outputs of this function helps instantly to understand what it does.%

% The following input files each one has to be also translated!
\section{Communication with the outer world (win32)}

Sometimes it's enough to observe some function's inputs and outputs in order to understand what it does.
That way you can save time.

Files and registry access: 
for the very basic analysis, Process Monitor\footnote{\url{http://go.yurichev.com/17301}}
utility from SysInternals can help.

For the basic analysis of network accesses, Wireshark\footnote{\url{http://go.yurichev.com/17303}} can be useful.

But then you will have to look inside anyway. \\
\\
The first thing to look for is which functions from the \ac{OS}'s \ac{API}s and standard libraries are used.

If the program is divided into a main executable file and a group of DLL files, sometimes the names of the functions in these DLLs can help.

If we are interested in exactly what can lead to a call to \TT{MessageBox()} with specific text, 
we can try to find this text in the data segment, find the references to it and find the points
from which the control may be passed to the \TT{MessageBox()} call we're interested in.

\myindex{\CStandardLibrary!rand()}
If we are talking about a video game and we're interested in which events are more or less random in it,
we may try to find the \rand function or its replacements (like the Mersenne twister algorithm) and find the places
from which those functions are called, and more importantly, how are the results used.
% BUG in varioref: http://tex.stackexchange.com/questions/104261/varioref-vref-or-vpageref-at-page-boundary-may-loop
One example: \ref{chap:color_lines}. 

But if it is not a game, and \rand is still used, it is also interesting to know why.
There are cases of unexpected \rand usage in data compression algorithms (for encryption imitation):
\href{http://go.yurichev.com/17221}{blog.yurichev.com}.

\subsection{Often used functions in the Windows API}

These functions may be among the imported.
It is worth to note that not every function might be used in the code that was written by the programmer.
A lot of functions might be called from library functions and \ac{CRT} code.

Some functions may have the \GTT{-A} suffix for the ASCII version and \GTT{-W} for the Unicode version.

\begin{itemize}

\item
Registry access (advapi32.dll): 
RegEnumKeyEx, RegEnumValue, RegGetValue, RegOpenKeyEx, RegQueryValueEx.

\item
Access to text .ini-files (kernel32.dll): 
GetPrivateProfileString.

\item
Dialog boxes (user32.dll): 
MessageBox, MessageBoxEx, CreateDialog, SetDlgItemText, GetDlgItemText.

\item
Resources access (\myref{PEresources}): (user32.dll): LoadMenu.

\item
TCP/IP networking (ws2\_32.dll):
WSARecv, WSASend.

\item
File access (kernel32.dll):
CreateFile, ReadFile, ReadFileEx, WriteFile, WriteFileEx.

\item
High-level access to the Internet (wininet.dll): WinHttpOpen.

\item
Checking the digital signature of an executable file (wintrust.dll):
WinVerifyTrust.

\item
The standard MSVC library (if it's linked dynamically) (msvcr*.dll):
assert, itoa, ltoa, open, printf, read, strcmp, atol, atoi, fopen, fread, fwrite, memcmp, rand,
strlen, strstr, strchr.

\end{itemize}

\subsection{Extending trial period}

Registry access functions are frequent targets for those who try to crack trial period of some software, which may save
installation date/time into registry.

Another popular target are GetLocalTime() and GetSystemTime() functions:
a trial software, at each startup, must check current date/time somehow anyway.

\subsection{Removing nag dialog box}

A popular way to find out what causing popping nag dialog box is intercepting MessageBox(), 
CreateDialog() and CreateWindow() functions.

\subsection{tracer: Intercepting all functions in specific module}
\myindex{tracer}

\myindex{x86!\Instructions!INT3}
There are INT3 breakpoints in the \tracer, that are triggered only once, however, they can be set for all functions
in a specific DLL.

\begin{lstlisting}
--one-time-INT3-bp:somedll.dll!.*
\end{lstlisting}

Or, let's set INT3 breakpoints on all functions with the \TT{xml} prefix in their name:

\begin{lstlisting}
--one-time-INT3-bp:somedll.dll!xml.*
\end{lstlisting}

On the other side of the coin, such breakpoints are triggered only once.
Tracer will show the call of a function, if it happens, but only once.
Another drawback---it is impossible to see the function's arguments.

Nevertheless, this feature is very useful when you know that the program uses a DLL,
but you do not know which functions are actually used.
And there are a lot of functions. 

\par
\myindex{Cygwin}
For example, let's see, what does the uptime utility from cygwin use:

\begin{lstlisting}
tracer -l:uptime.exe --one-time-INT3-bp:cygwin1.dll!.*
\end{lstlisting}

Thus we may see all that cygwin1.dll library functions that were called at least once, and where from:

\lstinputlisting{digging_into_code/uptime_cygwin.txt}


\section{Strings}
\label{sec:digging_strings}

\subsection{Text strings}

\subsubsection{\CCpp}

\label{C_strings}
The normal C strings are zero-terminated (\ac{ASCIIZ}-strings).

The reason why the C string format is as it is (zero-terminated) is apparently historical.
In [Dennis M. Ritchie, \IT{The Evolution of the Unix Time-sharing System}, (1979)]
we read:

\begin{framed}
\begin{quotation}
A minor difference was that the unit of I/O was the word, not the byte, because the PDP-7 was a word-addressed
machine. In practice this meant merely that all programs dealing with character streams ignored null
characters, because null was used to pad a file to an even number of characters.
\end{quotation}
\end{framed}

\myindex{Hiew}

In Hiew or FAR Manager these strings looks like this:

\begin{lstlisting}[style=customc]
int main()
{
	printf ("Hello, world!\n");
};
\end{lstlisting}

\begin{figure}[H]
\centering
\includegraphics[width=0.6\textwidth]{digging_into_code/strings/C-string.png}
\caption{Hiew}
\end{figure}

% FIXME видно \n в конце, потом пробел

\subsubsection{Borland Delphi}
\myindex{Pascal}
\myindex{Borland Delphi}

The string in Pascal and Borland Delphi is preceded by an 8-bit or 32-bit string length.

For example:

\begin{lstlisting}[caption=Delphi,style=customasmx86]
CODE:00518AC8                 dd 19h
CODE:00518ACC aLoading___Plea db 'Loading... , please wait.',0

...

CODE:00518AFC                 dd 10h
CODE:00518B00 aPreparingRun__ db 'Preparing run...',0
\end{lstlisting}

\subsubsection{Unicode}

\myindex{Unicode}

Often, what is called Unicode is a methods for encoding strings where each character occupies 2 bytes or 16 bits.
This is a common terminological mistake.
Unicode is a standard for assigning a number to each character in the many writing systems of the 
world, but does not describe the encoding method.

\myindex{UTF-8}
\myindex{UTF-16LE}
The most popular encoding methods are: UTF-8 (is widespread in Internet and *NIX systems) and UTF-16LE (is used in Windows).

\myparagraph{UTF-8}

\myindex{UTF-8}
UTF-8 is one of the most successful methods for
encoding characters.
All Latin symbols are encoded just like in ASCII,
and the symbols beyond the ASCII table are encoded using several bytes.
0 is encoded as
before, so all standard C string functions work with UTF-8 strings just like any other string.

Let's see how the symbols in various languages are encoded in UTF-8 and how it looks like in FAR, using the 437 codepage
\footnote{The example and translations was taken from here: 
\url{http://go.yurichev.com/17304}}:

\begin{figure}[H]
\centering
\includegraphics[width=0.6\textwidth]{digging_into_code/strings/multilang_sampler.png}
\end{figure}

% FIXME: cut it
\begin{figure}[H]
\centering
\myincludegraphics{digging_into_code/strings/multilang_sampler_UTF8.png}
\caption{FAR: UTF-8}
\end{figure}

As you can see, the English language string looks the same as it is in ASCII.

The Hungarian language uses some Latin symbols plus symbols with diacritic marks.

These symbols are encoded using several bytes, these are underscored with red.
It's the same story with the Icelandic and Polish languages.

There is also the \q{Euro} currency symbol at the start, which is encoded with 3 bytes.

The rest of the writing systems here have no connection with Latin.

At least in Russian, Arabic, Hebrew and Hindi we can see some recurring bytes, and that is not surprise:
all symbols from a writing system are usually located in the same Unicode table, so their code begins with
the same numbers.

At the beginning, before the \q{How much?} string we see 3 bytes, which are in fact the \ac{BOM}.
The \ac{BOM} defines the encoding system to be
used.

\myparagraph{UTF-16LE}

\myindex{UTF-16LE}
\myindex{Windows!Win32}
Many win32 functions in Windows have the suffixes \TT{-A} and \TT{-W}.
The first type of functions works
with normal strings, the other with UTF-16LE strings (\IT{wide}).

In the second case, each symbol is usually stored in a 16-bit value of type \IT{short}.

The Latin symbols in UTF-16 strings look in Hiew or FAR like they are interleaved with zero byte:

\begin{lstlisting}[style=customc]
int wmain()
{
	wprintf (L"Hello, world!\n");
};
\end{lstlisting}

\begin{figure}[H]
\centering
\includegraphics[width=0.6\textwidth]{digging_into_code/strings/UTF16-string.png}
\caption{Hiew}
\end{figure}

We can see this often in \gls{Windows NT} system files:

\begin{figure}[H]
\centering
\includegraphics[width=0.6\textwidth]{digging_into_code/strings/ntoskrnl_UTF16.png}
\caption{Hiew}
\end{figure}

\myindex{IDA}
Strings with characters that occupy exactly 2 bytes are called \q{Unicode} in \IDA:

\begin{lstlisting}[style=customasmx86]
.data:0040E000 aHelloWorld:
.data:0040E000                 unicode 0, <Hello, world!>
.data:0040E000                 dw 0Ah, 0
\end{lstlisting}

Here is how the Russian language string is encoded in UTF-16LE:

\begin{figure}[H]
\centering
\includegraphics[width=0.6\textwidth]{digging_into_code/strings/russian_UTF16.png}
\caption{Hiew: UTF-16LE}
\end{figure}

What we can easily spot is that the symbols are interleaved by the diamond character (which has the ASCII code of 4).
Indeed, the Cyrillic symbols are located in the fourth Unicode plane
\footnote{\href{http://go.yurichev.com/17003}{wikipedia}}.
Hence, all Cyrillic symbols in UTF-16LE are located in the \TT{0x400-0x4FF} range.

Let's go back to the example with the string written in multiple languages.
Here is how it looks like in UTF-16LE.

% FIXME: cut it
\begin{figure}[H]
\centering
\myincludegraphics{digging_into_code/strings/multilang_sampler_UTF16.png}
\caption{FAR: UTF-16LE}
\end{figure}

Here we can also see the \ac{BOM} at the beginning.
All Latin characters are interleaved with a zero byte.

Some characters with diacritic marks (Hungarian and Icelandic languages) are also underscored in red.

% subsection:
\subsubsection{Base64}
\myindex{Base64}

The base64 encoding is highly popular for the cases when you have to transfer binary data as a text string.

In essence, this algorithm encodes 3 binary bytes into 4 printable characters:
all 26 Latin letters (both lower and upper case), digits, plus sign (\q{+}) and slash sign (\q{/}),
64 characters in total.

One distinctive feature of base64 strings is that they often (but not always) ends with 1 or 2 \gls{padding}
equality symbol(s) (\q{=}), for example:

\begin{lstlisting}
AVjbbVSVfcUMu1xvjaMgjNtueRwBbxnyJw8dpGnLW8ZW8aKG3v4Y0icuQT+qEJAp9lAOuWs=
\end{lstlisting}

\begin{lstlisting}
WVjbbVSVfcUMu1xvjaMgjNtueRwBbxnyJw8dpGnLW8ZW8aKG3v4Y0icuQT+qEJAp9lAOuQ==
\end{lstlisting}

The equality sign (\q{=}) is never encounter in the middle of base64-encoded strings.

Now example of manual encoding.
Let's encode 0x00, 0x11, 0x22, 0x33 hexadecimal bytes into base64 string:

\lstinputlisting{digging_into_code/strings/base64_ex.sh}

Let's put all 4 bytes in binary form, then regroup them into 6-bit groups:

\begin{lstlisting}
|  00  ||  11  ||  22  ||  33  ||      ||      |
00000000000100010010001000110011????????????????
| A  || B  || E  || i  || M  || w  || =  || =  |
\end{lstlisting}

Three first bytes (0x00, 0x11, 0x22) can be encoded into 4 base64 characters (``ABEi''),
but the last one (0x33) --- cannot be,
so it's encoded using two characters (``Mw'') and \gls{padding} symbol (``='')
is added twice to pad the last group to 4 characters.
Hence, length of all correct base64 strings are always divisible by 4.

\myindex{XML}
\myindex{PGP}
Base64 is often used when binary data needs to be stored in XML.
``Armored'' (i.e., in text form) PGP keys and signatures are encoded using base64.

Some people tries to use base64 to obfuscate strings:
\url{http://blog.sec-consult.com/2016/01/deliberately-hidden-backdoor-account-in.html}
\footnote{\url{http://archive.is/nDCas}}.

\myindex{base64scanner}
There are utilities for scanning an arbitrary binary files for base64 strings.
One such utility is base64scanner\footnote{\url{https://github.com/dennis714/base64scanner}}.

\myindex{UseNet}
\myindex{FidoNet}
\myindex{Uuencoding}
Another encoding system which was much more popular in UseNet and FidoNet is Uuencoding.
It offers mostly the same features, but is different from base64 in the sense that file name
is also stored in header.

\myindex{Tor}
\myindex{base32}
By the way: there is also close sibling to base64: base32, alphabet of which has ~10 digits and ~26 Latin characters.
One well-known usage of it is onion addresses
\footnote{\url{https://trac.torproject.org/projects/tor/wiki/doc/HiddenServiceNames}},
like: \url{http://3g2upl4pq6kufc4m.onion/}.
\ac{URL} can't have mixed-case Latin characters, so apparently, this is why Tor developers used base32.





\subsection{Finding strings in binary}

\epigraph{Actually, the best form of Unix documentation is frequently running the
\textbf{strings} command over a program’s object code. Using \textbf{strings}, you can get
a complete list of the program’s hard-coded file name, environment variables,
undocumented options, obscure error messages, and so forth.}{The Unix-Haters Handbook}

\myindex{UNIX!strings}
The standard UNIX \IT{strings} utility is quick-n-dirty way to see strings in file.
For example, these are some strings from OpenSSH 7.2 sshd executable file:

\lstinputlisting{digging_into_code/sshd_strings.txt}

There are options, error messages, file paths, imported dynamic modules and functions, some other strange strings (keys?)
There is also unreadable noise---x86 code sometimes has chunks consisting of printable ASCII characters, up to ~8 characters.

Of course, OpenSSH is open-source program.
But looking at readable strings inside of some unknown binary is often a first step of analysis.
\myindex{UNIX!grep}

\IT{grep} can be applied as well.

\myindex{Hiew}
\myindex{Sysinternals}
Hiew has the same capability (Alt-F6), as well as Sysinternals ProcessMonitor.

\subsection{Error/debug messages}

Debugging messages are very helpful if present.
In some sense, the debugging messages are reporting
what's going on in the program right now. Often these are \printf-like functions,
which write to log-files, or sometimes do not writing anything but the calls are still present 
since the build is not a debug one but \IT{release} one.
\myindex{\oracle}

If local or global variables are dumped in debug messages, it might be helpful as well 
since it is possible to get at least the variable names.
For example, one of such function in \oracle is \TT{ksdwrt()}.

Meaningful text strings are often helpful.
The \IDA disassembler may show from which function and from which point this specific string is used.
Funny cases sometimes happen\footnote{\href{http://go.yurichev.com/17223}{blog.yurichev.com}}.

The error messages may help us as well.
In \oracle, errors are reported using a group of functions.\\
You can read more about them here: \href{http://go.yurichev.com/17224}{blog.yurichev.com}.

\myindex{Error messages}

It is possible to find quickly which functions report errors and in which conditions.

By the way, this is often the reason for copy-protection systems to inarticulate cryptic error messages 
or just error numbers. No one is happy when the software cracker quickly understand why the copy-protection
is triggered just by the error message.

One example of encrypted error messages is here: \myref{examples_SCO}.

\subsection{Suspicious magic strings}

Some magic strings which are usually used in backdoors looks pretty suspicious.

For example, there was a backdoor in the TP-Link WR740 home router\footnote{\url{http://sekurak.pl/tp-link-httptftp-backdoor/}}.
The backdoor can activated using the following URL:\\
\url{http://192.168.0.1/userRpmNatDebugRpm26525557/start_art.html}.\\

Indeed, the \q{userRpmNatDebugRpm26525557} string is present in the firmware.

This string was not googleable until the wide disclosure of information about the backdoor.

You would not find this in any \ac{RFC}.

You would not find any computer science algorithm which uses such strange byte sequences.

And it doesn't look like an error or debugging message.

So it's a good idea to inspect the usage of such weird strings.\\
\\
\myindex{base64}

Sometimes, such strings are encoded using
base64.

So it's a good idea to decode them all and to scan them visually, even a glance should be enough.\\
\\
\myindex{Security through obscurity}
More precise, this method of hiding backdoors is called \q{security through obscurity}.

\section{Строки}
\label{sec:digging_strings}

\subsection{Текстовые строки}

\subsubsection{\CCpp}

\label{C_strings}
Обычные строки в Си заканчиваются нулем (\ac{ASCIIZ}-строки).

Причина, почему формат строки в Си именно такой (оканчивающийся нулем) вероятно историческая.
В [Dennis M. Ritchie, \IT{The Evolution of the Unix Time-sharing System}, (1979)]
мы можем прочитать:

\begin{framed}
\begin{quotation}
A minor difference was that the unit of I/O was the word, not the byte, because the PDP-7 was a word-addressed
machine. In practice this meant merely that all programs dealing with character streams ignored null
characters, because null was used to pad a file to an even number of characters.
\end{quotation}
\end{framed}

\myindex{Hiew}
Строки выглядят в Hiew или FAR Manager точно так же:

\begin{lstlisting}[style=customc]
int main()
{
	printf ("Hello, world!\n");
};
\end{lstlisting}

\begin{figure}[H]
\centering
\includegraphics[width=0.6\textwidth]{digging_into_code/strings/C-string.png}
\caption{Hiew}
\end{figure}

% FIXME видно \n в конце, потом пробел

\subsubsection{Borland Delphi}
\myindex{Pascal}
\myindex{Borland Delphi}
Когда кодируются строки в Pascal и Delphi, сама строка предваряется 8-битным или 32-битным значением, в котором закодирована длина строки.

Например:

\begin{lstlisting}[caption=Delphi,style=customasmx86]
CODE:00518AC8                 dd 19h
CODE:00518ACC aLoading___Plea db 'Loading... , please wait.',0

...

CODE:00518AFC                 dd 10h
CODE:00518B00 aPreparingRun__ db 'Preparing run...',0
\end{lstlisting}

\subsubsection{Unicode}

\myindex{Unicode}
Нередко уникодом называют все способы передачи символов, когда символ занимает 2 байта или 16 бит.
Это распространенная терминологическая ошибка.
Уникод --- это стандарт, присваивающий номер каждому символу многих письменностей мира, но не описывающий
способ кодирования.

\myindex{UTF-8}
\myindex{UTF-16LE}
Наиболее популярные способы кодирования: 
UTF-8 (наиболее часто используется в Интернете и *NIX-системах) и UTF-16LE (используется в Windows).

\myparagraph{UTF-8}

\myindex{UTF-8}
UTF-8 это один из очень удачных способов кодирования символов.
Все символы латиницы кодируются так же, как и в ASCII-кодировке, а символы, выходящие за пределы
ASCII-7-таблицы, кодируются несколькими байтами.
0 кодируется, как и прежде, нулевыми байтом, так что все стандартные
функции Си продолжают работать с UTF-8-строками так же как и с обычными строками.

Посмотрим, как символы из разных языков кодируются в UTF-8 и как это выглядит в FAR, в кодировке 437

\footnote{Пример и переводы на разные языки были взяты здесь: 
\url{http://go.yurichev.com/17304}}:

\begin{figure}[H]
\centering
\includegraphics[width=0.6\textwidth]{digging_into_code/strings/multilang_sampler.png}
\end{figure}

% FIXME: cut it
\begin{figure}[H]
\centering
\myincludegraphics{digging_into_code/strings/multilang_sampler_UTF8.png}
\caption{FAR: UTF-8}
\end{figure}

Видно, что строка на английском языке выглядит точно так же, как и в ASCII-кодировке.
В венгерском языке используются латиница плюс латинские буквы с диакритическими знаками.
Здесь видно, что эти буквы кодируются несколькими байтами, они подчеркнуты красным.
То же самое с исландским и польским языками.
В самом начале имеется также символ валюты \q{Евро}, который кодируется тремя байтами.
Остальные системы письма здесь никак не связаны с латиницей.
По крайней мере о русском, арабском, иврите и хинди мы можем сказать, что здесь видны повторяющиеся
байты, что не удивительно, ведь, обычно буквы из одной и той же системы письменности расположены в одной
или нескольких таблицах уникода, поэтому часто их коды начинаются с одних и тех же цифр.

В самом начале, перед строкой \q{How much?}, видны три байта, которые на самом деле \ac{BOM}.
\ac{BOM} показывает, какой способ кодирования будет сейчас использоваться.

\myparagraph{UTF-16LE}

\myindex{UTF-16LE}
\myindex{Windows!Win32}
Многие функции win32 в Windows имеют суффикс \TT{-A} и \TT{-W}.
Первые функции работают с обычными строками, вторые с UTF-16LE-строками (\IT{wide}).
Во втором случае, каждый символ обычно хранится в 16-битной переменной типа \IT{short}.

Cтроки с латинскими буквами выглядят в Hiew или FAR как перемежающиеся с нулевыми байтами:

\begin{lstlisting}[style=customc]
int wmain()
{
	wprintf (L"Hello, world!\n");
};
\end{lstlisting}

\begin{figure}[H]
\centering
\includegraphics[width=0.6\textwidth]{digging_into_code/strings/UTF16-string.png}
\caption{Hiew}
\end{figure}

Подобное можно часто увидеть в системных файлах \gls{Windows NT}:

\begin{figure}[H]
\centering
\includegraphics[width=0.6\textwidth]{digging_into_code/strings/ntoskrnl_UTF16.png}
\caption{Hiew}
\end{figure}

\myindex{IDA}
В \IDA, уникодом называется именно строки с символами, занимающими 2 байта:

\begin{lstlisting}[style=customasmx86]
.data:0040E000 aHelloWorld:
.data:0040E000                 unicode 0, <Hello, world!>
.data:0040E000                 dw 0Ah, 0
\end{lstlisting}

Вот как может выглядеть строка на русском языке (\q{И снова здравствуйте!}) закодированная в UTF-16LE:

\begin{figure}[H]
\centering
\includegraphics[width=0.6\textwidth]{digging_into_code/strings/russian_UTF16.png}
\caption{Hiew: UTF-16LE}
\end{figure}

То что бросается в глаза --- это то что символы перемежаются ромбиками (который имеет код 4).
Действительно, в уникоде кирилличные символы находятся в четвертом блоке
\footnote{\href{http://go.yurichev.com/17003}{wikipedia}}.
Таким образом, все кирилличные символы в UTF-16LE находятся в диапазоне \TT{0x400-0x4FF}.

Вернемся к примеру, где одна и та же строка написана разными языками.
Здесь посмотрим в кодировке UTF-16LE.

% FIXME: cut it
\begin{figure}[H]
\centering
\myincludegraphics{digging_into_code/strings/multilang_sampler_UTF16.png}
\caption{FAR: UTF-16LE}
\end{figure}

Здесь мы также видим \ac{BOM} в самом начале.
Все латинские буквы перемежаются с нулевыми байтом.
Некоторые буквы с диакритическими знаками (венгерский и исландский языки) также подчеркнуты красным.

% subsection:
\subsubsection{Base64}
\myindex{Base64}

Кодировка base64 очень популярна в тех случаях, когда нужно передать двоичные данные как текстовую строку.

По сути, этот алгоритм кодирует 3 двоичных байта в 4 печатаемых символа:
все 26 букв латинского алфавита (в обоих регистрах), цифры, знак плюса (\q{+}) и слэша (\q{/}),
в итоге это получается 64 символа.

Одна отличительная особенность строк в формате base64, это то что они часто (хотя и не всегда) заканчиваются
одним или двумя символами знака равенства (\q{=}) для выравнивания (\gls{padding}), например:

\begin{lstlisting}
AVjbbVSVfcUMu1xvjaMgjNtueRwBbxnyJw8dpGnLW8ZW8aKG3v4Y0icuQT+qEJAp9lAOuWs=
\end{lstlisting}

\begin{lstlisting}
WVjbbVSVfcUMu1xvjaMgjNtueRwBbxnyJw8dpGnLW8ZW8aKG3v4Y0icuQT+qEJAp9lAOuQ==
\end{lstlisting}

Так что знак равенства (\q{=}) никогда не встречается в середине строк закодированных в base64.

Теперь пример кодирования вручную.
Попробуем закодировать шестнадцатеричные байты 0x00, 0x11, 0x22, 0x33 в строку в формате base64:

\lstinputlisting{digging_into_code/strings/base64_ex.sh}

Запишем все 4 байта в двоичной форме, затем перегруппируем их в 6-битные группы:

\begin{lstlisting}
|  00  ||  11  ||  22  ||  33  ||      ||      |
00000000000100010010001000110011????????????????
| A  || B  || E  || i  || M  || w  || =  || =  |
\end{lstlisting}

Первые три байта (0x00, 0x11, 0x22) можно закодировать в 4 base64-символа (``ABEi''),
но последний (0x33) --- нельзя,
так что он кодируется используя два символа (``Mw'') и \gls{padding}-символ (``='')
добавляется дважды, чтобы выровнять последнюю группу до 4-х символов.
Таким образом, длина всех корректных base64-строк всегда делится на 4.

\myindex{XML}
\myindex{PGP}
Base64 часто используется когда нужно закодировать двоичные данные в XML.
PGP-ключи и подписи в ``armored''-виде (т.е., в текстовом) кодируются в base64.

Некоторые люди пытаются использовать base64 для обфускации строк:
\url{http://blog.sec-consult.com/2016/01/deliberately-hidden-backdoor-account-in.html}
\footnote{\url{http://archive.is/nDCas}}.

\myindex{base64scanner}
Существуют утилиты для сканирования бинарных файлов и нахождения в них base64-строк.
Одна из них это base64scanner\footnote{\url{https://github.com/dennis714/base64scanner}}.

\myindex{UseNet}
\myindex{FidoNet}
\myindex{Uuencoding}
Еще одна система кодирования, которая была более популярна в UseNet и FidoNet это Uuencoding.
Ее возможности почти такие же, но разница с base64 в том, что имя файла также передавалось в заголовке.

\myindex{Tor}
\myindex{base32}
Кстати, у base64 есть близкий родственник - base32, алфавит которого стоит из ~10 цифр и ~26 латинских букв.
Известное многим использование, это onion-адрес
\footnote{\url{https://trac.torproject.org/projects/tor/wiki/doc/HiddenServiceNames}},
например: \url{http://3g2upl4pq6kufc4m.onion/}.
\ac{URL} не может содержать и строчные и заглавные латинские буквы, очевидно, по этой причине разработчики Tor
использовали base32.





\subsection{Поиск строк в бинарном фале}

\epigraph{Actually, the best form of Unix documentation is frequently running the
\textbf{strings} command over a program’s object code. Using \textbf{strings}, you can get
a complete list of the program’s hard-coded file name, environment variables,
undocumented options, obscure error messages, and so forth.}{The Unix-Haters Handbook}

\myindex{UNIX!strings}
Стандартная утилита в UNIX \IT{strings} это самый простой способ увидеть строки в файле.
Например, это строки найденные в исполняемом файле sshd из OpenSSH 7.2:

\lstinputlisting{digging_into_code/sshd_strings.txt}

Тут опции, сообщения об ошибках, пути к файлам, импортируемые модули, функции, и еще какие-то странные строки (ключи?)
Присутствует также нечитаемый шум---иногда в x86-коде бывают целые куски состоящие из печатаемых ASCII-символов,
вплоть до ~8 символов.

Конечно, OpenSSH это опенсорсная программа.
Но изучение читаемых строк внутри некоторого неизвестного бинарного файла это зачастую самый первый шаг в анализе.
\myindex{UNIX!grep}

Также можно использовать \IT{grep}.

\myindex{Hiew}
\myindex{Sysinternals}
В Hiew есть такая же возможность (Alt-F6), также как и в Sysinternals ProcessMonitor.

\subsection{Сообщения об ошибках и отладочные сообщения}

Очень сильно помогают отладочные сообщения, если они имеются. В некотором смысле, отладочные сообщения, 
это отчет о том, что сейчас происходит в программе.
Зачастую, это \printf-подобные функции, 
которые пишут куда-нибудь в лог, а бывает так что и не пишут ничего, но вызовы остались, так как эта сборка --- не
отладочная, а \IT{release}.

\myindex{\oracle}
Если в отладочных сообщениях дампятся значения некоторых локальных или глобальных переменных, 
это тоже может помочь, как минимум, узнать их имена. 
Например, в \oracle одна из таких функций: \TT{ksdwrt()}.

Осмысленные текстовые строки вообще очень сильно могут помочь. 
Дизассемблер \IDA может сразу указать, из какой функции и из какого её места используется эта строка. 
Встречаются и смешные случаи
\footnote{\href{http://go.yurichev.com/17223}{blog.yurichev.com}}.

Сообщения об ошибках также могут помочь найти то что нужно. 
В \oracle сигнализация об ошибках проходит при помощи вызова некоторой группы функций. \\
Тут еще немного об этом: \href{http://go.yurichev.com/17224}{blog.yurichev.com}.

\myindex{Error messages}
Можно довольно быстро найти, какие функции сообщают о каких ошибках, и при каких условиях.

Это, кстати, одна из причин, почему в защите софта от копирования, 
бывает так, что сообщение об ошибке заменяется 
невнятным кодом или номером ошибки. Мало кому приятно, если взломщик быстро поймет, 
из-за чего именно срабатывает защита от копирования, просто по сообщению об ошибке.

Один из примеров шифрования сообщений об ошибке, здесь: \myref{examples_SCO}.

\subsection{Подозрительные магические строки}

Некоторые магические строки, используемые в бэкдорах выглядят очень подозрительно.
Например, в домашних роутерах TP-Link WR740 был бэкдор
\footnote{\url{http://sekurak.pl/tp-link-httptftp-backdoor/}, на русском: \url{http://m.habrahabr.ru/post/172799/}}.
Бэкдор активировался при посещении следующего URL:\\
\url{http://192.168.0.1/userRpmNatDebugRpm26525557/start_art.html}.\\
Действительно, строка \q{userRpmNatDebugRpm26525557} присутствует в прошивке.

Эту строку нельзя было нагуглить до распространения информации о бэкдоре.

Вы не найдете ничего такого ни в одном \ac{RFC}.

Вы не найдете ни одного алгоритма, который бы использовал такие странные последовательности байт.

И это не выглядит как сообщение об ошибке, или отладочное сообщение.

Так что проверить использование подобных странных строк --- это всегда хорошая идея.
\\
\myindex{base64}
Иногда такие строки кодируются при помощи 
base64\footnote{Например, бэкдор в кабельном модеме Arris: 
\url{http://www.securitylab.ru/analytics/461497.php}}.
Так что неплохая идея их всех декодировать и затем просмотреть глазами, пусть даже бегло.
\\
\myindex{Security through obscurity}

Более точно, такой метод сокрытия бэкдоров называется \q{security through obscurity} (безопасность через
запутанность).

\section{Calls to assert()}
\myindex{\CStandardLibrary!assert()}

Sometimes the presence of the \TT{assert()} macro is useful too: 
commonly this macro leaves source file name, line number and condition in the code.

The most useful information is contained in the assert's condition, we can deduce variable names or structure field
names from it. Another useful piece of information are the file names---we can try to deduce what type of
code is there.
Also it is possible to recognize well-known open-source libraries by the file names.

\lstinputlisting[caption=Example of informative assert() calls]{digging_into_code/assert_examples.lst}

It is advisable to \q{google} both the conditions and file names, which can lead us to an open-source library.
For example, if we \q{google} \q{sp->lzw\_nbits <= BITS\_MAX}, this predictably 
gives us some open-source code that's related to the LZW compression.

\section{Constants}

Humans, including programmers, often use round numbers like 10, 100, 1000, 
in real life as well as in the code.

The practicing reverse engineer usually know them well in hexadecimal representation:
10=0xA, 100=0x64, 1000=0x3E8, 10000=0x2710.

The constants \TT{0xAAAAAAAA} (0b10101010101010101010101010101010) and \\
\TT{0x55555555} (0b01010101010101010101010101010101)  are also popular---those
are composed of alternating bits.

That may help to distinguish some signal from a signal where all bits are turned on (0b1111 \dots) or off (0b0000 \dots).
For example, the \TT{0x55AA} constant
is used at least in the boot sector, \ac{MBR}, 
and in the \ac{ROM} of IBM-compatible extension cards.

Some algorithms, especially cryptographical ones use distinct constants, which are easy to find
in code using \IDA.

\myindex{MD5}
\newcommand{\URLMD}{http://go.yurichev.com/17111}

For example, the MD5\footnote{\href{\URLMD}{wikipedia}} algorithm initializes its own internal variables like this:

\begin{verbatim}
var int h0 := 0x67452301
var int h1 := 0xEFCDAB89
var int h2 := 0x98BADCFE
var int h3 := 0x10325476
\end{verbatim}

If you find these four constants used in the code in a row, it is highly probable that this function is related to MD5.

\par Another example are the CRC16/CRC32 algorithms, 
whose calculation algorithms often use precomputed tables like this one:

\begin{lstlisting}[caption=linux/lib/crc16.c,style=customc]
/** CRC table for the CRC-16. The poly is 0x8005 (x^16 + x^15 + x^2 + 1) */
u16 const crc16_table[256] = {
	0x0000, 0xC0C1, 0xC181, 0x0140, 0xC301, 0x03C0, 0x0280, 0xC241,
	0xC601, 0x06C0, 0x0780, 0xC741, 0x0500, 0xC5C1, 0xC481, 0x0440,
	0xCC01, 0x0CC0, 0x0D80, 0xCD41, 0x0F00, 0xCFC1, 0xCE81, 0x0E40,
	...
\end{lstlisting}

See also the precomputed table for CRC32: \myref{sec:CRC32}.

In tableless CRC algorithms well-known polynomials are used, for example, 0xEDB88320 for CRC32.

\subsection{Magic numbers}
\label{magic_numbers}

\newcommand{\FNURLMAGIC}{\footnote{\href{http://go.yurichev.com/17112}{wikipedia}}}

A lot of file formats define a standard file header where a \IT{magic number(s)}\FNURLMAGIC{} is used, single one or even several.

\myindex{MS-DOS}

For example, all Win32 and MS-DOS executables start with the two characters \q{MZ}\footnote{\href{http://go.yurichev.com/17113}{wikipedia}}.

\myindex{MIDI}

At the beginning of a MIDI file the \q{MThd} signature must be present. 
If we have a program which uses MIDI files for something,
it's very likely that it must check the file for validity by checking at least the first 4 bytes.

This could be done like this:
(\IT{buf} points to the beginning of the loaded file in memory)

\begin{lstlisting}[style=customasmx86]
cmp [buf], 0x6468544D ; "MThd"
jnz _error_not_a_MIDI_file
\end{lstlisting}

\myindex{\CStandardLibrary!memcmp()}
\myindex{x86!\Instructions!CMPSB}

\dots or by calling a function for comparing memory blocks like \TT{memcmp()} or any other equivalent code
up to a \TT{CMPSB} (\myref{REPE_CMPSx}) instruction.

When you find such point you already can say where the loading of the MIDI file starts,
also, we could see the location
of the buffer with the contents of the MIDI file, what is used from the buffer, and how.

\subsubsection{Dates}

\myindex{UFS2}
\myindex{FreeBSD}
\myindex{HASP}

Often, one may encounter number like \TT{0x19870116}, which is clearly looks like a date (year 1987, 1th month (January), 16th day).
This may be someone's birthday (a programmer, his/her relative, child), or some other important date.
The date may also be written in a reverse order, like \TT{0x16011987}.
American-style dates are also popular, like \TT{0x01161987}.

Well-known example is \TT{0x19540119} (magic number used in UFS2 superblock structure), which is a birthday of Marshall Kirk McKusick, prominent FreeBSD contributor.

\myindex{Stuxnet}
Stuxnet uses the number ``19790509'' (not as 32-bit number, but as string, though), and this led to speculation
that the malware is connected to Israel
\footnote{This is a date of execution of Habib Elghanian, persian jew.}

Also, numbers like those are very popular in amateur-grade cryptography, for example, excerpt from the \IT{secret function} internals from HASP3 dongle
\footnote{\url{https://web.archive.org/web/20160311231616/http://www.woodmann.com/fravia/bayu3.htm}}:

\begin{lstlisting}[style=customc]
void xor_pwd(void) 
{ 
	int i; 
	
	pwd^=0x09071966;
	for(i=0;i<8;i++) 
	{ 
		al_buf[i]= pwd & 7; pwd = pwd >> 3; 
	} 
};

void emulate_func2(unsigned short seed)
{ 
	int i, j; 
	for(i=0;i<8;i++) 
	{ 
		ch[i] = 0; 
		
		for(j=0;j<8;j++)
		{ 
			seed *= 0x1989; 
			seed += 5; 
			ch[i] |= (tab[(seed>>9)&0x3f]) << (7-j); 
		}
	} 
}
\end{lstlisting}

\subsubsection{DHCP}

This applies to network protocols as well.
For example, the DHCP protocol's network packets contains the so-called \IT{magic cookie}: \TT{0x63538263}.
Any code that generates DHCP packets somewhere must embed this constant into the packet.
If we find it in the code we may find where this happens and, not only that.
Any program which can receive DHCP packet must verify the \IT{magic cookie}, comparing it with the constant.

For example, let's take the dhcpcore.dll file from Windows 7 x64 and search for the constant.
And we can find it, twice:
it seems that the constant is used in two functions with descriptive names\\
\TT{DhcpExtractOptionsForValidation()} and \TT{DhcpExtractFullOptions()}:

\begin{lstlisting}[caption=dhcpcore.dll (Windows 7 x64),style=customasmx86]
.rdata:000007FF6483CBE8 dword_7FF6483CBE8 dd 63538263h          ; DATA XREF: DhcpExtractOptionsForValidation+79
.rdata:000007FF6483CBEC dword_7FF6483CBEC dd 63538263h          ; DATA XREF: DhcpExtractFullOptions+97
\end{lstlisting}

And here are the places where these constants are accessed:

\begin{lstlisting}[caption=dhcpcore.dll (Windows 7 x64),style=customasmx86]
.text:000007FF6480875F  mov     eax, [rsi]
.text:000007FF64808761  cmp     eax, cs:dword_7FF6483CBE8
.text:000007FF64808767  jnz     loc_7FF64817179
\end{lstlisting}

And:

\begin{lstlisting}[caption=dhcpcore.dll (Windows 7 x64),style=customasmx86]
.text:000007FF648082C7  mov     eax, [r12]
.text:000007FF648082CB  cmp     eax, cs:dword_7FF6483CBEC
.text:000007FF648082D1  jnz     loc_7FF648173AF
\end{lstlisting}

\subsection{Specific constants}

Sometimes, there is a specific constant for some type of code.
For example, the author once dug into a code, where number 12 was encountered suspiciously often.
Size of many arrays is 12, or multiple of 12 (24, etc).
As it turned out, that code takes 12-channel audio file at input and process it.

And vice versa: for example, if a program works with text field which has length of 120 bytes,
there has to be a constant 120 or 119 somewhere in the code.
If UTF-16 is used, then $2 \cdot 120$.
If a code works with network packets of fixed size, it's good idea to search for this constant in the code as well.

This is also true for amateur cryptography (license keys, etc).
If encrypted block has size of $n$ bytes, you may want to try to find occurences of this number throughout the code.
Also, if you see a piece of code which is been repeated $n$ times in loop during execution,
this may be encryption/decryption routine.

\subsection{Searching for constants}

It is easy in \IDA: Alt-B or Alt-I.
\myindex{binary grep}
And for searching for a constant in a big pile of files, or for searching in non-executable files,
there is a small utility called \IT{binary grep}\footnote{\BGREPURL}.


\section{Finding the right instructions}

If the program is utilizing FPU instructions and there are very few of them in the code,
one can try to check each one manually with a debugger.

\par For example, we may be interested how Microsoft Excel calculates the formulae entered by user.
For example, the division operation.

\myindex{\GrepUsage}
\myindex{x86!\Instructions!FDIV}

If we load excel.exe (from Office 2010) version 14.0.4756.1000 into \IDA, make a full listing
and to find every \FDIV instruction (except the ones which use constants as a second 
operand---obviously, they do not suit us):

\begin{lstlisting}
cat EXCEL.lst | grep fdiv | grep -v dbl_ > EXCEL.fdiv
\end{lstlisting}

\dots then we see that there are 144 of them.

\par We can enter a string like \TT{=(1/3)} in Excel and check each instruction.

\myindex{tracer}

\par By checking each instruction in a debugger or \tracer
(one may check 4 instruction at a time),
we get lucky and the sought-for instruction is just the 14th:

\begin{lstlisting}
.text:3011E919 DC 33          fdiv    qword ptr [ebx]
\end{lstlisting}

\begin{lstlisting}
PID=13944|TID=28744|(0) 0x2f64e919 (Excel.exe!BASE+0x11e919)
EAX=0x02088006 EBX=0x02088018 ECX=0x00000001 EDX=0x00000001
ESI=0x02088000 EDI=0x00544804 EBP=0x0274FA3C ESP=0x0274F9F8
EIP=0x2F64E919
FLAGS=PF IF
FPU ControlWord=IC RC=NEAR PC=64bits PM UM OM ZM DM IM 
FPU StatusWord=
FPU ST(0): 1.000000
\end{lstlisting}

\ST{0} holds the first argument (1) and second one is in \TT{[EBX]}.\\
\\
\myindex{x86!\Instructions!FDIV}

The instruction after \FDIV (\TT{FSTP}) writes the result in memory:\\

\begin{lstlisting}
.text:3011E91B DD 1E          fstp    qword ptr [esi]
\end{lstlisting}

If we set a breakpoint on it, we can see the result:

\begin{lstlisting}
PID=32852|TID=36488|(0) 0x2f40e91b (Excel.exe!BASE+0x11e91b)
EAX=0x00598006 EBX=0x00598018 ECX=0x00000001 EDX=0x00000001
ESI=0x00598000 EDI=0x00294804 EBP=0x026CF93C ESP=0x026CF8F8
EIP=0x2F40E91B
FLAGS=PF IF
FPU ControlWord=IC RC=NEAR PC=64bits PM UM OM ZM DM IM 
FPU StatusWord=C1 P 
FPU ST(0): 0.333333
\end{lstlisting}

Also as a practical joke, we can modify it on the fly:

\begin{lstlisting}
tracer -l:excel.exe bpx=excel.exe!BASE+0x11E91B,set(st0,666)
\end{lstlisting}

\begin{lstlisting}
PID=36540|TID=24056|(0) 0x2f40e91b (Excel.exe!BASE+0x11e91b)
EAX=0x00680006 EBX=0x00680018 ECX=0x00000001 EDX=0x00000001
ESI=0x00680000 EDI=0x00395404 EBP=0x0290FD9C ESP=0x0290FD58
EIP=0x2F40E91B
FLAGS=PF IF
FPU ControlWord=IC RC=NEAR PC=64bits PM UM OM ZM DM IM 
FPU StatusWord=C1 P 
FPU ST(0): 0.333333
Set ST0 register to 666.000000
\end{lstlisting}

Excel shows 666 in the cell, finally convincing us that we have found the right point.

\begin{figure}[H]
\centering
\includegraphics[width=0.6\textwidth]{digging_into_code/Excel_prank.png}
\caption{The practical joke worked}
\end{figure}

If we try the same Excel version, but in x64,
we will find only 12 \FDIV instructions there,
and the one we looking for is the third one.

\begin{lstlisting}
tracer.exe -l:excel.exe bpx=excel.exe!BASE+0x1B7FCC,set(st0,666)
\end{lstlisting}

\myindex{x86!\Instructions!DIVSD}

It seems that a lot of division operations of \Tfloat and \Tdouble types, were replaced by the compiler with SSE instructions
like \TT{DIVSD} (\TT{DIVSD} is present 268 times in total).

\section{Suspicious code patterns}

\subsection{XOR instructions}
\myindex{x86!\Instructions!XOR}

Instructions like \TT{XOR op, op} (for example, \TT{XOR EAX, EAX}) 
are usually used for setting the register value
to zero, but if the operands are different, the \q{exclusive or} operation
is executed.

This operation is rare in common programming, but widespread in cryptography,
including amateur one.
It's especially suspicious if the
second operand is a big number.

This may point to encrypting/decrypting, checksum computing, etc.\\
\\

One exception to this observation worth noting is the \q{canary} (\myref{subsec:BO_protection}). 
Its generation and checking are often done using the \XOR instruction. \\
\\
\myindex{AWK}

This AWK script can be used for processing \IDA{} listing (.lst) files:

\begin{lstlisting}
gawk -e '$2=="xor" { tmp=substr($3, 0, length($3)-1); if (tmp!=$4) if($4!="esp") if ($4!="ebp") { print $1, $2, tmp, ",", $4 } }' filename.lst
\end{lstlisting}

It is also worth noting that this kind of script can also match incorrectly disassembled code 
(\myref{sec:incorrectly_disasmed_code}).

\subsection{Hand-written assembly code}

\myindex{Function prologue}
\myindex{Function epilogue}
\myindex{x86!\Instructions!LOOP}
\myindex{x86!\Instructions!RCL}

Modern compilers do not emit the \TT{LOOP} and \TT{RCL} instructions.
On the other hand, these instructions are well-known to coders who like to code directly in assembly language.
If you spot these, it can be said that there is a high probability that this fragment of code was hand-written.
Such instructions are marked as (M) in the instructions list in this appendix: \myref{sec:x86_instructions}.

\par

Also the function prologue/epilogue are not commonly present in hand-written assembly.
\par

Commonly there is no fixed system for passing arguments to functions in the hand-written code.

\par
Example from the Windows 2003 kernel 
(ntoskrnl.exe file):

\begin{lstlisting}[style=customasm]
MultiplyTest proc near               ; CODE XREF: Get386Stepping
             xor     cx, cx
loc_620555:                          ; CODE XREF: MultiplyTest+E
             push    cx
             call    Multiply
             pop     cx
             jb      short locret_620563
             loop    loc_620555
             clc
locret_620563:                       ; CODE XREF: MultiplyTest+C
             retn
MultiplyTest endp

Multiply     proc near               ; CODE XREF: MultiplyTest+5
             mov     ecx, 81h
             mov     eax, 417A000h
             mul     ecx
             cmp     edx, 2
             stc
             jnz     short locret_62057F
             cmp     eax, 0FE7A000h
             stc
             jnz     short locret_62057F
             clc
locret_62057F:                       ; CODE XREF: Multiply+10
                                     ; Multiply+18
             retn
Multiply     endp
\end{lstlisting}

Indeed, if we look in the 
\ac{WRK} v1.2 source code, this code
can be found easily in file \\
\IT{WRK-v1.2\textbackslash{}base\textbackslash{}ntos\textbackslash{}ke\textbackslash{}i386\textbackslash{}cpu.asm}.

\section{Using magic numbers while tracing}

Often, our main goal is to understand how the program uses a value that has been either read from file or received via network. 
The manual tracing of a value is often a very labor-intensive task. One of the simplest techniques for this (although not 100\% reliable) 
is to use your own \IT{magic number}.

This resembles X-ray computed tomography is some sense: a radiocontrast agent is injected into the patient's blood,
which is then used to improve the visibility of the patient's internal structure in to the X-rays.
It is well known how the blood of healthy humans
percolates in the kidneys and if the agent is in the blood, it can be easily seen on tomography, how blood is percolating,
and are there any stones or tumors.

We can take a 32-bit number like \TT{0x0badf00d}, or someone's birth date like \TT{0x11101979}
and write this 4-byte number to some point in a file used by the program we investigate.

\myindex{\GrepUsage}
\myindex{tracer}

Then, while tracing this program with \tracer in \IT{code coverage} mode, with the help of \IT{grep}
or just by searching in the text file (of tracing results), we can easily see where the value has been used and how.

Example 
of \IT{grepable} \tracer results in \IT{cc} mode:

\begin{lstlisting}
0x150bf66 (_kziaia+0x14), e=       1 [MOV EBX, [EBP+8]] [EBP+8]=0xf59c934 
0x150bf69 (_kziaia+0x17), e=       1 [MOV EDX, [69AEB08h]] [69AEB08h]=0 
0x150bf6f (_kziaia+0x1d), e=       1 [FS: MOV EAX, [2Ch]] 
0x150bf75 (_kziaia+0x23), e=       1 [MOV ECX, [EAX+EDX*4]] [EAX+EDX*4]=0xf1ac360 
0x150bf78 (_kziaia+0x26), e=       1 [MOV [EBP-4], ECX] ECX=0xf1ac360 
\end{lstlisting}
% TODO: good example!

This can be used for network packets as well.
It is important for the \IT{magic number} to be unique and not to be present in the program's code.

\newcommand{\DOSBOXURL}{\href{http://go.yurichev.com/17222}{blog.yurichev.com}}

\myindex{DosBox}
\myindex{MS-DOS}
Aside of 
the \tracer, DosBox (MS-DOS emulator) in heavydebug mode
is able to write information about all registers' states for each executed instruction of the program to a plain text file\footnote{See also my 
blog post about this DosBox feature: \DOSBOXURL{}}, so this technique may be useful for DOS programs as well.



\section{Other things}

\subsection{General idea}

Ein Reverse Engineer sollte versuchen so oft wie Möglich in den Schuhen des Programmierers zu laufen.
Um ihren/seinen Standpunkt zu betrachten uns sich selbst zu Fragen wie man einen Task in spezifischen Fällen lösen würde.

% A reverse engineer should try to be in programmer's shoes as often as possible. 
% To take his/her viewpoint and ask himself, how would one solve some task the specific case.

\subsection{Order of functions in binary code}

Sämmtliche Funktionen die in einer einzelenen .c oder .cpp-file gefunden werden, werden zu den entsprechenden Object Datein (.o) kompiliert. 
Später, fügt der Linker alle Objektdatein die er braucht zusammen, ohne die Reihenfolge oder die Funktionen in Ihnen zu verändern. 
Als eine Konsequenz, ergibt sich daraus wenn man zwei oder mehr aufeinanderfolgende Funktionen sieht, bedeutet dass das sie in der 
gleichen Source code Datei plaziert waren ( Ausser natürlich man bewegt sich an der Genze zwischen zwei Datein. ).  Das bedetet
das diese Funktionen etwas gemeinsam haben, das sie aus dem gleichen \ac{API} Level stammen oder aus der gleichen library, etc.

% All functions located in a single .c or .cpp-file are compiled into corresponding object (.o) file.
% Later, linker puts all object files it needs together, not changing order or functions in them.
% As a consequence, if you see two or more consecutive functions, it means, that they were placed together
% in a single source code file (unless you're on border of two object files, of course.)
% This means these functions have something in common, that they are from the same \ac{API} level, from same library, etc.

\subsection{Tiny functions}

Sehr kleine oder leere Funktionen  (\myref{empty_func})
oder Funtkionen die nur ``true'' (1) oder ``false'' (0) (\myref{ret_val_func}) sind weit verbreitet,
und fast jeder ordentlicher Kompiler tendiert dazu nur solche Funktionen in den resultierenden ausführbaren code zu stecken,
sogar wenn es mehrere gleiche Funktionen im Source Code bereits gibt. 
Also, wann immer man solche kleinen Funtkionen sieht die z.B nur aus \TT{mov eax, 1 / ret} bestehen und von mehreren 
Orten aus referenziert werden (und aufgerufen werden können), und scheinbar keine Verbindung zu einander haben, dann 
ist das warscheinlich das Ergebnis einer Optimisierung. 

% Tiny functions like empty functions (\myref{empty_func})
% or function which returns just ``true'' (1) or ``false'' (0) (\myref{ret_val_func}) are very common,
% and almost all decent compiler tends put only one such function into resulting executable code even if there was several
% similar functions in source code.
% So, whenever you see a tiny function consisting just of \TT{mov eax, 1 / ret}
% which is referenced (and can be called) from many places,
% which are seems unconnected to each other, this may be a result of such optimization.%

\subsection{\Cpp}

\ac{RTTI}~(\myref{RTTI})-data ist vielleicht auch nützlich für  \Cpp klassen identifikation.
% \ac{RTTI}~(\myref{RTTI})-data may be also useful for \Cpp class identification.

% sections
% DE Translation also needed for subfiles here
% TODO move section...

\subsection{Some binary file patterns}

All examples here were prepared on the Windows with active code page 437
\footnote{\url{https://en.wikipedia.org/wiki/Code_page_437}} in console.
Binary files internally may look visually different if another code page is set.

\clearpage
\subsubsection{Arrays}

Sometimes, we can clearly spot an array of 16/32/64-bit values visually, in hex editor.

Here is an example of array of 16-bit values.
We see that the first byte in pair is 7 or 8, and the second looks random:

\begin{figure}[H]
\centering
\myincludegraphics{digging_into_code/binary/16bit_array.png}
\caption{FAR: array of 16-bit values}
\end{figure}

I used a file containing 12-channel signal digitized using 16-bit \ac{ADC}.

\clearpage
\myindex{MIPS}
\par And here is an example of very typical MIPS code.

As we may recall, every MIPS (and also ARM in ARM mode or ARM64) instruction has size of 32 bits (or 4 bytes), 
so such code is array of 32-bit values.

By looking at this screenshot, we may see some kind of pattern.

Vertical red lines are added for clarity:

\begin{figure}[H]
\centering
\myincludegraphics{digging_into_code/binary/typical_MIPS_code.png}
\caption{Hiew: very typical MIPS code}
\end{figure}

Another example of such pattern here is book: 
\myref{Oracle_SYM_files_example}.

\clearpage
\subsubsection{Sparse files}

This is sparse file with data scattered amidst almost empty file.
Each space character here is in fact zero byte (which is looks like space).
This is a file to program FPGA (Altera Stratix GX device).
Of course, files like these can be compressed easily, but formats like this one are very popular in scientific and engineering software where efficient access is important while compactness is not.

\begin{figure}[H]
\centering
\myincludegraphics{digging_into_code/binary/sparse_FPGA.png}
\caption{FAR: Sparse file}
\end{figure}

\clearpage
\subsubsection{Compressed file}

% FIXME \ref{} ->
This file is just some compressed archive.
It has relatively high entropy and visually looks just chaotic.
This is how compressed and/or encrypted files looks like.

\begin{figure}[H]
\centering
\myincludegraphics{digging_into_code/binary/compressed.png}
\caption{FAR: Compressed file}
\end{figure}

\clearpage
\subsubsection{\ac{CDFS}}

\ac{OS} installations are usually distributed as ISO files which are copies of CD/DVD discs.
Filesystem used is named \ac{CDFS}, here is you see file names mixed with some additional data.
This can be file sizes, pointers to another directories, file attributes, etc.
This is how typical filesystems may look internally.

\begin{figure}[H]
\centering
\myincludegraphics{digging_into_code/binary/cdfs.png}
\caption{FAR: ISO file: Ubuntu 15 installation \ac{CD}}
\end{figure}

\clearpage
\subsubsection{32-bit x86 executable code}

This is how 32-bit x86 executable code looks like.
It has not very high entropy, because some bytes occurred more often than others.

\begin{figure}[H]
\centering
\myincludegraphics{digging_into_code/binary/x86_32.png}
\caption{FAR: Executable 32-bit x86 code}
\end{figure}

% TODO: Read more about x86 statistics: \ref{}. % FIXME blog post about decryption...

\clearpage
\subsubsection{BMP graphics files}

% TODO: bitmap, bit, group of bits...

BMP files are not compressed, so each byte (or group of bytes) describes each pixel.
I've found this picture somewhere inside my installed Windows 8.1:

\begin{figure}[H]
\centering
\myincludegraphicsSmall{digging_into_code/binary/bmp.png}
\caption{Example picture}
\end{figure}

You see that this picture has some pixels which unlikely can be compressed very good (around center), 
but there are long one-color lines at top and bottom.
Indeed, lines like these also looks as lines during viewing the file:

\begin{figure}[H]
\centering
\myincludegraphics{digging_into_code/binary/bmp_FAR.png}
\caption{BMP file fragment}
\end{figure}


% FIXME comparison!
\subsection{Memory \q{snapshots} comparing}
\label{snapshots_comparing}

The technique of the straightforward comparison of two memory snapshots in order to see changes was often used to hack
8-bit computer games and for hacking \q{high score} files.

For example, if you had a loaded game on an 8-bit computer (there isn't much memory on these, but the game usually
consumes even less memory) and you know that you have now, let's say, 100 bullets, you can do a \q{snapshot}
of all memory and back it up to some place. Then shoot once, the bullet count goes to 99, do a second \q{snapshot}
and then compare both: it must be a byte somewhere which has been 100 at the beginning, and now it is 99.

Considering the fact that these 8-bit games were often written in assembly language and such variables were global,
it can be said for sure which address in memory has holding the bullet count. If you searched for all references to the
address in the disassembled game code, it was not very hard to find a piece of code \glslink{decrement}{decrementing} the bullet count,
then to write a \gls{NOP} instruction there, or a couple of \gls{NOP}-s, 
and then have a game with 100 bullets forever.
\myindex{BASIC!POKE}
Games on these 8-bit computers were commonly loaded at the constant
address, also, there were not much different versions of each game (commonly just one version was popular for a long span of time),
so enthusiastic gamers knew which bytes must be overwritten (using the BASIC's instruction \gls{POKE}) at which address in
order to hack it. This led to \q{cheat} lists that contained \gls{POKE} instructions, published in magazines related to
8-bit games. See also: \href{http://go.yurichev.com/17114}{wikipedia}.

\myindex{MS-DOS}

Likewise, it is easy to modify \q{high score} files, this does not work with just 8-bit games. Notice 
your score count and back up the file somewhere. When the \q{high score} count gets different, just compare the two files,
it can even be done with the DOS utility FC\footnote{MS-DOS utility for comparing binary files} (\q{high score} files
are often in binary form).

There will be a point where a couple of bytes are different and it is easy to see which ones are
holding the score number.
However, game developers are fully aware of such tricks and may defend the program against it.

Somewhat similar example in this book is: \myref{Millenium_DOS_game}.

% TODO: пример с какой-то простой игрушкой?

\subsubsection{Windows registry}

It is also possible to compare the Windows registry before and after a program installation.

It is a very popular method of finding which registry elements are used by the program.
Perhaps, this is the reason why the \q{windows registry cleaner} shareware is so popular.

\subsubsection{Blink-comparator}

Comparison of files or memory snapshots remind us blink-comparator
\footnote{\url{http://go.yurichev.com/17348}}:
a device used by astronomers in past, intended to find moving celestial objects.

Blink-comparator allows to switch quickly between two photographies shot in different time,
so astronomer would spot the difference visually.

By the way, Pluto was discovered by blink-comparator in 1930.


