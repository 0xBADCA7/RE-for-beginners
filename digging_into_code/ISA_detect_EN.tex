\section{\ac{ISA} detection}
\label{ISA_detect}

Often, you can deal with a binary file for an unknown \ac{ISA}.
Perhaps, easiest way to detect \ac{ISA} is to try various ones in IDA, objdump or another disassembler.

To achieve this, one should understand a difference between incorrectly disassembled code and correctly one.

% subsection:
\renewcommand{\CURPATH}{digging_into_code/incorrect_disassembly}
\subsection{Incorrectly disassembled code}
\label{sec:incorrectly_disasmed_code}

Practicing reverse engineers often have to deal with incorrectly disassembled code.

\subsubsection{Disassembling from an incorrect start (x86)}

Unlike ARM and MIPS (where any instruction has a length of 2 or 4 bytes), x86 instructions have variable size,
so any disassembler that starts in the middle of a x86 instruction may produce incorrect results.

As an example:

\lstinputlisting[style=customasmx86]{\CURPATH/x86_wrong_start_EN.asm}

There are incorrectly disassembled instructions at the beginning, but eventually the disassembler gets on the right 
track.

\subsubsection{How does random noise looks disassembled?}

Common properties that can be spotted easily are:

\begin{itemize}
\item Unusually big instruction dispersion.
\myindex{x86!\Instructions!PUSH}
\myindex{x86!\Instructions!MOV}
\myindex{x86!\Instructions!CALL}
\myindex{x86!\Instructions!IN}
\myindex{x86!\Instructions!OUT}
The most frequent x86 instructions are \PUSH{}, \MOV{}, \CALL{}, 
but here we see
instructions from all instruction groups: \ac{FPU} instructions, \INS{IN}/\INS{OUT} instructions, rare and system instructions,
everything mixed up in one single place.

\item Big and random values, offsets and immediates.

\item Jumps having incorrect offsets, often jumping in the middle of another instructions.
\end{itemize}

\lstinputlisting[caption=\randomNoise{} (x86),style=customasmx86]{\CURPATH/x86.asm}

\myindex{x86-64}
\lstinputlisting[caption=\randomNoise{} (x86-64),style=customasmx86]{\CURPATH/x64.asm}

\myindex{ARM}
\lstinputlisting[caption=\randomNoise{} (ARM (\ARMMode)),style=customasmARM]{\CURPATH/ARM.asm}

\lstinputlisting[caption=\randomNoise{} (ARM (\ThumbMode)),style=customasmARM]{\CURPATH/ARM_thumb.asm}

\myindex{MIPS}
\lstinputlisting[caption=\randomNoise{} (MIPS little endian),style=customasmMIPS]{\CURPATH/MIPS.asm}

It is also important to keep in mind that 
cleverly constructed unpacking and decryption code 
(including self-modifying) may looks like noise as well, but still execute correctly.
% TODO таких примеров тоже бы добавить



\subsection{Correctly disassembled code}
\label{correctly_disasmed_code}

Each \ac{ISA} has a dozen of a most used instructions, all the rest are used much less often.

As of x86, it is interesting to know that the fact that function calls (\PUSH/\CALL/\ADD) and \MOV
instructions are the most frequently executed pieces of code in almost all
programs we use.
In other words, \ac{CPU} is very busy passing information between levels of abstractions, or,
it can be said, it's very busy switching between these levels.
Regardless type of \ac{ISA}.
This is a cost of splitting problems into several levels of abstractions (so humans could work with them easier).

