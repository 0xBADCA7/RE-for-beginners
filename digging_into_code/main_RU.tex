\chapter{Поиск в коде того что нужно}

Современное ПО, в общем-то, минимализмом не отличается.

\myindex{\Cpp!STL}
Но не потому, что программисты слишком много пишут, 
а потому что к исполняемым файлам обыкновенно прикомпилируют все подряд библиотеки. 
Если бы все вспомогательные библиотеки всегда выносили во внешние DLL, мир был бы иным.
(Еще одна причина для Си++ --- \ac{STL} и прочие библиотеки шаблонов.)

\newcommand{\FOOTNOTEBOOST}{\footnote{\url{http://go.yurichev.com/17036}}}
\newcommand{\FOOTNOTELIBPNG}{\footnote{\url{http://go.yurichev.com/17037}}}

Таким образом, очень полезно сразу понимать, какая функция из стандартной библиотеки или 
более-менее известной (как Boost\FOOTNOTEBOOST, libpng\FOOTNOTELIBPNG), 
а какая --- имеет отношение к тому что мы пытаемся найти в коде.

Переписывать весь код на \CCpp, чтобы разобраться в нем, безусловно, не имеет никакого смысла.

Одна из важных задач reverse engineer-а это быстрый поиск в коде того что собственно его интересует.

\myindex{\GrepUsage}
Дизассемблер \IDA позволяет делать поиск как минимум строк, последовательностей байт, констант.
Можно даже сделать экспорт кода в текстовый файл .lst или .asm и затем натравить на него \TT{grep}, \TT{awk}, итд.

Когда вы пытаетесь понять, что делает тот или иной код, это запросто может быть какая-то 
опенсорсная библиотека вроде libpng. Поэтому, когда находите константы, или текстовые строки, которые 
выглядят явно знакомыми, всегда полезно их \IT{погуглить}.
А если вы найдете искомый опенсорсный проект где это используется, 
то тогда будет достаточно будет просто сравнить вашу функцию с ней. 
Это решит часть проблем.

К примеру, если программа использует какие-то XML-файлы, первым шагом может быть
установление, какая именно XML-библиотека для этого используется, ведь часто используется какая-то
стандартная (или очень известная) вместо самодельной.

\myindex{SAP}
\myindex{Windows!PDB}
К примеру, автор этих строк однажды пытался разобраться как происходит компрессия/декомпрессия сетевых пакетов в SAP 6.0. 
Это очень большая программа, но к ней идет подробный .\gls{PDB}-файл с отладочной информацией, и это очень удобно. 
Он в конце концов пришел к тому что одна из функций декомпрессирующая пакеты называется CsDecomprLZC(). 
Не сильно раздумывая, он решил погуглить и оказалось, что функция с таким же названием имеется в MaxDB
(это опен-сорсный проект SAP) \footnote{Больше об этом в соответствующей секции~(\myref{sec:SAPGUI})}.

\url{http://www.google.com/search?q=CsDecomprLZC}

Каково же было мое удивление, когда оказалось, что в MaxDB используется точно такой же алгоритм, 
скорее всего, с таким же исходником.

\input{digging_into_code/identification/exec_RU}
% binary files might be also here

\section{Связь с внешним миром (на уровне функции)}
Очень желательно следить за аргументами ф-ции и возвращаемыми значениями, в отладчике или \ac{DBI}.
Например, автор этих строк однажды пытался понять значение некоторой очень запутанной ф-ции, которая, как потом оказалось,
была неверно реализованной пузырьковой сортировкой\footnote{\url{https://yurichev.com/blog/weird_sort/}}.
(Она работала правильно, но медленнее.)
В то же время, наблюдение за входами и выходами этой ф-ции помогает мгновенно понять, что она делает.

Часто, когда вы видите деление через умножение (\myref{sec:divisionbymult}),
но забыли все детали о том, как оно работает, вы можете просто наблюдать за входом и выходом, и так быстро найти делитель.

% sections:
\input{digging_into_code/communication_win32_RU}
\section{Строки}
\label{sec:digging_strings}

\subsection{Текстовые строки}

\subsubsection{\CCpp}

\label{C_strings}
Обычные строки в Си заканчиваются нулем (\ac{ASCIIZ}-строки).

Причина, почему формат строки в Си именно такой (оканчивающийся нулем) вероятно историческая.
В [Dennis M. Ritchie, \IT{The Evolution of the Unix Time-sharing System}, (1979)]
мы можем прочитать:

\begin{framed}
\begin{quotation}
A minor difference was that the unit of I/O was the word, not the byte, because the PDP-7 was a word-addressed
machine. In practice this meant merely that all programs dealing with character streams ignored null
characters, because null was used to pad a file to an even number of characters.
\end{quotation}
\end{framed}

\myindex{Hiew}
Строки выглядят в Hiew или FAR Manager точно так же:

\begin{lstlisting}[style=customc]
int main()
{
	printf ("Hello, world!\n");
};
\end{lstlisting}

\begin{figure}[H]
\centering
\includegraphics[width=0.6\textwidth]{digging_into_code/strings/C-string.png}
\caption{Hiew}
\end{figure}

% FIXME видно \n в конце, потом пробел

\subsubsection{Borland Delphi}
\myindex{Pascal}
\myindex{Borland Delphi}
Когда кодируются строки в Pascal и Delphi, сама строка предваряется 8-битным или 32-битным значением, в котором закодирована длина строки.

Например:

\begin{lstlisting}[caption=Delphi,style=customasmx86]
CODE:00518AC8                 dd 19h
CODE:00518ACC aLoading___Plea db 'Loading... , please wait.',0

...

CODE:00518AFC                 dd 10h
CODE:00518B00 aPreparingRun__ db 'Preparing run...',0
\end{lstlisting}

\subsubsection{Unicode}

\myindex{Unicode}
Нередко уникодом называют все способы передачи символов, когда символ занимает 2 байта или 16 бит.
Это распространенная терминологическая ошибка.
Уникод --- это стандарт, присваивающий номер каждому символу многих письменностей мира, но не описывающий
способ кодирования.

\myindex{UTF-8}
\myindex{UTF-16LE}
Наиболее популярные способы кодирования: 
UTF-8 (наиболее часто используется в Интернете и *NIX-системах) и UTF-16LE (используется в Windows).

\myparagraph{UTF-8}

\myindex{UTF-8}
UTF-8 это один из очень удачных способов кодирования символов.
Все символы латиницы кодируются так же, как и в ASCII-кодировке, а символы, выходящие за пределы
ASCII-7-таблицы, кодируются несколькими байтами.
0 кодируется, как и прежде, нулевыми байтом, так что все стандартные
функции Си продолжают работать с UTF-8-строками так же как и с обычными строками.

Посмотрим, как символы из разных языков кодируются в UTF-8 и как это выглядит в FAR, в кодировке 437

\footnote{Пример и переводы на разные языки были взяты здесь: 
\url{http://go.yurichev.com/17304}}:

\begin{figure}[H]
\centering
\includegraphics[width=0.6\textwidth]{digging_into_code/strings/multilang_sampler.png}
\end{figure}

% FIXME: cut it
\begin{figure}[H]
\centering
\myincludegraphics{digging_into_code/strings/multilang_sampler_UTF8.png}
\caption{FAR: UTF-8}
\end{figure}

Видно, что строка на английском языке выглядит точно так же, как и в ASCII-кодировке.
В венгерском языке используются латиница плюс латинские буквы с диакритическими знаками.
Здесь видно, что эти буквы кодируются несколькими байтами, они подчеркнуты красным.
То же самое с исландским и польским языками.
В самом начале имеется также символ валюты \q{Евро}, который кодируется тремя байтами.
Остальные системы письма здесь никак не связаны с латиницей.
По крайней мере о русском, арабском, иврите и хинди мы можем сказать, что здесь видны повторяющиеся
байты, что не удивительно, ведь, обычно буквы из одной и той же системы письменности расположены в одной
или нескольких таблицах уникода, поэтому часто их коды начинаются с одних и тех же цифр.

В самом начале, перед строкой \q{How much?}, видны три байта, которые на самом деле \ac{BOM}.
\ac{BOM} показывает, какой способ кодирования будет сейчас использоваться.

\myparagraph{UTF-16LE}

\myindex{UTF-16LE}
\myindex{Windows!Win32}
Многие функции win32 в Windows имеют суффикс \TT{-A} и \TT{-W}.
Первые функции работают с обычными строками, вторые с UTF-16LE-строками (\IT{wide}).
Во втором случае, каждый символ обычно хранится в 16-битной переменной типа \IT{short}.

Cтроки с латинскими буквами выглядят в Hiew или FAR как перемежающиеся с нулевыми байтами:

\begin{lstlisting}[style=customc]
int wmain()
{
	wprintf (L"Hello, world!\n");
};
\end{lstlisting}

\begin{figure}[H]
\centering
\includegraphics[width=0.6\textwidth]{digging_into_code/strings/UTF16-string.png}
\caption{Hiew}
\end{figure}

Подобное можно часто увидеть в системных файлах \gls{Windows NT}:

\begin{figure}[H]
\centering
\includegraphics[width=0.6\textwidth]{digging_into_code/strings/ntoskrnl_UTF16.png}
\caption{Hiew}
\end{figure}

\myindex{IDA}
В \IDA, уникодом называется именно строки с символами, занимающими 2 байта:

\begin{lstlisting}[style=customasmx86]
.data:0040E000 aHelloWorld:
.data:0040E000                 unicode 0, <Hello, world!>
.data:0040E000                 dw 0Ah, 0
\end{lstlisting}

Вот как может выглядеть строка на русском языке (\q{И снова здравствуйте!}) закодированная в UTF-16LE:

\begin{figure}[H]
\centering
\includegraphics[width=0.6\textwidth]{digging_into_code/strings/russian_UTF16.png}
\caption{Hiew: UTF-16LE}
\end{figure}

То что бросается в глаза --- это то что символы перемежаются ромбиками (который имеет код 4).
Действительно, в уникоде кирилличные символы находятся в четвертом блоке
\footnote{\href{http://go.yurichev.com/17003}{wikipedia}}.
Таким образом, все кирилличные символы в UTF-16LE находятся в диапазоне \TT{0x400-0x4FF}.

Вернемся к примеру, где одна и та же строка написана разными языками.
Здесь посмотрим в кодировке UTF-16LE.

% FIXME: cut it
\begin{figure}[H]
\centering
\myincludegraphics{digging_into_code/strings/multilang_sampler_UTF16.png}
\caption{FAR: UTF-16LE}
\end{figure}

Здесь мы также видим \ac{BOM} в самом начале.
Все латинские буквы перемежаются с нулевыми байтом.
Некоторые буквы с диакритическими знаками (венгерский и исландский языки) также подчеркнуты красным.

% subsection:
\subsubsection{Base64}
\myindex{Base64}

Кодировка base64 очень популярна в тех случаях, когда нужно передать двоичные данные как текстовую строку.

По сути, этот алгоритм кодирует 3 двоичных байта в 4 печатаемых символа:
все 26 букв латинского алфавита (в обоих регистрах), цифры, знак плюса (\q{+}) и слэша (\q{/}),
в итоге это получается 64 символа.

Одна отличительная особенность строк в формате base64, это то что они часто (хотя и не всегда) заканчиваются
одним или двумя символами знака равенства (\q{=}) для выравнивания (\gls{padding}), например:

\begin{lstlisting}
AVjbbVSVfcUMu1xvjaMgjNtueRwBbxnyJw8dpGnLW8ZW8aKG3v4Y0icuQT+qEJAp9lAOuWs=
\end{lstlisting}

\begin{lstlisting}
WVjbbVSVfcUMu1xvjaMgjNtueRwBbxnyJw8dpGnLW8ZW8aKG3v4Y0icuQT+qEJAp9lAOuQ==
\end{lstlisting}

Так что знак равенства (\q{=}) никогда не встречается в середине строк закодированных в base64.

Теперь пример кодирования вручную.
Попробуем закодировать шестнадцатеричные байты 0x00, 0x11, 0x22, 0x33 в строку в формате base64:

\lstinputlisting{digging_into_code/strings/base64_ex.sh}

Запишем все 4 байта в двоичной форме, затем перегруппируем их в 6-битные группы:

\begin{lstlisting}
|  00  ||  11  ||  22  ||  33  ||      ||      |
00000000000100010010001000110011????????????????
| A  || B  || E  || i  || M  || w  || =  || =  |
\end{lstlisting}

Первые три байта (0x00, 0x11, 0x22) можно закодировать в 4 base64-символа (``ABEi''),
но последний (0x33) --- нельзя,
так что он кодируется используя два символа (``Mw'') и \gls{padding}-символ (``='')
добавляется дважды, чтобы выровнять последнюю группу до 4-х символов.
Таким образом, длина всех корректных base64-строк всегда делится на 4.

\myindex{XML}
\myindex{PGP}
Base64 часто используется когда нужно закодировать двоичные данные в XML.
PGP-ключи и подписи в ``armored''-виде (т.е., в текстовом) кодируются в base64.

Некоторые люди пытаются использовать base64 для обфускации строк:
\url{http://blog.sec-consult.com/2016/01/deliberately-hidden-backdoor-account-in.html}
\footnote{\url{http://archive.is/nDCas}}.

\myindex{base64scanner}
Существуют утилиты для сканирования бинарных файлов и нахождения в них base64-строк.
Одна из них это base64scanner\footnote{\url{https://github.com/dennis714/base64scanner}}.

\myindex{UseNet}
\myindex{FidoNet}
\myindex{Uuencoding}
Еще одна система кодирования, которая была более популярна в UseNet и FidoNet это Uuencoding.
Ее возможности почти такие же, но разница с base64 в том, что имя файла также передавалось в заголовке.

\myindex{Tor}
\myindex{base32}
Кстати, у base64 есть близкий родственник - base32, алфавит которого стоит из ~10 цифр и ~26 латинских букв.
Известное многим использование, это onion-адрес
\footnote{\url{https://trac.torproject.org/projects/tor/wiki/doc/HiddenServiceNames}},
например: \url{http://3g2upl4pq6kufc4m.onion/}.
\ac{URL} не может содержать и строчные и заглавные латинские буквы, очевидно, по этой причине разработчики Tor
использовали base32.





\subsection{Поиск строк в бинарном фале}

\epigraph{Actually, the best form of Unix documentation is frequently running the
\textbf{strings} command over a program’s object code. Using \textbf{strings}, you can get
a complete list of the program’s hard-coded file name, environment variables,
undocumented options, obscure error messages, and so forth.}{The Unix-Haters Handbook}

\myindex{UNIX!strings}
Стандартная утилита в UNIX \IT{strings} это самый простой способ увидеть строки в файле.
Например, это строки найденные в исполняемом файле sshd из OpenSSH 7.2:

\lstinputlisting{digging_into_code/sshd_strings.txt}

Тут опции, сообщения об ошибках, пути к файлам, импортируемые модули, функции, и еще какие-то странные строки (ключи?)
Присутствует также нечитаемый шум---иногда в x86-коде бывают целые куски состоящие из печатаемых ASCII-символов,
вплоть до ~8 символов.

Конечно, OpenSSH это опенсорсная программа.
Но изучение читаемых строк внутри некоторого неизвестного бинарного файла это зачастую самый первый шаг в анализе.
\myindex{UNIX!grep}

Также можно использовать \IT{grep}.

\myindex{Hiew}
\myindex{Sysinternals}
В Hiew есть такая же возможность (Alt-F6), также как и в Sysinternals ProcessMonitor.

\subsection{Сообщения об ошибках и отладочные сообщения}

Очень сильно помогают отладочные сообщения, если они имеются. В некотором смысле, отладочные сообщения, 
это отчет о том, что сейчас происходит в программе.
Зачастую, это \printf-подобные функции, 
которые пишут куда-нибудь в лог, а бывает так что и не пишут ничего, но вызовы остались, так как эта сборка --- не
отладочная, а \IT{release}.

\myindex{\oracle}
Если в отладочных сообщениях дампятся значения некоторых локальных или глобальных переменных, 
это тоже может помочь, как минимум, узнать их имена. 
Например, в \oracle одна из таких функций: \TT{ksdwrt()}.

Осмысленные текстовые строки вообще очень сильно могут помочь. 
Дизассемблер \IDA может сразу указать, из какой функции и из какого её места используется эта строка. 
Встречаются и смешные случаи
\footnote{\href{http://go.yurichev.com/17223}{blog.yurichev.com}}.

Сообщения об ошибках также могут помочь найти то что нужно. 
В \oracle сигнализация об ошибках проходит при помощи вызова некоторой группы функций. \\
Тут еще немного об этом: \href{http://go.yurichev.com/17224}{blog.yurichev.com}.

\myindex{Error messages}
Можно довольно быстро найти, какие функции сообщают о каких ошибках, и при каких условиях.

Это, кстати, одна из причин, почему в защите софта от копирования, 
бывает так, что сообщение об ошибке заменяется 
невнятным кодом или номером ошибки. Мало кому приятно, если взломщик быстро поймет, 
из-за чего именно срабатывает защита от копирования, просто по сообщению об ошибке.

Один из примеров шифрования сообщений об ошибке, здесь: \myref{examples_SCO}.

\subsection{Подозрительные магические строки}

Некоторые магические строки, используемые в бэкдорах выглядят очень подозрительно.
Например, в домашних роутерах TP-Link WR740 был бэкдор
\footnote{\url{http://sekurak.pl/tp-link-httptftp-backdoor/}, на русском: \url{http://m.habrahabr.ru/post/172799/}}.
Бэкдор активировался при посещении следующего URL:\\
\url{http://192.168.0.1/userRpmNatDebugRpm26525557/start_art.html}.\\
Действительно, строка \q{userRpmNatDebugRpm26525557} присутствует в прошивке.

Эту строку нельзя было нагуглить до распространения информации о бэкдоре.

Вы не найдете ничего такого ни в одном \ac{RFC}.

Вы не найдете ни одного алгоритма, который бы использовал такие странные последовательности байт.

И это не выглядит как сообщение об ошибке, или отладочное сообщение.

Так что проверить использование подобных странных строк --- это всегда хорошая идея.
\\
\myindex{base64}
Иногда такие строки кодируются при помощи 
base64\footnote{Например, бэкдор в кабельном модеме Arris: 
\url{http://www.securitylab.ru/analytics/461497.php}}.
Так что неплохая идея их всех декодировать и затем просмотреть глазами, пусть даже бегло.
\\
\myindex{Security through obscurity}

Более точно, такой метод сокрытия бэкдоров называется \q{security through obscurity} (безопасность через
запутанность).

\section{Вызовы assert()}
\myindex{\CStandardLibrary!assert()}
Может также помочь наличие \TT{assert()} в коде: обычно этот макрос оставляет название файла-исходника, 
номер строки, и условие.

Наиболее полезная информация содержится в assert-условии, по нему можно судить по именам переменных
или именам полей структур. Другая полезная информация --- это имена файлов, по их именам можно попытаться
предположить, что там за код. Также, по именам файлов можно опознать какую-либо очень известную опен-сорсную
библиотеку.

\lstinputlisting[caption=Пример информативных вызовов assert(),style=customasm]{digging_into_code/assert_examples.lst}

Полезно \q{гуглить} и условия и имена файлов, это может вывести вас к опен-сорсной бибилотеке.
Например, если \q{погуглить} \q{sp->lzw\_nbits <= BITS\_MAX}, 
это вполне предсказуемо выводит на опенсорсный код, что-то связанное с LZW-компрессией.


\section{Константы}

Люди, включая программистов, часто используют круглые числа вроде 10, 100, 1000, в т.ч. и в коде.

Практикующие реверсеры, обычно, хорошо знают их в шестнадцатеричном представлении:
0b10=0xA, 0b100=0x64, 0b1000=0x3E8, 0b10000=0x2710.

Иногда попадаются константы \TT{0xAAAAAAAA} \\
(0b10101010101010101010101010101010) и
\TT{0x55555555} (0b01010101010101010101010101010101) --- это чередующиеся биты.
Это помогает отличить некоторый сигнал от сигнала где все биты включены (0b1111 \dots) или выключены (0b0000 \dots).

Например, константа \TT{0x55AA} используется как минимум в бут-секторе, \ac{MBR}, 
и в \ac{ROM} плат-расширений IBM-компьютеров.

Некоторые алгоритмы, особенно криптографические, используют хорошо различимые константы, 
которые при помощи \IDA легко находить в коде.

\myindex{MD5}
\newcommand{\URLMD}{http://go.yurichev.com/17110}

Например, алгоритм MD5\footnote{\href{\URLMD}{wikipedia}} инициализирует свои внутренние переменные так:

\begin{verbatim}
var int h0 := 0x67452301
var int h1 := 0xEFCDAB89
var int h2 := 0x98BADCFE
var int h3 := 0x10325476
\end{verbatim}

Если в коде найти использование этих четырех констант подряд --- очень высокая вероятность что эта функция имеет отношение к MD5.

\par
Еще такой пример это алгоритмы CRC16/CRC32, часто, алгоритмы вычисления контрольной суммы по CRC 
используют заранее заполненные таблицы, вроде:

\begin{lstlisting}[caption=linux/lib/crc16.c]
/** CRC table for the CRC-16. The poly is 0x8005 (x^16 + x^15 + x^2 + 1) */
u16 const crc16_table[256] = {
	0x0000, 0xC0C1, 0xC181, 0x0140, 0xC301, 0x03C0, 0x0280, 0xC241,
	0xC601, 0x06C0, 0x0780, 0xC741, 0x0500, 0xC5C1, 0xC481, 0x0440,
	0xCC01, 0x0CC0, 0x0D80, 0xCD41, 0x0F00, 0xCFC1, 0xCE81, 0x0E40,
	...
\end{lstlisting}

См. также таблицу CRC32: \myref{sec:CRC32}.

В бестабличных алгоритмах CRC используются хорошо известные полиномы, например 0xEDB88320 для CRC32.

\subsection{Магические числа}
\label{magic_numbers}

\newcommand{\FNURLMAGIC}{\footnote{\href{http://go.yurichev.com/17112}{wikipedia}}}

Немало форматов файлов определяет стандартный заголовок файла где используются \IT{магическое число} (magic number)\FNURLMAGIC{}, один или даже несколько.

\myindex{MS-DOS}
Скажем, все исполняемые файлы для Win32 и MS-DOS начинаются с двух символов \q{MZ}\footnote{\href{http://go.yurichev.com/17113}{wikipedia}}.

\myindex{MIDI}
В начале MIDI-файла должно быть \q{MThd}. Если у нас есть использующая для чего-нибудь MIDI-файлы программа,
наверняка она будет проверять MIDI-файлы на правильность хотя бы проверяя первые 4 байта.

Это можно сделать при помощи:
(\IT{buf} указывает на начало загруженного в память файла)

\begin{lstlisting}
cmp [buf], 0x6468544D ; "MThd"
jnz _error_not_a_MIDI_file
\end{lstlisting}

\myindex{\CStandardLibrary!memcmp()}
\myindex{x86!\Instructions!CMPSB}
\dots либо вызвав функцию сравнения блоков памяти \TT{memcmp()} или любой аналогичный код, 
вплоть до инструкции \TT{CMPSB} (\myref{REPE_CMPSx}).

Найдя такое место мы получаем как минимум информацию о том, где начинается загрузка MIDI-файла, во-вторых, 
мы можем увидеть где располагается буфер с содержимым файла, и что еще оттуда берется, и как используется.

\subsubsection{Даты}

\myindex{UFS2}
\myindex{FreeBSD}
\myindex{HASP}

Часто, можно встретить число вроде \TT{0x19870116}, которое явно выглядит как дата (1987-й год, 1-й месяц (январь), 16-й день).
Это может быть чей-то день рождения (программиста, его/её родственника, ребенка), либо какая-то другая важная дата.
Дата может быть записана и в другом порядке, например \TT{0x16011987}.

Известный пример это \TT{0x19540119} (магическое число используемое в структуре суперблока UFS2), это день рождения Маршала Кирка МакКузика, видного разработчика FreeBSD.

\myindex{Stuxnet}
В Stuxnet используется число ``19790509'' (хотя и не как 32-битное число, а как строка), и это привело к догадкам,
что этот зловред связан с Израелем\footnote{Это дата казни персидского еврея Habib Elghanian-а}.

Также, числа вроде таких очень популярны в любительской криптографии, например, это отрывок из внутренностей \IT{секретной функции} донглы HASP3
\footnote{\url{https://web.archive.org/web/20160311231616/http://www.woodmann.com/fravia/bayu3.htm}}:

\begin{lstlisting}
void xor_pwd(void) 
{ 
	int i; 
	
	pwd^=0x09071966;
	for(i=0;i<8;i++) 
	{ 
		al_buf[i]= pwd & 7; pwd = pwd >> 3; 
	} 
};

void emulate_func2(unsigned short seed)
{ 
	int i, j; 
	for(i=0;i<8;i++) 
	{ 
		ch[i] = 0; 
		
		for(j=0;j<8;j++)
		{ 
			seed *= 0x1989; 
			seed += 5; 
			ch[i] |= (tab[(seed>>9)&0x3f]) << (7-j); 
		}
	} 
}
\end{lstlisting}

\subsubsection{DHCP}

Это касается также и сетевых протоколов. 
Например, сетевые пакеты протокола DHCP содержат так называемую \IT{magic cookie}: \TT{0x63538263}. 
Какой-либо код, генерирующий пакеты по протоколу DHCP где-то и как-то должен внедрять в пакет также и эту константу. 
Найдя её в коде мы сможем найти место где происходит это и не только это. 
Любая программа, получающая DHCP-пакеты, должна где-то как-то проверять \IT{magic cookie}, 
сравнивая это поле пакета с константой.

Например, берем файл dhcpcore.dll из Windows 7 x64 и ищем эту константу. 
И находим, два раза: оказывается, эта константа используется в функциях с красноречивыми названиями \\
\TT{DhcpExtractOptionsForValidation()} и \TT{DhcpExtractFullOptions()}:

\begin{lstlisting}[caption=dhcpcore.dll (Windows 7 x64)]
.rdata:000007FF6483CBE8 dword_7FF6483CBE8 dd 63538263h          ; DATA XREF: DhcpExtractOptionsForValidation+79
.rdata:000007FF6483CBEC dword_7FF6483CBEC dd 63538263h          ; DATA XREF: DhcpExtractFullOptions+97
\end{lstlisting}

А вот те места в функциях где происходит обращение к константам:

\begin{lstlisting}[caption=dhcpcore.dll (Windows 7 x64)]
.text:000007FF6480875F  mov     eax, [rsi]
.text:000007FF64808761  cmp     eax, cs:dword_7FF6483CBE8
.text:000007FF64808767  jnz     loc_7FF64817179
\end{lstlisting}

И:

\begin{lstlisting}[caption=dhcpcore.dll (Windows 7 x64)]
.text:000007FF648082C7  mov     eax, [r12]
.text:000007FF648082CB  cmp     eax, cs:dword_7FF6483CBEC
.text:000007FF648082D1  jnz     loc_7FF648173AF
\end{lstlisting}

\subsection{Специфические константы}

Иногда, бывают какие-то специфические константы для некоторого типа кода.
Например, однажды автор сих строк пытался разобраться с кодом, где подозрительно часто встречалось число 12.
Размеры многих массивов также были 12, или кратные 12 (24, итд).
Оказалось, этот код брал на вход 12-канальный аудиофайл и обрабатывал его.

И наоборот: например, если программа работает с текстовым полем длиной 120 байт, значит где-то в коде должна
быть константа 120, или 119.
Если используется UTF-16, то тогда $2 \cdot 120$.
Если код работает с сетевыми пакетами фиксированной длины, то хорошо бы и такую константу поискать в коде.

\subsection{Поиск констант}

В \IDA это очень просто, Alt-B или Alt-I.

\myindex{binary grep}
А для поиска константы в большом количестве файлов, либо для поиска их в неисполняемых файлах, имеется небольшая утилита
\IT{binary grep}\footnote{\BGREPURL}.


\section{Поиск нужных инструкций}

Если программа использует инструкции сопроцессора, и их не очень много, 
то можно попробовать вручную проверить отладчиком какую-то из них.

\par К примеру, нас может заинтересовать, при помощи чего Microsoft Excel считает 
результаты формул, введенных пользователем. Например, операция деления.

\myindex{\GrepUsage}
\myindex{x86!\Instructions!FDIV}
Если загрузить excel.exe (из Office 2010) версии 14.0.4756.1000 в \IDA, затем сделать полный листинг 
и найти все инструкции \FDIV (но кроме тех, которые в качестве второго операнда используют константы --- они, 
очевидно, не подходят нам):

\begin{lstlisting}
cat EXCEL.lst | grep fdiv | grep -v dbl_ > EXCEL.fdiv
\end{lstlisting}

\dots то окажется, что их всего 144.

\par Мы можем вводить в Excel строку вроде \TT{=(1/3)} и проверить все эти инструкции.

\myindex{tracer}
\par Проверяя каждую инструкцию в отладчике или \tracer 
(проверять эти инструкции можно по 4 за раз), 
окажется, что нам везет и срабатывает всего лишь 14-я по счету:

\begin{lstlisting}
.text:3011E919 DC 33          fdiv    qword ptr [ebx]
\end{lstlisting}

\begin{lstlisting}
PID=13944|TID=28744|(0) 0x2f64e919 (Excel.exe!BASE+0x11e919)
EAX=0x02088006 EBX=0x02088018 ECX=0x00000001 EDX=0x00000001
ESI=0x02088000 EDI=0x00544804 EBP=0x0274FA3C ESP=0x0274F9F8
EIP=0x2F64E919
FLAGS=PF IF
FPU ControlWord=IC RC=NEAR PC=64bits PM UM OM ZM DM IM 
FPU StatusWord=
FPU ST(0): 1.000000
\end{lstlisting}

В \ST{0} содержится первый аргумент (1), второй содержится в \TT{[EBX]}.\\
\\
\myindex{x86!\Instructions!FDIV}
Следующая за \FDIV инструкция (\TT{FSTP}) записывает результат в память: \\

\begin{lstlisting}
.text:3011E91B DD 1E          fstp    qword ptr [esi]
\end{lstlisting}

Если поставить breakpoint на ней, то мы можем видеть результат:

\begin{lstlisting}
PID=32852|TID=36488|(0) 0x2f40e91b (Excel.exe!BASE+0x11e91b)
EAX=0x00598006 EBX=0x00598018 ECX=0x00000001 EDX=0x00000001
ESI=0x00598000 EDI=0x00294804 EBP=0x026CF93C ESP=0x026CF8F8
EIP=0x2F40E91B
FLAGS=PF IF
FPU ControlWord=IC RC=NEAR PC=64bits PM UM OM ZM DM IM 
FPU StatusWord=C1 P 
FPU ST(0): 0.333333
\end{lstlisting}

А также, в рамках пранка\footnote{practical joke}, модифицировать его на лету:

\begin{lstlisting}
tracer -l:excel.exe bpx=excel.exe!BASE+0x11E91B,set(st0,666)
\end{lstlisting}

\begin{lstlisting}
PID=36540|TID=24056|(0) 0x2f40e91b (Excel.exe!BASE+0x11e91b)
EAX=0x00680006 EBX=0x00680018 ECX=0x00000001 EDX=0x00000001
ESI=0x00680000 EDI=0x00395404 EBP=0x0290FD9C ESP=0x0290FD58
EIP=0x2F40E91B
FLAGS=PF IF
FPU ControlWord=IC RC=NEAR PC=64bits PM UM OM ZM DM IM 
FPU StatusWord=C1 P 
FPU ST(0): 0.333333
Set ST0 register to 666.000000
\end{lstlisting}

Excel показывает в этой ячейке 666, что окончательно убеждает нас в том, что мы нашли нужное место.

\begin{figure}[H]
\centering
\includegraphics[width=0.6\textwidth]{digging_into_code/Excel_prank.png}
\caption{Пранк сработал}
\end{figure}

Если попробовать ту же версию Excel, только x64, то окажется что там инструкций \FDIV всего 12, 
причем нужная нам --- третья по счету.

\begin{lstlisting}
tracer.exe -l:excel.exe bpx=excel.exe!BASE+0x1B7FCC,set(st0,666)
\end{lstlisting}

\myindex{x86!\Instructions!DIVSD}
Видимо, все дело в том, что много операций деления переменных типов \Tfloat и \Tdouble 
компилятор заменил на SSE-инструкции вроде \TT{DIVSD}, 
коих здесь теперь действительно много (\TT{DIVSD} присутствует в количестве 268 инструкций).


\section{Подозрительные паттерны кода}

\subsection{Инструкции XOR}
\myindex{x86!\Instructions!XOR}

Инструкции вроде \TT{XOR op, op} (например, \TT{XOR EAX, EAX}) 
обычно используются для обнуления регистра,
однако, если операнды разные, то применяется операция именно \q{исключающего или}.
Эта операция очень редко применяется в обычном программировании, но применяется очень часто в криптографии,
включая любительскую.

Особенно подозрительно, если второй операнд --- это большое число.
Это может указывать на шифрование, вычисление контрольной суммы, итд.  \\
\\
Одно из исключений из этого наблюдения о котором стоит сказать, то, что генерация и проверка значения \q{канарейки}
(\myref{subsec:BO_protection}) часто происходит, используя инструкцию \XOR.  \\
\\
\myindex{AWK}
Этот AWK-скрипт можно использовать для обработки листингов (.lst) созданных \IDA{}:

\begin{lstlisting}
gawk -e '$2=="xor" { tmp=substr($3, 0, length($3)-1); if (tmp!=$4) if($4!="esp") if ($4!="ebp") { print $1, $2, tmp, ",", $4 } }' filename.lst
\end{lstlisting}

Нельзя также забывать,
что если использовать подобный скрипт, то, возможно, он захватит и неверно дизассемблированный
код 
(\myref{sec:incorrectly_disasmed_code}).

\subsection{Вручную написанный код на ассемблере}

\myindex{Function prologue}
\myindex{Function epilogue}
\myindex{x86!\Instructions!LOOP}
\myindex{x86!\Instructions!RCL}
Современные компиляторы не генерируют инструкции \TT{LOOP} и \TT{RCL}. 
С другой стороны, эти инструкции хорошо знакомы кодерам, предпочитающим писать прямо на ассемблере. 
Подобные инструкции отмечены как (M) в списке инструкций в приложении: 
\myref{sec:x86_instructions}.
Если такие инструкции встретились, можно сказать с какой-то вероятностью, что этот фрагмент кода написан вручную.

\par
Также, пролог/эпилог функции обычно не встречается в ассемблерном коде, написанном вручную.

\par
Как правило, в вручную написанном коде, нет никакого четкого метода передачи аргументов в функцию.

\par
Пример из ядра Windows 2003 
(файл ntoskrnl.exe):

\begin{lstlisting}[style=customasmx86]
MultiplyTest proc near               ; CODE XREF: Get386Stepping
             xor     cx, cx
loc_620555:                          ; CODE XREF: MultiplyTest+E
             push    cx
             call    Multiply
             pop     cx
             jb      short locret_620563
             loop    loc_620555
             clc
locret_620563:                       ; CODE XREF: MultiplyTest+C
             retn
MultiplyTest endp

Multiply     proc near               ; CODE XREF: MultiplyTest+5
             mov     ecx, 81h
             mov     eax, 417A000h
             mul     ecx
             cmp     edx, 2
             stc
             jnz     short locret_62057F
             cmp     eax, 0FE7A000h
             stc
             jnz     short locret_62057F
             clc
locret_62057F:                       ; CODE XREF: Multiply+10
                                     ; Multiply+18
             retn
Multiply     endp
\end{lstlisting}

Действительно, если заглянуть в исходные коды 
\ac{WRK} v1.2, данный код можно найти в файле \\
\IT{WRK-v1.2\textbackslash{}base\textbackslash{}ntos\textbackslash{}ke\textbackslash{}i386\textbackslash{}cpu.asm}.

\section{Использование magic numbers для трассировки}

Нередко бывает нужно узнать, как используется то или иное значение, прочитанное из файла либо взятое из пакета,
принятого по сети. Часто, ручное слежение за нужной переменной это трудный процесс. Один из простых методов (хотя и не
полностью надежный на 100\%) это использование вашей собственной \IT{magic number}.

Это чем-то напоминает компьютерную томографию: пациенту перед сканированием вводят в кровь 
рентгеноконтрастный препарат, хорошо отсвечивающий в рентгеновских лучах.
Известно, как кровь нормального человека
расходится, например, по почкам, и если в этой крови будет препарат, то при томографии будет хорошо видно,
достаточно ли хорошо кровь расходится по почкам и нет ли там камней, например, и прочих образований.

Мы можем взять 32-битное число вроде \TT{0x0badf00d}, либо чью-то дату рождения вроде \TT{0x11101979} 
и записать это, занимающее 4 байта число, в какое-либо место файла используемого исследуемой нами программой.

\myindex{\GrepUsage}
\myindex{tracer}
Затем, при трассировке этой программы, в том числе, при помощи \tracer в режиме 
\IT{code coverage}, а затем при помощи
\IT{grep} или простого поиска по текстовому файлу с результатами трассировки, мы можем легко увидеть, в каких местах кода использовалось 
это значение, и как.

Пример результата работы \tracer в режиме \IT{cc}, к которому легко применить утилиту \IT{grep}:

\begin{lstlisting}[style=customasmx86]
0x150bf66 (_kziaia+0x14), e=       1 [MOV EBX, [EBP+8]] [EBP+8]=0xf59c934 
0x150bf69 (_kziaia+0x17), e=       1 [MOV EDX, [69AEB08h]] [69AEB08h]=0 
0x150bf6f (_kziaia+0x1d), e=       1 [FS: MOV EAX, [2Ch]] 
0x150bf75 (_kziaia+0x23), e=       1 [MOV ECX, [EAX+EDX*4]] [EAX+EDX*4]=0xf1ac360 
0x150bf78 (_kziaia+0x26), e=       1 [MOV [EBP-4], ECX] ECX=0xf1ac360 
\end{lstlisting}
% TODO: good example!
Это справедливо также и для сетевых пакетов.
Важно только, чтобы наш \IT{magic number} был как можно более уникален и не присутствовал в самом коде.

\newcommand{\DOSBOXURL}{\href{http://go.yurichev.com/17222}{blog.yurichev.com}}

\myindex{DosBox}
\myindex{MS-DOS}
Помимо \tracer, такой эмулятор MS-DOS как DosBox, в режиме heavydebug, может писать в отчет информацию обо всех
состояниях регистра на каждом шаге исполнения программы\footnote{См. также мой пост в блоге об этой возможности в 
DosBox: \DOSBOXURL{}}, так что этот метод может пригодиться и для исследования программ под DOS.


\section{Циклы}

Когда ваша программа работает с некоторым файлом, или буфером некоторой длины,
внутри кода где-то должен быть цикл с дешифровкой/обработкой.

Вот реальный пример выхода инструмента \tracer.
Был код, загружающий некоторый зашифрованный файл размером 258 байт.
Я могу запустить его с целью подсчета исполнения каждой инструкции (в наше время \ac{DBI} послужила бы куда лучше).
И я могу быстро найти место в коде, которое было исполнено 259/258 раз:

\lstinputlisting{digging_into_code/crypto_loop.txt}

Как потом оказалось, это цикл дешифрования.


% TODO move section...

\subsection{Некоторые паттерны в бинарных файлах}

Все примеры здесь были подготовлены в Windows с активной кодовой страницей 437
\footnote{\url{https://ru.wikipedia.org/wiki/CP437}} в консоли.
Двоичные файлы внутри могут визуально выглядеть иначе если установлена другая кодовая страница.

\clearpage
\subsubsection{Массивы}

Иногда мы можем легко заметить массив 16/32/64-битных значений визуально, в шестнадцатеричном 
редакторе.

Вот пример массива 16-битных значений.
Мы видим, что каждый первый байт в паре всегда равен 7 или 8, а второй выглядит случайным:

\begin{figure}[H]
\centering
\myincludegraphics{digging_into_code/binary/16bit_array.png}
\caption{FAR: массив 16-битных значений}
\end{figure}

Для примера я использовал файл содержащий 12-канальный сигнал оцифрованный при помощи 16-битного \ac{ADC}.

\clearpage
\myindex{MIPS}
\par А вот пример очень типичного MIPS-кода.

Как мы наверное помним, каждая инструкция в MIPS (а также в ARM в режиме ARM, или ARM64) имеет 
длину 32 бита (или 4 байта),
так что такой код это массив 32-битных значений.

Глядя на этот скриншот, можно увидеть некий узор.
Вертикальные красные линии добавлены для ясности:

\begin{figure}[H]
\centering
\myincludegraphics{digging_into_code/binary/typical_MIPS_code.png}
\caption{Hiew: очень типичный код для MIPS}
\end{figure}

Еще пример таких файлов в этой книге: 
\myref{Oracle_SYM_files_example}.

\clearpage
\subsubsection{Разреженные файлы}

Это разреженный файл, в котором данные разбросаны посреди почти пустого файла.
Каждый символ пробела здесь на самом деле нулевой байт (который выглядит как пробел).
Это файл для программирования FPGA (чип Altera Stratix GX).
Конечно, такие файлы легко сжимаются, но подобные форматы очень популярны в научном и инженерном ПО, где быстрый доступ важен, а компактность --- не очень.

\begin{figure}[H]
\centering
\myincludegraphics{digging_into_code/binary/sparse_FPGA.png}
\caption{FAR: Разреженный файл}
\end{figure}

\clearpage
\subsubsection{Сжатый файл}

% FIXME \ref{} ->
Этот файл это просто некий сжатый архив.
Он имеет довольно высокую энтропию и визуально выглядит просто хаотичным.
Так выглядят сжатые и/или зашифрованные файлы.

\begin{figure}[H]
\centering
\myincludegraphics{digging_into_code/binary/compressed.png}
\caption{FAR: Сжатый файл}
\end{figure}

\clearpage
\subsubsection{\ac{CDFS}}

Инсталляции \ac{OS} обычно распространяются в ISO-файлах, которые суть копии CD/DVD-дисков.
Используемая файловая система называется \ac{CDFS}, здесь видны имена файлов и какие-то допольнительные данные.
Это могут быть длины файлов, указатели на другие директории, атрибуты файлов, итд.
Так может выглядеть типичная файловая система внутри.

\begin{figure}[H]
\centering
\myincludegraphics{digging_into_code/binary/cdfs.png}
\caption{FAR: ISO-файл: инсталляционный \ac{CD} Ubuntu 15}
\end{figure}

\clearpage
\subsubsection{32-битный x86 исполняемый код}

Так выглядит 32-битный x86 исполняемый код.
У него не очень высокая энтропия, потому что некоторые байты встречаются чаще других.

\begin{figure}[H]
\centering
\myincludegraphics{digging_into_code/binary/x86_32.png}
\caption{FAR: Исполняемый 32-битных x86 код}
\end{figure}

% TODO: Read more about x86 statistics: \ref{}. % FIXME blog post about decryption...

\clearpage
\subsubsection{Графические BMP-файлы}

% TODO: bitmap, bit, group of bits...

BMP-файлы не сжаты, так что каждый байт (или группа байт) описывают каждый пиксель.
Я нашел эту картинку где-то внутри заинсталлированной Windows 8.1:

\begin{figure}[H]
\centering
\myincludegraphicsSmall{digging_into_code/binary/bmp.png}
\caption{Пример картинки}
\end{figure}

Вы видите, что эта картинка имеет пиксели, которые вряд ли могут быть хорошо сжаты (в районе центра),
но здесь есть длинные одноцветные линии вверху и внизу.
Действительно, линии вроде этих выглядят как линии при просмотре этого файла:

\begin{figure}[H]
\centering
\myincludegraphics{digging_into_code/binary/bmp_FAR.png}
\caption{Фрагмент BMP-файла}
\end{figure}


% FIXME comparison!
\subsection{Сравнение \q{снимков} памяти}
\label{snapshots_comparing}

Метод простого сравнения двух снимков памяти для поиска изменений часто применялся для взлома игр 
на 8-битных компьютерах и взлома файлов с записанными рекордными очками.

К примеру, если вы имеете загруженную игру на 8-битном компьютере (где самой памяти не очень много, но игра
занимает еще меньше), и вы знаете что сейчас у вас, условно, 100 пуль, вы можете сделать \q{снимок} всей
памяти и сохранить где-то. Затем просто стреляете куда угодно, у вас станет 99 пуль, сделать второй \q{снимок},
и затем сравнить эти два снимка: где-то наверняка должен быть байт, который в начале был 100, а затем стал 99.

Если учесть, что игры на тех маломощных домашних компьютерах обычно были написаны на ассемблере и подобные
переменные там были глобальные, то можно с уверенностью сказать, какой адрес в памяти всегда отвечает за количество
пуль. Если поискать в дизассемблированном коде игры все обращения по этому адресу, несложно найти код,
отвечающий за уменьшение пуль и записать туда инструкцию \gls{NOP}
или несколько \gls{NOP}-в, так мы получим игру в которой у игрока всегда будет 100 пуль, например.

\myindex{BASIC!POKE}
А так как игры на тех домашних 8-битных 
компьютерах всегда загружались по одним и тем же адресам, и версий одной игры редко когда было больше одной продолжительное время,
то геймеры-энтузиасты знали, по какому адресу (используя инструкцию языка BASIC \gls{POKE}) что записать после загрузки
игры, чтобы хакнуть её. Это привело к появлению списков \q{читов} состоящих из инструкций \gls{POKE}, публикуемых
в журналах посвященным 8-битным играм. См. также: \href{http://go.yurichev.com/17114}{wikipedia}.

\myindex{MS-DOS}
Точно так же легко модифицировать файлы с сохраненными рекордами (кто сколько очков набрал), впрочем, это может
сработать не только с 8-битными играми. Нужно заметить, какой у вас сейчас рекорд и где-то сохранить файл
с очками. Затем, когда очков станет другое количество, просто сравнить два файла, можно даже
DOS-утилитой FC\footnote{утилита MS-DOS для сравнения двух файлов побайтово} (файлы рекордов, часто, бинарные).

Где-то будут отличаться несколько байт, и легко будет увидеть, какие именно отвечают за количество очков. 
Впрочем, разработчики игр полностью осведомлены о таких хитростях и могут защититься от этого.

В каком-то смысле похожий пример в этой книге здесь: \myref{Millenium_DOS_game}.

% TODO: пример с какой-то простой игрушкой?

\subsubsection{Реестр Windows}

А еще можно вспомнить сравнение реестра Windows до инсталляции программы и после.
Это также популярный метод поиска, какие элементы реестра программа использует.

Наверное это причина, почему так популярны shareware-программы для очистки реестра в Windows.

\subsubsection{Блинк-компаратор}

Сравнение файлов или слепков памяти вообще, немного напоминает блинк-компаратор
\footnote{\url{http://go.yurichev.com/17349}}:
устройство, которое раньше использовали астрономы для поиска движущихся небесных объектов.

Блинк-компаратор позволял быстро переключаться между двух отснятых в разное время кадров,
и астроном мог увидеть разницу визуально.

Кстати, при помощи блинк-компаратора, в 1930 был открыт Плутон.


\section{Определение \ac{ISA}}
\label{ISA_detect}

Часто, вы можете иметь дело с бинарным файлом для неизвестной \ac{ISA}.
Вероятно, простейший способ определить \ac{ISA} это пробовать разные в IDA, objdump или другом дизассемблере.

Чтобы этого достичь, нужно понимать разницу между некорректно дизассемблированным кодом, и корректно дизассемблированным.

% subsection:
\renewcommand{\CURPATH}{digging_into_code/incorrect_disassembly}
\subsection{Неверно дизассемблированный код}
\label{sec:incorrectly_disasmed_code}

Практикующие reverse engineer-ы часто сталкиваются с неверно дизассемблированным кодом.

\subsubsection{Дизассемблирование началось в неверном месте (x86)}

В отличие от ARM и MIPS (где у каждой инструкции длина или 2 или 4 байта), x86-инструкции имеют переменную длину,
так что, любой дизассемблер, начиная работу с середины x86-инструкции, может выдать неверные результаты.

Как пример:

\lstinputlisting[style=customasmx86]{\CURPATH/x86_wrong_start_RU.asm}

В начале мы видим неверно дизассемблированные инструкции, но потом, так или иначе, дизассемблер находит верный след.

\subsubsection{Как выглядят случайные данные в дизассемблированном виде?}

Общее, что можно сразу заметить, это:

\begin{itemize}
\item Необычно большой разброс инструкций.
\myindex{x86!\Instructions!PUSH}
\myindex{x86!\Instructions!MOV}
\myindex{x86!\Instructions!CALL}
\myindex{x86!\Instructions!IN}
\myindex{x86!\Instructions!OUT}
Самые частые x86-инструкции это \PUSH{}, \MOV{}, \CALL{}, 
но здесь мы видим
инструкции из любых групп: \ac{FPU}-инструкции, инструкции \INS{IN}/\INS{OUT}, редкие и системные инструкции, всё друг с другом смешано в одном месте.

\item Большие и случайные значения, смещения, immediates.

\item Переходы с неверными смещениями часто имеют адрес перехода в середину другой инструкции.
\end{itemize}

\lstinputlisting[caption=\randomNoise{} (x86),style=customasmx86]{\CURPATH/x86.asm}

\myindex{x86-64}
\lstinputlisting[caption=\randomNoise{} (x86-64),style=customasmx86]{\CURPATH/x64.asm}

\myindex{ARM}
\lstinputlisting[caption=\randomNoise{} (ARM (\ARMMode)),style=customasmARM]{\CURPATH/ARM.asm}

\lstinputlisting[caption=\randomNoise{} (ARM (\ThumbMode)),style=customasmARM]{\CURPATH/ARM_thumb.asm}

\myindex{MIPS}
\lstinputlisting[caption=\randomNoise{} (MIPS little endian),style=customasmMIPS]{\CURPATH/MIPS.asm}

Также важно помнить, что хитрым образом написанный код для распаковки и дешифровки (включая самомодифицирующийся),
также может выглядеть как случайный шум, тем не менее, он исполняется корректно.

% TODO таких примеров тоже бы добавить


\subsection{Корректино дизассемблированный код}
\label{correctly_disasmed_code}

Каждая \ac{ISA} имеет десяток самых используемых инструкций, остальные используются куда реже.

Интересно знать тот факт, что в x86, инструкции вызовов ф-ций (\PUSH/\CALL/\ADD) и \MOV
это наиболее часто исполняющиеся инструкции в коде почти во всем ПО что мы используем.
Другими словами, \ac{CPU} очень занят передачей информации между уровнями абстракции, или, можно сказать, очень занят
переключением между этими уровнями.
Вне зависимости от \ac{ISA}.
Это цена расслоения программ на разные уровни абстракций (чтобы человеку было легче с ними управляться).



\section{Прочее}

\subsection{Общая идея}

Нужно стараться как можно чаще ставить себя на место программиста и задавать себе вопрос, 
как бы вы сделали ту или иную вещь в этом случае и в этой программе.

\subsection{Порядок функций в бинарном коде}

Все функции расположеные в одном .c или .cpp файле компилируются в соответствующий объектный (.o) файл.
Линкер впоследствии складывает все нужные объектные файлы вместе, не меняя порядок ф-ций в них.
Как следствие, если вы видите в коде две или более идущих подряд ф-ций, то это означает, что и в исходном коде они 
были расположены в одном и том же файле (если только вы не на границе двух объектных файлов, конечно).
Это может означать, что эти ф-ции имеют что-то общее между собой, что они из одного слоя \ac{API}, из одной библиотеки, итд.%

\subsection{Крохотные функции}

Крохотные ф-ции, такие как пустые ф-ции (\myref{empty_func})
или ф-ции возвращающие только ``true'' (1) или ``false'' (0) (\myref{ret_val_func}) очень часто встречаются,
и почти все современные компиляторы, как правило, помещают только одну такую ф-цию в исполняемый код,
даже если в исходном их было много одинаковых.
Так что если вы видите ф-цию состояющую только из \TT{mov eax, 1 / ret}, которая может вызываться из разных мест,
которые, судя по всему, друг с другом никак не связаны, это может быть результат подобной оптимизации.

\subsection{\Cpp}

\ac{RTTI}~(\myref{RTTI})-информация также может быть полезна для идентификации 
классов в \Cpp.

