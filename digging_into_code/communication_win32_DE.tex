\section{Kommunikation mit der Außen Welt (Win32)}

Manchmal reicht es aus die Ein- und Ausgaben einer Funktion zu beobachten um zu verstehen was sie tut.
Auf diese Weise kann man Zeit Sparren.

Datei und Regestry Zugriff:
Für einfache Analysen kann das Prozess Monitor\footnote{\url{http://go.yurichev.com/17301}}
Tool von SysInternals hilfreich sein.

Für grundlegende Analysen des Netzwerk Zugriffs ist Wireshark\footnote{\url{http://go.yurichev.com/17303}} ganz nützlich.
% For the basic analysis of network accesses, Wireshark\footnote{\url{http://go.yurichev.com/17303}} can be useful.

Dennoch muss man dann trotzdem hinein schauen. % <-- Der Satz ist kacke 
\\
Das erste wonach man schauen kann ist welche Funktionen die \ac{OS} \ac{API}s benutzen und was für Standard libraries
benutzt werden. 

Wenn das Programm unterteilt ist in eine Main executable und mehrere DLL Dateien, können manchmal die Namen der Funktionen innerhalb
der DLLs Helfen. 

Wenn wir daran interessiert sind was genau zum Aufruf von \TT{MessageBox()} mit einem spezifischen Text führt,
können wir versuchen diesen Text innerhalb des Data Segments zu finden, die Referenzen auf den Text und die 
Punkte von denen aus die Kontrolle an den \TT{MessageBox()} Aufruf an dem wir interessiert sind.

\myindex{\CStandardLibrary!rand()}
Wenn wir über Video Spiele sprechen sind wir daran interessiert welche mehr oder weniger zufällig darin vorkommen,
vielleicht versuchen wir die \rand Funktion oder Ersatz Funktionen zu finden ( wie z.B der Mersenne Twister Algorithmus) 
und wir versuchen die Orte zu finden von welchen aus diese Funktionen aufgerufen werden, und noch wichtiger was für
Ergebnisse verwertet werden. 
% BUG in varioref: http://tex.stackexchange.com/questions/104261/varioref-vref-or-vpageref-at-page-boundary-may-loop
Ein Beispiel: \ref{chap:color_lines}.

Aber wenn es sich nicht um ein Spiel handelt und \rand wird trotzdem benutzt, ist es interessant zu wissen warum.
Es gibt Fälle bei denen unerwartet \rand in Daten Kompression Algorithmen benutzt wird (für die Imitation von Verschlüsslung):
\href{http://go.yurichev.com/17221}{blog.yurichev.com}.

\subsection{Oft benutzte Funktionen in der Windows API}

Diese Funktionen sind vielleicht unter den importierten.
Es ist Sinnvoll an dieser Stelle zu erwähnen das nicht unbedingt jede Funktion benutzt wird aus 
dem Code den der Programmierer geschrieben hat.

Manche Funktionen haben eventuell das \GTT{-A} Suffix für die ASCII Version und das \GTT{-W} für die Unicode Version.


\begin{itemize}

\item
Registry zugriff (advapi32.dll): 
RegEnumKeyEx, RegEnumValue, RegGetValue, RegOpenKeyEx, RegQueryValueEx.

\item
Zugriff auf text .ini-files (kernel32.dll): 
GetPrivateProfileString.

\item
Dialog boxes (user32.dll): 
MessageBox, MessageBoxEx, CreateDialog, SetDlgItemText, GetDlgItemText.

\item
Resourcen zugriff (\myref{PEresources}): (user32.dll): LoadMenu.

\item
TCP/IP networking (ws2\_32.dll):
WSARecv, WSASend.

\item
Datei Zugriff (kernel32.dll):
CreateFile, ReadFile, ReadFileEx, WriteFile, WriteFileEx.

\item
High-level Zugriff auf das Internet (wininet.dll): WinHttpOpen.

\item
Die digitale Signatur einer ausführbaren Datei prüfen (wintrust.dll):

WinVerifyTrust.

\item
Die Standard MSVC library ( wenn sie dynamisch gelinked wurde) 

assert, itoa, ltoa, open, printf, read, strcmp, atol, atoi, fopen, fread, fwrite, memcmp, rand,
strlen, strstr, strchr.

\end{itemize}

\subsection{Verlängerung der Testphase}

Registry Zugriffs Funktionen sind häufig ziele für Leute die versuchen die Testphase einer Software zu cracken, die 
eventuell die Installations Zeit und Datum in der Regestry zu speichert. 

Ein weiteres beliebtes Ziel sind die GetLocalTime() und GetSystemTime() Funktionen:
eine Test Software, muss bei jedem Start die aktuelle Zeit und Datum überprüfen.

\subsection{Entfernen nerviger Dialog Boxenx}

Ein verbreiteter Weg raus zu finden was eine dieser nervigen Dialog boxen macht, ist den 
Aufruf von MessageBox(), CreateDialog() und CreateWindow() Funktionen abzufangen.


\subsection{tracer: Alle Funktionen innerhalb eines bestimmten Modules abfangen}

\myindex{tracer}

\myindex{x86!\Instructions!INT3}
Es gibt INT3 breakpoints in \tracer, die nur einmal ausgelöst werden. Jedoch können diese breakpoints für alle 
Funktionen in einer bestimmten DLL gesetzt werden.

\begin{lstlisting}
--one-time-INT3-bp:somedll.dll!.*
\end{lstlisting}

Oder, lasst uns einfach mal INT3 breakpoints für alle Funktionen setzen, die das \TT{xml} Präfix in ihrem Namen haben:

\begin{lstlisting}
--one-time-INT3-bp:somedll.dll!xml.*
\end{lstlisting}

Die andere Seite der Medaille ist, solche breakpoints werden nur einmal ausgelöst.
Tracer zeigt den Aufruf der Funktion, wenn er passiert, aber auch nur einmal. 
Ein weiterer Nachteil ist---es ist unmöglich die Argumente der Funktion zu betrachten.

Dennoch, dieses Feature ist sehr nützlich wenn man weiß das das Programm eine DLL benutzt,
aber man nicht weiß welche Funktionen aufgerufen werden. Und es gibt eine ganze Menge an 
Funktionen.


\par
\myindex{Cygwin}
Zum Beispiel, schauen wir uns einmal an was das uptime Kommando aus cygwin benutzt:

\begin{lstlisting}
tracer -l:uptime.exe --one-time-INT3-bp:cygwin1.dll!.*
\end{lstlisting}

Dadurch sehen wir alle cygwin1.dll library Funktionen die zumindest einmal aufgerufen wurden, und von welcher
Stelle: 
\lstinputlisting{digging_into_code/uptime_cygwin.txt}

