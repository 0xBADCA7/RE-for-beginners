\section{assert() Aufrufe}
\myindex{\CStandardLibrary!assert()}

Manchmal ist die Präsenz des \TT{assert()} macro's ebenfalls nützlich:
allgemein erlaubt dieses Makro Rückschlüsse auf source code Dateinamen,
Zeilen nummern und die Bedienung für das Makro im Code.

Die nützlichste Informationen ist enthalten in der Bedingung von assert, wir können Variablennamen oder Namen
von Struct Feldern ableiten. Ein weiteres nützliches Stück Information sind die Datei Namen---Wir können versuchen
abzuleiten von welcher Art der Code ist. 
Es ist ebenfalls möglich bekannte open-source library-Namen von den Datei Namen abzu leiten.

\lstinputlisting[caption=Example of informative assert() calls,style=customasmx86]{digging_into_code/assert_examples.lst}

Es ist Empfehlens wert beides die Konditionen und die Datei Namen in \q{google} zu suchen, was uns zu einer open-source library führen kann. 
Zum Beispiel, wenn wir \q{sp->lzw\_nbits <= BITS\_MAX} in \q{google} suchen, ist es absehbar das wir als Ergebnis den Code aus der 
Open-Source library für die LZW Kompression bekommen. 
