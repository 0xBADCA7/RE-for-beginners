\subsection*{mini-FAQ}

\par Q: Co trzeba wiedzieć zanim się przystąpi do książki?
\par A: Umiętności С/С++ są pożądane, ale nie są niezbędne.

\par Q: Czy powinnienem zacząć się uczyć naraz x86/x64/ARM i MIPS? Czy nie za dużo?
\par A: Myślę, że na początek, możesz czytać tylko o x86/x64.

\par Q: Czy jest możliwość zakupu książki w wersji papierowej w języku rozyjskim lub angielskim?
\par A: Niestety nie, żaden wydawca jeszcze się nie zainteresował wydaniem rosyjskiej lub angielskiej wersji. Natomiast można ją wydrukować i zbindować w każdym Xero.

\par Q: Czy istnieje wersja epub/mobi?
\par A: Książka jest napisana w bardzo specyficznym syntaksie TeX/LaTeX, dlatego przerobienie jej na HTML (epub/mobi to jest HTML)
nie jest łatwe.

\par Q: Po co uczyć się assemblera?
\par A: Jeśli się nie jest developerem \ac{OS}, to prawdopodobnie nie trzeba pisać nic w assemblerze: współczesne kompilatory optymalizują kod lepiej niż człowiek.

Do tego, współczesne \ac{CPU} są bardzo skomplikowanymi urządzeniami i wiedza assemblera nie pomoże poznać ich wnętrza.

Jednak zostają dwa obszary, w których umiętności pisania w assemblerze mogą bardzo pomóc:
1) research malwarów ; 2) lepsze zrozumienie kodu w trakcie debuggowania.
Z tego wynika, że ta książka jest napisana dla tych ludzi, którzy raczej chcą rozumieć assembler, a nie pisać w nim..

\par Q: Kliknąłem pod odnośnik wewnątrz PDF-pliku, jak teraz wrócić z powrotem?
\par A: W Adobe Acrobat Reader trzeba wcisnąć Alt+LeftArrow. W Evince kliknąć ``<''.

\par Q: Czy mogę wydrukować tę książkę? Korzystać z niej do nauki?
\par A: Oczywiście, właśnie dlatego ta książka ma licencję Creative Commons (CC BY-SA 4.0).

\par Q: Dlaczego ta książka jest darmowa? Jest to duży kawał roboty.
\par A: Moim zdaniem, autorzy literatury technicznej robią to dla samoreklamy. Taka robota nie przynosi za dużo pieniędzy.

\par Q: Jak znaleźć pracę w zawodzie reverse engineer-а?
\par A: Na reddit, dedykowanemu RE\FNURLREDDIT, od czasu od czasu pojawiają się hiring thread (\RedditHiringThread{}).
Proszę zobaczyć tam.

\par Q: Mam pytanie...
\par A: Proszę napisać je mailem (\EMAIL).

