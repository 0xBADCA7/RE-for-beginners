\section{
\q{QR9}: Rubik's cube inspired amateur crypto-algorithm}


Sometimes amateur cryptosystems appear to be pretty bizarre.

The author of this book was once asked to reverse engineer an amateur cryptoalgorithm of some data encryption utility, 
the source code for which was lost\footnote{He also got permission from the customer to publish the algorithm's details}.

Here is the listing exported from \IDA for the original encryption utility:

\lstinputlisting{examples/qr9/qr9_original.lst}

All function and label names were given by me during the analysis.

Let's start from the top. Here is a function that takes two file names and password.

\begin{lstlisting}
.text:00541320 ; int __cdecl crypt_file(int Str, char *Filename, int password)
.text:00541320 crypt_file      proc near
.text:00541320
.text:00541320 Str             = dword ptr  4
.text:00541320 Filename        = dword ptr  8
.text:00541320 password        = dword ptr  0Ch
.text:00541320
\end{lstlisting}

Open the file and report if an error occurs:

\begin{lstlisting}
.text:00541320                 mov     eax, [esp+Str]
.text:00541324                 push    ebp
.text:00541325                 push    offset Mode     ; "rb"
.text:0054132A                 push    eax             ; Filename
.text:0054132B                 call    _fopen          ; open file
.text:00541330                 mov     ebp, eax
.text:00541332                 add     esp, 8
.text:00541335                 test    ebp, ebp
.text:00541337                 jnz     short loc_541348
.text:00541339                 push    offset Format   ; "Cannot open input file!\n"
.text:0054133E                 call    _printf
.text:00541343                 add     esp, 4
.text:00541346                 pop     ebp
.text:00541347                 retn
.text:00541348
.text:00541348 loc_541348:
\end{lstlisting}

\myindex{\CStandardLibrary!fseek()}
\myindex{\CStandardLibrary!ftell()}
Get the file size via \TT{fseek()}/\TT{ftell()}:

\lstinputlisting{examples/qr9/1_EN}

This fragment of code calculates the file size aligned on a 64-byte boundary. 
This is because this cryptographic algorithm works with only 64-byte blocks. 
The operation is pretty straightforward: divide the file size by 64, forget about the remainder and add 1, 
then multiply by 64. 
The following code removes the remainder as if the value has already been divided by 64 and adds 64. 
It is almost the same.

\lstinputlisting{examples/qr9/2_EN}

Allocate buffer with aligned size:

\begin{lstlisting}
.text:00541373                 push    esi             ; Size
.text:00541374                 call    _malloc
\end{lstlisting}

\myindex{\CStandardLibrary!calloc()}
Call memset(), e.g., clear the allocated buffer\footnote{malloc() + memset() could 
be replaced by calloc()}.

\lstinputlisting{examples/qr9/3_EN}

Read file via the standard C function \TT{fread()}.

\begin{lstlisting}
.text:00541392                 mov     eax, [esp+38h+Str]
.text:00541396                 push    eax             ; ElementSize
.text:00541397                 push    ebx             ; DstBuf
.text:00541398                 call    _fread          ; read file
.text:0054139D                 push    ebp             ; File
.text:0054139E                 call    _fclose
\end{lstlisting}

Call \TT{crypt()}. This function takes a buffer, buffer size (aligned) and a password string.

\begin{lstlisting}
.text:005413A3                 mov     ecx, [esp+44h+password]
.text:005413A7                 push    ecx             ; password
.text:005413A8                 push    esi             ; aligned size
.text:005413A9                 push    ebx             ; buffer
.text:005413AA                 call    crypt           ; do crypt
\end{lstlisting}

Create the output file. By the way, the developer forgot to check if it has been created correctly! 
The file opening result is being checked, though.

\begin{lstlisting}
.text:005413AF                 mov     edx, [esp+50h+Filename]
.text:005413B3                 add     esp, 40h
.text:005413B6                 push    offset aWb      ; "wb"
.text:005413BB                 push    edx             ; Filename
.text:005413BC                 call    _fopen
.text:005413C1                 mov     edi, eax
\end{lstlisting}

The newly created file handle is in the \EDI register now. Write signature \q{QR9}.

\begin{lstlisting}
.text:005413C3                 push    edi             ; File
.text:005413C4                 push    1               ; Count
.text:005413C6                 push    3               ; Size
.text:005413C8                 push    offset aQr9     ; "QR9"
.text:005413CD                 call    _fwrite         ; write file signature
\end{lstlisting}

Write the actual file size (not aligned):

\begin{lstlisting}
.text:005413D2                 push    edi             ; File
.text:005413D3                 push    1               ; Count
.text:005413D5                 lea     eax, [esp+30h+Str]
.text:005413D9                 push    4               ; Size
.text:005413DB                 push    eax             ; Str
.text:005413DC                 call    _fwrite         ; write original file size
\end{lstlisting}

Write the encrypted buffer:

\begin{lstlisting}
.text:005413E1                 push    edi             ; File
.text:005413E2                 push    1               ; Count
.text:005413E4                 push    esi             ; Size
.text:005413E5                 push    ebx             ; Str
.text:005413E6                 call    _fwrite         ; write encrypted file
\end{lstlisting}

Close the file and free the allocated buffer:

\begin{lstlisting}
.text:005413EB                 push    edi             ; File
.text:005413EC                 call    _fclose
.text:005413F1                 push    ebx             ; Memory
.text:005413F2                 call    _free
.text:005413F7                 add     esp, 40h
.text:005413FA                 pop     edi
.text:005413FB                 pop     esi
.text:005413FC                 pop     ebx
.text:005413FD                 pop     ebp
.text:005413FE                 retn
.text:005413FE crypt_file      endp
\end{lstlisting}

Here is the reconstructed C code:

\begin{lstlisting}
void crypt_file(char *fin, char* fout, char *pw)
{
	FILE *f;
	int flen, flen_aligned;
	BYTE *buf;

	f=fopen(fin, "rb");
	
	if (f==NULL)
	{
		printf ("Cannot open input file!\n");
		return;
	};

	fseek (f, 0, SEEK_END);
	flen=ftell (f);
	fseek (f, 0, SEEK_SET);

	flen_aligned=(flen&0xFFFFFFC0)+0x40;

	buf=(BYTE*)malloc (flen_aligned);
	memset (buf, 0, flen_aligned);

	fread (buf, flen, 1, f);

	fclose (f);

	crypt (buf, flen_aligned, pw);
	
	f=fopen(fout, "wb");

	fwrite ("QR9", 3, 1, f);
	fwrite (&flen, 4, 1, f);
	fwrite (buf, flen_aligned, 1, f);

	fclose (f);

	free (buf);
};
\end{lstlisting}

The decryption procedure is almost the same:

\begin{lstlisting}
.text:00541400 ; int __cdecl decrypt_file(char *Filename, int, void *Src)
.text:00541400 decrypt_file    proc near
.text:00541400
.text:00541400 Filename        = dword ptr  4
.text:00541400 arg_4           = dword ptr  8
.text:00541400 Src             = dword ptr  0Ch
.text:00541400
.text:00541400                 mov     eax, [esp+Filename]
.text:00541404                 push    ebx
.text:00541405                 push    ebp
.text:00541406                 push    esi
.text:00541407                 push    edi
.text:00541408                 push    offset aRb      ; "rb"
.text:0054140D                 push    eax             ; Filename
.text:0054140E                 call    _fopen
.text:00541413                 mov     esi, eax
.text:00541415                 add     esp, 8
.text:00541418                 test    esi, esi
.text:0054141A                 jnz     short loc_54142E
.text:0054141C                 push    offset aCannotOpenIn_0 ; "Cannot open input file!\n"
.text:00541421                 call    _printf
.text:00541426                 add     esp, 4
.text:00541429                 pop     edi
.text:0054142A                 pop     esi
.text:0054142B                 pop     ebp
.text:0054142C                 pop     ebx
.text:0054142D                 retn
.text:0054142E
.text:0054142E loc_54142E:
.text:0054142E                 push    2               ; Origin
.text:00541430                 push    0               ; Offset
.text:00541432                 push    esi             ; File
.text:00541433                 call    _fseek
.text:00541438                 push    esi             ; File
.text:00541439                 call    _ftell
.text:0054143E                 push    0               ; Origin
.text:00541440                 push    0               ; Offset
.text:00541442                 push    esi             ; File
.text:00541443                 mov     ebp, eax
.text:00541445                 call    _fseek
.text:0054144A                 push    ebp             ; Size
.text:0054144B                 call    _malloc
.text:00541450                 push    esi             ; File
.text:00541451                 mov     ebx, eax
.text:00541453                 push    1               ; Count
.text:00541455                 push    ebp             ; ElementSize
.text:00541456                 push    ebx             ; DstBuf
.text:00541457                 call    _fread
.text:0054145C                 push    esi             ; File
.text:0054145D                 call    _fclose
\end{lstlisting}

Check signature (first 3 bytes):

\begin{lstlisting}
.text:00541462                 add     esp, 34h
.text:00541465                 mov     ecx, 3
.text:0054146A                 mov     edi, offset aQr9_0 ; "QR9"
.text:0054146F                 mov     esi, ebx
.text:00541471                 xor     edx, edx
.text:00541473                 repe cmpsb
.text:00541475                 jz      short loc_541489
\end{lstlisting}

Report an error if the signature is absent:

\begin{lstlisting}
.text:00541477                 push    offset aFileIsNotCrypt ; "File is not encrypted!\n"
.text:0054147C                 call    _printf
.text:00541481                 add     esp, 4
.text:00541484                 pop     edi
.text:00541485                 pop     esi
.text:00541486                 pop     ebp
.text:00541487                 pop     ebx
.text:00541488                 retn
.text:00541489
.text:00541489 loc_541489:
\end{lstlisting}

Call \TT{decrypt()}.

\begin{lstlisting}
.text:00541489                 mov     eax, [esp+10h+Src]
.text:0054148D                 mov     edi, [ebx+3]
.text:00541490                 add     ebp, 0FFFFFFF9h
.text:00541493                 lea     esi, [ebx+7]
.text:00541496                 push    eax             ; Src
.text:00541497                 push    ebp             ; int
.text:00541498                 push    esi             ; int
.text:00541499                 call    decrypt
.text:0054149E                 mov     ecx, [esp+1Ch+arg_4]
.text:005414A2                 push    offset aWb_0    ; "wb"
.text:005414A7                 push    ecx             ; Filename
.text:005414A8                 call    _fopen
.text:005414AD                 mov     ebp, eax
.text:005414AF                 push    ebp             ; File
.text:005414B0                 push    1               ; Count
.text:005414B2                 push    edi             ; Size
.text:005414B3                 push    esi             ; Str
.text:005414B4                 call    _fwrite
.text:005414B9                 push    ebp             ; File
.text:005414BA                 call    _fclose
.text:005414BF                 push    ebx             ; Memory
.text:005414C0                 call    _free
.text:005414C5                 add     esp, 2Ch
.text:005414C8                 pop     edi
.text:005414C9                 pop     esi
.text:005414CA                 pop     ebp
.text:005414CB                 pop     ebx
.text:005414CC                 retn
.text:005414CC decrypt_file    endp
\end{lstlisting}

Here is the reconstructed C code:

\begin{lstlisting}
void decrypt_file(char *fin, char* fout, char *pw)
{
	FILE *f;
	int real_flen, flen;
	BYTE *buf;

	f=fopen(fin, "rb");
	
	if (f==NULL)
	{
		printf ("Cannot open input file!\n");
		return;
	};

	fseek (f, 0, SEEK_END);
	flen=ftell (f);
	fseek (f, 0, SEEK_SET);

	buf=(BYTE*)malloc (flen);

	fread (buf, flen, 1, f);

	fclose (f);

	if (memcmp (buf, "QR9", 3)!=0)
	{
		printf ("File is not encrypted!\n");
		return;
	};

	memcpy (&real_flen, buf+3, 4);

	decrypt (buf+(3+4), flen-(3+4), pw);
	
	f=fopen(fout, "wb");

	fwrite (buf+(3+4), real_flen, 1, f);

	fclose (f);

	free (buf);
};
\end{lstlisting}

OK, now let's go deeper.

Function \TT{crypt()}:

\begin{lstlisting}
.text:00541260 crypt           proc near
.text:00541260
.text:00541260 arg_0           = dword ptr  4
.text:00541260 arg_4           = dword ptr  8
.text:00541260 arg_8           = dword ptr  0Ch
.text:00541260
.text:00541260                 push    ebx
.text:00541261                 mov     ebx, [esp+4+arg_0]
.text:00541265                 push    ebp
.text:00541266                 push    esi
.text:00541267                 push    edi
.text:00541268                 xor     ebp, ebp
.text:0054126A
.text:0054126A loc_54126A:
\end{lstlisting}

\myindex{x86!\Instructions!MOVSD}
This fragment of code copies a part of the input buffer to an internal array we later name \q{cube64}.
The size is in the \ECX register. \TT{MOVSD} stands for \IT{move 32-bit dword}, so, 
16 32-bit dwords are exactly 64 bytes.

\begin{lstlisting}
.text:0054126A                 mov     eax, [esp+10h+arg_8]
.text:0054126E                 mov     ecx, 10h
.text:00541273                 mov     esi, ebx   ; EBX is pointer within input buffer
.text:00541275                 mov     edi, offset cube64
.text:0054127A                 push    1
.text:0054127C                 push    eax
.text:0054127D                 rep movsd
\end{lstlisting}

Call \TT{rotate\_all\_with\_password()}:

\begin{lstlisting}
.text:0054127F                 call    rotate_all_with_password
\end{lstlisting}

Copy encrypted contents back from \q{cube64} to buffer:

\begin{lstlisting}
.text:00541284                 mov     eax, [esp+18h+arg_4]
.text:00541288                 mov     edi, ebx
.text:0054128A                 add     ebp, 40h
.text:0054128D                 add     esp, 8
.text:00541290                 mov     ecx, 10h
.text:00541295                 mov     esi, offset cube64
.text:0054129A                 add     ebx, 40h  ; add 64 to input buffer pointer
.text:0054129D                 cmp     ebp, eax  ; EBP = amount of encrypted data.
.text:0054129F                 rep movsd
\end{lstlisting}

If \EBP is not bigger that the size input argument, then continue to the next block.

\begin{lstlisting}
.text:005412A1                 jl      short loc_54126A
.text:005412A3                 pop     edi
.text:005412A4                 pop     esi
.text:005412A5                 pop     ebp
.text:005412A6                 pop     ebx
.text:005412A7                 retn
.text:005412A7 crypt           endp
\end{lstlisting}

Reconstructed \TT{crypt()} function:

\begin{lstlisting}
void crypt (BYTE *buf, int sz, char *pw)
{
	int i=0;
	
	do
	{
		memcpy (cube, buf+i, 8*8);
		rotate_all (pw, 1);
		memcpy (buf+i, cube, 8*8);
		i+=64;
	}
	while (i<sz);
};
\end{lstlisting}

OK, now let's go deeper in function \TT{rotate\_all\_with\_password()}. 
It takes two arguments: password string and a number.

In \TT{crypt()}, the number 1 is used, and in the \TT{decrypt()} function (where \TT{rotate\_all\_with\_password()} function 
is called too), the number is 3.

\begin{lstlisting}
.text:005411B0 rotate_all_with_password proc near
.text:005411B0
.text:005411B0 arg_0           = dword ptr  4
.text:005411B0 arg_4           = dword ptr  8
.text:005411B0
.text:005411B0                 mov     eax, [esp+arg_0]
.text:005411B4                 push    ebp
.text:005411B5                 mov     ebp, eax
\end{lstlisting}

Check the current character in the password. If it is zero, exit:

\begin{lstlisting}
.text:005411B7                 cmp     byte ptr [eax], 0
.text:005411BA                 jz      exit
.text:005411C0                 push    ebx
.text:005411C1                 mov     ebx, [esp+8+arg_4]
.text:005411C5                 push    esi
.text:005411C6                 push    edi
.text:005411C7
.text:005411C7 loop_begin:
\end{lstlisting}

\myindex{\CStandardLibrary!tolower()}
Call \TT{tolower()}, a standard C function.

\begin{lstlisting}
.text:005411C7                 movsx   eax, byte ptr [ebp+0]
.text:005411CB                 push    eax             ; C
.text:005411CC                 call    _tolower
.text:005411D1                 add     esp, 4
\end{lstlisting}

Hmm, if the password has non-Latin character, it is skipped! 
Indeed, when we run the encryption utility and try non-Latin characters in the password, 
they seem to be ignored.

\begin{lstlisting}
.text:005411D4                 cmp     al, 'a'
.text:005411D6                 jl      short next_character_in_password
.text:005411D8                 cmp     al, 'z'
.text:005411DA                 jg      short next_character_in_password
.text:005411DC                 movsx   ecx, al
\end{lstlisting}

Subtract the value of \q{a} (97) from the character.

\begin{lstlisting}
.text:005411DF                 sub     ecx, 'a'  ; 97
\end{lstlisting}

After subtracting, we'll get 0 for \q{a} here, 1 for \q{b}, etc. And 25 for \q{z}.

\begin{lstlisting}
.text:005411E2                 cmp     ecx, 24
.text:005411E5                 jle     short skip_subtracting
.text:005411E7                 sub     ecx, 24
\end{lstlisting}

It seems, \q{y} and \q{z} are exceptional characters too. 
After that fragment of code, \q{y} becomes 0 and \q{z}~---1. 
This implies that the 26 Latin alphabet symbols become values in the range of 0..23, (24 in total).

\begin{lstlisting}
.text:005411EA
.text:005411EA skip_subtracting:                       ; CODE XREF: rotate_all_with_password+35
\end{lstlisting}

This is actually division via multiplication. 
You can read more about it in the \q{\DivisionByMultSectionName} section~(\myref{sec:divisionbymult}).

The code actually divides the password character's value by 3.
% TODO1: add Mathematica calculations
\begin{lstlisting}
.text:005411EA                 mov     eax, 55555556h
.text:005411EF                 imul    ecx
.text:005411F1                 mov     eax, edx
.text:005411F3                 shr     eax, 1Fh
.text:005411F6                 add     edx, eax
.text:005411F8                 mov     eax, ecx
.text:005411FA                 mov     esi, edx
.text:005411FC                 mov     ecx, 3
.text:00541201                 cdq
.text:00541202                 idiv    ecx
\end{lstlisting}

\EDX is the remainder of the division.

\lstinputlisting{examples/qr9/4_EN}

If the remainder is 2, call \TT{rotate3()}. 
\EDI is the second argument of the \TT{rotate\_all\_with\_password()} function.
As we already noted, 1 is for the encryption operations and 3 is for the decryption. 
So, here is a loop. When encrypting, rotate1/2/3 are to be called the same number of times as 
given in the first argument.

\begin{lstlisting}
.text:00541215 call_rotate3:
.text:00541215                 push    esi
.text:00541216                 call    rotate3
.text:0054121B                 add     esp, 4
.text:0054121E                 dec     edi
.text:0054121F                 jnz     short call_rotate3
.text:00541221                 jmp     short next_character_in_password
.text:00541223
.text:00541223 call_rotate2:
.text:00541223                 test    ebx, ebx
.text:00541225                 jle     short next_character_in_password
.text:00541227                 mov     edi, ebx
.text:00541229
.text:00541229 loc_541229:
.text:00541229                 push    esi
.text:0054122A                 call    rotate2
.text:0054122F                 add     esp, 4
.text:00541232                 dec     edi
.text:00541233                 jnz     short loc_541229
.text:00541235                 jmp     short next_character_in_password
.text:00541237
.text:00541237 call_rotate1:
.text:00541237                 test    ebx, ebx
.text:00541239                 jle     short next_character_in_password
.text:0054123B                 mov     edi, ebx
.text:0054123D
.text:0054123D loc_54123D:
.text:0054123D                 push    esi
.text:0054123E                 call    rotate1
.text:00541243                 add     esp, 4
.text:00541246                 dec     edi
.text:00541247                 jnz     short loc_54123D
.text:00541249
\end{lstlisting}

Fetch the next character from the password string.

\begin{lstlisting}
.text:00541249 next_character_in_password:
.text:00541249                 mov     al, [ebp+1]
\end{lstlisting}

\Gls{increment} the character pointer in the password string:

\begin{lstlisting}
.text:0054124C                 inc     ebp
.text:0054124D                 test    al, al
.text:0054124F                 jnz     loop_begin
.text:00541255                 pop     edi
.text:00541256                 pop     esi
.text:00541257                 pop     ebx
.text:00541258
.text:00541258 exit:
.text:00541258                 pop     ebp
.text:00541259                 retn
.text:00541259 rotate_all_with_password endp
\end{lstlisting}

Here is the reconstructed C code:

\begin{lstlisting}
void rotate_all (char *pwd, int v)
{
	char *p=pwd;

	while (*p)
	{
		char c=*p;
		int q;

		c=tolower (c);

		if (c>='a' && c<='z')
		{
			q=c-'a';
			if (q>24)
				q-=24;

			int quotient=q/3;
			int remainder=q % 3;

			switch (remainder)
			{
			case 0: for (int i=0; i<v; i++) rotate1 (quotient); break;
			case 1: for (int i=0; i<v; i++) rotate2 (quotient); break;
			case 2: for (int i=0; i<v; i++) rotate3 (quotient); break;
			};
		};

		p++;
	};
};
\end{lstlisting}

Now let's go deeper and investigate the rotate1/2/3 functions. 
Each function calls another two functions. 
We eventually will name them \TT{set\_bit()} and \TT{get\_bit()}.

Let's start with \TT{get\_bit()}:

\begin{lstlisting}
.text:00541050 get_bit         proc near
.text:00541050
.text:00541050 arg_0           = dword ptr  4
.text:00541050 arg_4           = dword ptr  8
.text:00541050 arg_8           = byte ptr  0Ch
.text:00541050
.text:00541050                 mov     eax, [esp+arg_4]
.text:00541054                 mov     ecx, [esp+arg_0]
.text:00541058                 mov     al, cube64[eax+ecx*8]
.text:0054105F                 mov     cl, [esp+arg_8]
.text:00541063                 shr     al, cl
.text:00541065                 and     al, 1
.text:00541067                 retn
.text:00541067 get_bit         endp
\end{lstlisting}

\dots in other words: calculate an index in 
the cube64 array: \IT{arg\_4 + arg\_0 * 8}.
Then shift a byte from the array by arg\_8 bits right. 
Isolate the lowest bit and return it.

Let's see another function, \TT{set\_bit()}:

\begin{lstlisting}
.text:00541000 set_bit         proc near
.text:00541000
.text:00541000 arg_0           = dword ptr  4
.text:00541000 arg_4           = dword ptr  8
.text:00541000 arg_8           = dword ptr  0Ch
.text:00541000 arg_C           = byte ptr  10h
.text:00541000
.text:00541000                 mov     al, [esp+arg_C]
.text:00541004                 mov     ecx, [esp+arg_8]
.text:00541008                 push    esi
.text:00541009                 mov     esi, [esp+4+arg_0]
.text:0054100D                 test    al, al
.text:0054100F                 mov     eax, [esp+4+arg_4]
.text:00541013                 mov     dl, 1
.text:00541015                 jz      short loc_54102B
\end{lstlisting}

The value in the \TT{DL} is 1 here. It gets shifted left by arg\_8.
For example, if arg\_8 is 4, the value in the \TT{DL} register is to be 
0x10 or 1000b in binary form.

\begin{lstlisting}
.text:00541017                 shl     dl, cl
.text:00541019                 mov     cl, cube64[eax+esi*8]
\end{lstlisting}

Get a bit from array and explicitly set it. % TODO1: rewrite

\begin{lstlisting}
.text:00541020                 or      cl, dl
\end{lstlisting}

Store it back: % TODO1: rewrite

\begin{lstlisting}
.text:00541022                 mov     cube64[eax+esi*8], cl
.text:00541029                 pop     esi
.text:0054102A                 retn
.text:0054102B
.text:0054102B loc_54102B:
.text:0054102B                 shl     dl, cl
\end{lstlisting}

If arg\_C is not zero\dots

\begin{lstlisting}
.text:0054102D                 mov     cl, cube64[eax+esi*8]
\end{lstlisting}

\myindex{x86!\Instructions!NOT}

\dots invert DL. For example, if DL's state after the shift is 0x10 or 0b1000, 
there is 0xEF to be after the \NOT instruction (or 0b11101111b).

\begin{lstlisting}
.text:00541034                 not     dl
\end{lstlisting}

This instruction clears the bit, in other words, it saves all bits in \TT{CL} which are also set in 
\TT{DL} except those in \TT{DL} which are cleared.
This implies that if \TT{DL} is 11101111b in binary form,
all bits are to be saved except the 5th (counting from lowest bit).

\begin{lstlisting}
.text:00541036                 and     cl, dl
\end{lstlisting}

Store it back:

\begin{lstlisting}
.text:00541038                 mov     cube64[eax+esi*8], cl
.text:0054103F                 pop     esi
.text:00541040                 retn
.text:00541040 set_bit         endp
\end{lstlisting}

It is almost the same as \TT{get\_bit()}, except, if arg\_C is zero, the function clears the specific bit in the array, 
or sets it otherwise.

We also know that the array's size is 64. The first two arguments both in the \TT{set\_bit()} and \TT{get\_bit()} functions
could be seen as 2D coordinates. Then the array is to be an 8*8 matrix.

Here is a C representation of what we know up to now:

\begin{lstlisting}
#define IS_SET(flag, bit)       ((flag) & (bit))
#define SET_BIT(var, bit)       ((var) |= (bit))
#define REMOVE_BIT(var, bit)    ((var) &= ~(bit))

static BYTE cube[8][8];

void set_bit (int x, int y, int shift, int bit)
{
	if (bit)
		SET_BIT (cube[x][y], 1<<shift);
	else
		REMOVE_BIT (cube[x][y], 1<<shift);
};

bool get_bit (int x, int y, int shift)
{
	if ((cube[x][y]>>shift)&1==1)
		return 1;
	return 0;
};
\end{lstlisting}

Now let's get back to the rotate1/2/3 functions.

\begin{lstlisting}
.text:00541070 rotate1         proc near
.text:00541070
\end{lstlisting}

Internal array allocation in the local stack, with size of 64 bytes:

\begin{lstlisting}
.text:00541070 internal_array_64= byte ptr -40h
.text:00541070 arg_0           = dword ptr  4
.text:00541070
.text:00541070                 sub     esp, 40h
.text:00541073                 push    ebx
.text:00541074                 push    ebp
.text:00541075                 mov     ebp, [esp+48h+arg_0]
.text:00541079                 push    esi
.text:0054107A                 push    edi
.text:0054107B                 xor     edi, edi        ; EDI is loop1 counter
\end{lstlisting}

\EBX is a pointer to the internal array:

\begin{lstlisting}
.text:0054107D                 lea     ebx, [esp+50h+internal_array_64]
.text:00541081
\end{lstlisting}

Here we have two nested loops:

\lstinputlisting{examples/qr9/5_EN}

\dots we see that both loops' counters are in the range of 0..7. 
Also they are used as the first and second argument for the \TT{get\_bit()} function.
The third argument to \TT{get\_bit()} is the only argument of \TT{rotate1()}. 
The return value from \TT{get\_bit()} is placed in the internal array.

Prepare a pointer to the internal array again:

\lstinputlisting{examples/qr9/6_EN}

\dots this code is placing the contents of the internal array to the cube global array via the \TT{set\_bit()} function, 
\IT{but} in a different order!
Now the counter of the first loop is in the range of 7 to 0, \glslink{decrement}{decrementing} at each iteration!

The C code representation looks like:

\begin{lstlisting}
void rotate1 (int v)
{
	bool tmp[8][8]; // internal array
	int i, j;

	for (i=0; i<8; i++)
		for (j=0; j<8; j++)
			tmp[i][j]=get_bit (i, j, v);

	for (i=0; i<8; i++)
		for (j=0; j<8; j++)
			set_bit (j, 7-i, v, tmp[x][y]);
};
\end{lstlisting}

Not very understandable, but if we take a look at \TT{rotate2()} function:

\lstinputlisting{examples/qr9/7_EN}

It is \IT{almost} the same, except the order of the arguments of the \TT{get\_bit()} and \TT{set\_bit()} is different. 
Let's rewrite it in C-like code:

\begin{lstlisting}
void rotate2 (int v)
{
	bool tmp[8][8]; // internal array
	int i, j;

	for (i=0; i<8; i++)
		for (j=0; j<8; j++)
			tmp[i][j]=get_bit (v, i, j);

	for (i=0; i<8; i++)
		for (j=0; j<8; j++)
			set_bit (v, j, 7-i, tmp[i][j]);
};
\end{lstlisting}

Let's also rewrite the \TT{rotate3()} function:

\begin{lstlisting}
void rotate3 (int v)
{
	bool tmp[8][8];
	int i, j;

	for (i=0; i<8; i++)
		for (j=0; j<8; j++)
			tmp[i][j]=get_bit (i, v, j);

	for (i=0; i<8; i++)
		for (j=0; j<8; j++)
			set_bit (7-j, v, i, tmp[i][j]);
};
\end{lstlisting}

Well, now things are simpler. If we consider cube64 as a 3D cube of size 8*8*8, where each element is a bit, 
\TT{get\_bit()} and \TT{set\_bit()} take just the coordinates of a bit as input.

The rotate1/2/3 functions are in fact rotating all bits in a specific plane. 
These three functions are one for each cube side and the \TT{v} argument sets the plane in the range of 0..7.

Maybe the algorithm's author was thinking of a 8*8*8 Rubik's cube
\footnote{\href{http://go.yurichev.com/17115}{wikipedia}}?!

Yes, indeed.

Let's look closer into the \TT{decrypt()} function, here is its rewritten version:

\begin{lstlisting}
void decrypt (BYTE *buf, int sz, char *pw)
{
	char *p=strdup (pw);
	strrev (p);
	int i=0;

	do
	{
		memcpy (cube, buf+i, 8*8);
		rotate_all (p, 3);
		memcpy (buf+i, cube, 8*8);
		i+=64;
	}
	while (i<sz);
	
	free (p);
};
\end{lstlisting}

It is almost the same as for \TT{crypt()}, \IT{but} the password string is reversed by the
strrev() \footnote{\href{http://go.yurichev.com/17249}{MSDN}}
standard C function and \TT{rotate\_all()} is called with argument 3. 

This implies that in case of decryption, each corresponding rotate1/2/3 call is to be performed thrice.

This is almost as in Rubik'c cube! 
If you want to get back, do the same in reverse order and direction! 
If you want to undo the effect of rotating one place in clockwise direction, 
rotate it once in counter-clockwise direction, or thrice in clockwise direction.

\TT{rotate1()} is apparently for rotating the \q{front} plane. 
\TT{rotate2()} is apparently for rotating the \q{top} plane. 
\TT{rotate3()} is apparently for rotating the \q{left} plane.

Let's get back to the core of the \TT{rotate\_all()} function:

\begin{lstlisting}
q=c-'a';
if (q>24)
	q-=24;

int quotient=q/3; // in range 0..7
int remainder=q % 3;

switch (remainder)
{
    case 0: for (int i=0; i<v; i++) rotate1 (quotient); break; // front
    case 1: for (int i=0; i<v; i++) rotate2 (quotient); break; // top
    case 2: for (int i=0; i<v; i++) rotate3 (quotient); break; // left
};
\end{lstlisting}

Now it is much simpler to understand: each password character defines a side (one of three) and a plane (one of 8). 
3*8 = 24, that is why two the last two characters of the Latin alphabet are remapped to fit an alphabet of exactly 
24 elements.

The algorithm is clearly weak: in case of short passwords you can see
that in the encrypted file there are 
the original bytes of the original file in a binary file editor.

Here is the whole source code reconstructed:

\lstinputlisting{examples/qr9/qr9.cpp}

