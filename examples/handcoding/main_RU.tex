\section{Вручную написанный на ассемблере код}

\subsection{Тестовый файл EICAR}
\label{subsec:EICAR}

\myindex{MS-DOS}
\myindex{EICAR}
Этот .COM-файл предназначен для тестирования антивирусов, его можно запустить в MS-DOS
и он выведет такую строку: \q{EICAR-STANDARD-ANTIVIRUS-TEST-FILE!}
\footnote{\href{http://go.yurichev.com/17005}{wikipedia}}.
% FIXME1 \myref{} -> about .COM files

Он примечателен тем, что он полностью состоит только из печатных ASCII-символов, следовательно, его можно
набрать в любом текстовом редакторе:

\begin{lstlisting}
X5O!P%@AP[4\PZX54(P^)7CC)7}$EICAR-STANDARD-ANTIVIRUS-TEST-FILE!$H+H*
\end{lstlisting}

Попробуем его разобрать:

\lstinputlisting[style=customasmx86]{examples/handcoding/EICAR_RU.lst}

Добавим везде комментарии, показывающие состояние регистров и стека после каждой инструкции.

Собственно, все эти инструкции нужны только для того чтобы исполнить следующий код:

\begin{lstlisting}[style=customasmx86]
B4 09     MOV AH, 9
BA 1C 01  MOV DX, 11Ch
CD 21     INT 21h
CD 20     INT 20h
\end{lstlisting}

\myindex{x86!\Instructions!INT}
\TT{INT 21h} с функцией 9 (переданной в \TT{AH}) просто выводит строку, адрес которой передан в \TT{DS:DX}.
Кстати, строка должна быть завершена символом '\$'.
Надо полагать, это наследие \gls{CP/M} 
и эта функция в DOS осталась для совместимости.
\TT{INT 20h} возвращает управление в DOS.

Но, как видно, далеко не все опкоды этих инструкций печатные.
Так что основная часть EICAR-файла это:

\begin{itemize}
\item подготовка нужных значений регистров (AH и DX);
\item подготовка в памяти опкодов для INT 21 и INT 20;
\item исполнение INT 21 и INT 20.
\end{itemize}

\myindex{Shellcode}
Кстати, подобная техника широко используется для создания шеллкодов, 
где нужно создать x86-код, который будет нужно передать в виде текстовой строки.

Здесь также список всех x86-инструкций с печатаемыми опкодоами: \myref{printable_x86_opcodes}.
