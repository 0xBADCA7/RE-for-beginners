\section{Handwritten assembly code}

\subsection{ EICAR test file}
\label{subsec:EICAR}

\myindex{MS-DOS}
\myindex{EICAR}
This .COM-file is intended for testing antivirus software, it is possible to run in
in MS-DOS and it prints this string: \q{EICAR-STANDARD-ANTIVIRUS-TEST-FILE!}
\footnote{\href{http://go.yurichev.com/17006}{wikipedia}}.
% FIXME1 \myref{} -> about .COM files

Its important property is that it's consists entirely of printable 
ASCII-symbols, which, in turn, makes it possible to create it in any text editor:

\begin{lstlisting}
X5O!P%@AP[4\PZX54(P^)7CC)7}$EICAR-STANDARD-ANTIVIRUS-TEST-FILE!$H+H*
\end{lstlisting}

Let's decompile it:

\lstinputlisting[style=customasm]{examples/handcoding/EICAR_EN.lst}

We will add comments about the registers and stack after each instruction.

Essentially, all these
instructions are here only to execute this code:

\begin{lstlisting}[style=customasm]
B4 09     MOV AH, 9
BA 1C 01  MOV DX, 11Ch
CD 21     INT 21h
CD 20     INT 20h
\end{lstlisting}

\myindex{x86!\Instructions!INT}
\TT{INT 21h} with 9th
function (passed in \TT{AH}) just prints a string, the address of which is passed in \TT{DS:DX}.
By the way, the string has to be terminated
with the '\$' sign.
Apparently, it's inherited from \gls{CP/M} 
and this function was left in DOS for compatibility.
\TT{INT 20h} exits to DOS.

But as we can see, these instruction's
opcodes are not strictly printable.
So the main part of EICAR file is:

\begin{itemize}
\item preparing the register (AH and DX) values that we need;
\item preparing INT 21 and INT 20 opcodes in memory;
\item executing INT 21 and INT 20.
\end{itemize}

\myindex{Shellcode}

By the way, this technique is widely used in shellcode construction, when one have to pass x86 code
in string form.

Here is also a list of all 
x86 instructions which have printable opcodes: \myref{printable_x86_opcodes}.
