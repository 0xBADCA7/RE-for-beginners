\subsection{\TT{V\$VERSION} table in the \oracle}

\myindex{\oracle}
\myindex{Linux}
\myindex{Windows!ntoskrnl.exe}
\oracle 11.2 is a huge program, its main module \TT{oracle.exe} contain approx. 124,000 functions. For comparison, the Windows 7 x86 kernel (ntoskrnl.exe) contains approx. 11,000 functions and the Linux 3.9.8 kernel
(with default drivers compiled)---31,000 functions.

Let's start with an easy question. Where does \oracle get all this information, when we execute this simple statement in SQL*Plus:

\begin{lstlisting}
SQL> select * from V$VERSION;
\end{lstlisting}

And we get:

\begin{lstlisting}
BANNER
--------------------------------------------------------

Oracle Database 11g Enterprise Edition Release 11.2.0.1.0 - Production
PL/SQL Release 11.2.0.1.0 - Production
CORE    11.2.0.1.0      Production
TNS for 32-bit Windows: Version 11.2.0.1.0 - Production
NLSRTL Version 11.2.0.1.0 - Production
\end{lstlisting}

Let's start. Where in the \oracle can we find the string \TT{V\$VERSION}?

In the win32-version, \TT{oracle.exe} file contains the string,
it's easy to see.
But we can also use the object (.o) files from the Linux version of \oracle since, unlike the win32 version \TT{oracle.exe}, the function names (and global variables as well) are preserved there.

So, the \TT{kqf.o} file contains the \TT{V\$VERSION} string.
The object file is in the main Oracle-library \TT{libserver11.a}.

A reference to this text string can find in the \TT{kqfviw} table stored in the same file, \TT{kqf.o}:

\begin{lstlisting}[caption=kqf.o]
.rodata:0800C4A0 kqfviw dd 0Bh    ; DATA XREF: kqfchk:loc_8003A6D
.rodata:0800C4A0                  ; kqfgbn+34
.rodata:0800C4A4        dd offset _2__STRING_10102_0 ; "GV$WAITSTAT"
.rodata:0800C4A8        dd 4
.rodata:0800C4AC        dd offset _2__STRING_10103_0 ; "NULL"
.rodata:0800C4B0        dd 3
.rodata:0800C4B4        dd 0
.rodata:0800C4B8        dd 195h
.rodata:0800C4BC        dd 4
.rodata:0800C4C0        dd 0
.rodata:0800C4C4        dd 0FFFFC1CBh
.rodata:0800C4C8        dd 3
.rodata:0800C4CC        dd 0
.rodata:0800C4D0        dd 0Ah
.rodata:0800C4D4        dd offset _2__STRING_10104_0 ; "V$WAITSTAT"
.rodata:0800C4D8        dd 4
.rodata:0800C4DC        dd offset _2__STRING_10103_0 ; "NULL"
.rodata:0800C4E0        dd 3
.rodata:0800C4E4        dd 0
.rodata:0800C4E8        dd 4Eh
.rodata:0800C4EC        dd 3
.rodata:0800C4F0        dd 0
.rodata:0800C4F4        dd 0FFFFC003h
.rodata:0800C4F8        dd 4
.rodata:0800C4FC        dd 0
.rodata:0800C500        dd 5
.rodata:0800C504        dd offset _2__STRING_10105_0 ; "GV$BH"
.rodata:0800C508        dd 4
.rodata:0800C50C        dd offset _2__STRING_10103_0 ; "NULL"
.rodata:0800C510        dd 3
.rodata:0800C514        dd 0
.rodata:0800C518        dd 269h
.rodata:0800C51C        dd 15h
.rodata:0800C520        dd 0
.rodata:0800C524        dd 0FFFFC1EDh
.rodata:0800C528        dd 8
.rodata:0800C52C        dd 0
.rodata:0800C530        dd 4
.rodata:0800C534        dd offset _2__STRING_10106_0 ; "V$BH"
.rodata:0800C538        dd 4
.rodata:0800C53C        dd offset _2__STRING_10103_0 ; "NULL"
.rodata:0800C540        dd 3
.rodata:0800C544        dd 0
.rodata:0800C548        dd 0F5h
.rodata:0800C54C        dd 14h
.rodata:0800C550        dd 0
.rodata:0800C554        dd 0FFFFC1EEh
.rodata:0800C558        dd 5
.rodata:0800C55C        dd 0
\end{lstlisting}

By the way, often, while analyzing \oracle's internals, you may ask yourself, why are the names of the functions and global variable so weird.

Probably, because \oracle is a very old product and was developed in C in the 1980s.

And that was a time when the C standard guaranteed that the function names/variables can support only up to 6 characters inclusive: <<6 significant initial characters in an external identifier>>\footnote{\href{http://go.yurichev.com/17142}{Draft ANSI C Standard (ANSI X3J11/88-090) (May 13, 1988) (yurichev.com)}}

Probably, the table \TT{kqfviw} contains most (maybe even all) views prefixed with V\$, these are \IT{fixed views}, present all the time.
Superficially, by noticing the cyclic recurrence of data, we can easily see that each \TT{kqfviw} table element has 12 32-bit fields.
It is very simple to create a 12-elements structure in \IDA and apply it to all table elements.
As of \oracle version 11.2, there are 1023 table elements, i.e., in it are described 1023 of all possible \IT{fixed views}.

We are going to return to this number later.

As we can see, there is not much information in these numbers in the fields. The first number is always equals to the name of the view (without the terminating zero.
This is correct for each element. But this information is not very useful.

We also know that the information about all fixed views can be retrieved from a \IT{fixed view} named \TT{V\$FIXED\_VIEW\_DEFINITION}
(by the way, the information for this view is also taken from the \TT{kqfviw} and \TT{kqfvip} tables.)
By the way, there are 1023 elements in those too. Coincidence? No.

\begin{lstlisting}
SQL> select * from V$FIXED_VIEW_DEFINITION where view_name='V$VERSION';

VIEW_NAME
------------------------------
VIEW_DEFINITION
------------------------------

V$VERSION
select  BANNER from GV$VERSION where inst_id = USERENV('Instance')
\end{lstlisting}

So, \TT{V\$VERSION} is some kind of a \IT{thunk view} for another view, named \TT{GV\$VERSION}, which is, in turn:

\begin{lstlisting}
SQL> select * from V$FIXED_VIEW_DEFINITION where view_name='GV$VERSION';

VIEW_NAME
------------------------------
VIEW_DEFINITION
------------------------------

GV$VERSION
select inst_id, banner from x$version
\end{lstlisting}

The tables prefixed with X\$ in the \oracle are service tables too, undocumented, cannot be changed by the user and are refreshed dynamically.

If we search for the text \\
\begin{lstlisting}
select BANNER from GV\$VERSION where inst\_id = 
USERENV('Instance')
\end{lstlisting}
... 
in the \TT{kqf.o} file, we find it in the \TT{kqfvip} table:

\begin{lstlisting}[caption=kqf.o]
.rodata:080185A0 kqfvip dd offset _2__STRING_11126_0 ; DATA XREF: kqfgvcn+18
.rodata:080185A0                                ; kqfgvt+F
.rodata:080185A0                                ; "select inst_id,decode(indx,1,'data bloc"...
.rodata:080185A4        dd offset kqfv459_c_0
.rodata:080185A8        dd 0
.rodata:080185AC        dd 0

...

.rodata:08019570        dd offset _2__STRING_11378_0 ; "select  BANNER from GV$VERSION where in"...
.rodata:08019574        dd offset kqfv133_c_0
.rodata:08019578        dd 0
.rodata:0801957C        dd 0
.rodata:08019580        dd offset _2__STRING_11379_0 ; "select inst_id,decode(bitand(cfflg,1),0"...
.rodata:08019584        dd offset kqfv403_c_0
.rodata:08019588        dd 0
.rodata:0801958C        dd 0
.rodata:08019590        dd offset _2__STRING_11380_0 ; "select  STATUS , NAME, IS_RECOVERY_DEST"...
.rodata:08019594        dd offset kqfv199_c_0
\end{lstlisting}


The table appear to have 4 fields in each element. By the way, there are 1023 elements in it, again, the number we already know.

The second field points to another table that contains the table fields for this \IT{fixed view}.
As for \TT{V\$VERSION}, this table has only two elements, the first is 6 and the second is 
the \TT{BANNER} string (the number 6 is this string's length) and after, a \IT{terminating} element that contains 
0 and a \IT{null} C string:

\begin{lstlisting}[caption=kqf.o]
.rodata:080BBAC4 kqfv133_c_0 dd 6     ; DATA XREF: .rodata:08019574
.rodata:080BBAC8             dd offset _2__STRING_5017_0 ; "BANNER"
.rodata:080BBACC             dd 0
.rodata:080BBAD0             dd offset _2__STRING_0_0
\end{lstlisting}

By joining data from both \TT{kqfviw} and \TT{kqfvip} tables, we can get the SQL statements which are executed when the user wants to query information from a specific \IT{fixed view}.

So we can write an \oracletables program, to gather all this information from \oracle for Linux's object files.
For \TT{V\$VERSION}, we find this:

\begin{lstlisting}[caption=Result of \OracleTablesName]
kqfviw_element.viewname: [V$VERSION] ?: 0x3 0x43 0x1 0xffffc085 0x4
kqfvip_element.statement: [select  BANNER from GV$VERSION where inst_id = USERENV('Instance')]
kqfvip_element.params:
[BANNER] 
\end{lstlisting}

And:

\begin{lstlisting}[caption=Result of \OracleTablesName]
kqfviw_element.viewname: [GV$VERSION] ?: 0x3 0x26 0x2 0xffffc192 0x1
kqfvip_element.statement: [select inst_id, banner from x$version]
kqfvip_element.params:
[INST_ID] [BANNER] 
\end{lstlisting}

The \TT{GV\$VERSION} \IT{fixed view} is different from \TT{V\$VERSION} only in that it has one more field with the identifier \IT{instance}.

Anyway, we are going to stick with the \TT{X\$VERSION} table. Just like any other X\$-table, it is undocumented, however, we can query it:

\begin{lstlisting}
SQL> select * from x$version;

ADDR           INDX    INST_ID
-------- ---------- ----------
BANNER
------------------------------

0DBAF574          0          1
Oracle Database 11g Enterprise Edition Release 11.2.0.1.0 - Production

...
\end{lstlisting}

This table has some additional fields, like \TT{ADDR} and \TT{INDX}.

While scrolling \TT{kqf.o} in \IDA we can spot another table that contains a pointer to the \TT{X\$VERSION} string, this is \TT{kqftab}:

\begin{lstlisting}[caption=kqf.o]
.rodata:0803CAC0      dd 9                    ; element number 0x1f6
.rodata:0803CAC4      dd offset _2__STRING_13113_0 ; "X$VERSION"
.rodata:0803CAC8      dd 4
.rodata:0803CACC      dd offset _2__STRING_13114_0 ; "kqvt"
.rodata:0803CAD0      dd 4
.rodata:0803CAD4      dd 4
.rodata:0803CAD8      dd 0
.rodata:0803CADC      dd 4
.rodata:0803CAE0      dd 0Ch
.rodata:0803CAE4      dd 0FFFFC075h
.rodata:0803CAE8      dd 3
.rodata:0803CAEC      dd 0
.rodata:0803CAF0      dd 7
.rodata:0803CAF4      dd offset _2__STRING_13115_0 ; "X$KQFSZ"
.rodata:0803CAF8      dd 5
.rodata:0803CAFC      dd offset _2__STRING_13116_0 ; "kqfsz"
.rodata:0803CB00      dd 1
.rodata:0803CB04      dd 38h
.rodata:0803CB08      dd 0
.rodata:0803CB0C      dd 7
.rodata:0803CB10      dd 0
.rodata:0803CB14      dd 0FFFFC09Dh
.rodata:0803CB18      dd 2
.rodata:0803CB1C      dd 0
\end{lstlisting}

There are a lot of references to the X\$-table names, apparently, to all \oracle 11.2 X\$-tables.
But again, we don't have enough information.

It's not clear what does the \TT{kqvt} string stands for. 

The \TT{kq} prefix may mean \IT{kernel} or \IT{query}. 

\TT{v} apparently stands for \IT{version} and \TT{t}---\IT{type}? 
Hard to say.

A table with a similar name can be found in \TT{kqf.o}:

\begin{lstlisting}[caption=kqf.o]
.rodata:0808C360 kqvt_c_0 kqftap_param <4, offset _2__STRING_19_0, 917h, 0, 0, 0, 4, 0, 0>
.rodata:0808C360                                  ; DATA XREF: .rodata:08042680
.rodata:0808C360                                  ; "ADDR"
.rodata:0808C384          kqftap_param <4, offset _2__STRING_20_0, 0B02h, 0, 0, 0, 4, 0, 0> ; "INDX"
.rodata:0808C3A8          kqftap_param <7, offset _2__STRING_21_0, 0B02h, 0, 0, 0, 4, 0, 0> ; "INST_ID"
.rodata:0808C3CC          kqftap_param <6, offset _2__STRING_5017_0, 601h, 0, 0, 0, 50h, 0, 0> ; "BANNER"
.rodata:0808C3F0          kqftap_param <0, offset _2__STRING_0_0, 0, 0, 0, 0, 0, 0, 0>
\end{lstlisting}

It contains information about all fields in the \TT{X\$VERSION} table.
The only reference to this table is in the \TT{kqftap} table:

\begin{lstlisting}[caption=kqf.o]
.rodata:08042680                 kqftap_element <0, offset kqvt_c_0, offset kqvrow, 0> ; element 0x1f6
\end{lstlisting}

It is interesting that this element here is \TT{0x1f6th} (502nd), just like the pointer to the \TT{X\$VERSION} string in 
the \TT{kqftab} table.

Probably, the \TT{kqftap} and \TT{kqftab} tables complement each other, just like \TT{kqfvip} and \TT{kqfviw}.

We also see a pointer to the \TT{kqvrow()} function. Finally, we got something useful!

So we will add these tables to our \oracletables utility too. For \TT{X\$VERSION} we get:

\begin{lstlisting}[caption=Result of \OracleTablesName]
kqftab_element.name: [X$VERSION] ?: [kqvt] 0x4 0x4 0x4 0xc 0xffffc075 0x3
kqftap_param.name=[ADDR] ?: 0x917 0x0 0x0 0x0 0x4 0x0 0x0
kqftap_param.name=[INDX] ?: 0xb02 0x0 0x0 0x0 0x4 0x0 0x0
kqftap_param.name=[INST_ID] ?: 0xb02 0x0 0x0 0x0 0x4 0x0 0x0
kqftap_param.name=[BANNER] ?: 0x601 0x0 0x0 0x0 0x50 0x0 0x0
kqftap_element.fn1=kqvrow
kqftap_element.fn2=NULL
\end{lstlisting}

\myindex{tracer}
With the help of \tracer, it is easy to check that this function is called 6 times in row (from the \TT{qerfxFetch()} function) while querying the \TT{X\$VERSION} table.

Let's run \tracer in \TT{cc} mode (it comments each executed instruction):

\begin{lstlisting}
tracer -a:oracle.exe bpf=oracle.exe!_kqvrow,trace:cc
\end{lstlisting}

\lstinputlisting[style=customasm]{examples/oracle/VERSION_kqvrow.asm}

Now it is easy to see that the row number is passed from outside. The function returns the string, constructing it as follows:

\begin{center}
\begin{tabular}{ | l | l | }
\hline                        
String 1	& Using \TT{vsnstr}, \TT{vsnnum}, \TT{vsnban} global variables. \\
                                & Calls \TT{sprintf()}. \\
String 2	& Calls \TT{kkxvsn()}. \\
String 3	& Calls \TT{lmxver()}. \\
String 4	& Calls \TT{npinli()}, \TT{nrtnsvrs()}. \\
String 5	& Calls \TT{lxvers()}. \\
\hline  
\end{tabular}
\end{center}

That's how the corresponding functions are called for determining each module's version.

