\subsection{Функции проверки пароля в SAP 6.0}

\myindex{SAP}
Когда автор этой книги в очередной раз вернулся к своему SAP 6.0 IDES заинсталлированному в виртуальной машине VMware, он обнаружил что забыл пароль, впрочем, затем он вспомнил его, но теперь получаем такую ошибку:
 
\IT{<<Password logon no longer possible - too many failed attempts>>}, 
потому что были потрачены все попытки на то, чтобы вспомнить его.

\myindex{Windows!PDB}
Первая очень хорошая новость состоит в том, что с SAP поставляется полный \gls{PDB}-файл \IT{disp+work.pdb}, он содержит все: имена функций, структуры, типы, локальные переменные, имена аргументов, \etc{}. Какой щедрый подарок!

Существует утилита TYPEINFODUMP\footnote{\url{http://go.yurichev.com/17038}} для дампа содержимого \gls{PDB}-файлов во что-то более читаемое и grep-абельное.

Вот пример её работы: информация о функции + её аргументах + её локальных переменных:

\begin{lstlisting}
FUNCTION ThVmcSysEvent 
  Address:         10143190  Size:      675 bytes  Index:    60483  TypeIndex:    60484 
  Type: int NEAR_C ThVmcSysEvent (unsigned int, unsigned char, unsigned short*)
Flags: 0 
PARAMETER events 
  Address: Reg335+288  Size:        4 bytes  Index:    60488  TypeIndex:    60489 
  Type: unsigned int
Flags: d0 
PARAMETER opcode 
  Address: Reg335+296  Size:        1 bytes  Index:    60490  TypeIndex:    60491 
  Type: unsigned char
Flags: d0 
PARAMETER serverName 
  Address: Reg335+304  Size:        8 bytes  Index:    60492  TypeIndex:    60493 
  Type: unsigned short*
Flags: d0 
STATIC_LOCAL_VAR func 
  Address:         12274af0  Size:        8 bytes  Index:    60495  TypeIndex:    60496 
  Type: wchar_t*
Flags: 80 
LOCAL_VAR admhead 
  Address: Reg335+304  Size:        8 bytes  Index:    60498  TypeIndex:    60499 
  Type: unsigned char*
Flags: 90 
LOCAL_VAR record 
  Address: Reg335+64  Size:      204 bytes  Index:    60501  TypeIndex:    60502 
  Type: AD_RECORD
Flags: 90 
LOCAL_VAR adlen 
  Address: Reg335+296  Size:        4 bytes  Index:    60508  TypeIndex:    60509 
  Type: int
Flags: 90 
\end{lstlisting}

А вот пример дампа структуры:

\begin{lstlisting}
STRUCT DBSL_STMTID 
Size: 120  Variables: 4  Functions: 0  Base classes: 0
MEMBER moduletype 
  Type:  DBSL_MODULETYPE
  Offset:        0  Index:        3  TypeIndex:    38653
MEMBER module 
  Type:  wchar_t module[40]
  Offset:        4  Index:        3  TypeIndex:      831
MEMBER stmtnum 
  Type:  long
  Offset:       84  Index:        3  TypeIndex:      440
MEMBER timestamp 
  Type:  wchar_t timestamp[15]
  Offset:       88  Index:        3  TypeIndex:     6612
\end{lstlisting}

Вау!

Вторая хорошая новость: \IT{отладочные} вызовы, коих здесь очень много, очень полезны. 

Здесь вы можете увидеть глобальную переменную \IT{ct\_level}\footnote{Еще об уровне трассировки: \url{http://go.yurichev.com/17039}}, отражающую уровень трассировки.

В \IT{disp+work.exe} очень много таких отладочных вставок:

\begin{lstlisting}[style=customasm]
cmp     cs:ct_level, 1
jl      short loc_1400375DA
call    DpLock
lea     rcx, aDpxxtool4_c ; "dpxxtool4.c"
mov     edx, 4Eh        ; line
call    CTrcSaveLocation
mov     r8, cs:func_48
mov     rcx, cs:hdl     ; hdl
lea     rdx, aSDpreadmemvalu ; "%s: DpReadMemValue (%d)"
mov     r9d, ebx
call    DpTrcErr
call    DpUnlock
\end{lstlisting}

Если текущий уровень трассировки выше или равен заданному в этом коде порогу, 
отладочное сообщение будет записано в лог-файл вроде \IT{dev\_w0}, \IT{dev\_disp} 
и прочие файлы \IT{dev*}.

\myindex{\GrepUsage}
Попробуем grep-ать файл недавно полученный при помощи утилиты TYPEINFODUMP:

\begin{lstlisting}
cat "disp+work.pdb.d" | grep FUNCTION | grep -i password
\end{lstlisting}

Получаем:

\begin{lstlisting}
FUNCTION rcui::AgiPassword::DiagISelection 
FUNCTION ssf_password_encrypt 
FUNCTION ssf_password_decrypt 
FUNCTION password_logon_disabled 
FUNCTION dySignSkipUserPassword 
FUNCTION migrate_password_history 
FUNCTION password_is_initial 
FUNCTION rcui::AgiPassword::IsVisible 
FUNCTION password_distance_ok 
FUNCTION get_password_downwards_compatibility 
FUNCTION dySignUnSkipUserPassword 
FUNCTION rcui::AgiPassword::GetTypeName 
FUNCTION `rcui::AgiPassword::AgiPassword'::`1'::dtor$2 
FUNCTION `rcui::AgiPassword::AgiPassword'::`1'::dtor$0 
FUNCTION `rcui::AgiPassword::AgiPassword'::`1'::dtor$1 
FUNCTION usm_set_password 
FUNCTION rcui::AgiPassword::TraceTo 
FUNCTION days_since_last_password_change 
FUNCTION rsecgrp_generate_random_password 
FUNCTION rcui::AgiPassword::`scalar deleting destructor' 
FUNCTION password_attempt_limit_exceeded 
FUNCTION handle_incorrect_password 
FUNCTION `rcui::AgiPassword::`scalar deleting destructor''::`1'::dtor$1 
FUNCTION calculate_new_password_hash 
FUNCTION shift_password_to_history 
FUNCTION rcui::AgiPassword::GetType 
FUNCTION found_password_in_history 
FUNCTION `rcui::AgiPassword::`scalar deleting destructor''::`1'::dtor$0 
FUNCTION rcui::AgiObj::IsaPassword 
FUNCTION password_idle_check 
FUNCTION SlicHwPasswordForDay 
FUNCTION rcui::AgiPassword::IsaPassword 
FUNCTION rcui::AgiPassword::AgiPassword 
FUNCTION delete_user_password 
FUNCTION usm_set_user_password 
FUNCTION Password_API 
FUNCTION get_password_change_for_SSO 
FUNCTION password_in_USR40 
FUNCTION rsec_agrp_abap_generate_random_password 
\end{lstlisting}

Попробуем так же искать отладочные сообщения содержащие слова \IT{<<password>>} и \IT{<<locked>>}.
Одна из таких это строка \IT{<<user was locked by subsequently failed password logon attempts>>} на которую есть ссылка в \\
функции \IT{password\_attempt\_limit\_exceeded()}.

Другие строки, которые эта найденная функция может писать в лог-файл это: 
\IT{<<password logon attempt will be rejected immediately (preventing dictionary attacks)>>}, \IT{<<failed-logon lock: expired (but not removed due to 'read-only' operation)>>}, \IT{<<failed-logon lock: expired => removed>>}.

Немного поэкспериментировав с этой функцией, мы быстро понимаем, что проблема именно в ней.
Она вызывается из функции \IT{chckpass()} --- одна из функций проверяющих пароль.

В начале, давайте убедимся, что мы на верном пути:

Запускаем \tracer:
\myindex{tracer}

\begin{lstlisting}
tracer64.exe -a:disp+work.exe bpf=disp+work.exe!chckpass,args:3,unicode
\end{lstlisting}

\begin{lstlisting}
PID=2236|TID=2248|(0) disp+work.exe!chckpass (0x202c770, L"Brewered1                               ", 0x41) (called from 0x1402f1060 (disp+work.exe!usrexist+0x3c0))
PID=2236|TID=2248|(0) disp+work.exe!chckpass -> 0x35
\end{lstlisting}

Функции вызываются так: \IT{syssigni()} -> \IT{DyISigni()} -> \IT{dychkusr()} -> \IT{usrexist()} -> \IT{chckpass()}.

Число 0x35 возвращается из \IT{chckpass()} в этом месте:

\begin{lstlisting}[style=customasm]
.text:00000001402ED567 loc_1402ED567:                          ; CODE XREF: chckpass+B4
.text:00000001402ED567                 mov     rcx, rbx        ; usr02
.text:00000001402ED56A                 call    password_idle_check
.text:00000001402ED56F                 cmp     eax, 33h
.text:00000001402ED572                 jz      loc_1402EDB4E
.text:00000001402ED578                 cmp     eax, 36h
.text:00000001402ED57B                 jz      loc_1402EDB3D
.text:00000001402ED581                 xor     edx, edx        ; usr02_readonly
.text:00000001402ED583                 mov     rcx, rbx        ; usr02
.text:00000001402ED586                 call    password_attempt_limit_exceeded
.text:00000001402ED58B                 test    al, al
.text:00000001402ED58D                 jz      short loc_1402ED5A0
.text:00000001402ED58F                 mov     eax, 35h
.text:00000001402ED594                 add     rsp, 60h
.text:00000001402ED598                 pop     r14
.text:00000001402ED59A                 pop     r12
.text:00000001402ED59C                 pop     rdi
.text:00000001402ED59D                 pop     rsi
.text:00000001402ED59E                 pop     rbx
.text:00000001402ED59F                 retn
\end{lstlisting}

Отлично, давайте проверим:

\begin{lstlisting}
tracer64.exe -a:disp+work.exe bpf=disp+work.exe!password_attempt_limit_exceeded,args:4,unicode,rt:0
\end{lstlisting}

\begin{lstlisting}
PID=2744|TID=360|(0) disp+work.exe!password_attempt_limit_exceeded (0x202c770, 0, 0x257758, 0) (called from 0x1402ed58b (disp+work.exe!chckpass+0xeb))
PID=2744|TID=360|(0) disp+work.exe!password_attempt_limit_exceeded -> 1
PID=2744|TID=360|We modify return value (EAX/RAX) of this function to 0
PID=2744|TID=360|(0) disp+work.exe!password_attempt_limit_exceeded (0x202c770, 0, 0, 0) (called from 0x1402e9794 (disp+work.exe!chngpass+0xe4))
PID=2744|TID=360|(0) disp+work.exe!password_attempt_limit_exceeded -> 1
PID=2744|TID=360|We modify return value (EAX/RAX) of this function to 0
\end{lstlisting}

Великолепно! Теперь мы можем успешно залогиниться.

Кстати, мы можем сделать вид что вообще забыли пароль, заставляя \IT{chckpass()} всегда возвращать ноль, и этого достаточно для отключения проверки пароля:

\begin{lstlisting}
tracer64.exe -a:disp+work.exe bpf=disp+work.exe!chckpass,args:3,unicode,rt:0
\end{lstlisting}

\begin{lstlisting}
PID=2744|TID=360|(0) disp+work.exe!chckpass (0x202c770, L"bogus                                   ", 0x41) (called from 0x1402f1060 (disp+work.exe!usrexist+0x3c0))
PID=2744|TID=360|(0) disp+work.exe!chckpass -> 0x35
PID=2744|TID=360|We modify return value (EAX/RAX) of this function to 0
\end{lstlisting}

Что еще можно сказать, бегло анализируя функцию \\
\IT{password\_attempt\_limit\_exceeded()}, это то, что в начале
можно увидеть следующий вызов:

\begin{lstlisting}[style=customasm]
lea     rcx, aLoginFailed_us ; "login/failed_user_auto_unlock"
call    sapgparam
test    rax, rax
jz      short loc_1402E19DE
movzx   eax, word ptr [rax]
cmp     ax, 'N'
jz      short loc_1402E19D4
cmp     ax, 'n'
jz      short loc_1402E19D4
cmp     ax, '0'
jnz     short loc_1402E19DE
\end{lstlisting}

Очевидно, функция \IT{sapgparam()} используется чтобы узнать значение какой-либо переменной конфигурации. Эта функция может вызываться из 1768 разных мест.

Вероятно, при помощи этой информации, мы можем легко находить те места кода, на которые влияют определенные переменные конфигурации.

Замечательно! Имена функций очень понятны, куда понятнее чем в \oracle.
 
\myindex{\Cpp}
По всей видимости, процесс \IT{disp+work} весь написан на \Cpp. Должно быть, он был переписан не так давно?

