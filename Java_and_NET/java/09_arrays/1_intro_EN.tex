% TODO proof-reading
\subsubsection{Simple example}

Let's first create an array of 10 integers and fill it:

\begin{lstlisting}
	public static void main(String[] args) 
	{
		int a[]=new int[10];
		for (int i=0; i<10; i++)
			a[i]=i;
		dump (a);
	}
\end{lstlisting}

\begin{lstlisting}
  public static void main(java.lang.String[]);
    flags: ACC_PUBLIC, ACC_STATIC
    Code:
      stack=3, locals=3, args_size=1
         0: bipush        10
         2: newarray       int
         4: astore_1      
         5: iconst_0      
         6: istore_2      
         7: iload_2       
         8: bipush        10
        10: if_icmpge     23
        13: aload_1       
        14: iload_2       
        15: iload_2       
        16: iastore       
        17: iinc          2, 1
        20: goto          7
        23: aload_1       
        24: invokestatic  #4     // Method dump:([I)V
        27: return        
\end{lstlisting}

The \TT{newarray} instruction creates an array object of 10 \IT{int} elements.

The array's size is set with \TT{bipush} and left at \ac{TOS}.

The array's type is set in \TT{newarray} instruction's operand.

After \TT{newarray}'s execution, a \IT{reference} (or pointer) to the newly created array in the heap 
is left at the \ac{TOS}.

\TT{astore\_1} stores the \IT{reference} to the 1st slot in \ac{LVA}.

The second part of the \main function is the loop which stores \IT{i} into the corresponding
array element.

\TT{aload\_1} gets a \IT{reference} of the array and places it in the stack.

\TT{iastore} then stores the integer value from the stack in the array, 
\IT{reference} of which is currently in \ac{TOS}.

The third part of the \main function calls the \TT{dump()} function.

An argument for it is prepared by \TT{aload\_1} (offset 23).

Now let's proceed to the \TT{dump()} function:

\begin{lstlisting}
	public static void dump(int a[])
	{
		for (int i=0; i<a.length; i++)
			System.out.println(a[i]);
	}
\end{lstlisting}

\begin{lstlisting}
  public static void dump(int[]);
    flags: ACC_PUBLIC, ACC_STATIC
    Code:
      stack=3, locals=2, args_size=1
         0: iconst_0      
         1: istore_1      
         2: iload_1       
         3: aload_0       
         4: arraylength   
         5: if_icmpge     23
         8: getstatic     #2      // Field java/lang/System.out:Ljava/io/PrintStream;
        11: aload_0       
        12: iload_1       
        13: iaload        
        14: invokevirtual #3      // Method java/io/PrintStream.println:(I)V
        17: iinc          1, 1
        20: goto          2
        23: return        
\end{lstlisting}

The incoming \TT{reference} to the array is in the zeroth slot.

The \TT{a.length} expression in the source code is converted to an \TT{arraylength} instruction: 
it takes a \IT{reference} to the array and leaves the array size at \ac{TOS}.

\TT{iaload} at offset 13 is used to load array elements, 
it requires to array \IT{reference} be present
in the stack (prepared by \TT{aload\_0} at 11), 
and also an index (prepared by \TT{iload\_1} at offset 12).

Needless to say, instructions prefixed with \IT{a} may be mistakenly comprehended 
as \IT{array} instructions.

It's not correct.
These instructions works with \IT{references} to objects.

And arrays and strings are objects too.

