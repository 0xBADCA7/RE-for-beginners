% TODO proof-reading
\subsection{Simple function calling}

\TT{Math.random()} returns a pseudorandom number in range of [0.0 \dots 1.0), but let's say that
for some reason we need to devise a function that returns a number in range of [0.0 \dots 0.5):


\begin{lstlisting}[style=customjava]
public class HalfRandom
{ 
	public static double f()
	{
		return Math.random()/2;
	}
}
\end{lstlisting}

\begin{lstlisting}[caption=Constant pool]
...
   #2 = Methodref          #18.#19    //  java/lang/Math.random:()D
   #3 = Double             2.0d
...
  #12 = Utf8               ()D
...
  #18 = Class              #22        //  java/lang/Math
  #19 = NameAndType        #23:#12    //  random:()D
  #22 = Utf8               java/lang/Math
  #23 = Utf8               random
\end{lstlisting}

\begin{lstlisting}
  public static double f();
    flags: ACC_PUBLIC, ACC_STATIC
    Code:
      stack=4, locals=0, args_size=0
         0: invokestatic  #2          // Method java/lang/Math.random:()D
         3: ldc2_w        #3          // double 2.0d
         6: ddiv          
         7: dreturn       
\end{lstlisting}

\TT{invokestatic} calls the \TT{Math.random()} function and leaves the result at the \ac{TOS}.

Then the result is divided by 2.0 and returned.

But how is the function name encoded?

It's encoded in the constant pool using a \TT{Methodref} expression.

It defines the class and method names.

The first field of \TT{Methodref} points to a \TT{Class} expression which, in turn, points to
the usual text string (\q{java/lang/Math}).

The second \TT{Methodref} expression points to a \TT{NameAndType} expression which also 
has two links to the strings.

The first string is \q{random}, which is the name of the method.

The second string is \q{()D}, which encodes the function's type.
It means that it returns a \IT{double} value (hence the \IT{D} in the string).

This is the way 
1) JVM can check data for type correctness; 
2) Java decompilers can restore data types from a compiled class file.


Now let's try the \q{Hello, world!} example:


\begin{lstlisting}[style=customjava]
public class HelloWorld
{
	public static void main(String[] args)
	{
		System.out.println("Hello, World");
	}
}
\end{lstlisting}

\begin{lstlisting}[caption=Constant pool]
...
   #2 = Fieldref           #16.#17        //  java/lang/System.out:Ljava/io/PrintStream;
   #3 = String             #18            //  Hello, World
   #4 = Methodref          #19.#20        //  java/io/PrintStream.println:(Ljava/lang/String;)V
...
  #16 = Class              #23            //  java/lang/System
  #17 = NameAndType        #24:#25        //  out:Ljava/io/PrintStream;
  #18 = Utf8               Hello, World
  #19 = Class              #26            //  java/io/PrintStream
  #20 = NameAndType        #27:#28        //  println:(Ljava/lang/String;)V
...
  #23 = Utf8               java/lang/System
  #24 = Utf8               out
  #25 = Utf8               Ljava/io/PrintStream;
  #26 = Utf8               java/io/PrintStream
  #27 = Utf8               println
  #28 = Utf8               (Ljava/lang/String;)V
...
\end{lstlisting}

\begin{lstlisting}
  public static void main(java.lang.String[]);
    flags: ACC_PUBLIC, ACC_STATIC
    Code:
      stack=2, locals=1, args_size=1
         0: getstatic     #2        // Field java/lang/System.out:Ljava/io/PrintStream;
         3: ldc           #3        // String Hello, World
         5: invokevirtual #4        // Method java/io/PrintStream.println:(Ljava/lang/String;)V
         8: return        
\end{lstlisting}

\TT{ldc} at offset 3 takes a pointer to the \q{Hello, World} string in the constant pool
and pushes in the stack.

It's called a \IT{reference} in the Java world, but it's rather a pointer, or an address

\footnote{About difference in pointers and \IT{reference}'s in C++ see: \myref{cpp_references}.}.

The familiar \TT{invokevirtual} instruction takes the information about the \TT{println} 
function (or method) from the constant pool and calls it.

As we may know, there are several \TT{println} methods, one for each data type.

Our case is the version of \TT{println} intended for the \IT{String} data type.


But what about the first \TT{getstatic} instruction?

This instruction takes a \IT{reference} (or address of) a field of the object \TT{System.out} 
and pushes it in the stack.

This value is acts like the \IT{this} pointer for the \TT{println} method.

Thus, internally, the \TT{println} method takes two arguments for input:
1) \IT{this}, i.e., a pointer to an object; 
2) the address of the \q{Hello, World} string.


Indeed, \TT{println()} is called as a method within an initialized \TT{System.out} object.


For convenience, the \TT{javap} utility writes all this information in the comments.

