% TODO proof-reading
\subsection{Argumente übergeben}

%Let's extend our \TT{min()}/\TT{max()} example:
%
%
%\begin{lstlisting}[style=customjava]
%public class minmax
%{
%	public static int min (int a, int b)
%	{
%		if (a>b)
%			return b;
%		return a;
%	}
%
%	public static int max (int a, int b)
%	{
%		if (a>b)
%			return a;
%		return b;
%	}
%
%	public static void main(String[] args)
%	{
%		int a=123, b=456;
%		int max_value=max(a, b);
%		int min_value=min(a, b);
%		System.out.println(min_value);
%		System.out.println(max_value);
%	}
%}
%\end{lstlisting}
%
%Here is \main function code:
%
%
%\begin{lstlisting}
%  public static void main(java.lang.String[]);
%    flags: ACC_PUBLIC, ACC_STATIC
%    Code:
%      stack=2, locals=5, args_size=1
%         0: bipush        123
%         2: istore_1      
%         3: sipush        456
%         6: istore_2      
%         7: iload_1       
%         8: iload_2       
%         9: invokestatic  #2       // Method max:(II)I
%        12: istore_3      
%        13: iload_1       
%        14: iload_2       
%        15: invokestatic  #3       // Method min:(II)I
%        18: istore        4
%        20: getstatic     #4       // Field java/lang/System.out:Ljava/io/PrintStream;
%        23: iload         4
%        25: invokevirtual #5       // Method java/io/PrintStream.println:(I)V
%        28: getstatic     #4       // Field java/lang/System.out:Ljava/io/PrintStream;
%        31: iload_3       
%        32: invokevirtual #5       // Method java/io/PrintStream.println:(I)V
%        35: return        
%\end{lstlisting}
%
%Arguments are passed to the other function in the stack, and the return value is left on \ac{TOS}.
%
%
