% TODO proof-reading
\subsection{Условные переходы}

Перейдем к условным переходам.

\begin{lstlisting}[style=customjava]
public class abs
{
	public static int abs(int a)
	{
		if (a<0)
			return -a;
		return a;
	}
}
\end{lstlisting}

\begin{lstlisting}
  public static int abs(int);
    flags: ACC_PUBLIC, ACC_STATIC
    Code:
      stack=1, locals=1, args_size=1
         0: iload_0       
         1: ifge          7
         4: iload_0       
         5: ineg          
         6: ireturn       
         7: iload_0       
         8: ireturn       
\end{lstlisting}


\TT{ifge} переходит на смещение 7 если значение на \ac{TOS} больше или равно 0.

Не забывайте, любая инструкция \TT{ifXX} выталкивает значение (с которым будет производиться
сравнение) из стека.


\TT{ineg} просто меняет знак значения на \ac{TOS}.

Еще пример:

\begin{lstlisting}[style=customjava]
	public static int min (int a, int b)
	{
		if (a>b)
			return b;
		return a;
	}
\end{lstlisting}

Получаем:

\begin{lstlisting}
  public static int min(int, int);
    flags: ACC_PUBLIC, ACC_STATIC
    Code:
      stack=2, locals=2, args_size=2
         0: iload_0       
         1: iload_1       
         2: if_icmple     7
         5: iload_1       
         6: ireturn       
         7: iload_0       
         8: ireturn       
\end{lstlisting}

\TT{if\_icmple} выталкивает два значения и сравнивает их.

Если второе меньше первого (или равно), происходит переход на смещение 7.


Когда мы определяем функцию \TT{max()} \dots

\begin{lstlisting}[style=customjava]
	public static int max (int a, int b)
	{
		if (a>b)
			return a;
		return b;
	}
\end{lstlisting}


\dots итоговый код точно такой же, только последние инструкции \TT{iload} 
(на смещениях 5 и 7) поменяны местами:

\begin{lstlisting}
  public static int max(int, int);
    flags: ACC_PUBLIC, ACC_STATIC
    Code:
      stack=2, locals=2, args_size=2
         0: iload_0       
         1: iload_1       
         2: if_icmple     7
         5: iload_0       
         6: ireturn       
         7: iload_1       
         8: ireturn       
\end{lstlisting}

Более сложный пример:

\begin{lstlisting}[style=customjava]
public class cond
{
	public static void f(int i)
	{
		if (i<100)
			System.out.print("<100");
		if (i==100)
			System.out.print("==100");
		if (i>100)
			System.out.print(">100");
		if (i==0)
			System.out.print("==0");
	}
}
\end{lstlisting}

\begin{lstlisting}
  public static void f(int);
    flags: ACC_PUBLIC, ACC_STATIC
    Code:
      stack=2, locals=1, args_size=1
         0: iload_0       
         1: bipush        100
         3: if_icmpge     14
         6: getstatic     #2        // Field java/lang/System.out:Ljava/io/PrintStream;
         9: ldc           #3        // String <100
        11: invokevirtual #4        // Method java/io/PrintStream.print:(Ljava/lang/String;)V
        14: iload_0       
        15: bipush        100
        17: if_icmpne     28
        20: getstatic     #2        // Field java/lang/System.out:Ljava/io/PrintStream;
        23: ldc           #5        // String ==100
        25: invokevirtual #4        // Method java/io/PrintStream.print:(Ljava/lang/String;)V
        28: iload_0       
        29: bipush        100
        31: if_icmple     42
        34: getstatic     #2        // Field java/lang/System.out:Ljava/io/PrintStream;
        37: ldc           #6        // String >100
        39: invokevirtual #4        // Method java/io/PrintStream.print:(Ljava/lang/String;)V
        42: iload_0       
        43: ifne          54
        46: getstatic     #2        // Field java/lang/System.out:Ljava/io/PrintStream;
        49: ldc           #7        // String ==0
        51: invokevirtual #4        // Method java/io/PrintStream.print:(Ljava/lang/String;)V
        54: return        
\end{lstlisting}

\TT{if\_icmpge} Выталкивает два значения и сравнивает их.

Если второй больше первого, происходит переход на смещение 14.

\TT{if\_icmpne} и \TT{if\_icmple} работают одинаково, но используются разные условия.


По смещению 43 есть также инструкция \TT{ifne}.

Название неудачное, её было бы лучше назвать \TT{ifnz} 
(переход если переменная на \ac{TOS} не равна нулю).

И вот что она делает: производит переход на смещение 54, если входное значение не ноль.

Если ноль, управление передается на смещение 46, где выводится строка \q{==0}.


N.B.: В \ac{JVM} нет беззнаковых типов данных, так что инструкции сравнения работают
только со знаковыми целочисленными значениями.
