% TODO proof-reading
\subsection{Conditional jumps}

Now let's proceed to conditional jumps.

\begin{lstlisting}
public class abs
{
	public static int abs(int a)
	{
		if (a<0)
			return -a;
		return a;
	}
}
\end{lstlisting}

\begin{lstlisting}
  public static int abs(int);
    flags: ACC_PUBLIC, ACC_STATIC
    Code:
      stack=1, locals=1, args_size=1
         0: iload_0       
         1: ifge          7
         4: iload_0       
         5: ineg          
         6: ireturn       
         7: iload_0       
         8: ireturn       
\end{lstlisting}

\TT{ifge} jumps to offset 7 if the value at \ac{TOS} is greater or equal to 0.

Don't forget, any \TT{ifXX} instruction pops the value (to be compared) from the stack.


\TT{ineg} just negates value at \ac{TOS}.


Another example:

\begin{lstlisting}
	public static int min (int a, int b)
	{
		if (a>b)
			return b;
		return a;
	}
\end{lstlisting}

We get:

\begin{lstlisting}
  public static int min(int, int);
    flags: ACC_PUBLIC, ACC_STATIC
    Code:
      stack=2, locals=2, args_size=2
         0: iload_0       
         1: iload_1       
         2: if_icmple     7
         5: iload_1       
         6: ireturn       
         7: iload_0       
         8: ireturn       
\end{lstlisting}

\TT{if\_icmple} pops two values and compares them.
If the second one is lesser than (or equal to) the first, a jump to offset 7 is performed.


When we define \TT{max()} function \dots


\begin{lstlisting}
	public static int max (int a, int b)
	{
		if (a>b)
			return a;
		return b;
	}
\end{lstlisting}

\dots the resulting code is the same, but the last two \TT{iload} instructions 
(at offsets 5 and 7) are swapped:


\begin{lstlisting}
  public static int max(int, int);
    flags: ACC_PUBLIC, ACC_STATIC
    Code:
      stack=2, locals=2, args_size=2
         0: iload_0       
         1: iload_1       
         2: if_icmple     7
         5: iload_0       
         6: ireturn       
         7: iload_1       
         8: ireturn       
\end{lstlisting}

A more complex example:

\begin{lstlisting}
public class cond
{
	public static void f(int i)
	{
		if (i<100)
			System.out.print("<100");
		if (i==100)
			System.out.print("==100");
		if (i>100)
			System.out.print(">100");
		if (i==0)
			System.out.print("==0");
	}
}
\end{lstlisting}

\begin{lstlisting}
  public static void f(int);
    flags: ACC_PUBLIC, ACC_STATIC
    Code:
      stack=2, locals=1, args_size=1
         0: iload_0       
         1: bipush        100
         3: if_icmpge     14
         6: getstatic     #2        // Field java/lang/System.out:Ljava/io/PrintStream;
         9: ldc           #3        // String <100
        11: invokevirtual #4        // Method java/io/PrintStream.print:(Ljava/lang/String;)V
        14: iload_0       
        15: bipush        100
        17: if_icmpne     28
        20: getstatic     #2        // Field java/lang/System.out:Ljava/io/PrintStream;
        23: ldc           #5        // String ==100
        25: invokevirtual #4        // Method java/io/PrintStream.print:(Ljava/lang/String;)V
        28: iload_0       
        29: bipush        100
        31: if_icmple     42
        34: getstatic     #2        // Field java/lang/System.out:Ljava/io/PrintStream;
        37: ldc           #6        // String >100
        39: invokevirtual #4        // Method java/io/PrintStream.print:(Ljava/lang/String;)V
        42: iload_0       
        43: ifne          54
        46: getstatic     #2        // Field java/lang/System.out:Ljava/io/PrintStream;
        49: ldc           #7        // String ==0
        51: invokevirtual #4        // Method java/io/PrintStream.print:(Ljava/lang/String;)V
        54: return        
\end{lstlisting}

\TT{if\_icmpge} pops two values and compares them.
If the second one is larger than the first, a jump to offset 14 is performed.

\TT{if\_icmpne} and \TT{if\_icmple} work just the same, but implement different conditions.


There is also a \TT{ifne} instruction at offset 43.

Its name is misnomer, it would've be better to name it \TT{ifnz} 
(jump if the value at \ac{TOS} is not zero).

And that is what it does: it jumps to offset 54 if the input value is not zero.

If zero,the  execution flow proceeds to offset 46, where the \q{==0} string is printed.


N.B.: \ac{JVM} has no unsigned data types, so the comparison instructions operate 
only on signed integer values.

