% TODO proof-reading
\subsection{Bit-Felder}

%All bit-wise operations work just like in any other \ac{ISA}:
%
%
%\begin{lstlisting}[style=customjava]
%	public static int set (int a, int b) 
%	{
%		return a | 1<<b;
%	}
%
%	public static int clear (int a, int b) 
%	{
%		return a & (~(1<<b));
%	}
%\end{lstlisting}
%
%\begin{lstlisting}
%  public static int set(int, int);
%    flags: ACC_PUBLIC, ACC_STATIC
%    Code:
%      stack=3, locals=2, args_size=2
%         0: iload_0       
%         1: iconst_1      
%         2: iload_1       
%         3: ishl          
%         4: ior           
%         5: ireturn       
%
%  public static int clear(int, int);
%    flags: ACC_PUBLIC, ACC_STATIC
%    Code:
%      stack=3, locals=2, args_size=2
%         0: iload_0       
%         1: iconst_1      
%         2: iload_1       
%         3: ishl          
%         4: iconst_m1     
%         5: ixor          
%         6: iand          
%         7: ireturn       
%\end{lstlisting}
%
%\TT{iconst\_m1} loads $-1$ in the stack, it's the same as the \TT{0xFFFFFFFF} number.
%
%XORing with \TT{0xFFFFFFFF} has the same effect of inverting all bits
% (\myref{XOR_property}).
%
%Let's extend all data types to 64-bit \IT{long}:
%
%
%\begin{lstlisting}[style=customjava]
%	public static long lset (long a, int b) 
%	{
%		return a | 1<<b;
%	}
%
%	public static long lclear (long a, int b) 
%	{
%		return a & (~(1<<b));
%	}
%\end{lstlisting}
%
%\begin{lstlisting}
%  public static long lset(long, int);
%    flags: ACC_PUBLIC, ACC_STATIC
%    Code:
%      stack=4, locals=3, args_size=2
%         0: lload_0       
%         1: iconst_1      
%         2: iload_2       
%         3: ishl          
%         4: i2l           
%         5: lor           
%         6: lreturn       
%
%  public static long lclear(long, int);
%    flags: ACC_PUBLIC, ACC_STATIC
%    Code:
%      stack=4, locals=3, args_size=2
%         0: lload_0       
%         1: iconst_1      
%         2: iload_2       
%         3: ishl          
%         4: iconst_m1     
%         5: ixor          
%         6: i2l           
%         7: land          
%         8: lreturn       
%\end{lstlisting}
%
%The code is the same, but instructions with \IT{l} prefix are used, which operate 
%on 64-bit values.
%
%Also, the second argument of the function still is of type \IT{int}, and when the 32-bit value in it 
%needs to be promoted to 64-bit value the \TT{i2l} instruction is used, 
%which essentially extend the value of an \IT{integer} type to a \IT{long} one.
%
