\subsection{Иногда вместо массива можно использовать структуру в Си}

\subsubsection{Арифметическое среднее}

\begin{lstlisting}[style=customc]
#include <stdio.h>

int mean(int *a, int len)
{
	int sum=0;
	for (int i=0; i<len; i++)
		sum=sum+a[i];
	return sum/len;
};

struct five_ints
{
	int a0;
	int a1;
	int a2;
	int a3;
	int a4;
};

int main()
{
	struct five_ints a;
	a.a0=123;
	a.a1=456;
	a.a2=789;
	a.a3=10;
	a.a4=100;
	printf ("%d\n", mean(&a, 5));
	// test: https://www.wolframalpha.com/input/?i=mean(123,456,789,10,100)
};
\end{lstlisting}

Это работает: ф-ция \IT{mean()} никогда не будет читать за концом структуры \IT{five\_ints},
потому что передано 5, означая, что только 5 целочисленных значений будет прочитано.

\subsubsection{Сохраняем строку в структуре}

\begin{lstlisting}[style=customc]
#include <stdio.h>

struct five_chars
{
	char a0;
	char a1;
	char a2;
	char a3;
	char a4;
} __attribute__ ((aligned (1),packed));

int main()
{
	struct five_chars a;
	a.a0='h';
	a.a1='i';
	a.a2='!';
	a.a3='\n';
	a.a4=0;
	printf (&a); // §это печатает "hi!"§
};
\end{lstlisting}

Нужно использовать атрибут \IT{((aligned (1),packed))} потому что иначе, каждое поле структуры будет выровнено по 4-байтной
или 8-байтной границе.

\subsubsection{Итог}

Это просто еще один пример, как структуры и массивы сохраняются в памяти.
Вероятно, ни один программист в трезвом уме не будет делать так, как в этом примере, за исключением, может быть, какого-то
очень специального хака.
Или может быть в случае обфускации исходных текстов?

