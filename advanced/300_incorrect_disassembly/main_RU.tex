\section{Неверно дизассемблированный код}
\label{sec:incorrectly_disasmed_code}

Практикующие reverse engineer-ы часто сталкиваются с неверно дизассемблированным кодом.

\subsection{Дизассемблирование началось в неверном месте (x86)}

В отличие от ARM и MIPS (где у каждой инструкции длина или 2 или 4 байта), x86-инструкции имеют переменную длину,
так что, любой дизассемблер, начиная работу с середины x86-инструкции, может выдать неверные результаты.

Как пример:

\lstinputlisting[style=customasmx86]{\CURPATH/x86_wrong_start_RU.asm}

В начале мы видим неверно дизассемблированные инструкции, но потом, так или иначе, дизассемблер находит верный след.

\subsection{Как выглядят случайные данные в дизассемблированном виде?}

Общее, что можно сразу заметить, это:

\begin{itemize}
\item Необычно большой разброс инструкций.
\myindex{x86!\Instructions!PUSH}
\myindex{x86!\Instructions!MOV}
\myindex{x86!\Instructions!CALL}
\myindex{x86!\Instructions!IN}
\myindex{x86!\Instructions!OUT}
Самые частые x86-инструкции это \PUSH{}, \MOV{}, \CALL{}, 
но здесь мы видим
инструкции из любых групп: \ac{FPU}-инструкции, инструкции \TT{IN}/\TT{OUT}, редкие и системные инструкции, всё друг с другом смешано в одном месте.

\item Большие и случайные значения, смещения, immediates.

\item Переходы с неверными смещениями часто имеют адрес перехода в середину другой инструкции.
\end{itemize}

\lstinputlisting[caption=\randomNoise{} (x86),style=customasmx86]{\CURPATH/x86.asm}

\myindex{x86-64}
\lstinputlisting[caption=\randomNoise{} (x86-64),style=customasmx86]{\CURPATH/x64.asm}

\myindex{ARM}
\lstinputlisting[caption=\randomNoise{} (ARM (\ARMMode)),style=customasmARM]{\CURPATH/ARM.asm}

\lstinputlisting[caption=\randomNoise{} (ARM (\ThumbMode)),style=customasmARM]{\CURPATH/ARM_thumb.asm}

\myindex{MIPS}
\lstinputlisting[caption=\randomNoise{} (MIPS little endian),style=customasmMIPS]{\CURPATH/MIPS.asm}

Также важно помнить, что хитрым образом написанный код для распаковки и дешифровки (включая самомодифицирующийся),
также может выглядеть как случайный шум, тем не менее, он исполняется корректно.

% TODO таких примеров тоже бы добавить
