\subsubsection{strcat()}
\myindex{\CStandardLibrary!strcat()}

\ifdefined\RUSSIAN
Это ф-ция strcat() в том виде, в котором её сгенерировала MSVC 6.0.
Здесь видны 3 части:
1) измерение длины исходной строки (первый \INS{scasb});
2) измерение длины целевой строки (второй \INS{scasb});
3) копирование исходной строки в конец целевой (пара \INS{movsd}/\INS{movsb}).
\fi % RUSSIAN

\ifdefined\ENGLISH
This is inlined strcat() as it has been generated by MSVC 6.0.
There are 3 parts visible:
1) getting source string length (first \INS{scasb});
2) getting destination string length (second \INS{scasb});
3) copying source string into the end of destination string (\INS{movsd}/\INS{movsb} pair).
\fi % ENGLISH

\lstinputlisting[caption=strcat(),style=customasmx86]{\CURPATH/str_mem/strcat.lst}

