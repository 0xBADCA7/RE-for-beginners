\subsection{Returning string}

This is classic bug from \RobPikePractice{}:

\begin{lstlisting}
#include <stdio.h>

char* amsg(int n, char* s)
{
        char buf[100];

        sprintf (buf, "error %d: %s\n", n, s) ;

        return buf;
};

int main()
{
        amsg ("%s\n", interim (1234, "something wrong!"));
};
\end{lstlisting}

It would crash.
First, let's understand, why.

This is a stack state before amsg() return:

\begin{lstlisting}
[amsg(): 100 bytes][RA][two amsg arguments][something else][main() local variables]
                    ^^
                 current SP
\end{lstlisting}

When amsg() returns control flow to \main, so far so good.
But \printf is called from \main, which is, in turn, use stack for its own needs, zapping 100-byte buffer.
A random garbage will be printed at the best.

Hard to believe, but I know how to fix this problem:

\begin{lstlisting}
#include <stdio.h>

char* amsg(int n, char* s)
{
        char buf[100];

        sprintf (buf, "error %d: %s\n", n, s) ;

        return buf;
};

char* interim (int n, char* s)
{
        char large_buf[8000];
        // make use of local array
        // it will be optimized away otherwise
        large_buf[0]=0;
        return amsg (n, s);
};

int main()
{
        printf ("%s\n", interim (1234, "something wrong!"));
};
\end{lstlisting}

It will work if compiled by MSVC 2013 with no optimizations and with \TT{/GS-} option\footnote{Turn off buffer security check}.
MSVC will warn: ``warning C4172: returning address of local variable or temporary'', but the code will run and message
will be printed.
Let's see stack state at the moment when amsg() returns control to interim():

\begin{lstlisting}
[amsg(): 100 bytes][RA][two amsg() arguments][interim() stuff, incl. 8000 bytes][something else][main() local variables]
                    ^^
                 current SP
\end{lstlisting}

Now the stack state at the moment when interim() returns control to \main{}:

\begin{lstlisting}
[amsg(): 100 bytes][RA][two amsg() arguments][interim() stuff, incl. 8000 bytes][something else][main() local variables]
                                                                                        ^^
                                                                                     current SP
\end{lstlisting}

So when \main calls \printf, it uses stack at the place where interim()'s buffer was allocated,
and doesn't zap 100 bytes with error message inside, because 8000 bytes (or maybe much less) is just enough for everything
\printf and other descending functions do!

It may also work if there are many functions between, like:
\main $\rightarrow$ f1() $\rightarrow$ f2() $\rightarrow$ f3() ... $\rightarrow$ amsg(),
and then the result of amsg() is used in \main.
The distance between \ac{SP} in \main and address of \TT{buf[]} must be long enough,

This is why bugs like these are dangerous: sometimes your code works (and bug can be hiding unnoticed), sometimes not.
\myindex{Heisenbug}
Bugs like these are jokingly called heisenbugs or schrödinbugs\footnote{\url{https://en.wikipedia.org/wiki/Heisenbug}}.

