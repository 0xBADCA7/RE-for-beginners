\subsection{Возврат строки}

Классическая ошибка из \RobPikePractice{}:

\begin{lstlisting}[style=customc]
#include <stdio.h>

char* amsg(int n, char* s)
{
        char buf[100];

        sprintf (buf, "error %d: %s\n", n, s) ;

        return buf;
};

int main()
{
        amsg ("%s\n", interim (1234, "something wrong!"));
};
\end{lstlisting}

Она упадет.
В начале, попытаемся понять, почему.

Это состояние стека перед возвратом из amsg():

\begin{lstlisting}
[amsg(): 100 байт][RA][два аргумента amsg][что-то еще][локальные переменные main()]
                   ^^
                текущий SP
\end{lstlisting}

Когда управление возвращается из amsg() в \main, пока всё хорошо.
Но когда \printf вызывается из \main, который, в свою очередь, использует стек для своих нужд, затирая 100-байтный буфер.
В лучшем случае, будет выведен случайный мусор.

Трудно поверить, но я знаю, как это исправить:

\begin{lstlisting}[style=customc]
#include <stdio.h>

char* amsg(int n, char* s)
{
        char buf[100];

        sprintf (buf, "error %d: %s\n", n, s) ;

        return buf;
};

char* interim (int n, char* s)
{
        char large_buf[8000];
        // используем локальный массив.
        // а иначе компилятор выбросит его при оптимизации, как неиспользуемый.
        large_buf[0]=0;
        return amsg (n, s);
};

int main()
{
        printf ("%s\n", interim (1234, "something wrong!"));
};
\end{lstlisting}

Это заработает если скомпилировано в MSVC 2013 без оптимизаций и с опцией \TT{/GS-}\footnote{Выключить защиту от переполнения буфера}.
MSVC предупредит: ``warning C4172: returning address of local variable or temporary'', но код запустится и сообщение выведется.
Посмотрим состояние стека в момент, когда amsg() возвращает управление в interim():

\begin{lstlisting}
[amsg(): 100 байт][RA][два аргумента amsg()][вледения interim(), включая 8000 байт][еще что-то][локальные переменные main()]
                   ^^
                текущий SP
\end{lstlisting}

Теперь состояние стека на момент, когда interim() возвращает управление в \main{}:

\begin{lstlisting}
[amsg(): 100 байт][RA][два аргумента amsg()][вледения interim(), включая 8000 байт][еще что-то][локальные переменные main()]
                                                                                        ^^
                                                                                     текущий SP
\end{lstlisting}

Так что когда \main вызывает \printf, он использует стек в месте, где выделен буфер в interim(),
и не затирает 100 байт с сообщение об ошибке внутри, потому что 8000 байт (или может быть меньше) это достаточно для всего,
что делает \printf и другие нисходящие ф-ции!

Это также может сработать, если между ними много ф-ций, например:
\main $\rightarrow$ f1() $\rightarrow$ f2() $\rightarrow$ f3() ... $\rightarrow$ amsg(),
и тогда результат amsg() используется в \main.
Дистанция между \ac{SP} в \main и адресом буфера \TT{buf[]} должна быть достаточно длинной.

Вот почему такие ошибки опасны: иногда ваш код работает (и бага прячется незамеченной). иногда нет.
\label{heisenbug}
\myindex{Хейзенбаги}
Такие баги в шутку называют хейзенбаги или шрёдинбаги\footnote{\url{https://en.wikipedia.org/wiki/Heisenbug}}.

