\subsection{Accessing arguments/local variables of caller}

From C/C++ basics we know that this is impossible for a function to access arguments of caller function or its
local variables.

Nevertheless, it's possible using dirty hacks.
For example:

\begin{lstlisting}[style=customc]
	#include <stdio.h>

	void f(char *text)
	{
		// print stack
		int *tmp=&text;
		for (int i=0; i<20; i++)
		{
			printf ("0x%x\n", *tmp);
			tmp++;
		};
	};

	void draw_text(int X, int Y, char* text)
	{
		f(text);

		printf ("We are going to draw [%s] at %d:%d\n", text, X, Y);
	};

	int main()
	{
		printf ("address of main()=0x%x\n", &main);
		printf ("address of draw_text()=0x%x\n", &draw_text);
		draw_text(100, 200, "Hello!");
	};
\end{lstlisting}

On 32-bit Ubuntu 16.04 and GCC 5.4.0, I got this:

\begin{lstlisting}
	address of main()=0x80484f8
	address of draw_text()=0x80484cb
	0x8048645	first argument to f()
	0x8048628
	0xbfd8ab98
	0xb7634590
	0xb779eddc
	0xb77e4918
	0xbfd8aba8
	0x8048547	return address into the middle of main()
	0x64		first argument to draw_text()
	0xc8		second argument to draw_text()
	0x8048645	third argument to draw_text()
	0x8048581
	0xb779d3dc
	0xbfd8abc0
	0x0
	0xb7603637
	0xb779d000
	0xb779d000
	0x0
	0xb7603637
\end{lstlisting}

(Comments are mine.)

Since \IT{f()} starting to enumerate stack elements at its first argument, the first stack element is indeed a pointer
to \q{Hello!} string. We see its address is also used as third argument to \IT{draw\_text()} function.

In \IT{f()} we could read all functions arguments and local variables if we know exact stack layout, but it's always
changed, from compiler to compiler.
Various optimization levels affects stack layout greatly.

But if we can somehow detect information we need, we can use it and even modify it.
As an example, I'll rework \IT{f()} function:

\begin{lstlisting}[style=customc]
	void f(char *text)
	{
		...

		// find 100, 200 values pair and modify the second on
		tmp=&text;
		for (int i=0; i<20; i++)
		{
			if (*tmp==100 && *(tmp+1)==200)
			{
				printf ("found\n");
				*(tmp+1)=210; // change 200 to 210
				break;
			};
			tmp++;
		};
	};
\end{lstlisting}

Holy moly, it works:

\begin{lstlisting}
	found
	We are going to draw [Hello!] at 100:210
\end{lstlisting}

\myparagraph{Summary}

It's extremely dirty hack, intended to demonstrate stack internals.
I never ever seen or heard that anyone used this in a real code.
But still, this is a good example.

\myparagraph{\Exercise}

The example has been compiled without optimization on 32-bit Ubuntu using GCC 5.4.0 and it works.
But when I turn on \TT{-O3} maximum optimization, it's failed.
Try to find why.

Use your favorite compiler and OS, try various optimization levels, find if it's works and if it's not, find why.

