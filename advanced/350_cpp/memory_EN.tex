\subsection{Memory}

Sometimes you may hear from C++ programmers \q{allocate memory on stack} and/or \q{allocate memory on \gls{heap}}.

Allocating object \IT{on stack}:

\begin{lstlisting}[style=customc]
void f()
{
	...

	Class o=Class(...);

	...
};
\end{lstlisting}

The memory for object (or structure) is allocated in stack, using simple \ac{SP} shift.
The memory is deallocated upon function exit, or, more precisely, at the end of \IT{scope}---\ac{SP} is returning to its state (same as at the start of function) and destructor of \IT{Class} is called.
In the same manner, memory for allocated structure in C is deallocated upon function exit.

Allocating object \IT{on \gls{heap}}:

\begin{lstlisting}[style=customc]
void f1()
{
	...

	Class *o=new Class(...);

	...
};

void f2()
{
	...

	delete o;

	...
};
\end{lstlisting}

\myindex{\CStandardLibrary!malloc()}
\myindex{\CStandardLibrary!free()}
This is the same as allocating memory for a structure using \IT{malloc()} call.
In fact, \IT{new} in C++ is wrapper for \IT{malloc()}, and \IT{delete} is wrapper for \IT{free()}.
Since memory block has been allocated in \gls{heap}, it must be deallocated explicitly, using \IT{delete}.
Class destructor will be automatically called right before that moment.

Which method is better?
Allocating \IT{on stack} is very fast, and good for small, short-lived object, which will be used only in the current function.

Allocating \IT{on heap} is slower, and better for long-lived object, which will be used across many functions.
Also, objects allocated in \gls{heap} are prone to memory leakage, because they must to be freed explicitly, but one can
forget about it.

Anyway, this is matter of taste.
