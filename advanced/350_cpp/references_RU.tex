\subsection{References}
\myindex{\Cpp!References}
\label{cpp_references}

References в \Cpp это тоже указатели (\myref{label_pointers}), но их называют \IT{безопасными} (safe), потому что работая с ними, труднее сделать ошибку (C++11 8.3.2). % FIXME link
Например, reference всегда должен указывать объект того же типа и не может быть NULL
\InSqBrackets{\ParashiftCPPFAQ 8.6}.

Более того, reference нельзя менять, нельзя его заставить указывать на другой объект (reseat)
\InSqBrackets{\ParashiftCPPFAQ 8.5}.

Если мы попробуем изменить пример с указателями (\myref{label_pointers}) 
чтобы он использовал reference вместо указателей \dots

\begin{lstlisting}[style=customc]
void f2 (int x, int y, int & sum, int & product)
{
	sum=x+y;
	product=x*y;
};
\end{lstlisting}

\dots то выяснится, что скомпилированный код абсолютно такой 
же как и в примере с указателями (\myref{label_pointers}):

\begin{lstlisting}[caption=\Optimizing MSVC 2010,style=custom]
_x$ = 8		; size = 4
_y$ = 12	; size = 4
_sum$ = 16	; size = 4
_product$ = 20	; size = 4
?f2@@YAXHHAAH0@Z PROC	; f2
	mov	ecx, DWORD PTR _y$[esp-4]
	mov	eax, DWORD PTR _x$[esp-4]
	lea	edx, DWORD PTR [eax+ecx]
	imul eax, ecx
	mov ecx, DWORD PTR _product$[esp-4]
	push esi
	mov	esi, DWORD PTR _sum$[esp]
	mov	DWORD PTR [esi], edx
	mov	DWORD PTR [ecx], eax
	pop	esi
	ret	0
?f2@@YAXHHAAH0@Z ENDP	; f2
\end{lstlisting}

(Почему у функций в \Cpp такие странные имена, описано здесь: \myref{namemangling}.)

Следовательно, references в С++ эффективны настолько, насколько и обычные указатели.

