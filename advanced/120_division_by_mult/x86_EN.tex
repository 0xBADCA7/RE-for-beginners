\subsection{x86}

\dots is compiled in a very predictable way:

\lstinputlisting[caption=MSVC]{\CURPATH/11_1_msvc_EN.asm}

\myindex{x86!\Instructions!IDIV}

\IDIV divides the 64-bit number stored in the \TT{EDX:EAX} register pair by the value in the \ECX.
As a result, \EAX will contain the \gls{quotient}, and \EDX --- the remainder.
The result is returned from the \ttf function in the \EAX register, 
so the value is not moved after the division 
operation, it is in right place already.

Since \IDIV uses the value in the \TT{EDX:EAX} register pair, the \TT{CDQ} instruction (before \IDIV) extends 
the value in \EAX to a 64-bit value taking its sign into account, just as \MOVSX does.

If we turn optimization on (\Ox), we get:

\lstinputlisting[caption=\Optimizing MSVC]{\CURPATH/11_1_msvc_Ox.asm}

This is division by multiplication. Multiplication operations work much faster. 
And it is possible to use this trick
\footnote{Read more about division by multiplication in \InSqBrackets{\HenryWarren 10-3}}
to produce code which is effectively equivalent and faster.

This is also called \q{strength reduction} in compiler optimizations.

GCC 4.4.1 generates almost the same code even without additional optimization flags, 
just like MSVC with optimization turned on:

\lstinputlisting[caption=\NonOptimizing GCC 4.4.1]{\CURPATH/11_2_gcc.asm}
