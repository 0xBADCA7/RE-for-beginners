\subsection{Getting the divisor}

\subsubsection{Variant \#1}

Often, the code looks like this:

\lstinputlisting{\CURPATH/form_EN.asm}

Let's denote the 32-bit \IT{magic} coefficient as $M$, the shifting coefficient as $C$ and the divisor as $D$.

The divisor we want to get is:

\[
D=\frac{2^{32 + C}}{M}
\]

For example:

\lstinputlisting[caption=\Optimizing MSVC 2012]{\CURPATH/ex1.asm}

This is:

\[
D=\frac{2^{32 + 3}}{2021161081}
\]

\myindex{Wolfram Mathematica}

The numbers are larger than 32-bit, so we can use Wolfram Mathematica for convenience:

\begin{lstlisting}[caption=Wolfram Mathematica]
In[1]:=N[2^(32+3)/2021161081]
Out[1]:=17.
\end{lstlisting}

So the divisor from the code we used as example is 17.

For x64 division, things 
are the same, but $2^{64}$ has to be used instead of $2^{32}$:

\begin{lstlisting}
uint64_t f1234(uint64_t a)
{
	return a/1234;
};
\end{lstlisting}

\begin{lstlisting}[caption=\Optimizing MSVC 2012 x64]
f1234	PROC
	mov	rax, 7653754429286296943 ; 6a37991a23aead6fH
	mul	rcx
	shr	rdx, 9
	mov	rax, rdx
	ret	0
f1234	ENDP
\end{lstlisting}

\begin{lstlisting}[caption=Wolfram Mathematica]
In[1]:=N[2^(64+9)/16^^6a37991a23aead6f]
Out[1]:=1234.
\end{lstlisting}

\subsubsection{Variant \#2}

A variant with omitted arithmetic shift also exist:

\begin{lstlisting}
		mov     eax, 55555556h ; 1431655766
		imul    ecx
		mov     eax, edx
		shr     eax, 1Fh
\end{lstlisting}

The method of getting divisor is simplified:

\[
D=\frac{2^{32}}{M}
\]

As of my example, this is:

\[
D=\frac{2^{32}}{1431655766}
\]

\myindex{Wolfram Mathematica}

And again we use Wolfram Mathematica:

\begin{lstlisting}[caption=Wolfram Mathematica]
In[1]:=N[2^32/16^^55555556]
Out[1]:=3.
\end{lstlisting}

The divisor is 3.
