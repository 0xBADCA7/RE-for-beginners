\section{Обрезка строк}
\newcommand{\CRLF}{\ac{CR}/\ac{LF}}

Весьма востребованная операция со строками --- это удаление некоторых символов в начале и/или конце
строки.

В этом примере, мы будем работать с функцией, удаляющей все символы перевода строки 
(\CRLF{}) в конце входной строки:

\lstinputlisting[style=customc]{\CURPATH/strtrim_RU.c}

Входной аргумент всегда возвращается на выходе, это удобно, когда вам нужно объединять
функции обработки строк в цепочки, как это сделано здесь в функции \main.

\myindex{\CLanguageElements!Short-circuit}
Вторая часть for() (\TT{str\_len>0 \&\& (c=s[str\_len-1])}) называется в \CCpp \q{short-circuit} 
(короткое замыкание) и это очень удобно: \InSqBrackets{\CNotes 1.3.8}.

Компиляторы \CCpp гарантируют последовательное вычисление слева направо.

Так что если первое условие не истинно после вычисления, второе никогда не будет
вычисляться.

% subsections
\subsection{x64}

\myindex{x86-64}
В x86-64 всё немного иначе, здесь аргументы функции (4 или 6) передаются через регистры, 
а \gls{callee} из читает их из регистров, а не из стека.

\subsubsection{MSVC}

\Optimizing MSVC:

\lstinputlisting[caption=\Optimizing MSVC 2012 x64,style=customasm]{patterns/05_passing_arguments/x64_MSVC_Ox_RU.asm}

Как видно, очень компактная функция \ttf берет аргументы прямо из регистров.

Инструкция \LEA используется здесь для сложения чисел. 
Должно быть компилятор посчитал, что это будет эффективнее использования \TT{ADD}.

\myindex{x86!\Instructions!LEA}
В самой \main{} \LEA{} также используется для подготовки первого и третьего аргумента: должно быть,
компилятор решил, что \LEA{} будет работать здесь быстрее, чем загрузка значения в регистр при помощи \MOV.

Попробуем посмотреть вывод неоптимизирующего MSVC:

\lstinputlisting[caption=MSVC 2012 x64,style=customasm]{patterns/05_passing_arguments/x64_MSVC_IDA_RU.asm}

Немного путанее: все 3 аргумента из регистров зачем-то сохраняются в стеке.

\myindex{Shadow space}
\label{shadow_space}
Это называется \q{shadow space} \footnote{\href{http://go.yurichev.com/17256}{MSDN}}: 
каждая функция в Win64 может (хотя и не обязана) сохранять значения 4-х регистров там.

Это делается по крайней мере из-за двух причин: 
1) в большой функции отвести целый регистр (а тем более 4 регистра) для входного аргумента 
слишком расточительно, так что к нему будет обращение через стек;

2) отладчик всегда знает, где найти аргументы функции в момент останова
\footnote{\href{http://go.yurichev.com/17257}{MSDN}}.

Так что, какие-то большие функции могут сохранять входные аргументы в \q{shadows space} 
для использования в будущем, а небольшие функции, как наша, могут этого и не делать.

Место в стеке для \q{shadow space} выделяет именно \gls{caller}.

\subsubsection{GCC}

\Optimizing GCC также делает понятный код:

\lstinputlisting[caption=\Optimizing GCC 4.4.6 x64,style=customasm]{patterns/05_passing_arguments/x64_GCC_O3_RU.s}

\NonOptimizing GCC:

\lstinputlisting[caption=GCC 4.4.6 x64,style=customasm]{patterns/05_passing_arguments/x64_GCC_RU.s}

\myindex{Shadow space}
В соглашении о вызовах System V *NIX (\SysVABI) нет \q{shadow space}, но \gls{callee} тоже иногда
должен сохранять где-то аргументы, потому что, опять же, регистров может и не хватить на все действия.
Что мы здесь и видим.

\subsubsection{GCC: uint64\_t вместо int}

Наш пример работал с 32-битным \Tint, поэтому использовались 32-битные части регистров с префиксом \TT{E-}.

Его можно немного переделать, чтобы он заработал с 64-битными значениями:

\lstinputlisting[style=customc]{patterns/05_passing_arguments/ex64.c}

\lstinputlisting[caption=\Optimizing GCC 4.4.6 x64,style=customasm]{patterns/05_passing_arguments/ex64_GCC_O3_IDA_RU.asm}

Собствено, всё то же самое, только используются регистры \IT{целиком}, с префиксом \TT{R-}.


\subsubsection{ARM64}

\myparagraph{GCC}

Компилируем пример в GCC 4.8.1 для ARM64:

\lstinputlisting[numbers=left,label=hw_ARM64_GCC,caption=\NonOptimizing GCC 4.8.1 + objdump,style=customasmARM]{patterns/01_helloworld/ARM/hw.lst}

В ARM64 нет режима Thumb и Thumb-2, только ARM, так что тут только 32-битные инструкции.

Регистров тут в 2 раза больше: \myref{ARM64_GPRs}.
64-битные регистры теперь имеют префикс 
\TT{X-}, а их 32-битные части --- \TT{W-}.

\myindex{ARM!\Instructions!STP}
Инструкция \TT{STP} (\IT{Store Pair}) 
сохраняет в стеке сразу два регистра: \RegX{29} и \RegX{30}.
Конечно, эта инструкция может сохранять эту пару где угодно в памяти, но здесь указан регистр \ac{SP}, так что
пара сохраняется именно в стеке.

Регистры в ARM64 64-битные, каждый имеет длину в 8 байт, так что для хранения двух регистров нужно именно 16 байт.

Восклицательный знак (``!'') после операнда означает, что сначала от \ac{SP} будет отнято 16 и только затем
значения из пары регистров будут записаны в стек.

Это называется \IT{pre-index}.
Больше о разнице между \IT{post-index} и \IT{pre-index} 
описано здесь: \myref{ARM_postindex_vs_preindex}.

Таким образом, в терминах более знакомого всем процессора x86, первая инструкция~--- это просто аналог 
пары инструкций \TT{PUSH X29} и \TT{PUSH X30}.
\RegX{29} в ARM64 используется как \ac{FP}, а \RegX{30} 
как \ac{LR}, поэтому они сохраняются в прологе функции и
восстанавливаются в эпилоге.

Вторая инструкция копирует \ac{SP} в \RegX{29} (или \ac{FP}).
Это нужно для установки стекового фрейма функции.

\label{pointers_ADRP_and_ADD}
\myindex{ARM!\Instructions!ADRP/ADD pair}
Инструкции \TT{ADRP} и \ADD нужны для формирования адреса строки \q{Hello!} в регистре \RegX{0}, 
ведь первый аргумент функции передается через этот регистр.
Но в ARM нет инструкций, при помощи которых можно записать в регистр длинное число 
(потому что сама длина инструкции ограничена 4-я байтами. Больше об этом здесь: \myref{ARM_big_constants_loading}).
Так что нужно использовать несколько инструкций.
Первая инструкция (\TT{ADRP}) записывает в \RegX{0} адрес 4-килобайтной страницы где находится строка, 
а вторая (\ADD) просто прибавляет к этому адресу остаток.
Читайте больше об этом: \myref{ARM64_relocs}.

\TT{0x400000 + 0x648 = 0x400648}, и мы видим, что в секции данных \TT{.rodata} по этому адресу как раз находится наша
Си-строка \q{Hello!}.

\myindex{ARM!\Instructions!BL}
Затем при помощи инструкции \TT{BL} вызывается \puts. Это уже рассматривалось ранее: \myref{puts}.

Инструкция \MOV записывает 0 в \RegW{0}. 
\RegW{0} это младшие 32 бита 64-битного регистра \RegX{0}:

\begin{center}
\begin{tabular}{ | l | l | }
\hline
\RU{Старшие 32 бита}\EN{High 32-bit part}\ES{Parte alta de 32 bits}\PTBRph{}\PLph{}\ITAph{}\DEph{}\THAph{}\NLph{}\FR{Partie 32 bits haute} & \RU{младшие 32 бита}\EN{low 32-bit part}\ES{parte baja de 32 bits}\PTBRph{}\PLph{Starsze 32 bity}\ITAph{}\DEph{}\THAph{}\NLph{}\FR{Partie 32 bits basse} \\
\hline
\multicolumn{2}{ | c | }{X0} \\
\hline
\multicolumn{1}{ | c | }{} & \multicolumn{1}{ c | }{W0} \\
\hline
\end{tabular}
\end{center}


А результат функции возвращается через \RegX{0}, и \main возвращает 0, 
так что вот так готовится возвращаемый результат.

Почему именно 32-битная часть?
Потому что в ARM64, как и в x86-64, тип \Tint оставили 32-битным, для лучшей совместимости.

Следовательно, раз уж функция возвращает 32-битный \Tint, то нужно заполнить только 32 младших бита регистра \RegX{0}.

Для того, чтобы удостовериться в этом, немного отредактируем этот пример и перекомпилируем его.%

Теперь \main возвращает 64-битное значение:

\begin{lstlisting}[caption=\main возвращающая значение типа \TT{uint64\_t},style=customc]
#include <stdio.h>
#include <stdint.h>

uint64_t main()
{
        printf ("Hello!\n");
        return 0;
}
\end{lstlisting}

Результат точно такой же, только \MOV в той строке теперь выглядит так:

\begin{lstlisting}[caption=\NonOptimizing GCC 4.8.1 + objdump]
  4005a4:       d2800000        mov     x0, #0x0      // #0
\end{lstlisting}

\myindex{ARM!\Instructions!LDP}
Далее при помощи инструкции \INS{LDP} (\IT{Load Pair}) восстанавливаются регистры \RegX{29} и \RegX{30}.

Восклицательного знака после инструкции нет. Это означает, что сначала значения достаются из стека, и только потом \ac{SP} увеличивается на 16.

Это называется \IT{post-index}.

\myindex{ARM!\Instructions!RET}
В ARM64 есть новая инструкция: \RET. 
Она работает так же как и \INS{BX LR}, но там добавлен специальный бит,
подсказывающий процессору, что это именно выход из функции, а не просто переход, чтобы процессор
мог более оптимально исполнять эту инструкцию.

Из-за простоты этой функции оптимизирующий GCC генерирует точно такой же код.


\subsection{ARM: \OptimizingKeilVI (\ARMMode)}

И снова, компилятор пользуется условными инструкциями в режиме ARM, поэтому код более 
компактный.

\lstinputlisting[caption=\OptimizingKeilVI (\ARMMode),style=customasm]{\CURPATH/Keil_ARM_O3_RU.s}

\subsection{ARM: \OptimizingKeilVI (\ThumbMode)}
\myindex{\CompilerAnomaly}
\label{Keil_anomaly}

В режиме Thumb куда меньше условных инструкций, так что код более простой.

Но здесь есть одна странность со сдвигами на 0x20 и 0x1F (строки 22 и 23).

Почему компилятор Keil сделал так?
Честно говоря, трудно сказать.
Возможно, это выверт процесса оптимизации компилятора.

Тем не менее, код будет работать корректно.

\lstinputlisting[caption=\OptimizingKeilVI (\ThumbMode),numbers=left,style=customasm]{\CURPATH/Keil_thumb_O3_RU.s}


\subsubsection{MIPS}

Одна отличительная особенность MIPS это отсутствие регистра флагов.
Очевидно, так было сделано для упрощения анализа зависимости данных (data dependency).

\myindex{x86!\Instructions!SETcc}
\myindex{MIPS!\Instructions!SLT}
\myindex{MIPS!\Instructions!SLTU}
Так что здесь есть инструкция, похожая на \INS{SETcc} в x86: \INS{SLT} (\q{Set on Less Than}~--- установить если
меньше чем, знаковая версия) и \INS{SLTU} (беззнаковая версия).
Эта инструкция устанавливает регистр-получатель в 1 если условие верно или в 0 в противном случае.

\myindex{MIPS!\Instructions!BEQ}
\myindex{MIPS!\Instructions!BNE}
Затем регистр-получатель проверяется, используя инструкцию 
\INS{BEQ} (\q{Branch on Equal} --- переход если равно) или \INS{BNE} (\q{Branch on Not Equal} --- переход если не равно) 
и может произойти переход.
Так что эта пара инструкций должна использоваться в MIPS для сравнения и перехода.
Начнем с знаковой версии нашей функции:

\lstinputlisting[caption=\NonOptimizing GCC 4.4.5 (IDA),style=customasmMIPS]{patterns/07_jcc/simple/O0_MIPS_signed_IDA_RU.lst}

\INS{SLT REG0, REG0, REG1} сокращается в IDA до более короткой формы \INS{SLT REG0, REG1}.
\myindex{MIPS!\Pseudoinstructions!BEQZ}
Мы также видим здесь псевдоинструкцию \INS{BEQZ} (\q{Branch if Equal to Zero}~--- переход если равно нулю), 
которая, на самом деле, \INS{BEQ REG, \$ZERO, LABEL}.

\myindex{MIPS!\Instructions!SLTU}
Беззнаковая версия точно такая же, только здесь используется \INS{SLTU} (беззнаковая версия, 
отсюда \q{U} в названии) вместо \INS{SLT}:

\lstinputlisting[caption=\NonOptimizing GCC 4.4.5 (IDA),style=customasmMIPS]{patterns/07_jcc/simple/O0_MIPS_unsigned_IDA.lst}



