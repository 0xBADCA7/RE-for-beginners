\subsection{x64: \Optimizing MSVC 2013}

\lstinputlisting[caption=\Optimizing MSVC 2013 x64,style=customasmx86]{\CURPATH/MSVC2013_x64_Ox_EN.asm}

First, MSVC inlined the \strlen{} function code, because it concluded this 
is to be faster than the usual \strlen{} work + the cost of calling it and returning from it.
This is called inlining: \myref{inline_code}.

\myindex{x86!\Instructions!OR}
\myindex{\CStandardLibrary!strlen()}
\label{using_OR_instead_of_MOV}
The first instruction of the inlined \strlen{} is\\
\TT{OR RAX, 0xFFFFFFFFFFFFFFFF}. 

MSVC often uses \TT{OR} instead of \TT{MOV RAX, 0xFFFFFFFFFFFFFFFF}, because resulting opcode is shorter.

And of course, it is equivalent: all bits are set, and a number with all bits set is $-1$ 
in two's complement arithmetic: \myref{sec:signednumbers}.

Why would the $-1$ number be used in \strlen{}, one might ask.
Due to optimizations, of course.
Here is the code that MSVC generated:

\lstinputlisting[caption=Inlined \strlen{} by MSVC 2013 x64,style=customasmx86]{\CURPATH/strlen_MSVC_EN.asm}

Try to write shorter if you want to initialize the counter at 0!
OK, let' try:

\lstinputlisting[caption=Our version of \strlen{},style=customasmx86]{\CURPATH/my_strlen_EN.asm}

We failed. We have to use additional \INS{JMP} instruction!

So what the MSVC 2013 compiler did is to move the \TT{INC} instruction to the place before 
the actual character loading.

If the first character is 0, that's OK, \RAX is 0 at this moment, 
so the resulting string length is 0.

The rest in this function seems easy to understand.

\subsection{x64: \NonOptimizing GCC 4.9.1}

\lstinputlisting[style=customasmx86]{\CURPATH/GCC491_x64_O0_EN.asm}

Comments are added by the author of the book.

After the execution of \strlen{}, the control is passed to the L2 label, 
and there two clauses are checked, one after another.

\myindex{\CLanguageElements!Short-circuit}
The second will never be checked, if the first one (\IT{str\_len==0}) is false 
(this is \q{short-circuit}).

Now let's see this function in short form:

\begin{itemize}
\item First for() part (call to \strlen{})
\item goto L2
\item L5: for() body. goto exit, if needed
\item for() third part (decrement of str\_len)
\item L2: 
for() second part: check first clause, then second. goto loop body begin or exit.
\item L4: // exit
\item return s
\end{itemize}

\subsection{x64: \Optimizing GCC 4.9.1}
\label{string_trim_GCC_x64_O3}

\lstinputlisting[style=customasmx86]{\CURPATH/GCC491_x64_O3_EN.asm}

Now this is more complex.

The code before the loop's body start is executed only once, but it has the \CRLF{} 
characters check too!
What is this code duplication for?

The common way to implement the main loop is probably this:

\begin{itemize}
\item (loop start) check for 
\CRLF{} characters, make decisions
\item store zero character
\end{itemize}

But GCC has decided to reverse these two steps. 

Of course, \IT{store zero character} cannot be first step, so another check is needed:

\begin{itemize}
\item workout first character. match it to \CRLF{}, exit if character is not \CRLF{}

\item (loop begin) store zero character

\item check for \CRLF{} characters, make decisions
\end{itemize}

Now the main loop is very short, which is good for latest \ac{CPU}s.

The code doesn't use the str\_len variable, but str\_len-1.
So this is more like an index in a buffer.

Apparently, GCC notices that the str\_len-1 statement is used twice.

So it's better to allocate a variable which always holds a value that's smaller than 
the current string length by one, 
and decrement it (this is the same effect as decrementing the str\_len variable).
