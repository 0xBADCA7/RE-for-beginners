\subsection{Издевательство над указателями в ядре Windows}

Секция ресурсов в исполняемых файлах типа PE в Windows это секция, содержащая картинки, иконки, строки, итд.
Ранние версии Windows позволяли иметь к ним доступ только при помощи идентификаторов, но потом в Microsoft добавили
также и способ адресовать ресурсы при помощи строк.

Так что потом стало возможным передать идентификатор или строку в ф-цию
\href{https://msdn.microsoft.com/en-us/library/windows/desktop/ms648042%28v=vs.85%29.aspx}{FindResource()}.
Которая декларирована вот так:

\myindex{win32!FindResource()}

\begin{lstlisting}[style=customc]
HRSRC WINAPI FindResource(
  _In_opt_ HMODULE hModule,
  _In_     LPCTSTR lpName,
  _In_     LPCTSTR lpType
);
\end{lstlisting}

\IT{lpName} и \IT{lpType} имеют тип \IT{char*} или \IT{wchar*}, и когда кто-то всё еще хочет передать идентификатор,
нужно использовать макрос
\href{https://msdn.microsoft.com/en-us/library/windows/desktop/ms648029%28v=vs.85%29.aspx}{MAKEINTRESOURCE}, вот так:

\myindex{win32!MAKEINTRESOURCE()}

\begin{lstlisting}[style=customc]
result = FindResource(..., MAKEINTRESOURCE(1234), ...);
\end{lstlisting}

Очень интересно то, что всё что делает MAKEINTRESOURCE это приводит целочисленное к указателю.
В MSVC 2013, в файле\\
\IT{Microsoft SDKs\textbackslash{}Windows\textbackslash{}v7.1A\textbackslash{}Include\textbackslash{}Ks.h},
мы можем найти это:

\begin{lstlisting}[style=customc]
...

#if (!defined( MAKEINTRESOURCE )) 
#define MAKEINTRESOURCE( res ) ((ULONG_PTR) (USHORT) res)
#endif

...
\end{lstlisting}

Звучит безумно. Заглянем внутрь древнего, когда-то утекшего, исходного кода Windows NT4.
В \IT{private/windows/base/client/module.c} мы можем найти исходный код \IT{FindResource()}:

\begin{lstlisting}[style=customc]
HRSRC
FindResourceA(
    HMODULE hModule,
    LPCSTR lpName,
    LPCSTR lpType
    )

...

{
    NTSTATUS Status;
    ULONG IdPath[ 3 ];
    PVOID p;

    IdPath[ 0 ] = 0;
    IdPath[ 1 ] = 0;
    try {
        if ((IdPath[ 0 ] = BaseDllMapResourceIdA( lpType )) == -1) {
            Status = STATUS_INVALID_PARAMETER;
            }
        else
        if ((IdPath[ 1 ] = BaseDllMapResourceIdA( lpName )) == -1) {
            Status = STATUS_INVALID_PARAMETER;
...
\end{lstlisting}

Посмотрим в \IT{BaseDllMapResourceIdA()} в том же исходном файле:

\begin{lstlisting}[style=customc]
ULONG
BaseDllMapResourceIdA(
    LPCSTR lpId
    )
{
    NTSTATUS Status;
    ULONG Id;
    UNICODE_STRING UnicodeString;
    ANSI_STRING AnsiString;
    PWSTR s;

    try {
        if ((ULONG)lpId & LDR_RESOURCE_ID_NAME_MASK) {
            if (*lpId == '#') {
                Status = RtlCharToInteger( lpId+1, 10, &Id );
                if (!NT_SUCCESS( Status ) || Id & LDR_RESOURCE_ID_NAME_MASK) {
                    if (NT_SUCCESS( Status )) {
                        Status = STATUS_INVALID_PARAMETER;
                        }
                    BaseSetLastNTError( Status );
                    Id = (ULONG)-1;
                    }
                }
            else {
                RtlInitAnsiString( &AnsiString, lpId );
                Status = RtlAnsiStringToUnicodeString( &UnicodeString,
                                                       &AnsiString,
                                                       TRUE
                                                     );
                if (!NT_SUCCESS( Status )){
                    BaseSetLastNTError( Status );
                    Id = (ULONG)-1;
                    }
                else {
                    s = UnicodeString.Buffer;
                    while (*s != UNICODE_NULL) {
                        *s = RtlUpcaseUnicodeChar( *s );
                        s++;
                        }

                    Id = (ULONG)UnicodeString.Buffer;
                    }
                }
            }
        else {
            Id = (ULONG)lpId;
            }
        }
    except (EXCEPTION_EXECUTE_HANDLER) {
        BaseSetLastNTError( GetExceptionCode() );
        Id =  (ULONG)-1;
        }
    return Id;
}
\end{lstlisting}

К \IT{lpId} применяется операция ``И'' с \IT{LDR\_RESOURCE\_ID\_NAME\_MASK}. Маску можно найти в \IT{public/sdk/inc/ntldr.h}:

\begin{lstlisting}[style=customc]
...

#define LDR_RESOURCE_ID_NAME_MASK 0xFFFF0000

...
\end{lstlisting}

Так что к \IT{lpId} применяется операция ``И'' c \IT{0xFFFF0000}, и если присутствуют какие-либо биты за младшими 16 битами,
исполняется первая часто ф-ции и (\IT{lpId} принимается за адрес строки).
Иначе --- вторая часть ф-ции (\IT{lpId} принимается за 16-битное значение).

Этот же код можно найти и в Windows 7, в файле kernel32.dll:

\begin{lstlisting}[style=customasmx86]
....

.text:0000000078D24510 ; __int64 __fastcall BaseDllMapResourceIdA(PCSZ SourceString)
.text:0000000078D24510 BaseDllMapResourceIdA proc near         ; CODE XREF: FindResourceExA+34
.text:0000000078D24510                                         ; FindResourceExA+4B
.text:0000000078D24510
.text:0000000078D24510 var_38          = qword ptr -38h
.text:0000000078D24510 var_30          = qword ptr -30h
.text:0000000078D24510 var_28          = _UNICODE_STRING ptr -28h
.text:0000000078D24510 DestinationString= _STRING ptr -18h
.text:0000000078D24510 arg_8           = dword ptr  10h
.text:0000000078D24510
.text:0000000078D24510 ; FUNCTION CHUNK AT .text:0000000078D42FB4 SIZE 000000D5 BYTES
.text:0000000078D24510
.text:0000000078D24510                 push    rbx
.text:0000000078D24512                 sub     rsp, 50h
.text:0000000078D24516                 cmp     rcx, 10000h
.text:0000000078D2451D                 jnb     loc_78D42FB4
.text:0000000078D24523                 mov     [rsp+58h+var_38], rcx
.text:0000000078D24528                 jmp     short $+2
.text:0000000078D2452A ; ---------------------------------------------------------------------------
.text:0000000078D2452A
.text:0000000078D2452A loc_78D2452A:                           ; CODE XREF: BaseDllMapResourceIdA+18
.text:0000000078D2452A                                         ; BaseDllMapResourceIdA+1EAD0
.text:0000000078D2452A                 jmp     short $+2
.text:0000000078D2452C ; ---------------------------------------------------------------------------
.text:0000000078D2452C
.text:0000000078D2452C loc_78D2452C:                           ; CODE XREF: BaseDllMapResourceIdA:loc_78D2452A
.text:0000000078D2452C                                         ; BaseDllMapResourceIdA+1EB74
.text:0000000078D2452C                 mov     rax, rcx
.text:0000000078D2452F                 add     rsp, 50h
.text:0000000078D24533                 pop     rbx
.text:0000000078D24534                 retn
.text:0000000078D24534 ; ---------------------------------------------------------------------------
.text:0000000078D24535                 align 20h
.text:0000000078D24535 BaseDllMapResourceIdA endp

....

.text:0000000078D42FB4 loc_78D42FB4:                           ; CODE XREF: BaseDllMapResourceIdA+D
.text:0000000078D42FB4                 cmp     byte ptr [rcx], '#'
.text:0000000078D42FB7                 jnz     short loc_78D43005
.text:0000000078D42FB9                 inc     rcx
.text:0000000078D42FBC                 lea     r8, [rsp+58h+arg_8]
.text:0000000078D42FC1                 mov     edx, 0Ah
.text:0000000078D42FC6                 call    cs:__imp_RtlCharToInteger
.text:0000000078D42FCC                 mov     ecx, [rsp+58h+arg_8]
.text:0000000078D42FD0                 mov     [rsp+58h+var_38], rcx
.text:0000000078D42FD5                 test    eax, eax
.text:0000000078D42FD7                 js      short loc_78D42FE6
.text:0000000078D42FD9                 test    rcx, 0FFFFFFFFFFFF0000h
.text:0000000078D42FE0                 jz      loc_78D2452A

....

\end{lstlisting}

Если значение больше чем 0x10000, происходит переход в то место, где обрабатывается строка.
Иначе, входное значение \IT{lpId} возвращается как есть.
Маска \IT{0xFFFF0000} здесь больше не используется, т.к., это все же 64-битный код, но всё-таки,
маска \IT{0xFFFFFFFFFFFF0000} могла бы здесь использоваться.

Внимательный читатель может спросить, что если адрес входной строки будет ниже 0x10000?
Этот код полагается на тот факт, что в Windows нет ничего по адресам ниже 0x10000, по крайней мере, в Win32.

Raymond Chen \href{https://blogs.msdn.microsoft.com/oldnewthing/20130925-00/?p=3123}{пишет} об этом:

\begin{framed}
\begin{quotation}
How does MAKE­INT­RESOURCE work? It just stashes the integer in the bottom 16 bits of a pointer, leaving the upper bits zero. This relies on the convention that the first 64KB of address space is never mapped to valid memory, a convention that is enforced starting in Windows 7.
\end{quotation}
\end{framed}

Коротко говоря, это грязный хак, и наверное не стоит его использовать, если только нет большой необходимости.
Вероятно, аргумент ф-ции \IT{FindResource()} в прошлом имел тип \IT{SHORT}, а потом в Microsoft добавили возможность
передавать здесь и строки, но старый код также нужно было поддерживать.

Вот мой короткий очищенный пример:

\begin{lstlisting}[style=customc]
#include <stdio.h>
#include <stdint.h>

void f(char* a)
{
	if (((uint64_t)a)>0x10000)
		printf ("Pointer to string has been passed: %s\n", a);
	else
		printf ("16-bit value has been passed: %d\n", (uint64_t)a);
};

int main()
{
	f("Hello!"); // pass string
	f((char*)1234); // pass 16-bit value
};
\end{lstlisting}

Работает!

\subsubsection{Издевательство над указателями в ядре Linux}

Как было упомянуто среди \href{https://news.ycombinator.com/item?id=11823647}{комментариев на Hacker News},
в ядре Linux также есть что-то подобное.

Например, эта ф-ция может возвращать и код ошибки и указатель:

\begin{lstlisting}[style=customc]
struct kernfs_node *kernfs_create_link(struct kernfs_node *parent,
				       const char *name,
				       struct kernfs_node *target)
{
	struct kernfs_node *kn;
	int error;

	kn = kernfs_new_node(parent, name, S_IFLNK|S_IRWXUGO, KERNFS_LINK);
	if (!kn)
		return ERR_PTR(-ENOMEM);

	if (kernfs_ns_enabled(parent))
		kn->ns = target->ns;
	kn->symlink.target_kn = target;
	kernfs_get(target);	/* ref owned by symlink */

	error = kernfs_add_one(kn);
	if (!error)
		return kn;

	kernfs_put(kn);
	return ERR_PTR(error);
}
\end{lstlisting}

( \url{https://github.com/torvalds/linux/blob/fceef393a538134f03b778c5d2519e670269342f/fs/kernfs/symlink.c#L25} )

\IT{ERR\_PTR} это макрос, приводящий целочисленное к указателю:

\begin{lstlisting}[style=customc]
static inline void * __must_check ERR_PTR(long error)
{
	return (void *) error;
}
\end{lstlisting}

( \url{https://github.com/torvalds/linux/blob/61d0b5a4b2777dcf5daef245e212b3c1fa8091ca/tools/virtio/linux/err.h} )

Этот же заголовочный файл имеет также макрос, который можно использовать, чтобы отличить код ошибки от указателя:

\begin{lstlisting}[style=customc]
#define IS_ERR_VALUE(x) unlikely((x) >= (unsigned long)-MAX_ERRNO)
\end{lstlisting}

Это означает, коды ошибок это ``указатели'' очень близкие к -1, и, будем надеяться, в памяти ядра ничего не находится
по адресам вроде 0xFFFFFFFFFFFFFFFF, 0xFFFFFFFFFFFFFFFE, 0xFFFFFFFFFFFFFFFD, итд.

Намного более популярный способ это возвращать \IT{NULL} в случае ошибки и передавать код ошибки через дополнительный
аргумент.
Авторы ядры Linux так не делают, но все кто пользуется этими ф-циями, должны помнить, что возвращаемый указатель
должен быть вначале проверен при помощи \IT{IS\_ERR\_VALUE} перед разыменовыванием.

Например:

\begin{lstlisting}[style=customc]
	fman->cam_offset = fman_muram_alloc(fman->muram, fman->cam_size);
	if (IS_ERR_VALUE(fman->cam_offset)) {
		dev_err(fman->dev, "%s: MURAM alloc for DMA CAM failed\n",
			__func__);
		return -ENOMEM;
	}
\end{lstlisting}

( \url{https://github.com/torvalds/linux/blob/aa00edc1287a693eadc7bc67a3d73555d969b35d/drivers/net/ethernet/freescale/fman/fman.c#L826} )

\subsubsection{Издевательство над указателями в пользовательской среде UNIX}

\myindex{UNIX!mmap()}
Ф-ция mmap() возвращает -1 в случае ошибки (или \TT{MAP\_FAILED}, что равно -1).
Некоторые люди говорят, что в некоторых случаях, mmap() может подключить память по нулевому адресу, так что использовать
0 или NULL как код ошибки нельзя.

