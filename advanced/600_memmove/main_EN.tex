\section{memmove() and memcpy()}
\label{memmove_and_DF}
\myindex{\CStandardLibrary!memmove()}
\myindex{\CStandardLibrary!memcpy()}

The difference between these standard functions is that \IT{memcpy()} blindly copies a block to another place,
while \IT{memmove()} correctly handles overlapping blocks.
For example, you want to tug a string two bytes forward:

% TODO TikZ
\begin{lstlisting}
`|.|.|h|e|l|l|o|...` -> `|h|e|l|l|o|...`
\end{lstlisting}

\IT{memcpy()} which copies 32-bit or 64-bit words at once, or even \ac{SIMD},
will obviously fail here, a byte-wise copy routine must be used instead.

Now even more advanced example, insert two bytes in front of string:

\begin{lstlisting}
`|h|e|l|l|o|...` -> `|.|.|h|e|l|l|o|...`
\end{lstlisting}

Now even byte-wise memory copy routine will fail, you have to copy bytes starting at the end.

\myindex{x86!\Registers!DF}
That's a rare case where \TT{DF} x86 flag is to be set before \INS{REP MOVSB} instruction:
\TT{DF} defines direction, and now we must move backwardly.

The typical \IT{memmove()} routine works like this:
1) if source is below destination, copy forward;
2) if source is above destination, copy backward.

\myindex{uClibc}
This is \IT{memmove()} from uClibc:

\begin{lstlisting}[style=customc]
void *memmove(void *dest, const void *src, size_t n)
{
	int eax, ecx, esi, edi;
	__asm__ __volatile__(
		"	movl	%%eax, %%edi\n"
		"	cmpl	%%esi, %%eax\n"
		"	je	2f\n" /* (optional) src == dest -> NOP */
		"	jb	1f\n" /* src > dest -> simple copy */
		"	leal	-1(%%esi,%%ecx), %%esi\n"
		"	leal	-1(%%eax,%%ecx), %%edi\n"
		"	std\n"
		"1:	rep; movsb\n"
		"	cld\n"
		"2:\n"
		: "=&c" (ecx), "=&S" (esi), "=&a" (eax), "=&D" (edi)
		: "0" (n), "1" (src), "2" (dest)
		: "memory"
	);
	return (void*)eax;
}
\end{lstlisting}

In the first case, \INS{REP MOVSB} is called with \TT{DF} flag cleared.
In the second, \TT{DF} is set, then cleared.

More complex algorithm has the following piece in it:

\q{if difference between \IT{source} and \IT{destination} is larger than width of \gls{word},
copy using words rather than bytes, and use byte-wise copy to copy unaligned parts}.

\myindex{Glibc}
This how it happens in Glibc 2.24 in non-optimized C part.

Given all that, \IT{memmove()} may be slower than \IT{memcpy()}.
But some people, including Linus Torvalds, argue\footnote{\url{https://bugzilla.redhat.com/show_bug.cgi?id=638477\#c132}}
that \IT{memcpy()} should be an alias (or synonym) of \IT{memmove()}, and the latter
function must just check at start, if the buffers are overlapping or not, and then behave as \IT{memcpy()} or \IT{memmove()}.
Nowadays, check for overlapping buffers is very cheap, after all.

\subsection{Anti-debugging trick}

I've heard about anti-debugging trick where all you need is just set \TT{DF} to crash the process: the very next \IT{memcpy()}
routine will lead to crash because it copies backwardly.
But I can't check this: it seems all memory copy routines clear/set \TT{DF} as they want to.
On the other hand, \IT{memmove()} from uClibc I cited here,
has no explicit clear of \TT{DF} (it assumes \TT{DF} is always clear?),
so it can really crash.

