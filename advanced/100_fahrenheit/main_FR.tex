\section{Conversion de température}

Un autre exemple très populaire dans
les livres de programmation est un petit programme qui convertit une température de Fahrenheit vers Celsius ou inversemment.

\[
	C=\frac{5 \cdot (F-32)}{9}
\]

Nous pouvons aussi ajouter une gestion des erreurs simples:
1) nous devons vérifier si l'utilisateur a entré un nombre correct;
2) nous devons tester si la température en Celsius n'est pas en dessous de $-273$ 
(qui est en dessous du zéro absolu, comme vu pendant les cours de physique à l'école)

\myindex{\CStandardLibrary!exit()}
La fonction \TT{exit()} termine 
le programme instantanément, sans retourner à la fonction \gls{caller}.

\subsection{Valeurs entières}

\lstinputlisting[style=customc]{\CURPATH/i.c}

\subsubsection{MSVC 2012 x86 \Optimizing}

\lstinputlisting[caption=MSVC 2012 x86 \Optimizing,style=customasmx86]{\CURPATH/i_MSVC_2012_Ox_x86_EN.asm}

Ce que l'on peut en dire:

\begin{itemize}
\item L'adresse de \printf est d'abord chargée dans le
registre \ESI, donc les futurs
appels à \printf seront faits juste par l'instruction \TT{CALL ESI}.
C'est une technique très populaire
des compilateurs, possible si plusieurs appels consécutifs vers la même fonction sont présents
dans le code, et/ou s'il y a un registre disponible qui peut être utilisé pour ça.

\item Nous voyons l'instruction \TT{ADD EAX, -32} 
à l'endroit où 32 doit être soustrait de la valeur.
$EAX=EAX+(-32)$ est équivalent à $EAX=EAX-32$ 
et curieusement, le compilateur a décidé d'utiliser \TT{ADD} au lieu de \TT{SUB}.
Peut-être que ca vaut la peine, difficile d'en être sûr.

\myindex{x86!\Instructions!LEA}
\item L'instruction \LEA est utilisée quand 
la valeur est multipliée par 5: \TT{lea ecx, DWORD PTR [eax+eax*4]}.
Oui, $i+i*4$ équivaut à $i*5$ et \LEA 
marche plus rapidement que \TT{IMUL}.\\
D'ailleurs, la paire d'instructions \TT{SHL EAX, 2 / ADD EAX, EAX} peut aussi être utilisée ici---
certains compilateur le font.

\item La division par l'astuce de la multiplication (\myref{sec:divisionbymult}) 
est aussi utilisée ici.

\item \main retourne 0 si nous n'avons pas \TT{return 0} 
à la fin.
Le standard C99 nous dit \InSqBrackets{\CNineNineStd 5.1.2.2.3} que \main 
va retourner 0 dans le cas où 
la déclaration \TT{return} est manquante.
Cette règle ne fonctionne uniquement que pour la fonction \main.

Cependant, MSVC ne supporte pas officiellement C99, mais peut-être qu'il le supporte partiellement ?
\end{itemize}

\subsubsection{MSVC 2012 x64 \Optimizing}

Le code est quasiment le même, mais nous pouvons trouver les instructions \TT{INT 3} après chaque appel à \TT{exit()}.

\begin{lstlisting}[style=customasmx86]
	xor	ecx, ecx
	call	QWORD PTR __imp_exit
	int	3
\end{lstlisting}

\TT{INT 3} est un point d'arrêt du debugger.

C'est connu que \TT{exit()} est l'une des fonctions qui ne retourne jamais
\footnote{une autre populaire est \TT{longjmp()}},
donc si elle le fait, quelque chose de vraiment étrange est arrivé et il est temps de lancer le debugger.

\subsection{Floating-point values}

\lstinputlisting[style=customc]{\CURPATH/f.c}

MSVC 2010 x86 utilise des instructions \ac{FPU}\dots

\lstinputlisting[caption=MSVC 2010 x86 \Optimizing,style=customasmx86]{\CURPATH/f_MSVC_2010_x86_Ox_EN.asm}

\dots mais MSVC 2012 utilise à la place des instructions \ac{SIMD}:

\lstinputlisting[caption=MSVC 2010 x86 \Optimizing,style=customasmx86]{\CURPATH/f_MSVC_2012_x86_Ox_EN.asm}

Bien sûr, les instructions \ac{SIMD} sont disponibles dans le mode x86,
incluants celles qui fonctionnent avec les nombres à virgule flottante.

C'est quelque part plus simple de les utiliser pour les calculs, donc le nouveau compilateur Microsoft les utilise.

Nous pouvons aussi voir que la valeur $-273$
est chargée dans le registre \XMM{0} trop tôt.
Et c'est OK, parce que le compilateur peut mettre des instructions dans un ordre
différent de celui du code source.
