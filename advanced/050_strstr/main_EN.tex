% TODO translate to RU
\section{strstr() example}
\label{strstr_example}
\myindex{\CStandardLibrary!strstr()}

Let's back to the fact that GCC sometimes can use part of string: \myref{use_parts_of_C_strings}.

The \IT{strstr()} \CCpp standard library function is used to find any occurrence in a string.
This is what we will do:

\begin{lstlisting}[style=customc]
#include <string.h>
#include <stdio.h>

int main()
{
	char *s="Hello, world!";
	char *w=strstr(s, "world");

	printf ("%p, [%s]\n", s, s);
	printf ("%p, [%s]\n", w, w);
};
\end{lstlisting}

The output is:

\begin{lstlisting}
0x8048530, [Hello, world!]
0x8048537, [world!]
\end{lstlisting}

The difference between the address of the original string and the address of the substring that \IT{strstr()} has returned is 7.
Indeed, \q{Hello, } string has length of 7 characters.

The \printf{} function during second call has no idea there are some other characters before
the passed string and it prints characters
from the middle of original string till the end (marked by zero byte).

