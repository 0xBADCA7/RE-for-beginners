\section{Duff's device}
\myindex{Duff's device}
\myindex{Unrolled loop}

Duff's device 
\footnote{\href{http://go.yurichev.com/17137}{wikipedia}}
is an unrolled loop with the possibility to jump to the middle of it.
The unrolled loop is implemented using a fallthrough switch() statement.
We would use here a slightly simplified version of Tom Duff's original code.
Let's say, we have to write a function that clears a region in memory.
One can come with a simple loop, clearing byte by byte.
It's obviously slow, since all modern computers have much wider memory bus.
So the better way is to clear the memory region using 4 or 8 bytes blocks.
Since we are going to work with a 64-bit example here, we are going to clear the memory in 8 bytes blocks.
So far so good.
But what about the tail? 
Memory clearing routine can also be called for regions of size that's not a multiple of 8.
So here is the algorithm:

\begin{itemize}
\item calculate the number of 8-bytes blocks, clear them using 8-bytes (64-bit) memory accesses;

\item calculate the size of the tail, clear it using 1-byte memory accesses.
\end{itemize}

The second step can be implemented using a simple loop.
But let's implement it as an unrolled loop:

\lstinputlisting[style=customc]{\CURPATH/duff_EN.c}

Let's first understand how the calculation is performed.
The memory region size comes as a 64-bit value.
And this value can be divided in two parts:

% .... 7 6 5 4 3 2 1 0
%|....|B|B|B|B|B|S|S|S|

\begin{center}
\begin{bytefield}[endianness=big,bitwidth=0.03\linewidth]{11}
\bitheader{7,6,5,4,3,2,1,0} \\
\bitbox{3}{\dots} & 
\bitbox{1}{B} & 
\bitbox{1}{B} & 
\bitbox{1}{B} & 
\bitbox{1}{B} & 
\bitbox{1}{B} & 
\bitbox{1}{S} & 
\bitbox{1}{S} & 
\bitbox{1}{S}
\end{bytefield}
\end{center}

( \q{B} is number of 8-byte blocks and \q{S} is length of the tail in bytes ).

When we divide the input memory region size by 8, the value is just shifted right by 3 bits.
But to calculate the remainder, we can just to isolate the lowest 3 bits!
So the number of 8-byte blocks is calculated as $count>>3$ and remainder as $count \& 7$.
We also have to find out if we are going to execute the 8-byte procedure at all, so we need
to check if the value of $count$ is greater than 7.
We do this by clearing the 3 lowest bits and comparing the resulting number with zero, because 
all we need here is to answer the question, is the high part of $count$ non-zero.
Of course, this works because 8 is $2^{3}$ and division by numbers that are $2^n$ is easy.
It's not possible for other numbers.
It's actually hard to say if these hacks are worth using, because they lead
to hard-to-read code.
However, these tricks are very popular and a practicing programmer, 
even if he/she is not using them, nevertheless has to understand them.

So the first part is simple: get the number of 8-byte blocks and write 64-bit zero values to memory.
The second part is an unrolled loop implemented as fallthrough switch() statement.

First, let's express in plain English what we have to do here.

We have to \q{write as many zero bytes in memory, as $count\&7$ value tells us}.
If it's 0, jump to the end, there is no work to do.
If it's 1, jump to the place inside switch() statement where only one storage operation
is to be executed.
If it's 2, jump to another place, where two storage operation are to be executed, etc.
7 as input value leads to the execution of all 7 operations.
There is no 8, because a memory region of 8 bytes is to be processed by the first part of our function.
So we wrote an unrolled loop.
It was definitely faster on older computers than normal loops (and conversely,
latest \ac{CPU}s works better for short loops than for unrolled ones).
Maybe this is still meaningful on modern low-cost embedded \ac{MCU}s.

Let's see what the optimizing MSVC 2012 does:

\lstinputlisting[style=customasmx86]{\CURPATH/duff_MSVC2012_x64_Ox_EN.asm}

The first part of the function is predictable.
The second part is just an unrolled loop and a jump passing control flow to the correct instruction
inside it.
There is no other code between the \TT{MOV}/\TT{INC} instruction pairs, 
so the execution is to fall until the very end, executing as many pairs as needed.
By the way, we can observe that the \TT{MOV}/\TT{INC} pair consumes a fixed number of bytes (3+3).
So the pair consumes 6 bytes.
Knowing that, we can get rid of the switch() jumptable, we can just multiple the input value by 6
and jump to $current\_RIP + input\_value * 6$.

This can also be faster because we are not in need to fetch a value from the jumptable.

It's possible that 6 probably is not a very good constant for fast multiplication and maybe it's not worth it,
but you get the idea\footnote{As an exercise, you can try to rework the code to get rid of 
the jumptable. 
\myindex{x86!\Instructions!STOSB}
The instruction pair can be rewritten in a way that it will consume 4 bytes or maybe 8. 
1 byte is also possible (using \TT{STOSB} instruction).}.

That is what old-school demomakers did in the past with unrolled loops.

\subsection{Should one use unrolled loops?}
\myindex{Unrolled loop}

Unrolled loops can have benefits if there is no fast cache memory between \ac{RAM} and \ac{CPU}, and the \ac{CPU},
in order to get the code of the next instruction, must load it from \ac{RAM} each time.
This is a case of modern low-cost \ac{MCU} and old \ac{CPU}s.

Unrolled loops are slower than short loops if there is a fast cache between \ac{RAM} and \ac{CPU} and the body of loop
can fit into cache, and \ac{CPU} will load the code from it not touching the \ac{RAM}.
Fast loops are the loops which body's size can fit into L1 cache, but even faster loops are those small ones
which can fit into micro-operation cache.

