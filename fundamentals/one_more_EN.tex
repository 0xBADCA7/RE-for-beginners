\subsection{Couple of additions about two's complement form}

\subsubsection{Getting maximum number of some \gls{word}}

Maximum number in unsigned form is just a number where all bits are set: \IT{0xFF....FF}
(this is -1 if the \gls{word} is treated as signed integer).
So you take a \gls{word}, set all bits and get the value:

\begin{lstlisting}
	#include <stdio.h>

	int main()
	{
		int val=~0; // change to unsigned char to get maximal value for the unsigned 8-bit byte
		// 0-1 will also work, or just -1
		printf ("%u\n", val); // %u for unsigned
	};
\end{lstlisting}

This is 4294967295 for 32-bit integer.

\subsubsection{Getting minimum number for some signed \gls{word}}

Minimum signed number has form of \IT{0x80....00}, i.e., most significant bit is set, while others are cleared.
Maximum signed number has the same form, but all bits are inverted: \IT{0x7F....FF}.

Let's shift a lone bit left until it disappears:

\begin{lstlisting}
	#include <stdio.h>

	int main()
	{
		int val=1; // change to "signed char" to find values for signed byte
		while (val!=0)
		{
			printf ("%d %d\n", val, ~val);
			val=val<<1;
		};
	};
\end{lstlisting}

Output is:

\begin{lstlisting}
	...

	536870912 -536870913
	1073741824 -1073741825
	-2147483648 2147483647
\end{lstlisting}

Two last numbers are minimum and maximum signed 32-bit \IT{int} respectively.

