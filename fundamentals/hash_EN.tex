\newcommand{\HashFuncChapterName}{Hash functions}
\section{\HashFuncChapterName}
\label{hash_func}

\myindex{\HashFuncChapterName}
\myindex{CRC32}
A very simple example is CRC32, an algorithm that provides \q{stronger} checksum for integrity checking purposes.
It is impossible to restore the original text from the hash value, it has much less information:
But CRC32 is not cryptographically secure: it is known how to alter a text in a way that the resulting
CRC32 hash value will be the one we need.
Cryptographic hash functions are protected from this. \\
\\
\myindex{MD5}
\myindex{SHA1}
MD5, SHA1, etc. are such functions and they are widely used to hash user passwords in order to store them in a database.
Indeed: an Internet forum database may not contain user passwords 
(a stolen database can compromise all users' passwords) but only hashes 
(so a cracker can't reveal the passwords).
Besides, an Internet forum engine does not need to know your password exactly, it needs only to check if its hash
is the same as the one in the database, and give you access if they match.
One of the simplest password cracking methods is just to try hashing all possible passwords in order
to see which matches the resulting value that we need.
Other methods are much more complex.
% TODO1 add about Rainbow tables

\subsection{How do one-way functions work?}

A one-way function is a function which is able to transform one value into another,
while it is impossible (or very hard) to reverse it.
Some people have difficulties while understanding how this is possible at all.
Here is a simple demonstration.

We have a vector of 10 numbers in range 0..9, each is present only once, for example:

\begin{lstlisting}
4 6 0 1 3 5 7 8 9 2
\end{lstlisting}

The algorithm for the simplest possible one-way function is:

\begin{itemize}
\item take the number at zeroth position (4 in our case);
\item take the number at first position (6 in our case);
\item swap numbers at positions of 4 and 6.
\end{itemize}

Let's mark the numbers at positions 4 and 6:

\begin{lstlisting}
4 6 0 1 3 5 7 8 9 2
        ^   ^
\end{lstlisting}

Let's swap them and we get this result:

\begin{lstlisting}
4 6 0 1 7 5 3 8 9 2
\end{lstlisting}

While looking at the result, and even if we know the algorithm, we can't know unambiguously the initial
state, because the first two numbers could be 0 and/or 1, and then they could participate in the swapping procedure.

This is an utterly simplified example for demonstration. Real one-way functions are much more complex.
