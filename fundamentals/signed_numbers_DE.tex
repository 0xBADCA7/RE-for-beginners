\section{\SignedNumbersSectionName}
\label{sec:signednumbers}
\myindex{Signed numbers}

\newcommand{\URLS}{\href{http://go.yurichev.com/17117}{wikipedia}}

Es gibt verschiedene Arten vorzeichenbehaftete Zahlen\footnote{\URLS} darzustellen,
jedoch ist das \q{Zweierkomplement} die am weitesten verbreitete in Computern.

Hier ist eine Tabelle für einige Byte-Werte:

\begin{center}
\begin{tabular}{ | l | l | l | l | }
\hline
\HeaderColor binär & \HeaderColor hexadezimal & \HeaderColor vorzeichenlos & \HeaderColor vorzeichenbehaftet \\
\hline
01111111 & 0x7f & 127 & 127 \\
\hline
01111110 & 0x7e & 126 & 126 \\
\hline
\multicolumn{4}{ |c| }{...} \\
\hline
00000110 & 0x6 & 6 & 6 \\
\hline
00000101 & 0x5 & 5 & 5 \\
\hline
00000100 & 0x4 & 4 & 4 \\
\hline
00000011 & 0x3 & 3 & 3 \\
\hline
00000010 & 0x2 & 2 & 2 \\
\hline
00000001 & 0x1 & 1 & 1 \\
\hline
00000000 & 0x0 & 0 & 0 \\
\hline
11111111 & 0xff & 255 & -1 \\
\hline
11111110 & 0xfe & 254 & -2 \\
\hline
11111101 & 0xfd & 253 & -3 \\
\hline
11111100 & 0xfc & 252 & -4 \\
\hline
11111011 & 0xfb & 251 & -5 \\
\hline
11111010 & 0xfa & 250 & -6 \\
\hline
\multicolumn{4}{ |c| }{...} \\
\hline
10000010 & 0x82 & 130 & -126 \\
\hline
10000001 & 0x81 & 129 & -127 \\
\hline
10000000 & 0x80 & 128 & -128 \\
\hline
\end{tabular}
\end{center}

\myindex{x86!\Instructions!JA}
\myindex{x86!\Instructions!JB}
\myindex{x86!\Instructions!JL}
\myindex{x86!\Instructions!JG}
Der Unterschied zwischen vorzeichenbehafteten und vorzeichenlosen Zahlen ist, dass wenn \TT{0xFFFFFFFE}
und \TT{0x00000002} ohne Vorzeichen repräsentiert werden, die erste Zahl (4294967294) größer ist als
die zweite Zahl (2).
Wenn beide Zahlen als vorzeichenbehaftet repräsentiert werden, wird die erste Zahl $-2$ und ist kleiner
als die zweite Zahl (2).
Das ist der Grund, warum bedingte Sprünge ~(\myref{sec:Jcc}) sowohl für vorzeichenbehaftete (e.g. \JG, \JL)
als auch vorzeichenlose (\JA, \JB) Operationen vorhanden sind.

Aus Gründen der Einfachheit ist hier was man wissen muss:

\begin{itemize}
\item Zahlen können mit oder ohne Vorzeichen sein.

\item vorzeichenbehaftete Datentypen in \CCpp:

  \begin{itemize}
    \item \TT{int64\_t} (-9,223,372,036,854,775,808 .. 9,223,372,036,854,775,807)
	  (-~9.2..~9.2 Quintillionen) oder \\
                \TT{0x8000000000000000..0x7FFFFFFFFFFFFFFF}),
    \item \Tint (-2,147,483,648..2,147,483,647 (-~2.15..~2.15Gb) oder \\
	    \TT{0x80000000..0x7FFFFFFF}),
    \item \Tchar (-128..127 oder \TT{0x80..0x7F}),
    \item \TT{ssize\_t}.
   \end{itemize}

	Vorzeichenlos:
  \begin{itemize}
	  \item \TT{uint64\_t} (0..18,446,744,073,709,551,615 
		  (~18 Quintillionen) oder \TT{0..0xFFFFFFFFFFFFFFFF}),
   \item \TT{unsigned int} (0..4,294,967,295 (~4.3Gb) oder \TT{0..0xFFFFFFFF}),
   \item \TT{unsigned char} (0..255 oder \TT{0..0xFF}), 
   \item \TT{size\_t}.
  \end{itemize}

\item Vorzeichenbehaftete Typen haben das Vorzeichen am MSB: 1 bedeutet \q{Minus}, 0 bedeutet \q{Plus}.

\item Erweitern auf größere Datentypen ist einfach:
\myref{subsec:sign_extending_32_to_64}.

\label{sec:signednumbers:negation}
\item Negieren ist einfach: es müssen lediglich alle Bits invertiert und anschließend 1 addiert werden.

%TODO Not sure how to translate this into an understandable in german.
%We can keep in mind that a number of inverse sign is located on the opposite side at the same proximity from zero.
%The addition of one is needed because zero is present in the middle.

\myindex{x86!\Instructions!IDIV}
\myindex{x86!\Instructions!DIV}
\myindex{x86!\Instructions!IMUL}
\myindex{x86!\Instructions!MUL}
\myindex{x86!\Instructions!CBW}
\myindex{x86!\Instructions!CWD}
\myindex{x86!\Instructions!CWDE}
\myindex{x86!\Instructions!CDQ}
\myindex{x86!\Instructions!CDQE}
\myindex{x86!\Instructions!MOVSX}
\myindex{x86!\Instructions!SAR}
\item 
	Die Addition und Subtraktion funktioniert sowohl für Zahlen mit als auch ohne Vorzeichen.
	Für Multiplikation und Division gibt es bei x86 unterschiedliche Anweisungen:
	\TT{IDIV}/\TT{IMUL} für vorzeichenbehaftete und \TT{DIV}/\TT{MUL} für vorzeichenlose Zahlen.
\item
	Hier sind einige weitere Anweisungen die mit vorzeichenbehafteten Zahlen arbeiten:\\
	\TT{CBW/CWD/CWDE/CDQ/CDQE} (\myref{ins:CBW_CWD_etc}), \TT{MOVSX} (\myref{MOVSX}), \TT{SAR} (\myref{ins:SAR}).
\end{itemize}

% subsections:
%% TODO translate
\subsection{Using IMUL over MUL}
\label{IMUL_over_MUL}

\myindex{x86!\Instructions!MUL}
\myindex{x86!\Instructions!IMUL}
Example like \lstref{unsigned_multiply_C} where two unsigned values are multiplied compiles into \lstref{unsigned_multiply_lst} where \IMUL is used instead of \MUL.

This is important property of both \MUL and \IMUL instructions.
First of all, they both produce 64-bit value if two 32-bit values are multiplied, or 128-bit value if two 64-bit values are multiplied (biggest possible \gls{product} in 32-bit environment is \\
\GTT{0xffffffff*0xffffffff=0xfffffffe00000001}).
But \CCpp standards have no way to access higher half of result, and a \gls{product} always has the same size as multiplicands. % TODO \gls{}?
And both \MUL and \IMUL instructions works in the same way if higher half is ignored, i.e., they both generate
the same lower half.
This is important property of \q{two's complement} way of representing signed numbers.

So \CCpp compiler can use any of these instructions.

But \IMUL is more versatile than \MUL because it can take any register(s) as source, while \MUL requires one of multiplicands stored in \AX/\EAX/\RAX register.
Even more than that: \MUL stores result in \GTT{EDX:EAX} pair in 32-bit environment, or \GTT{RDX:RAX} in 64-bit one, so it always calculates the whole result.
On contrary, it's possible to set a single destination register while using \IMUL instead of pair, and then \ac{CPU} will calculate only lower half, which works faster [see Torborn Granlund, \IT{Instruction latencies and throughput for AMD and Intel x86 processors}\footnote{\url{http://yurichev.com/mirrors/x86-timing.pdf}]}).

Given than, \CCpp compilers may generate \IMUL instruction more often then \MUL.

\myindex{Compiler intrinsic}
Nevertheless, using compiler intrinsic, it's still possible to do unsigned multiplication and get \IT{full} result.
This is sometimes called \IT{extended multiplication}.
MSVC has intrinsic for this called \IT{\_\_emul}\footnote{\url{https://msdn.microsoft.com/en-us/library/d2s81xt0(v=vs.80).aspx}} and another one: \IT{\_umul128}\footnote{\url{https://msdn.microsoft.com/library/3dayytw9%28v=vs.100%29.aspx}}.
GCC offer \IT{\_\_int128} data type, and if 64-bit multiplicands are first promoted to 128-bit ones,
then a \gls{product} is stored into another \IT{\_\_int128} value, then result is shifted by 64 bits right,
you'll get higher half of result\footnote{Example: \url{http://stackoverflow.com/a/13187798}}.

\subsubsection{MulDiv() function in Windows}
\myindex{Windows!Win32!MulDiv()}

Windows has MulDiv() function
\footnote{\url{https://msdn.microsoft.com/en-us/library/windows/desktop/aa383718(v=vs.85).aspx}},
fused multiply/divide function, it multiplies two 32-bit integers into intermediate 64-bit value
and then divides it by a third 32-bit integer.
It is easier than to use two compiler intrinsic, so Microsoft developers made a special function for it.
And it seems, this is busy function, judging by its usage.


%\subsection{Couple of additions about two's complement form}

\subsubsection{Getting maximum number of some \gls{word}}

Maximum number in unsigned form is just a number where all bits are set: \IT{0xFF....FF}
(this is -1 if the \gls{word} is treated as signed integer).
So you take a \gls{word}, set all bits and get the value:

\begin{lstlisting}
	#include <stdio.h>

	int main()
	{
		unsigned int val=~0; // change to "unsigned char" to get maximal value for the unsigned 8-bit byte
		// 0-1 will also work, or just -1
		printf ("%u\n", val); // %u for unsigned
	};
\end{lstlisting}

This is 4294967295 for 32-bit integer.

\subsubsection{Getting minimum number for some signed \gls{word}}

Minimum signed number has form of \IT{0x80....00}, i.e., most significant bit is set, while others are cleared.
Maximum signed number has the same form, but all bits are inverted: \IT{0x7F....FF}.

Let's shift a lone bit left until it disappears:

\begin{lstlisting}
	#include <stdio.h>

	int main()
	{
		signed int val=1; // change to "signed char" to find values for signed byte
		while (val!=0)
		{
			printf ("%d %d\n", val, ~val);
			val=val<<1;
		};
	};
\end{lstlisting}

Output is:

\begin{lstlisting}
	...

	536870912 -536870913
	1073741824 -1073741825
	-2147483648 2147483647
\end{lstlisting}

Two last numbers are minimum and maximum signed 32-bit \IT{int} respectively.



Eine Tabelle mit negativen und positiven Werten (\ref{signed_tbl}) sieht aus wie ein Thermometer mit Celsius-Skala.
Das ist der Grund warum Addition und Subtraktion für Zahlen mit und ohne Vorzeichen gleich funktioniert:
wenn der erste Summand eine Markierung auf dem Thermometer ist und ein weiterer Summand addiert werden soll,
der positiv ist, muss lediglich die Markierung auf dem Thermometer um den Wert des zweiten Summanden nach
oben verschoben werden.
Ist der zweite Summand negativ, wird die Markierung um den entsprechenden, absoluten Wert nach unten verschoben.

Die Addition zweier negativer Zahlen funktioniert wie folgt:
Wenn beispielsweise -2 und -3 unter Verwendung eines 16-Bit-Registers addiert werden sollen,
ist die Darstellung 0xfffe beziehungsweise 0xfffd.
Wenn diese Werte ohne Vorzeichen addiert werden, ist das Ergebnis 0xfffe+0xfffd=0x1fffb.
Allerdings sollen 16-Bit-Register verwendet werden, also wird beim Ergebis die erste 1 abgeschnitten.
Es bleibt 0xfffb was -5 entspricht.
Dies funktioniert, weil -2 (oder 0xfffe) in natürlicher Sprache wie folgt repräsentiert werden kann:
``2 fehlt bei diesem Werte bis zum maximalen Wert in einem 16-Bit-Register + 1''.
-3 kann repräsentiert werden als ``\dots 3 fehlt in diesem Wert bis zu \dots''.
Der maximale Wert eines 16-Bit-Registers + 1 ist 0x10000.
Bei der Addition der beiden Zahlen und \IT{Abschneiden}  durch $2^{16}$ modulo,
wird $2+3=5$ \IT{fehlen}.
