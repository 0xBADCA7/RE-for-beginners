\subsection{Einige weitere Anmerkungen zum Zweierkomplement}

\subsubsection{Das Maximum eines \gls{word} finden}
Die maximale Zahl in vorzeichenloser Form ist lediglich eine Zahl in der alle
Bits gesetzt sind: \IT{0xFF....FF}
(das entspricht -1 wenn das \gls{word} als vorzeichenbehaftete Zahl behandelt wird).
Also nimmt man ein \gls{word}, setzt alle Bits und bekommt den Wert:

\begin{lstlisting}
	#include <stdio.h>

	int main()
	{
		unsigned int val=~0; // Aendern zu "unsigned char" um den maximalen Wert fur vorzeichenlose 8-Bit-Zahlen
		// 0-1 funktionieren auch oder nur -1
		printf ("%u\n", val); // %u fuer vorzeichenlos
	};
\end{lstlisting}

Dies entspricht 4294967295 für 32-Bit-Integer.

\subsubsection{Das Minimum eines vorzeichenbehafteten \gls{word} finden}

%Minimum signed number has form of \IT{0x80....00}, i.e., most significant bit is set, while others are cleared.
%Maximum signed number has the same form, but all bits are inverted: \IT{0x7F....FF}.

%Let's shift a lone bit left until it disappears:

\begin{lstlisting}
	#include <stdio.h>

	int main()
	{
		signed int val=1; // Aendern zu signed char um den maximalen Wert fur vorzeichenbehaftete 8-Bit-Zahlen
		while (val!=0)
		{
			printf ("%d %d\n", val, ~val);
			val=val<<1;
		};
	};
\end{lstlisting}

Die Ausgabe ist:

\begin{lstlisting}
	...

	536870912 -536870913
	1073741824 -1073741825
	-2147483648 2147483647
\end{lstlisting}

Die letzten beiden Zahlen sind minimaler beziehungsweise maximaler Wert eines 32-Bit \IT{int}.
